% $Author: oscar$
% $Translate: mathieu chappuis + martial boniou $
% $Date: 2007-12-13 15:59:51 +0100 (Thu, 13 Dec 2007) $
% $Revision$
% $french: Sun Dec 16 14:25:37 CET 2007$ 
%%%%%%%%%%%%%%%%%%%%%%
% note temporaire de Martial destine aux relectures:
% collapse a window --> ranger la fenetre (et pas reduire, si vous
% voyez cette erreur, SVP merci de corriger)
%%%%%%%%%%%%%%%%%%%%%%
% relecture: Rene Mages (fusion par martial: Wed Dec 26 17:28:17 CET 2007)
% relecture: Rene Mages (Sat Jan 12 17:28:17 CET 2007)
% note de martial: revoir les index pour les termes liés à Morphic
% adaptation pour PBE - 28729
% sync avec la version: 29170
%=================================================================
\ifx\wholebook\relax\else
% --------------------------------------------
% Lulu:
	\documentclass[a4paper,10pt,twoside]{book}
	\usepackage[
		papersize={6.13in,9.21in},%,
		%% Martial: j'ai enlevé les lignes pour tester la mise en page "manuelle" de la figure colouredMouse
		hmargin={.75in,.75in},
		vmargin={.75in,1in}
%		ignoreheadfoot
	]{geometry}
	\input{../common.tex}
	\pagestyle{headings}
	\setboolean{lulu}{true}
% --------------------------------------------
% A4:
%	\documentclass[a4paper,11pt,twoside]{book}
%	\input{../common.tex}
%	\usepackage{a4wide}
% --------------------------------------------
    \graphicspath{{figures/} {../figures/}}
	\begin{document}
	%\renewcommand{\nnbb}[2]{} % Disable editorial comments
	\sloppy
\fi
%=================================================================
\newcommand{\clover}{%
	\raisebox{-0.8ex}[0pt][0pt]{%
		\includegraphics[width=1em]{cloverleafKey}}}
%=================================================================

\chapter{Une visite de \pharo}
\chalabel{quick}

Nous vous proposons dans ce chapitre une premi\`ere visite de \pharo afin de vous familiariser avec son environnement.
De nombreux aspects seront abordés; il est conseillé d'avoir une
machine pr\^ete \`a l'emploi pour suivre ce chapitre. 

Cette icône \dothisicon{} dans le texte signalera les étapes où vous devrez essayer quelque chose vous-même.
Vous apprendrez à lancer \pharo et les différentes manières d'utiliser l'environnement et les outils de base.
La création des méthodes, des objets et les envois de messages seront également abordés.

%=================================================================
\section{Premiers pas}

\pharo est librement disponible au \ind{téléchargement} depuis le site
web: \pharoweb. % CHANGE
Vous devez y télécharger 3 archives (pour 4 fichiers principaux qui
constituent une installation courante de \pharo; voir \figref{download}) 

\begin{figure}[htb]
\centerline {\includegraphics[width=\textwidth]{annotatedDownload}}
\caption{Les fichiers à télécharger de \pharo. \figlabel{download}}
\end{figure}

\begin{enumerate}

\item La \emphind{machine virtuelle} (abr\'eg\'ee en VM pour
  \emph{virtual machine}) est la seule partie de l'environnement qui
  est particulière à chaque système d'exploitation. Des machines
  virtuelles pré-compilées sont disponibles pour la plupart des
  systèmes (Linux, \macosx, Win32). Dans \figref{download}, vous
  avez par exemple l'ic\^one de la machine virtuelle pour le syst\`eme
  \macosx: \textit{Squeak 4.0.1beta1U.app}~\footnote{\pharo est dérivé
  de \squeak{} 3.9 et il partage actuellement la machine virtuelle
  avec \squeak.}.
%martial: ajout des index dans la vf
\index{machine virtuelle}
\seeindex{VM}{machine virtuelle}

  \item Le fichier \emphind{source} contient le code source du système
    \pharo. Ce fichier ne change pas tr\`es fréquement. Dans \figref{download}, il
    correspond au fichier \emph{SqueakV39.sources}.
%ajout fr index
\index{fichier!source}
\seeindex{fichier-source}{fichier, source}
\seeindex{SqueakV39.sources}{fichier, source}

\item Le \emph{système} \emphind{image} est un cliché d'un système
  \pharo{} en fonctionnement, figé à un instant donné. 
Il est composé de deux fichiers: le premier nommé avec l'extension
\emph{.}\emphind{image} contient l'état de tous les objets du système
dont les classes et les méthodes (qui sont aussi des
objets). Le second avec l'extension \emph{.}\emphind{changes} contient
le journal de toutes les modifications apportées au code source du
système (contenu dans le fichier source).
\arelire{Dans \figref{download}, nous voyons que nous utilisons les images et
fichiers \emph{changes} de \textit{PBE}.
\arevoir{Nous utiliserons en fait une image légérement différente dans
ce livre.}} \on{fix the figure to use Damien's PBE image} % CHANGE
%ajout fr index
\index{fichier!image}
\index{fichier!changes}

\end{enumerate}

\dothis{Téléchargez et installez \pharo sur votre ordinateur.}
\arelire{Nous vous recommandons d'utiliser l'image fournie sur la page web du
livre\footnote{\ppe}.}
\index{téléchargement}
\seclabel{sbeImage} %????


Sachez que si vous avez déjà une autre version de \pharo qui fonctionne sur votre
machine, la plupart des exemples d'introduction de ce livre
fonctionneront. Il n'est donc pas nécessaire de mettre à jour \pharo.
D\`es lors, ne soyez pas surpris de constater parfois des différences dans l'apparence ou le comportement que nous décrirons.
% D'un autre c\^ot\'e, si vous téléchargez \pharo pour la première fois,
% vous devriez rapatrier et utiliser l'image \emph{\pharo par l'Exemple}.

Pendant que vous travaillez avec \pharo les fichiers \emph{.image} et \emph{.changes} sont modifiés, vous devez vous assurer qu'ils sont accessibles en écriture.
Conservez toujours ces deux fichiers ensemble, \cad dans le même dossier.
Et surtout, ne tentez pas de les modifier avec un éditeur de texte, \pharo les utilise pour stocker vos objets de travail et vos changements dans le code source.
Faire une copie de sauvegarde de vos images téléchargées et de vos
fichiers \emph{changes} est une bonne id\'ee; vous pourrez ainsi
toujours démarrer à partir d'une image propre et y recharger votre code.

Les fichiers \emphind{source}{}\emph{s} et l'exécutable de la VM peuvent être
en lecture seule\,---\,il est donc possible de les partager entre plusieurs utilisateurs.
Ces quatre fichiers peuvent résider dans le même dossier, mais vous pouvez également placer la machine virtuelle et les fichiers sources dans un dossier partagé distinct.
Vous pouvez adapter l'installation de \pharo à vos habitudes de travail
et \`a votre système d'exploitation.

%-----------------------------------------------------------------
\begin{figure}[htb]
%\centerline {\includegraphics[width=0.6\textwidth]{download}}
\centerline {\includegraphics[width=0.95\textwidth]{startup}}
\caption{Une image \arevoir{\ppe} fra\^{\i}chement d\'emarr\'ee.\figlabel{startup}}
\end{figure}

\index{lancement de \pharo}

\paragraph{Lancement.} Pour lancer \pharo, selon votre système: glissez
le fichier \emph{.}\emphind{image} sur l'icône de l'exécutable de la
machine virtuelle, ou double-cliquez sur le fichier
\emph{.}\emphind{image}, ou encore, depuis une ligne de commande,
tapez le nom du fichier binaire correspondant \`a la machine virtuelle
suivi du chemin d'accès au fichier \emph{.}\emphind{image} (si vous
avez installé plusieurs machines virtuelles, le système ne choisira
pas forcément celle qui convient, il sera préférable de
glisser-déposer l'image sur la VM ou d'utiliser la ligne de commande).

Une fois lancé, \pharo vous présente une large fenêtre qui contient des
espaces de travail nommés \emphind{Workspace} (voir \figref{startup}).
\arelire{Vous pourriez remarquer un barre de menus mais \pharo{}
  emploie principalement des menus contextuels.}


\dothis{Lancez \pharo. Vous pouvez fermer les fen\^etres d\'ej\`a
  ouvertes en \clickant sur \arevoir{l'icône {\sf X}} situ\'e sur le coin
  supérieur gauche des fenêtres ou les ranger (ce qui normalement les
  r\'eduit \`a  leur barre de titre) en \clickant sur le symbole
  \raisebox{-0.2ex}{{\Large $\circ$}} au coin supérieur droit.} % REVOIR dans la VO PBE x devrait être le bouton rouge si le style par défaut est OS X


%-----------------------------------------------------------------
\paragraph{Première interaction.}

Les options du menu World (``Monde'' en anglais) pr\'esent\'ees dans
\figref{threeButtons:click} sont un bon point de d\'epart.

\dothis{Cliquez à l'aide de la souris dans l'arrière plan de la
  fenêtre principale pour afficher le menu World, puis sélectionnez
  \menu{Workspace} pour créer un nouvel espace de
  travail ou Workspace.}

\begin{figure}[tbh]
	\centering
	\subfigure[Le menu World]{\figlabel{threeButtons:click}% click
		\includegraphics[width=0.40\linewidth]{worldMenu}}\hfill
	\subfigure[le menu contextuel]{\figlabel{threeButtons:actclick}% action click
		\includegraphics[width=0.55\linewidth]{yellowButtonMenuOnWorkspace}}\hfill
	\subfigure[Le halo Morphic]{\figlabel{threeButtons:metaclick}% meta click
		\includegraphics[width=0.60\linewidth]{morphicHaloOnWorkspace}}% these braces needed (else no whitespace at end of line)
	\caption{\arelire{Le menu World (affiché en \clickant{} avec la
        souris), un menu contextuel (affiché en \actclickant{}) et un
      \subind{Morphic}{halo} Morphic (affiché en \metaclickant).}\figlabel{threeButtons}}
\end{figure} % CHANGE
%\seeindex{morphic halo}{Morphic}
\seeindex{halo}{Morphic} % REVOIR

% ON: I had to shrink this and move it up to avoid
% it running over the end of the page.
% position d'origine de la figure colouredMouse

\st a été conçu à l'origine pour être utilisé avec une souris à trois
boutons. Si votre souris en a moins, vous pourrez utiliser des touches
du clavier en complément de la souris pour simuler les boutons
manquants. Une souris à deux boutons fonctionne bien avec \pharo, mais si
la v\^otre n'a qu'un seul bouton vous devriez songer à adopter un
modèle récent avec une molette qui fera office de troisième bouton: votre travail avec \pharo n'en sera que plus agréable.

\pharo évite les termes ``clic gauche'' ou ``clic droit'' car leurs
effets peuvent varier selon les systèmes, le mat\'eriel ou les
réglages utilisateur.
\arelire{Originellement, \st{} introduit des couleurs pour définir les
différents boutons de souris~\footnote{Les couleurs de boutons sont
  \emph{rouge}, \emph{jaune} et \emph{bleu}. Les auteurs de ce livre
  n'ont jamais pu se souvenir à quelle couleur se réfère chaque
  bouton.}.}
\index{bouton!rouge}
\index{bouton!jaune}
\index{bouton!bleu}
\arelire{Puisque de nombreux utilisateurs utiliseront diverses touches de
  modifications (\emph{Ctrl}, \emph{Alt}, \emph{Meta} \etc) pour
  réaliser les mêmes actions, nous utiliserons plutôt les termes
  suivants:}
\begin{description}
\item[\clickbtn:] \arelire{il s'agit du bouton de la souris le plus fréquemment
  utilisé et correspond au fait de \click{} avec une souris à un seul
  bouton sans aucun touche de modifications; \click{} sur
  l'arrière-plan de l'image fait apparaître le menu ``World'' (voir
  \figref{threeButtons:click}); 
nous utiliserons le terme \emph{\click} pour définir cette action;} % vf-only
\item[\actclickbtn:] \arelire{c'est le second bouton le plus utilisé; il est
  utilisé pour afficher un menu contextuel \ie{} un menu qui fournit
  différentes actions dépendant de où se trouve la souris comme le
  montre \figref{threeButtons:actclick}. Si vous n'avez pas de souris
  à multiples boutons, vous configurerez normalement la touche de
  modifications \emph{Ctrl} pour effectuer cette même action avec
  votre unique bouton de souris;
nous utiliserons l'expression ``\emph{\actclick}\footnote{En anglais,
  le terme utilisé est ``to actclick''.}''.} % vf-only: 2 dernières
                                % lignes
\item[\metaclickbtn:]  \arelire{vous pouvez finalement \emph{\metaclick{}} sur
  un objet affiché dans l'image pour activer le
  ``\subind{Morphic}{halo} Morphic'' qui est une consellation d'icônes
  autour de l'objet actif à l'écran; chaque icône repr\'esentant une poign\'ee de contr\^ole
permettant des actions telles que \emph{changer la taille} ou
\emph{faire pivoter l'objet}, comme vous pouvez le voir sur
\figref{threeButtons:metaclick}~\footnote{Notez que les icônes
    Morphic sont habituellement actives dans \pharo, mais vous pouvez
    les désactiver via le Preferences Browser que nous verrons plus
    loin.}.
En survolant lentement une icône avec le pointeur de votre souris,
une bulle d'aide en affichera un descriptif de sa fonction.
Dans \pharo, \metaclick dépend de votre système d'exploitation:
Soit vous devez maintenir {\sc Shift} \emph{Ctrl} ou {\sc Shift}
\emph{Alt} tout en \clickant.}
% \ab{This makes it sound like either {\sc shift} \emph{ctrl} or {\sc shift} \emph{alt} will work.  On my (Mac OS) system, only the latter works.  Perhaps we want to say: In \pharo, how you meta-click depends on your operating system. On Linux \ldots}
% Typically you will use a third modifier key, such as \emph{command} or \emph{meta} to \metaclick.
\end{description}


% %martial: il faut regler a la main la position et la hauteur de
% %l'encart avec la souris 
% \begin{wrapfigure}[15]{r}{0.25\linewidth}
% % The parameters are the number of narrow lines to the right of the figure [19],
% % the placement {r} for right, and the width of the figure. Capital R will allow some float.
% % Inside the wrapfig environment, linewidth is special --- the width of the figure.
% \includegraphics[width=0.95\linewidth]{colouredMouse}
% \caption{La souris de l'auteur. Le clic avec la molette correspond au bouton bleu.\figlabel{colouredMouse}}
% \end{wrapfigure}
% \newpage

\dothis{Saisissez \ct{Time now}
%ajout 
(expression retournant l'heure actuelle) dans le Workspace.
Puis \actclickz{} dans le Workspace et sélectionnez
\menu{print it} 
%ajout
(en fran\c{c}ais, ``imprimez-le'')
dans le menu qui apparaît.}

% \dothis{\Metaclickz{} sur le Workspace.
% Déplacez la poignée 
% %martial: j'ai decide de mettre tout les Morphic handles dans le
% %repertoire 'figures' de la racine 
% \rotateHandle{}
% %\raisebox{-0.4ex}{\includegraphics[width=1em]{morphicRotate}}
% située à proximité du coin inférieur gauche pour faire pivoter le Workspace.}

\arelire{Nous recommandons aux droitiers de configurer leur souris pour
\click{} avec le bouton gauche
% ajout vf
(qui devient donc le bouton de \clickbtn),
\actclick{} avec le bouton droit et \metaclick{} avec la
\arevoir{molette de défilement cliquable}, si elle est disponible.}
% Avec une souris sans molette il est possible d'invoquer le menu halo
% en maintenant \ct{alt}, \ct{ctrl}
% ou \ct{option} pendant que vous cliquez sur le \ind{bouton rouge}.
% \ab{This doesn't work any more.  This sentence either repeats or
% contradicts the meta-click item above; neither is a good idea.}
Si vous utilisez un Macintosh avec une souris à un bouton, vous pouvez
simuler le second bouton en maintenant la touche \clover{} enfoncée 
en \clickant. Cependant, si vous prévoyez d'utiliser \pharo souvent, nous
vous recommandons d'investir dans un modèle à deux boutons au minimum.

%  j'ai ajouté CTRL car sur mon linux ni alt ni fn.. ne marchent pour
%  ça. seul ctrl le fait..
% note de martial: ca depend aussi du windowmanager; c'est une bonne
% idee de le mettre en tout cas 
Vous pouvez configurer votre souris selon vos souhaits en utilisant
les préférences de votre système ou le pilote de votre dispostif de
pointage.
\ab{How can I get meta-click without a three-finger salute?  Is this a secret?}
\pharo vous propose des réglages pour adapter votre souris et les
touches spéciales de votre clavier. 
Dans l'outil de r\'eglage des pr\'ef\'erences nomm\'e \ind{Preference
  Browser} (\menu{System \go Preferences {\ldots\go}
  Preference Browser\ldots} dans le menu \menu{World}), la catégorie
\menu{keyboard} contient une option \emph{swapControlAndAltKeys}
permettant de \arelire{permuter les fonctions ``\actclick{}'' et
  ``\metaclick''}.
Cette catégorie
propose aussi des options afin de dupliquer les touches de modifications.%  et 
% rendre une pression sur \ct{alt} \'equivalente à une pression sur \ct{ctrl}.

\begin{figure}[htb]
	{\centerline {\includegraphics[width=\textwidth]{PreferenceBrowser}}}
\caption{Le Preference Browser.\figlabel{prefBrowser}}
\end{figure}

%=================================================================
\section{Le menu World}
\index{menu World}

\dothis{\Clickz dans l'arrière plan de \pharo.}
Le menu \menu{World} apparaît à nouveau.
La plupart des menus de \pharo ne sont pas modaux; ils ne bloquent pas
le système dans l'attente d'une réponse.
Avec \pharo vous pouvez maintenir ces menus sur l'écran en \clickant{} sur
l'icône en forme d'épingle au coin supérieur droit. Essayez!%  Vous
% remarquerez que les menus apparaissent  quand  vous cliquez  mais ne
% disparaissent pas quand vous relâchez votre bouton, ils restent
% visibles jusqu'à que vous ayez fait une sélection ou que vous ayez
% cliqué en dehors du menu. Tous les menus affichés à l'écran peuvent se déplacer en glissant leur barre de titre, comme n'importe quelle fenêtre.

Le menu World vous offre un moyen simple d'accéder à la plupart des
outils disponibles dans \pharo.

\dothis{\'Etudiez attentivement \arelire{le menu \menu{World} et, en
    particulier, son sous-menu \menu{Tools}} (voir \figref{threeButtons:click}).}

Vous y trouverez une liste des principaux outils de \pharo.
Nous aurons affaire à eux dans les prochains chapitres.

%=================================================================
\section{\arelire{Envoyer des messages}} % CHANGE spécial Pharo

\dothis{Ouvrez un espace de travail Workspace et saisissez-y
    le texte suivant:}

\begin{code}{}
BouncingAtomsMorph new openInWorld
\end{code}

\dothis{Maintenant \actclickz. Un menu devrait
  apparaître. Sélectionnez l'option \menu{do it (d)} (en français,
  ``faîtes-le!'') comme le montre la \figref{doit}.}

\begin{figure}[htb]
\centerline {\includegraphics[width=0.8\textwidth]{Doit}}
\caption{Évaluer une expression avec \menu{do it}.\figlabel{doit}}
\end{figure}

Une fenêtre contenant un grand nombre d'\bamfr
(en anglais, ``\emph{bouncing atoms}'') s'ouvre dans le coin supérieur
gauche de votre image \pharo.
%A window containing a large number of bouncing atoms should open in the top left of the \pharo image.

Vous venez tout simplement d'évaluer votre première expression \st. 
%You have just evaluated your first \st expression!
Vous avez juste envoyé le message 
 \ct{new} à la classe \bam ce qui résulte de la création d'une
 nouvelle instance qui à son tour \arevoir{reçoit le message}
 \ct{openInWorld}. % REVOIR dans le texte orig. pas de notion de
                   % receveur mais on parle uniquement d'envoyer (dur
                   % à reformuler) - martial
% You just sent the message \ct{new} to the \bam class, resulting in a new \bam instance, followed by the message \ct{openInWorld} to this instance.
La classe \bam{} a décidé de ce qu'il fallait faire avec le message
\ct{new}: elle recherche dans ces \emph{méthodes} pour répondre de
façon appropriée au message \ct{new}
% vf
\arevoir{(\ie{} ``nouveau'' en français; ce que nous traduirons par
  \emph{nouvelle instance})}.
% The \bam class decided what to do with the \ct{new} message, that is, it looked up its \emph{methods} for handling \ct{new} message and reacted appropriately.
De même, l'instance \bam recherchera dans ces méthodes comment
répondre à \ct{openInWorld}.
% Similarly the \bam instance looked up its method for responding to \ct{openInWorld} and took appropriate action.

Si vous discutez avec des habitués de \st, vous constaterez rapidement
qu'ils n'emploient généralement pas les expressions comme ``faire appel
à une opération'' ou ``invoquer une méthode'': ils diront ``envoyer
un message''.
% If you talk to Smalltalkers for a while, you will quickly notice that they generally do not use expressions like ``call an operation'' or ``invoke a method'', but instead they will say ``send a message''.
Ceci reflète l'idée que les objets sont responsables de leurs propres
actions.
% This reflects the idea that objects are responsible for their own actions. 
Vous ne \emph{direz} jamais à un objet quoi faire\,---\,vous lui
\emph{demanderez} polimment de faire quelque chose en lui envoyant un
message.
% You never \emph{tell} an object what to do\,---\,instead you politely \emph{ask} it to do something by sending it a message. 
C'est l'objet, et non pas vous, qui choisit la méthode appropriée pour
répondre à votre message.
% The object, not you, selects the appropriate method for responding to your message.

%=================================================================
\section{Enregistrer, quitter et redémarrer une session \pharo.}

\dothis{\arelire{\Clickz{} sur la fenêtre de démo des \bamfr{} et
    déplacez-la où vous voulez. Vous avons maintenant la démo ``dans
    la main''. Posez-la en \clickant.}}

\begin{figure}[htb]
\begin{minipage}[b]{0.48\textwidth}
	{\centerline{\includegraphics[width=0.7\textwidth]{atoms}}}
	\caption{Une instance de \bam.\figlabel{atoms}}
\end{minipage}
\hfill
\begin{minipage}[b]{0.48\textwidth}
	{\centerline{\includegraphics[width=0.7\textwidth]{saveAs}}}
	\caption{La bo\^{\i}te de dialogue \menu{save as\ldots}.\figlabel{saveas}}
\end{minipage}
\end{figure}

\dothis{Sélectionnez \menu{World\go{}Save as\ldots} et entrez le nom
  ``myPharo'', puis \clickz sur le bouton \button{OK}.
%ajout
pour sauvegarder sous un nouveau nom d'image.
Pour quitter, sélectionnez \menu{World\go{}Save and quit}.} % CHANGE

Le dossier qui contenait les fichiers image et \emph{changes} lorsque
vous avez lancé cette session de travail avec \pharo contient d\'esormais
deux nouveaux fichiers: ``myPharo.\ind{image}'' et ``myPharo.\ind{changes}''. 
%image vivante = working state of the pharo image
Ils repr\'esentent l'image ``vivante'' de votre session \pharo au moment qui précédait votre enregistrement avec \menu{Save and quit}.
Ces deux fichiers peuvent être copiés à votre convenance dans les
dossiers  de votre disque pour y être utilisés plus tard. \`A vous de
les invoquer en prenant soin (selon votre syst\`eme de fichiers) de
%note de martial: j'ai transforme pour plus de logique mais c'est
%lourd: a revoir
d\'eplacer, copier ou lier le fichier \emph{.source} correspondant,
tout en veillant \`a ex\'ecuter la bonne machine virtuelle.

\dothis{Lancez \pharo avec l'image que vous venez de créer \cad le
  fichier ``myPharo.image''.}

Vous retrouvez l'état de votre session exactement tel qu'il était
avant que vous quittiez \pharo. La démo des \bamfr{} est toujours sur votre
fenêtre de travail, en train de se déplacer d'o\`u vous l'aviez
abandonné.

En lançant pour la première fois \pharo, la \ind{machine virtuelle}
charge le fichier image que vous spécifiez. Ce fichier contient
l'instantané d'un grand nombre d'objets et surtout le code
pré-existant accompagné des outils de développement qui sont
d'ailleurs des objets comme les autres. En travaillant dans \pharo, vous
allez envoyer des messages à ces objets, en créer de nouveaux, et
certains seront supprimés et l'espace-mémoire utilisé sera récupéré
(\ie pass\'e au ramasse-miettes ou \emph{garbage collector}).

En quittant \pharo vous sauvegardez un instantané de tous vos objets. En sauvegardant (par ``Save''), vous remplacerez l'image courante par l'instantané de votre session. Pour préserver l'image courante, vous devez enregister sous un nouveau nom comme nous venons de le faire.

Chaque fichier \emph{.image} est accompagné d'un fichier \emph{.changes}.
Ce fichier contient un journal de toutes les modifications que vous avez faites en utilisant l'environnement de développement.
Vous n'avez pas à vous soucier de ce fichier la plupart du temps.
Mais comme nous allons le voir plus tard, le fichier \emph{.changes} pourra être utilisé pour rétablir votre système \pharo à la suite d'erreurs.

L'image sur laquelle vous travaillez provient d'une image de \st-80 créée à la fin des années 1970.
Beaucoup des objets qu'elle contient sont là depuis des décennies!

Vous pourriez penser que l'utilisation d'une image est incontournable pour stocker et gérer des projets, mais comme nous le verrons bientôt il existe des outils plus adaptés pour gérer le code et travailler en équipe sur des projets.
Les images sont très utiles mais nous consid\'erons comme une pratique un peu dépassée et fragile pour diffuser et partager vos projets alors qu'il existe des outils tels que Monticello qui proposent de biens meilleurs moyens de suivre les évolutions du code et de le partager entre plusieurs développeurs.

\dothis{\arelire{\Metaclickz{} sur la fenêtre
  d'\bamfr\footnote{Souvenez-vous que vous pourriez avoir besoin
    d'activer l'option \ct{halosEnabled} dans le Preference Browser.}.}}

%martial: le choix des noms 'poignee' 'icone' ... pourra etre change
Vous verrez tout autour une collection d'icônes colorées nomm\'ee
\subind{Morphic}{halo} de \bam; l'ic\^one
% \emphsubind{halo}{ic\^one}
\subind{Morphic}{halo} est aussi appel\'ee \emphsubind{halo}{poignée}.
Cliquez sur la poignée rose p\^ale qui contient une croix; la fenêtre
de démo disparaît. % CHANGE
%(index Morphic vs halo) a revoir?
\seeindex{poignée}{halo, poignée}
\seeindex{Morphic!poignée}{halo, poignée}
\seeindex{icône}{halo, icône}
\seeindex{halo!icône}{halo, poignée}
\seeindex{Morphic!icône}{halo, icône}
\seeindex{halo}{Morphic, halo}

%=================================================================
\section{Les fenêtres Workspace et Transcript}
\seclabel{transcript}

% martial - REVOIR - j'ai choisi de merger les deux dothis
\dothis{Fermez toutes fenêtres actuellement ouvertes. 
Ouvrez un \ind{Transcript} (via le menu \menu{World \go{} Tools}) et un \ind{Workspace}.
Positionnez et redimensionnez le Transcript et le Workspace
  pour que ce dernier recouvre le Transcript.} % CHANGE
Vous pouvez redimensionner les fenêtres en glissant l'un de leurs
coins ou en \metaclickant qui affiche les poignées
\emph{halo}: utilisez alors l'icône jaune située en bas à droite.

Une seule fenêtre est active à la fois; elle s'affiche au premier plan
et \arelire{son contour est alors mis en relief}. % CHANGE

Le Transcript est un objet qui est couramment utilisé pour afficher
des messages du système. C'est un genre de ``console''.
% Sachez que l'affichage dans la fenêtre Transcript est extr\^ement
% lent, donc si vous la conservez ouverte et que vous y affichez des
% résultats, certaines opérations peuvent \^etre 10 fois plus lentes.
% De plus, le Transcript n'est pas conçu pour recevoir
% simultanément des messages à afficher provenant de plusieurs objets:
% il n'est pas prot\'eg\'e contre les acc\`es concourrants (en anglais,
% \emph{thread-safe}), donc vous pourriez \^etre t\'emoin de
% comportements \'etranges si plusieurs objets tentent d'\'ecrire de
% mani\`ere concourrante dans le Transcript. 
% ON: I think the transcript has been made thread-safe now, right?

%%%% martial: Sat Dec 15 14:13:47 CET 2007
Les fen\^etres de Workspace ou espace de travail sont destin\'ees \`a
y saisir vos expressions de code \st \`a exp\'erimenter.
Vous pouvez aussi les utiliser simplement pour taper une quelconque
note de texte \`a retenir, comme une liste de choses \`a faire (en
anglais, \emph{todo-list}) ou des instructions pour quiconque est
amen\'e \`a utiliser votre image.
Les Workspaces sont souvent employ\'es pour maintenir une
documentation \`a propos de l'image courante, comme c'est le cas
dans l'image standard pr\'ec\'edemment charg\'ee (voir
\figref{startup}).

% originellement 'hello world'
\dothis{Saisissez le texte suivant dans l'espace de travail Workspace:}
\begin{code}{}
Transcript show: 'hello world'; cr.
\end{code}

%ajout
Exp\'erimentez la s\'election
en double-\clickant{} dans l'espace de travail \`a diff\'erents points dans
le texte que vous venez de saisir.
% entire word, entire string, or the whole text ((diff: string and word?))
Remarquez comment un mot entier ou tout un texte est
s\'electionn\'e %selon l'endroit o\`u vous cliquez.
\arelire{selon que vous \clickz{} sur un mot, à la fin d'une chaîne de
caractères ou à la fin d'une expression entière.}

\dothis{S\'electionnez le texte que vous avez saisi puis \actclickz{}.
Choisissez \menu{do it (d)} 
%ajout
(dans le sens ``faites-le!'', \cad \emph{\'evaluer le code
  s\'electionn\'e})
dans le menu contextuel.}
Notez que le texte ``hello world''~\footnote{\arelire{NdT: C'est une tradition de la
  programmation: tout premier programme dans un nouveau langage de
  programmation consiste à afficher la phrase en anglais ``hello world''
  signifiant ``bonjour le monde''.}}
 appara\^{\i}t dans la
fen\^etre Transcript (voir \figref{helloworld}).
Refaites encore un \menu{do it (d)}
(Le \menu{(d)} dans l'option de menu \menu{do it (d)} vous indique que
le raccourci-clavier correspondant est \short{d}. Pour plus
d'informations, rendez-vous dans la prochaine section!).

\begin{figure}[htb]
\ifluluelse
	{\centerline {\includegraphics[width=\textwidth]{HelloWorld}}}
\caption{\arelire{Les fenêtres sont superposées. Le Workspace est actif.}\figlabel{helloworld}}
\end{figure}

% Vous venez d'\'evaluer votre premi\`ere
% expression \st!
% Vous avez seulement envoyé le message \ct{show: hello world} \`a
% l'objet \ct{Transcript} (\ct{show:} veut dire: afficher), suivi du
% message \ct{cr} 
% %ajout
% (qui a le sens de \emph{carriage return}, \cad retour-chariot
% permettant de forcer le passage \`a la ligne suivante).
% Le \ct{Transcript} d\'ecide ensuite de quoi faire avec ce message; il
% cherche parmi ses \emph{m\'ethodes} celles qui g\`erent une r\'eponse
% aux messages \ct{show:} et \ct{cr} et qui r\'eagissent de fa\c{c}on
% appropri\'ee.

%=================================================================
\section{Les raccourcis-clavier}

Si vous voulez \'evaluer une expression, vous n'avez pas besoin de
toujours passer par le menu accessible en \actclickant: les
raccourcis-clavier sont l\`a pour vous. Ils sont mentionn\'es
dans les expressions parenth\'es\'ees des options des menus. Selon
votre plateforme, vous pouvez \^etre amen\'e \`a presser l'une des
touches de modifications soit \texttt{Control}, \texttt{Alt},
\texttt{Command} ou \texttt{Meta} (nous les indiquerons de mani\`ere
g\'en\'erique par \short{\emph{touche}}).
\index{raccourci-clavier}
\seeindex{clavier!raccourci-clavier}{raccourci-clavier}
\seeindex{clavier!événement}{événement, clavier} % martial: pour
                                % Morphic surtout

\dothis{R\'e\'evaluez l'expression dans le Workspace en utilisant
  cette fois-ci le raccourci-clavier: \short{d}.}
\index{raccourci-clavier!do it}

En plus de \menu{do it}, vous aurez not\'e la pr\'esence de
\menu{print it} 
%ajout
(pour \'evaluer et afficher le r\'esultat dans le m\^eme espace de travail), 
de \menu{inspect it} (pour inspecter) et de \menu{explore it} (pour
explorer). 
Jetons un coup d'\oe il \`a ceux-ci.

\dothis{Entrez l'expression \ct{3 + 4} dans le Workspace. Maintenant
  \'evaluez en faisant un \menu{do it} avec le raccourci-clavier.}

Ne soyez pas surpris que rien ne se passe!
Ce que vous venez de faire, c'est d'envoyer le message \ct{+} avec
l'argument \ct{4} au nombre \ct{3}. Le r\'esultat \ct{7} aura
normalement \'et\'e calcul\'e et retourn\'e, mais puisque votre espace de
travail Workspace ne savait que faire de ce r\'esultat, la r\'eponse a
simplement \'et\'e jet\'ee dans le vide. Si vous voulez voir le
r\'esultat, vous devriez faire \menu{print it} au lieu
de \menu{do it}. En fait, \menu{print it} compile l'expression,
l'ex\'ecute et envoie le message \ct{printString} au r\'esultat puis
affiche la cha\^{\i}ne de caract\`ere r\'esultante.

\dothis{S\'electionnez \ct{3+4} et faites \menu{print it} (\short{p}).}
Cette fois, nous pouvons lire le r\'esultat que nous attendions (voir
\figref{printit}).
\index{raccourci-clavier!print it}

\begin{figure}[htb]
% \centerline {\includegraphics[width=0.4\textwidth]{PrintIt}}
\centerline {\includegraphics[width=0.8\textwidth]{PrintIt}}
\caption{Afficher le r\'esultat sous forme de cha\^{\i}ne de
  caract\`eres avec \menu{print it} plut\^ot que de simplement
  \'evaluer avec \menu{do it}.\figlabel{printit}}
\end{figure}

\needlines{3}
\begin{code}{@TEST}
3 + 4 --> 7
\end{code}
\noindent
Nous utilisons la notation \ct{-->} comme convention dans tout le
livre pour indiquer qu'une expression particuli\`ere donne un certain
r\'esultat quand vous l'\'evaluez avec \menu{print it}.

\dothis{Effacez le texte surlign\'e ``\ct{7}''; comme \pharo devrait l'avoir
  s\'electionn\'e pour vous, vous n'avez qu'\`a presser sur la touche
  de suppression (suivant votre type de clavier \texttt{Suppr.} ou
  \texttt{Del.}). S\'electionnez \ct{3+4} \`a nouveau et, cette fois,
  faites une inspection avec \menu{inspect it} (\short{i}).}
%ajout
\index{raccourci-clavier!inspect it}
\index{inspecteur}
\seeindex{Inspector}{inspecteur}

\noindent
Vous devriez maintenant voir une nouvelle fen\^etre appel\'ee
\emphind{inspecteur} avec pour titre 
 \ct{SmallInteger: 7} (voir \figref{inspectit}).
L'inspecteur ou (sous son nom de classe) Inspector est un outil
extr\^emement utile: il vous permet de naviguer et d'interagir avec
n'importe quel objet du syst\`eme.
Le titre nous dit que \ct{7} est une instance de la classe
\clsind{SmallInteger} 
%ajout
(classe des entiers sur 31 bits).
Le panneau de gauche nous offre une vue des variables d'instance de
l'objet en cours d'inspection. Nous pouvons naviguer entre ces
variables et le panneau de droite nous affiche leur valeur.
Le panneau inf\'erieur peut \^etre utilis\'e pour \'ecrire des
expressions envoyant des messages \`a l'objet.

\begin{figure}[htb]
\centerline {\includegraphics[width=\textwidth]{InspectIt}}
\caption{Inspecter un objet.\figlabel{inspectit}}
\end{figure}

\dothis{Saisissez \ct{self squared} dans le panneau inf\'erieur de
  l'inspecteur que vous aviez ouvert sur l'entier \ct{7} et faites un
  \menu{print it}.
%ajout
Le message \ct{squared} (carr\'e) va \'elever le nombre \ct{7} lui-m\^eme (\ct{self}).}

\needlines{2}
\dothis{Fermez l'inspecteur. Saisissez dans un Workspace le
  mot-expression \ct{Object} et explorez-le via \menu{explore it}
  (\short{I}, i majuscule).}
\index{raccourci-clavier!explore it}
\index{explorateur}
\seeindex{Explorer}{explorateur}

Vous devriez voir maintenant une fen\^etre intitul\'ee \clsind{Object}
contenant le texte \mbox{$\triangleright$ \ct{root: Object}}.
Cliquez sur le triangle pour l'ouvrir (voir \figref{exploreit}).

\begin{figure}[htb]
\centerline {\includegraphics[width=0.7\textwidth]{ExploreIt}}
\caption{Explorer \ct{Object}.\figlabel{exploreit}}
\end{figure}

Cet explorateur (ou Explorer) est similaire \`a l'inspecteur mais il
offre une vue arborescente d'un objet complexe.
Dans notre cas, l'objet que nous observons est la classe \ct{Object}.
Nous pouvons voir directement toutes les informations stock\'ees dans
cette classe et naviguer facilement dans toutes ses parties.

%=================================================================
% \section{\sqmap}
% \index{SqueakMap}

% %web-based catalogue
% \sqmap est un catalogue web des ``packages'' ou
% \ind{paquetage}{}s\,---\,applications et biblioth\`eques de programmes (dites
% aussi librairies)\,---\,que vous pouvez t\'el\'echarger dans votre
% image.
% Les paquetages sont h\'eberg\'es sur de nombreux serveurs de
% par le monde et sont maintenus par un grand nombre de personnes. Certains de ces paquetages ne fonctionnent que sur une version spécifique de \pharo.
% \lr{Maybe mention Package Universes (SqueakMap is not maintained anymore)}

% \dothis{Ouvrez \menu{World \go open\ldots \go \sqmap Package Loader}.}
% Vous aurez besoin d'une connection Internet pour que cela
% fonctionne. Au bout d'un certain temps, la fen\^etre du gestionnaire
% de chargement \sqmap devrait appara\^{\i}tre (voir \figref{sokoban}).
% Sur le c\^ot\'e gauche, vous pouvez voir une longue liste de
% paquetages. Le champ de saisie situ\'e dans le coin sup\'erieur gauche
% est un panneau de recherche pour vous aider \`a trouver ce que vous
% cherchez dans la liste.

% Saisissez ``\ind{Sokoban}'' dans ce champ de recherche et
%   tapez sur la touche \textsc{Entr\'ee}.
% Cliquer sur le triangle dirig\'e vers le nom du paquetage vous
% r\'ev\`ele une liste des versions disponibles. Quand un paquetage ou
% une version est s\'electionn\'e, des informations \`a leur sujet sont
% affich\'ees dans le panneau de droite.
% Naviguez dans la derni\`ere version du jeu \ct{Sokoban}.
% Activez le menu contextuel du panneau de liste en cliquant dans cet
% espace avec le \ind{bouton jaune} et choisissez \menu{install} pour
% installer le paquetage s\'electionn\'e
% (si \pharo se plaint qu'il n'est pas s\^ur que cette version du jeu
% fonctionne dans votre image, r\'epondez aux questions par ``yes'' 
% %ajout
% pour confirmer l'installation).
% Remarquez qu'une fois que le paquetage a \'et\'e install\'e, il est
% marqu\'e d'une ast\'erisque dans la liste du \sqmap Package Loader.

% \begin{figure}[htb]
% \ifluluelse
% 	{\centerline {\includegraphics[width=\textwidth]{SqueakMap}}}
% 	{\centerline {\includegraphics[width=0.8\textwidth]{SqueakMap}}}
% \caption{Utiliser \sqmap pour installer le jeu Sokoban.\figlabel{sokoban}}
% \end{figure}

% \dothis{Apr\`es avoir installé ce paquetage, d\'emarrez \ct{Sokoban}
%   en \'evaluant \ct{SokobanMorph random openInWorld} dans un Workspace
% %ajout
% (souvenez-vous de faire \menu{do it} sur toute la s\'election).}

% % You can also try the \ct{NsGame}; execute it using \ct{NsGame new openInWorld}.
% % ON: I could not find NsGame anywhere!

% Le panneau inf\'erieur gauche du \sqmap Package Loader fournit
% plusieurs possibilit\'es pour filtrer la liste des paquetages. Vous
% pouvez choisir de ne voir que les paquetages qui sont compatibles avec
% une version particuli\`ere de \pharo
% %ajout
% (\emph{Squeak versions}), 
% ou qui sont de la famille des jeux
% %ajout
% (\emph{Entertainment\go{}Games}), 
% \etc.

%=================================================================
\section{Le navigateur de classes Class Browser}

Le navigateur de classes nomm\'e
\ind{Class Browser}~\footnote{\arevoir{Ce navigateur est confusément référé sous les noms ``System Browser''
  ou ``Code Browser''. \pharo{} utilise l'implémentation
  \ind{OmniBrowser} du navigateur connue aussi comme ``OB'' ou
  ``Package Browser''. Dans ce livre, nous utiliserons simplement le
  terme de Browser ou, s'il y a ambiguïté, nous parlerons de
  navigateur de classes}.} est un des
outils-cl\'e pour programmer. % CHANGE
Comme nous le verrons bient\^ot, il y a plusieurs navigateurs ou
\emph{browsers} int\'eressants disponibles pour \pharo, mais c'est le
plus simple que vous pourrez trouver dans n'importe quelle image, que
nous allons utiliser ici. % REVOIR (toujours vrai?)
\seeindex{navigateur de classes}{Browser}
\seeindex{Class Browser}{Browser}

\dothis{Ouvrez un navigateur de classes en s\'electionnant \menu{World
    \go{} Class Browser}~\footnote{\arelire{Si votre Browser ne
      ressemble pas à celui visible sur \figref{classBrowser}, vous
      pourriez avoir besoin de changer le navigateur par défaut. Voyez
    \faqref{packagebrowser}}}.} 

\begin{figure}[htb]
\ifluluelse
	{\centerline {\includegraphics[width=\textwidth]{ClassBrowser1}}}
	{\centerline {\includegraphics[width=0.7\textwidth]{ClassBrowser1}}}
\caption{Le navigateur de classes (ou Browser) affichant la
  m\'ethode \ct{printString} de la classe Object.
\figlabel{classBrowser}}
\end{figure}

Nous pouvons voir un navigateur de classes sur \figref{classBrowser}.
La barre de titre indique que nous sommes en train de parcourir la
classe \clsind{Object}.

\`A l'ouverture du Browser, tous les panneaux sont vides except\'e
le premier \`a gauche.
Ce premier panneau liste tous les \emph{paquetages} 
% vf
(en anglais, \emph{packages})
connus; \arelire{elles contiennent des groupes de classes apparentées}.
%\index{catégorie}

\dothis{\clickz{} sur le paquetage \scatind{Kernel}.}
Cette manipulation permet l'affichage dans le second panneau de toutes les
classes du paquetage s\'electionn\'e.

\dothis{S\'electionnez la classe \clsind{Object}.}
D\'esormais les deux panneaux restants se remplissent.
Le troisi\`eme panneau affiche les \emph{protocoles} de la classe
s\'electionn\'ee.
Ce sont des regroupements commodes pour relier des m\'ethodes
connexes. Si aucun \ind{protocole} n'est s\'electionn\'e, vous devriez
voir toutes les m\'ethodes disponibles de la classe dans le
quatri\`eme panneau.

\dothis{S\'electionnez le protocole \protind{printing}, 
%ajout
protocole de l'affichage.}
Vous pourriez avoir besoin de faire d\'efiler (avec la barre de
d\'efilement) la liste des protocoles pour le trouver.
Vous ne voyez maintenant que les m\'ethodes relatives \`a
l'affichage.

\dothis{S\'electionnez la m\'ethode \mthind{Object}{printString}.}
D\`es lors, vous voyez dans la partie inf\'erieure du Browser
le code source de la m\'ethode \ct{printString} partag\'e par tous
les objets 
%ajout
(tous dérivés de la classe Object,
exception faite de ceux qui la surcharge).

%=================================================================
\section{Trouver les classes}

Il existe plusieurs moyens de trouver une classe dans \pharo.
Tout d'abord, comme nous l'avons vu plus haut, nous pouvons savoir (ou
deviner) dans quelle cat\'egorie elle se trouve et, de l\`a, naviguer
jusqu'\`a elle via le navigateur de classes.
\index{Browser}
\seeindex{Browser!trouver une classe}{classe, recherche}
\index{classe!recherche}
\seeindex{classe!trouver}{classe, recherche}

Une seconde technique consiste \`a envoyer le message \ct{browse}
(ce mot a le sens de ``naviguer'') \`a la classe, ce qui a pour effet
d'ouvrir un navigateur de classes sur celle-ci
%ajout
(si elle existe bien s\^ur).
Supposons que nous voulions naviguer dans la classe \clsind{Boolean}
(la classe des bool\'eens).

\dothis{Saisissez \ct{Boolean browse} dans un Workspace et faites un \menu{do it}.}
Un navigateur s'ouvrira sur la classe \ct{Boolean} (voir \figref{browseBoolean}).
Il existe aussi un \ind{raccourci-clavier} \short{b} (browse) que vous
pouvez utiliser dans n'importe quel outil o\`u vous trouvez un nom de
classe;
\index{raccourci-clavier!browse it}
s\'electionnez le nom de la classe 
%ajout
(\parex \ct{Boolean})
puis tapez \short{b}.

\dothis{Utilisez le raccourci-clavier pour naviguer dans la classe \ct{Boolean}.}

\begin{figure}[hbt]
	{\centerline {\includegraphics[width=\textwidth]{Kernel-objects-boolean}}}
\caption{Le navigateur de classes affichant la d\'efinition de la
  classe \ct{Boolean}.\figlabel{browseBoolean}}
\end{figure}

Remarquez \arelire{que nous voyons une \emph{définition de classe}
  quand la classe \ct{Boolean} est sélectionnée mais sans qu'aucun
  protocole ni aucune méthode ne le soit}
% martial: on utilise le PLURIEL lorsque l’action ou l’état peut être
% rapporté aux deux sujets ; et le SINGULIER, lorsqu’il ne peut être
% rapporté qu’à un seul à la fois.
(voir \figref{browseBoolean}).
Ce n'est rien de plus qu'un message \st ordinaire qui est envoy\'e \`a
la classe parente lui r\'eclamant de cr\'eer une sous-classe.
Ici nous voyons qu'il est demand\'e \`a la classe \ct{Object} de
cr\'eer une sous-classe nomm\'ee \ct{Boolean} sans aucune variables
d'instance, ni variables de classe ou ``pool dictionaries'' et de 
mettre la classe \ct{Boolean} dans la cat\'egorie \scatind{Kernel-Objects}.
% The lower pane shows the \emph{class comment} --- a piece of plain text describing the class.
\arelire{Si vous \clickz{} sur le bouton \button{?} en bas du panneau
  de classes, vous verrez le \subind{classe}{commentaire} de classe
  dans un panneau dédié comme le montre \figref{classComment}.} % CHANGE

% Le nouveau panneau en dessous nous montre le \emph{commentaire de
%   classe}\,---\,quelques paragraphes de texte d\'ecrivant la classe.
% Si vous cliquez sur le bouton \button{?} \`a la base du panneau des
% classes 
% %ajout
% (\cad le second),
% vous pouvez voir le \subind{classe}{commentaire} de classe dans un
% panneau d\'edi\'e.

% \ab{I thought that this was supposed to be a \emph{Quick} tour!  And here we are describing a tool that I have used maybe twice in 10 years!   In any case, this description should be deferred to the \textbf{Environment} chapter}
% \on{I don't see why.  I use the hierarchy browser a lot!  I think it is really useful to know from the beginning, to help you find your through the hierarchy.}

% Si vous souhaitez explorer la hi\'erarchie des h\'eritages de \pharo, le
% navigateur nomm\'e \emphind{Hierarchy Browser} vous y aidera.
% Ça peut \^etre utile si vous \^etes en train de chercher une
% sous-classe ou une super-classe inconnue d'une classe connue.
% Le Hierarchy Browser ou navigateur hi\'erarchique est similaire au Browser except\'e que la liste des classes est arrang\'ee comme
% une arborescence indent\'ee refl\'etant l'h\'eritage.

% \dothis{Cliquez sur le bouton \button{hierarchy} dans le navigateur de
%   classes lorsque la classe \ct{Boolean} est s\'electionn\'ee.}
% \noindent
% Il est r\'esulte l'ouverture d'un Hierarchy Browser affichant les
% super-classes et les sous-classes de \clsind{Boolean}.
% % (\figref{booleanhierarchybrowser}).
% Naviguez un peu dans la super-classe et les sous-classes imm\'ediates
% de \ct{Boolean}.

\begin{figure}[hbt]
\centerline {\includegraphics[width=\textwidth]{classComment}}
\caption{Le commentaire de classe de \ct{Boolean}.
\figlabel{classComment}}
\end{figure}

Souvent, la m\'ethode la plus rapide de trouver une classe consiste
\`a la rechercher par son nom. Par exemple, supposons que vous \^etes
\`a la recherche d'une classe inconnue qui repr\'esente les jours et
les heures.% dates and times.

\dothis{Placez la souris dans le panneau des paquetages
  du Browser et tapez \short{f} ou s\'electionnez \menu{find
    class\ldots (f)} dans le menu contextuel accessible en
  \actclickant.
Saisissez ``time'' 
%ajout
(\cad le temps, puisque c'est l'objet de notre qu\^ete) 
dans la bo\^{\i}te de dialogue et acceptez cette entr\'ee.} 
\noindent
Une liste de classes dont le nom contient ``time'' vous sera
pr\'esent\'ee (voir \figref{findit}). Choisissez-en une, disons,
\ct{Time}; 
%martial: ca fait longtemps qu'il n'y a plus ce comportement
un navigateur l'affichera avec un commentaire de classe
sugg\'erant d'autres classes pouvant \^etre utiles. Si vous voulez
naviguer dans l'une des autres classes, s\'electionnez son nom (dans
n'importe quelle zone de texte) et tapez \short{b}.
\index{raccourci-clavier!find\ldots}
\index{raccourci-clavier!browse it}

\begin{figure}[hbt]
\centerline{
	\includegraphics[width=0.45\textwidth]{FindIt}
	\hspace{1cm}
	\includegraphics[width=0.45\textwidth]{TimeClasses}
}
\caption{Rechercher une classe d'apr\`es son nom.\figlabel{findit}}
\end{figure}

Notez que si vous tapez le nom complet (et correctement capitalis\'e 
%ajout
\cad en respectant la casse)
de la classe dans la bo\^{\i}te de dialogue de recherche (find), le
navigateur ira directement \`a cette classe sans montrer aucune liste
de classes \`a choisir.

%=================================================================
\section{Trouver les m\'ethodes}
\seclabel{quick:methodFinder}

Vous pouvez parfois deviner le nom de la m\'ethode ou, tout au moins,
une partie de son nom plus facilement que le nom d'une classe.
Par exemple, si vous \^etes int\'eress\'e par la connaissance du temps
actuel, vous pouvez vous attendre \`a ce qu'il y ait 
%martial: phrase differente pour le sens en francais
une m\'ethode affichant le temps \emph{maintenant}: comme la langue de \st
est l'anglais et que \emph{maintenant} se dit ``now'', une m\'ethode
contenant le mot ``now'' a de forte chance d'exister.
Mais o\`u pourrait-elle \^etre?
L'outil \emphind{Method Finder} peut vous aider \`a la trouver.
\seeindex{Browser!trouver une méthode}{méthode, recherche}
\index{méthode!recherche}
\seeindex{méthode!trouver}{méthode, recherche}

\dothis{\arelire{Sélectionnez \menu{World \go{} Tools \go{} Method Finder}.}
Saisissez ``now'' dans le panneau sup\'erieur gauche et cliquez sur
\menu{accept} (ou tapez simplement la touche \textsc{Entr\'ee}).}
Le chercheur de m\'ethodes Method Finder affichera une liste de tous
les noms de m\'ethodes contenant la sous-cha\^{\i}ne de caract\`eres ``now''.  

Pour d\'efiler jusqu'\`a \ct{now} lui-m\^eme, tapez ``\ct{n}''; cette
astuce fonctionne sur toutes les zones \`a d\'efilement de n'importe
quelle fen\^etre. En s\'electionnant ``now'', le panneau de droite
vous pr\'esentera les classes qui d\'efinissent une m\'ethode
avec ce nom, comme le montre \figref{MethodFinder}.
S\'electionner une de ces classes vous ouvrira un navigateur sur
celle-ci.

\begin{figure}[hbt]
\centerline {\includegraphics[width=0.7\textwidth]{methodFinder-now}}
\caption{Le Method Finder affichant toutes les classes qui
  d\'efinissent une m\'ethode appel\'ee \ct{now}.
\figlabel{MethodFinder}}
\end{figure}

\`A d'autres moments, vous pourriez avoir en t\^ete qu'une m\'ethode
existe bien sans savoir comment elle s'appelle.
Le Method Finder peut encore vous aider! Par exemple, partons de la
situation suivante: vous voulez trouvez une m\'ethode qui transforme
une cha\^{\i}ne de caract\`eres en sa version majuscule, \cad qui
transforme \ct{'eureka'} en \ct{'EUREKA'}.

\dothis{Saisissez \ct{'eureka' . 'EUREKA'} dans le Method Finder,
  comme le montre \figref{methodFinder-example1}.}
\noindent
Le Method Finder vous sugg\`ere une m\'ethode qui fait ce
que vous voulez~\footnote{\arelire{Si une fenêtre s'ouvre soudain avec un
  message d'alerte à propos d'une méthode obsolète\,---\, le terme
  anglais est \emph{deprecated method}\,---\, ne paniquez pas: le
  Method Finder est simplement en train d'essayer de chercher parmi
  tous les candidats incluant ainsi les méthodes obsolètes. \Clickz{}
  alors sur le bouton~\button{Proceed}.}}.

Un ast\'erisque au d\'ebut d'une ligne dans le panneau de droite du
Method Finder vous indique que cette m\'ethode est celle qui a \'et\'e
effectivement utilis\'ee pour obtenir le r\'esultat requis.
Ainsi, l'ast\'erisque devant \ct{String asUppercase} vous fait savoir
que la m\'ethode \mthind{String}{asUppercase} 
%ajout
(traduisible par ``en tant que majuscule'')
d\'efinie dans la classe \clsind{String} 
%ajout
(la classe des cha\^{\i}nes de caract\`eres)
a \'et\'e ex\'ecut\'ee et a renvoy\'e le r\'esultat voulu.
Les m\'ethodes qui n'ont pas d'ast\'erisque ne sont que d'autres
m\'ethodes que celles qui retournent le r\'esultat attendu.
\cmind{Character}{asUppercase} n'a pas \'et\'e ex\'ecut\'ee dans notre
exemple, parce que \ct{'eureka'} n'est pas un caract\`ere de classe \clsind{Character}.

\begin{figure}[hbt]
	{\centerline {\includegraphics[width=\textwidth]{MethodFinder-example1}}}
\caption{Trouver une méthode par l'exemple.
\figlabel{methodFinder-example1}}
\end{figure}

Vous pouvez aussi utiliser le Method Finder pour trouver des
m\'ethodes avec plusieurs arguments; par exemple, si vous recherchez
une m\'ethode qui trouve le plus grand commun diviseur de deux
entiers, vous pouvez essayer de saisir \ct{25. 35. 5} comme exemple.
Vous pouvez aussi donner au Method Finder de multiples exemples pour
restreindre le champ des recherches; le texte d'aide situ\'e dans le
panneau inf\'erieure vous apprendra \arelire{comment faire}. % CHANGE

%=================================================================
\section{D\'efinir une nouvelle m\'ethode}
\seeindex{développement orienté tests}{Test Driven Development}
L'av\`enement de la m\'ethodologie de d\'eveloppement orient\'e tests
ou \emphind{Test Driven Development}\cite{Beck03a} a chang\'e la
fa\c{c}on d'\'ecrire du code.
L'id\'ee derri\`ere cette technique aussi appel\'ee TDD se r\'esume par l'\'ecriture
du test qui d\'efinit le comportement d\'esir\'e de notre
code \emph{avant} celle du code proprement dit.
\`A partir de l\`a seulement, nous \'ecrivons le code qui satisfait au
test.
\seeindex{développement dirigé par le comportement}{Test Driven Development} 
\seeindex{Behavior Driven Development}{Test Driven Development}


%says something loudly and with emphasis
Supposons que nous voulions \'ecrire une m\'ethode qui ``hurle quelque
chose''. Qu'est-ce que cela veut dire au juste? Quelle serait le nom
le plus convenable pour une telle m\'ethode? Comment pourrions-nous
\^etre s\^urs que les programmeurs en charge de la maintenance future
du code auront une description sans ambigu\"{\i}t\'e de ce que ce code
est cens\'e faire?
Nous pouvons r\'epondre \`a toutes ces questions en proposant
l'exemple suivant:

%martial: j'ai change figures/testShoutConfirm.png et j'ai remplace
%"Don't panic" par "Pas de panique"
\begin{quote}
Quand nous envoyons le message \ct{shout} (qui veut dire ``crier'' en anglais)
\`a la cha\^{\i}ne de caract\`eres ``Pas de panique'', le r\'esultat
devrait \^etre ``PAS DE PANIQUE!''.
\end{quote}

\noindent
Pour faire de cet exemple quelque chose que le syst\`eme peut
utiliser, nous le transformons en m\'ethode de test:
\index{test}
%\seeindex{testing}{test}
\index{SUnit}

\needlines{3}
\begin{method}[testShout]{Un test pour la m\'ethode shout}
testShout
	self assert: ('Pas de panique' shout = 'PAS DE PANIQUEBANG')
\end{method} % BANG is the escape for !

Comment cr\'eons-nous une nouvelle m\'ethode dans \pharo? Premi\`erement,
nous devons d\'ecider quelle classe va accueillir la m\'ethode.
Dans ce cas, la m\'ethode \ct{shout} que nous testons ira dans la
classe \clsind{String}
%ajout
car c'est la classe des cha\^{\i}nes de caract\`eres et ``Pas de panique'' en est une.
Donc, par convention, le test correspondant ira dans une classe
nomm\'ee \clsind{StringTest}.

\begin{figure}[hbt]
	{\centerline {\includegraphics[width=\textwidth]{StringTest-newMethodTemplate}}}
\caption{Le patron de la nouvelle m\'ethode dans la classe \ct{StringTest}.
\figlabel{newMethodTemplate}}
\end{figure}

\dothis{Ouvrez un navigateur de classes sur la classe
  \ct{StringTest}. S\'electionnez un protocole appropri\'e pour notre
  m\'ethode; dans notre cas, \menu{tests - converting} 
%ajout
(signifiant tests de conversion, puisque notre m\'ethode modifiera le texte en retour),
comme nous pouvons le voir sur \figref{newMethodTemplate}.
Le texte surlign\'e dans le panneau inf\'erieur est un patron de
m\'ethode qui vous rappelle ce \`a quoi ressemble une m\'ethode.
Effacez-le et saisissez le code de  \tmthref{testShout}.}
Une fois que vous avez commenc\'e \`a entrer le texte dans le
navigateur, l'espace de saisie est entour\'e de rouge pour vous
rappeler que ce panneau contient des changements non-sauvegard\'es.
%ajout
Lorsque vous avez fini de saisir le texte de la m\'ethode de test,
s\'electionnez \menu{accept (s)} via le menu activ\'e en \actclickant{} dans ce panneau ou utilisez le raccourci-clavier
\short{s}: ainsi, vous compilerez et sauvegarderez votre m\'ethode.
\index{raccourci-clavier}
\index{raccourci-clavier!accept}
\seeindex{accept it}{raccourci-clavier, accept}
%ajout
\seeindex{méthode!accepter}{raccourci-clavier, accept}

\arelire{Si c'est la première fois que vous acceptez du code dans
  votre image, vous serez invité à saisir votre nom dans une
  fenêtre spécifique. Beaucoup de personnes ont contribué au code de
  l'image; c'est important de garder une trace de tous ceux qui créent
  ou modifient les méthodes. Entrez simplement votre prénom suivi de
  votre nom sans espaces ou de point de séparation.}
% If this is the first time you have accepted any code in your image, you will likely be prompted to enter your name. Since many people have contributed code to the image, it is important to keep track of everyone who creates or modifies methods. Simply enter your first and last names, without any spaces, or separated by a dot.

%\begin{figure}[hbt]
%\centerline {\includegraphics[width=0.35\textwidth]{initials}}
%\caption{Saisir ses initiales.
%\figlabel{initials}}
%\end{figure}

Puisqu'il n'y a pas encore de m\'ethode nomm\'ee \ct{shout}, le 
Browser vous demandera confirmation que c'est bien le nom que vous
d\'esirez\,---\,il vous sugg\`erera d'ailleurs d'autres noms de
m\'ethodes existantes dans le syst\`eme (voir \figref{testShoutConfirm}).
Ce comportement du navigateur est utile si vous aviez effectivement
fait une erreur de frappe. Mais ici, nous voulons \emph{vraiment}
\'ecrire \ct{shout} puisque c'est la m\'ethode que nous voulons
cr\'eer. D\`es lors, nous n'avons qu'\`a confirmer cela en
s\'electionnant la premi\`ere option parmi celles du menu, comme vous
le voyez sur \figref{testShoutConfirm}. 

%\begin{figure}[htb]
%\begin{minipage}[b]{0.48\textwidth}
%\centerline {\includegraphics[width=0.9\textwidth]{name}}
%\caption{Saisir son nom.\figlabel{name}}
%\end{minipage}
%\hfill
% \begin{figure}[hbt]
% \ifluluelse
% 	{\centerline {\includegraphics[width=\textwidth]{testShoutConfirm}}}
% 	{\centerline {\includegraphics[width=0.8\textwidth]{testShoutConfirm}}}
% \caption{Accepter la m\'ethode testShout dans la classe \ct{StringTest}.
% \figlabel{testShoutConfirm}}
% \end{figure}

\begin{figure}[htb]
 \centerline {\includegraphics[width=0.6\textwidth]{name}}
 \caption{Saisir son nom.\figlabel{name}}
 \end{figure}

\begin{figure}[htb]
 	{\centerline {\includegraphics[width=\textwidth]{testShoutConfirm}}}
 \caption{Accepter la m\'ethode \ct{testShout} dans la classe \ct{StringTest}.
 \figlabel{testShoutConfirm}}
 \end{figure}

 \dothis{Lancez votre test nouvellement cr\'e\'e: ouvrez le programme
   \ind{SUnit} nomm\'e \emphind{TestRunner} depuis le menu \menu{World}.}

 Les deux panneaux les plus \`a gauche se pr\'esentent un peu comme les
 panneaux sup\'erieurs du Browser. Le panneau de gauche contient
 une liste de cat\'egories restreintes aux cat\'egories qui
 contiennent des classes de test.

\dothis{S\'electionnez \scat{CollectionsTests-Text} et
le panneau juste à droite vous affichera alors toutes les classes de test
de cette cat\'egorie dont la classe \ct{StringTest}.
Les classes sont déjà séléctionnées dans cette catégorie; \clickz{}
alors sur \menu{Run Selected} pour lancer tous ces tests.} % CHANGE

\begin{figure}[hbt]
\centerline {\includegraphics[width=\textwidth]{testRunnerOnStringTest}}
\caption{Lancer les tests de \ct{String}.
\figlabel{testRunnerTestShout}}
\end{figure}

Vous devriez voir un message comme celui de
\figref{testRunnerTestShout}, vous indiquant qu'il y a eu une erreur
lors de l'ex\'ecution des tests. La liste des tests qui donne
naissance \`a une erreur est affich\'ee dans le panneau inf\'erieur de
droite; comme vous pouvez le voir, c'est bien
\ct{StringTest>>>#testShout} le coupable
(remarquez que la notation \ct{StringTest>>#testShout} est la fa\c{c}on dont \st
identifie la m\'ethode de la classe \ct{StringTest}).
Si vous \clickz{} sur cette ligne de texte, le test erron\'e sera
lanc\'e \`a nouveau mais, cette fois-ci, de telle fa\c{c}on que vous
voyez l'erreur surgir:
``\ct{MessageNotUnderstood: ByteString>>>shout}''.
\seeindex{\ct{>>}}{Behavior, \ct{>>}}
\cmindex{Behavior}{>>}

La fen\^etre qui s'ouvre avec le message d'erreur est le d\'ebogueur \st (voir \figref{predebugger}).
Nous verrons le d\'ebogueur nomm\'e \ind{Debugger} et ses
fonctionnalit\'es dans \charef{env}.
\seeindex{Debugger}{débogueur}

\begin{figure}[hbt]

	\centerline {\includegraphics[width=\textwidth]{Predebugger}}
\caption{La fenêtre de démarrage du d\'ebogueur.}
\figlabel{predebugger}
\end{figure}

L'erreur \'etait bien s\^ur attendue; lancer le test g\'en\`ere une
erreur parce que nous n'avons pas encore \'ecrit la m\'ethode qui dit
aux cha\^{\i}nes de caract\`eres comment hurler 
%ajout pour le francais
\cad comment r\'epondre au message \ct{shout}.
De toutes fa\c{c}ons, c'est une bonne pratique de s'assurer que le test
\'echoue; cela confirme que nous avons correctement
configur\'e notre machine \`a tests % testing machinery
et que le nouveau test est actuellement en cours d'ex\'ecution.
Une fois que vous avez vu l'erreur, vous pouvez cliquer sur le bouton
\button{Abandon} pour abandonner le test en cours, ce qui fermera la
fen\^etre du d\'ebogueur.
Sachez qu'en \st vous pouvez souvent d\'efinir la m\'ethode manquante
%ajout
directement depuis le d\'ebogueur 
en utilisant le bouton \button{Create}, en y \'editant la m\'ethode
nouvellement cr\'e\'ee puis, \emph{in fine}, en appuyant sur le bouton
\button{Proceed} pour poursuivre le test.

D\'efinissons maintenant la m\'ethode qui fera du test un succ\`es!

\dothis{S\'electionnez la classe \clsind{String} dans le 
  Browser et rendez-vous dans le protocole 
%ajout
d\'ej\`a existant des m\'ethodes de conversion et appel\'e
\menu{converting}. \`A la place du patron de cr\'eation de m\'ethode,
saisissez le texte de \tmthref{shout} et faites \menu{accept}
(saisissez \caret pour obtenir un \mbox{\ct{^}})}
\begin{method}[shout]{La m\'ethode shout}
shout
	^ self asUppercase, 'BANG'
\end{method}

La virgule est un op\'erateur de concat\'enation de cha\^{\i}nes de
caract\`eres, donc, le corps de cette m\'ethode ajoute un point
d'exclamation \`a la version majuscule
%martial: ajout pour rappeler au francais que uppercase eleve au majuscule
(obtenue avec la m\'ethode \mthind{String}{asUppercase})
de l'objet \ct{String} auquel le message \ct{shout} a \'et\'e
envoy\'e.
Le $\uparrow$ dit \`a \pharo que l'expression qui suit est la r\'eponse
que la m\'ethode doit retourner; dans notre cas, il s'agit de la
nouvelle cha\^{\i}ne concat\'en\'ee.
\seeindex{virgule}{Collection, opérateur virgule}
\index{Collection!opérateur virgule}

Est-ce que cette m\'ethode fonctionne? Lan\c{c}ons tout simplement
notre test afin de le savoir.

\dothis{Cliquez encore sur le bouton \menu{Run Selected} du Test
  Runner. Cette fois vous devriez obtenir une barre de signalisation
  verte (et non plus rouge) et son texte vous confirmera que tous les
  tests lanc\'es se feront sans aucun \'echec (ni \emph{failures}, ni
  \emph{errors}).}
Vous voyez une barre verte~\footnotemark\ dans le Test Runner? Bravo!
Sauvegardez votre image et faites une pause. 
%martial: ajout (ca fait toujours plaisir!)
Vous l'avez bien m\'erit\'e. 
% \footnotetext{En r\'ealit\'e, vous pourriez ne pas obtenir de barre
%   verte car certaines images contiennent des tests pour des
%   \emph{bugs} \`a corriger. Ne vous inqui\'etez pas!
% \pharo est en perp\'etuelle \'evolution.
% }

\begin{figure}[hbt]
	{\centerline{\includegraphics[width=0.7\textwidth]{String-Shout}}}
\caption{La m\'ethode \ct{shout} dans la classe \ct{String}.\figlabel{String-shout}}
\end{figure}

%=================================================================
\section{R\'esum\'e du chapitre}
%martial: j'ai mis 'intronise' si trop solennel, corriger par 'presente'
Dans ce chapitre, nous vous avons introduit \`a l'environnement de
\pharo et nous vous avons montr\'e comment utiliser certains de ses
principaux outils comme le   Browser, le
Method Finder et le Test Runner. Vous avez pu avoir un aper\c{c}u de la
syntaxe sans que vous puissiez encore la comprendre suffisamment \`a ce stade.

\begin{itemize}
  \item Un syst\`eme \pharo fonctionnel comprend une \emph{machine
      virtuelle} (souvent abr\'eg\'ee par VM), un fichier
    \emph{sources} et un couple de fichiers: une \emph{image} et un
    fichier \emph{changes}. Ces deux derniers sont les seuls \`a
    \^etre susceptibles de changer, puisqu'ils sauvegardent un clich\'e
    du syst\`eme actif.
  \item Quand vous restorez une image \pharo, vous vous retrouvez
    exactement dans le m\^eme \'etat\,---\,avec les m\^emes objets
    lanc\'es\,---\,que lorsque vous l'avez laiss\'ee au moment de votre derni\`ere
    sauvegarde de cette image.
  \item \pharo est destin\'e \`a fonctionner avec une souris \`a trois
    boutons \arelire{pour \click, \actclick ou \metaclick}.
 Si vous n'avez pas de souris \`a trois boutons, vous pouvez utiliser
 des touches de modifications au clavier pour obtenir le m\^eme effet.
  \item Vous \clickz sur l'arri\`ere-plan de
    \pharo pour faire appara\^{\i}tre le \emph{menu World} et pouvoir
    lancer depuis celui-ci divers outils.
  \item Un \emph{Workspace} ou espace de travail est un outil
    destin\'e \`a \'ecrire et \'evaluer des fragments de code. Vous
    pouvez aussi l'utiliser pour y stocker un texte quelconque.
  \item Vous pouvez utiliser des raccourcis-clavier sur du texte
    dans un Workspace ou tout autre outil pour en
    \'evaluer le code. Les plus importants sont \menu{do it}
    (\short{d}), \menu{print it} (\short{p}), \menu{inspect it}
    (\short{i}) et \menu{explore it} (\short{I}).
\index{raccourci-clavier}
  \item \sqmap est un outil pour t\'el\'echarger des paquetages utiles
    depuis Internet.
  \item Le navigateur de classes \emph{Browser} est le
    principal outil pour naviguer dans le code \pharo et pour
    d\'evelopper du nouveau code.
  \item Le \emph{Test Runner} permet d'effectuer des tests
    unitaires. Il supporte pleinement la m\'ethodologie de
    programmation orient\'ee tests connue sous le nom de \emph{Test
      Driven Development}.
\end{itemize}

%=================================================================
\ifx\wholebook\relax\else 
   \bibliographystyle{jurabib}
   \nobibliography{scg}
   \end{document}
\fi
%=================================================================

%%% Local Variables:
%%% coding: utf-8
%%% mode: latex
%%% TeX-master: t
%%% TeX-PDF-mode: t
%%% ispell-local-dictionary: "english"
%%% End:

