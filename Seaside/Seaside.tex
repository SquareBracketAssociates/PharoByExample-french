% $Author: oscar $
% $Date: 2009-08-16 20:11:18 +0200 (Sun, 16 Aug 2009) $
% $Revision: 28482 $
% $Id: Seaside.tex 28482 2009-08-16 18:11:18Z oscar $

% HISTORY:
% 2007-10-29 - Oscar started chapter
% 2007-11-30 - Oscar first draft
% 2007-12-07 - Orla Greevy reviewed
% 2007-12-09 - Lukas Renggli reviewed
% 2008-01-11 - Andrew revised
% 2009-04-17 - Fabrizio Perin reviewed
% 2009-04-18 - Jorge Ressia reviewed
% 2009-05-06 - Oscar converted to Pharo; fixed review comments
% 2009-11-18 - Martial started the french translation
% 2010-01-08 - Rene started the first rereading
% 
% sync avec la version: 29667
%
%=================================================================
\ifx\wholebook\relax\else
% --------------------------------------------
% Lulu:
	\documentclass[a4paper,10pt,twoside]{book}
	\usepackage[
		papersize={6.13in,9.21in},
		hmargin={.75in,.75in},
		vmargin={.75in,1in},
		ignoreheadfoot
	]{geometry}
	\input{../common.tex}
	\pagestyle{headings}
	\setboolean{lulu}{true}
% --------------------------------------------
% A4:
%	\documentclass[a4paper,11pt,twoside]{book}
%	\input{../common.tex}
%	\usepackage{a4wide}
% --------------------------------------------
    \localizedgpath{{figures/lang/fr/}}%
    {{figures/lang/en/}}%
    {{figures/} {../figures/}}
	\begin{document}
	% \renewcommand{\nnbb}[2]{} % Disable editorial comments
	\sloppy
\fi
%=================================================================
\chapter{Seaside par l'exemple}
\chalabel{seaside}

%=================================================================

\ind{Seaside} est une bibliothèque destinée à la construction
d'applications web en Smalltalk. 
Elle a été initialement développée par Avi Bryant \index{Bryant, Avi} 
en 2002; 
une fois maîtrisé, Seaside permet la création d'applications
web aussi facilement que se fait l'écriture d'applications classiques.
%once mastered, Seaside makes web applications almost as easy to write as desktop applications.

Deux des applications basées sur Seaside les plus connues sont 
 \ind{SqueakSource}\footnote{\url{http://SqueakSource.com}} et \ind{Dabble DB}\footnote{\url{http://DabbleDB.com}}.
Seaside n'est pas commun dans le sens qu'il est résolument orienté
objet: il n'y a aucune structure de type \emph{template} XHTML, aucun
flux de contrôle complexe par des pages web, ni même encodage de
l'état dans les URLs. Vous envoyez simplement des messages aux
objets. Quelle merveilleuse idée!

\section{Pourquoi avons-nous besoin de Seaside?}

Les applications web modernes essayent d'intéragir avec l'utilisateur
de la même façon que les applications de bureau: elles interrogent
l'utilisateur et celui-ci leur répond, habituellement en remplissant
un formulaire ou en cliquant sur un bouton.
Mais le web fonctionne d'une autre manière: le navigateur de
l'utilisateur fait des requêtes au serveur et le serveur répond avec
une nouvelle page web.
Dès lors les bibliothèques (ou \emph{frameworks}) destinées au
\ind{développement d'applications web} doivent contourner une foule de
problèmes, le principal étant la gestion du flux de contrôle ``inversé''.
% control flow
C'est pour cette raison que beaucoup d'applications web tentent
d'interdire l'usage du bouton \backbtn (ou ``back'' en anglais, nommé
\emph{précédent(e)} sur certaines navigateurs) du fait de la
difficulté de garder une trace de l'état d'une session.
Diffuser des flux de contrôle significatifs entre plusieurs pages web 
est souvent lourd et la diffusion de plusieurs flux peut être difficile.
%Expressing non-trivial control flows across multiple web pages is often cumbersome, 
%and multiple control flows can be difficult or impossible to express.

% Seaside is a component-based framework that uses ``\ind{continuations}''\footnote{A \emph{continuation} represents ``the rest of the computation'' at any point in a computation. In Smalltalk, a continuation is just an object that captures the current state of the computation, and which can be resumed at any point.} to keep track of multiple points in the control flow of web applications. Continuations are managed automatically by Seaside, so web developers do not even have to be aware of the underlying machinery. It just works.

\index{Seaside!\backtracking}
%\index{Seaside!backtracking state}
\index{Seaside!transactions}
\index{Seaside!composant}
%\index{Seaside!components}
Seaside est une bibliothèque orienté composant
% component-based 
qui allège le développement web de plusieurs façons.
Tout d'abord, le flux de contrôle
% control flow
peut être exprimé naturellement en utilisant les envois de messages.
\arevoir{Seaside garde une trace de la correspondance entre page web et 
un point précis dans l'exécution de l'application web.}
% Seaside keeps track of which web page corresponds to which point in
% the execution of the web application.
Donc le bouton \backbtn du navigateur web fonctionnement correctement.

Deuxièmement, la gestion de l'état est automatique. % 
%Second, state is managed for you.
En tant que développeur, vous aurez le choix de permettre le
\backtracking (\ie le chaînage arrière) de l'état ou non, ainsi la
navigation en \backbtn{} annulera les effets de bord.
% backtracking of state, so that navigation ``back'' in time will undo side-effects.
Vous pouvez autrement utiliser le \arelire{support de transaction}
inclus dans Seaside pour empêcher les utilisateurs d'annuler les
effets de bord permanents lorsqu'ils utilisent le bouton \backbtn.
Vous n'avez pas à encoder l'information de l'état dans
l'URL\,---\,Seaside s'en occupe pour vous. 

Ensuite, les pages web sont générées depuis les composants imbriqués,
chacun pouvant avoir son propre flux de contrôle indépendant.
% control flow
Il n'y a pas de \emph{templates} XHTML\,---\,le code XHTML valide est
généré de façon programmatique via un simple protocole Smalltalk.
% generated programmatically using a simple Smalltalk protocol.
Seaside supporte les feuilles de style en cascade 
 (\ind{CSS}), ainsi le contenu et la mise en page sont clairement
 séparés.
\seeindex{feuille de style en cascade}{CSS} % REVOIR
%\seeindex{Cascading Style Sheets}{CSS}

Finalement, Seaside offre une interface web pratique pour le
développement en facilitant la programmation itérative, le débogage
interactif et la recompilation et l'extension des applications alors
que le serveur reste en activité.

%=================================================================
\section{Démarrer avec Seaside}

Télécharger le logiciel ``Seaside \subind{Seaside}{One-Click
  Experience}'' depuis le \subind{Seaside}{site
  web}\footnote{\url{http://seaside.st}} est la façon la plus simple
pour commencer avec Seaside.
Il s'agit d'une \arevoir{version empaqueté}
% proposition de René : version packagée 
% prepackaged version 
de Seaside 2.8 pour \ind{Mac OS X}, \ind{Linux} and \ind{Windows}.
Le site web de Seaside liste aussi de nombreux liens vers des
ressources complémentaires dont de la documentation et des
tutoriels (en anglais).
Sachez, cependant, que Seaside a considérablement évolué durant ces
dernières années et toutes ces ressources disponibles ne se rapportent
pas nécessairement à la dernière version de Seaside.

% If you are feeling more adventurous, an alternative to the ``one-click'' image is to start with the latest \ind{\pharo web image}\footnote{\url{http://pharo-project.org/download}}, and install Seaside yourself by following the manual \subind{Seaside}{installation} instructions on the Seaside web site.
\mb{It should be nice to explain how to load it from
  Monticello. The reader should be able to do this at this step.}

Seaside inclut un serveur web: vous pouvez lancer ce serveur sur le
port 8080 en évaluant \clsind{WAKom} \ct{startOn: 8080} et le stopper
en évaluant \ct{WAKom stop}.
Dans l'installation par défaut, le \subind{Seaside}{compte
  administrateur} (en anglais, \emph{administrator login}) est
\lct{admin} et le mot de passe (ou \emph{password}) est \lct{seaside}.
Pour les changer, évaluez: 
 \clsind{WADispatcherEditor} \ct{initialize}.
Ceci affichera une fenêtre de dialogue pour la saisie d'un nouveau nom
et d'un nouveau mot de passe. 

\begin{figure}[tbh]
\begin{center}
\includegraphics[width=\textwidth]{seasideStartup}
\caption{La page de démarrage de Seaside.}
\figlabel{seasideStartup}
\end{center}
\end{figure}

\dothis{Démarrez le serveur Seaside et rendez-vous sur la page
  \url{http://localhost:8080/seaside/} avec votre navigateur favori.}

\noindent
Vous devriez voir une page web semblable à celle de \figref{seasideStartup}.

\noindent
\dothis{Naviguez vers la page \lct{examples{\go}counter} (voir \figref{counter}).}


\begin{figure}[htb]
\begin{center}
\includegraphics[width=0.8\textwidth]{counter}
\caption{La démo du compteur avec l'application ``counter''.}
\figlabel{counter}
\end{center}
\end{figure}

\noindent
Cette page est une petite application Seaside nommée
\subind{Seaside}{counter}: elle affiche un
compteur qui peut être incrémenté et décrementé en cliquant sur les
liens \link{++} et \link{--\,--}.

\noindent
\dothis{Amusez-vous un peu avec l'application ``counter'' en cliquant sur
  ces liens.
Utiliser le bouton \backbtn du navigateur pour revenir à l'état
précédent et cliquer ensuite sur \link{++}.
Remarquez comment le compteur est correctement incrémenté depuis
l'état actuellement affiché au lieu d'être incrémenté depuis l'état
dans lequel le compteur était lorsque vous aviez commencé à utiliser
le bouton \backbtn.}

Notez la 
% \subind{Seaside}{toolbar}
\subind{Seaside}{\toolbar} en bas de la page web dans 
 \figref{seasideStartup}.
Seaside supporte la notion de ``sessions'' pour suivre l'état de
l'application pour différents utilisateurs.
\button{New Session} démarre une nouvelle session sur l'application
``counter''.
\button{Configure} vous permet de 
de configurer les paramètres de votre application via une interface
web (pour fermer la vue \button{Configure}, cliquez sur le \link{x}
dans le côté supérieur droit de la page).
\button{Toggle Halos} propose d'explorer l'état de l'application
lancée sur le serveur Seaside.
\button{Profiler} et \button{Memory} offre une information détaillée 
à propos des performances à l'exécution de l'application.
\button{XHTML} peut être utilisé pour valider la page web générée;
ce lien ne fonctionne que lorsque la page est accessible depuis
Internet car il utilise le service de validation W3C.
\index{Seaside!halos}


Les applications Seaside sont construites à l'aide de
\arevoir{``composants'' connectables}.
% pluggable ``components''.
Ces composants sont en réalité des objets Smalltalk ordinaires.
Leur seule particularité est qu'elles doivent être des instances des
classes héritées de la classe de la bibliothèque Seaside appelée
\ct{WAComponent}.
Nous pouvons explorer les composants et leurs classes dans l'image
Pharo ou directement, depuis l'interface web, en utilisant les halos.


\begin{figure}[ht]
\begin{center}
\includegraphics[width=\textwidth]{counterHalos}
\caption{Les halos dans Seaside.}
\figlabel{counterHalos}
\end{center}
\end{figure}

\dothis{Sélectionnez \button{Toggle Halos}. Vous devriez voir une 
page web comme celle de \figref{counterHalos}.
Dans le coin supérieur gauche, le texte \ct{WACounter} nous indique le
nom de la classe du composant Seaside qui implémente le comportement
de cette page. \arelire{À côté de ce texte,} il y a trois icônes
cliquables.
Le premier, représentant un crayon, active un navigateur de classes
Seaside sur la classe \ct{WACounter}. Le second icône avec une loupe
ouvre un inspecteur d'objets sur l'instance \ct{WACounter} en cours
d'activité.
Le dernier, sous la forme de cercles colorés, affiche la feuille de
style \ind{CSS} de ce composant.
Dans le coin supérieur droit, le 
 \link{R} et le  \link{S} vous laissent basculer entre la vue du
 code rendue et celle du code source.
Essayez donc en cliquant sur ces différents liens.
Remarquez que les liens \link{++} et \link{--} sont aussi actifs dans
la vue du code source.
\mb{More and more browsers, Safari/Firefox, provide pretty-print code view!}
\arevoir{Observez d'ailleurs le formatage en couleurs du code source fourni
par les halos Seaside \aretirer{avec} la vue du code source non-formatée fournie
par votre navigateur.}}

Le navigateur de classes Seaside et l'inspecteur d'objets peuvent être
très pratiques lorsque le serveur tourne sur un autre ordinateur,
surtout si celui-ci n'a pas d'écran ou qu'il se trouve dans un lieu
distant.
Néanmoins, lorsque vous commencez à développer une application
Seaside, le serveur sera lancé localement et il sera plus facile
d'utiliser les outils de développement standards disponibles dans
l'image \pharo dans laquelle votre serveur est en activité.
 
\begin{figure}[ht]
\begin{center}
\includegraphics[width=0.7\textwidth]{haltingCounter}
\caption{Suspendre l'application ``counter''.}
\figlabel{haltingCounter}
\end{center}
\end{figure}

\dothis{Cliquez sur le lien 
 Object Inspector représenté par l'icône de loupe du halo
pour ouvrir un inspecteur d'objets sur \arelire{l'objet-compteur
  Smalltalk} et évaluez \ct{self halt} (en cliquant sur 
\button{doit}). % REVOIR
La page web s'arrêtera de charger. Allez maintenant dans votre image
Seaside. Vous devriez voir une fenêtre \arevoir{de pré-débogueur} % ATTENDRE changement dans le debugger de Pharo?
affichant un objet \ct{WACounter} exécutant un \ct{halt} (voir
\figref{haltingCounter}).
Examinez cette exécution dans le débogueur
% ajout vf - rappel
en cliquant sur la première ligne de la pile,
puis cliquez sur \button{Proceed}.
Retourner dans votre navigateur web et notez que l'application ``counter''
fonctionne à nouveau.}

Les composants Seaside peuvent être instanciés plusieurs fois et dans
différents contextes.

\begin{figure}[ht]
\begin{center}
\includegraphics[width=\textwidth]{multiCounterHalos}
\caption{Les sous-composants indépendants.}
\figlabel{multiCounterHalos}
\end{center}
\end{figure}

\dothis{Allez sur la page
  \url{http://localhost:8080/seaside/examples/multicounter avec votre
    navigateur web}.
Vous verrez une application construite grâce à un certain nombre
d'instances indépendantes du composant ``counter''.
Incrémentez et décrémentez plusieurs de ces compteurs.
Vérifiez que ces compteurs se comportent correctement même lorsque
vous utilisez le bouton \backbtn.
Activez les halos pour voir que la structure de l'application est
faite de composants imbriqués.
Utilisez le navigateur de classes Seaside pour visualiser
l'implémentation de \ct{WAMultiCounter}.
Vous devriez voir trois méthodes du côté classe (\ct{canBeRoot},
\ct{description} et \ct{initialize}) ainsi que trois méthodes du côté
instance (\ct{children}, \ct{initialize} et \ct{renderContentOn:}).
\arelire{Remarquez qu'une application est simplement un composant
qui peut être à la racine de la hiérarchie de composants; cette
possibilité de devenir racine, et donc application, est indiquée en
définissant une méthode de classe \ct{canBeRoot} (en anglais, ``peut
être racine'') pour qu'elle réponde
\ct{true}.}}
\index{Seaside!multicounter}

Vous pouvez utiliser l'interface web Seaside, copier ou enlever une
ou plusieurs applications (\ie les composants racines). Essayez de
changer la configuration comme suit.

\dothis{Allez sur \url{http://localhost:8080/seaside/config} avec
  votre navigateur.
Saisissez le \emph{login} et le mot de passe 
 (\ct{admin} et \ct{seaside} par défaut).
Sélectionnez 
 \link{Configure} à côté de ``examples''.
Sous l'entête  ``Add entry point'' (\ie{} \emph{ajouter un point
  d'entrée}),
saisissez le nouveau nom ``counter2'' pour le type \emph{Application} 
et cliquez sun le bouton \button{Add} comme le montre \figref{counter2}.
Sur  l'écran suivant, choisissez \clsind{WACounter} comme composant racine 
 \emph{Root Component} puis, cliquez sur \button{Save} et enfin,
 \button{Close}.
Nous avons maintenant un nouveau compteur installé à l'URL
 \url{http://localhost:8080/seaside/examples/counter2}.
Utilisez la même interface de configuration pour enlever ce point
d'entrée.}
\index{Seaside!configuration}


\begin{figure}[ht]
\begin{center}
\includegraphics[width=\textwidth]{counter2}
\caption{Configurer une nouvelle application.}
\figlabel{counter2}
\end{center}
\end{figure}

Seaside fonctionne dans deux modes: soit le mode de développement dit 
\emph{development} qui est le mode dans lequel nous venons de travailler,
soit dans le mode de diffusion appelé \emph{deployment} dans lequel la
\toolbar n'est pas accessible. 
\index{Seaside!mode!deployment}
\index{Seaside!mode!development}
Vous pouvez mettre Seaside dans le mode \emph{deployment} en utilisant
soit la page de configuration\,---\,pour cela naviguer jusqu'à
l'entrée de l'application et cliquez sur le lien \link{Configure}\,---\,soit
% \ab{How?  I couldn't find this}
cliquez directement sur le bouton  \button{Configure} dans la
\toolbar.
Dans chacun de ces cas, mettez le mode \emph{deployment} à
\emph{true}.
Notez que ceci n'affecte que les nouvelles sessions.
Vous pouvez aussi activer ce mode en évaluant 
\clsind{WAGlobalConfiguration} \lct{setDeploymentMode}, et
inversement, réactiver le mode \emph{development} en évaluant
\ct{WAGlobalConfiguration setDevelopmentMode}.
\index{Seaside!deployment mode}
\index{Seaside!development mode}

En fait, la page web de configuration n'est qu'une autre application
Seaside. Ainsi elle peut aussi être pilotée depuis la page de
configuration. Si vous enlevez l'application
``config'', vous pouvez toujours la réinstaller en évaluant
\clsind{WADispatcherEditor} \ct{initialize}.

%=================================================================
\section{Les composants Seaside}
\seclabel{components}
\seeindex{Seaside!component}{Seaside, composant}

%\ab{This section was too long\,---\,18 pages.  It also contained several self-references (``see section 1.3''). So I broke into smaller sections, by promoting some of the subsections and subsubsections.}

Comme nous l'avons mentionné précédemment, les applications de Seaside sont bâties sur une aggrégation de
  \subind{Seaside}{composants}\,---\,en anglais \emph{component}.
%As we mentioned in the previous section, Seaside applications are built out of \emph{\subind{Seaside}{components}.}
Regardons comment Seaside fonctionne en programmant le composant
``hello world''.
%\emph{Hello World}.

Tout composant Seaside doit hérité directement ou indirectement de
\clsind{WAComponent} comme le montre la \figref{WACounter}.

\dothis{Définissez une sous-classe de \ct{WAComponent} appelée
  \ct{HelloWorld}.}

Les composants doivent savoir comment faire \arelire{leur propre rendu
  visuel}.
Ceci est fait habituellement en implémentant la méthode
\mthind{WAPresenter}{renderContentOn:} qui prend comme argument une
instance de  \clsind{WAHtmlCanvas}; ce dernier sait comment faire le
rendu en XHTML.
\index{Seaside!rendu}
%\index{Seaside!rendering}

\dothis{Implémentez la méthode suivante et mettez-la dans le protocole
  \prot{rendering}:}
\needlines{2}
\begin{code}{}
WAHelloWorld>>>renderContentOn: html
	html text: 'hello world'
\end{code}

\noindent
Nous devons dire maintenant à Seaside que ce composant est destiné à
être autonome
% ajout vf - plus clair
\arelire{(donc une possible racine de l'arborescence de composants).} % CHANGE

\dothis{Écrivez la méthode de classe suivante pour \ct{WAHelloWorld}.}

\begin{code}{}
WAHelloWorld class>>>canBeRoot
	^ true
\end{code}

\noindent
Nous avons fini!

\dothis{Pointez votre navigateur web
  \url{http://localhost:8080/seaside/config}, ajoutez une nouvelle
  entrée qui vous appelerez ``hello'' et choisissez \ct{WAHelloWorld}
  comme composant racine.}
Pointez maintenant votre navigateur 
à l'URL \url{http://localhost:8080/seaside/hello}.
Vous y êtes! Vous devriez voir une page web comme sur
 \figref{WAHelloWorld}.

\begin{figure}[htb]
\begin{center}
\includegraphics[width=\textwidth]{WAHelloWorld}
\caption{``hello world'' en Seaside.}
\figlabel{WAHelloWorld}
\end{center}
\end{figure}

%-----------------------------------------------------------------
\subsection{Le \backtracking d'état et l'application ``counter''}
\seeindex{Seaside!chaînage arrière}{Seaside!\backtracking} % REVOIR
%{Simple and nested components}

L'application ``counter'' est seulement un peu plus compliquée que
l'application ``hello world''.
\seclabel{backtracking}

\begin{figure}[ht]
\begin{center}
\includegraphics[width=\textwidth]{WACounter}
\caption{La classe \ct{WACounter} implémentant l'application
  ``counter''. Les méthodes avec les noms soulignés sont des méthodes
  de classe; les autres sont des méthodes d'instance.}
\figlabel{WACounter}
\end{center}
\end{figure}

La classe  \clsind{WACounter} est une \arelire{application autonome},
ainsi \ct{WACounter class} doit répondre \ct{true} au message 
 \mthind{WAComponent class}{canBeRoot}.
Elle doit aussi être enregistrée en tant qu'application; sa méthode de
classe \ct{initialize} est utilisée pour cela (voir
\figref{WACounter}).

\ct{WACounter} définit deux méthodes, \ct{increase} et \ct{decrease},
qui seront déclenchés par les liens \link{++} et \link{--\,--} sur la
page web.
Cette classe définit aussi une variable d'instance 
 \ct{count} pour enregistrer l'état du compteur.
Cependant, nous voulons aussi que Seaside synchronise le compteur avec la page du navigateur web:
lorsque l'utilisateur clique sur le bouton \backbtn{} de son
navigateur web, nous voulons que Seaside rappelle l'état de l'objet
\ct{WACounter}.
%when the user clicks on the browser's ``back'' button, we want seaside to ``backtrack'' the state of the \ct{WACounter} object.
Seaside inclut un mécanisme général pour le chaînage arrière ou
\backtracking mais chaque application doit signaler à Seaside quelle
partie de son état doit être suivie.

Un composant permet le \backtracking en implémentant la méthode
d'instance \ct{states}:
% \ab{note that xspace messes up again, by inserting a space at the start of this line}
\ct{states} devrait répondre un tableau contenant tous les objets à
suivre. Dans notre cas, l'objet \ct{WACounter} s'ajoute lui-même à la
table Seaside des objets à suivre pour le \backtracking en retournant
\ct{Array with: self}.

\paragraph{\emph{Avertissement.}}
Il y a un point subtil mais important à surveiller lorsque vous
déclarez des objets pour le \backtracking.
Seaside suit l'état en faisant une \arelire{\emph{copie}}
% \emph{copy} % REVOIR - martial - nom de méthode??
de tous les objets déclarés dans le tableau \ct{states}.
Ceci se fait via l'objet
 \clsind{WASnapshot}; \ct{WASnapshot} est une sous-classe
de \clsind{IdentityDictionary} qui enregistre les objets à être suivis
en tant que clés et des \arelire{copies superficielles 
% ajout vf
(\ie{} en \ct{shallowCopy})} en tant que valeurs.
\arelire{S'il y a \backtracking{} de l'état d'une application vers une
  capture particulière de l'état, l'état de chaque objet entré dans ce
dictionnaire est écrasé par la copie sauvegardée dans la capture.}
%If the state of an application is backtracked to a particular snapshot, the state of each object entered into the snapshot dictionary is overwritten by the copy saved in the snapshot.

\tradalert{martial}{revoir ce paragraphe.}

Voilà le point critique:
Dans le cas de \ct{WACounter}, vous pourriez vous dire que l'état à
être suivi est une valeur numérique\,---\,la valeur de la variable
d'instance \ct{count}.
Cependant, la méthode \ct{states} répondant \ct{Array with: count} ne
marche pas.
%However, having the \ct{states} method answer \ct{Array with: count}
%won't work. 
Ceci est du au fait que l'objet \ct{count} est un entier et un entier
est immuable.
%This is because the object named by \ct{count} is an integer, and integers are immutable.
Les méthodes \ct{increase} et \ct{decrease} ne changent pas l'état de
l'objet \ct{0} en \ct{1} ou l'objet \ct{3} en \ct{2}.
Au lieu de ça, elles associent \ct{count} à un autre entier:
%Instead, they make \ct{count} name a different integer: 
chaque fois que le compteur est incrémenté ou décrémenté, l'objet
nommé \ct{count} est \emph{remplacé} par un autre.
C'est pourquoi \ct{WACounter>>>states} doit retourner \ct{Array with: self}.
Lorsque l'état d'un objet \mbox{\ct{WACounter}} est remplacé par un
état précédent, la \emph{valeur} de chacune des variables d'instance
dans l'objet est remplacée par une valeur précédente; ceci remplace
correctement la valeur actuelle de \ct{count} par une valeur
précédente.
\index{Seaside!\backtracking} % CHANGE
%\index{Seaside!backtracking state}
\index{WAPresenter!states@\ct{states}}

\section{Le rendu XHTML}

Le but d'une application web est la création, ou ``
:
%\ab{Too long!}

Maintenant nous allons changer l'implémentation du \ct{display} pour
afficher la valeur du sommet de la pile.

\dothis{%
Utilisez une table HTML avec la classe ``keypad'' contenant une ligne
avec une seule cellule de classe ``stackcell''
% ajout - vf - martial - plus clair comme ça:
dans \ct{MyDisplay>>>renderContentOn:}.
Changez maintenant la méthode de rendu de \ct{keypad} pour que le
nombre \ct{0} soit mis sur la pile dans le cas où celle-ci est vide
%Change the rendering method of the keypad to ensure that the number 0 is pushed on the stack in case it is empty.
(définissez et utilisez \ct{MyKeypad>>>ensureMyStackMachineNotEmpty}).
Faites en sorte que \ct{keypad} affiche un table vide de classe
``keypad''.
Le calculateur devrait afficher une seule cellule contenant la valeur 0.
Si vous activez les halos, vous devriez voir quelque chose comme suit:}

\begin{figure}[ht]
\begin{center}
\includegraphics[width=0.8\textwidth]{firstStackDisplay}
\caption{Affichage du sommet de la pile.}
\figlabel{firstStackDisplay}
\end{center}
\end{figure}

Implémentons désormais une interface pour interagir avec la pile.

\dothis{
Définissez tout d'abord les méthodes adjointes facilitant l'écriture
de l'interface:
}

\needlines{3}
\begin{code}{}
MyKeypad>>>renderStackButton: text callback: aBlock colSpan: anInteger on: html 
	html tableData
		class: 'key';
		colSpan: anInteger;
		with: 
				[html anchor
					callback: aBlock;
					with: [html html: text]]
\end{code}


\begin{code}{}
MyKeypad>>>renderStackButton: text callback: aBlock on: html 
	self 
		renderStackButton: text
		callback: aBlock
		colSpan: 1
		on: html
\end{code}

Nous utiliserons ces deux méthodes pour définir les boutons du
\ct{keypad} avec les \callbacks appropriés.
Certains boutons peuvent s'étaler sur plusieurs colonnes
% ajout - vf
(avec \ct{colSpan:}) mais, par défaut, ils occupent une seule colonne.

\dothis{%
Utiliser ces deux méthodes pour écrire le \ct{keypad} comme suit
(une astuce: commencez par faire en sorte que les chiffres et du bouton
``Enter'' fonctionnent 
% ajout - vf
\arelire{(en laissant la méthode \ct{setClearMode})} % REVOIR -
                                % martial - à reporter dans la VO
                                % parce que pas gagné! 
avant de vous attardez aux opérateurs arithmétiques):}

\needlines{4}
\begin{code}{}
MyKeypad>>>renderContentOn: html 
  self ensureStackMachineNotEmpty.
  html table
    class: 'keypad';
    with: [
      html tableRow: [
          self renderStackButton: '+' callback: [self stackOp: #add] on: html.
          self renderStackButton: '&ndash;' callback: [self stackOp: #min] on: html.
          self renderStackButton: '&times;' callback: [self stackOp: #mul] on: html.
          self renderStackButton: '&divide;' callback: [self stackOp: #div] on: html.
          self renderStackButton: '&plusmn;' callback: [self stackOp: #neg] on: html ].
        html tableRow: [
          self renderStackButton: '1' callback: [self type: '1'] on: html.
          self renderStackButton: '2' callback: [self type: '2'] on: html.
          self renderStackButton: '3' callback: [self type: '3'] on: html.
          self renderStackButton: 'Drop' callback: [self stackOp: #pop]
          	colSpan: 2 on: html ].
" et ainsi de suite ... "
        html tableRow: [
          self renderStackButton: '0' callback: [self type: '0'] colSpan: 2 on: html.
          self renderStackButton: 'C' callback: [self stackClearTop] on: html.
          self renderStackButton: 'Enter'
          	callback: [self stackOp: #dup. self setClearMode]
			colSpan: 2 on: html ]]
\end{code}

Vérifiez que le \ct{keypad} s'affiche proprement.
Si vous essayez de cliquer sur les boutons, vous verrez que la
calculatrice ne marche pas encore\ldots{}

\dothis{%
Implémentez \ct{MyKeypad>>>type:} pour mettre à jour le sommet de la
pile en annexant les chiffres saisis.
Vous aurez besoin de convertir la valeur du sommet de la pile en
chaîne de caractères, la mettre à jour
% ajout - vf
en la concatenant avec l'argument de \ct{type:} et la convertir enfin
en entier, en faisant de la sorte:}
\begin{code}{}
MyKeypad>>>type: aString
	stackMachine push: (stackMachine pop asString, aString) asNumber.
\end{code}
Lorsque vous cliquez sur les boutons-chiffre, l'affichage devrait être
mis à jour
(soyez donc sûr que la méthode \ct{MyStackMachine>>>pop} retourne la
valeur sortie de la pile, sinon ça ne marchera pas!).


\dothis{Now we must implement \ct{MyKeypad>>>stackOp:}
Something like this will do the trick:}

\begin{code}{}
MyKeypad>>>stackOp: op
	[ stackMachine perform: op ] on: AssertionFailure do: [ ].
\end{code}

Nous ne sommes pas sûrs du succès de toutes les opérations. Par
exemple, une addition peut échouer si nous n'avons pas deux nombres
sur la pile.
Pour l'instant, nous pouvons simplement ignorer de telles erreurs.
\arelire{%
Si nous nous sentons plus ambitieux plus tard, nous pourrons ajouter
un retour informatif à l'utilisateur dans un bloc de gestion d'erreur.}
%If we are feeling more ambitious later on, we can provide some user feedback in the error handler block.

\dothis{La première version de la calculatrice devrait fonctionner
  maintenant. Essayez d'entrer des nombres en cliquant sur les
  boutons-chiffre, puis
cliquez sur \menu{Enter} pour dupliquer la valeur actuelle, et cliquez
enfin sur \menu{+} pour faire l'addition de ces deux valeurs.}

Vous remarquerez que la saisie des chiffres ne se passent pas comme
prévu.
En fait, la calculatrice devrait savoir si vous saisissez un
\emph{nouveau} nombre ou que vous complétez une nombre existant.

\dothis{Adaptez la méthode \ct{MyKeypad>>>type:} pour se comporter
  différemment suivant le mode actuel de saisie.
Introduisez une variable d'instance \ct{mode} pouvant prendre trois
valeurs \lct{\#typing} (lorsque vous saisissez), \lct{\#push} (après que
vous ayez effectué une opération de calcul \arevoir{et que la saisie
  devrait forcer l'entrée de la valeur sur la pile}) 
%after you have performed a calculator operation and typing should
%force the top value to be pushed)
ou \lct{\#clear} (après que vous ayez cliqué sur le bouton \menu{Enter}
et que \arevoir{le sommet de la pile devrait se mettre dans l'état
  initial avant la prochaine saisie}).
% and the top value should be cleared before typing).
La nouvelle méthode \ct{type:} pourrait ressembler à ceci:}

\tradalert{martial}{pour TODO-cb, type: devrait utiliser
  "stackMachine dup." au lieu de "stackMachine push: stackMachine
  top."; j'ai déjà changé en prévision...}
\begin{code}{}
MyKeypad>>>type: aString
	self inPushMode ifTrue: [
		stackMachine dup.
		self stackClearTop ].
	self inClearMode ifTrue: [ self stackClearTop ].
	stackMachine push: (stackMachine pop asString, aString) asNumber.
\end{code} % ATTENDRE

La saisie devrait mieux fonctionner maintenant mais le fait de ne
pouvoir voir la pile intégralement est frustrant.
%Typing might work better now, but it is still frustrating not to be able to see what is on the stack.

\dothis{%
Définissez la classe \ct{MyDisplayStack} comme une sous-classe de
\ct{MyDisplay}. Ajoutez un bouton dans la méthode de rendu de
\ct{MyDisplay} qui appelera une nouvelle instance de
\ct{MyDisplayStack}.
Vous aurez besoin d'un lien HTML (ou \emph{anchor}) ressemblant à ceci:}

\begin{code}{}
html anchor
	callback: [ self call: (MyDisplayStack new setStackMachine: stackMachine)];
	with: 'open'
\end{code}
% ATTENDRE setMyStackMachine: dans la VO, setStackMachine c'est plus
% logique

Le \callback{} entraînera le remplacement temporaire de l'actuelle
instance de \ct{MyDisplay} en une nouvelle instance de la classe
\ct{MyDisplayStack} dont le travail consiste à afficher la pile
entière.
Lorsque ce composant signale que le travail est fini (en envoyant
\ct{self answer}), l'instance originel de \ct{MyDisplay} reviendra
dans le \arelire{flux}.
%then control will return to the original instance of \ct{MyDisplay}.

\dothis{%
Définissez la méthode de rendu de la classe \ct{MyDisplayStack} pour
qu'elle affiche \arelire{toutes les valeurs sur la pile
(vous aurez besoin de définir soit un accesseur \ct{contents} pour
atteindre le contenu de la machine à pile, soit une méthode
\ct{MyStackMachine>>>do:} pour itérer sur les valeurs de la pile)}.
L'interface de la pile devra aussi proposer un bouton ``close'' dont
le \callback{} effectuera simplement en \ct{self answer}.}
%The stack display should also have a button labelled ``close'' whose callback will simply perform \ct{self answer}.


\begin{code}{}
html anchor
	callback: [ self answer];
	with: 'close'
\end{code}

Vous devriez maintenant être capable de ouvrir (avec \menu{open}) et de
fermer (avec \menu{close}) l'interface \ct{display} de la pile durant
l'utilisation de la calculatrice.
%Now you should be able to \emph{open} and \emph{close} the stack while you are using the calculator.

Il y a cependant une chose que nous avons oubliée.
Essayez d'effectuer des opérations sur la pile.
Utilisez maintenant le bouton \backbtn{} de votre navigateur web et
essayez encore d'effectuer des opérations sur la pile (\parex
ouvrez la pile, saisissez \menu{1} une fois et \menu{Enter} deux fois
puis \menu{+}. La pile devrait afficher ``2'' et ``1''. Cliquez
maintenant sur la bouton \backbtn. La pile montre encore trois fois
``1''. Mais si vous cliquez sur \menu{+}, la pile affichera ``3''
% ajout - vf
au lieu de ``2''. Le \backtracking{} ne marche pas encore).

\dothis{%
Codez la méthode \ct{MyCalculator>>>states} pour retourner 
% ajout - vf
un tableau contenant le contenu \ct{contents} de la machine à pile.
Vérifiez que le \backtracking{} fonctionne correctement désormais!}

Détendez-vous et prenez un rafraîchissant: vous l'avez bien mérité!
%Sit back and enjoy a tall glass of something cool!

%=================================================================
\section{Un bref coup d'\oe il sur la technologie AJAX}

% Original text by Lukas Renggli

\ind{AJAX} (Asynchronous \ind{JavaScript} and \ind{XML}) est une
technique pour créer des applications web plus interactive en
exploitant les fonctionalités offertes par \jscript du côté client.

Deux bibliothèques \jscript{} populaires sont \ind{\pjs}\footnote{\url{http://www.prototypejs.org}.} et \ind{\sau}\footnote{\url{http://script.aculo.us}.}.


%Prototype provides a framework to ease writing JavaScript.
% CHANGE - retrait - vf - martial: phrase non traduite parce que redondant
\sau{} étend la librairie \pjs{} en ajoutant des fonctionnalités pour
l'animation et le glissé-déposé (\emph{drag-and-drop}).
%script.aculo.us provides some additional features to support
%animations and drag-and-drop on top of Prototype.
Ces deux bibliothèques sont incluses dans Seaside via la paquetage
\scat{Scriptaculous}.

Toutes les images prêtes à l'emploi ont ce paquetage déjà chargé. La
dernière version est disponible sur
\url{http://www.squeaksource.com/Seaside}.
Une démo en ligne est visible à l'adresse 
\url{http://scriptaculous.seasidehosting.st}.
Si vous avez une image actuellement lancée, pointez tout simplement
votre navigateur web sur la page
\url{http://localhost:8080/seaside/tests/scriptaculous}.

Les extensions \sau{} suivent la même approche que Seaside
lui-même\,---\,configurez des objets Smalltalk pour modéliser votre
application et le code \jscript{} nécessaire sera généré pour vous.

Jetons un \oe il sur un simple exemple pour voir comment le support
\jscript{} côté client peut rendre le réactivité de notre calculatrice
RPN plus naturelle.
%of how client-side Javascript support can make our RPN calculator behave more naturally.
Actuellement tout clic sur un chiffre entraîne une requête pour
rafraîchir la page. Nous aimerions plutôt gérer l'édition de
l'affichage côté client en mettant à jour l'affichage 
% ajout - vf (plus clair)
de la partie \ct{display} de la page existante.
%We would like instead to handle editing of the display on the client-side by updating the display in the existing page.

\dothis{Pour pouvoir communiquer depuis \jscript{} avec des éléments
  précis de l'interface, nous devons tout d'abord donner à ces
  éléments un attribut \emph{id} unique.
%To address the display from JavaScript code we must first give it a unique id.
Changez la méthode de rendu de la calculatrice\footnote{Si
  vous n'avez pas implémenté le tutoriel vous-même, vous pouvez \aretirer{%
  télécharger l'exemple complet \scat{PBE-SeasideRPN} depuis
  \url{http://www.squeaksource.com/PharoByExample} et} appliquer les
modifications suggérées aux classes en \ct{RPN*} au lieu des classes
en \ct{My*}.} comme suit:}

\begin{code}{}
MyCalculator>>>renderContentOn: html
	html div id: 'keypad'; with: keypad.
	html div id: 'display'; with: display.	
\end{code}
				
\dothis{%
% martial - confusion keyboard et keypad dans la version de Lukas -
% REVOIR à signaler dans TODO-cb
Pour pouvoir refaire le rendu du \ct{display} lorsqu'un bouton du
\ct{keypad} est pressé, le composant \ct{keypad} a besoin de connaître
le composant \ct{display}.
%To be able to re-render the display when a keyboard button is
%pressed, the keyboard needs to know the display component.
Ajoutez une variable d'instance \ct{display} à la classe \ct{MyKeypad}
et une méthode d'initialisation \ct{MyKeypad>>>setDisplay:} et
utilisez-la dans la méthode \ct{MyCalculator>>initialize}.
Nous sommes maintenant capable d'adjoindre du code \jscript aux
boutons en mettant à jour
\ct{MyKeypad>>>renderStackButton:callback:colSpan:on:} comme suit:}

\begin{code}{}
MyKeypad>>>renderStackButton: text callback: aBlock colSpan: anInteger on: html 
	html tableData
		class: 'key';
		colSpan: anInteger;
		with: [
			html anchor
				callback: aBlock;
				onClick:				"!gère! les !événements! JavaScript"
					(html updater
						id: 'display';
						callback: [ :r |
							aBlock value.
							r render: display ];
						return: false);
				with: [ html html: text ] ]
\end{code}

\mthind{WATagBrush}{onClick:} indique une gestionnaire d'événements
\ind{\jscript}.
%\ind{JavaScript} event handler.
\ct{html updater} renvoie un \updater \ie une instance de
\ct{SUUpdater}, un objet Smalltalk représentant l'objet JavaScript
Ajax.Updater (\url{http://www.prototypejs.org/api/ajax/updater}).
Cet objet fait une requête AJAX et met à jour, par le texte en
réponse, le contenu d'un conteneur.
%updates a container's contents based on the response text.
Le message \ct{id:} dit à l'\updater quel élément DOM XHTML mettre à jour; ici,
il s'agit du contenu de l'élément \emph{div} d'attribut \emph{id}
``display''.
Un bloc est passé en argument à \ct{callback:} pour se déclencher
quand l'utilisateur cliquera sur le bouton.
L'argument de bloc \ct{r} est \arelire{une nouvelle interface de rendu ou
  \emph{renderer}} qui peut être utilisé pour le rendu du composant
\ct{display}
(remarquez que \arelire{même si le code HTML est toujours accessible,
  il n'est plus valide au moment où ce \callback est évalué}).
%(Even though html is still accessible, it is not valid anymore at the
%time this callback block is evaluated).
Avant de faire le rendu du composant \ct{display}, nous évaluons
\ct{aBlock} pour effectuer l'action désirée.

\ct{return: false} interdit au moteur \jscript de déclencher le
\callback d'origine du lien, ce qui engendrerait un rafraîchissement
complet. %JavaScript engine
Nous pouvons aussi retirer l'ancre d'origine \ct{callback:}, mais en
le laissant, nous sommes sûrs que la calculatrice marchera même si le
\jscript est désactivé.

\dothis{Essayez la calculatrice à nouveau et remarquez que le
  rafraîchissement complet de la page se produit toujours lorsque vous
  cliquez sur un chiffre du \ct{keypad} (autrement dit, l'URL de la
  page web change à chaque clic).}

Bien que nous ayons bien implémenté le comportement du côté client,
nous ne l'avons pas encore activé.
Nous devons donc permettre la gestion des événements \jscript.

\dothis{%
Cliquez sur le lien \link{Configure} dans la \toolbar de la
calculatrice.
Sélectionnez ``Add Library:'' \ct{SULibrary} (pour configurer l'ajout de la
bibliothèque \ct{SULibrary}) et cliquez sur \button{Add} puis
\button{Close}.}

Vous pouvez aussi ajouter la bibliothèque de manière programmatique
(plutôt que manuellement) lorsque vous enregistrez l'application:
\begin{code}{}
MyCalculator class>>>initialize
	(self registerAsApplication: 'rpn')
		addLibrary: SULibrary}}
\end{code}

\begin{figure}[ht]
\begin{center}
\includegraphics[width=\textwidth]{ajax-processing}
\caption{Diagramme de séquences simplifié des interactions AJAX dans notre application Seaside.}
%\caption{Seaside AJAX processing (simplified)}
\figlabel{ajax-processing}
\end{center}
\end{figure}

\dothis{Essayez l'application revisitée. 
Notez que la réponse est beaucoup plus naturelle. En particulier,
aucune nouvelle URL n'est générée lors du clic.}

Vous devez vous demander: ``oui mais, comment ça marche?''.
%You may well ask, \emph{yes, but how does this work?}
\Figref{ajax-processing} montre comment les deux versions\,---\,avec et
sans AJAX\,---\,de l'application RPN (avec et sans AJAX) fonctionnent.
AJAX court-circuite simplement le rendu de façon à mettre à jour le
composant \ct{display} \emph{uniquement}. 
\jscript est responsable à la fois du déclenchement de la requête et
de la mise à jour de l'élément DOM correspondant.
Regardons le code source généré, principalement le code \jscript.

\begin{code}{}
new Ajax.Updater(
	'display',
	'http://localhost/seaside/RPN+Calculator',
	{'evalScripts': true,
	  'parameters': ['UNDERSCOREs=zcdqfonqwbeYzkza', 'UNDERSCOREk=jMORHtqr','9'].join('&')});
return false
\end{code}

Pour des exemples plus avancés, nous vous invitons à visiter la page
\url{http://localhost:8080/seaside/tests/scriptaculous} avec votre
navigateur web.

\paragraph{\emph{Astuces.}}
En cas de soucis du côté serveur, servez-vous du débogueur
Smalltalk. Pour parer aux problèmes côté client, utiliser FireFox
(\url{http://www.mozilla.com}) et FireBug, son débogueur \jscript
(\url{http://www.getfirebug.com/}) en \emph{plugin}.

%=================================================================
\section{Résumé du chapitre}

% ajout - vf (martial: impression de bacler (11/2009): il y a bcp de
% copier-coller dans la VO)
Nous avons vu que:

\begin{itemize}
  \item La façon la plus simple de commencer avec Seaside est de
    télécharger le programme ``Seaside One-Click Experience'' sur
    \url{http://seaside.st};
  \item Lancer ou arrêter le serveur se fait en évaluant \ct{WAKom startOn: 8080} 
ou \ct{WAKom stop} respectivement;
  \item Changer le \emph{login} et mot de passe de l'administrateur
    peut se faire en évaluant \ct{WADispatcherEditor initialize};
  \item \menu{Toggle Halos} permet de visualiser directement le code
    source de l'application, les objets à l'exécution, les feuilles de
    style CSS et le code XHTML;
  \item Envoyer \ct{WAGlobalConfiguration setDeploymentMode} masque la
    \toolbar.
  \item Les applications web Seaside sont composées de composants,
    chacun étant une sous-classe de \ct{WAComponent};
  \item Seul un composant racine (\emph{Root Component}) peut être
    enregistré comme application. 
Il devrait implémenter la méthode de classe \ct{canBeRoot}. Il est
possible d'enregistrer le composant comme application dans la
méthode de classe \ct{initialize} en envoyant
\ct{self registerAsApplication:} \emph{chemin de l'application}.
\aretirer{%
Si vous surchargez \ct{description}, il est possible de retourner une
nom descriptif pour l'application qui sera affiché dans l'éditeur de
configuration}; % ATTENDRE - martial: pas vu du tout, ça!
  \item Pour gérer le chaînage arrière ou \backtracking, un composant
    devait disposer d'une méthode \ct{states} renvoyant un tableau des
    objets dont l'état sera restauré quand l'utilisateur clique sur le
    bouton \backbtn de son navigateur web;
  \item Le rendu d'un composant se fait via la méthode
    \ct{renderContentOn:}. L'argument de cette méthode est un
    \emph{canevas} destiné au rendu XHTML (généralement appelé
    \ct{html});
  \item Un composant peut faire le rendu d'un sous-composant en
    envoyant \ct{self render:} \emph{sous-component};
  \item Le code XHTML est généré de manière programmatique en envoyant
    des messages à des \brushes. Un \brush est obtenu en envoyant un
    message, par exemple \ct{paragraph} ou \ct{div}, au canevas HTML;
  \item Si vous envoyez des messages en cascade à un \brush qui
    contient le message \ct{with:}, ce \ct{with:} devra être le
    dernier message envoyé.
Le message \ct{with:} envoie le contenu \emph{et} le fait le rendu du résultat;
  \item Les actions devrait apparaître uniquement dans des \callbacks
    (ou fonctions de rappel).
Vous ne devez jamais changer l'état de l'application durant la phase de rendu;
  \item Vous pouvez attacher plusieurs éléments graphiques de
    formulaire (ou \emph{widgets}) et autre ancres (ou liens) à des
    variables d'instance munies de méthodes d'accès en envoyant le
    message \ct{on:} \emph{variable d'instance} \ct{of:} \emph{objet}
    au \brush;
  \item Vous pouvez définir la feuille de style CSS pour une
    hiérarchie de composants en définissant la méthode \ct{style} de
    sorte à ce qu'elle retourne une chaîne de caractères contenant la
    feuille de style (pour les applications officiellement déployées,
    il est commun de se référer à une feuille de style externe se
    trouvant à une URL statique);
  \item Les flux de contrôle
%Control flows can be programmed
peuvent être programmés en envoyant \ct{x call: y} où le composant
\ct{x} sera replacé par \ct{y} jusqu'à ce qu'\ct{y} répondent en
envoyant le message \ct{answer:} \arelire{avec un résultat dans un
  \callback}.
%by sending \ct{answer:} with a result in a callback.
Le receveur de \ct{call:} est habituellement \ct{self} mais peut être
de façon générale n'importe quel composant visible;
  \item Il existe un flux de contrôle
%A control flow 
nommé \emph{task}\,---\,instance d'une sous-classe de \ct{WATask}. Il
devrait implémenter la méthode \ct{go} pour appeler avec le message
\ct{call:} une série de composants dans une séquence de tâches;
  \item Vous pouvez vous \arelire{simplifier le travail de création
    d'interactions simples en utilisant les méthodes utilitaires de
    \ct{WAComponent} telles que \ct{request}, \ct{inform:},
    \ct{confirm:} et \ct{chooseFrom:caption:};}
  \item Pour interdire à l'utilisateur de se servir du bouton \backbtn
    de son navigateur web pour accéder un état d'exécution passé de
    l'application web, vous pouvez isoler des parties de la séquence
    de tâches dans des \transactions en les incluant dans un
    bloc-argument du message \ct{isolate:}.
\end{itemize}
%-----------------------------------------------------------------

%=================================================================
\ifx\wholebook\relax\else 
   \bibliographystyle{jurabib}
   \nobibliography{scg}
   \end{document}
\fi
%=================================================================
