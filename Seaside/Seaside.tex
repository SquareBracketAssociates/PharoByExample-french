% $Author: oscar $
% $Date: 2009-08-16 20:11:18 +0200 (Sun, 16 Aug 2009) $
% $Revision: 28482 $
% $Id: Seaside.tex 28482 2009-08-16 18:11:18Z oscar $

% HISTORY:
% 2007-10-29 - Oscar started chapter
% 2007-11-30 - Oscar first draft
% 2007-12-07 - Orla Greevy reviewed
% 2007-12-09 - Lukas Renggli reviewed
% 2008-01-11 - Andrew revised
% 2009-04-17 - Fabrizio Perin reviewed
% 2009-04-18 - Jorge Ressia reviewed
% 2009-05-06 - Oscar converted to Pharo; fixed review comments
% 2009-11-18 - Martial started the french translation
% 2010-01-08 - Rene started the first rereading
% 
% sync avec la version: 29667
%
%=================================================================
\ifx\wholebook\relax\else
% --------------------------------------------
% Lulu:
	\documentclass[a4paper,10pt,twoside]{book}
	\usepackage[
		papersize={6.13in,9.21in},
		hmargin={.75in,.75in},
		vmargin={.75in,1in},
		ignoreheadfoot
	]{geometry}
	\input{../common.tex}
	\pagestyle{headings}
	\setboolean{lulu}{true}
% --------------------------------------------
% A4:
%	\documentclass[a4paper,11pt,twoside]{book}
%	\input{../common.tex}
%	\usepackage{a4wide}
% --------------------------------------------
    \localizedgpath{{figures/lang/fr/}}%
    {{figures/lang/en/}}%
    {{figures/} {../figures/}}
	\begin{document}
	% \renewcommand{\nnbb}[2]{} % Disable editorial comments
	\sloppy
\fi
%=================================================================
\chapter{Seaside par l'exemple}
\chalabel{seaside}

%=================================================================

\ind{Seaside} est un \framework destiné à la construction
d'applications web en Smalltalk. 
Il a été initialement développé par Avi Bryant \index{Bryant, Avi} 
en 2002; 
une fois maîtrisé, Seaside permet la création d'applications
web aussi facilement que se fait l'écriture d'applications classiques.
%once mastered, Seaside makes web applications almost as easy to write as desktop applications.

Deux des applications basées sur Seaside les plus connues sont 
 \ind{SqueakSource}\footnote{\url{http://SqueakSource.com}} et \ind{Dabble DB}\footnote{\url{http://DabbleDB.com}}.
Seaside n'est pas commun dans le sens qu'il est résolument orienté
objet: il n'y a aucune structure de type \emph{template} XHTML, aucun
flux de contrôle complexe par des pages web, ni même encodage de
l'état dans les URLs. Vous envoyez simplement des messages aux
objets. Quelle merveilleuse idée!

\section{Pourquoi avons-nous besoin de Seaside?}

Les applications web modernes essayent d'intéragir avec l'utilisateur
de la même façon que les applications de bureau: elles interrogent
l'utilisateur et celui-ci leur répond, habituellement en remplissant
un formulaire ou en cliquant sur un bouton.
Mais le web fonctionne d'une autre manière: le navigateur de
l'utilisateur fait des requêtes au serveur et le serveur répond avec
une nouvelle page web.
Dès lors les \frameworks destinées au
\ind{développement d'applications web} doivent contourner une foule de
problèmes, le principal étant la gestion du flux de contrôle ``inversé''.
% control flow
C'est pour cette raison que beaucoup d'applications web tentent
d'interdire l'usage du bouton \backbtn (ou ``back'' en anglais, nommé
\emph{précédent(e)} sur certaines navigateurs) du fait de la
difficulté de garder une trace de l'état d'une session.
Diffuser des flux de contrôle significatifs entre plusieurs pages web 
est souvent lourd et la diffusion de plusieurs flux peut être difficile.
%Expressing non-trivial control flows across multiple web pages is often cumbersome, 
%and multiple control flows can be difficult or impossible to express.

% Seaside is a component-based framework that uses ``\ind{continuations}''\footnote{A \emph{continuation} represents ``the rest of the computation'' at any point in a computation. In Smalltalk, a continuation is just an object that captures the current state of the computation, and which can be resumed at any point.} to keep track of multiple points in the control flow of web applications. Continuations are managed automatically by Seaside, so web developers do not even have to be aware of the underlying machinery. It just works.

\index{Seaside!\backtracking}
%\index{Seaside!backtracking state}
\index{Seaside!transactions}
\index{Seaside!composant}
%\index{Seaside!components}
Seaside est un \framework orienté composant
% component-based 
qui allège le développement web de plusieurs façons.
Tout d'abord, le flux de contrôle
% control flow
peut être exprimé naturellement en utilisant les envois de messages.
\arevoir{Seaside garde une trace de la correspondance entre page web et 
un point précis dans l'exécution de l'application web.}
% Seaside keeps track of which web page corresponds to which point in
% the execution of the web application.
Donc le bouton \backbtn du navigateur web fonctionnement correctement.

Deuxièmement, la gestion de l'état est automatique. % 
%Second, state is managed for you.
En tant que développeur, vous aurez le choix de permettre le
\backtracking (\ie le chaînage arrière) de l'état ou non, ainsi la
navigation en \backbtn{} annulera les effets de bord.
% backtracking of state, so that navigation ``back'' in time will undo side-effects.
Vous pouvez autrement utiliser le \arelire{support de transaction}
inclus dans Seaside pour empêcher les utilisateurs d'annuler les
effets de bord permanents lorsqu'ils utilisent le bouton \backbtn.
Vous n'avez pas à encoder l'information de l'état dans
l'URL\,---\,Seaside s'en occupe pour vous. 

Ensuite, les pages web sont générées depuis les composants imbriqués,
chacun pouvant avoir son propre flux de contrôle indépendant.
% control flow
Il n'y a pas de \emph{templates} XHTML\,---\,le code XHTML valide est
généré de façon programmatique via un simple protocole Smalltalk.
% generated programmatically using a simple Smalltalk protocol.
Seaside supporte les feuilles de style en cascade 
 (\ind{CSS}), ainsi le contenu et la mise en page sont clairement
 séparés.
\seeindex{feuille de style en cascade}{CSS} % REVOIR
%\seeindex{Cascading Style Sheets}{CSS}

Finalement, Seaside offre une interface web pratique pour le
développement en facilitant la programmation itérative, le débogage
interactif et la recompilation et l'extension des applications alors
que le serveur reste en activité.

%=================================================================
\section{Démarrer avec Seaside}

Télécharger le logiciel ``Seaside \subind{Seaside}{One-Click
  Experience}'' depuis le \subind{Seaside}{site
  web}\footnote{\url{http://seaside.st}} est la façon la plus simple
pour commencer avec Seaside.
Il s'agit d'une \arevoir{version ``clé en main''}
% proposition de René : version packagée
% proposition de Martial: version ``clé en main'': idée de paquet ("bundle") et de "prêt à l'emploi"
% prepackaged version 
de Seaside 2.8 pour \ind{Mac OS X}, \ind{Linux} et \ind{Windows}.
Le site web de Seaside liste aussi de nombreux liens vers des
ressources complémentaires dont de la documentation et des
tutoriels (en anglais).
Sachez, cependant, que Seaside a considérablement évolué durant ces
dernières années et toutes ces ressources disponibles ne se rapportent
pas nécessairement à la dernière version de Seaside.

% If you are feeling more adventurous, an alternative to the ``one-click'' image is to start with the latest \ind{\pharo web image}\footnote{\url{http://pharo-project.org/download}}, and install Seaside yourself by following the manual \subind{Seaside}{installation} instructions on the Seaside web site.
\mb{It should be nice to explain how to load it from
  Monticello. The reader should be able to do this at this step.}

Seaside inclut un serveur web: vous pouvez lancer ce serveur sur le
port 8080 en évaluant \clsind{WAKom} \ct{startOn: 8080} et le stopper
en évaluant \ct{WAKom stop}.
Dans l'installation par défaut, le \subind{Seaside}{compte
  administrateur} (en anglais, \emph{administrator login}) est
\lct{admin} et le mot de passe (ou \emph{password}) est \lct{seaside}.
Pour les changer, évaluez: 
 \clsind{WADispatcherEditor} \ct{initialize}.
Ceci affichera une fenêtre de dialogue pour la saisie d'un nouveau nom
et d'un nouveau mot de passe. 

\begin{figure}[tbh]
\begin{center}
\includegraphics[width=\textwidth]{seasideStartup}
\caption{La page de démarrage de Seaside.}
\figlabel{seasideStartup}
\end{center}
\end{figure}

\dothis{Démarrez le serveur Seaside et rendez-vous sur la page
  \url{http://localhost:8080/seaside/} avec votre navigateur favori.}

\noindent
Vous devriez voir une page web semblable à celle de \figref{seasideStartup}.

\noindent
\dothis{Naviguez vers la page \lct{examples{\go}counter} (voir \figref{counter}).}


\begin{figure}[htb]
\begin{center}
\includegraphics[width=0.8\textwidth]{counter}
\caption{La démo du compteur avec l'application ``counter''.}
\figlabel{counter}
\end{center}
\end{figure}

\noindent
Cette page est une petite application Seaside nommée
\subind{Seaside}{counter}: elle affiche un
compteur qui peut être incrémenté et décrementé en cliquant sur les
liens \link{++} et \link{--\,--}.

\noindent
\dothis{Amusez-vous un peu avec l'application ``counter'' en cliquant sur
  ces liens.
Utiliser le bouton \backbtn du navigateur pour revenir à l'état
précédent et cliquer ensuite sur \link{++}.
Remarquez comment le compteur est correctement incrémenté depuis
l'état actuellement affiché au lieu d'être incrémenté depuis l'état
dans lequel le compteur était lorsque vous aviez commencé à utiliser
le bouton \backbtn.}

Notez la 
% \subind{Seaside}{toolbar}
\subind{Seaside}{\toolbar} en bas de la page web dans 
 \figref{seasideStartup}.
Seaside supporte la notion de ``sessions'' pour suivre l'état de
l'application pour différents utilisateurs.
\button{New Session} démarre une nouvelle session sur l'application
``counter''.
\button{Configure} vous permet de 
configurer les paramètres de votre application via une interface
web (pour fermer la vue \button{Configure}, cliquez sur le \link{x}
dans le côté supérieur droit de la page).
\button{Toggle Halos} propose d'explorer l'état de l'application
lancée sur le serveur Seaside.
\button{Profiler} et \button{Memory} offre une information détaillée 
à propos des performances à l'exécution de l'application.
\button{XHTML} peut être utilisé pour valider la page web générée;
ce lien ne fonctionne que lorsque la page est accessible depuis
Internet car il utilise le service de validation W3C.
\index{Seaside!halos}


Les applications Seaside sont construites à l'aide de
\arevoir{``composants'' connectables}.
% pluggable ``components''.
Ces composants sont en réalité des objets Smalltalk ordinaires.
Leur seule particularité est qu'elles doivent être des instances des
classes héritées de la classe du \framework Seaside appelée
\ct{WAComponent}.
Nous pouvons explorer les composants et leurs classes dans l'image
Pharo ou directement, depuis l'interface web, en utilisant les halos.


\begin{figure}[ht]
\begin{center}
\includegraphics[width=\textwidth]{counterHalos}
\caption{Les halos dans Seaside.}
\figlabel{counterHalos}
\end{center}
\end{figure}

\dothis{Sélectionnez \button{Toggle Halos}. Vous devriez voir une 
page web comme celle de \figref{counterHalos}.
Dans le coin supérieur gauche, le texte \ct{WACounter} nous indique le
nom de la classe du composant Seaside qui implémente le comportement
de cette page. \arelire{À côté de ce texte,} il y a trois icônes
cliquables.
Le premier, représentant un crayon, active un navigateur de classes
Seaside sur la classe \ct{WACounter}. Le second icône avec une loupe
ouvre un inspecteur d'objets sur l'instance \ct{WACounter} en cours
d'activité.
Le dernier, sous la forme de cercles colorés, affiche la feuille de
style \ind{CSS} de ce composant.
Dans le coin supérieur droit, le 
 \link{R} et le  \link{S} vous laissent basculer entre la vue du
 code rendue et celle du code source.
Essayez donc en cliquant sur ces différents liens.
Remarquez que les liens \link{++} et \link{--} sont aussi actifs dans
la vue du code source.
\mb{More and more browsers, Safari/Firefox, provide pretty-print code view!}
\arevoir{Observez d'ailleurs le formatage en couleurs du code source fourni
par les halos Seaside \aretirer{avec} la vue du code source non-formatée fournie
par votre navigateur.}}

Le navigateur de classes Seaside et l'inspecteur d'objets peuvent être
très pratiques lorsque le serveur tourne sur un autre ordinateur,
surtout si celui-ci n'a pas d'écran ou qu'il se trouve dans un lieu
distant.
Néanmoins, lorsque vous commencez à développer une application
Seaside, le serveur sera lancé localement et il sera plus facile
d'utiliser les outils de développement standards disponibles dans
l'image \pharo dans laquelle votre serveur est en activité.
 
\begin{figure}[ht]
\begin{center}
\includegraphics[width=0.7\textwidth]{haltingCounter}
\caption{Suspendre l'application ``counter''.}
\figlabel{haltingCounter}
\end{center}
\end{figure}

\dothis{Cliquez sur le lien 
 Object Inspector représenté par l'icône de loupe du halo
pour ouvrir un inspecteur d'objets sur \arelire{l'objet-compteur
  Smalltalk} et évaluez \ct{self halt} (en cliquant sur 
\button{doit}). % REVOIR
La page web s'arrêtera de charger. Allez maintenant dans votre image
Seaside. Vous devriez voir une fenêtre \arevoir{de pré-débogueur} % ATTENDRE changement dans le debugger de Pharo?
affichant un objet \ct{WACounter} exécutant un \ct{halt} (voir
\figref{haltingCounter}).
Examinez cette exécution dans le débogueur
% ajout vf - rappel
en cliquant sur la première ligne de la pile,
puis cliquez sur \button{Proceed}.
Retourner dans votre navigateur web et notez que l'application ``counter''
fonctionne à nouveau.}

Les composants Seaside peuvent être instanciés plusieurs fois et dans
différents contextes.

\begin{figure}[ht]
\begin{center}
\includegraphics[width=\textwidth]{multiCounterHalos}
\caption{Les sous-composants indépendants.}
\figlabel{multiCounterHalos}
\end{center}
\end{figure}

\dothis{Allez sur la page
  \url{http://localhost:8080/seaside/examples/multicounter avec votre
    navigateur web}.
Vous verrez une application construite grâce à un certain nombre
d'instances indépendantes du composant ``counter''.
Incrémentez et décrémentez plusieurs de ces compteurs.
Vérifiez que ces compteurs se comportent correctement même lorsque
vous utilisez le bouton \backbtn.
Activez les halos pour voir que la structure de l'application est
faite de composants imbriqués.
Utilisez le navigateur de classes Seaside pour visualiser
l'implémentation de \ct{WAMultiCounter}.
Vous devriez voir trois méthodes du côté classe (\ct{canBeRoot},
\ct{description} et \ct{initialize}) ainsi que trois méthodes du côté
instance (\ct{children}, \ct{initialize} et \ct{renderContentOn:}).
\arelire{Remarquez qu'une application est simplement un composant
qui peut être à la racine de la hiérarchie de composants; cette
possibilité de devenir racine, et donc application, est indiquée en
définissant une méthode de classe \ct{canBeRoot} (en anglais, ``peut
être racine'') pour qu'elle réponde
\ct{true}.}}
\index{Seaside!multicounter}

Vous pouvez utiliser l'interface web Seaside, copier ou enlever une
ou plusieurs applications (\ie les composants racines). Essayez de
changer la configuration comme suit.

\dothis{Allez sur \url{http://localhost:8080/seaside/config} avec
  votre navigateur.
Saisissez le \emph{login} et le mot de passe 
 (\ct{admin} et \ct{seaside} par défaut).
Sélectionnez 
 \link{Configure} à côté de ``examples''.
Sous l'en-tête  ``Add entry point'' (\ie{} \emph{ajouter un point
  d'entrée}),
saisissez le nouveau nom ``counter2'' pour le type \emph{Application} 
et cliquez sun le bouton \button{Add} comme le montre \figref{counter2}.
Sur  l'écran suivant, choisissez \clsind{WACounter} comme composant racine 
 \emph{Root Component} puis, cliquez sur \button{Save} et enfin,
 \button{Close}.
Nous avons maintenant un nouveau compteur installé à l'URL
 \url{http://localhost:8080/seaside/examples/counter2}.
Utilisez la même interface de configuration pour enlever ce point
d'entrée.}
\index{Seaside!configuration}


\begin{figure}[ht]
\begin{center}
\includegraphics[width=\textwidth]{counter2}
\caption{Configurer une nouvelle application.}
\figlabel{counter2}
\end{center}
\end{figure}

Seaside fonctionne dans deux modes: soit le mode de développement dit 
\emph{development} qui est le mode dans lequel nous venons de travailler,
soit dans le mode de diffusion appelé \emph{deployment} dans lequel la
\toolbar n'est pas accessible. 
\index{Seaside!mode!deployment}
\index{Seaside!mode!development}
Vous pouvez mettre Seaside dans le mode \emph{deployment} en utilisant
soit la page de configuration\,---\,pour cela naviguer jusqu'à
l'entrée de l'application et cliquez sur le lien \link{Configure}\,---\,soit
% \ab{How?  I couldn't find this}
cliquez directement sur le bouton  \button{Configure} dans la
\toolbar.
Dans chacun de ces cas, mettez le mode \emph{deployment} à
\emph{true}.
Notez que ceci n'affecte que les nouvelles sessions.
Vous pouvez aussi activer ce mode en évaluant 
\clsind{WAGlobalConfiguration} \lct{setDeploymentMode}, et
inversement, réactiver le mode \emph{development} en évaluant
\ct{WAGlobalConfiguration setDevelopmentMode}.
\index{Seaside!deployment mode}
\index{Seaside!development mode}

En fait, la page web de configuration n'est qu'une autre application
Seaside. Ainsi elle peut aussi être pilotée depuis la page de
configuration. Si vous enlevez l'application
``config'', vous pouvez toujours la réinstaller en évaluant
\clsind{WADispatcherEditor} \ct{initialize}.

%=================================================================
\section{Les composants Seaside}
\seclabel{components}
\seeindex{Seaside!component}{Seaside, composant}

%\ab{This section was too long\,---\,18 pages.  It also contained several self-references (``see section 1.3''). So I broke into smaller sections, by promoting some of the subsections and subsubsections.}

Comme nous l'avons mentionné précédemment, les \arelire{applications
%de
Seaside} sont bâties sur une agrégation de
  \subind{Seaside}{composants}\,---\,en anglais \emph{component}.
%As we mentioned in the previous section, Seaside applications are built out of \emph{\subind{Seaside}{components}.}
Regardons comment Seaside fonctionne en programmant le composant
``hello world''.
%\emph{Hello World}.

Tout composant Seaside doit hériter directement ou indirectement de
\clsind{WAComponent} comme le montre \figref{WACounter}.

\dothis{Définissez une sous-classe de \ct{WAComponent} appelée
  \ct{HelloWorld}.}

Les composants doivent savoir comment faire \arelire{leur propre rendu
  visuel}.
Ceci est fait habituellement en implémentant la méthode
\mthind{WAPresenter}{renderContentOn:} qui prend comme argument une
instance de  \clsind{WAHtmlCanvas}; ce dernier sait comment faire le
rendu en XHTML.
\index{Seaside!rendu}
%\index{Seaside!rendering}

\dothis{Implémentez la méthode suivante et mettez-la dans le protocole
  \prot{rendering}:}
\needlines{2}
\begin{code}{}
WAHelloWorld>>>renderContentOn: html
	html text: 'hello world'
\end{code}

\noindent
Nous devons dire maintenant à Seaside que ce composant est destiné à
être autonome
% ajout vf - plus clair
\arelire{(donc une possible racine de l'arborescence de composants).} % CHANGE

\dothis{Écrivez la méthode de classe suivante pour \ct{WAHelloWorld}.}

\begin{code}{}
WAHelloWorld class>>>canBeRoot
	^ true
\end{code}

\noindent
Nous avons fini!

\dothis{Pointez votre navigateur web
  \url{http://localhost:8080/seaside/config}, ajoutez une nouvelle
  entrée qui vous appelerez ``hello'' et choisissez \ct{WAHelloWorld}
  comme composant racine.}
Pointez maintenant votre navigateur 
à l'URL \url{http://localhost:8080/seaside/hello}.
Vous y êtes! Vous devriez voir une page web comme sur
 \figref{WAHelloWorld}.

\begin{figure}[htb]
\begin{center}
\includegraphics[width=\textwidth]{WAHelloWorld}
\caption{``hello world'' en Seaside.}
\figlabel{WAHelloWorld}
\end{center}
\end{figure}

%-----------------------------------------------------------------
\subsection{Le \backtracking d'état et l'application ``counter''}
\seeindex{Seaside!chaînage arrière}{Seaside!\backtracking} % REVOIR
%{Simple and nested components}

L'application ``counter'' est seulement un peu plus compliquée que
l'application ``hello world''.
\seclabel{backtracking}

\begin{figure}[ht]
\begin{center}
\includegraphics[width=\textwidth]{WACounter}
\caption{La classe \ct{WACounter} implémentant l'application
  ``counter''. Les méthodes avec les noms soulignés sont des méthodes
  de classe; les autres sont des méthodes d'instance.}
\figlabel{WACounter}
\end{center}
\end{figure}

La classe  \clsind{WACounter} est une \arelire{application autonome},
ainsi \ct{WACounter class} doit répondre \ct{true} au message 
 \mthind{WAComponent class}{canBeRoot}.
Elle doit aussi être enregistrée en tant qu'application; sa méthode de
classe \ct{initialize} est utilisée pour cela (voir
\figref{WACounter}).

\ct{WACounter} définit deux méthodes, \ct{increase} et \ct{decrease},
qui seront déclenchés par les liens \link{++} et \link{--\,--} sur la
page web.
Cette classe définit aussi une variable d'instance 
 \ct{count} pour enregistrer l'état du compteur.
Cependant, nous voulons aussi que Seaside synchronise le compteur avec la page web:
lorsque l'utilisateur clique sur le bouton \backbtn{} de son
navigateur web, nous voulons que Seaside rappelle l'état de l'objet
\ct{WACounter}.
%when the user clicks on the browser's ``back'' button, we want seaside to ``backtrack'' the state of the \ct{WACounter} object.
Seaside inclut un mécanisme général pour le chaînage arrière ou
\backtracking mais chaque application doit signaler à Seaside quelle
partie de son état doit être suivie.

Un composant permet le \backtracking en implémentant la méthode
d'instance \ct{states}:
% \ab{note that xspace messes up again, by inserting a space at the start of this line}
\ct{states} devrait répondre un tableau contenant tous les objets à
suivre. Dans notre cas, l'objet \ct{WACounter} s'ajoute lui-même à la
table Seaside des objets à suivre pour le \backtracking en retournant
\ct{Array with: self}.

\paragraph{\emph{Avertissement.}}
Il y a un point subtil mais important à surveiller lorsque vous
déclarez des objets pour le \backtracking.
Seaside suit l'état en faisant une \arelire{\emph{copie}}
% \emph{copy} % REVOIR - martial - nom de méthode??
de tous les objets déclarés dans le tableau \ct{states}.
Ceci se fait via l'objet
 \clsind{WASnapshot}; \ct{WASnapshot} est une sous-classe
de \clsind{IdentityDictionary} qui enregistre les objets à être suivis
en tant que clés et des \arelire{copies superficielles 
% ajout vf
(\ie{} en \ct{shallowCopy})} en tant que valeurs.
\arelire{S'il y a \backtracking{} de l'état d'une application vers une
  capture particulière de l'état, l'état de chaque objet entré dans ce
dictionnaire est écrasé par la copie sauvegardée dans la capture.}
%If the state of an application is backtracked to a particular snapshot, the state of each object entered into the snapshot dictionary is overwritten by the copy saved in the snapshot.

\tradalert{martial}{revoir ce paragraphe.}

Voilà le point critique:
Dans le cas de \ct{WACounter}, vous pourriez vous dire que l'état à
être suivi est une valeur numérique\,---\,la valeur de la variable
d'instance \ct{count}.
Cependant, la méthode \ct{states} répondant \ct{Array with: count} ne
marche pas.
%However, having the \ct{states} method answer \ct{Array with: count}
%won't work. 
Ceci est du au fait que l'objet \ct{count} est un entier et un entier
est immuable.
%This is because the object named by \ct{count} is an integer, and integers are immutable.
Les méthodes \ct{increase} et \ct{decrease} ne changent pas l'état de
l'objet \ct{0} en \ct{1} ou l'objet \ct{3} en \ct{2}.
Au lieu de ça, elles associent \ct{count} à un autre entier:
%Instead, they make \ct{count} name a different integer: 
chaque fois que le compteur est incrémenté ou décrémenté, l'objet
nommé \ct{count} est \emph{remplacé} par un autre.
C'est pourquoi \ct{WACounter>>>states} doit retourner \ct{Array with: self}.
Lorsque l'état d'un objet \mbox{\ct{WACounter}} est remplacé par un
état précédent, la \emph{valeur} de chacune des variables d'instance
dans l'objet est remplacée par une valeur précédente; ceci remplace
correctement la valeur actuelle de \ct{count} par une valeur
précédente.
\index{Seaside!\backtracking} % CHANGE
%\index{Seaside!backtracking state}
\index{WAPresenter!states@\ct{states}}

%=================================================================
\section{Le rendu XHTML}

Le but d'une application web est la création, ou ``rendu'', de pages web.
Comme nous l'avons dit dans \secref{components}, chaque composant
Seaside est responsable de son propre rendu.
% REVOIR - ajout vf (cette phrase n'existe pas dans PBE
En anglais, nous utilisons le terme \emph{rendering} que nous
retrouvons en tant que nom de protocole pour les méthodes de rendu.
Commençons notre exploration du rendu par l'exemple du composant ``counter''.
% by seeing how the counter component renders itself.

\subsection{Le rendu de l'application ``counter''}

Le rendu de l'application ``counter'' est relativement simple; le code
est visible sur \figref{WACounter}.
La valeur actuelle du compteur est affichée comme une en-tête (ou
\emph{heading}) XHTML et les opérations d'incrémentation et de
décrémentation sont implémentées sous la forme de lien HTML (ou
\emph{anchor}) avec des \callbacks (\ie{} des fonctions
de rappel) sous la forme de \emph{blocs} qui enverront soit
\ct{increase} soit \ct{decrease} à l'objet ``counter''.
%The current value of the counter is displayed as an XHTML heading, and the increment and decrement operations are implemented as html anchors (that is, links) with callbacks to blocks that will send \ct{increase} and \ct{decrease} to the counter object.
Avant de voir plus en détails le protocole \prot{rendering}, regardons
un peu le cas de l'application ``\subind{Seaside}{multicounter}''

\subsection{De la version ``counter'' à la version ``multicounter''}

\Figref{WAMultiCounter} montre \ct{WAMultiCounter} qui est
aussi une application
% ajout - vf
de compteurs multiples. Elle est autonome; elle surcharge donc la méthode de
classe \mthind{WAComponent class}{canBeRoot} pour qu'elle réponde
\ct{true}.
De plus, c'est un composant \emph{composite}. Seaside a besoin d'une
déclaration de ses composants enfants faite via la méthode
\ct{children} qui répond un tableau de tous les composants que
l'application contient.
Le rendu de ce composant se fait en faisant le rendu de ces
sous-composants séparés par une barre horizontale.
En dehors des méthodes d'instance et de classe destinées à
l'initialisation, il n'y a rien d'autre pour faire l'application
``multicounter''!

\begin{figure}[bht]
\begin{center}
\includegraphics[width=\textwidth]{WAMultiCounter}
\caption{L'application de compteurs multiples ``multicounter'' basée
  sur la classe WAMultiCounter.}
\figlabel{WAMultiCounter}
\end{center}
\end{figure}

%-----------------------------------------------------------------
\subsection{Quelques mots encore à propos du rendu XHTML}

Comme nous l'avons vu sur ces exemples, Seaside n'utilise pas de
patrons ou \emph{templates} pour générer les pages web.
Au lieu de ça, il génère le code XHTML de manière programmatique.
Tout composant Seaside devrait surcharger la méthode
 \mthind{WAPresenter}{renderContentOn:}. C'est ce message qui sera
 envoyé par Seaside pour chaque composant nécessitant d'être rendu.
 \ct{renderContentOn:} a un argument qui est un
 \seeindex{canevas}{canevas HTML} \emphind{canevas HTML}
% ajout -vf
 (en anglais, \emph{canvas}\seeindex{canvas}{canevas HTML}) 
sur lequel le composant devra être rendu.
Par convention, le paramètre de canevas HTML est appelé 
 \ct{html}.
Un canevas HTML est analogue au canevas graphique utilisé par Morphic
(et la plupart des autres bibliothèques graphiques) pour
\arevoir{abstraire le graphisme des détails dépendants des
  périphériques}.
% An html canvas is analogous to the graphics canvas used by Morphic (and most other drawing frameworks) to abstract away from the device-dependent details of drawing.
 
Voici quelques-unes des méthodes de rendu les plus simples:
\begin{code}{}
html text: 'hello world'.  "rendu du texte ordinaire"
html html: '&ndash;'.     "rendu d'un !élément! XHTML"
html render: 1.              "rendu d'un object"
\end{code}

Le message \ct{render: anyObject} 
% ajout - vf
(ou \lct{anyObject} défini un objet quelconque)
peut être envoyé à un canevas HTML pour faire le rendu de \ct{anyObject}; 
ce message est normalement utilisé pour faire le rendu de sous-composants.
Le message \ct{renderContentOn:} sera alors envoyé à \lct{anyObject}.
C'est ce qu'il se produit avec l'application ``multicounter''
 (voir \figref{WAMultiCounter}).

\subsection{Utiliser les \brushes}
\seclabel{brushes}
\seeindex{\brushes}{\brush}
Un canevas offre un grand nombre de \brushes (\ie{}
des pinceaux) utilisés pour le rendu d'un contenu sur le canevas.
Il y a des \brushes\,---\,au singulier, \brush\,---\,pour toutes
sortes d'éléments XHTML: des paragraphes, des tables, des listes\ldots
Pour voir l'ensemble des \brushes et des méthodes utilitaires
associées, vous devez naviguer avec votre Browser dans la classe
\clsind{WACanvas} et ses sous-classes. 
%To see the full protocol of brushes and convenience methods, you should browse the class \clsind{WACanvas} and its subclasses.
L'argument de \ct{renderContentOn:} est en fait une instance de la
sous-classe \clsind{WARenderCanvas}.

Nous avons déjà vu le \brush suivant dans les exemples ``counter'' et ``multicounter'':
\needlines{2}
\begin{code}{}
html horizontalRule.
\end{code}

\begin{figure}[ht]
\begin{center}
\includegraphics[width=\textwidth]{RenderingDemo}
\caption{La démo des rendus: ``Rendering Demo''.}
\figlabel{RenderingDemo}
\end{center}
\end{figure}

Dans \figref{RenderingDemo}, nous pouvons voir la sortie de beaucoup
de \brushes simples inclus dans Seaside\footnote{Le code source de
  \mthref{renderdemo} se trouve dans le paquetage \ct{PBE-SeasideDemo}
  dans le projet
  \url{http://www.squeaksource.com/PharoByExample}.
  \frsays{\arevoir{NdT: Les textes
  ont été traduits dans le présent ouvrage. Attendez-vous à ce que la
  version du code d'exemple que vous téléchargerez soit en anglais.}}{}}.
% REVOIR une version francaise du repository de code à prévoir!?!
Le composant racine \ct{SeasideDemo} fait simplement le rendu
de ses sous-composants qui sont instances de
\ct{SeasideHtmlDemo}, \ct{SeasideFormDemo}, \ct{SeasideEditCallDemo}
et \ct{SeasideDialogDemo}, comme le montre \tmthref{renderdemo}.

\needspace{7ex}
\localcode{renderdemo}

\noindent
Rappelez vous qu'un composant racine doit toujours déclarer ses
enfants sinon Seaside refusera d'en faire le rendu.
\begin{code}{}
SeasideDemo>>>children
	^ { htmlDemo . formDemo . editDemo . dialogDemo }
\end{code}

Remarquez qu'il y a deux différentes façons d'instancier le \brush
d'en-tête nommé \ct{heading}:
soit vous placez le texte directement en argument du message
\ct{heading:}
% ajout - vf
envoyé au canevas,
soit vous instanciez le \brush via l'envoi de \ct{heading} puis 
vous envoyez une cascade de messages au \brush{} en question pour
définir ses propriétés et en faire le rendu.
% and then to send a cascade of messages to the brush to set its
% properties and render it.
Beaucoup des \brushes disponibles peuvent être utilisés de ces deux
manières.

\important{Si vous envoyez à un \brush une \ind{cascade} de messages
  incluant \mthind{WABrush}{with:}, \ct{with:} doit être le message
  \emph{final}.}
L'association du contenu au \brush{} et le rendu de ce dernier est
fait par \ct{with:}.

Dans \mthref{renderdemo}, la première en-tête est de niveau
(ou \emph{level}) 1 puisque c'est la valeur par défaut.
Nous mettons explicitement le niveau de la seconde en-tête à 2.
Le sous-composant est rendu sous la forme de \emph{div} XHTML avec
``subcomponent'' comme nom de classe \ind{CSS} (pour en savoir plus
sur les feuilles de style CSS, rendez-vous à \secref{css}).
Notez aussi que l'argument du message à mots-clés \ct{with:} n'a pas
besoin d'être une chaîne de caractères litéral: ce peut être un autre
composant ou même\,---\,comme dans l'exemple suivant\,---\,un
bloc contenant \arevoir{des actions de rendu appliquées plus tard}.
%block containing further rendering actions.

Le composant \ct{SeasideHtmlDemo} fait la démonstration des \brushes{}
les plus basiques.
Le code devrait parler de lui-même~\footnote{NdT: les \emph{div}{s} et
  les \emph{span}{s} sont des éléments HTML\/XHTML et ``link with
  callback'' peut se traduire par ``lien avec fonction de rappel''.}.
%Most of the code should be self-explanatory.


\begin{code}{}
SeasideHtmlDemo>>>renderContentOn: html 
	self renderParagraphsOn: html.
	self renderListsAndTablesOn: html.
	self renderDivsAndSpansOn: html.
	self renderLinkWithCallbackOn: html
\end{code}

La division de longue méthode de rendu en plusieurs méthodes adjointes
est une pratique commune: c'est ce que nous avons fait ici.
%It is common practice to break up long rendering methods into many helper methods, as we have done here.

\important{Ne mettez pas tout votre code de rendu dans une seule
  méthode.
Séparez-le dans différentes méthodes adjointes dont le nom est de la
forme \ct{render*On:}. Toutes les méthodes de rendu vont dans le
protocole \prot{rendering}.
N'envoyez pas \ct{renderContentOn:} depuis votre propre code: utilisez
plutôt \ct{render:}.}

Observons le code suivant.
La première méthode, \ct{SeasideHtmlDemo>>>renderParagraphsOn:}, vous
montre comment générer des paragraphes XHTML, du texte ordinaire ou en
emphase (\ie{} en italique) et des images.
Remarquez qu'en Seaside les éléments simples sont rendus en spécifiant
le texte qu'ils contiennent directement alors que les éléments
complexes sont spécifiés dans des blocs.
C'est une convention simple pour vous aider à structurer votre code de
rendu.

\localcode{renderParagraphs}

La méthode suivante, \ct{SeasideHtmlDemo>>>renderListsAndTablesOn:},
vous montre comment générer des listes et des tables.
Une table utilise deux niveaux de blocs pour afficher chacune
de ses lignes et les cellules que ces dernières contiennent.

\localcode{renderListsAndTables}

Nous pouvons voir sur l'exemple suivant comment nous pouvons spécifier
les éléments \emph{div}s et \emph{span}s avec leurs attributs CSS
\emph{class} ou \emph{id}.
Bien sûr, les messages \ct{class:} et \ct{id:} peuvent être aussi
envoyés à d'autres \brushes, et pas seulement aux \emph{div}s et aux
\emph{span}s.
La méthode \ct{SeasideDemoWidget>>>style} définit comment ces éléments
XHTML devraient être affichés (voir \secref{css}).

\localcode{renderDivsAndSpans}

Finalement nous avons un simple exemple d'un lien, crée en liant un
simple \subind{Seaside}{\callback} (ou fonction de rappel) à une
``ancre'' (en anglais, ``\emph{anchor}'').
Cliquez sur le lien aura pour conséquence de basculer le texte entre
``true'' et ``false'' en faisant varier la variable d'instance
booléenne \ct{toggleValue}.
%by toggling the instance variable \ct{toggleValue}.

\needlines{3}
\localcode{renderLinkWithCallback}

\important{Remarquez que les actions ne devraient apparaître que dans les
\callbacks. Le code exécuté durant le rendu ne devrait pas changer
l'état de l'application!}
%Note that actions should appear only in callbacks.
%The code executed while rendering should not change the state of the
%application!

%-----------------------------------------------------------------
\subsection{Les formulaires}
\seeindex{Seaside!\emph{form}}{Seaside, formulaire}

Les formulaires ou \emph{forms} sont simplement rendus comme dans les
autres exemples que nous avons vus précédemment.
Voici le code pour le composant \ct{SeasideFormDemo} sur
\figref{RenderingDemo}.

\index{Seaside!formulaire}
%\index{Seaside!XHTML forms} % REVOIR IMPORTANT Seaside!XHTML!form

\begin{code}{} % REVOIR
SeasideFormDemo>>>renderContentOn: html
	| radioGroup |
	html heading: heading.
	html form: [
		html span: 'Heading: '.
		html textInput on: #heading of: self.
		html select
			list: self colors;
			on: #color of: self.
		radioGroup := html radioGroup.
		html text: 'Radio on:'.
		radioGroup radioButton
			selected: radioOn;
			callback: [radioOn := true].
		html text: 'off:'.
		radioGroup radioButton
			selected: radioOn not;
			callback: [radioOn := false].
		html checkbox on: #checked of: self.
		html submitButton
			text: 'done' ]
\end{code}{}

En raison de sa nature complexe, le formulaire est rendu en utilisant
un bloc.
\arevoir{%
Notez que tous les changements d'état se produisent dans les
\callbacks et non dans les parties de code liées au rendu.} 
% REVOIR - martial - déjà dit plus haut dans \important
%Note that all the state changes happen in the callbacks, not as part of the rendering.

Le message \mthind{WAAnchorTag}{on:of:} est une particularité Seaside
qui mérite une explication.
Dans l'exemple, ce message est utilisé pour attacher un zone de saisie
de texte à la variable \ct{heading}.
Les ancres (\ie{} les liens) et les boutons prennent en charge, eux
aussi, ce message.
Le premier argument est le nom d'une variable d'instance pour laquelle
des méthodes d'accès
ont été définies; le second argument est l'objet auquel appartient
cette variable d'instance.
Les messages accesseur et mutateur avec la convention de noms usuelle
(\ie{} \ct{heading} et \ct{heading:} respectivement) 
doivent être compris par l'objet.
Dans le cas présent d'une zone de saisie de texte, 
\arelire{cela nous évite le problème d'avoir à définir un \callback
  pour mettre à jour le champ de saisie et d'avoir à attacher le
  contenu par défaut de l'entrée à la valeur actuelle de la variable
  d'instance.}
%this saves us the trouble of having to define a callback that updates the field as well as having to bind the default contents of the html input field to the current value of the instance variable.
En utilisant \ct{on: #heading of: self}, la variable \ct{heading} est
mise à jour automatiquement à chaque fois l'utilisateur \arevoir{change le
texte de la zone de saisie.}
%variable is updated automatically whenever the user updates the text input field.

Le même message est utilisé deux fois encore dans cet exemple pour
mettre à jour la variable \ct{color} en fonction de la sélection de la
couleur dans le formulaire HTML et pour attacher l'état de la case à
cocher (ou \emph{checkbox}) à la variable \ct{checked}.
Vous pouvez trouver beaucoup d'autres exemples dans les tests
fonctionnels de Seaside.
Jetez un coup d'\oe{}il à la \arelire{catégorie
  \scat{Seaside-Tests-Functional}} % REVOIR - martial - ne devrait-on
                                % pas parler de paquetage/catégorie et
                                % non catégorie seule depuis Pharo
                                % (besoin d'une image One-Click
                                % Experience en Pharo plus récent)
ou allez simplement sur la page
\url{http://localhost:8080/seaside/tests/alltests} avec votre
navigateur web.
Sélectionnez \menu{WAInputTest} 
% ajout vf
\arelire{(cliquez sur le bouton \button{Restart} si vous avez
  désactivé le support JavaScript de votre navigateur web)} % CHANGE -
                                % ajout vf - automatique si JavaScript
 pour voir la plupart des types d'éléments de formulaire.

N'oubliez pas que, si vous activez le bouton \button{Toggle Halos} de
la \toolbar, vous pouvez naviguer dans le code source des exemples
directement via le navigateur de classes Seaside. 

%-----------------------------------------------------------------
\section{Les feuilles de style CSS}
\seclabel{css}

%\ab{I think that it just needs a few paragraphs telling the reader the key ideas behind CSS, and the new terminology that the CSS folks introduce, before going in to the details of how you define their "thingies".  Now I have forgotten what they call their "thingies" --- I know that there are effectively paragraph styles (divs) and character styles (spans), but I've forgotten what they call them.  So, I think that the text needs to tell the reader, for each thingie, (1) the CSS concept behind the thingie, (2) what it looks like in a CSS style sheet , (3) what it looks like in html, and (4) how to do it in Seaside.   Maybe (3) can be omitted, because it's not needed to use Seaside.}
% \on{I think we do most of that already.}

Les feuilles de style en cascade ou 
\emph{Cascading Style
  Sheets}\footnote{\url{http://www.w3.org/Style/CSS/}} abrégé en
 \ind{CSS} sont devenues une technique standard pour séparer le style
 du contenu des applications web.
Seaside utilise les CSS pour éviter l'encombrement de votre code de
rendu par la mise en page.

Vous pouvez mettre un feuille de style CSS pour vos composants web en
définissant la méthode \ct{style} qui devrait retourner les règles CSS
de ce composant sous forme d'une chaîne de caractères.
Les styles de tous les composants affichés sur un page web sont
assemblés. Ainsi chaque composant peut avoir son propre style.
Une meilleur approche serait de définir une classe abstraite pour
votre application web définissant un style commun pour toutes ses
sous-classes. % martial - ( un trait aussi )

En réalité, il est préférable de définir les feuilles de style des
applications déployées comme des fichiers externes.
Ce faisant, l'apparence du composant est complètement séparée de sa
fonctionnalité\,---\,regardez la classe \clsind{WAFileLibrary} qui
permet de servir des fichiers statiques sans le besoin d'un autre
serveur autonome
% ajout - vf
dédié à ces fichiers.

Si vous n'êtes pas déjà familier avec les CSS, veuillez lire
\arelire{la brève introduction aux feuilles de style CSS}.

En utilisant les CSS, vous définirez différentes classes aux éléments
de texte et de paragraphe de vos pages web et vous déclarerez toutes
les propriétés graphiques dans une feuille de style séparée, plutôt
que d'écrire ces propriétés directement sous forme d'attributs de ces
éléments.
Les entités-paragraphes sont appelées \emph{div}s et les
entités-textes sont appelées \emph{span}s.
Vous devriez définir alors définir des noms symboliques pour ces
classes, tels que ``highlight'' (\emph{subbrillance} en français) pour
le texte à mettre en surbrillance, et préciser dans votre feuille de
style comment le texte en surbrillance doit être affiché.
Un feuille de style comprend simplement un ensemble de règles qui
décrivent le format des éléments XHTML donnés. 
Chaque règle se scinde en deux parties: un \emph{sélecteur} qui
dit sur quels éléments XHTML la règle s'applique et une
\emph{déclaration} qui liste un certain nombre d'attributs pour ce ou
ces éléments.

\begin{figure}[tb]
\begin{code}{}
SeasideDemoWidget>>>style
	^ '
body {
	font: 10pt Arial, Helvetica, sans-serif, Times New Roman;
}
h2 {
	font-size: 12pt;
	font-weight: normal;
	font-style: italic;
}
table { border-collapse: collapse; }
td {
	border: 2px solid #CCCCCC;
	padding: 4px;
}
#author {
	border: 1px solid black;
	padding: 2px;
	margin: 2px;
}
.subcomponent {
	border: 2px solid lightblue;
	padding: 2px;
	margin: 2px;
}
.highlight { background-color: yellow; }
.boolean { background-color: lightgrey; }
.field { background-color: lightgrey; }
'
\end{code}
\caption{La feuille de style commune \lct{SeasideDemoWidget}.
\figlabel{democss}}
\end{figure}
\figref{democss} illustre l'exemple d'une simple feuille de style pour
l'application ``Rendering Demo'' vue plus tôt dans
\figref{RenderingDemo}.
La première règle signale une préférence pour les fontes à utiliser
pour le corps de la page web correspondant toujours à l'élément \ct{body}.
Les quelques règles suivantes donnent les propriétés des en-têtes de
niveau 2 (\ct{h2}) et celles des tables (\ct{table}) et de leur
cellule (\ct{td} pour \emph{table data}).

Les règles restantes ont des sélecteurs qui correspondent aux éléments
XHTML qui ont le même nom d'attributs ``class'' ou ``id''.
Les sélecteurs CSS pour les attributs de classe nommés ``class''
commencent par un point (``\ct{.}'') et ceux pour les attributs \emph{id}
commencent par un dièse (``\ct{#}'').
La principale différence entre les classes et les attributs \emph{id}
est la suivante: plusieurs éléments peuvent avoir la même classe mais
seul un élément peut avoir un attribut \emph{id} donné (\ie{} un
\emph{id}{entifant}).
Ainsi, alors qu'une classe comme \ct{highlight} peut être utilisée
plusieurs fois dans un page, un attribut \emph{id} doit identifier un
élément \emph{unique} sur la page, comme un menu particulier par
exemple, ou encore une date modifiée ou un auteur.
Remarquez qu'un élément XHTML particulier peut avoir de multiples
classes; dans ce cas, tous les attributs d'affichage seront appliqués
\arelire{séquentiellement}.
%all the applicable display attributes will be applied in sequence.

% This style sheet expects at most one element to specify the \emph{author} of the web page.

Des conditions au sélecteur peuvent être ajoutées. Ainsi le sélecteur
\ct{div.subcomponent} correspondera seulement à un élément XHTML qui
est à la fois un élément \emph{div} \emph{et} un élément de classe
``subcomponent''.

Il est aussi possible de préciser des éléments imbriqués, bien que
cela soit rarement nécessaire. Par exemple, le sélecteur
``\ct{paragraphe span}'' correspondera à un élément \emph{span} inclus dans un
paragraphe (élément \ct{p}) mais excluera ceux inclus dans un élément
\emph{div}.

Il existe un grand nombre de livres et de sites web sur le sujet des
CSS que vous pourrez consulter pour en apprendre plus.
Pour une démonstration spectaculaire du pouvoir des CSS, nous vous
recommandons de voir le site CSS Zen
Garden\footnote{\url{http://www.csszengarden.com/}} qui présente
comment un même contenu peut avoir un rendu totallement différent
uniquement en changeant de feuille de style.

%-----------------------------------------------------------------
\section{Gérer les flux de contrôle} % REVOIR

Seaside facilite considérablement l'élaboration d'applications web
ayant un flux de contrôle (en anglais, \emph{control flow}) complexe.
%non-trivial control flow.
Deux mécanismes peuvent être utilisés:

\begin{enumerate}
  \item un composant peut appeler\,---\,en anglais,
    \emph{call}\,---\,un autre composant en envoyant
\ct{caller call: callee}
% ajout vf
où \ct{caller} est le composant appelant et \ct{callee} est le
composant appelé.
Le composant appelant est temporairement remplacé par le composant
appelé jusqu'à ce que le \arelire{composant appelé lui rende la main
  en envoyant \ct{answer:} (\ie la réponse).}
%until the callee returns control by sending \ct{answer:}.
L'appelant est généralement \ct{self} mais ce peut être un autre
composant visible;
\seeindex{Seaside!tâche}{Seaside, \task}
  \item une modélisation des tâches (ou \emph{workflow}) peut être définie
    sous forme de \subind{Seaside}{\task}{\emph{s}}~\footnote{En
      français, \emph{tâche}.}.
    C'est un type particulier de composant, sous-classe de
    \clsind{WATask} (et non pas de \clsind{WAComponent}).\seclabel{task}
Au lieu de définir \ct{renderContentOn:}, ce composant ne définit
aucun contenu de lui-même mais définit une méthode \ct{go} qui envoie
une série de messages \ct{call:} pour activer un certain nombre de
composants en retour.
\end{enumerate}
\index{Seaside!flux de contrôle}
\seeindex{Seaside!control flow}{Seaside, flux de contrôle} % REVOIR

%-----------------------------------------------------------------
\subsection{\emph{Call} et \emph{answer}}

\emph{Call} (\cad{} l'appel) et \emph{answer} (\cad la réponse) sont utilisés
pour réaliser des dialogues simples.

L'application ``Rendering Demo'' présente un exemple simple de
\ct{call:} et \ct{answer:}.
Le composant \ct{SeasideEditCallDemo} affiche une zone de saisie de
texte et un lien \emph{edit}.
Le \callback pour ce lien appelle (en anglais, \emph{call}) une nouvelle instance de
\ct{SeasideEditAnswerDemo} initialisée à la valeur du texte dans la
zone de saisie.
Le \callback met aussi à jour cette zone de saisie au résultat qui est
envoyé en réponse (en anglais, \emph{answer})\,---\,dans les codes
suivants, nous avons souligné les messages \ct{call:} et
\ct{answer:} pour plus de clarté. % REVOIR - un peu changé par
                                   % rapport à la VO

\begin{code}{}
SeasideEditCallDemo>>>renderContentOn: html
	html span
		class: 'field';
		with: self text.
	html space.
	html anchor
		callback: [self text: (self !\underline{call:}! (SeasideEditAnswerDemo new text: self text))];
		with: 'edit'
\end{code}{}

Le code ne fait absolument aucune référence à la nouvelle page web qui
doit être créée, ce qui est particulièrement élégant.
À l'exécution, une nouvelle page est créée dans laquelle le composant
\ct{SeasideEditCallDemo} est remplacé par un composant
\ct{SeasideEditAnswerDemo}; le composant parent et les autres et les
autres composants homologues restent inchangés.
%the other peer components are untouched.

\important{Les messages \mthind{WAComponent}{call:} et
  \mthind{WAComponent}{answer:} ne doivent jamais être utilisés durant
  la phase de rendu.
Ils peuvent être envoyés sans problème depuis un
\subind{Seaside}{\callback} ou depuis la méthode \mthind{WATask}{go}
d'une \task.}

Le composant \ct{SeasideEditAnswerDemo} est aussi remarquable simple.
Il se contente d'afficher un formulaire avec une zone de saisie.
Le bouton \emph{submit} 
% ajout - vf
(\brush accessible par le message \ct{submitButton})
est lié au \callback qui répondra la valeur finale du texte dans la
zone de saisie.

\begin{code}{}
SeasideEditAnswerDemo>>>renderContentOn: html
	html form: [
		html textInput
			on: #text of: self.
		html submitButton
			callback: [ self !\underline{answer:}! self text ];
			text: 'ok'.
		]
\end{code}{}

C'est tout!

Seaside prend soin du flux de contrôle et du rendu de tous les
composants.
Il est intéressant de voir que le bouton \backbtn{} du navigateur web
fonctionnera bien (bien que les effets de bord ne soient pas
rembobinés à moins d'avoir pris des dispositions complémentaires).
%though side effects are not rolled back unless we take additional steps).

%-----------------------------------------------------------------
\subsection{Les méthodes utilitaires}
%\subsection{Convenience methods}

Puisque certains dialogues de type \emph{call--answer} sont très
communs, Seaside fournit des méthodes utilitaires pour vous éviter
l'écriture des composants tels que \ct{SeasideEditAnswerDemo}.
Les dialogues générés sont montrés sur \figref{dialogs}.
Nous pouvons voir que ces méthodes utilitaires sont utilisées dans la
méthode \ct{SeasideDialogDemo>>>renderContentOn:}
\index{Seaside!méthodes utilitaires}
%\index{Seaside!convenience methods} % REVOIR

\begin{figure}[b]
\begin{center}
\includegraphics[width=\textwidth]{dialogs}
\caption{Certains dialogues standards.}
\figlabel{dialogs}
\end{center}
\end{figure}

Le message \mthind{WAComponent}{request:} fait un appel à un composant
qui vous laissera éditer une zone de saisie. Le composant répond la
chaîne de caractères éditée
% ajout - vf
dans cette zone de saisie.
Un \emph{label} optionnel et une valeur par défaut peuvent aussi être
spécifiés.

\needlines{3}
\begin{code}{}
SeasideDialogDemo>>>renderContentOn: html
	html anchor
		callback: [ self request: 'edit this' label: 'done' default: 'some text' ];
		with: 'self request:'.
...
\end{code}

Le message \mthind{WAComponent}{inform:} appelle un composant qui
affiche simplement l'argument-message et attend que l'utilisateur
clique sur ``ok''.
Le composant appelé retourne simplement \ct{self}.
%The called component just returns \ct{self}.

\begin{code}{}
...
	html space.
	html anchor
		callback: [ self inform: 'yesBANG' ];
		with: 'self inform:'.
...
\end{code}

Le message \mthind{WAComponent}{confirm:} pose une question et attend
que l'utilisateur clique sur ``Yes'' ou ``No''.
Le composant répond un booléen qui peut être utilisé pour faire
d'autres actions.

\begin{code}{}
...
	html space.
	html anchor
		callback: [
			(self confirm: 'Are you happy?') "!Êtes-vous! content?"
				ifTrue: [ self inform: ':-)' ]
				ifFalse: [ self inform: ':-(' ]
			];
		with: 'self confirm:'.
\end{code}

Quelques méthodes utilitaires, tels que
\mthind{WAComponent}{chooseFrom:caption:}, sont définies dans le
protocole \prot{convenience} de la classe \clsind{WAComponent}.

%-----------------------------------------------------------------
\subsection{Les \tasks Seaside}

Une \subind{Seaside}{task} (ou \emph{tâche}) est un composant qui est
sous-classe de \clsind{WATask}.
Elle ne fait le rendu du rien elle-même mais appelle simplement
d'autres composants dans un flux de contrôle défini en implémentant la
méthode \mthind{WATask}{go}.


\clsind{WAConvenienceTest} est une simple exemple de l'utilisation
d'une \task définie dans \arelire{la catégorie
  \scat{Seaside-Tests-Functional}}. % REVOIR - en attendant une
                                % version packagée de One-Click...
Pour voir son effet, pointez votre navigateur web sur l'URL
\url{http://localhost:8080/seaside/tests/alltests} et sélectionnez
\menu{WAConvenienceTest} et cliquez sur \button{Restart}.

\begin{code}{}
WAConvenienceTest>>>go
	[ self chooseCheese.
	  self confirmCheese ] whileFalse.
	self informCheese
\end{code}

Cette \task appelle trois composants.
Le premier, généré par la méthode utilitaire
\mthind{WAComponent}{chooseFrom: caption:}, est une instance de
\clsind{WAChoiceDialog} qui demande à l'utilisateur de choisir un
fromage (\emph{cheese}).

\begin{code}{}
WAConvenienceTest>>>chooseCheese
	cheese := self
		chooseFrom: #('Greyerzer' 'Tilsiter' 'Sbrinz')
		caption: 'What''s your favorite Cheese?'. "Quel est votre fromage !préféré!"
	cheese isNil ifTrue: [ self chooseCheese ]
\end{code}

% \alex{Is there a situation where cheese may be nil? Maybe if a browser authorizes an empty selection...}

Le second est un dialogue \clsind{WAYesOrNoDialog} pour confirmer le
choix (généré par la méthode utilitaire \mthind{WAComponent}{confirm:}).

\begin{code}{}
WAConvenienceTest>>>confirmCheese
    "Ce fromage est-il votre !préféré!?"
	^self confirm: 'Is ', cheese,  ' your favorite cheese?'
\end{code}

\emph{In fine}, un dialogue \clsind{WAFormDialog} est appelé (via la
méthode utilitaire \mthind{WAComponent}{inform:}).

\begin{code}{}
WAConvenienceTest>>>informCheese
    "Votre fromage !préféré! est ..."
	self inform: 'Your favorite cheese is ', cheese, '.'
\end{code}

Vous pouvez voir ces dialogues générés sur \figref{chooseCheese}.

\begin{figure}[ht]
\begin{center}
\includegraphics[width=\textwidth]{chooseCheese}
\caption{Un simple exemple de \task en action.}
\figlabel{chooseCheese}
\end{center}
\end{figure}

%-----------------------------------------------------------------
\subsection{Les \transactions} % REVOIR TRADUIRE

Nous avons vu dans \secref{backtracking} que Seaside peut garder une
trace de la correspondance entre l'état des composants et les pages
web en ayant des composants qui enregistrent leur état pour le
\backtracking:
% Seaside can keep track of the correspondence between the state of components and individual web pages by having components register their state for backtracking:
tout ce qu'un composant a besoin de faire, c'est d'implémenter la
méthode \ct{states} pour répondre un tableau de tous les objets dont
l'état doit être suivi.

Cependant, certaines fois, nous ne voudrions pas avoir de \backtracking
de l'état: nous aimerions seulement \emph{prévenir} l'utilisateur
qu'il risque \arelire{d'effacer accidentellement des effets qui
  devraient être permanents}.
%from accidentally undoing effects that should be permanent.
Ce problème est souvent appelé ``le problème du panier'' (en anglais, ``the
shopping cart problem'').
Une fois que vous avez validé votre panier virtuel 
% ajout - vf
sur un site webmarchand
et que vous avez payé pour les articles que vous avez acheté, il ne
devrait pas être possible de revenir en arrière avec le bouton
\backbtn{} de votre navigateur web et d'ajouter de nouveaux articles à
votre panier!

Seaside vous permet de prévenir cela en définissant une \task 
dans laquelle certaines actions sont regroupées en \transactions.
Vous pouvez utiliser le \backtracking{} dans une \transaction mais,
une fois la \transaction terminée, vous ne pouvez plus revenir sur
celle-ci.
Les pages correspondantes sont \emph{invalidées} et toute tentative
de revenir sur celles-ci entraînera la création d'une alerte par
Seaside et redirigera l'utilisateur sur la page la plus récemment valide.

\begin{figure}[ht]
\begin{center}
\includegraphics[width=\textwidth]{sushiStore}
\caption{L'application ``Sushi Store''.}
\figlabel{sushiStore}
\end{center}
\end{figure}

L'application ``\emphsubind{Seaside}{Sushi Store}'' est une
application de démonstration illustrant de nombreuses particularités
de Seaside dont les \transactions.
Cette application est incluse dans votre installation de Seaside. Dès
lors vous pouvez l'essayer en vous rendant sur la page 
\url{http://localhost:8080/seaside/examples/store}\footnote{%
Si vous ne la trouvez pas dans votre image, vous pouvez charger une
version de l'application ``Sushi Store'' sur SqueakSource dans le dépôt
  \url{http://www.squeaksource.com/SeasideExamples/}
% ajout vf
sous le nom de paquetage \scat{Store}%
.} avec votre navigateur web.

L'application ``Sushi Store''
% ajout -vf
(traduisible par le \emph{restaurant de sushis en ligne})
présente la modélisation des tâches suivante:
%The sushi store supports the following workflow:
\begin{enumerate}[itemsep=0pt]
  \item Visiter la boutique.
  \item Naviguer ou chercher un sushi.
  \item Ajouter un sushi à votre panier (ou \emph{shopping cart}).
  \item Valider (\emph{checkout}).
  \item Vérifier votre ordre.
  \item Entrer une adresse d'expédition.
  \item Vérifier une adresse d'expédition.
  \item Entrer les informations de paiement.
  \item Votre poisson cru est en route!
\end{enumerate}

Si vous activez les \subind{Seaside}{halos}, vous verrez que le
composant racine de l'application ``Sushi Store'' est un instance de
\clsind{WAStore}.
Ce composant ne fait rien d'autre que le rendu d'une barre de titre et
de l'objet \ct{task}; ce dernier est une variable d'instance de
\clsind{WAStoreTask}.

\begin{code}{}
WAStore>>>renderContentOn: html
	"... rendu de la barre de titre ..."
	html div id: 'body'; with: task
\end{code}

\clsind{WAStoreTask} retient cette séquence de modélisation de tâches.
%captures this workflow sequence.
C'est important qu'à deux moments l'utilisateur ne puisse pas revenir
en arrière et changer les informations déjà envoyées.

% vf - martial: c'est pas clair dans la VO REVOIR
\dothis{\,Achetez avec la commande ``Checkout'' des sushis,
    confirmez en cliquant ``Proceed with checkout'' et utilisez
  ensuite le bouton \backbtn{} pour essayer de mettre plus de sushis
  dans votre panier (dans l'application, ``\emph{Your cart}'').
  Vous aurez le message ``That page has expired'' (\ie{} ``la
  page a expiré'').}
%You will get the message ``That page has expired.''}

Seaside laisse le programmeur écrire si une certaine partie d'une
séquence de tâches agit comme une \transaction: une fois la
\transaction complète, l'utilisateur ne pourra plus revenir en arrière
et l'annuler.
Vous l'écrivez en envoyant le message \mthind{WAComponent}{isolate:} à
la \task avec le bloc transactionnel comme argument.
Nous pouvons le voir dans la séquence des tâches de l'application
``Sushi Store'' dans le code suivant:

\begin{code}{}
WAStoreTask>>>go
	| shipping billing creditCard |
	cart := WAStoreCart new.
	self isolate:
		[[self fillCart.
		self confirmContentsOfCart]
			whileFalse].

	self isolate:
		[shipping := self getShippingAddress.
		billing := (self useAsBillingAddress: shipping)
					ifFalse: [self getBillingAddress]
					ifTrue: [shipping].
		creditCard := self getPaymentInfo.
		self shipTo: shipping billTo: billing payWith: creditCard].

	self displayConfirmation.
\end{code}

Nous voyons assez clairement que nous avons ici deux \transactions.
La première remplit le panier (en anglais, \emph{fill cart}) et clôt
la phase des achats (les méthodes adjointes, \ct{fillCart} \etc,
prennent soin d'instancier et d'appeler les bons sous-composants).
Une fois que vous avez confirmé le contenu du panier, vous ne pouvez
plus revenir sans démarrer une nouvelle session.
La seconde \transaction concerne la saisie des informations d'envoi et
de paiement.
Vous pouvez naviguer en arrière et revenir dans la seconde
\transaction{} jusqu'à votre confirmation du paiement.
Cependant, une fois les deux \transactions{} concluent, toute tentative
de navigation en arrière échouera.

Les \transactions{} peuvent aussi être imbriquées.
Une simple démonstration de ceci se trouve dans la classe
\clsind{WANestedTransaction}.
Le premier message \ct{isolate:} prend un bloc comme argument;
celui-ci a un autre envoi de message \ct{isolate:} imbriqué.

\begin{code}{}
WANestedTransaction>>>go
	self inform: 'Before parent txn'.
	self isolate:
			[self inform: 'Inside parent txn'.
			self isolate: [self inform: 'Inside child txn'].
			self inform: 'Outside child txn'].
	self inform: 'Outside parent txn'
\end{code}

\dothis{Allez sur \url{http://localhost:8080/seaside/tests/alltests},
  sélectionnez \menu{WATransactionTest} et cliquez sur
  \button{Restart}.
Essayez de naviguer en arrière et en avant dans la transaction parent
et enfant (\emph{child}) en cliquant sur le bouton \backbtn et
en cliquant ensuite sur le bouton \button{ok}.
%Try to navigate back and forth within the parent and child transaction by clicking the \button{back} button and then clicking \button{ok}.
Remarquez que, dès qu'un \transaction{} est complète, vous ne pouvez
plus revenir dans la \transaction, sans
générer une erreur en cliquant sur \button{ok}.}

%=================================================================
\section{Un tutoriel complet}

% ON: Should take about two hours

Voyons comment nous pouvons construire une application Seaside
complète de zéro\footnote{L'exercice devrait vous prendre deux bonnes
  heures. Si vous préférez simplement regarder le code source final,
  vous pouvez le récupérer depuis le projet SqueakSource à l'adresse:
 \url{http://www.squeaksource.com/PharoByExample}. \scat{PBE-SeasideRPN}
 est le paquetage à charger. Le tutoriel utilise des noms de classe
 légèrement différents ainsi vous pourrez comparer votre
 implémentation avec la notre.%
}.
Nous allons construire une calculatrice RPN~\footnote{La RPN pour
  ``Reverse Polish Notation'' ou notation polonaise inversée est une
  des trois écritures mathématiques; elle est dite aussi
  \emph{postfixe}.} % martial - perso, c'est ma notation favorite (PS,
                    % factor, forth...)
                    % rene préfère aussi les machines RPN :-)
comme une application Seaside qui utilise une machine à pile simple
comme modèle (``pile'' en anglais se disant ``\emph{stack}'').
En outre, l'interface Seaside nous laissera choisir entre deux modes
d'affichage\,---\,l'un nous montrant simplement la valeur
actuelle au sommet de la pile et l'autre nous montrant l'état complet
de la pile.
Vous pouvez voir la calculatrice munie de ses deux options d'affichage
sur \figref{stackMachine}.

\begin{figure}[ht]
\begin{center}
\includegraphics[width=\textwidth]{stackMachine}
\caption{La calculatrice RPN et sa machine à pile.}
\figlabel{stackMachine}
\end{center}
\end{figure}

Nous commençons par implémenter la machine à piles et ses tests.

\dothis{Définissez une nouvelle classe \ct{MyStackMachine} 
% ajout - vf
\arelire{dans une nouvelle catégorie de votre choix et} % rappel
avec une variable d'instance \ct{contents} initialisée comme une
nouvelle collection ordonnée, instance de \ct{OrderedCollection}.}

\begin{code}{}
MyStackMachine>>>initialize
	super initialize.
	contents := OrderedCollection new.
\end{code}

La machine à pile devrait fournir des opérateurs \ct{push:} et
\ct{pop}
% ajout - vf
pour ajouter et enlever des valeurs respectivement ainsi que des
commandes pour voir le sommet de la pile que nous appellerons \ct{top} 
et pour faire plusieurs opérations arithmétiques telles que l'addition (\emph{add}),
la soustraction (\emph{substract}), la multiplication (\emph{multiply}) 
et la division (\emph{divide}) des valeurs au sommet de la pile.

\dothis{Écrivez des tests pour les opérations de la pile et
  implémentons ensuite ces opérations. Voici un exemple de test
% ajout - vf 
(avec la classe \ct{MyStackMachineTest}, instance de \ct{TestCase},
pour le cas de la division \ct{div} en ayant pris soin d'initialiser
la variable d'instance \ct{stack} dans \ct{setUp}): % martial - pour plus de clareté
}

\needlines{4}
\begin{code}{}
MyStackMachineTest>>>testDiv
	stack
		push: 3;
		push: 4;
		div.
	self assert: stack size = 1.
	self assert: stack top = (4/3).
\end{code}

Vous devriez penser à utiliser des méthodes utilitaires pour les
opérations arithmétiques pour vérifier qu'il y a toujours deux nombres
sur la pile avant de faire quoi que ce soit et lever une erreur si
ce prérequis n'est pas atteint\footnote{C'est une bonne idée d'utiliser \ct{Object>>>assert:} 
% ajout - vf
dans \ct{MyStackMachine}
pour écrire les préconditions requises pour une opération. Cette méthode lévera un 
\ct{AssertionFailure} si l'utilisateur essaye d'utiliser la machine à pile dans un état invalide.}.
%You might consider using some helper methods for the arithmetic operations to check that there are two numbers on the stack before doing anything, and raising an error if this precondition is not fulfilled.\footnote{It's a good idea to use \ct{Object>>>assert:} to specify the preconditions for an operation.
Si vous faites ainsi, la plupart de vos méthodes tiendront sur une ou deux lignes.

Vous pourriez aussi considérer l'implémentation d'une méthode
\ct{MyStackMachine>>>printOn:} pour faciliter le débogage de votre
implémentation de la machine à pile avec l'aide de l'Inspector
(une petite astuce: déléguer simplement l'impression \ct{printOn:} à
la variable d'instance \ct{contents}).
\index{Object!printOn:@\ct{printOn:}}

\dothis{Complétez la classe \ct{MyStackMachine} en écrivant les
  opérateurs \ct{dup} qui duplique la valeur du sommet de la pile
et la met dans la pile, \ct{exch} (abrégé de \emph{exchange}) qui échange
les deux valeurs du sommet de la pile et \ct{rotUp} qui fait une
permutation circulaire du contenu entier de la pile\,---\,la valeur en
\ct{top} se retrouvera ainsi en bas de la pile.}

Nous avons maintenant une implémentation simple d'une machine à
pile. Nous pouvons commencer la programmation de notre calculatrice
RPN Seaside.

Nous allons créer 5 classes:
\begin{itemize}
  \item \ct{MyRPNWidget}\,---\,c'est une classe que nous voulons
    abstraite et qui définit la feuille de style CSS commune pour
    l'application et d'autres comportements communs pour les
    composants de la calculatrice RPN.
C'est une sous-classe de \ct{WAComponent} et super-classe directe des
quatre classes suivantes;
  \item \ct{MyCalculator}\,---\,c'est le composant racine.
      Il devrait enregistrer l'application dans Seaside (côté classe),
      instancier et faire le rendu de ses sous-composants et il
      devrait enfin déclarer un état pour le \backtracking;
  \item \ct{MyKeypad}\,---\,ce composant affiche les boutons
    (\emph{key}) que nous utiliserons pour interagir avec la
    calculatrice;
  \item \ct{MyDisplay}\,---\,ce composant affiche le sommet de la pile
    et fournit un bouton pour appeler un autre composant pour afficher
    la vue détaillée
% ajout - vf
(\ie pour gérer un \emph{call});
  \item \ct{MyDisplayStack}\,---\,ce composant affiche la vue
    détaillée de la pile et offre un bouton pour répondre en retour
% ajout - vf
(\ie pour gérer un \emph{answer}).
C'est une sous-classe de \lct{MyDisplay}.
\end{itemize}

\dothis{Définissez la classe \ct{MyRPNWidget} \arevoir{dans la catégorie
    \ct{MyCalculator}}. % REVOIR - martial - je pense qu'il faudrait
                        % faire un paquetage avec les deux catégories
                        % Core pour MyStackMachine et Display pour
                        % MyRPNWidget etc
Définissez le \ct{style} commun pour l'application.}

Voici une feuille de style CSS minimale pour l'application.
Vous pourrez à loisir la rendre plus chic si vous voulez.
\begin{code}{}
MyRPNWidget>>>style
	^ 'table.keypad { float: left; }
td.key {
	border: 1px solid grey;
	background: lightgrey;
	padding: 4px;
	text-align: center;
}
table.stack { float: left; }
td.stackcell {
	border: 2px solid white;
	border-left-color: grey;
	border-right-color: grey;
	border-bottom-color: grey;
	padding: 4px;
	text-align: right;
}
td.small { font-size: 8pt; }'
\end{code}

\dothis{Définissez la classe \ct{MyCalculator} de façon à être un
  composant racine et enregistrez le en tant qu'application Seaside
(autrement dit, implémentez \ct{canBeRoot} et la méthode de classe
\ct{initialize}).
Commencez à implémenter la méthode
\ct{MyCalculator>>>renderContentOn:} pour faire le rendu de quelque
chose de simple comme, par exemple, afficher le nom de l'application
et vérifiez que l'application tourne correctement dans un navigateur
web.}

\ct{MyCalculator} est responsable de l'instanciation des classes \ct{MyStackMachine}, \ct{MyKeypad} et \ct{MyDisplay}.

\dothis{%
Définissez \ct{MyKeypad} et \ct{MyDisplay} comme sous-classes
de \lct{MyRPNWidget}.
Tous ces trois composants auront besoin d'accéder à une instance
commune de la machine à pile; définissez donc la variable d'instance 
\ct{stackMachine} et une méthode d'initialisation
\ct{setMyStackMachine:} dans la classe mère \ct{MyRPNWidget}.
Ajoutez les variables d'instance \ct{keypad} et \ct{display} à la
classe \ct{MyCalculator} et initialisez-les dans
\ct{MyCalculator>>>initialize} (sans oublier d'envoyer 
 \lct{super initialize}).}

\dothis{%
Passez l'instance partagée de la machine à pile à l'objet \ct{keypad}
et à l'objet \ct{display} dans la même méthode
\ct{MyCalculator>>>initialize}.
Implémentez \ct{MyCalculator>>>renderContentOn:} de façon à faire le
rendu des instances \ct{keypad} et \ct{display}.
Pour afficher correctement les sous-composants, vous devez coder la
méthode \ct{MyCalculator>>>children} de manière à retourner un tableau
avec \ct{keypad} et \ct{display}.
Implémentez une méthode de rendu quelconque,
% ajout - vf
par exemple pour afficher un texte ordinaire,
pour le \ct{keypad} et le \ct{display} et vérifiez que l'application
calculatrice affiche ces deux sous-composants.}

%\ab{Too long!}

Maintenant nous allons changer l'implémentation du \ct{display} pour
afficher la valeur du sommet de la pile.

\dothis{%
Utilisez une table HTML avec la classe ``keypad'' contenant une ligne
avec une seule cellule de classe ``stackcell''
% ajout - vf - martial - plus clair comme ça:
dans \ct{MyDisplay>>>renderContentOn:}.
Changez maintenant la méthode de rendu de \ct{keypad} pour que le
nombre \ct{0} soit mis sur la pile dans le cas où celle-ci est vide
%Change the rendering method of the keypad to ensure that the number 0 is pushed on the stack in case it is empty.
(définissez et utilisez \ct{MyKeypad>>>ensureMyStackMachineNotEmpty}).
Faites en sorte que \ct{keypad} affiche un table vide de classe
``keypad''.
Le calculateur devrait afficher une seule cellule contenant la valeur
0.
Si vous activez les halos, vous devriez voir quelque chose comme suit:}

\begin{figure}[ht]
\begin{center}
\includegraphics[width=0.8\textwidth]{firstStackDisplay}
\caption{Affichage du sommet de la pile.}
\figlabel{firstStackDisplay}
\end{center}
\end{figure}

Implémentons désormais une interface pour interagir avec la pile.

\dothis{
Définissez tout d'abord les méthodes adjointes facilitant l'écriture
de l'interface:
}

\needlines{3}
\begin{code}{}
MyKeypad>>>renderStackButton: text callback: aBlock colSpan: anInteger on: html 
	html tableData
		class: 'key';
		colSpan: anInteger;
		with: 
				[html anchor
					callback: aBlock;
					with: [html html: text]]
\end{code}


\begin{code}{}
MyKeypad>>>renderStackButton: text callback: aBlock on: html 
	self 
		renderStackButton: text
		callback: aBlock
		colSpan: 1
		on: html
\end{code}

Nous utiliserons ces deux méthodes pour définir les boutons du
\ct{keypad} avec les \callbacks appropriés.
Certains boutons peuvent s'étaler sur plusieurs colonnes
% ajout - vf
(avec \ct{colSpan:}) mais, par défaut, ils occupent une seule colonne.

\dothis{%
Utiliser ces deux méthodes pour écrire le \ct{keypad} comme suit
(une astuce: commencez par faire en sorte que les chiffres et du bouton
``Enter'' fonctionnent 
% ajout - vf
\arelire{(en laissant la méthode \ct{setClearMode})} % REVOIR -
                                % martial - à reporter dans la VO
                                % parce que pas gagné! 
avant de vous attardez aux opérateurs arithmétiques):}

\needlines{4}
\begin{code}{}
MyKeypad>>>renderContentOn: html 
  self ensureStackMachineNotEmpty.
  html table
    class: 'keypad';
    with: [
      html tableRow: [
          self renderStackButton: '+' callback: [self stackOp: #add] on: html.
          self renderStackButton: '&ndash;' callback: [self stackOp: #min] on: html.
          self renderStackButton: '&times;' callback: [self stackOp: #mul] on: html.
          self renderStackButton: '&divide;' callback: [self stackOp: #div] on: html.
          self renderStackButton: '&plusmn;' callback: [self stackOp: #neg] on: html ].
        html tableRow: [
          self renderStackButton: '1' callback: [self type: '1'] on: html.
          self renderStackButton: '2' callback: [self type: '2'] on: html.
          self renderStackButton: '3' callback: [self type: '3'] on: html.
          self renderStackButton: 'Drop' callback: [self stackOp: #pop]
          	colSpan: 2 on: html ].
" et ainsi de suite ... "
        html tableRow: [
          self renderStackButton: '0' callback: [self type: '0'] colSpan: 2 on: html.
          self renderStackButton: 'C' callback: [self stackClearTop] on: html.
          self renderStackButton: 'Enter'
          	callback: [self stackOp: #dup. self setClearMode]
			colSpan: 2 on: html ]]
\end{code}

Vérifiez que le \ct{keypad} s'affiche proprement.
Si vous essayez de cliquer sur les boutons, vous verrez que la
calculatrice ne marche pas encore\ldots{}

\dothis{%
Implémentez \ct{MyKeypad>>>type:} pour mettre à jour le sommet de la
pile en annexant les chiffres saisis.
Vous aurez besoin de convertir la valeur du sommet de la pile en
chaîne de caractères, la mettre à jour
% ajout - vf
en la concatenant avec l'argument de \ct{type:} et la convertir enfin
en entier, en faisant de la sorte:}
\begin{code}{}
MyKeypad>>>type: aString
	stackMachine push: (stackMachine pop asString, aString) asNumber.
\end{code}
Lorsque vous cliquez sur les boutons-chiffre, l'affichage devrait être
mis à jour
(soyez donc sûr que la méthode \ct{MyStackMachine>>>pop} retourne la
valeur sortie de la pile, sinon ça ne marchera pas!).


\dothis{Now we must implement \ct{MyKeypad>>>stackOp:}
Something like this will do the trick:}

\begin{code}{}
MyKeypad>>>stackOp: op
	[ stackMachine perform: op ] on: AssertionFailure do: [ ].
\end{code}

Nous ne sommes pas sûrs du succès de toutes les opérations. Par
exemple, une addition peut échouer si nous n'avons pas deux nombres
sur la pile.
Pour l'instant, nous pouvons simplement ignorer de telles erreurs.
Si nous nous sentons plus ambitieux plus tard, nous pourrons ajouter
un retour informatif à l'utilisateur dans un bloc de gestion d'erreur.
%If we are feeling more ambitious later on, we can provide some user feedback in the error handler block.

\dothis{%
La première version de la calculatrice devrait fonctionner
maintenant. Essayez d'entrer des nombres en cliquant sur les
boutons-chiffre, puis cliquez sur \menu{Enter} pour dupliquer 
la valeur actuelle, et cliquez enfin sur \menu{+} pour faire 
l'addition de ces deux valeurs.}

Vous remarquerez que la saisie des chiffres ne se passent pas comme
prévu.
En fait, la calculatrice devrait savoir si vous saisissez un
\emph{nouveau} nombre ou que vous complétez une nombre existant.

\dothis{%
Adaptez la méthode \ct{MyKeypad>>>type:} pour se comporter
différemment suivant le mode actuel de saisie.
Introduisez une variable d'instance \ct{mode} pouvant prendre trois
valeurs \lct{\#typing} (lorsque vous saisissez), \lct{\#push} (après que
vous ayez effectué une opération de calcul \arevoir{et que la saisie
  devrait forcer l'entrée de la valeur sur la pile}) 
%after you have performed a calculator operation and typing should
%force the top value to be pushed)
ou \lct{\#clear} (après que vous ayez cliqué sur le bouton \menu{Enter}
et que \arevoir{le sommet de la pile devrait se mettre dans l'état
  initial avant la prochaine saisie}).
% and the top value should be cleared before typing).
La nouvelle méthode \ct{type:} pourrait ressembler à ceci:}

\tradalert{martial}{pour TODO-cb, type: devrait utiliser
  "stackMachine dup." au lieu de "stackMachine push: stackMachine
  top."; j'ai déjà changé en prévision...}
\begin{code}{}
MyKeypad>>>type: aString
	self inPushMode ifTrue: [
		stackMachine dup.
		self stackClearTop ].
	self inClearMode ifTrue: [ self stackClearTop ].
	stackMachine push: (stackMachine pop asString, aString) asNumber.
\end{code} % ATTENDRE

La saisie devrait mieux fonctionner maintenant mais le fait de ne
pouvoir voir la pile intégralement est frustrant.
%Typing might work better now, but it is still frustrating not to be able to see what is on the stack.

\dothis{%
Définissez la classe \ct{MyDisplayStack} comme une sous-classe de
\ct{MyDisplay}. Ajoutez un bouton dans la méthode de rendu de
\ct{MyDisplay} qui appelera une nouvelle instance de
\ct{MyDisplayStack}.
Vous aurez besoin d'un lien HTML (ou \emph{anchor}) ressemblant à ceci:}

\begin{code}{}
html anchor
	callback: [ self call: (MyDisplayStack new setStackMachine: stackMachine)];
	with: 'open'
\end{code}
% ATTENDRE setMyStackMachine: dans la VO, setStackMachine c'est plus
% logique

Le \callback{} entraînera le remplacement temporaire de l'actuelle
instance de \ct{MyDisplay} en une nouvelle instance de la classe
\ct{MyDisplayStack} dont le travail consiste à afficher la pile
entière.
Lorsque ce composant signale que le travail est fini (en envoyant
\ct{self answer}), l'instance originel de \ct{MyDisplay} reviendra
dans le \arelire{flux}.
%then control will return to the original instance of \ct{MyDisplay}.

\dothis{%
Définissez la méthode de rendu de la classe \ct{MyDisplayStack} pour
qu'elle affiche toutes les valeurs sur la pile
(vous aurez besoin de définir soit un accesseur \ct{contents} pour
atteindre le contenu de la machine à pile, soit une méthode
\ct{MyStackMachine>>>do:} pour itérer sur les valeurs de la pile).
L'interface de la pile devra aussi proposer un bouton ``close'' dont
le \callback{} effectuera simplement en \ct{self answer}.}
%The stack display should also have a button labelled ``close'' whose callback will simply perform \ct{self answer}.


\begin{code}{}
html anchor
	callback: [ self answer];
	with: 'close'
\end{code}

Vous devriez maintenant être capable d'ouvrir (avec \menu{open}) et de
fermer (avec \menu{close}) l'interface \ct{display} de la pile durant
l'utilisation de la calculatrice.
%Now you should be able to \emph{open} and \emph{close} the stack while you are using the calculator.

Il y a cependant une chose que nous avons oubliée.
Essayez d'effectuer des opérations sur la pile.
Utilisez maintenant le bouton \backbtn{} de votre navigateur web et
essayez encore d'effectuer des opérations sur la pile (\parex
ouvrez la pile, saisissez \menu{1} une fois et \menu{Enter} deux fois
puis \menu{+}. La pile devrait afficher ``2'' et ``1''. Cliquez
maintenant sur la bouton \backbtn. La pile montre encore trois fois
``1''. Mais si vous cliquez sur \menu{+}, la pile affichera ``3''
% ajout - vf
au lieu de ``2''. Le \backtracking{} ne marche pas encore).

\dothis{%
Codez la méthode \ct{MyCalculator>>>states} pour retourner 
% ajout - vf
un tableau contenant
le contenu \ct{contents} de la machine de pile.
Vérifiez que le \backtracking{} fonctionne correctement désormais!}

Détendez-vous et prenez un rafraîchissant: vous l'avez bien mérité!
%Sit back and enjoy a tall glass of something cool!


%=================================================================
\section{Un bref coup d'\oe il sur la technologie AJAX}

% Original text by Lukas Renggli

\ind{AJAX} (Asynchronous \ind{JavaScript} and \ind{XML}) est une
technique pour créer des applications web plus interactives en
exploitant les fonctionalités offertes par \jscript du côté client.

Deux bibliothèques \jscript{} populaires sont \ind{\pjs}\footnote{\url{http://www.prototypejs.org}.} et \ind{\sau}\footnote{\url{http://script.aculo.us}.}.


%Prototype provides a framework to ease writing JavaScript.
% CHANGE - retrait - vf - martial: phrase non traduite parce que redondant
\sau{} étend la librairie \pjs{} en ajoutant des fonctionnalités pour
l'animation et le glissé-déposé (\emph{drag-and-drop}).
%script.aculo.us provides some additional features to support
%animations and drag-and-drop on top of Prototype.
Ces deux bibliothèques sont incluses dans Seaside via la paquetage
\scat{Scriptaculous}.

Toutes les images prêtes à l'emploi ont ce paquetage déjà chargé. La
dernière version est disponible sur
\url{http://www.squeaksource.com/Seaside}.
Une démo en ligne est visible à l'adresse 
\url{http://scriptaculous.seasidehosting.st}.
Si vous avez une image actuellement lancée, pointez tout simplement
votre navigateur web sur la page
\url{http://localhost:8080/seaside/tests/scriptaculous}.

Les extensions \sau{} suivent la même approche que Seaside
lui-même\,---\,configurez des objets Smalltalk pour modéliser votre
application et le code \jscript{} nécessaire sera généré pour vous.

Jetons un \oe il sur un simple exemple pour voir comment le support
\jscript{} côté client peut rendre le réactivité de notre calculatrice
RPN plus naturelle.
%of how client-side Javascript support can make our RPN calculator behave more naturally.
Actuellement tout clic sur un chiffre entraîne une requête pour
rafraîchir la page. Nous aimerions plutôt gérer l'édition de
l'affichage côté client en mettant à jour l'affichage 
% ajout - vf (plus clair)
de la partie \ct{display} de la page existante.
%We would like instead to handle editing of the display on the client-side by updating the display in the existing page.

\dothis{Pour pouvoir communiquer depuis \jscript{} avec des éléments
  précis de l'interface, nous devons tout d'abord donner à ces
  éléments un attribut \emph{id} unique.
%To address the display from JavaScript code we must first give it a unique id.
Changez la méthode de rendu de la calculatrice\footnote{Si
  vous n'avez pas implémenté le tutoriel vous-même, vous pouvez \aretirer{%
  télécharger l'exemple complet \scat{PBE-SeasideRPN} depuis
  \url{http://www.squeaksource.com/PharoByExample} et} appliquer les
modifications suggérées aux classes en \ct{RPN*} au lieu des classes
en \ct{My*}.} comme suit:}

\begin{code}{}
MyCalculator>>>renderContentOn: html
	html div id: 'keypad'; with: keypad.
	html div id: 'display'; with: display.	
\end{code}
				
\dothis{%
% martial - confusion keyboard et keypad dans la version de Lukas -
% REVOIR à signaler dans TODO-cb
Pour pouvoir refaire le rendu du \ct{display} lorsqu'un bouton du
\ct{keypad} est pressé, le composant \ct{keypad} a besoin de connaître
le composant \ct{display}.
%To be able to re-render the display when a keyboard button is
%pressed, the keyboard needs to know the display component.
Ajoutez une variable d'instance \ct{display} à la classe \ct{MyKeypad}
et une méthode d'initialisation \ct{MyKeypad>>>setDisplay:} et
utilisez-la dans la méthode \ct{MyCalculator>>initialize}.
Nous sommes maintenant capable d'adjoindre du code \jscript aux
boutons en mettant à jour
\ct{MyKeypad>>>renderStackButton:callback:colSpan:on:} comme suit:}

\begin{code}{}
MyKeypad>>>renderStackButton: text callback: aBlock colSpan: anInteger on: html 
	html tableData
		class: 'key';
		colSpan: anInteger;
		with: [
			html anchor
				callback: aBlock;
				onClick:				"!gère! les !événements! JavaScript"
					(html updater
						id: 'display';
						callback: [ :r |
							aBlock value.
							r render: display ];
						return: false);
				with: [ html html: text ] ]
\end{code}

\mthind{WATagBrush}{onClick:} indique une gestionnaire d'événements
\ind{\jscript}.
%\ind{JavaScript} event handler.
\ct{html updater} renvoie un \updater \ie une instance de
\ct{SUUpdater}, un objet Smalltalk représentant l'objet JavaScript
Ajax.Updater (\url{http://www.prototypejs.org/api/ajax/updater}).
Cet objet fait une requête AJAX et met à jour, par le texte en
réponse, le contenu d'un conteneur.
%updates a container's contents based on the response text.
Le message \ct{id:} dit à l'\updater quel élément DOM XHTML mettre à jour; ici,
il s'agit du contenu de l'élément \emph{div} d'attribut \emph{id}
``display''.
Un bloc est passé en argument à \ct{callback:} pour se déclencher
quand l'utilisateur cliquera sur le bouton.
L'argument de bloc \ct{r} est \arelire{une nouvelle interface de rendu ou
  \emph{renderer}} qui peut être utilisé pour le rendu du composant
\ct{display}
(remarquez que \arelire{même si le code HTML est toujours accessible,
  il n'est plus valide au moment où ce \callback est évalué}).
%(Even though html is still accessible, it is not valid anymore at the
%time this callback block is evaluated).
Avant de faire le rendu du composant \ct{display}, nous évaluons
\ct{aBlock} pour effectuer l'action désirée.

\ct{return: false} interdit au moteur \jscript de déclencher le
\callback d'origine du lien, ce qui engendrerait un rafraîchissement
complet. %JavaScript engine
Nous pouvons aussi retirer l'ancre d'origine \ct{callback:}, mais en
le laissant, nous sommes sûrs que la calculatrice marchera même si le
\jscript est désactivé.

\dothis{Essayez la calculatrice à nouveau et remarquez que le
  rafraîchissement complet de la page se produit toujours lorsque vous
  cliquez sur un chiffre du \ct{keypad} (autrement dit, l'URL de la
  page web change à chaque clic).}

Bien que nous ayons bien implémenté le comportement du côté client,
nous ne l'avons pas encore activé.
Nous devons donc permettre la gestion des événements \jscript.

\dothis{%
Cliquez sur le lien \link{Configure} dans la \toolbar de la
calculatrice.
Sélectionnez ``Add Library:'' \ct{SULibrary} (pour configurer l'ajout de la
bibliothèque \ct{SULibrary}) et cliquez sur \button{Add} puis
\button{Close}.}

Vous pouvez aussi ajouter la bibliothèque de manière programmatique
(plutôt que manuellement) lorsque vous enregistrez l'application:
\begin{code}{}
MyCalculator class>>>initialize
	(self registerAsApplication: 'rpn')
		addLibrary: SULibrary}}
\end{code}

\begin{figure}[ht]
\begin{center}
\includegraphics[width=\textwidth]{ajax-processing}
\caption{Diagramme de séquences simplifié des interactions AJAX dans notre application Seaside.}
%\caption{Seaside AJAX processing (simplified)}
\figlabel{ajax-processing}
\end{center}
\end{figure}

\dothis{Essayez l'application revisitée. 
Notez que la réponse est beaucoup plus naturelle. En particulier,
aucune nouvelle URL n'est générée lors du clic.}

Vous devez vous demander: ``oui mais, comment ça marche?''.
%You may well ask, \emph{yes, but how does this work?}
\Figref{ajax-processing} montre comment les deux versions\,---\,avec et
sans AJAX\,---\,de l'application RPN (avec et sans AJAX) fonctionnent.
AJAX court-circuite simplement le rendu de façon à mettre à jour le
composant \ct{display} \emph{uniquement}. 
\jscript est responsable à la fois du déclenchement de la requête et
de la mise à jour de l'élément DOM correspondant.
Regardons le code source généré, principalement le code \jscript.

\begin{code}{}
new Ajax.Updater(
	'display',
	'http://localhost/seaside/RPN+Calculator',
	{'evalScripts': true,
	  'parameters': ['UNDERSCOREs=zcdqfonqwbeYzkza', 'UNDERSCOREk=jMORHtqr','9'].join('&')});
return false
\end{code}

Pour des exemples plus avancés, nous vous invitons à visiter la page
\url{http://localhost:8080/seaside/tests/scriptaculous} avec votre
navigateur web.

\paragraph{\emph{Astuces.}}
En cas de soucis du côté serveur, servez-vous du débogueur
Smalltalk. Pour parer aux problèmes côté client, utiliser FireFox
(\url{http://www.mozilla.com}) et FireBug, son débogueur \jscript
(\url{http://www.getfirebug.com/}) en \emph{plugin}.

%=================================================================
\section{Résumé du chapitre}

% ajout - vf (martial: impression de bacler (11/2009): il y a bcp de
% copier-coller dans la VO)
Nous avons vu que:

\begin{itemize}
  \item La façon la plus simple de commencer avec Seaside est de
    télécharger le programme ``Seaside One-Click Experience'' sur
    \url{http://seaside.st};
  \item Lancer ou arrêter le serveur se fait en évaluant \ct{WAKom startOn: 8080} 
ou \ct{WAKom stop} respectivement;
  \item Changer le \emph{login} et mot de passe de l'administrateur
    peut se faire en évaluant \ct{WADispatcherEditor initialize};
  \item \menu{Toggle Halos} permet de visualiser directement le code
    source de l'application, les objets à l'exécution, les feuilles de
    style CSS et le code XHTML;
  \item Envoyer \ct{WAGlobalConfiguration setDeploymentMode} masque la
    \toolbar.
  \item Les applications web Seaside sont composées de composants,
    chacun étant une sous-classe de \ct{WAComponent};
  \item Seul un composant racine (\emph{Root Component}) peut être
    enregistré comme application. 
Il devrait implémenter la méthode de classe \ct{canBeRoot}. Il est
possible d'enregistrer le composant comme application dans la
méthode de classe \ct{initialize} en envoyant
\ct{self registerAsApplication:} \emph{chemin de l'application}.
\aretirer{%
Si vous surchargez \ct{description}, il est possible de retourner une
nom descriptif pour l'application qui sera affiché dans l'éditeur de
configuration}; % ATTENDRE - martial: pas vu du tout, ça!
  \item Pour gérer le chaînage arrière ou \backtracking, un composant
    devait disposer d'une méthode \ct{states} renvoyant un tableau des
    objets dont l'état sera restauré quand l'utilisateur clique sur le
    bouton \backbtn de son navigateur web;
  \item Le rendu d'un composant se fait via la méthode
    \ct{renderContentOn:}. L'argument de cette méthode est un
    \emph{canevas} destiné au rendu XHTML (généralement appelé
    \ct{html});
  \item Un composant peut faire le rendu d'un sous-composant en
    envoyant \ct{self render:} \emph{sous-component};
  \item Le code XHTML est généré de manière programmatique en envoyant
    des messages à des \brushes. Un \brush est obtenu en envoyant un
    message, par exemple \ct{paragraph} ou \ct{div}, au canevas HTML;
  \item Si vous envoyez des messages en cascade à un \brush qui
    contient le message \ct{with:}, ce \ct{with:} devra être le
    dernier message envoyé.
Le message \ct{with:} envoie le contenu \emph{et} le fait le rendu du résultat;
  \item Les actions devrait apparaître uniquement dans des \callbacks
    (ou fonctions de rappel).
Vous ne devez jamais changer l'état de l'application durant la phase de rendu;
  \item Vous pouvez attacher plusieurs éléments graphiques de
    formulaire (ou \emph{widgets}) et autre ancres (ou liens) à des
    variables d'instance munies de méthodes d'accès en envoyant le
    message \ct{on:} \emph{variable d'instance} \ct{of:} \emph{objet}
    au \brush;
  \item Vous pouvez définir la feuille de style CSS pour une
    hiérarchie de composants en définissant la méthode \ct{style} de
    sorte à ce qu'elle retourne une chaîne de caractères contenant la
    feuille de style (pour les applications officiellement déployées,
    il est commun de se référer à une feuille de style externe se
    trouvant à une URL statique);
  \item Les flux de contrôle
%Control flows can be programmed
peuvent être programmés en envoyant \ct{x call: y} où le composant
\ct{x} sera remplacé par \ct{y} jusqu'à ce qu'\ct{y} réponde en
envoyant le message \ct{answer:} \arelire{avec un résultat dans un
  \callback}.
%by sending \ct{answer:} with a result in a callback.
Le receveur de \ct{call:} est habituellement \ct{self} mais peut être
de façon générale n'importe quel composant visible;
  \item Il existe un flux de contrôle
%A control flow 
nommé \emph{task}\,---\,instance d'une sous-classe de \ct{WATask}. Il
devrait implémenter la méthode \ct{go} pour appeler avec le message
\ct{call:} une série de composants dans une séquence de tâches;
  \item Vous pouvez vous \arelire{simplifier le travail de création
    d'interactions simples en utilisant les méthodes utilitaires de
    \ct{WAComponent} telles que \ct{request}, \ct{inform:},
    \ct{confirm:} et \ct{chooseFrom:caption:};}
  \item Pour interdire à l'utilisateur de se servir du bouton \backbtn
    de son navigateur web pour accéder un état d'exécution passé de
    l'application web, vous pouvez isoler des parties de la séquence
    des tâches dans des \transactions en les incluant dans un
    bloc-argument du message \ct{isolate:}.
\end{itemize}
%-----------------------------------------------------------------

%=================================================================
\ifx\wholebook\relax\else 
   \bibliographystyle{jurabib}
   \nobibliography{scg}
   \end{document}
\fi
%=================================================================
