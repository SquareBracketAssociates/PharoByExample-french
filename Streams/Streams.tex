% $Author: oscar $
% $Date: 2007-09-23 11:56:47 +0200 (Sun, 23 Sep 2007) $
% $Revisionn: 12130 $
% traduit par Martial 
% relecture par Rene (Fri, 21 Dec 2007)
% relecture par Rene (Sun, 13 Jan 2008)
% relecture par Rene (Fri,  8 Jan 2010)
% relecture par Rene (Mon,  9 Aug 2010)
% relecture par Rene (Sat, 16 Apr 2011)
% update: Tue Dec 25 12:52:59 CET 2007
% adaptation pour PBE: martial - Wed Sep  9 22:33:36 CEST 2009 from
% $Author: oscar $ % $Date: 2009-08-16 16:37:09 +0200 (Sun, 16 Aug
% 2009) $ % $Revision: 28477 $
% sync avec la version: 30278
%=================================================================
\ifx\wholebook\relax\else
% --------------------------------------------
% Lulu:
	\documentclass[a4paper,10pt,twoside]{book}
	\usepackage[
		papersize={6.13in,9.21in},
		hmargin={.75in,.75in},
		vmargin={.75in,1in},
		ignoreheadfoot
	]{geometry}
	\input{../common.tex}
	\pagestyle{headings}
	\setboolean{lulu}{true}
% --------------------------------------------
% A4:
%	\documentclass[a4paper,11pt,twoside]{book}
%	\input{../common.tex}
%	\usepackage{a4wide}
% --------------------------------------------
    \graphicspath{{figures/} {../figures/}}
	\begin{document}
	\renewcommand{\nnbb}[2]{} % Disable editorial comments
	\sloppy
\fi
%=================================================================
\newcommand{\stream}{\emph{stream}\xspace}
\newcommand{\streams}{\emph{streams}\xspace}
% remarque generale: beaucoup de commentaires et de chaines de caracteres dans les codes ont ete traduites dans ce chapitre
\chapter{Streams: les flux de données}\chalabel{streams}

\ew{Streams are presented as a way to navigate collection. From my point of view, stream are important not to navigate collection, but to produce/consume data:
(a)	memory constraint. Data can not hold into memory and must be processed in a stream fashion, e.g: encryption
(b)	blocking IO. A stream is a nice abstraction to deal with, and the stream manages internally data availability, buffering, etc. to simplify the consumption/production of data
Only few streams have random access capability.}

Les flux de données ou \streams sont utilisés pour itérer dans
une séquence d'éléments comme des
% sequenced
collections, des fichiers ou des flux réseau.
Les \streams peuvent être en lecture ou en écriture ou les deux.
La lecture et l'écriture est toujours relative à la position courante
dans le \stream. Les \streams peuvent être facilement convertis en
collections 
%ajout d'apres la remarque de lukas
(enfin presque toujours)
et les collections en \streams.
\lr{"Streams can easily be converted into collections." I wouldn't say it like this, because it is not true for all streams (infinite streams). According to Kent Beck we should only talk about conversion when the same protocol is supported. Collections and Streams do not support the same protocol. (p. 249)}

%=============================================================
\section{Deux séquences d'éléments}
Voici une bonne métaphore pour comprendre ce qu'est un flux de données:
un flux de données ou \stream peut être représenté comme deux
séquences d'éléments: une séquence d'éléments passée 
et une séquence d'éléments future.
Le \stream est positionné entre les deux séquences.
Comprendre ce modèle est important car toutes les opérations
sur les \streams en \st en dépendent.
C'est pour cette raison que la plupart des
% Attention \clsindmain n'est pas au tout debut 
classes \clsindmain{Stream} sont des sous-classes de \clsind{PositionableStream}.
\Figref{_abcde} présente un flux de données contenant cinq caractères.
Ce \stream est dans sa position originale \ie qu'il n'y a aucun élément
dans le passé. Vous pouvez revenir à cette position en envoyant le message 
\mthind{PositionableStream}{reset}.

\begin{figure}[ht]
\centerline{\includegraphics[scale=0.5]{_abcdeStef}}
\caption{Un flux de données positionné à son origine.}
\figlabel{_abcde}
\vspace{.2in}
\end{figure}

Lire un élément revient conceptuellement à effacer le premier élément de la séquence 
d'éléments future et le mettre après le dernier élément dans la séquence d'éléments passée.
Aprés avoir lu un élément avec le message \ct{next}, l'état de votre \stream est celui de \figref{a_bcde}.

\begin{figure}[ht]
\centerline{\includegraphics[scale=0.5]{a_bcdeStef}}
\caption{Le même flux de données après l'exécution de la méthode \ct{next}: le caractère \ct{a} est ``dans le passé'' alors que \ct{b}, \ct{c}, \ct{d} and \ct{e} sont ``dans le futur''.}
\figlabel{a_bcde}
\vspace{.2in}
\end{figure}

Écrire un élément revient à remplacer le premier élément de la séquence future par le nouveau et le déplacer dans le passé. \Figref{ax_cde} montre l'état du même \stream après avoir écrit un \ct{x} via le message \mthind{Stream}{nextPut:} \ct{anElement}.

\begin{figure}[h!t]
\centerline{\includegraphics[scale=0.5]{ax_cdeStef}}
\caption{Le même flux de données après avoir écrit un \ct{x}.}
\figlabel{ax_cde}
\vspace{.2in}
\end{figure}

%=============================================================
\section{Streams contre Collections}

Le protocole des collections supporte le stockage, l'effacement et l'énumération
des éléments d'une collection mais il ne permet pas que ces opérations
soient combinées ensemble. Par exemple, si les éléments d'une 
\clsind{OrderedCollection} sont traités par une méthode \mthind{OrderedCollection}{do:}, il n'est pas possible d'ajouter ou d'enlever des éléments à l'intérieur du bloc \ct{do:}.
Ce protocole ne permet pas non plus d'itérer dans deux collections
en même temps en choisissant quelle collection on itère, laquelle on n'itère pas.
% ?choosing which collection goes forward
% Serge : oui la phrase n'est pas très claire ... Précision à demander sur la liste sbe
De telles procédures requièrent qu'un index de parcours ou une référence
de position soit maintenu hors de la collection elle-même:
c'est exactement le rôle de  
%traversal index or
\clsind{ReadStream} (pour la lecture), \clsind{WriteStream} (pour l'écriture) et \clsind{ReadWriteStream} (pour les deux).

Ces trois classes sont définies pour \emph{glisser à travers}~\footnote{En anglais, nous dirions ``stream over''.} une collection.
Par exemple, le code suivant crée un \stream sur un intervalle puis y lit deux éléments.
\needlines{5}
\begin{code}{@TEST |r|}
r := ReadStream on: (1 to: 1000).
r next.   --> 1
r next.   --> 2
r atEnd.--> false
\end{code}

Les \ct{WriteStream}{}s peuvent écrire des données dans la collection:
\begin{code}{@TEST |w|}
w := WriteStream on: (String new: 5).
w nextPut: $a.
w nextPut: $b.
w contents. -->  'ab'
\end{code}

Il est aussi possible de créer des \ct{ReadWriteStream}{}s qui supportent
les protocoles de lecture et d'écriture.

Le principal problème de \ct{WriteStream} et de \ct{ReadWriteStream}
est que, dans \pharo, ils ne supportent que les tableaux et les 
chaînes de caractères. Cette limitation est en cours de
disparition grâce au développement d'une nouvelle librairie
nommée \emph{Nile}~\footnote{Disponible à \url{www.squeaksource.com/Nile.html}}. 
mais en attendant, vous obtiendrez une erreur
si vous essayez d'utiliser les \streams avec un autre type de collection:
% REVOIR 

\needlines{3}
\begin{code}{}
w := WriteStream on: (OrderedCollection new: 20).
w nextPut: 12. -->  !\normcomment{lève une erreur}!
\end{code}

Les \streams ne sont pas seulement destinés aux collections mais
aussi aux fichiers et aux \emph{sockets}.
L'exemple suivant crée un fichier appelé \ct{test.txt}, 
y écrit deux chaînes de caractères, séparées par un retour-chariot et enfin ferme le fichier.

\needlines{3}
\begin{code}{}
StandardFileStream
  fileNamed: 'test.txt'
  do: [:str | str
                nextPutAll: '123';
                cr;
                nextPutAll: 'abcd'].
\end{code}
\cmindex{FileStream class}{fileNamed:do:}

Les sections suivantes s'attardent sur les protocoles.

%=============================================================
\section{Utiliser les streams avec les collections}

Les \streams sont vraiment utiles pour traiter des collections d'éléments.
Ils peuvent être utilisés pour la lecture et l'écriture d'éléments
dans des collections. Nous allons explorer maintenant les caractéristiques
des \streams dans le cadre des collections.

%-----------------------------------------------------------------
\subsection{Lire les collections}

Cette section présente les propriétés utilisées pour lire des collections. 
Utiliser les flux de données pour lire une collection repose essentiellement 
sur le fait de disposer d'un pointeur sur le contenu de la collection.
Vous pouvez placer où vous voulez ce pointeur qui avancera dans le contenu pour lire.
La classe \clsindmain{ReadStream} devrait être utilisée pour lire les éléments dans 
les collections.

Les méthodes \mthind{ReadStream}{next} et \mthind{ReadStream}{next:} 
sont utilisées pour récupérer un ou plusieurs éléments dans la collection.

\begin{code}{@TEST |stream|}
stream := ReadStream on: #(1 (a b c) false).
stream next. -->   1
stream next. -->   #(#a #b #c)
stream next. -->   false
\end{code}
\cmindex{PositionableStream class}{on:}

\begin{code}{@TEST |stream|}
stream := ReadStream on: 'abcdef'.
stream next: 0. -->   ''
stream next: 1. -->   'a'
stream next: 3. -->   'bcd'
stream next: 2. -->   'ef'
\end{code}

Le message \mthind{PositionableStream}{peek} est utilisé quand vous voulez
connaître l'élément suivant dans le \stream sans avancer dans le flux.

\begin{code}{@TEST |stream negative number|}
stream := ReadStream on: '-143'.
negative := (stream peek = $-).    "regardez le premier !élément! sans le lire"
negative. --> true
negative ifTrue: [stream next].       "ignore le !caractère! moins"
number := stream upToEnd.
number. --> '143'
\end{code}%$
\noindent
Ce code affecte la variable booléenne \ct{negative} en fonction du signe du nombre dans le \stream et \ct{number} est assigné à sa valeur absolue. 
La méthode \mbox{\mthind{ReadStream}{upToEnd}} (qui en français se traduirait par ``jusqu'à la fin'') renvoie tout depuis la position courante jusqu'à
la fin du flux de données et positionne ce dernier à sa fin.
Ce code peut être simplifié grâce à \mthind{PositionableStream}{peekFor:} qui déplace le pointeur si et seulement si l'élément est égal au paramètre passé en argument.

\begin{code}{@TEST |stream negative number|}
stream := '-143' readStream.
(stream peekFor: $-) --> true
stream upToEnd         --> '143'
\end{code}%$
\noindent
\ct{peekFor:} retourne aussi un booléen indiquant si le paramètre est égal à l'élément courant.

Vous avez dû remarquer une nouvelle façon de construire un \stream dans l'exemple précédent: vous pouvez simplement envoyer  
\mthind{SequenceableCollection}{readStream} à une collection séquentielle pour avoir un flux de données en lecture seule sur une collection.

\paragraph{Positionner.} Il existe des méthodes pour positionner le pointeur du \stream. Si vous connaissez l'emplacement, vous pouvez vous y rendre directement en utilisant \mthind{PositionableStream}{position:}. Vous pouvez demander la position actuelle avec \mthind{PositionableStream}{position}. Souvenez-vous bien 
qu'un \stream n'est pas positionné sur un élément, mais entre deux éléments. L'index 0 correspond au début du flux.

Vous pouvez obtenir l'état du \stream montré dans 
\figref{ab_cde} avec le code suivant:

\begin{code}{@TEST |stream|}
stream := 'abcde' readStream.
stream position: 2.
stream peek --> $c
\end{code}%$

\begin{figure}[h!t]
\centerline{\includegraphics[scale=0.5]{ab_cdeStef}}
\caption{Un flux de données à la position 2.}
\figlabel{ab_cde}
\vspace{.2in}
\end{figure}

Si vous voulez aller au début ou à la fin, vous pouvez utiliser
   %martial: signaler la tournure lourde dans la VO: If you want to go to the beginning or at the end is what you want, you can use
\mthind{PositionableStream}{reset} ou \mthind{PositionableStream}{setToEnd}.
Les messages \mthind{PositionableStream}{skip:} et \mthind{PositionableStream}{skipTo:} sont utilisés pour avancer d'une position relative à la position actuelle: la méthode \ct{skip:} accepte un nombre comme
argument et saute sur une distance de ce nombre d'éléments alors que
\ct{skipTo:} saute tous les éléments dans le flux jusqu'à trouver
un élément égal à son argument. Notez que cette méthode positionne
le \stream après l'élément identifié.

\begin{code}{@TEST |stream|}
stream := 'abcdef' readStream.
stream next.      --> $a    "!le flux est à la position juste après a!"
stream skip: 3.                           "le flux est !après! d"
stream position.  -->   4
stream skip: -2.                          "le flux est !après! b"
stream position.  --> 2
stream reset.
stream position.  --> 0
stream skipTo: $e.                        "le flux est !après! e"
stream next.        --> $f
stream contents. --> 'abcdef'
\end{code}%$

Comme vous pouvez le voir, la lettre \ct{e} a été sautée.

La méthode \mthind{PositionableStream}{contents} retourne toujours une copie de l'intégralité du flux de données.

\paragraph{Tester.} Certaines méthodes vous permettent de tester l'état d'un \stream courant: 
la méthode \mthind{PositionableStream}{atEnd} renvoie \emph{true} si et seulement si aucun élément ne peut être trouvé aprés la position actuelle alors que \mthind{PositionableStream}{isEmpty} renvoie \emph{true} si et seulement si aucun élément ne se trouve dans la collection.

Voici une implémentation possible d'un algorithme utilisant \ct{atEnd} et prenant deux collections triées comme paramètres puis les fusionnant dans une autre collection triée:

\needlines{4}
\begin{code}{@TEST |stream1 stream2 result|}
stream1 := #(1 4 9 11 12 13) readStream.
stream2 := #(1 2 3 4 5 10 13 14 15) readStream.

"!La variable résultante contiendra la collection triée.!"
result := OrderedCollection new.
[stream1 atEnd not & stream2 atEnd not]
  whileTrue: [stream1 peek < stream2 peek
    "!Enlève le plus petit élément de chaque flux et l'ajoute au résultat!"
    ifTrue: [result add: stream1 next]
    ifFalse: [result add: stream2 next]].

"!Un des deux flux peut ne pas être à la position finale. Copie ce qu'il reste!"
result
  addAll: stream1 upToEnd;
  addAll: stream2 upToEnd.

result. -->   an OrderedCollection(1 1 2 3 4 4 5 9 10 11 12 13 13 14 15)
\end{code}
%	from either stream and add it to the result."

%-----------------------------------------------------------------
\subsection{Écrire dans les collections}

Nous avons déjà vu comment lire une collection en itérant sur ses
éléments via un objet \ct{ReadStream}. Apprenons maintenant à créer
des collections avec la classe \clsindmain{WriteStream}.

Les flux de données \ct{WriteStream} sont utiles pour adjoindre des données en plusieurs endroits dans une collection. Ils sont souvent utilisés pour construire des chaînes de caractères basées sur des parties à la fois statiques et dynamiques comme dans l'exemple suivant:

\begin{code}{NB: can't be tested due to the changing number of classes}
stream := String new writeStream.
stream
  nextPutAll: 'Cette image Smalltalk contient: ';
  print: Smalltalk allClasses size;
  nextPutAll: ' classes.';
  cr;
  nextPutAll: 'C'est vraiment beaucoup.'.

stream contents. --> 'Cette image Smalltalk contient: 2322 classes. C'est vraiment beaucoup.'
\end{code}

Par exemple, cette technique est utilisée dans différentes 
implémentations de la méthode \ct{printOn:}. Il existe une manière
plus simple et plus efficace de créer des flux de données si vous êtes
seulement interessé au contenu du \stream:

\begin{code}{@TEST |string|}
string := String streamContents:
		[:stream |
			stream
                 print: #(1 2 3);
                 space;
                 nextPutAll: 'size';
                 space;
                 nextPut: $=;
                 space;
                 print: 3.	].
string. -->   '#(1 2 3) size = 3'
\end{code}%$

La méthode \mthind{SequenceableCollection class}{streamContents:} \seclabel{streamContents} crée une collection et un \stream sur cette collection.
Elle exécute ensuite le bloc que vous lui donnez en passant le \stream comme argument de bloc. Quand le bloc se termine, \ct{streamContents:}
renvoie le contenu de la collection.

Les méthodes de \ct{WriteStream} suivantes sont spécialement utiles dans ce contexte:

\begin{description}
\item[\lct{nextPut:}] ajoute le paramètre au flux de données;
\item[\lct{nextPutAll:}] ajoute chaque élément de la collection passé en argument au flux;
\item[\lct{print:}] ajoute la représentation textuelle du paramètre au flux.
	\cmindex{Stream}{print:}
\end{description}

Il existe aussi des méthodes utiles pour imprimer différentes sortes
de caractères au \stream comme \mthind{WriteStream}{space} (pour un espace), 
   \mthind{WriteStream}{tab} (pour une tabulation) et
   \mthind{WriteStream}{cr} (pour \emph{Carriage Return} \cad le retour-chariot).
Une autre méthode s'avère utile pour s'assurer que le dernier caractère
dans le flux de données est un espace: il s'agit de \mthind{WriteStream}{ensureASpace}; si le dernier caractère n'est pas un espace, il en ajoute un.

\paragraph{Au sujet de la concaténation.}
L'emploi de \mthind{WriteStream}{nextPut:} et de \mthind{WriteStream}{nextPutAll:} sur un \ct{WriteStream} est souvent le meilleur moyen pour concaténer 
les caractères.
L'utilisation de l'opérateur virgule (\ct{,}) est beaucoup moins efficace:
\index{Collection!opérateur virgule}

\begin{code}{}
[| temp |
  temp := String new.
  (1 to: 100000)
    do: [:i | temp := temp, i asString, ' ']] timeToRun --> 115176 "(ms)"

[| temp |
  temp := WriteStream on: String new.
  (1 to: 100000)
    do: [:i | temp nextPutAll: i asString; space].
  temp contents] timeToRun --> 1262 "(milliseconds)"
\end{code}

La raison pour laquelle l'usage d'un \stream est plus efficace provient
du fait que l'opérateur virgule crée une nouvelle chaîne de caractères
contenant la concaténation du receveur et de l'argument, donc il doit
les copier tous les deux.
Quand vous concaténez de manière répétée sur le même receveur,
ça prend de plus en plus de temps à chaque fois; le nombre
de caractères copiés s'accroît de façon exponentielle.
Cet opérateur implique aussi une surcharge de travail pour le ramasse-miettes qui collecte ces chaînes. 
% ajout de 'Pour ce cas' pour suggerer que ca ne concerne specialement les gros travaux sur les chaines, pour les petites accumulations, je suis assez d'accord avec lukas (a discuter dans la sbe-discussion)
Pour ce cas, utiliser un \stream plutôt qu'une concaténation de chaînes est une optimisation bien connue.
\lr{About Concatenation. This is not true for real world examples (the example benchmark is unrealistic). Most of the time concatenation is simpler, cleaner and much faster, for example when being used on a small number of longer strings. (p. 257)}
En fait, vous pouvez utiliser la méthode de classe \mthind{SequenceableCollection class}{streamContents:} (mentionnée à la page \pageref{sec:streamContents}) pour parvenir à ceci:

\begin{code}{}
String streamContents: [ :tempStream |
  (1 to: 100000)
       do: [:i | tempStream nextPutAll: i asString; space]] 
\end{code}

%-----------------------------------------------------------------
\subsection{Lire et écrire en même temps}

Vous pouvez utiliser un flux de données pour accéder à une collection
en lecture et en écriture en même temps.
Imaginez que vous voulez créer une classe d'historique que nous appelerons \ct{History} et qui gérera
les boutons ``Retour'' (\emph{Back}) et ``Avant'' (\emph{Forward}) d'un
navigateur web.
Un historique réagirait comme le montrent les illustrations depuis \ref{fig:emptyStream} jusqu'à \ref{fig:page4}.

\begin{figure}[!ht]
\centerline{\includegraphics[scale=0.5]{emptyStef}}
\caption{Un nouvel historique est vide. Rien n'est affiché dans le navigateur web.}
\figlabel{emptyStream}
\vspace{.2in}
\end{figure}

\begin{figure}[!ht]
\centerline{\includegraphics[scale=0.5]{page1Stef}}
\caption{L'utilisateur ouvre la page 1.}
\figlabel{page1}
\vspace{.2in}
\end{figure}

\begin{figure}[!ht]
\centerline{\includegraphics[scale=0.5]{page2Stef}}
\caption{L'utilisateur clique sur un lien vers la page 2.}
\figlabel{page2}
\vspace{.2in}
\end{figure}

\begin{figure}[!ht]
\centerline{\includegraphics[scale=0.5]{page3Stef}}
\caption{L'utilisateur clique sur un lien vers la page 3.}
\figlabel{page3}
\vspace{.2in}
\end{figure}

\begin{figure}[!ht]
\centerline{\includegraphics[scale=0.5]{page2_Stef}}
\caption{L'utilisateur clique sur le bouton ``Retour'' (Back). Il visite désormais la page 2 à nouveau.}
\figlabel{page2_}
\vspace{.2in}
\end{figure}

\begin{figure}[!ht]
\centerline{\includegraphics[scale=0.5]{page1_Stef}}
\caption{L'utilisateur clique sur le bouton ``Retour'' (Back). La page 1 est affichée maintenant.}
\figlabel{page1_}
\vspace{.2in}
\end{figure}

\begin{figure}[!ht]
\centerline{\includegraphics[scale=0.5]{page4Stef}}
\caption{Depuis la page 1, l'utilisateur clique sur un lien vers la page 4. L'historique oublie les pages 2 et 3.}
\figlabel{page4}
\vspace{.2in}
\end{figure}

Ce comportement peut être programmé avec un \clsind{ReadWriteStream}.

\needlines{6}
\begin{code}{}
Object subclass: #History
  instanceVariableNames: 'stream'
  classVariableNames: ''
  poolDictionaries: ''
  category: 'PBE-Streams'

History>>initialize
    super initialize.
    stream := ReadWriteStream on: Array new.
\end{code}

Nous n'avons rien de compliqué ici; nous définissons une nouvelle classe
qui contient un \stream. Ce \stream est créé dans la méthode \ct{initialize} 
%ajout
depuis un tableau.

Nous avons besoin d'ajouter les méthodes \ct{goBackward} et \ct{goForward} pour aller respectivement en arrière (``Retour'') et en avant:

\begin{code}{}
History>>goBackward
  self canGoBackward ifFalse: [self error: '!\normcode{Déjà sur le premier élément}!'].
  stream skip: -2. 
  ^ stream next

History>>goForward
  self canGoForward ifFalse: [self error: '!\normcode{Déjà sur le dernier élément}!'].
  ^ stream next
\end{code}

Jusqu'ici le code est assez simple. Maintenant, nous devons nous occuper
de la méthode \ct{goTo:} (que nous pouvons traduire en français par ``aller à'') qui devrait être activée quand l'utilisateur clique sur un lien. Une solution possible est la suivante:

\begin{code}{}
History>>goTo: aPage
    stream nextPut: aPage.
\end{code}

Cette version est cependant incomplète. Ceci vient du fait que lorsque l'utilisateur clique sur un lien, il ne devrait plus y avoir de pages futurs \ie que le bouton ``Avant'' devrait être désactivé.
Pour ce faire, la solution la plus simple est d'écrire \ct{nil}
juste après la position courante pour indiquer la fin de l'historique:

\begin{code}{}
History>>goTo: anObject
  stream nextPut: anObject.
  stream nextPut: nil.
  stream back.
\end{code}

Maintenant, seules les méthodes \ct{canGoBackward} (pour dire si oui ou non nous pouvons aller en arrière) et \ct{canGoForward} (pour dire si oui ou non nous pouvons aller en avant) sont à coder.

Un flux de données est toujours positionné entre deux éléments.
Pour aller en arrière, il doit y avoir deux pages avant la position courante:
une est la page actuelle et l'autre est la page que nous voulons atteindre.

\begin{code}{}
History>>canGoBackward
  ^ stream position > 1

History>>canGoForward
  ^ stream atEnd not and: [stream peek notNil]
\end{code}

Ajoutons pour finir une méthode pour accéder au contenu du \stream:
\begin{code}{}
History>>contents
  ^ stream contents
\end{code}

Faisons fonctionner maintenant notre historique 
%ajout (plus claire)
comme dans la séquence illustrée plus haut:
\begin{code}{}
History new
	goTo: #page1;
	goTo: #page2;
	goTo: #page3;
	goBackward;
	goBackward;
	goTo: #page4;
	contents --> #(#page1 #page4 nil nil)
\end{code}

%=============================================================
\section{Utiliser les streams pour accéder aux fichiers}

Vous avez déjà vu comment glisser sur une collection d'éléments via
un \stream. Il est aussi possible d'en faire de même avec un flux 
sur des fichiers de votre disque dur.
Une fois créé, un \stream sur un fichier est comme un \stream sur
une collection: vous pourrez utiliser le même protocole pour lire, écrire
ou positionner le flux.
La principale différence apparaît à la création du flux de données.
Nous allons voir qu'il existe plusieurs manières de créer un \stream sur un fichier.

%-----------------------------------------------------------------
\subsection{Créer un flux pour fichier}
\seclabel{creat-file-stre}

Créer un \stream sur un fichier consiste à utiliser une des méthodes
de création d'instance suivantes mises à disposition par la classe
\clsindmain{FileStream}:

\begin{description}

\item[\lct{fileNamed:}] ouvre en lecture et en écriture un fichier 
  avec le nom donné. Si le fichier existe déjà, son contenu pourra
  être modifié ou remplacé mais le fichier ne sera pas tronqué
  à la fermeture. Si le nom n'a pas de chemin spécifié pour répertoire,
  le fichier sera créé dans le répertoire par défaut.
  \cmindex{FileStream class}{fileNamed:}
  
\item[\lct{newFileNamed:}] crée un nouveau fichier avec le nom donné
	et retourne un \stream ouvert en écriture pour ce fichier.
	Si le fichier existe déjà, il est demandé à l'utilisateur
	de choisir la marche à suivre.
  \cmindex{FileStream class}{newFileNamed:}
  
\item[\lct{forceNewFileNamed:}] crée un nouveau fichier avec le nom donné
	et répond un \stream ouvert en écriture sur ce fichier.
	Si le fichier existe déjà, il sera effacé avant qu'un nouveau
	ne soit créé.
  \cmindex{FileStream class}{forceNewFileNamed:}

\item[\lct{oldFileNamed:}] ouvre en lecture et en écriture un fichier 
	existant avec le nom donné. Si le fichier existe déjà, son 
	contenu pourra être modifié ou remplacé mais le fichier ne sera
	pas tronqué à la fermeture. Si le nom n'a pas de chemin spécifié
	pour répertoire, le fichier sera créé dans le répertoire par
	défaut.
  \cmindex{FileStream class}{oldFileNamed:}

\item[\lct{readOnlyFileNamed:}] ouvre en lecture seule un fichier 
	existant avec le nom donné.
  \cmindex{FileStream class}{readOnlyFileNamed:}

\end{description}

Vous devez vous remémorer de fermer le \stream sur le fichier que vous avez ouvert. Ceci se fait grâce à la méthode \mthind{FileStream}{close}.

\begin{code}{@TEST |stream|}
stream := FileStream forceNewFileNamed: 'test.txt'.
stream
    nextPutAll: '!\normcode{Ce texte est écrit dans un fichier nommé}! ';
    print: stream localName.
stream close.

stream := FileStream readOnlyFileNamed: 'test.txt'.
stream contents. --> '!\normcode{Ce fichier est écrit dans un fichier nommé}! ''test.txt'''
stream close.
\end{code}

% \on{need way to clean up test files afterwards}

%[:fileName | (FileDirectory forFileName: fileName)
%	deleteFileNamed: fileName
%	ifAbsent: [ 'Could not delete ', fileName ] ]
%	value: 'test.txt'

La méthode \mthind{FileStream}{localName} retourne le dernier composant du nom du fichier. 
Vous pouvez accéder au chemin entier en utilisant la méthode
\mthind{StandardFileStream}{fullName}.

Vous remarquerez bientôt que la fermeture manuelle de \stream de fichier
est pénible et source d'erreurs. C'est pourquoi \ct{FileStream}
offre un message appelé \mthind{FileStream class}{forceNewFileNamed:do:} 
pour fermer automatiquement un nouveau flux de données après
avoir évalué un bloc qui modifie son contenu.

\begin{code}{@TEST |string|}
FileStream
    forceNewFileNamed: 'test.txt'
    do: [:stream |
        stream
            nextPutAll: '!\normcode{Ce texte est écrit dans un fichier nommé}! ';
            print: stream localName].
string := FileStream
            readOnlyFileNamed: 'test.txt'
            do: [:stream | stream contents].
string --> '!\normcode{Ce fichier est écrit dans un fichier nommé}! ''test.txt'''
\end{code}

Les méthodes de création de flux de données prenant un bloc comme
argument créent d'abord un \stream sur un fichier, puis exécute
un argument et enfin ferme le \stream. Ces méthodes retournent ce qui est retourné par 
le bloc, \ie la valeur de la dernière
expression dans le bloc. C'est ce que nous avons utilisé dans
l'exemple précédent pour récupérer le contenu d'un fichier
et le mettre dans la variable \ct{string}.

%-----------------------------------------------------------------
\subsection{Les flux binaires}
\seclabel{binary-streams}

Par défaut, les \streams créés sont à base textuelle ce qui signifie
que vous lirez et écrirez des caractères.
Si votre flux doit être binaire, vous devez lui envoyer le message 
\mthind{FileStream}{binary}.

Quand votre \stream est en mode binaire, vous pouvez seulement écrire
des nombres de 0 à 255 (ce qui correspond à un octet). Si
vous voulez utiliser \ct{nextPutAll:} pour écrire plus d'un
nombre à la fois, vous devez passer comme
argument un tableau d'octets de la classe \clsind{ByteArray}.

\begin{code}{@TEST}
FileStream
  forceNewFileNamed: 'test.bin'
  do: [:stream |
          stream
            binary;
            nextPutAll: #(145 250 139 98) asByteArray].

FileStream
  readOnlyFileNamed: 'test.bin'
  do: [:stream |
          stream binary.
          stream size.         --> 4
          stream next.         --> 145
          stream upToEnd. --> #[250 139 98]
      ].
\end{code}

Voici un autre exemple créant une image dans un fichier nommé
``test.pgm'' que vous pourrez ouvrir avec votre outil graphique
préféré.

% The following does not assert anything, but @TEST is used to ensure
% that no error is thrown.
\begin{code}{@TEST}
FileStream
  forceNewFileNamed: 'test.pgm' 
  do: [:stream |
	stream
		nextPutAll: 'P5'; cr;
		nextPutAll: '4 4'; cr;
		nextPutAll: '255'; cr;
		binary;
		nextPutAll: #(255 0 255 0) asByteArray;
		nextPutAll: #(0 255 0 255) asByteArray;
		nextPutAll: #(255 0 255 0) asByteArray;
		nextPutAll: #(0 255 0 255) asByteArray
	]
\end{code}

Cela crée un échiquier 4 par 4 comme nous montre \figref{checkerboard4x4}.

\begin{figure}[!ht]
\centerline{\includegraphics[width=0.25\textwidth]{checkerboard4x4}}
\caption{Un échiquier 4 par 4 que vous pouvez dessiner en utilisant des \streams binaires.}
\figlabel{checkerboard4x4}
\vspace{.2in}
\end{figure}

%=============================================================
\section{Résumé du chapitre}

Par rapport aux collections, les flux de données ou \streams offrent
un bien meilleur moyen de lire et d'écrire de manière
incrémentale dans une séquence d'éléments.
Il est très facile de passer par conversion de \streams à collections et vice-versa.

\begin{itemize}
  \item Les flux peuvent être soit en lecture, soit en écriture, soit à la fois en lecture-écriture.
  \item Pour convertir une collection en un \stream, définissez un \stream
sur une collection grâce au message \ct{on:}, \eg \ct{ReadStream on: (1 to: 1000)}, ou via les messages \ct{readStream}, \etc ... sur la collection.
  \item Pour convertir un \stream en collection, envoyer le message \ct{contents}.
  \item Pour concaténer des grandes collections, il est plus efficace d'abandonner l'emploi de l'opérateur virgule \ct{,} et de créer un \stream et y adjoindre les collections avec le message \ct{nextPutAll:} puis extraire enfin le résultat en lui envoyant \ct{contents}.
  \item Par défaut, les \streams de fichiers sont à base de caractères.
Envoyer le message \ct{binary} en fait explicitement des \streams binaires.
\end{itemize}

%=================================================================
\ifx\wholebook\relax\else\end{document}\fi
%=================================================================

%-----------------------------------------------------------------

%%% Local Variables: 
%%% coding: utf-8
%%% mode: latex
%%% TeX-master: t
%%% TeX-PDF-mode: t
%%% End:
