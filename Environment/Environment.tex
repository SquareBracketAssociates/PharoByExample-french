% $Author: oscar $
% $Date: 2007-09-23 11:56:47 +0200 (Sun, 23 Sep 2007) $
% $Translation: martial $
% $Date: Wed Oct 10 17:22:05 CEST 2007 $
% $Revision: 12715 $ 
% translated by Martial.Boniou@ifrance.com start: (Fri, 5 Oct 2007)
% relecture : par Rene Mages  (Wed 9 Jan 2008) de la version 14879
% adaptation pour PBE - martial - Tue Sep  8 21:34:54 CEST 2009 
% relecture : par Rene Mages  (Sun 10 Jan 2010) de la version 30225
% relecture : par Rene Mages  (Sun 8 Aug 2010) de la version 34361
% relecture : par Rene Mages  (Wed 13 Apr 2011) 
% $Author: black$ $Date: 2009-09-06 15:31:38 +0200 (Sun, 06 Sep 2009) $
% $Revision: 28940 $
% $Revision: 29158 $
% 2010-03-05 - Alexandre minor correction (thanks Ralph Boland)
% sync avec la revision: 33691
%==================================================================
\ifx\wholebook\relax\else
% --------------------------------------------
% Lulu:
	\documentclass[a4paper,10pt,twoside]{book}
	\usepackage[
		papersize={6.13in,9.21in},
		hmargin={.75in,.75in},
		vmargin={.75in,1in},
		ignoreheadfoot
	]{geometry}
	\input{../common.tex}
	\pagestyle{headings}
	\setboolean{lulu}{true}
% --------------------------------------------
% A4:
%	\documentclass[a4paper,11pt,twoside]{book}
%	\input{../common.tex}
%	\usepackage{a4wide}
% --------------------------------------------
    \graphicspath{{figures/} {../figures/}}
	\begin{document}
	%\renewcommand{\nnbb}[2]{} % Disable editorial comments
	\sloppy
\fi
%=================================================================
%\newcommand{\debugHandle}{\raisebox{-0.4ex}{\includegraphics[width=1em]{debugHandle}}}

%=================================================================
\chapter{\titreEnvironment}
\chalabel{env}

% cf. Email 2009-10-21 \hw dans PharoBook/Environment/Environment.tex

% martial: global index
%\seeindex{souris!bouton jaune}{bouton jaune} % REVOIR
%\seeindex{souris!bouton rouge}{bouton rouge} % REVOIR
%\seeindex{souris!bouton bleu}{bouton bleu}   % REVOIR

L'objectif de ce chapitre est de vous montrer comment développer des programmes 
dans l'environnement de programmation de \pharo.
Vous avez déjà vu comment définir des méthodes et des classes
en utilisant le navigateur de classes; ce chapitre
vous présentera plus de caractéristiques du Browser ainsi que
d'autres navigateurs. % CHANGE

Bien entendu, vous pouvez occasionnellement rencontrer
des situations dans lesquelles votre programme ne marche pas comme voulu.
\pharo a un excellent débogueur, mais comme la plupart des outils puissants, il peut s'avérer déroutant au début.
Nous vous en parlerons au travers de sessions de débogages et vous
montrerons certaines de ses possibilités.

Lorsque vous programmez, vous le faites dans un monde d'objets vivants et
non dans un monde de programmes textuels statiques; c'est 
une des particularités uniques de Smalltalk.
Elle permet d'obtenir une réponse très rapide de vos programmes et vous rend plus productif. 
Il y a deux outils vous permettant l'observation et aussi la modification de ces objets 
vivants: l'\emph{Inspector} (ou inspecteur) et l'\emph{Explorer} (ou explorateur).

La programmation dans un monde d'objets vivants plutôt qu'avec des fichiers et un éditeur 
de texte vous oblige à agir explicitement pour exporter votre programme depuis l'image \st.

La technique traditionnelle, aussi supportée par tous les dialectes \st consiste à créer un 
fichier d'exportation \emph{fileout} ou une archive d'échange dit \changeset. Il s'agit 
principalement de fichiers textes encodés pouvant être importés dans un autre système.
Une technique plus récente de \pharo est le chargement de code dans un dépôt de versions sur un serveur.
Elle est plus efficace surtout en travail coopératif et est rendue possible via un outil nommé \ind{Monticello}.
%\seeindex{change set}{file, filing out}
\seeindex{change set}{fichier, exportation}
%\index{file!filing out}
\index{fichier!exportation}

Finalement, en travaillant, vous pouvez trouver un \emph{bug} (dit aussi bogue) dans \pharo;
nous vous expliquerons aussi comment reporter les bugs
et comment soumettre les corrections de bugs ou \emph{bug fixes}.
\ab{Or I would, if I knew how.   We should do this, or remove the paragraph.}

%=========================================================
\section{Une vue générale}
%Overview
\seclabel{overview}

Smalltalk et les interfaces graphiques modernes ont été développées ensemble.
Bien avant la première sortie publique de Smalltalk en 1983, Smalltalk
avait un environnement de développement graphique écrit lui-même en Smalltalk et
tout le développement est intégré à cet environnement.
Commençons par jeter un coup d'\oe il sur les principaux outils de
\pharo. % CHANGE

\begin{itemize}
	\item Le {\menu{Browser}} ou \emph{navigateur de classes} est l'outil de développement central.
Vous l'utiliserez pour créer, définir et organiser vos classes et vos méthodes. Avec lui, vous pourrez aussi naviguer dans toutes les classes de la bibliothèque de classes: contrairement aux autres environnements où le code source est réparti dans des fichiers séparés, en \st toutes les classes et méthodes sont contenues dans l'image.
	\index{Browser}
	\index{navigateur de classes}
    \seeindex{Browser}{navigateur de classes}

	\item L'outil {\menu{Message Names}} sert à voir toutes les méthodes ayant un sélecteur (noms de messages sans argument) spécifique ou dont le sélecteur contient une certaine sous-chaîne de caractères.
	\index{Message Name Finder}
	
	\item Le {\menu{Method Finder}} vous permet aussi de trouver des méthodes, soit selon leur \emph{comportement}, soit en fonction de leur nom.
	\index{Method Finder}
	
	\item Le {\menu{Monticello Browser}} est le point de départ pour le chargement ou la sauvegarde de code via des paquetages \ind{Monticello} dit aussi \emph{packages}.
	
	\item Le {\menu{Process Browser} offre une vue sur l'ensemble des processus (threads) exécutés dans \st.}
	\index{Process Browser}
	
	\item Le {\menu{Test Runner}} permet de lancer et de déboguer les tests unitaires \SUnit. Il est décrit dans \charef{SUnit}.
	\index{Test Runner}
	\index{SUnit}
	
	\item Le {\menu{Transcript}} est une fenêtre sur le flux de données sortant de \glbind{Transcript}. Il est utile pour écrire des fichiers-journaux ou \emph{log} et a déjà été étudié dans \secref{transcript}.
	
	\item Le {\menu{Workspace}} ou \emph{espace de travail} est une fenêtre dans laquelle vous pouvez entrer des commandes.  
	Il peut être utilisé dans plusieurs buts mais il l'est plus généralement pour taper des expressions Smalltalk et les exécuter avec 
\menu{do it}~\footnote{En anglais, \emph{do it} correspond à l'exclamation ``fais-le!''.}. L'utilisation de \ind{Workspace} a déjà été vu dans \secref{transcript}.
\end{itemize}

L'outil \menu{Debugger} a un rôle évident, mais vous découvrirez qu'il a une place plus centrale comparé aux débogueurs d'autres langages de programmation
% en comparaison des débogueurs dans d'autres langages de programmation 
car en \st vous pouvez \emph{programmer} dans l'outil \ind{Debugger}.  Il n'est pas lancé depuis un menu; il apparaît normalement en situation d'erreur,
en tapant \short{\textbf{.}} pour interrompre un processus lancé ou
encore en insérant une expression \ct{self halt} dans le code.
\index{processus!interruption}


%=========================================================
\section{Le Browser} % CHANGE
\seclabel{browser} % (fold)
%apb: what does the fold comment mean?

De nombreux navigateurs de classes ou \emph{browsers} ont été
développés pour \st durant des années. \pharo simplifie l'histoire
en proposant un unique navigateur disposant de multiples vues,
le \ind{Browser}. 
\Figref{SystemBrowser0} présente le Browser tel qu'il apparaît
lorsque vous l'ouvrez pour la première fois\footnote{Rappelez-vous que si votre navigateur ne ressemble pas à 
celui présenté sur \figref{SystemBrowser0}, vous pourriez avoir besoin 
de changer le navigateur par défaut (voir \faqref{packagebrowser}).}.
% REVOIR figref SystemBrowser0  au lieu  de classbrowser

\begin{figure}[htbp]
   \centering
   \ifluluelse
	 {\includegraphics[width=\textwidth]{SystemBrowser0} }
	 {\includegraphics[width=0.7\textwidth]{SystemBrowser0} }
   \caption{Le navigateur de classes.}
   \figlabel{SystemBrowser0}
\end{figure}

Les quatre petits panneaux en haut du Browser représentent la vue
hiérarchique des méthodes dans le système de la même
manière que le \textit{File Viewer} de \ind{NeXTstep} et le
\textit{Finder} de Mac OS X fournissent une vue en colonnes 
des fichiers du disque.

Le premier panneau de gauche liste les \emph{paquetages} de
classes ou, en anglais, \emph{packages};
sélectionnez-en une (disons \scat{Kernel}) et alors le
panneau immédiatemment à droite affichera toutes les classes incluses
dans ce paquetage. % CHANGE

\begin{figure}[htbp]
   \centering
   \ifluluelse
	   {\includegraphics[width=\textwidth]{SystemBrowser1} }
	   {\includegraphics[width=0.7\textwidth]{SystemBrowser1} }
   \caption{Le Browser avec la classe \ct{Model} sélectionnée.}
   \figlabel{SystemBrowserModel}
\end{figure}

De même, si vous sélectionnez une des classes de ce second panneau,
disons \menu{Model} (voir  \figref{SystemBrowserModel}), le troisième
panneau vous affichera tous les \emph{protocoles} définis pour cette
classe ainsi qu'un protocole virtuel \mbox{\prot{-{}-all-{}-}} (désignant l'ensemble 
des méthodes). Ce dernier est sélectionné par défaut. 
Les protocoles sont une façon de catégoriser les méthodes;
ils rendent la recherche des méthodes plus facile et détaillent
le comportement d'une classe en le découpant en petites divisions 
cohérentes.
% they make it easier to find and think about the behaviour of a class by breaking it up into smaller, conceptually coherent pieces.  
Le quatrième panneau montre les noms de toutes les méthodes définies dans le protocole sélectionné.
Si vous sélectionnez enfin un nom de méthode, le code source de la 
méthode correspondante apparaît dans le grand panneau inférieur
du navigateur. Là, vous pouvez voir, éditer et sauvegarder la version éditée.
Si vous sélectionnez la classe
\mbox{\menu{Model},} le protocole \mbox{\protind{dependents}} et la méthode 
\mbox{\menu{myDependents},} 
le navigateur devrait ressembler à \figref{SystemBrowserMyDependents}.
\protindex{all}
\cmindex{Model}{myDependents}

\begin{figure}[htbp]
   \centering
   \ifluluelse
	   {\includegraphics[width=\textwidth]{SystemBrowserMyDependents}}
	   {\includegraphics[width=0.7\textwidth]{SystemBrowserMyDependents}}
   \caption{Le Browser affichant la méthode \ct{myDependents} de la classe \ct{Model}.
   \figlabel{SystemBrowserMyDependents}}
\end{figure}

Contrairement aux répertoires du \emph{Finder} de Mac OS X, les quatre panneaux 
supérieurs ne sont aucunement égaux.
Là où les classes et les méthodes font partie du langage \st,
les paquetages et les protocoles ne sont que des commodités
introduites par le navigateur pour limiter la quantité d'information que chaque panneau pourrait présenter.
Par exemple, s'il n'y avait pas de protocoles, le navigateur devrait afficher
la liste de toutes les méthodes dans la classe choisie; pour la plupart des classes, 
cette liste sera trop importante pour être parcourue aisément.
\index{Mac OS X Finder}

De ce fait, la façon dont vous créez un nouveau paquetage ou
un nouveau protocole est différent de la manière avec laquelle
vous créez une nouvelle classe ou une nouvelle méthode. 
Pour créer un nouveau paquetage, sélectionnez 
 \menu{new package} dans le menu contextuel accessible 
en \actclickant dans le  panneau des paquetages; 
pour créer un nouveau protocole, sélectionnez \menu{new protocol} via 
le menu accessible en \actclickant dans le panneau des protocoles.
Entrez le nom de la nouvelle entité (paquetage ou protocole) dans
la zone de saisie, et voilà! 
% dialog
Un paquetage ou un protocole, ça n'est qu'un nom et son contenu.
\index{paquetage!création}

\begin{figure}[htbp]
   \centering
   \ifluluelse
	   {\includegraphics[width=\textwidth]{SystemBrowserClassCreation}}
	   {\includegraphics[width=0.7\textwidth]{SystemBrowserClassCreation}}
   \caption{Le Browser montrant le patron de création de classe.
   \figlabel{SystemBrowserClassCreation}}
\end{figure}

À l'opposé, créer une classe ou une méthode nouvelle nécessite 
l'écriture de code \st.
Si vous \clickz sur le paquetage actuellement sélectionné dans le
panneau de gauche, le panneau inférieur affichera un patron de création de classe
(voir \figref{SystemBrowserClassCreation}).
Vous créez une nouvelle classe en éditant ce patron ou \emph{template}. 
Pour ce faire, remplacez \ct{Object} par le nom de la classe existante
que vous voulez dériver, puis remplacez \ct{NameOfSubclass} par le nom
que vous avez choisi pour votre nouvelle classe (sous-classe de la première) 
et enfin, remplissez la liste des noms de variables d'instance si vous en connaissez.  
La catégorie pour la nouvelle classe est par défaut la
catégorie du paquetage actuellement sélectionné\footnote{Rappelez-vous que paquetages et catégories ne sont pas exactement la même chose. 
Nous verrons la relation qui existe entre eux dans \secref{packages}.}, 
mais vous pouvez à loisir la changer si vous voulez.
Si vous avez déjà la classe à dériver sélectionnée dans 
le Browser, vous pouvez obtenir le même patron avec une initialisation
quelque peu différente en \actclickant dans le panneau des classes et en sélectionnant 
\menu{class templates \ldots \go subclass template}.
Vous pouvez aussi éditer simplement la définition de la classe existante en changeant 
le nom de la classe en quelque chose d'autre.
Dans tous les cas, à chaque fois que vous acceptez la nouvelle définition, la nouvelle classe 
(celle dont le nom est précédé par \ct{#}) est créée (ainsi que sa méta-classe associée).  
Créer une classe crée aussi une variable globale référençant
la classe. En fait, l'existence de celle-ci vous permet de vous 
référer à toutes les classes existantes en utilisant leur nom.
\index{classe!création}
\seeindex{classe!création}{Browser, définir une classe}
\index{Browser!définir une classe}
\seeindex{navigateur de classes!définition d'une classe}{Browser, définir une classe}

Voyez-vous pourquoi le nom d'une nouvelle classe doit apparaître
comme un \clsind{Symbol} (\ie préfixé avec \ct{#}) dans le
patron de création de classe, mais qu'après la création
de classe, le code peut s'y référer en utilisant
son nom comme identifiant (\ie sans le \ct{#}).

Le processus de création d'une nouvelle méthode
est similaire. Premièrement sélectionnez la classe dans laquelle vous
voulez que la méthode apparaisse, puis sélectionnez un protocole.
Le navigateur affichera un patron de création de méthode que
vous pouvez remplir et éditer, comme indiqué par
\figref{SystemBrowserMethodTemplate}.
\index{méthode!création}
\seeindex{méthode!création}{Browser, définir une méthode}
\index{Browser!définir une méthode}
\seeindex{navigateur de classes!définition d'une méthode}{Browser, définir une méthode}

\begin{figure}[htbp]
   \centering
   \ifluluelse
	   {\includegraphics [width=\textwidth]{SystemBrowserMethodTemplate}}
	   {\includegraphics[width=.7\textwidth]{SystemBrowserMethodTemplate}}
   \caption{Le Browser montrant le patron de création de méthode.
   \figlabel{SystemBrowserMethodTemplate}}
\end{figure}

%---------------------------------------------------------
\subsection{Naviguer dans l'espace de code} % CHANGE
\seclabel{ButtonBar}

Le navigateur de classes fournit plusieurs outils pour l'exploration et 
l'analyse de code. 
Ces outils sont accessibles en \actclickant dans divers menus
contextuels ou (pour les outils les plus communs) via des
raccourcis-clavier.

\subsubsection{Ouvrir une nouvelle fenêtre de Browser}
\seclabel{browsing}

Vous aurez besoin parfois d'ouvrir de multiples navigateurs de classes.
Lorsque vous écrivez du code, vous aurez presque toujours besoin d'au moins deux
fenêtres: une pour la méthode que vous éditez et une pour naviguer
dans le reste du système pour y voir ce dont vous aurez besoin pour la
méthode éditée dans la première. % CHANGE
Vous pouvez ouvrir un Browser sur une classe en sélectionnant son nom et en
utilisant le raccourci-clavier \short{b} \ind{raccourci-clavier}.
\index{Browser!browse} % REVOIR !bouton!
\seeindex{navigateur de classes!browse}{Browser, browse} % CHANGE
\index{raccourci-clavier!browse it}

\dothis{Essayez ceci: dans un espace de travail ou Workspace, saisissez le nom d'une 
classe (par exemple, \ct{Morph}), sélectionnez-le et pressez \short{b}. Cette astuce 
est souvent utile; elle marche depuis n'importe quelle fenêtre de texte.}

\subsubsection{\Senders et \implementors d'un message}
\seclabel{sendersImplementors}

\index{Browser!senders}%\index{Browser!bouton!senders}

\Actclick sur \menu{browse \ldots \go senders (n)} dans le
panneau des méthodes vous renvoie une liste de toutes les méthodes
pouvant utiliser la méthode sélectionnée. En prenant le Browser
ouvert sur \ct{Morph}, \clickz sur la méthode  \mthind{Morph}{drawOn:}
dans le panneau des méthodes; le corps de \ct{drawOn:} s'affiche dans
la partie inférieure du navigateur.
Si vous sélectionnez \menu{senders (n)} (voir \figref{SendersOfDrawOn}), un 
menu apparaîtra avec \ct{drawOn:} comme premier élement de
la pile,  suivi de tous les messages que 
\ct{drawOn:} envoie (voir \figref{SendersOfDrawOn2}). %CHANGE
Sélectionner un message dans ce menu ouvrira un navigateur avec la
liste de toutes les méthodes dans l'image qui envoie le message
choisi (voir \figref{CanvasDraw}).

% \caption{Un navigateur de classes ouvert sur la classe \ct{ScaleMorph}. Notez la barre horizontale de boutons en son centre; nous appuyons ici sur le bouton \button{senders}.
%\label{fig:SendersOfCheckEvent}}

%\begin{figure}[htb]
%\begin{minipage}[b]{0.74\textwidth}
%\centerline {\includegraphics[width=\textwidth]{SendersOfDrawOn}}
%\caption{The \menu{senders (n)} menu item.\figlabel{SendersOfDrawOn}}
%\end{minipage}
%\hfill
%\begin{minipage}[b]{0.24\textwidth}
%\centerline {\includegraphics[width=\textwidth]{SendersOfDrawOn2}}
%\caption{Choose senders of which message.\figlabel{SendersOfDrawOn2}}
%\end{minipage}
%\end{figure}

\begin{figure}[htb]
\centerline {\includegraphics[width=\textwidth]{SendersOfDrawOn}}
\caption{L'élement de menu \menu{senders (n)}.\figlabel{SendersOfDrawOn}}
 \end{figure}

\begin{figure}[htb]
\centerline {\includegraphics[width=0.4\textwidth]{SendersOfDrawOn2}}
\caption{Choisir un message dans la liste pour avoir ses \emph{senders}.\figlabel{SendersOfDrawOn2}}
\end{figure}

%\index{Browser!bouton!implementors}
%Le bouton \button{implementors} fonctionne de la même manière mais,
% au lieu de renvoyer une liste de \senders d'un message (ou méthodes-envoyeuses), il sort toutes les
% classes qui implémentent une méthode avec le même sélecteur.
% Pour le voir, sélectionnez \lct{drawOn:} dans le panneau des méthodes
% puis affichez le navigateur ``implementors of drawOn:'', 
% soit en utilisant le bouton \button{implementors}, soit via le 
% \ind{bouton jaune}, soit encore en tapant simplement \short{m} (pour {i\textbf{m}ple\textbf{m}entors}) avec la méthode \menu{drawOn:} sélectionnée dans le panneau des méthodes. 
% Vous devriez avoir une fenêtre à ascenseur montrant une liste des 96
% classes implémentant une méthode \ct{drawOn:}.
% Il n'y a rien de surprenant à ce qu'autant de classes implémentent cette
% méthode: \ct{drawOn:} est le message compris par tout objet apte à se
% dessiner lui-même sur l'écran.
%% REVOIR ? il y a du 'senders' implementators' 'emettrices'...
%\arevoir{Essayez de naviguer dans la liste des \senders du message \ct{drawOn:}; nous nous trouvons face à 63 méthodes émettrices. Vous pouvez aussi ouvrir
%un navigateur d'implementors chaque fois que vous sélectionnez un message
% (en incluant les arguments s'il s'agit d'un message à mots-clés) 
%\arevoir{et que vous appuyez sur \short{m}}.}

Le ``n'' dans \menu{senders (n)} vous informe que le
raccourci-clavier pour trouver les \senders (\ie les méthodes émettrices) d'un message
est \short{n}. Cette commande fonctionne depuis \emph{n'importe
quelle} fenêtre de texte. % CHANGE text window

\dothis{Sélectionnez le texte ``drawOn:'' dans le panneau de
    code et pressez \short{n} pour faire apparaître immédiatement les
     \senders de \ct{drawOn:}.}

\begin{figure}[htbp]
	\begin{center}
   \ifluluelse
		{\includegraphics[width=\textwidth]{CanvasDraw}}
		{\includegraphics[width=0.7\textwidth]{CanvasDraw}}
	\end{center}
	\caption{Le navigateur Senders Browser montrant que la méthode \ct{Canvas>>>>draw} envoie le message \ct{drawOn:} à son argument.	\figlabel{CanvasDraw}}
\end{figure}

%% REVOIR martial: c'est totalement faux dans l'image Core actuelle
%% rene : c'est vrai avec l'image https://gforge.inria.fr/frs/download.php/27023/PBE-1.0.zip
% \arevoir{Si vous regardez bien les \senders de
%  \ct{drawOn:} dans \ct{AtomMorph>>>drawOn:}, vous verrez qu'il s'agit
%  d'un \subpvind{super}{envoi}{} à \super. Nous savons ainsi que la
%  méthode qui sera exécutée est dans la superclasse de \ct{AtomMorph}.
% Quelle est cette classe? \Actclickz sur \menu{browse~\go~hierarchy~implementors} et vous 
% verrez qu'il s'agit de \ct{EllipseMorph}.}
% \index{Browser!bouton!hierarchy} % CHANGE Browser!bouton!hierarchy
% \seeindex{navigateur de classes!hiérarchie}{Browser, hierarchy} % CHANGE

Si vous regardez bien les \senders de \ct{drawOn:} dans \mbox{\ct{AtomMorph>>>drawOn:},} vous verrez que la méthode implémente un \subpvind{super}{envoi}{} à \super. Nous savons ainsi que la méthode qui sera exécutée est dans la superclasse de \ct{AtomMorph}. Quelle est cette classe? \Actclickz sur \menu{browse~\go~hierarchy~implementors} et vous verrez qu'il s'agit de \ct{EllipseMorph}.
\index{Browser!bouton!hierarchy} % CHANGE Browser!bouton!hierarchy
\seeindex{navigateur de classes!hiérarchie}{Browser, hierarchy} % CHANGE

Maintenant observons le sixième \emph{sender} de la liste,
\ct{Canvas>>>draw}, comme le montre \figref{CanvasDraw}.
Vous pouvez voir que cette méthode envoie \ct{drawOn:} à n'importe quel objet 
passé en argument; ce peut être une instance de n'importe quelle classe.
L'analyse du flux de données peut nous aider à mettre la main sur la classe du 
receveur de certains messages, mais de manière générale, il n'a pas de moyen simple 
pour que le navigateur sache quelle méthode sera exécutée à l'envoi d'un message.
%Dataflow analysis can help figure out the class of the receiver of some messages, but in general, there is no simple way for the browser to know which message-sends might cause which methods to be executed.

C'est pourquoi, le navigateur de ``\senders'' (\ie le Browser des
méthodes émettrices) nous montre exactement ce que son nom
suggère: tous  les \senders d'un message relatifs à un sélecteur donné. 
%CHANGE martial: la phrase trop lourde 
% rene: tous les "senders" d'un message relatifs à un selecteur donné
Ce navigateur devient grandement indispensable quand vous avez besoin
de comprendre le \emph{rôle} d'une méthode: il vous permet de
naviguer rapidement à travers les exemples d'usage. % CHANGE
%The \button{senders} button is nevertheless extremely useful when you need to understand how you can \emph{use} a method: it lets you navigate quickly through example uses.  
Puisque toutes les méthodes avec un même sélecteur devraient être utilisées de la même manière, toutes les utilisations d'un message donné devraient être semblables.
%\index{Browser!bouton!senders}
\index{Browser!senders} % REVOIR navigateur de classes!senders >
                        % Browser!senders
\seeindex{Senders Browser}{Browser!senders}
\seeindex{navigateur de classes!méthodes émettrices}{Browser, senders}
% CHANGE TEMP
\seeindex{navigateur de classes!senders}{Browser, senders} % CHANGE

\index{Browser!implementors}
\seeindex{Implementors Browser}{Browser!implementors}
\seeindex{navigateur de classes!classes contenantes}{Browser, implementors} % CHANGE TEMP
\seeindex{navigateur de classes!implementors}{Browser, implementors} % CHANGE

L'Implementors Browser fonctionne de même mais, au lieu de
lister les \senders d'un message, il affiche toutes les classes
contenantes ou \implementors, \ie les classes qui implémentent une
méthode avec le même selecteur
% martial: ajout de 'que celui sélectionné' plus clair!?
que celui sélectionné.
Sélectionnez, par exemple, \lct{drawOn:} dans le panneau des méthodes
et sélectionne \menu{browse \go implementors (m)} (ou selectionnez le
texte ``drawOn:'' dans la zone inférieure du code et pressez 
\short{m}).
Vous devriez voir une fenêtre listant des méthodes montrant ainsi la
liste déroulante des \arevoir{\drawOnImplNumber} % 90 dans l'original
% rene : ni 90 ni 63 mais 77
% rene : il me semble préférable de s'éloigner un peu de l'original et
%        ne plus préciser le nombre exact
% martial: j'ai mis une variable globale (ligne 761 de common.tex)
classes qui implémentent une méthode \ct{drawOn:}.
Ceci ne devrait pas être si surprenant qu'il y ait tant de classes
implémentant cette méthode: \ct{drawOn:} est le message qui est
compris par chaque objet capable de se représenter graphiquement à
l'écran.

\subsubsection{Les versions d'une méthode}
\seclabel{versions}

Quand vous sauvegardez une nouvelle \subind{méthode}{version} d'une méthode,
l'ancienne version n'est pas perdue. \pharo garde toutes les versions passées et vous permet 
de comparer les différentes versions entre elles et de revenir (en anglais, ``revert'') à une ancienne version.
\begin{figure}[btp]
   \centering
	   {\includegraphics[width=\textwidth]{Versions} }
   \caption{\arelire{Le \ind{Versions Browser} montre deux versions de la
     méthode \ct{TheWorldMenul>>>buildWorldMenu:}.}}
   \figlabel{buildWorldMenuVersions}
\end{figure} % CHANGE

%\index{Browser!bouton!versions}
% Rene n'arrive pas à voir les deux versions de la methode buildWorldMenu
\index{Browser!versions}
\seeindex{Versions Browser}{Browser, versions} % CHANGE
\arelire{L'élement de menu \menu{browse \go versions (v)}  donne accès aux modifications successives
effectuées sur la méthode sélectionnée.
Dans \figref{buildWorldMenuVersions} nous pouvons voir deux versions de la
méthode \ct{buildWorldMenu:}.}


\index{System Browser!bouton!versions}

Le panneau supérieur affiche une ligne pour chaque version de la méthode
incluant les initiales du programmeur qui l'a écrite, la date et l'heure de
sauvegarde, les noms de la classe et de la méthode et le protocole dans 
lequel elle est définie.
La version courante (active) est au sommet de la liste; quelle que soit la version
sélectionnée affichée dans le panneau inférieur.
%Si le bouton ou \emph{checkbox} \menu{diffs} est sélectionné, 
%comme c'est le cas dans \figref{mouseUpVersions}, 
%les différences entre la version sélectionnée et celle qui la précède
%immédiatemment sont affichées.
Les boutons offrent aussi l'affichage des différences entre la méthode sélectionnée et la version courante et la possibilité de revenir à la version choisie.  
%Le bouton \menu{prettyDiffs} est utile s'il y a eu changement dans la mise en
%pages: il affiche en mode \emph{pretty-print} (affichage élégant) 
%à la fois les versions antérieures et choisies de façon à
%ce que les différences liées au formatage ne soient pas prises en compte.

Le \ind{Versions Browser} existe pour que vous ne vous inquiétez jamais
de la préservation de code que vous pensiez ne plus avoir besoin: effacez-le simplement. 
Si vous vous rendez compte que vous en avez \emph{vraiment} besoin, 
vous pouvez toujours revenir à l'ancienne version ou copier le morceau de 
code utile de la version antérieure pour le coller dans une autre méthode.

Ayez pour habitude d'utiliser les versions; ``commenter'' le code qui n'est 
plus utile n'est pas une bonne pratique car ça rend le code courant plus difficile à lire. 
Les Smalltalkiens~\footnote{En anglais, nous les appelons
 \emph{Smalltalkers}.} accordent une extrême importance à la lisibilité du code.

\hint{Qu'en est-il du cas où vous décidez de revenir à une méthode que vous
avez entièrement effacée?  
Vous pouvez trouver l'effacement dans un \changeset
dans lequel vous pouvez demander à visiter les versions 
en \actclickant{}. % CHANGE
Le navigateur de \changeset est décrit dans
\secref{env:changeSet}}

\subsubsection{Les surcharges de méthodes}
\seclabel{overriding}

Le Inheritance Browser est un navigateur spécialisé
affichant toutes les méthodes surchargées par la méthode affichée.
Pour le voir à l'action, sélectionnez la méthode
\cmind{ImageMorph}{drawOn:} dans le Browser.
Remarquez les icônes triangulaires juxtant le nom des méthodes (voir \figref{OBinheritanceBrowser}).
Le triangle pointant vers le haut vous indique que 
\ct{ImageMorph>>>drawOn:} surcharge une méthode héritée
 (\ie \ct{Morph>>>drawOn:}) et triangle pointant vers le bas vous
 indique que cette méthode est surchargée dans ses sous-classes (vous
 pouvez aussi \click sur les icônes pour naviguer vers ces méthodes).
Sélectionnez maintenant \menu{browse \go inheritance}.
L'Inheritance Browser vous montre la hiérarchie de méthodes
surchargées (voir \figref{OBinheritanceBrowser}).

\begin{figure}[tbp]
	\begin{center}
   \ifluluelse
		{\includegraphics[width=\textwidth]{OBInheritanceOverriding}}
		{\includegraphics[width=0.7\textwidth]{OBInheritanceOverriding}}
	\end{center}
	\caption{\ct{ImageMorph>>>drawOn:} et les méthodes qu'il
      surcharge. \arelire{Les méthodes apparentées ou \emph{siblings} des
        méthodes sélectionnées sont visibles dans les listes déroulantes .}} % CHANGE
	\figlabel{OBinheritanceBrowser}
\end{figure}

\subsubsection{La vue hiérarchique}
\seclabel{hierarchy}

Par défaut, le navigateur présente une liste de paquetages dans son
panneau supérieur gauche.
Cependant, il est possible de changer le contenu de ce panneau pour
avoir une vue hiérarchique des classes. Pour cela, sélectionner tout
simplement une classe de votre choix, disons \ct{ImageMorph} et \click
sur le bouton \button{hier.}.
Vous verrez alors dans le panneau le plus à gauche une hiérarchie de
classes affichant toutes les super-classes et sous-classes de la
classe sélectionnée.
Le second panneau liste les paquetages implémentant les
méthodes de la classe sélectionnée. % CHANGE à vérifier
%The \button{hierarchy} button opens a \ind{hierarchy browser} on the current class; this
%browser can also be opened by using the \menu{browse hierarchy} menu item in the class pane.
%The hierarchy browser is similar to the browser, but instead of displaying the categories and the classes in each category, it shows a single list of classes, indented to represent inheritance.
%The category of the selected class is displayed in the small annotation pane at the top of the browser.
Sur \figref{OBinheritanceBrowser}, la vue hiérarchique dans le navigateur 
% rene : s /hierarchyBrowser/OBinheritanceBrowser/
montre que \clsind{ImageMorph} est la super-classe directe de
\mbox{\clsind{Morph}.}
\index{Browser!bouton!hierarchy} % REVOIR

\begin{figure}[btp]
	\begin{center}
   \ifluluelse
		{\includegraphics[width=\textwidth]{hierarchyBrowser}}
		{\includegraphics[width=0.7\textwidth]{hierarchyBrowser}}
	\end{center}
	\caption{Une vue hiérarchique de \ct{ImageMorph}.}
	\figlabel{OBinheritanceBrowser}
\end{figure}

\subsubsection{Trouver les références aux variables}
\seclabel{variables}

\index{Browser!variables}
En \actclickant sur une classe dans le panneau de classes du
navigateur, puis en sélectionnant
\menu{browse \go chase variables}~\footnote{\emph{Chase} signifie
  ``poursuivre'' en anglais.}, vous pouvez trouver 
où une certaine variable\,---\,d'instance ou de classe\,---\,est
utilisée.
Vous naviguez au travers des méthodes d'accès de toutes les variables
d'instance ou de classe via ce \emph{navigateur de poursuite} et, en
retour, visitez les méthodes qui envoyent ces accesseurs, et
ainsi de suite (voir \figref{chasingBrowser}).

\begin{figure}[btp]
	\begin{center}
	\ifluluelse
		{\includegraphics[width=\textwidth]{chasingBrowser}}
		{\includegraphics[width=0.7\textwidth]{chasingBrowser}}
	\end{center}
	\caption{Un navigateur de poursuite ouvert sur \ct{Morph}.}
	\figlabel{chasingBrowser}
\end{figure}

% Le menu permet aussi d'afficher le jeu 
% %subset 
% des références aux variables d'instance qui affecte la variable choisie
% par \menu{inst var defs}.
% Une fois que vous avez cliqué sur le bouton ou que vous avez choisi une
% des propositions de menu, un menu flottant s'affichera, vous invitant ainsi
% à sélectionner une variable parmi toutes les variables définies et
% héritées dans la classe courante.
% La liste suit l'ordre d'héritage; il peut d'ailleurs être utile d'afficher
% cette liste à chaque fois vous avez besoin de vous remémorer le nom d'une
% variable d'instance. Si vous cliquer en dehors de la liste, cette dernière
% disparaîtra sans avoir affiché le navigateur de variable.


\subsubsection{Le panneau de code}
% rene propose Le code source
\seclabel{sources}

\index{Browser!panneau de code}
\seeindex{Browser!panneau source}{Browser, panneau de code} % CHANGE
\seeindex{Browser!view}{Browser, panneau de code} % REVOIR
L'option de menu \menu{various \go view \ldots} 
% ajout - martial
disponible en \actclickant{} dans le panneau des méthodes
affiche le menu ``comment faut-il l'afficher''
qui vous permet de choisir comment le navigateur va afficher
% ajout - martial
la méthode sélectionnée dans le panneau inférieur \ie le panneau de code
(ou \emph{panneau source}).
%Le bouton \button{source} affiche un menu que nous pourrions appeler
%%``what to show'' menu, 
%``ce qui est à voir''; il nous permet de choisir ce que le navigateur
%affiche dans le panneau inférieur ou panneau source.
Parmi les propositions, nous avons l'affichage du code \menu{source}, 
du code source en mode \menu{prettyPrint} (affichage élégant), 
du code compilé ou \menu{byteCode}, ou encore du code source
décompilé depuis le \emph{bytecode} via \menu{decompile}.
%Le label du bouton change pour afficher le mode choisi. Il y a d'autres
%options; si vous promenez la souris sur ces options, vous verrez
%apparaître un ballon d'aide (ou \emph{help balloon}). Essayez-en
%quelques-uns. 
\index{méthode!pretty-print}
\index{méthode!decompile}
\index{méthode!byte code}

Remarquez que le choix de \menu{prettyPrint} dans ce menu n'est
\emph{absolument pas} le même que le travail en mode \emph{pretty-print} d'une méthode
avant sa sauvegarde\footnote{\menu{pretty print (r)} est la première option de menu 
dans le panneau des méthodes ou celui à mi-hauteur dans le menu du panneau de code.}.
Le menu contrôle seulement l'affichage du navigateur et n'a aucun effet sur
le code enregistré dans le système.
Vous pouvez le vérifier en ouvrant deux navigateurs et en sélectionnant
\menu{prettyPrint} pour l'un et \menu{source} pour l'autre.
Pointer les deux navigateurs sur la même méthode et en choisissant
\menu{byteCode} dans l'un et \menu{decompile} dans l'autre est vraiment
une bonne manière d'en apprendre plus sur le jeu d'instructions codées
(dit aussi \emph{byte-codées}) de la machine virtuelle \pharo. % CHANGE

\subsubsection{La refactorisation}


Les menus contextuels proposent un grand nombre de refactorisations (ou
\emph{refactoring}) classiques. \Actclickz sur l'un des quatre panneaux
supérieurs pour voir les opérations de refactorisation actuellement
disponibles (voir \figref{refactoring}).
À l'origine, cette fonction était disponible uniquement
par un navigateur spécifique nommé Refactoring Browser, mais
elle peut désormais être accessible depuis n'importe quel
navigateur. % CHANGE

\begin{figure}[btp]
	\begin{center}
	\ifluluelse
		{\includegraphics[width=\textwidth]{refactoring}}
		{\includegraphics[width=0.7\textwidth]{refactoring}}
	\end{center}
	\caption{Le refactorisation à la souris.}
	\figlabel{refactoring}
\end{figure} % REVOIR
%---------------------------------------------------------
\subsection{Les menus du navigateur}

De nombreuses fonctions complémentaires sont disponibles 
en \actclickant{} dans les panneaux du Browser.
Même si les options de menus portent le même nom,
leur \emph{signification} dépend du contexte.
Par exemple, le panneau des paquetages, le
panneau des classes, le panneau des protocoles et enfin, celui des méthodes
ont tous \menu{file out} dans leurs menus respectifs. Cependant, chaque 
\menu{file out} fait une chose différente: dans le panneau des paquetages,
il enregistre entièrement dans un fichier le paquetage
sélectionné; dans le celui des classes, des protocoles ou des
méthodes, il exporte respectivement la classe entière, le protocole
entier ou la méthode affichée. % CHANGE REVOIR
Bien qu'apparemment évident, ce peut être une source de confusion pour
les débutants.
\index{fichier!filing in}
\index{fichier!importation}
\index{fichier!filing out}
\index{fichier!exportation}

L'option de menu probablement la plus utile est \menu{find class\ldots (f)} 
dans le panneau des paquetages. 
Elle permet de trouver une classe.
% ajout
Bien que les catégories soient utiles pour arranger le code que nous 
sommes en train de développer, la plupart d'entre nous ne connaissent pas
la catégorisation de tout le système, et c'est beaucoup plus rapide
en tapant \short{f} suivi par les premiers caractères du nom d'une 
classe que de deviner dans quel paquetage elle peut bien être.
\menu{recent classes\ldots } vous aide aussi à retrouver rapidement
une classe parmi celles que vous avez visitées récemment, même si vous
avez oublié son nom.
% FAUX dans la VO
% rene : tout semble fonctionne (mais il faut supprimer (r) dans \menu{recent classes\ldots (r)}  
\index{classe!recherche}
\index{classe!récente}

Vous pouvez aussi rechercher une classe ou une méthode en particulier en
saisissant son nom dans la boîte de requête située dans la partie 
supérieure gauche de votre navigateur. Quand vous tapez
sur la touche ``entrée'', une requête sera envoyée au système et son
résultat sera affiché. % REVOIR CHANGE
Notez qu'en préfixant votre requête avec \ct{#}, vous pouvez chercher
toutes les références à une classe ou tous les \senders d'un
message.
%Dans le panneau de classes, le menu propose \menu{find method} (pour
%``trouver une méthode'') et \menu{find method wildcard\ldots} qui
%s'avèrent utiles si vous souhaitez naviguer dans une méthode
%particulière.
Cependant, si vous recherchez une méthode dans une classe sélectionnée, il est
souvent plus efficace de naviguer dans le protocole \prot{-{}-all-{}-} 
(qui d'ailleurs est le choix par défaut), placer la souris dans le
panneau des méthodes et taper la première lettre du nom de la méthode
que vous recherchez. % CHANGE
Ceci va faire glisser l'ascenseur du panneau jusqu'à ce que la méthode souhaitée soit visible.
\seeindex{méthode!trouver}{méthode, recherche}
\index{méthode!recherche}
\protindex{all}

\dothis{Essayez les deux techniques de navigation pour \cmind{OrderedCollection}{removeAt:}}

Il y a beaucoup d'autres options dans les menus.  Passer quelques
minutes à tester les possibilités du navigateur est véritablement payant.  

\dothis{Comparez le résultat de \menu{Browse Protocol}, \menu{Browse Hierarchy},  et \menu{Show Hierarchy} dans le menu contextuel du panneau de classes.}

%---------------------------------------------------------
\subsection{Naviguer par programme} %{Browsing programmatically}

La classe \glbind{SystemNavigation} offre de nombreuses méthodes utiles
pour naviguer dans le système.
Beaucoup de fonctionnalités offertes par le navigateur classique
sont programmées par
\ct{SystemNavigation}.
\index{navigation par programme}

\dothis{
Ouvrez un espace de travail Workspace et exécutez le code suivant pour naviguer dans la liste des \senders du message 
\ct{drawOn:} en utilisant \menu{do it}:}

\begin{code}{}
SystemNavigation default browseAllCallsOn: #drawOn: .
\end{code}
Pour restreindre le champ de la recherche à une classe spécifique:
\begin{code}{}
SystemNavigation default browseAllCallsOn: #drawOn: from: ImageMorph .
\end{code}
Les outils de développement sont complètement accessibles depuis un
programme car \emph{ceux-ci sont aussi des objets}. Vous pouvez dès lors
développer vos propres outils ou adapter ceux qui existent déjà
selon vos besoins.

L'équivalent programmatique de l'option de menu
\menu{implementors} est: % CHANGE
\begin{code}{}
SystemNavigation default browseAllImplementorsOf: #drawOn: .
\end{code}

Pour en apprendre plus sur ce qui est disponible, explorez la classe
\ct{SystemNavigation} avec le navigateur.

Des exemples supplémentaires peuvent être trouvés dans
 \charef{faq}.

%=========================================================
\section{Monticello}

Nous vous avons donné un aperçu de \ind{Monticello}, l'outil de gestion
de paquetages de \pharo dans \secref{Monticello}.  
Cependant Monticello a beaucoup plus de fonctions que celles dont nous allons
discuter ici.
Comme Monticello gère des \emph{paquetages} dits \emph{packages}, nous allons expliquer ce qu'est
un \ind{paquetage} avant d'aborder Monticello proprement dit.

%---------------------------------------------------------
\subsection{Les paquetages: une catégorisation déclarative du code de Pharo}
\seclabel{packages}

Dans \secref{categoriesPackages}, nous avons pointé le fait
que les paquetages sont plus ou moins équivalents aux catégories. 
Nous allons désormais voir la relation qui existe entre eux.
Le système du paquetage est une façon simple et légère
d'organiser le code source de Smalltalk; il exploite une simple
convention de nommage pour les catégories et les protocoles.

Prenons l'exemple suivant en guise d'explication.
Supposons que nous soyons en train de développer une librairie pour
nous faciliter l'utilisation d'une base de données relationnelles depuis
\pharo. Vous avez décidé d'appeler votre librairie (ou \emph{framework})
\ct{PharoLink} et vous avez créé une série de catégories
contenant toutes les classes que vous avez écrites, par exemple la catégorie
\ct{'PharoLink-Connections'} contient les classes
\ct{OracleConnection MySQLConnection PostgresConnection} et la catégorie
\ct{'PharoLink-Model'} contient les classes
\ct{DBTable DBRow DBQuery} % CHANGE
et ainsi de suite. Cependant, tout le code ne résidera pas dans ces classes.
Par exemple, vous pouvez aussi avoir une série de méthodes pour 
convertir des objets dans un format sympathique pour notre format
SQL~\footnote{Nous dirions que ce format est SQL-friendly.}:

\begin{code}{}
Object>>>asSQL
String>>>asSQL
Date>>>asSQL
\end{code}

\noindent
Ces méthodes appartiennent au même paquetage que les classes
dans les catégories \ct{PharoLink-Connections} et \ct{PharoLink-Model}.
Mais la classe \ct{Object} n'appartient clairement pas à notre paquetage!
Vous avez donc besoin de trouver un moyen pour associer certaines
\emph{méthodes} à un paquetage même si le reste de la classe est dans
un autre. 
\index{paquetage!extension}
\seeindex{extension de paquetage}{paquetage, extension}
\seeindex{extension de package}{paquetage, extension}

Pour ce faire, nous plaçons ces méthodes (de \ct{Object}, \ct{String}, \ct{Date} \etc) dans un protocole nommé \prot{*PharoLink} (remarquez l'astérisque en début de nom). L'association des
catégories en \scat{PharoLink-\ldots} et des protocoles \prot{*PharoLink} 
forme un paquetage nommé \ct{PharoLink}.
Précisement, les règles de formation d'un paquetage s'énoncent comme suit.

Un paquetage appelé \ct{Foo} contient:

\begin{enumerate}		\seclabel{packageRules}
	\item{} toutes les \emph{définitions de classe} des classes présentes dans
la catégorie \scat{Foo} ou toutes catégories avec un nom commençant par
\scat{Foo-};
	\item{} \label{env:extensions} toutes les \emph{méthodes}
dans \emph{n'importe quelle classe} dont le protocole se nomme
\prot{*Foo} ou \prot{*foo}~\footnote{Durant la comparaison
de ces noms, la casse des lettres est parfaitement ignorée.}, et;
%%ou n'importe quel nom commençant par \prot{*foo-} et;
\item{} toutes les \emph{méthodes} dans les classes présentes dans
\scat{Foo} ou toutes catégories avec un nom commençant par \scat{Foo-}, 
\emph{exception} faite des méthodes dont le nom des protocoles débute par 
\prot{*}.
\end{enumerate}

\noindent
Une conséquence de ces règles est que chaque définition de classe et chaque 
méthode appartiennent exactement à un paquetage. 
L'\emph{exception} de la dernière règle est justifiée parce que
ces méthodes doivent appartenir à d'autres paquetages.
La raison pour laquelle la casse
%%~\footnote{La hauteur minuscule ou majuscule d'une lettre.} 
est ignorée dans la règle \ref{env:extensions} 
% Rene rewording
% est que, par convention, les noms de protocole sont tous en
% minuscules (et peuvent inclure des  
% espaces), alors que les noms de catégorie utilise une écriture
% en chameau c'est-à-dire les 
% mots composants ces noms sont en capitales et forment les noms sans espaces comme
% dans CamelCase (nom anglais de cette technique de formatage de nom).
est que, conventionnellement les noms de protocoles sont typiquement
(mais pas nécessairement) en minuscules (et peuvent inclure des espaces); alors que 
les noms de catégories utilisent un format d'écriture dit \emph{casse de chameau} comme par 
exemple AlanKay, LargePositiveInteger ou CamelCase (d'ailleurs CamelCase est 
le nom anglais de ce type de format de noms).
\index{écriture en chameau}
\seeindex{CamelCase}{écriture en chameau}
% martial: au-dessus, dans la correction de rene, je n'ai mis
% LargePositiveInteger a la place de SmallTalk parce que ce n'est pas
% (plus) la convention d'ecriture

La classe \ct{PackageInfo} implémente ces règles et vous pouvez mieux les 
appréhender en expérimentant cette classe.

\dothis{Évaluez l'expression suivante dans un espace de travail:}
%\seeindex{refactoring}{refactorisation} REVOIR


\begin{code}{}
mc := PackageInfo named: 'Monticello'
\end{code}

Il est possible maintenant de faire une introspection de ce paquetage.
Par exemple, imprimer via \menu{print it} le code \ct{mc classes} dans l'espace de travail nous retourne la longue liste des classes qui font le paquetage Monticello. L'expression \ct{mc coreMethods}
nous renvoie une liste de \mbox{\ct{MethodReference}{s}} ou références de méthodes
pour toutes les méthodes de ces classes.
La requête \ct{mc extensionMethods} est peut-être une des plus
intéressantes: elle retourne la liste de toutes les méthodes contenues
dans le paquetage \ct{Monticello} qui ne sont pas dans une classe de
\lct{Monticello}. % CHANGE

\arevoir{Les paquetages sont des ajouts à \pharo relativement récents
mais, puisque les conventions de nommage de paquetage sont basées
sur celles déjà existantes, il est possible d'utiliser
\ct{PackageInfo} pour analyser du code plus ancien qui n'a pas 
été explicitement adapté pour pouvoir y répondre.} % REVOIR -
% martial : FAUX? DANS PBE
% rene : nous pourrions remplacer \pharo par \st

\dothis{Imprimer le code \ct{(PackageInfo named: 'Collections') externalSubclasses}; 
cette expression répond une liste de toutes les sous-classes de \ct{Collection}
qui ne sont \emph{pas} dans le paquetage \ct{Collections}.}

%---------------------------------------------------------

\subsection{Les fondamentaux de Monticello}
% les fondamentaux, les bases

\ind{Monticello} est nommé ainsi d'après la villégiature 
de Thomas Jefferson, troisième président des États-Unis d'Amérique
et auteur de la statue pour les libertés religieuses (Religious Freedom) en
Virginie. Le nom signifie ``petite montagne'' en italien, en ainsi, il est
toujours prononcé avec un ``c'' italien, \ie avec le son \emph{tch} comme
dans ``quetsche'':
Monn-ti-tchel-lo%
%Mont-y'-che-llo.
~\footnote{Note du traducteur: c'est aussi une commune de Haute-Corse.}.

\begin{figure}[btp]
	\begin{center}
	\ifluluelse
		{\includegraphics[width=\textwidth]{freshMonticello}}
		{\includegraphics[width=0.7\textwidth]{freshMonticello}}
	\end{center}
	\caption{Le navigateur Monticello.}
	\figlabel{freshMonticello}
\end{figure}

Quand vous ouvrez le navigateur Monticello, vous voyez deux panneaux
de listes et une ligne de boutons, comme sur \figref{freshMonticello}.

Le panneau de gauche liste tous les paquetages qui ont été chargés
dans l'image actuelle; la version courante du paquetage est
présentée entre parenthèses à la suite de son nom.

Celle de droite liste tous les dépôts (ou \emph{repository}) de code
source que Monticello connaît généralement pour les avoir utilisés
pour charger le code. Si vous sélectionnez un paquetage dans le panneau de 
gauche, celui de droite est filtré pour ne montrer que les dépôts
qui contiennent des versions du paquetage choisi.

Un des dépôts est un répertoire nommé \emph{package-cache} qui
est un sous-répertoire du répertoire courant où vous avez
votre image.
Quand vous chargez du code depuis un dépôt distant (ou remote repository)
ou quand vous écrivez du code, une copie est effectuée aussi dans ce
répertoire de cache. Il peut être utile si le réseau n'est pas 
disponible et que vous ayez besoin d'accéder à un paquetage. De plus,
si vous avez directement reçu un fichier Monticello (.mcz), par exemple, 
en pièce jointe dans un courriel, la façon la plus convenable d'y accéder
depuis \pharo est de le placer dans le répertoire package-cache.
\index{package!cache}

Pour ajouter un nouveau dépôt à la liste, cliquez sur le bouton 
\button{+Repository} et choisissez le type de dépôt dans le menu
flottant. Disons que nous voulons ajouter un dépôt HTTP.

\dothis{Ouvrez Monticello, cliquez sur \button{+Repository} et choisissez \menu{HTTP}.
Éditez la zone de texte à lire:}
%\ab{How does one continue the $\backslash$dothis to include the code?}
%\on{Don't.  Just close the \dothis{} and follow with the code.}
\needlines{4}
\begin{code}{}
MCHttpRepository
	location: 'http://squeaksource.com/PharoByExample'
	user: ''
	password: ''
\end{code}

\begin{figure}[btp]
	\begin{center}
	\ifluluelse
		{\includegraphics[width=0.7\textwidth]{SqueakSource-PBE}}
		{\includegraphics[width=0.7\textwidth]{SqueakSource-PBE}}
	\end{center}
	\caption{Un navigateur de dépôts ou Repository Browser.}
	\figlabel{SqueakSource:PBE}
\end{figure}
\noindent

Ensuite cliquez sur \button{Open} pour ouvrir un navigateur de dépôts ou
Repository Browser. Vous devriez voir quelque chose comme
 \figref{SqueakSource:PBE}.  
Sur la gauche, nous voyons une liste de tous les paquetages présents dans le
dépôt; si vous en sélectionnez un, le panneau de droite affichera
toutes les versions du paquetage choisi dans ce dépôt.

Si vous choisissez une des versions, vous pourrez naviguer dans son contenu (sans le charger dans votre image) via le bouton \button{Browse}, le charger
par le bouton \button{Load} ou encore inspecter les modifications
via \button{Changes} qui seront faites à votre image en chargeant la version
sélectionnée. Vous pouvez aussi créer une copie grâce au bouton \button{Copy}
d'une version d'un paquetage que vous pourriez ensuite écrire dans un
autre dépôt.

Comme vous pouvez le voir, les noms des versions contiennent le nom du paquetage, les initiales de l'auteur de la version et un numéro de version.
Le nom d'une version est aussi le nom du fichier dans le dépôt. 
Ne changez jamais ces noms; le déroulement correct des opérations
effectuées dans Monticello dépend d'eux!
Les fichiers de version de Monticello sont simplement des archives compressées
 et, si vous êtes curieux vous pouvez les décompresser avec un outil 
de décompression ou \emph{dézippeur}, mais la meilleure façon 
d'explorer leur contenu consiste à faire appel à Monticello lui-même.

Pour créer un paquetage avec Monticello, vous n'avez que deux choses à faire:
écrire du code et le mentionner à Monticello.

\dothis{Créez une paquetage appelé \scat{PBE-Monticello}, et mettez-y
une paire de classes, comme vu sur \figref{MCnewcategory}. 
Créez une méthode dans une classe existante, par exemple
\ct{Object}, et mettez-la dans le 
même paquetage que vos classes en utilisant les règles de la page~\pageref{sec:packageRules}\,---\,voir \figref{MCnewmethod}.}

\begin{figure}[btp]
	\begin{center}
	\ifluluelse
		{\includegraphics[width=\textwidth]{MCnewcategory}}
		{\includegraphics[width=0.7\textwidth]{MCnewcategory}}
	\end{center}
	\caption{Deux classes dans le paquetage ``PBE''.}
	\figlabel{MCnewcategory}
\end{figure}

\begin{figure}[btp]
	\begin{center}
	\ifluluelse
		{\includegraphics[width=\textwidth]{MCnewmethod}}
		{\includegraphics[width=0.7\textwidth]{MCnewmethod}}
	\end{center}
	\caption{Une extension de méthode qui sera aussi incluse dans le paquetage ``PBE''.}
	\figlabel{MCnewmethod}
\end{figure}

Pour mentionner à Monticello l'existence de votre paquetage, 
cliquez sur le bouton \button{+Package} et tapez le nom du paquetage,
dans notre cas ``PBE''.
Monticello ajoutera \ct{PBE} à sa liste de paquetages;
l'entrée du paquetage sera marquée avec une astérisque pour
montrer que la version présente dans votre image n'a pas
été encore écrite dans le dépôt.
Remarquez que vous devriez avoir maintenant deux paquetages;
un nommé \ct{PBE} et un autre nommé \ct{PBE-Monticello}. C'est
normal puisque \ct{PBE} contiendra \ct{PBE-Monticello} ainsi que
tout autre paquetage dont le nom commence par \ct{PBE-}. % CHANGE

Initialement, le seul dépôt associé à ce paquetage sera votre
\emph{package cache} comme sur \figref{MC+PBE}.
C'est parfait: vous pouvez toujours sauvegarder le code en l'écrivant
dans ce répertoire local de cache.
Maintenant, cliquez sur \button{Save} et vous serez invité à
fournir des informations ou \emph{log message} pour la version de ce 
paquetage, comme le montre \figref{PBE-on}; 
quand vous acceptez le message entré, Monticello sauvegardera votre paquetage
et l'astérisque décorant le nom du paquetage du panneau de gauche
de Monticello disparaîtra avec le changement du numéro de version.

Si vous faites ensuite une modification dans votre paquetage,---\,disons
en ajoutant une méthode à une des classes\,---\,l'astérisque réapparaîtra pour 
signaler que vous avez des changements non-sauvegardés.
Si vous ouvrez un Repository Browser sur le package cache, vous
pouvez choisir une version sauvée et utiliser le bouton \button{Changes}
ou d'autres boutons.
Vous pouvez aussi bien sûr sauvegarder la nouvelle version dans
ce dépôt; une fois que vous rafraîchissez la vue
du dépôt via le bouton \button{Refresh}, vous devriez voir
la même chose que sur
\figref{package-cache-browser}.
\index{paquetage!package cache}
\seeindex{package!cache}{paquetage, package cache}
\seeindex{Monticello!package cache}{paquetage, package cache}

\begin{figure}[tbp]
	\begin{center}
		\includegraphics[width=\textwidth]{MC+PBE}
	\end{center}
	\caption{Le paquetage PBE pas encore sauvegardé dans Monticello.}
	\figlabel{MC+PBE}
\end{figure}

\begin{figure}[tbp]
	\begin{center}
		{\includegraphics[width=0.8\textwidth]{PBE-on}}
	\end{center}
	\caption{Fournir un \emph{log message} pour une version d'un paquetage.}
	\figlabel{PBE-on}
\end{figure}

\begin{figure}[tbp]
	\begin{center}
		{\includegraphics[width=\textwidth]{package-cache-browser}}
	\end{center}
	\caption{Deux versions de notre paquetage sont maintenant le dépôt \emph{package cache}.}
	\figlabel{package-cache-browser}
\end{figure}

Pour sauvegarder notre nouveau paquetage dans un autre
dépôt (autre que package-cache), vous avez besoin de vous 
assurer tout d'abord que Monticello connaît
ce dépôt en l'ajoutant si nécessaire.
Alors vous pouvez utiliser le bouton \button{Copy} dans le 
Repository Browser de package-cache et choisir le dépôt vers lequel
le paquetage doit être copié.
Vous pouvez aussi associer le dépôt désiré avec le paquetage
en sélectionnant \menu{add to package \ldots} 
dans le menu contextuel du répertoire accessible en \actclickant,
comme nous pouvons le voir dans \figref{associateRepository}.
Une fois que le paquetage est lié à un dépôt, vous pouvez sauvegarder
toute nouvelle version en sélectionnant le dépôt et le paquetage
dans le Monticello Browser puis en cliquant sur le bouton 
 \button{Save}.  
Bien entendu, vous devez avoir une permission d'écrire dans un dépôt.
Le dépôt \ct{PharoByExample} sur \emphind{\sqsrc} est
lisible pour tout le monde mais n'est pas ouvert en écriture à tout le monde; 
ainsi, si vous essayez d'y sauvegarder quelque chose, vous aurez un message d'erreur.
Cependant, vous pouvez créer votre propre dépôt sur 
\sqsrc en utilisant l'interface web de \url{http://www.squeaksource.com} et 
en l'utilisant pour sauvegarder votre travail.
Ceci est particulièrement utile pour partager votre code avec vos amis ou
si vous utilisez plusieurs ordinateurs.

\begin{figure}[tbp]
	\begin{center}
		\includegraphics[width=\textwidth]{MCaddToPackage}
	\end{center}
	\caption{Ajouter un dépôt à l'ensemble des dépôts liés
au paquetage.}
	\figlabel{associateRepository}
\end{figure}

Si vous essayez de sauvegarder dans un répertoire dans lequel vous n'avez
pas les droits en écriture, une version sera de toute façon écrite
dans le package-cache.
Donc vous pourrez corriger en éditant les informations du dépôt
(en \actclickant{} dans Monticello Browser) ou
en choisissant un dépôt différent puis, en le copiant
depuis le navigateur ouvert sur package-cache avec le bouton \button{Copy}.

%=========================================================
\section{L'inspecteur  et l'explorateur} % CHANGE
%\section{L'inspecteur Inspector et l'explorateur Explorer}
\seclabel{inspector} % (fold)

Une des caractéristiques de \st qui le rend différent de nombreux 
environnements de programmation est qu'il vous offre une fenêtre 
sur une monde d'objets vivants et non pas sur un monde de codes statiques.
Chacun de ces objets peut être examiné par le programmeur et même
changé\,---\,bien qu'un certain soin doit être apporté lorsqu'il s'agit
de modifier des objets bas niveau qui soutiennent le système.
De toute façon, expérimentez à votre guise, mais sauvegardez votre
image avant!

%---------------------------------------------------------
\subsection{Inspector}

\dothis{Pour illustrer ce que vous pouvez faire avec
  l'\ind{inspecteur} ou \ind{Inspector}, tapez \ct{TimeStamp now} dans
  un espace de travail puis \actclickz et choisissez \menu{inspect it}.} 
(Il n'est pas nécessaire de sélectionner le texte avant d'utiliser le menu;
si aucun texte n'est sélectionné, les opérations du menu fonctionnent
sur la ligne entière.
Vous pouvez aussi entrer \short{i} pour \menu{\textbf{i}nspect it}.)
\clsindex{TimeStamp}
\index{raccourci clavier!inspect it}

\begin{figure}[btp]
	\begin{center}
		\includegraphics[width=\textwidth]{inspectTimeNow1}
	\end{center}
	\caption{Inspecter \ct{TimeStamp now}.}
	\figlabel{inspectTimeNow1}
\end{figure}

Une fenêtre comme celle de \figref{inspectTimeNow1} apparaîtra.
Cet inspecteur peut être vu comme une fenêtre sur les états internes
d'un objet particulier\,---\,dans ce cas, l'instance particulière
de
 \mbox{\ct{TimeStamp}} 
% the \mbox is here because without it, the listings macros puts a space between TimeStamp 
% and the following word, and that space happens to come out at the start of a line.
qui a été créée en évaluant l'expression 
\ct{TimeStamp now}.
La barre de titre de la fenêtre affiche  \arelire{la représentation textuelle} de l'objet
en cours d'inspection.
Si vous sélectionnez la ligne la plus haute dans le panneau supérieur de gauche,
le panneau de droite affichera \arelire{la description textuelle de l'objet,
dite aussi \emph{printString} de l'objet.}
% Si vous sélectionnez \menu{all inst vars} dans le panneau de gauche,
% celui de droite vous présentera une liste de 
% toutes les variables d'instance de l'objet accompagnées de leur
% description printstring.
% Les éléments à suivre dans la liste de ce panneau de gauche
% représentent les variables d'instance; une par une, elles peuvent ainsi
% être facilement examinées et même modifiées dans le panneau
% de droite. % CHANGE

Le panneau de gauche montre une vue arborescente de l'objet
avec \self{} pour racine. Les variables d'instance peuvent être
explorées en \clickant{} sur les triangles à côté de leurs noms.% REVOIR

% The left pane shows a tree view of the object, with \self at the root.
% Instance variables can be explored by expanding the triangles next to their names.

Le panneau horizontal inférieur de l'Inspector est un petit espace de
travail ou Workspace.
C'est utile car dans cette fenêtre, la pseudo-variable \ct{self}
correspond à l'objet que vous avez sélectionné dans le panneau de gauche. 
% CHANGE
% is bound to the object that you have selected in the left pane.
Ainsi, si vous inspectez via \menu{inspect it} l'expression:
\begin{code}{}
self - TimeStamp today
\end{code}
dans ce panneau-espace de travail, le résultat sera un objet 
\clsind{Duration} qui représente l'intervalle temporel entre 
%midnight today and the instant at which you evaluated  \ct{TimeStamp now} and created the \ct{TimeStamp} object that you are inspecting.
la date d'aujourd'hui (en anglais, \ct{today}, le nom du message envoyé) à
minuit et le moment où vous avez évalué \ct{TimeStamp now}
et ainsi créé l'objet \ct{TimeStamp} que vous inspectez.
Vous pouvez aussi essayer d'evaluer \ct{TimeStamp now - self}; 
ce qui vous donnera le temps que vous avez mis à lire la section de ce livre!

En plus de \ct{self}, toutes les variables d'instance de l'objet sont
visibles dans le panneau-espace de travail; dès lors vous pouvez
les utiliser dans des expressions ou même les affecter.
Par exemple, si vous sélectionnez l'objet racine et que vous
évaluez \ct{jdn  := jdn - 1} dans ce panneau,
vous verrez que la valeur de la variable d'instance \ct{jdn} 
changera réellement et que la valeur de \ct{TimeStamp now - self} 
sera augmentée d'un jour.

% ON: Does not work anymore
%Vous pouvez changer les variables d'instance directement en les sélectionnant,
%puis en remplaçant l'ancienne valeur dans le panneau de droite
%par une expression \pharo et en acceptant cette dernière.
%\pharo évaluera l'expression et assignera le résultat à la variable
%d'instance.

Il y a des variantes spécifiques de l'inspecteur pour les dictionnaires (sous-classes 
de Dictionaries), pour les collections ordonnées (sous-classes de OrderedCollections), 
pour les CompiledMethods (objets des méthodes compilées) et pour quelques autres classes 
facilitant ainsi l'examen du contenu de ces objets spéciaux.

%---------------------------------------------------------
\subsection{Object Explorer}

\arevoir{L'\emph{Object Explorer} ou \ind{explorateur} d'objets est sur le plan
conceptuel semblable à l'inspecteur mais présente ses informations
de manière différente.
Pour voir la différence, nous allons \emph{explorer} le même objet
que nous venons juste d'inspecter.} % martial: OBSOLETE ou pas

\begin{figure}[tbp]
\begin{minipage}{0.48\textwidth}
	\begin{center}
	\ifluluelse
		{\includegraphics[width=\textwidth]{exploreTimeStampNow}}
		{\includegraphics[width=0.7\textwidth]{exploreTimeStampNow}}
	\end{center}
	\caption{Explorer \ct{TimeStamp now}.}
	\figlabel{exploreTimeStampNow}
\end{minipage}
\hfill
\begin{minipage}{0.48\textwidth}
	\begin{center}
	\ifluluelse
		{\includegraphics[width=\textwidth]{exploreTimeStampNow2}}
		{\includegraphics[width=0.7\textwidth]{exploreTimeStampNow2}}
	\end{center}
	\caption{Explorer les variables d'instance.}
	\figlabel{exploreTimeStampNow2}
\end{minipage}
\end{figure}

% rene : pourquoi ne pas remplacer self par "la plus haute ligne" ?
% martial : bonne idée mais ce n'est plus (explore (I) mais Explore Pointers (e)) 
\dothis{Sélectionnez %\arevoir{\menu{self}} dans le panneau gauche de notre
la plus haut ligne dans le panneau gauche de notre
inspecteur et choisissez % \menu{explore (I)} dans le menu contextuel
\menu{Explore Pointers (e)} dans le menu contextuel
obtenu en \actclickant.}
La fenêtre \ind{Explorer} apparaît alors comme sur
 \figref{exploreTimeStampNow}.
Si vous cliquez sur le petit triangle à gauche de \ct{root} (racine, en anglais), 
la vue changera comme dans \figref{exploreTimeStampNow2} qui
nous montre les variables d'instance de l'objet que nous explorons.
Cliquez sur le triangle proche d'\ct{offset} et vous verrez
\emph{ses} variables d'instance.
L'explorateur est véritablement un outil puissant lorsque vous avez besoin
d'explorer une structure hiérarchique complexe\,---\,d'où son nom.
\index{raccourci clavier!explore it}

Le panneau Workspace de l'Object Explorer fonctionne de façon
 légèrement différente de celui de l'Inspector.
\ct{self} n'est pas lié à l'objet racine root mais plutôt
à l'objet actuellement sélectionné; les variables d'instance de
l'objet sélectionné sont aussi à portée~\footnote{En anglais, vous
entendrez souvent le terme ``scope'' pour désigner la portée des
variables d'instance.}.

Pour comprendre l'importance de l'explorateur, employons-le pour
explorer une structure profonde imbriquant beaucoup d'objets.

\dothis{Évaluez \ct{Object explore} dans un espace de travail.}
C'est l'objet qui représente la classe  \ct{Object} dans \pharo.
Notez que vous pouvez naviguer directement dans les objets
représentants le dictionnaire de méthodes et même explorer les
méthodes compilées de cette classe (voir \figref{ExploreObject}). % CHANGE

\begin{figure}[tbp]
	\begin{center}
		\includegraphics[width=0.5\textwidth]{ExploreObject}
	\end{center}
	\caption{Explorer un \ct{Object}.}
	\figlabel{ExploreObject}
\end{figure}

% \dothis{Ouvrez un navigateur et \actclickz{} cinq fois 
%  sur le panneau des méthodes
% de manière à afficher le \emph{halo} Morphic sur
% le morph \ct{PluggableListMorph} qui est utilisé pour représenter
% la liste des messages.
% \Clickz{}sur l'icône
% \emph{debug} \debugHandle{} et sélectionnez dans le menu flottant
% \menu{explore morph}.  
% Ceci ouvrira un Explorer sur l'objet \clsind{OBPluggableListMorph} qui
% représente la liste de méthodes du navigateur à l'écran.
% Ouvrez l'objet root (en cliquant sur son triangle), ouvrez ses sous-morphs
% \ct{submorphs} et continuez d'explorer la structure des objets sur lesquels
% reposent ce morph comme nous pouvons le voir sur
% \figref{explorePluggableListMorph}.}

% \begin{figure}[tbp]
% 	\begin{center}
% 		\includegraphics[width=0.7\textwidth]{explorePluggableListMorph}
% 	\end{center}
% 	\caption{Explorer une \ct{OBPluggableListMorph}.}
% 	\figlabel{explorePluggableListMorph}
% \end{figure}

%=========================================================
\section{Le débogueur}
%\section{Debugger, le débogueur} % CHANGE
\seclabel{debugger} % (fold)

Le \ind{débogueur} \ind{Debugger} est sans conteste l'outil le plus
puissant dans la suite d'outils de \pharo. 
%is arguably the most powerful tool in the Squeak tool suite.
Il est non seulement employé pour déboguer c'est-à-dire pour corriger les erreurs
%ajout
mais aussi pour écrire du code nouveau.
Pour démontrer la richesse du Debugger, commençons par
créer un \emph{bug}!

\dothis{Via le navigateur, ajouter la méthode suivante dans la classe \ct{String}:}

\needlines{7} % REVOIR
\begin{method}[buggy]{Une méthode boguée}
suffix
	"disons que je suis un nom de fichier et que je fournis mon suffixe, la partie suivant le dernier point"
	| dot dotPosition |
	dot := FileDirectory dot.
	dotPosition := (self size to: 1 by: -1) detect: [ :i | (self at: i) = dot ].
	^ self copyFrom: dotPosition to: self size 
\end{method}

Bien sûr, nous sommes certain qu'une méthode si triviale fonctionnera.
Ainsi plutôt que d'écrire un test \emph{SUnit} 
%ajout
(que nous verrons dans \charef{SUnit}),
nous entrons simplement \ct{'readme.txt' suffix} dans un Workspace
et nous en imprimons l'exécution via \menu{print it (p)}.
Quelle surprise! Au lieu d'obtenir la réponse attendu \ct{'txt'}, 
une notification \clsind{PreDebugWindow} s'ouvre comme sur
\figref{PreDebugWindow}.

\begin{figure}[btp]
	\begin{center}
		{\includegraphics[width=0.8\textwidth]{PreDebugWindow}}
	\end{center}
	\caption{Un \ct{PreDebugWindow} nous alarme de la présence d'un bug.}
	\figlabel{PreDebugWindow}
\end{figure}

\tradalert{martial}{modif récente 2011-04-19}\\
Le \ct{PreDebugWindow} nous indique dans sa barre de titre
qu'une erreur s'est produite et nous affiche une trace de la pile d'exécution
ou \emphind{stack trace} des messages qui ont conduit à l'erreur:
% ajout vf martial
\arelire{la dernière exécution est représentée par la plus haute ligne sur la pile.
En descendant la trace (donc en remontan le temps), nous tombons sur la ligne
 \ct{UndefinedObject>>>DoIt} qui représente le code qui vient d'être compilé
et lancé quand nous avons demandé à \pharo d'imprimer 
le code \ct{'readme.txt' suffix} dans notre espace de travail
par \menu{print it}.}
Ce code a envoyé le message \ct{suffix} à
l'objet \clsind{ByteString} (\ct{'readme.txt'}).
S'en est suivi l'exécution de la méthode \ct{suffix} héritée de la
classe \ct{String};
toutes ces informations sont disponibles dans la ligne précédente de la trace,
\ct{ByteString(String)>>>suffix}.
En remontant la pile, nous pouvons voir que \ct{suffix} envoie
à son tour \ct{detect:}; cette dernière méthode envoie à son tour \ct{detect:ifNone} qui émet 
\ct{errorNotFound}.
\clsindex{UndefinedObject}

\begin{figure}[btp]
	\begin{center}
	\ifluluelse
		{\includegraphics[width=\textwidth]{debuggerDetectIfNone}}
		{\includegraphics[width=0.7\textwidth]{debuggerDetectIfNone}}
	\end{center}
	\caption{Le débogueur.}
	\figlabel{debuggerDetectIfNone}
\end{figure}

Pour trouver \emph{pourquoi} le point (\ct{dot}) n'a pas été trouvé,
nous avons besoin du débogueur lui-même que nous pouvons appeler en
\clickant{} sur le bouton \button{Debug}
%ajout vf martial
\arelire{ou en \clickant{} sur une ligne de la pile}.

% \dothis{Vous pouvez aussi ouvrir Debugger en \clickant
% sur n'importe quelle ligne du \emph{stack trace}. 
% Si vous faites ainsi, le débogueur s'ouvrira sur la méthode correspondante.}

Le débogueur est visible sur \figref{debuggerDetectIfNone}; 
il semble intimidant au début, mais il est assez facile à utiliser.
La barre de titre et le panneau supérieur sont très similaires
à ceux que nous avons vu dans le notificateur \lct{PreDebugWindow}.
Cependant, le Debugger combine la trace de la pile avec un navigateur de
méthode, ainsi quand vous sélectionnez une ligne dans le \emph{stack
trace}, la méthode correspondante s'affiche dans le panneau inférieur.
Vous devez absolument comprendre que l'exécution qui a causée l'erreur
est toujours dans l'image mais dans un état suspendu.
Chaque ligne de la trace représente une tranche de la pile
d'exécution qui contient toutes les informations nécessaires
pour poursuivre l'exécution. Ceci comprend tous les objets impliqués
dans le calcul, avec leurs variables d'instance et toutes les variables
temporaires des méthodes exécutées.

Dans \figref{debuggerDetectIfNone} nous avons sélectionné
la méthode \ct{detect:ifNone:} dans le panneau supérieur.
Le corps de la méthode est affiché dans le panneau central;
%le surlignage?
la sélection bleue entourant le message \ct{value} nous montre
que la méthode actuelle a envoyé le message \ct{value} et
attend une réponse.

Les quatre panneaux inférieurs du débogueur sont véritablement deux
mini-inspecteurs (sans panneaux-espace de travail).
L'inspecteur de gauche affiche l'objet actuel,
c'est-à-dire l'objet nommé \self dans le panneau central.
En sélectionnant différentes lignes de la pile, l'identité de \self
peut changer ainsi que le contenu de
l'inspecteur du \self{}.
Si vous cliquez sur \self dans le panneau inférieur gauche, vous verrez
que \self est un intervalle \ct{(10 to: 1 by -1)}, ce à quoi nous devions
nous attendre.
Les panneaux Workspace ne sont pas nécessaires dans les mini-inspecteurs
de Debugger car toutes les variables sont aussi à portée dans
le panneau de méthode; vous pouvez entrer et évaluer à loisir 
n'importe quelle expression.
Vous pouvez toujours annuler vos changements en utilisant 
\menu{cancel (l)} dans le menu ou en tapant \short{\textit{l}}. 
% apb: that lower-case-L is in italics so that it doesn't look like a 1 or a |
\index{raccourci clavier!cancel}

L'inspecteur de droite affiche les variables temporaires du contexte courant.
\arelire{%
% ajout vf martial
Sur \figref{debuggerDetectIfNone},
nous voyons 
dans le panneau de méthode au centre du débogueur que le message
\ct{value} a été envoyé au paramètre-bloc \ct{exceptionBlock} passé en argument de la
méthode \ct{detect:ifNone:};
ce paramètre se retrouve bien dans notre troisième colonne de variables temporaires.}
% rene : il nous faudrait clarifier l'explication
% martial: et voilà! (j'ai préféré rattaché au paragraphe du haut)
\arelire{Comme nous pouvons le voir sur la méthode \ct{detect:} plus bas sur la
  pile, notre paramètre \ct{exceptionBlock} est \ct{[self errorNotFound: aBlock]}. Il n'y
a donc rien de surprenant à voir le message d'erreur correspondant.}

Si vous voulez ouvrir un inspecteur complet sur une des variables
affichées dans les mini-inspecteurs, vous n'avez qu'à double-cliquer
sur le nom de la variable ou alors sélectionner le nom de la variable et 
\actclickz pour choisir
 \menu{inspect (i)} ou \menu{explore (I)}: % ce raccourci-clavier ``I'' est toujours valable ici
utile si vous voulez suivre le changement d'une variable lorsque vous exécutez un autre code.
\index{raccourci clavier!inspect it}
\index{raccourci clavier!explore it}

\arelire{% modif
En revenant sur le panneau de méthode, nous voyons que nous
nous attendions à trouver \ct{dot} dans la chaîne de caractère
\ct{'readme.txt'} à l'avant-dernière 
ligne de la méthode et que l'exécution n'aurait jamais du atteindre 
la dernière ligne.}
\pharo ne nous permet pas de lancer une exécution en arrière mais
il permet de \emph{relancer une méthode}, ce qui marche parfaitement dans
notre code qui ne change pas les objets mais qui en crée de nouveaux.

\dothis{\arelire{En restant sur la ligne \ct{detect:ifNone:}, cliquez sur 
le bouton \button{Restart} et vous verrez que 
l'exécution retournera dans l'état premier de la méthode courante.}
La sélection bleue englobe maintenant le message  
\mbox{\ct{do:}} (voir \figref{RestartDetectIfNone}).}

\begin{figure}[btp]
	\begin{center}
	\ifluluelse
		{\includegraphics[width=\textwidth]{RestartDetectIfNone}}
		{\includegraphics[width=0.7\textwidth]{RestartDetectIfNone}}
	\end{center}
	\caption{Debugger après avoir relancé la méthode \ct{detect: ifNone:}.}
	\figlabel{RestartDetectIfNone}
\end{figure}

Les boutons \button{Into} et \button{Over} offrent deux façons différentes de 
parcourir l'exécution pas-à-pas.
Si vous cliquez sur le bouton \button{Over} \arelire{(en français, ``par dessus'')},
\pharo exécutera sauf erreur l'envoi du message actuel (dans notre cas \ct{do:}) d'un
seul pas (en anglais, \emph{step}).
Ainsi \button{Over} nous amènera sur le prochain message à envoyer dans la méthode courante.
Ici nous passons à \ct{value}\,---\,c'est exactement l'endroit d'où nous avons démarré
et ça ne nous aide pas beaucoup.
En fait, nous avons besoin de trouver pourquoi \ct{do:} ne trouve pas
le caractère que nous cherchons.

\dothis{Après avoir \clicke sur le bouton \button{Over}, \clickz sur le 
bouton \button{Restart} pour revenir encore une fois au début de la méthode
dans le même état que sur \figref{RestartDetectIfNone}.}

%ajout vf
\arelire{Après ce coup pour rien, essayons de parcourir l'exécution autrement. Pour ce faire, nous utiliserons le bouton \button{Into} (en français, ``dedans'') permettant de rentrer dans la méthode pour une exécution pas-à-pas.}

\dothis{Cliquez sur le bouton \button{Into}; \pharo ira dans la méthode correspondante 
au message surligné par la sélection bleue; dans ce cas, \ct{Collection>>>do:}.}

\arelire{Malheureusement,} ceci ne nous aide pas plus: nous pouvons être confiant
dans le fait que la méthode \ct{Collection>>>do:} n'est pas erronée. 
Le bug se situe plutôt dans \emph{ce que} nous demandons à \pharo de faire.
%\emph{what}
\button{Through} (en français, ``à travers'') est le bouton approprié à ce cas: nous
voulons ignorer les détails de \ct{do:} lui-même et se focaliser sur
l'\emph{exécution du bloc}, argument de \ct{do:}.

\dothis{Sélectionnez encore la méthode \ct{detect:ifNone:} et
\clickz{} sur le bouton \button{Restart} pour revenir à l'état de
\figref{RestartDetectIfNone}.
\Clickz maintenant sur le bouton \button{Through} plusieurs fois. 
Sélectionnez \ct{each} dans le mini-inspecteur de contexte (en bas à droite).
%in the context window as you do so.
Vous remarquez que \ct{each} décompte depuis \ct{10} au fur et à mesure de
l'exécution de la méthode \ct{do:}.}

Quand \ct{each} est \ct{7} 
% ajout vf 
\arelire{(normalement après sept \clickbtn{}s sur \button{Through}\,--\,si vous êtes perdus vous pouvez toujours redémarrer en \clickant{} sur le bouton \button{Restart}),}
nous nous attendons à ce que le bloc \ct{ifTrue:} soit exécuté, mais ce n'est pas le cas:
% ajout vf
\arelire{si vous \clickz{} encore sur le bouton \button{Through}, vous passerez à \ct{6} comme
si notre point (\ct{dot}) n'était pas vu.}
Pour voir ce qui ne marche pas, \arelire{rendez-vous
après septième \clickbtn sur le bouton \button{Through} dans l'état illustré par
\figref{steppingIntoValue}. De là, allez \emph{dans} l'exécution de 
\ct{value:}:
% ajout vf
\clickz{} pour ce faire sur le bouton \button{Into}}.

\begin{figure}[btp]
	\begin{center}
	\ifluluelse
		{\includegraphics[width=\textwidth]{steppingIntoValue}}
		{\includegraphics[width=0.7\textwidth]{steppingIntoValue}}
	\end{center}
	\caption{Debugger après un \emph{pas} \arelire{\button{Into}} dans la méthode
      \ct{do:} plusieurs fois grâce au bouton \button{Through}.}
	\figlabel{steppingIntoValue}
\end{figure}

\arelire{%
Après avoir \clicke{} sur le bouton \button{Into}, 
vous obtiendrez une fenêtre de Debugger dans la même position que sur
\figref{dotIsAString}.}
Tout d'abord, il semble que nous soyons \emph{revenus} à la méthode 
\ct{suffix} mais c'est parce que nous exécutons désormais le bloc
que \ct{suffix} fourni en argument au message \ct{detect:}.
%\on{does not work any more! the debugger does not know about block variables!}  
% Si vous sélectionnez \ct{i} dans le mini-inspecteur contextuel, 
% vous pouvez voir sa valeur actuelle, qui devrait être \ct{7} 
% si vous avez suivi jusqu'ici la procédure.
% Vous pouvez alors sélectionner l'élément correspondant de \self
% dans l'inspecteur de \self.
% %\self{}-inspector.
% Dans \figref{dotIsAString}, vous pouvez voir que l'élément
% \ct{7} de la chaîne de caractères est le caractère 46: ce
% n'est pas un caractère-point.
Si vous sélectionnez \ct{dot} dans l'inspecteur contextuel, 
vous verrez que sa valeur est \ct{'.'}.
Vous constatez maintenant qu'ils ne sont pas égaux: le septième caractère
de \ct{'readme.txt'} est pourtant un objet \ct{Character} (donc un caractère), 
alors que \ct{dot} est un \ct{String} (\ie une chaîne de caractères entre \emph{quotes}).

\begin{figure}[btp]
	\begin{center}
	\ifluluelse
		{\includegraphics[width=\textwidth]{dotIsAString}}
		{\includegraphics[width=0.7\textwidth]{dotIsAString}}
	\end{center}
	\caption{Debugger montrant pourquoi \ct{'readme.txt' at: 7} n'est pas égal à \ct{dot}.}
	\figlabel{dotIsAString}
\end{figure}

Maintenant nous pouvons mettre le doigt sur le bug, 
la correction~\footnote{En anglais, nous parlons de \emph{bug fix}.} 
est évidente: nous devons convertir \ct{dot} en un caractère avant de 
recommencer la recherche.
%before starting to search for it.  

\begin{figure}[btp]
 	\begin{center}
 	\ifluluelse
 		{\includegraphics[width=\textwidth]{revertDialog}}
 		{\includegraphics[width=0.7\textwidth]{revertDialog}}
 	\end{center}
 	\caption{Changer la méthode \ct{suffix} dans Debugger: demander
      la confirmation de la sortie du bloc interne. La boîte
      d'alerte nous dit: ``Je devrais revenir à la méthode d'où ce bloc est originaire. Est-ce bon?''. }
 	\figlabel{revertDialog}
 \end{figure}

\dothis{Changez le code directement dans le débogueur de 
façon à ce que l'affectation soit de la forme
\ct{dot := FileDirectory dot first}:
%ajout vf
\arelire{%
\ct{SequenceableCollection>>>first} renvoie le élément de la collection donc ici le premier caractère de la chaîne de caractères et ainsi, \ct{dot} correspond bien au caractère \ct{.} désormais}.
Acceptez la modifications via l'option \menu{accept} du menu contextuel.}
% rene propose : \menu{accept} pour accepter la modification  ?
% martial : pas utile; à ce stade le lecteur connait bien accept/print it/do it; bon! c'est fait

Puisque nous sommes en train d'exécuter le code dans un bloc à
l'intérieur d'un \lct{detect:}, plusieurs trames de la pile 
devront être abandonnées de manière à valider le changement.
\pharo nous demande si c'est ce que nous voulons (voir \figref{revertDialog})
et, à condition de \click{} sur \menu{yes}, \pharo sauvegardera
(et compilera) la nouvelle méthode.

%\dothis{Cliquez sur le bouton \button{Restart} et ensuite \button{Proceed}; Debugger disparaîtra et l'évaluation de l'expression \ct{'readme.txt' suffix} sera complète et affichera la réponse \ct{'.txt'}}

\arelire{L'évaluation de l'expression \ct{'readme.txt' suffix} sera complète et affichera la réponse \ct{'.txt'}.} 

Est-ce pour autant une réponse correcte?  Malheureusement nous ne pouvons
répondre avec certitude.
%Unfortunately, we can't say for sure.  
Le suffixe devrait-il être \ct{.txt} ou \ct{txt}?
Le commentaire dans la méthode \ct{suffix} n'est pas très précis.
% note de martial: commentaire traduit car en reference
La façon d'éviter ce type de problème est d'écrire
un test \ind{SUnit} pour définir la réponse.

\begin{method}[testSuffix]{Un simple test pour la méthode \ct{suffix}}
testSuffixFound
	self assert: 'readme.txt' suffix = 'txt'
\end{method}

L'effort requis pour ce faire est à peine plus important que celui
qui consiste à lancer le même test dans un espace de travail;
l'avantage de \sunit est de sauvegarder ce test sous la forme d'une
documentation exécutable et de faciliter l'accessibilité des usagers
de la méthode.
% note de martial: j'ai retourne un peu le sens de la phrase: The effort required to do that was little more than to run the same test in the workspace, but using \sunit saves the test as executable documentation, and makes it easy for others to run.
% rene approuve
En plus, si vous ajoutez \tmthref{testSuffix} à la classe
\ct{StringTest} et que vous lancez ce test avec \sunit, vous
pouvez très facilement revenir pour déboguer l'actuelle erreur.
% very quickly get back to debugging the error.
\sunit ouvre Debugger sur l'assertion fautive mais là vous
avez simplement besoin de descendre d'une ligne dans la pile,
% you need only go back down the stack one frame,
redémarrez le test avec le bouton \button{Restart} et allez
dans la méthode \ct{suffix} par le bouton \button{Into}. Vous
pouvez alors corriger l'erreur, comme nous l'avons fait dans
\figref{fixOffByOne}.
Il s'agit maintenant de cliquer sur le bouton \button{Run Failures} dans
le \sunit Test Runner et de se voir confirmer que le test passe (en anglais, \emph{pass}) 
normalement. Rapide, non?

\begin{figure}[btp]
	\begin{center}
		\includegraphics[width=\textwidth]{fixOffByOne}
	\end{center}
	\caption{Changer la méthode \ct{suffix} dans Debugger: corriger l'erreur du plus-d'un-point après l'assertion fautive \sunit.}
	\figlabel{fixOffByOne}
\end{figure} % CHANGE

Voici un meilleur test:

\begin{method}[testSuffix2]{Un meilleur test pour la méthode \ct{suffix}}
testSuffixFound
	self assert: 'readme.txt' suffix = 'txt'.
	self assert: 'read.me.txt' suffix = 'txt'
\end{method}
\noindent
Pourquoi ce test est-il meilleur? Simplement parce que
nous informons le lecteur de ce que la méthode devrait faire 
s'il y a plus d'un point dans la chaîne de caractères, instance de String.

Il y a d'autres moyens d'obtenir une fenêtre de débogueur en plus de ceux
qui consistent à capturer une erreur effective ou à faire une assertion
fautive (ou \emph{assertion failures}).
Si vous exécutez le code qui conduit à une boucle infinie, vous pouvez
l'interrompre et ouvrir un débogueur durant le calcul en tapant \short{.}%
% (that's a full stop or a period, depending  on where you learned English).
~\footnote{Sachez que vous pouvez ouvrir un débogueur d'urgence n'importe quand en tapant
\short{{\sc shift}.}}
Vous pouvez aussi éditer simplement le code suspect en insérant l'expression \ct{self halt}.
Ainsi, par exemple, nous pourrions éditer la méthode \ct{suffix} comme suit:
\index{processus!interruption}

\needspace{11ex}
\begin{method}[suffix]{Insérer une pause par \ct{halt} dans la méthode \ct{suffix}}
suffix
	"disons que je suis un nom de fichier et que je fournis mon suffixe, la partie suivant le dernier point"
	| dot dotPosition |
	dot := FileDirectory dot first.
	dotPosition := (self size to: 1 by: -1) detect: [ :i | (self at: i) = dot ].
	self halt.
	^ self copyFrom: dotPosition to: self size 
\end{method}

Quand nous lançons cette méthode, l'exécution de \ct{self halt} ouvre 
un \ind{notificateur} ou \emph{\ind{pre-debugger}} d'où nous pouvons continuer 
%ajout
en cliquant sur \menu{proceed}
ou déboguer et explorer l'état des variables, parcourir pas-à-pas la pile d'exécution et éditer le code.

C'est tout pour le débogueur mais nous n'en avons pas fini avec la méthode \ct{suffix}.
Le bug initial aurait dû vous faire réaliser que s'il n'y a pas de point dans la chaîne 
cible la méthode \ct{suffix} lèvera une erreur.
Ce n'est pas le comportement que nous voulons. Ajoutons ainsi un second test
pour signaler ce qu'il pourrait arriver dans ce cas.  

\begin{method}[testNoSuffix]{Un second test pour la méthode \ct{suffix}: la cible n'a pas de suffixe}
testSuffixNotFound
	self assert: 'readme' suffix = ''
\end{method}

\dothis{Ajoutez \tmthref{testNoSuffix} à la suite de tests dans la classe \clsind{StringTest} 
et observez l'erreur levée par le test.
Entrez dans Debugger en sélectionnant le test erroné dans \sunit puis éditez
le code de façon à passer normalement le test (donc sans erreur).
La méthode la plus facile et la plus claire consiste à remplacer le message \ct{detect:} 
par \ct{detect: ifNone:}~\footnote{En anglais, \emph{if none} signifie ``s'il n'y a rien''.}  
où le second argument un bloc qui retourne tout simplement une chaîne.}

Nous en apprendrons plus sur SUnit dans \charef{SUnit}.

% section debugger (end)

%=========================================================
\section{Le navigateur de processus}
% Process Browser

\st est un systême multitâche: plusieurs processus légers (aussi
connu sous le nom de \emph{threads}) tournent simultanément dans
votre image.
%fonctionnent de façon concourrante dans votre image.
Dans l'avenir la machine virtuelle de \pharo bénéficiera davantage
des multi-processeurs lorsqu'ils seront disponibles, mais le partage
d'accès est actuellement programmé sur le principe de 
tranches temporelles (ou \emph{time-slice}).
%concurrency is implemented by time-slicing.

\begin{figure}[btp]
	\begin{center}
	\ifluluelse
		{\includegraphics[width=\textwidth]{processBrowser}}
		{\includegraphics[width=0.7\textwidth]{processBrowser}}
	\end{center}
	\caption{Le Process Browser.}
	\figlabel{processBrowser}
\end{figure}

Le Process \subind{processus}{Browser} ou navigateur de \subind{navigateur}{processus} 
est un cousin de Debugger qui vous permet d'observer les divers processus tournant
dans le système \pharo.
\Figref{processBrowser} nous en présente une capture d'écran.
Le panneau supérieur gauche liste tous les processus présents dans \pharo, 
dans l'ordre de leur priorité depuis le \emph{timer interrupt watcher} 
%?
(système de surveillance d'interruption d'horloge) de priorité
80 au \emph{idle process} ou processus inactif du système de priorité 10.
Bien sûr, sur un système mono-processeur, le seul processus pouvant être 
lancé en phase de visualisation est le \emph{UI~\footnote{UI désigne 
\emph{User Interface}; en français, interface utilisateur.} process} 
ou processus graphique;
%when you look is the UI process; 
tous les autres processus seront en attente d'un quelconque événement.
%:===> Process browser context menu is broken!
\on{broken -- to be fixed!}
Par défaut, l'affichage des processus est statique; il peut être
mis à jour en \actclickant{} et en sélectionnant \menu{turn on auto-update (a)}. % CHANGE

Si vous sélectionnez un processus dans le panneau supérieur gauche, le panneau de droite
affichera son \emph{stack trace} tout comme le fait le débogueur.
Si vous en sélectionnez un, la méthode correspondante est affichée dans le panneau
inférieur.
Le Process Browser n'est pas équipé de mini-inspecteurs pour 
\self et \lct{thisContext} mais \actclick{} sur les
tranches de la pile offre une fonctionnalité  équivalente. % CHANGE

%=========================================================
\section{Trouver les méthodes}
\seclabel{methodFinder} 

Il y a deux outils dans \pharo pour vous aider à trouver des
messages. %CHANGE
Ils diffèrent en termes d'interface et de fonctionnalité.

Le \emph{Method Finder} (ou chercheur de méthodes) a été longuement décrit
dans \secref{quick:methodFinder}; vous pouvez l'utiliser pour trouver des méthodes
par leur nom ou leur fonction.
Cependant, pour observer le corps d'une méthode, le Method Finder ouvre 
un nouveau navigateur. Cela peut vite devenir pénible.
% overwhelming.

\begin{figure}[btp]
	\begin{center}
	\ifluluelse
		{\includegraphics[width=\textwidth]{methodNamesRandom}}
		{\includegraphics[width=0.7\textwidth]{methodNamesRandom}}
	\end{center}
	\caption{Le Message Names Browser montrant toutes les méthodes
      contenant le sous-élement de chaîne \ct{random} dans leur
      sélecteur.} % CHANGE : Method Names Browser -> Message Names Browser
	\figlabel{methodNamesRandom}
\end{figure}

\index{Message Names Browser}
\seeindex{navigateur de noms de messages}{Message Names Browser}

La fonctionnalité de recherche du Message Names Browser ou navigateur de 
\emph{noms de messages} est plus limitante: vous entrez un morceau d'un sélecteur de 
message dans la boîte de recherche et le navigateur liste toutes les méthodes
contenant ce fragment dans leurs noms, comme nous pouvons le voir dans
\figref{methodNamesRandom}.
Cependant, c'est un navigateur complet:
%full-fledged browser:
si vous sélectionnez un des noms dans le panneau de gauche, toutes les méthodes
ayant ce nom seront listées dans celui de droite et vous pourrez alors naviguer dans
le panneau inférieur.
Le Message Names Browser a une barre de bouton, comme le Browser, % CHANGE
pouvant être utilisée pour ouvrir d'autres navigateurs sur la méthode choisie
ou sur sa classe.
% section methodFinder (end)

%=========================================================
\section{Change set et son gestionnaire Change Sorter}
% Change sets and the Change Sorter
\seclabel{env:changeSet} % (fold)

Chaque fois que vous travaillez dans \pharo, tous les changements que vous effectuez
sur les méthodes et les classes sont enregistrés dans un
\ct{change set} (traduisible par ``ensemble des modifications'').
%note temporaire de martial: je dois changer sauf contre-ordre change set en \ct{change set} sans traduction
Ceci inclus la création de nouvelles classes, le renommage de classes, le changement de
catégories, l'ajout de méthodes dans une classe existante\,---\,en bref, tout ce qui a un impact sur le système.
Cependant, les exécutions arbitraires avec \emph{do it} ne sont pas
incluses; par exemple, si vous créez une nouvelle variable globale par affectation dans 
un espace de travail, la création de variable ne sera pas dans un 
\subind{fichier}{change set}.
%%\index{Change Set Browser}
%%\seeindex{navigateur de change set}{Change Set Browser}
\index{Change Sorter}

A tout moment, beaucoup de \changesets existent, mais un seul d'entre eux\,---\,\ct{ChangeSet current}\,---\,collecte les changements qui sont en cours dans l'image actuelle.
Vous pouvez voir quel \changeset est le \changeset actuel et vous pouvez examiner
tous les \changesets en utilisant le Change Sorter
(ou trieur de \changeset) disponible dans le menu
principal dans \menu{World\go{} Tools \ldots \go{} Change Sorter}.

\begin{figure}[btp]
	\begin{center}
		\includegraphics[width=\linewidth]{changeSorter}
	\end{center}
	\caption{Le Change Sorter.}
	\figlabel{changeSorter}
\end{figure}

\Figref{changeSorter} nous montre ce navigateur. La barre de titre affiche le \changeset actuel et ce \changeset est sélectionné quand le navigateur s'ouvre.

Les autres \changesets peuvent être choisis dans le panneau supérieur de gauche;
le menu contextuel accessible via le \ind{bouton jaune} vous permet de faire de
n'importe quel \changeset votre \changeset actuel ou de créer un nouveau \changeset.
Le panneau supérieur de droite liste toutes les classes 
(accompagnées de leurs catégories) affectées par le \changeset sélectionné.
Sélectionner une des classes affiche les noms de ses méthodes qui sont aussi dans
le \changeset (\emph{pas} toutes les méthodes de la classe) dans le panneau central
et sélectionner un de ces noms de méthodes affiche sa définition dans le panneau
inférieur.
Remarquez que le navigateur ne montre \emph{pas} si la création de la classe elle-même
fait partie du \changeset bien que cette information soit stockée dans la structure
de l'objet qui est utilisé pour représenter le \changeset.

Le Change Sorter vous permet d'effacer des classes et des méthodes du \changeset
en  \actclickant{} sur les élements correspondants.
% Cependant, pour une édition plus élaborée, vous devez utiliser un deuxième
% programme, le \textit{Change Sorter} (ou trieuse de \changeset), disponible sous ce nom dans 
% \toolsflap ou en passant par \menu{World\go{}open...\go{}dual change sorter}. Nous
% pouvons le voir dans \figref{changeSorter}.

Le Change Sorter vous permet de voir simultanément deux
\changesets: un \changeset à gauche et un autre à droite.
Cette fonctionalité offre les principales fonctions du Change Sorter 
% rene propose :  \arevoir{disposition offre} 
telles que la possibilité de déplacer ou copier les changements d'un \changeset à un autre,
comme nous pouvons le voir sur \figref{changeSorter},
dans le menu contextuel accessible en \actclickant.
Nous pouvons aussi copier des méthodes d'un pan à un autre.

Vous pouvez vous demander pourquoi vous devez accorder de l'importance à la composition
d'un \changeset: la réponse est que les \changesets fournissent un mécanisme simple
pour exporter du code depuis \pharo vers le système de fichiers d'où il peut
être importé dans une autre image \pharo ou vers un autre \st que \pharo.
L'exportation de \changeset est connu sous le nom ``filing-out'' et peut être réalisé
en utilisant le menu contextuel obtenu en \actclickant{} sur n'importe quel \changeset, classe ou
méthode dans n'importe quel navigateur.
Des exportations (ou fileouts) répétées créent une nouvelle version du fichier
mais les \changesets ne sont pas un outil de versionnage (gestion de
versions) comme peut l'être Monticello:
ils ne conservent pas les dépendances.
\index{fichier!filing-out}
\index{fichier!exportation}

Avant l'avènement de Monticello, les \changesets étaient la
technique majeure d'échange de code entre les Smalltalkiens. % AREVOIR
%%utilisateurs %de \pharo (en anglais \emph{\pharo{}ers}). % avant 33691

% [squeak-fr] \sq{}ers: squeakiens, squeakois, squeakais, squeakeux?
Ils ont l'avantage d'être simples et relativement portables (le fichier d'exportation 
n'est qu'un fichier texte; \emph{nous ne vous recommandons pas} d'éditer ce fichier 
avec un éditeur de texte).
Il est assez facile aussi de créer un \changeset qui modifie
beaucoup de parties différentes du système sans aucun rapport
entre elles\,---\,ce pour quoi Monticello n'est pas encore équipé.
% martial: ou 'ce pourquoi'; je crois que les deux marches
% Rene : ce pour quoi est correct
%\ab{Or is it?}
%\on{you mean something different than extensions to foreign packages using the *package protocol notation?}

Le principal inconvénient des \changesets par rapport aux paquetages \ind{Monticello}
est leur absence de notion de dépendances.
Une exportation de \changeset est un ensemble d'\emph{actions} transformant n'importe quelle
image dans laquelle elle est chargée. Pour en charger avec succès, l'image doit être
dans un état approprié.
Par exemple, le \changeset pourrait contenir une action pour ajouter une méthode à une
classe; ceci ne peut être fait que si la classe est déjà définie dans l'image.
De même, le \changeset pourrait renommer ou re-catégoriser une classe, ce qui ne 
fonctionnerait évidemment que si la classe est présente dans l'image; les méthodes
pourraient utiliser des variables d'instance déclarées lors de l'exportation mais
inexistantes dans l'image dans laquelle elles sont importées.
Le problème est que les \changesets ne contiennent pas explicitemment les conditions 
sous lesquelles ils peuvent être chargés:
le fichier en cours de chargement marche \emph{au petit bonheur la chance} jusqu'à
ce qu'un message d'erreur énigmatique et un \emph{stack trace} surviennent
quand les choses tournent mal.
%the file in process just hopes for the best, usually resulting in a cryptic error message and a stack trace when things go wrong.
Même si le fichier fonctionne, un \changeset peut annuler silencieusement 
un changement fait par un autre.

À l'inverse, les paquetages (dits aussi packages) de Monticello représentent le code d'une manière 
déclarative: ils décrivent l'état que l'image devrait avoir une fois le chargement
effectué.
Ceci permet à Monticello de vous avertir des conflits (quand deux paquetages ont des
objectifs incompatibles)
%require contradictory final states
et vous permet de charger une série de paquetages dans un ordre de dépendances.

Malgré ces imperfections, les \changesets reste utiles; vous pouvez, en particulier, en trouver sur Internet pour en observer le contenu, voire les utiliser.
Maintenant que nous avons vu comment exporter des \changesets avec le Change Sorter,
nous allons voir comment les importer.
Cette étape requiert l'usage d'un autre outil, le File List Browser.
% section changeSet (end)

%=========================================================
\section{Le navigateur de fichiers File List Browser}

\begin{figure}[btp]
	\begin{center}
	\ifluluelse
		{\includegraphics[width=\textwidth]{fileList}}
		{\includegraphics[width=0.7\textwidth]{fileList}}
	\end{center}
	\caption{Le File List Browser.}
	\figlabel{fileList}
\end{figure}

Le navigateur de \subind{navigateur}{fichiers} ou \ind{File List Browser} est
en réalité un outil générique pour naviguer au travers d'un système de fichiers
(et aussi sur des serveurs FTP) depuis \pharo.
Vous pouvez l'ouvrir depuis le menu \menu{World \go{}Tools \ldots{}
  \go{}File Browser}. % CHANGE
Ce que vous y voyez dépend bien sûr du contenu de votre système de fichiers local
mais une vue typique du navigateur est illustrée sur 
\figref{fileList}.
\seeindex{fichier!navigation}{File List Browser}

Quand vous ouvrez un navigateur de fichiers, il pointera tout d'abord le répertoire
actuel, \ie celui depuis lequel vous avez démarré \pharo. La barre de titre
montre le chemin de ce répertoire.
Le panneau de gauche est utilisé pour naviguer dans le système de fichiers de manière
conventionnelle.
% note de martial; j'ai enleve le terme: larger pane parce que ce n'est pas forcement le plus grand 

Quand un répertoire est sélectionné, les fichiers qu'ils contiennent (mais pas les répertoires) 
sont affichés sur la droite. Cette liste de fichiers peut être filtrée en entrant dans la petite boîte
dans la zone supérieure gauche de la fenêtre un modèle de filtrage ou 
\emph{pattern} dans le style Unix.
Initialement, ce \emph{pattern} est \ct{*}, ce qui est égal à l'ensemble des fichiers, mais vous 
pouvez entrer une chaîne de caractères différente et l'accepter pour changer ce filtre. 
%ajout/sans les parentheses
Notez qu'un \ct{*} est implicitement joint ou pré-joint au \emph{pattern}
que vous entrez.
L'ordre de tri des fichiers peut être modifié via les boutons \button{name} (par nom), 
\button{date} (par date) et \button{size} (par taille).
Le reste des boutons dépend du nom du fichier sélectionné dans le navigateur.
Dans \figref{fileList}, le nom des fichiers a le suffixe \ct{.cs}, donc le navigateur
suppose qu'il s'agit de \changeset et ajoute les boutons \button{install} (pour
\textit{l'importer} dans un nouveau \changeset dont le nom est dérivé
de celui du fichier), \button{changes} (pour naviguer dans le changement du fichier),
\button{code} (pour l'examiner) et \button{filein} (pour charger
le code dans le \changeset \emph{actuel}).
Vous pourriez penser que le bouton \button{conflicts} vous informerait 
des modifications du \changeset pouvant être source de conflits dans le code existant
dans l'image mais ça n'est pas le cas.
\ab{Does anyone know what it does do?  I've never found it useful.}
\on{I tried it and found that it complained about linefeeds.}
En réalité, il vérifie juste d'éventuels problèmes dans le fichier (tel que
la présence de sauts de lignes ou \textit{linefeeds})
pouvant indiquer qu'il ne pourrait pas être proprement chargé.

\begin{figure}[btp]
	\begin{center}
	\ifluluelse
		{\includegraphics[width=\textwidth]{fileContentsBrowser}}
		{\includegraphics[width=0.7\textwidth]{fileContentsBrowser}}
	\end{center}
	\caption{Le File Contents Browser.}
	\figlabel{fileContentsBrowser}
\end{figure}

Puisque le choix des boutons affichés dépend du \emph{nom} du fichier et
non de son contenu, parfois le bouton dont vous avez besoin pourrait ne pas être
affiché.
De toutes façons, le jeu complet des options est toujours disponible
grâce
% martial: "de toute façon" est plus correct selon l'Académie française
à l'option \menu{more \ldots} du menu contextuel accessible en \actclickant,
ainsi vous pouvez facilement contourner ce problème.

Le bouton \button{code} est certainement le plus utile pour travailler avec les \changesets;
il ouvre un navigateur sur le contenu du fichier. Un exemple est présenté dans
\figref{fileContentsBrowser}.
Le File Contents Browser est proche d'un Browser à l'exception
des catégories; seuls les classes, les protocoles et les méthodes sont présentés.
Pour chaque classe, ce navigateur précise si la classe existe déjà dans
le système ou non et si elle est définie dans le fichier (mais \emph{pas} si
les définitions sont identiques).
Il affichera les méthodes de chaque classe
% note: doublon de 'the' a signaler dans l'original
ainsi que les différences entre la version actuelle et celle dans le fichier; ce que
nous montre \figref{fileContentsBrowser}.
Les options du menu contextuel de chacun des quatre panneaux supérieurs vous
permettront de charger (en anglais, \emph{file in}) le \changeset complet, la classe, 
le protocole ou la méthode correspondante.

%=========================================================
\section{En Smalltalk, pas de perte de codes}
%you can't lose code
\seclabel{cantLoseCode} % (fold)

\pharo peut parfois planter: en tant que système expérimental, \pharo vous permet
de changer n'importe quoi dont les élements vitaux qui font que \pharo fonctionne!
%It is quite possible to crash \pharo

\dothis{Pour \emph{crasher} malicieusement \pharo, évaluez \ct{Object become: nil}.}

La bonne nouvelle est que vous ne perdez jamais votre travail, même si votre 
image plante et revient dans l'état de la dernière version sauvegardée il y 
a de cela peut être des heures.
La raison en est que tout code exécuté est sauvegardé dans le fichier
\emph{.changes}.
Tout ceci inclut les expressions que vous évaluez dans un espace de travail Workspace,
tout comme le code que vous ajoutez à une classe en la programmant.
\index{fichier!changes} %voir QuickTour.tex

Ainsi, voici les instructions permetant de retrouver ce code.
Il n'est pas utile de lire ce qui suit tant que vous n'en avez pas besoin.
Cependant, quand vous en aurez besoin, vous saurez où le trouver.

Dans le pire des cas, vous pouvez toujours utiliser un éditeur de texte
sur le fichier \emph{.changes}, mais quand celui-ci pèse plusieurs méga-octets,
cette technique pourrait s'avérer lente et peu recommandable.
\pharo vous offre de meilleures façons de vous en sortir.

%---------------------------------------------------------
\subsection{La pêche au code}
%\subsection{Comment ramener du code}

Redémarrez \pharo depuis la sauvegarde (ou \emph{snapshot}) la plus récente et
sélectionnez \menu{World\go{}Tools \ldots \go{}Recover lost changes}. 
% Ceci vous ouvrira un Workspace plein d'expressions utiles. Les trois premières,

% \begin{code}{}
% Smalltalk recover: 10000.
% ChangeList browseRecentLog.
% ChangeList browseRecent: 2000.
% \end{code}

% \noindent
% sont les plus utiles pour le recouvrement de données.

% Si vous exécutez \ct{ChangeList browseRecentLog},
Vous aurez ainsi l'opportunité
de décider jusqu'où vous souhaitez revenir dans l'historique.
Normalement, naviguer dans les changements depuis la dernière sauvegarde
est suffisant (et vous pouvez obtenir le même effet en éditant 
\ct{ChangeList browseRecent: 2000} en tâtonnant sur le chiffre empirique
\ct{2000}).

Une fois que vous avez le navigateur des modifications récentes nommé
\emph{Recent Changes Browser} vous affichant les changements, disons, depuis votre
dernière sauvegarde, vous aurez une liste de tout ce que vous avez effectué 
dans \pharo durant tout ce temps.
Vous pouvez effacer des articles de cette liste en utilisant le menu
accessible en \actclickant.
Quand vous êtes satisfait, vous
pouvez charger (c'est-à-dire faire un \emph{file-in}) ce qui a été laissé
et ainsi incorporer les modifications dans un nouveau \changeset.

Une chose utile à faire dans le \emph{Recent Changes} Browser est
d'effacer les évaluations \emph{do it} via \menu{remove doIts}. 
Habituellement vous ne voudriez pas charger (c'est-à-dire re-exécuter) ses expressions.
Cependant, il existe une exception.
Créer une classe apparaît comme un \menu{doIt}.
\emph{Avant de charger les méthodes d'une classe, la classe doit exister.}
Donc, si vous avez créer des nouvelles classes, chargez \emph{en premier lieu} 
les \emph{doIts} créateur de classes, ensuite utilisez \menu{remove doIts} 
%ajout
(pour ne pas charger les expressions d'un espace de travail)
et enfin charger les méthodes.
%\lr{Maybe mention that class renames are not logged and completely screw up the change-set mechanism. (p. 174)}

Quand j'en ai fini avec le recouvrement (en anglais, \emph{recover}), 
j'aime exporter (par \emph{file-out}) mon nouveau \changeset, quitter \pharo
sans sauvegarder l'image, redémarrer et m'assurer que mon nouveau fichier
se charge parfaitement.
% section cantLoseCode (end)

%=========================================================
\section{Résumé du chapitre}

Pour développer efficacement avec \pharo, il est
important d'investir quelques efforts dans l'apprentissage des outils
disponibles dans l'environnement.

\begin{itemize}
  \item Le navigateur de classes standard ou \emph{Browser} est votre principale interface pour naviguer dans les catégories, les classes, les protocoles et les méthodes existants et pour en définir de nouveaux.
Ce navigateur offre plusieurs onglets pour accéder directement aux \senders (\ie{} les méthodes émettrices) ou aux \implementors (\ie{} les méthodes contenantes) d'un message, aux versions d'une méthode, 
\etc.
  \item Plusieurs navigateurs différents existent
(comme l'OmniBrowser )  
et plusieurs sont spécialisés (comme le Hierarchy Browser) pour fournir différentes vues sur les classes et les méthodes.
  \item Depuis n'importe quel outil, vous pouvez sélectionner en surlignant le nom d'une classe ou celui d'une méthode pour obtenir immédiatemment
un navigateur en utilisant le raccourci-clavier \short{b}.
  \item Vous pouvez aussi naviguer dans le système \st de manière
programmatique en envoyant des messages à \ct{SystemNavigation default}.
  \item \emph{Monticello} est un outil d'import-export, de versionnage
    (organisation et maintien de versions, en anglais, \emph{versioning}) et de partage de \emph{paquetages} de classes et de méthodes nommés aussi \emph{packages}.
  Un paquetage Monticello comprend une catégorie, des sous-catégories et des protocoles de méthodes associés dans d'autres catégories. 
  \item L'\emph{Inspector} et l'\emph{Explorer} sont deux outils utiles
pour explorer et interagir avec les objets vivants dans votre image.
Vous pouvez même inspecter des outils en \actclickant{} pour
afficher leur \emph{halo} et en sélectionnant l'icône
\emph{debug} \debugHandle .
  \item Le \emph{Debugger} ou débogueur est un outil qui non seulement vous permet
d'inspecter la pile d'exécution (\emph{runtime stack}) de votre
programme lorsqu'une erreur est signalée, mais aussi, 
vous assure une interaction avec tous les objets de votre application,
incluant le code source. Souvent, vous pouvez modifier votre 
code source depuis le Debugger et continuer l'exécution.
Ce débogueur est particulièrement efficace comme outil pour
le développement orienté test (ou, en anglais, \emph{test-first development}) en tandem
avec 
SUnit (\charef{SUnit}).
  \item Le \emph{Process Browser} ou navigateur de processus vous permet de piloter (monitoring), chercher (querying) et interagir avec les processus courants lancés dans votre image.
  \item Le \emph{Method Finder} et le \emph{Message Names Browser} sont 
deux outils destinés à la localisation de méthodes. Le
premier excelle lorsque vous n'êtes pas sûr du nom mais que vous
connaissez le comportement.
% note perso de martial: (attendu) 
Le second dispose d'une interface de navigation plus avancée pour le cas où
vous connaissez au moins une partie du nom.
  \item Les \changesets sont des journaux de bord (ou log) automatiquement générés pour 
tous les changements du code source dans l'image.
Bien que rendus obsolètes par la présence de Monticello comme
moyen de stockage et d'échange des versions de votre code source,
ils sont toujours utiles, en particulier pour réparer des erreurs
catastrophiques aussi rares soient-elles.
  \item Le \emph{File List Browser} est un programme pour parcourir le
système de fichiers. Il vous permet aussi d'insérer du code
source depuis le système de fichiers via \menu{fileIn}.
  \item Dans le cas où votre image plante~\footnote{Nous parlons de \emph{crash}, en anglais.} avant que
vous l'ayez sauvegardée ou que vous ayez enregistré le code source
avec Monticello, vous pouvez toujours retrouver vos modifications les
plus récentes en utilisant un \emph{Change List Browser}.
Vous pouvez alors sélectionner les changements ou \emph{changes} (en anglais) que vous voulez 
reprendre et les charger dans la copie la plus récente de votre image.
\end{itemize}

%=================================================================
\ifx\wholebook\relax\else\end{document}\fi
%=================================================================

%=========================================================
%---------------------------------------------------------
