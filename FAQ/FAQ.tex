% $Author: oscar $
% $Date: 2007-09-23 11:56:47 +0200 (dim, 23 sep 2007) $
% $Revision: 12130 $
% Traduction: Benoît TUDURI 18-10-2007
% Relecture: Martial Boniou - Fri Nov  9  16:49:46 CET 2007
% Relecture: Rene Mages     - Fri Jan  10 22:23:44 CET 2008
% Relecture: Martial Boniou - Wed Jan 30 23:21:34 CET 2008
% adaptation pour Pharo: martial - Fri Sep 11 10:50:59 CEST 2009 from
% $Author: oscar $ % $Date: 2009-09-09 09:57:42 +0200 (Wed, 09 Sep 2009) $ % $Revision: 29006 $
% martial: remplacement de la faq:omnibrowser en faq:packagebrowser pour la selection via "choose new default browser"
% Relecture: Rene Mages     - Thu Aug  12 22:23:44 CET 2010
%=================================================================
\ifx\wholebook\relax\else
% --------------------------------------------
% Lulu:
	\documentclass[a4paper,10pt,twoside]{book}
	\usepackage[
		papersize={6.13in,9.21in},
		hmargin={.75in,.75in},
		vmargin={.75in,1in},
		ignoreheadfoot
	]{geometry}
	\input{../common.tex}
	\pagestyle{headings}
	\setboolean{lulu}{true}
% --------------------------------------------
% A4:
%	\documentclass[a4paper,11pt,twoside]{book}
%	\input{../common.tex}
%	\usepackage{a4wide}
% --------------------------------------------
    \graphicspath{{figures/} {../figures/}}
	\begin{document}
	\renewcommand{\nnbb}[2]{} % Disable editorial comments
	\sloppy
\fi
  
%=================================================================
%\chapter{Frequently Asked Questions}
%\chapter{Questions Fréquemments Posées}
%% note de Martial: le terme FAQ est utilise dans certaines pages; sinon "questions freq. posees" est valable aussi
\chapter{Foire Aux Questions}
\label{cha:faq}

%\on{These should be *real* (not invented) FAQs.
%Here are a few that I have collected.
%Check the ST lecture notes for more FAQs.
%We should also try to mine more from newbies.}
\on{Ceci devrait être une *vraie* FAQs.
Celle-ci contient quelques questions que j'ai collectées.
Pour plus de FAQ, vous pouvez consulter les références de lectures sur ST.
Nous devrions essayer de nous mêmes pour les newbies...}

%=================================================================
%\section{Getting started}
\section{Prémisses}
\begin{faq}
%Where do I get the latest Squeak?
Où puis-je trouver la dernière version de Pharo?
\end{faq}
\answer
\pharoweb
\index{download}

\begin{faq}
%Which \pharo image should I use with this book?
Quelle image de \pharo devrais-je utiliser avec ce livre?
\end{faq}
\answer
<<<<<<< .mine
\arelire{Vous pouvez utiliser n'importe quelle image \pharo mais nous vous
recommandons d'utiliser l'image préparée sur le site web de \pharo Par
l'Exemple: {\ppe}.
Vous risquez d'avoir des comportements surprenants lors de la saisie
=======
Vous pouvez utiliser n'importe quelle image de \pharo mais nous vous
recommandons d'utiliser l'image préparée sur le site web de \pharo par
l'exemple: {\ppe}.
Vous risqueriez d'avoir des comportements surprenants lors de la saisie
>>>>>>> .r34584
des exercices en utilisant une autre image. % CHANGE

%=================================================================
\section{Collections}

\begin{faq}
%How do I sort an \clsind{OrderedCollection}?
Comment puis-je trier une \clsind{OrderedCollection}~?

\end{faq}
\answer
%Send it the message \mthind{Collection}{asSortedCollection}.
Envoyez le message suivant \mthind{Collection}{asSortedCollection}.

\begin{code}{@TEST}
#(7 2 6 1) asSortedCollection --> a SortedCollection(1 2 6 7)
\end{code}

\begin{faq}
%How do I convert a collection of characters to a \clsind{String}?
Comment puis-je convertir une collection de caractères en une 
%% ajout
chaîne de caractères
\clsind{String}~?
\end{faq}
\answer
\begin{code}{@TEST}
String streamContents: [:str | str nextPutAll: 'hello' asSet] --> 'hleo'
\end{code}

%=================================================================
%\section{Browsing the system}
\section{Naviguer dans le système}

% voir \faqref{packagebrowser}
\begin{faq}
Le navigateur de classes ne ressemble pas à celui décrit dans le
livre. Que se passe-t-il?
\end{faq}
\answer
Vous utilisez probablement une image disposant d'une version
différente d'\ind{OmniBrowser}
(abrégé en OB) % ajout: martial: Fri Sep 11 11:41:49 CEST 2009
installé comme Browser par défaut.
Dans ce livre, nous présumons que le navigateur Omnibrowser
\emph{Package Browser} (navigateur par paquetages) est installé par
défaut.
Vous pouvez changer cela en \clickant sur la bulle grise (le bouton si vous voulez...)
à droite de la barre de titre du navigateur, puis en sélectionnant
dans le menu du Browser ``Choose new default Browser'' (en français,
\emph{choisissez le nouveau Browser par défaut}). Dans la liste des
navigateurs proposés, \clickz sur O2PackageBrowserAdaptor. 
Le prochain navigateur de classes que vous ouvrirez sera le Package Browser. %CHANGE

\begin{figure}[tbh]
	\centering
	\subfigure[Choisir un nouveau Browser.]{\figlabel{chooseNewBrowser}
		\includegraphics[width=0.45\linewidth]{chooseNewBrowser}}\hfill
	\subfigure[Sélectionnez l'OB Package Browser]{\figlabel{O2PackageBrowserAdaptor}
		\includegraphics[width=0.45\linewidth]{O2PackageBrowserAdaptor}}\hfill
	\caption{Changer le navigateur de classes par défaut.}
\end{figure}
\seeindex{halo}{Morphic} % \seeindex{morphic halo}{Morphic} dans PBE

\begin{faq}
%How to I search for a class?
Comment puis-je chercher une classe~?
\end{faq}
\answer
%\short{b} (browse) on the class name, or \short{f} in the category pane of the class browser.
\short{b} (pour \emph{browse} \cad parcourir à l'aide du navigateur
de classes) sur le nom de la classe ou \short{f} dans le panneau des
catégories du Browser. % REVOIR
\index{raccourci-clavier!browse it}
\index{raccourci-clavier!find...}

\begin{faq}
%How do I find/browse all sends to super?
Comment puis-je trouver/naviguer dans tous les envois 
%%de
à
 super~?
\end{faq}
\answer
%The second solution is much faster:
La deuxième solution est la plus rapide:
\begin{code}{}
SystemNavigation default browseMethodsWithSourceString: 'super'.
SystemNavigation default browseAllSelect: [:method | method sendsToSuper ].
\end{code}
%\index{browsing programmatically}
\index{naviguer de manière pragmatique}
\clsindex{SystemNavigation}

\begin{faq}
%How do I browse all super \subind{super}{send}{}s within a hierarchy?
Comment puis-je naviguer au travers de tous les 
\subpvind{super}{envoi}{}s de messages à \ct{super} dans une hiérarchie~?
\end{faq}
\answer
\begin{code}{}
browseSuperSends:= [:aClass | SystemNavigation default
	browseMessageList: (aClass withAllSubclasses gather: [:each |
		(each methodDict associations
			select: [:assoc | assoc value sendsToSuper ])
				collect: [:assoc | MethodReference class: each selector: assoc key ] ])
	name: 'Supersends of ' , aClass name , ' and its subclasses'].
browseSuperSends value: OrderedCollection.
\end{code}

\begin{faq}
%How do I find out which are the new methods implemented in a class?
Comment puis-je découvrir quelles sont les nouvelles méthodes implémentées dans une classe?
\end{faq}
\answer
%Here we ask which new methods are introduced by \ct{True}:
Dans le cas présent nous demandons quelles sont les nouvelles méthodes introduites par \ct{True}:
\begin{code}{@TEST | newMethods |}
newMethods:= [:aClass| aClass methodDict keys select:
	[:aMethod | (aClass superclass canUnderstand: aMethod) not ]].
newMethods value: True --> an IdentitySet(#asBit)
\end{code}

\begin{faq}
%How do I tell which methods of a class are abstract?
Comment puis-je 
%rapporter
trouver
les méthodes d'une classe qui sont abstraites~?
\end{faq}
\answer
\begin{code}{@TEST | abstractMethods |}
abstractMethods:=
	[:aClass | aClass methodDict keys select:
		[:aMethod | (aClass>>aMethod) isAbstract ]].
abstractMethods value: Collection --> an IdentitySet(#remove:ifAbsent: #add: #do:)
\end{code}

\begin{faq}
%How do I generate a view of the \ind{AST} of an expression?
Comment puis-je créer une vue de 
%% ajout
l'arbre syntaxique abstrait ou 
\ind{AST} d'une expression~?
\end{faq}
\answer
%Load AST from squeaksource.com. Then evaluate:
Charger le paquetage AST depuis \url{http://squeaksource.com}. Ensuite évaluer:
\begin{code}{}
(RBParser parseExpression: '3+4') explore
\end{code}
%(Alternatively \emph{explore it}.)
(D'autre part \emph{explorer le}.)
\clsindex{RBParser}

\begin{faq}
%How do I find all the Traits in the system?
Comment puis-je trouver tout les \emph{traits} dans le système~?
\end{faq}
\answer
\begin{code}{}
Smalltalk allTraits
\end{code}

\begin{faq}
%How do I find which classes use traits?
Comment puis-je trouver quelles classes utilisent les \emph{traits}~?
\end{faq}
\answer
\begin{code}{}
Smalltalk allClasses select: [:each | each hasTraitComposition ]
\end{code}

%=================================================================
%\section{Using Monticello and SqueakSource}
\section{Utilisation de Monticello et de SqueakSource}

\begin{faq}
%How do I load a \ind{SqueakSource} project?
Comment puis-je charger un projet du \ind{SqueakSource}~?
\index{Monticello}
\end{faq}
\answer
\begin{enumerate}
%  \item Find the project you want in \url{squeaksource.com}
  \item Trouvez le projet que vous souhaitez sur \url{http://squeaksource.com}
%  \item Copy the registration code snippet
  \item Copiez le code d'enregistrement
%  \item Select \menu{open \go Monticello browser}
  \item Sélectionnez \menu{open \go Monticello browser}
%  \item Select \menu{+Repository \go HTTP}
  \item Sélectionnez \menu{+Repository \go HTTP}
%  \item Paste and accept the Registration code snippet; enter your password
  \item Collez et acceptez le code d'enregistrement ; entrez votre mot de passe
%  \item Select the new repository and \menu{Open} it
  \item Sélectionnez le nouveau dépôt et ouvrez-le avec le bouton \menu{Open}
%  \item Select and load the latest version
  \item Sélectionnez et chargez la version la plus récente
\end{enumerate}

\begin{faq}
%How do I create a SqueakSource project?
Comment puis-je créer un projet SqueakSource~?
\end{faq}
\answer
\begin{enumerate}
%  \item Go to \url{http://squeaksource.com}
  \item Allez à \url{http://squeaksource.com}
%  \item Register yourself as a new member
  \item Enregistrez-vous comme un nouveau membre
%  \item Register a project (name = category)
  \item Enregistrez un projet (nom = catégorie)
%  \item Copy the Registration code snippet
  \item Copiez le code d'enregistrement
%  \item \menu{open \go Monticello browser}
  \item \menu{open \go Monticello browser}
%  \item \menu{+Package} to add the category
  \item \menu{+Package} pour ajouter une catégorie
%  \item Select the package
  \item Sélectionnez le package
%  \item \menu{+Repository \go HTTP}
  \item \menu{+Repository \go HTTP}
%  \item Paste and accept the Registration code snippet; enter your password
  \item Collez et acceptez le code d'enregistrement ; entrez votre mot de passe
%  \item \menu{Save} to save the first version
  \item \menu{Save} pour enregistrer la première version
\end{enumerate}

\begin{faq}
%How do I extend \ct{Number} with \ct{Number>>>chf} but have Monticello recognize it as being part of my \ct{Money} project?
Comment puis-je étendre \ct{Number} avec
%% ajout
la méthode \ct{Number>>>chf} 
%% changement
tel que Monticello la reconnaissent comme étant une partie de mon projet \ct{Money}~?
\end{faq}
\answer
%Put it in a method-category named \ct{*Money}.
%% changement
Mettez-la 
dans une catégorie de méthodes nommée \ct{*Money}.
%Monticello gathers all methods that are in other categories named like *package and includes them in your package.
Monticello réunit toutes les méthodes 
dont les noms de catégories 
%% changement: nommées autrement telles que 
ont la forme \emph{*package} et les insére dans votre package.

%=================================================================
%\section{Tools}
\section{Outils}

\begin{faq}
%How do I programmatically open the \ind{SUnit} \ind{TestRunner}?
Comment puis-je ouvrir de manière pragmatique le \ind{SUnit} \ind{TestRunner}~?
\end{faq}
\answer
%Evaluate \ct{TestRunner open}.
Évaluez \ct{TestRunner open}.

\begin{faq}
%Where can I find the \ind{Refactoring Browser}?
Où puis-je trouver le \ind{Refactoring Browser}~?
\end{faq}
\answer
%Load AST then Refactoring Engine from squeaksource.com:
Chargez le paquetage AST puis le moteur de 
%%plutot que refrabrication 
refactorisation sur le site \url{http://squeaksource.com}:\\
\url{http://www.squeaksource.com/AST}\\
\url{http://www.squeaksource.com/RefactoringEngine}

\begin{faq}
%How do I register the browser that I want to be the default?
Comment puis-je enregistrer le navigateur comme navigateur par défaut~?
\end{faq}
\answer
%Click the menu icon in the top left of the Browser window.
\Clickz{} sur l'icône (une bulle grise) du menu situé à droite dans la barre de
titre de la fenêtre du Browser. % REVOIR (erreur dans PBE)
%% à côté de \aretirer{la croix de destruction de la fenêtre}. % martial - ce n'est plus une croix mais une bulle rouge 
%% ajout (pour la traduction)
Choisissez \menu{Register this Browser as default} pour enregistrer le navigateur courant comme navigateur par défaut ou bien, sélectionnez \menu{Choose new default Browser} pour obtenir un menu flottant d'où vous pourrez faire votre choix parmi les différentes classes de Browser.

%=================================================================
%\section{Regular expressions and parsing}
\section{Expressions régulières et analyse grammaticale}

%\begin{faq}
%%How can I work with regular expressions?
%Comment puis-je travailler avec les expressions régulières~?
%\index{paquetage!expressions régulières}
%\end{faq}
%\answer
%Chargez le paquetage de RegEx de Vassili Bykov à l'adresse: \\
%\url{http://www.squeaksource.com/Regex.html}
%\index{Bykov, Vassili}

\begin{faq}
%Where is the documentation for the RegEx package?
Où est la documentation pour le \arevoir{paquetage RegEx~?}
% rene : paquetage VB-Regex ?
\end{faq}
\answer
%Look at the \menu{DOCUMENTATION} protocol of \ct{RxParser class} in the \menu{VB-Regex} category.
%% changement
Regardez dans le protocole \menu{DOCUMENTATION} de \ct{RxParser class} situé dans la catégorie \menu{VB-Regex}.

\begin{faq}
%Are there tools for writing parsers?
Y a-t'il des outils pour l'écriture d'un outil d'analyse grammaticale~?
\end{faq}
\answer
%Use \ind{SmaCC}\,---\,the Smalltalk Compiler Compiler.
Utilisez \ind{SmaCC}\,---\,le compilateur de compilateur (ou générateur de compilateur)~\footnote{En anglais, Compiler-Compiler.} Smalltalk.
%You should install at least SmaCC-lr.13.
Vous devrez installer au moins SmaCC-lr.13.
%Load it from \url{http://www.squeaksource.com/SmaccDevelopment.html}.
Chargez-le depuis \url{http://www.squeaksource.com/SmaccDevelopment.html}.
%There is a nice tutorial online:
Il y a un bon tutoriel en ligne à l'adresse:
\url{http://www.refactory.com/Software/SmaCC/Tutorial.html}

\begin{faq}
%Which packages should I load from SqueakSource SmaccDevelopment to write parsers?
Quels paquetages devrais-je charger depuis \emph{SqueakSource SmaccDevelopment} pour écrire un analyseur grammatical~?
\end{faq}
\answer
%Load the latest version of \ind{SmaCCDev}{}\,---\,the runtime is already there.
Chargez la dernière version de \ind{SmaCCDev}{}\,---\,le lanceur de programme est déjà actif.
(Attention: SmaCC-Development est destiné à la version 3.8 de \squeak)

%=================================================================
\ifx\wholebook\relax\else\end{document}\fi
%=================================================================

%\begin{faq}
%How do I run a headless image with a file argument?
%\end{faq}
%\answer
%Right now you can't do it with the MacOSX VM.

%\begin{faq}
%How do I find out which methods access the Smalltalk dictionary?
%\end{faq}
%\answer
%???

%\begin{faq}
%How do I get the tree view of an AST?
%\end{faq}
%\answer
%???

%\begin{faq}
%How do I browse all references to a given class?
%\end{faq}
%\answer
%???

%%%%%%%%%%%%%%%%%%%%%%%%%%%%%%%%%%%%%

%  Benoît

% Rappel LaTeX:
%  é    \'e
%  è    \`e
%  ê    \^e
%  ë    \"e
%  ç    \c{c}
%  $    \$
%  &    \&
%  %    \%
%  #    \#
%  _    \_
%  {    \{
%  }    \}
%  ~    \~
%  ^    \^
%  \    $\backslash$

% Paramètre VIM 7.1
% pour l'édition de ce document
% :set encoding=UTF-8
%%%%%%%%%%%%%%%%%%%%%%%%%%%%%%%%%%%%%

