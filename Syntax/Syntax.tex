% $Author$
% $Date$
% $Revision: 14633$
% $Id$
% %=================================================================
% translated by Rene Mages squeak@rmages.com start: (Fri, 5 Oct 2007)
% relecture par Rene Mages et Martial Boniou: (Fri, 20 Dec 2007)
% relecture par Rene Mages : (Sun, 13 Jan 2008) de la version #14938
% adaptation pour PBE - martial - Thu Sep 10 22:45:08 CEST 2009 from
% $Id: Syntax.tex 28624 2009-08-27 10:59:05Z oscar $
%=================================================================
\ifx\wholebook\relax\else
% --------------------------------------------
% Lulu:
	\documentclass[a4paper,10pt,twoside]{book}
	\usepackage[
		papersize={6.13in,9.21in},
		hmargin={.75in,.75in},
		vmargin={.75in,1in},
		ignoreheadfoot
	]{geometry}
	\input{../common.tex}
	\pagestyle{headings}
	\setboolean{lulu}{true}
% --------------------------------------------
% A4:
%	\documentclass[a4paper,11pt,twoside]{book}
%	\input{../common.tex}
%	\usepackage{a4wide}
% --------------------------------------------
    \graphicspath{{figures/} {../figures/}}
	\begin{document}
	\renewcommand{\nnbb}[2]{} % Disable editorial comments
	\sloppy
\fi
%=================================================================
\chapter{Un r\'esum\'e de la syntaxe}
\chalabel{syntax}

\sd{We should add pragmas.}
\on{Please do so.}

% \sd{It would be good to add link to the chapter where the reader can learn about conditional, exceptions and loops.}
% \on{There are links already.}

\pharo, comme la plupart des dialectes modernes de \st, adopte une syntaxe proche de celle de \st-80.
La \ind{syntaxe} est con\c{c}ue de telle sorte que le texte d'un
programme lu \`{a} haute voix ressemble \`{a} de l'\emph{English pidgin} ou ``anglais simplifié'':

\begin{code}{}
(Smalltalk includes: Class) ifTrue: [ Transcript show: Class superclass ]
\end{code}

\noindent
La syntaxe de \pharo (\ie les expressions) est minimaliste; pour l'essentiel, con\c{c}ue uniquement pour \emph{envoyer des messages}.
%% et  \emph{des d\'{e}clarations de m\'{e}thodes}.
Les expressions sont construites \`{a} partir d'un nombre tr\`{e}s r\'{e}duit de primitives.
\st dispose seulement de 6 mots-cl\'{e}s et d'aucune syntaxe pour les structures de contr\^{o}le, ni pour les d\'{e}clarations de nouvelles classes.
En revanche, tout ou presque est r\'{e}alisable en envoyant des messages \`{a} des objets.
Par exemple, \`{a} la place de la structure de contr\^{o}le conditionnelle \emph{si-alors-sinon}, \st envoie des messages comme \ct{ifTrue:} \`{a} des \arevoir{objets bool\'eens}.
% rene propose la suppresion de :  de la classe \clsind{Boolean} (ne figure pas dans le texte original).
Les nouvelles \mbox{(sous-)classes} sont cr\'{e}\'{e}es en envoyant un message \`{a} leur super-classe.

%=================================================================
\section{Les \'{e}l\'{e}ments syntaxiques }

Les expressions sont compos\'{e}es des blocs constructeurs suivants:
\begin{enumerate}[label=(\small\itshape\roman{*}), ref=(\small\itshape\roman{*})]
\item six mots-cl\'{e}s r\'{e}serv\'{e}s ou \emph{pseudo-variables}:
\pvind{self}, \pvind{super}, \pvind{nil}, \pvind{true}, \pvind{false}, and \pvind{thisContext};
\item des expressions constantes pour des \emph{objets littéraux} comprenant les nombres, les caract\`{e}res, les chaînes de caract\`{e}res, les symboles et les tableaux;
\item des d\'{e}clarations de variables;
\item des affectations;
\item des \ind{bloc}{}s ou fermetures lexicales -- \emph{block closures} en anglais -- et;
\item des messages.
\end{enumerate}
\index{littéral!objet}
\seeindex{pseudo-variable}{variable, pseudo}
\seeindex{objet!littéral}{littéral, objet}

\begin{table}\centering
	\begin{tabular}{ll}
		\toprule
		Syntaxe & ce qu'elle repr\'{e}sente \\
		\midrule
		\lct{startPoint}			&	un nom de variable\\
		\lct{Transcript}			&	un nom de variable globale\\
		\lct{self}				&	une pseudo-variable \\
		\midrule
		\lct{1}				 	&	un entier decimal \\
		\lct{2r101}				&	un entier binaire \\
		\lct{1.5}					& un nombre flottant \\
		\lct{2.4e7}				&	une notation exponentielle \\
		\lct{\$a}					& le caract\`{e}re `a' \\
		\lct{'Bonjour'}				&	la cha\^{\i}ne ``Bonjour'' \\
		\lct{\#Bonjour}				&	le symbole \lct{\#Bonjour} \\
		\lct{\#(1 2 3)}			&	un tableau de litt\'{e}raux \\
		\lct{\{1. 2. 1+2\}}		&	un tableau dynamique \\
		\midrule
		\lct{"c'est mon commentaire"} 		&	un commentaire  \\
		\midrule
		\lct{| x y |}				&	une d\'eclaration de 2 variables \lct{x} et \lct{y}	\\
		\lct{x := 1}				&	affectation de 1 \`a \lct{x} \\
		\lct{[ x + y ]}			&	un bloc qui \'evalue \lct{x+y} \\
		\lct{<primitive: 1>}		&	une primitive de la VM~\footnote{Machine Virtuelle.} ou annotation\\
		\midrule
		\lct{3 factorial}			&	un message unaire \\
		\lct{3 + 4}					&	un message binaire \\
		\lct{2 raisedTo: 6 modulo: 10}		&	un message \`a mots-cl\'es \\
		\midrule
		\lct{$\uparrow$ true}
 			&	retourne la valeur \lct{true} pour vrai \\
		\lct{Transcript show: 'bonjour'. Transcript cr }		& un
        s\'eparateur d'expression (\lct{.})	\\ 
		\lct{Transcript show: 'bonjour'; cr}	& un message en cascade (\lct{;}) \\
		\bottomrule
	\end{tabular}
\caption{R\'esum\'e de la syntaxe de \pharo \tablabel{syntax}}
\end{table}


Dans \tabref{syntax}, nous pouvons voir des exemples divers d'\'{e}l\'{e}ments syntaxiques.
\begin{description}
\item[Les variables locales.] \ct{startPoint} est un nom de variable ou identifiant.
Par convention, les identifiants sont compos\'{e}s de mots au format
d'écriture \emph{casse chameau} (``\ind{camelCase}''): chaque mot
except\'{e} le premier d\'{e}bute par une lettre majuscule.
La premi\`{e}re lettre d'une variable d'instance, d'une m\'{e}thode ou d'un bloc argument ou d'une variable temporaire doit \^{e}tre en minuscule.
Ce qui indique au lecteur que la port\'{e}e de la variable est priv\'{e}e .

\item[Les variables partag\'{e}es.]	Les identifiants qui d\'{e}butent par
  une lettre majuscule sont des  variables
  \subind{variable}{globale}{s}, des  \subind{classe}{variable}{s} de
  classes, des dictionnaires de \subind{variable}{pool} ou des noms de classes.
\ct{Transcript} est une variable globale, une instance de la classe \ct{TranscriptStream}.
\seeindex{variable globale}{variable, globale}
\seeindex{dictionnaire de pool}{variable, pool}
\seeindex{variable!classe}{classe, variable}

\item[Le receveur.] \pvind{self} est un mot-cl\'{e} qui pointe vers l'objet sur lequel la m\'{e}thode courante s'ex\'{e}cute. Nous le nommons  ``le receveur'' car cet objet devra normalement re\c{c}evoir le message qui provoque l'ex\'{e}cution de la m\'{e}thode.
\self est appel\'{e} une ``\subind{variable}{pseudo-variable}'' puisque nous ne pouvons rien lui affecter.

\item[Les entiers.] En plus des entiers d\'{e}cimaux habituels comme
  \ct{42}, \pharo propose aussi une \ind{notation en base num\'{e}rique}
  ou \emph{radix}.
\ct{2r101} est \ct{101} en base 2 (\ie en binaire), qui est \'{e}gal \`{a} l'entier d\'{e}cimal 5.
\index{littéral}
\index{littéral!nombre}

\item[Les nombres flottants.] Ils peuvent \^{e}tre sp\'{e}cifi\'{e}s avec leur \ind{exposant} en base dix: \mbox{\ct{2.4e7}} est $2.4 \times 10^7$.
\index{nombres flottants}

\item[Les caract\`{e}res.] Un signe dollar d\'{e}finit un \subind{littéral}{caractère}: \ct{$a}\ignoredollar$ est le litt\'{e}ral pour `a'.
Des instances de caract\`{e}res non-imprimables peuvent \^{e}tre
obtenues en envoyant des messages ad hoc \`{a} la classe
\clsind{Character}, tel que \ct{Character space} \cmindex{Character class}{space} et \ct{Character tab}\cmindex{Character class}{tab}.
		
\item[Les chaînes de caract\`{e}res.] Les apostrophes sont utilis\'{e}es pour d\'{e}finir un  litt\'{e}ral \subind{littéral}{chaîne}.
Si vous d\'{e}sirez une chaine comportant une apostrophe, il suffira de doubler l'apostrophe, comme dans \ct{'aujourd''hui'}.

\item[Les symboles.] Ils ressemblent \`a des chaînes de caract\`{e}res, en ce sens qu'ils comportent une suite de caract\`{e}res.  
Mais contrairement \`{a} une chaîne, un \subind{littéral}{symbole} doit \^{e}tre globalement unique.
Il y a seulement un objet symbole \ct{#Bonjour} mais il peut y avoir plusieurs objets chaînes de caract\`{e}res ayant la valeur \ct{'Bonjour'}.
\seeindex{\#@{\textsf{\#}}}{littéral, symbole}
\seeindex{symbole!littéral}{littéral, symbole}

% martial (25 dec 2007): nous avons dit 'tableaux \`a la compilation'
% dans d'autres chapitres? 
\item[Les tableaux définis \`{a} la compilation.] Ils sont d\'{e}finis par \ct{#( )}, les objets litt\'{e}raux sont s\'{e}par\'{e}s par des espaces.
À l'int\'{e}rieur des parenth\`{e}ses, tout doit \^{e}tre constant durant la compilation.
Par exemple,  \ct{#(27 #(true false) abc)} est un
\subind{littéral}{tableau} litt\'{e}ral de trois \'{e}l\'{e}ments: l'entier \ct{27}, le tableau \`{a} la compilation contenant deux bool\'{e}ens et le symbole \ct{#abc}.
\seeindex{tableau!littéral}{littéral, tableau}

\item[Les tableaux définis \`{a} l'ex\'{e}cution.] Les accolades \ct|{ }|
  d\'{e}finissent un tableau (\subind{tableau}{dynamique}) \`{a} l'ex\'{e}cution.
Ses \'{e}l\'{e}ments sont des expressions s\'epar\'{e}es par des points.
Ainsi \ct|{ 1. 2. 1+2 }| d\'{e}finit un tableau dont les \'{e}l\'{e}ments sont 1, 2 et le r\'{e}sultat de l'\'{e}valuation de 1+2
(la notation entre accolades est \arelire{particuli\`{e}re \`{a} \pharo et à
\squeak}. % CHANGE
Dans d'autres \st{}s vous devez explicitement construire des tableaux dynamiques).

\item[Les commentaires.] Ils sont encadr\'{e}s par des guillemets.
\ct{"Bonjour le commentaire"} est un \ind{commentaire} et non une
chaîne; donc il est ignor\'{e} par le compilateur de \pharo.
Les commentaires peuvent se r\'{e}partir sur plusieurs lignes.
		
\item[Les d\'{e}finitions des variables locales.] Des barres
  verticales \ct{| |} limitent les
  \subind{variable}{déclaration}{}s d'une ou plusieurs variables
  locales dans une m\'{e}thode (ainsi que dans un bloc).
% \seeindex{\|@{\textsf{\|\|}}}{assignment}
% Can't seem to index or-bars! (special char for index macro)
\seeindex{déclaration de variable}{variable!déclaration}

\item[L'affectation.]	\ct{:=} affecte un objet \`{a} une variable.
%Quelquefois vous verrez \`{a} la place une $\leftarrow$ .
%Malheureusement, tant qu'elle ne sera pas un caract\`{e}re
%\textsc{ASCII}, elle apparaîtra sous la forme d'un signe souligné (en
%anglais, \emph{underscore} \`{a} moins que vous n'utilisiez une fonte 
%sp\'{e}ciale.
%Ainsi, \ct{x := 1} est identique \`{a} \ct{x _ 1} ou \ct{x UNDERSCORE
%1}. Il est préférable d'utiliser  \ct{:=} puisque les autres
%repr\'{e}sentations ont \'{e}t\'{e} déclarées comme obsolètes depuis
%la version 3.9 de \squeak. % CHANGE
\index{affectation}
\seeindex{:=@{\textsf{:=}}}{affectation}
\seeindex{\_@{\textsf{\_}}}{affectation}
%\seeindex{<-@{$\leftarrow$}}{affectation}

\item[Les blocs.] Des crochets \ct{[ ]} définissent un \ind{bloc},
  aussi connu sous le nom de \emph{block closure} ou fermeture lexicale, laquelle est un objet \`{a} part enti\`{e}re repr\'{e}sentant une fonction.
Comme nous le verrons, les blocs peuvent avoir des arguments et des variables locales.
\seeindex{[ ]@{\textsf{[ ]}}}{bloc}
\seeindex{block closure}{bloc}
\seeindex{fermeture lexicale}{bloc}

\item[Les primitives.]	\ct{<primitive: ...>} marque l'invocation
  d'une \ind{primitive} de la VM ou \ind{machine virtuelle}
(\ct{<primitive: 1>} est la primitive de \ct{SmallInteger>>>+}).
Tout code suivant la primitive est ex\'{e}cut\'{e} seulement si la
primitive \'{e}choue.
La m\^{e}me syntaxe est aussi employ\'{e}e pour des annotations de
m\'{e}thode. % CHANGE

\item[Les messages unaires.] Ce sont des simples mots (comme \ct{factorial}) envoy\'{e}s \`{a} un receveur (comme \ct{3}).
\index{message!unaire}
%\seeindex{message unaire}{message, unaire}

\item[Les messages binaires.] Ce sont des op\'{e}rateurs (comme \ct{+}) envoy\'{e}s \`{a} un receveur et ayant un seul argument. Dans \ct{3+4}, le receveur est \ct{3} et l'argument est \ct{4}.
\index{message!binaire}
%\seeindex{message binaire}{message, binaire}

\item[Les messages \`{a} mots-cl\'{e}s.] Ce sont des mots-cl\'{e}s multiples (comme \ct{raisedTo:modulo:}), chacun se terminant par un deux-points (:) et ayant un seul argument. 
Dans l'expression \ct{2 raisedTo: 6 modulo: 10}, le \emph{sélecteur de message} \ct{raisedTo:modulo:} prend les deux arguments \ct{6} et \ct{10}, chacun suivant le \lct{:}. Nous envoyons le message au receveur \ct{2}.
\index{message!sélecteur}
\index{message!à mots-clés}
%\seeindex{message de mot-cl\'{e}}{message, mot-cl\'{e}}

\item[Le retour d'une m\'{e}thode.] \ct{^} est employ\'{e} pour
  obtenir le \emphind{retour} ou \emph{renvoi} d'une m\'{e}thode.  Il
  vous faut taper \verb|^| pour obtenir le caract\`{e}re \ct{^}.
\md{\ct{^} donne toujours un retour de la m\'{e}thode, m\^{e}me s'il est utilis\'{e} dans un bloc, il donnera le retour de la m\'{e}thode inser\'{e}e dans le bloc.}

\item[Les séquences d'instructions.]	Un point (\ct{.}) est le
  \emphsubind{instruction}{séparateur} \emph{d'instructions}. Placer un point entre deux expressions les transforme en deux instructions ind\'{e}pendantes.	
\index{instruction!séquence}
\seeindex{instruction!séparateur}{expression, séparateur}
\seeindex{point}{expression, séparateur}
\seeindex{\ct{.}}{expression, séparateur}

\item[Les cascades.] un point virgule peut \^{e}tre utilis\'{e} pour
  envoyer une \emphind{cascade} de messages \`{a} un receveur
  unique. Dans \ct{Transcript show: 'bonjour'; cr}, nous envoyons
  d'abord le message à mots-cl\'{e}s \ct{show: 'bonjour'} au receveur  \ct{Transcript}, puis nous envoyons au m\^{e}me receveur le message unaire \ct{cr}.
	\seeindex{;}{cascade}

\end{description}

Les classes \ct{Number}, \ct{Character}, \ct{String} et \ct{Boolean} sont d\'{e}crites avec plus de d\'{e}tails dans \charef{basic}.
\on{Blocks are described in \charef{blocks}. (Control flow and Iterators).}

%=================================================================
\section{Les pseudo-variables}

Dans \st, il y a 6 mots-cl\'{e}s r\'{e}serv\'{e}s ou  \emph{pseudo-variables}:
\pvind{nil}, \pvind{true},  \pvind{false},  \pvind{self},
\pvind{super} et \pvind{thisContext}.
Ils sont appel\'{e}s \subind{variable}{pseudo-variable}{s} car ils sont pr\'{e}d\'{e}finis et ne peuvent pas \^{e}tre l'objet d'une affectation.
\ct{true}, \ct{false} et \ct{nil} sont des constantes tandis que les valeurs de \ct{self}, \ct{super} et de \ct{thisContext} varient de fa\c{c}on dynamique lorsque le code est ex\'{e}cut\'{e}. 

\ct{true} et \ct{false} sont les uniques instances des classes
\clsind{Boolean}: \clsind{True} et \clsind{False} (voir \charef{basic} 
pour plus de d\'{e}tails).

\pvind{self} se r\'{e}f\`{e}re toujours au receveur de la m\'{e}thode en cours d'ex\'{e}cution.
\ct{super} se r\'{e}f\`{e}re aussi au receveur de la m\'{e}thode en
cours, mais quand vous envoyez un message \`{a} \super, la recherche
de m\'{e}thode change en d\'{e}marrant de la super-classe relative \`{a} la classe contenant la m\'{e}thode qui utilise \ct{super}
(pour plus de d\'{e}tails, voyez \charef{model}).

\ct{nil} est l'objet non d\'{e}fini.
C'est l'unique instance de la classe \clsind{UndefinedObject}. 
Les variables d'instance, les variables de classe et les variables locales  sont initialis\'{e}es \`{a} \ct{nil}.

\ct{thisContext} est une pseudo-variable qui repr\'{e}sente la structure du sommet de la pile d'ex\'{e}cution.
En d'autres termes, il repr\'{e}sente le \clsind{MethodContext} ou le \clsind{BlockClosure} en cours d'ex\'{e}cution.
En temps normal, \ct{thisContext} ne doit pas intéresser la plupart
des programmeurs, mais il est essentiel pour impl\'{e}menter des
outils de d\'{e}veloppement tels que le débogueur et il est aussi utilis\'{e} pour g\'{e}rer exceptions et continuations.

%=================================================================
\section{Les envois de messages}

Il y a trois types de messages dans \pharo.
\begin{enumerate}
  \item Les messages \emph{unaires}: messages sans argument.
  \ct{1 factorial} envoie le message  \ct{factorial} \`{a} l'objet \ct{1}.
  \item Les messages \emph{binaires}: messages avec un seul argument.
  	\ct{1 + 2} envoie le message \ct{+} avec l'argument \ct{2} \`{a} l'objet \ct{1}.
  \item Les messages \`{a} \emph{mots-cl\'{e}s}: messages qui comportent un nombre arbitraire d'arguments.
  	\ct{2 raisedTo: 6 modulo: 10} envoie le message comprenant le s\'{e}lecteur 	\ct{raisedTo: modulo:} et les arguments \ct{6} et \ct{10} vers l'objet \ct{2}.
\end{enumerate}

Les s\'{e}lecteurs des messages unaires sont constitu\'{e}s de caract\`{e}res alphanum\'{e}riques et d\'{e}butent par une lettre minuscule. 
\index{message!unaire}

Les s\'{e}lecteurs des messages binaires sont constitu\'{e}s par un ou plusieurs caract\`{e}res de l'ensemble suivant:
\index{message!binaire}
\begin{code}{}
+ - / \ * ~ < > = @ % | & ! ? ,
\end{code}
\noindent
% [\~\!\@\%\&\*\-\+\=\\\|\?\/\>\<\,]
\on{Il semble que 3 caract\`{e}res ou plus fonctionnent bien, mais il n'est pas possible d'avoir plus d'un ``-'' dans un s\'{e}lecteur binaire. Sans doute \`{a} cause d'un conflit avec l'analyseur (parser) des nombres n\'{e}gatifs?}
\ab{Bon; $-$ est \'{e}trange .}

Les s\'{e}lecteurs des messages \`{a} mots-cl\'{e}s sont form\'{e}s d'une suite de mots-cl\'{e}s alphanum\'{e}riques qui commencent par une lettre minuscule et se terminent par \lct{:}.
\index{message!à mots-clés}

Les messages unaires ont la plus haute priorit\'{e}, puis viennent les messages binaires et, pour finir, les messages \`{a} mots-cl\'{e}s; ainsi:
\begin{code}{@TEST}
2 raisedTo: 1 + 3 factorial --> 128
\end{code}
D'abord nous envoyons \ct{factorial} \`{a} \ct{3}, puis nous envoyons \ct{+ 6} \`{a} \ct{1}, et pour finir, nous envoyons \ct{raisedTo: 7} \`{a} \ct{2}.  
Rappelons que nous utilisons la notation \lct{\emph{expression}}\ct{-->}\lct{\emph{result}} pour montrer le r\'{e}sultat de l'\'{e}valuation d'une expression. 

Priorit\'{e} mise \`{a} part, l'\'{e}valuation s'effectue strictement de la gauche vers la droite, donc: 
\begin{code}{@TEST}
1 + 2 * 3 --> 9
\end{code}
et non \ct{7}.
Les parenth\`{e}ses permettent de modifier l'ordre d'une \'{e}valuation:
\begin{code}{@TEST}
1 + (2 * 3) --> 7
\end{code}
Les envois de message peuvent \^{e}tre compos\'{e}s gr\^{a}ce \`{a} des points et des points-virgules. Une suite d'expressions s\'{e}par\'{e}es par des points provoque  l'\'{e}valuation de chaque expression dans la suite comme une \emphind{instruction}, une apr\`{e}s l'autre. 
%\index{s\'{e}parateur!instruction}
\index{expression!séparateur}

\begin{code}{}
Transcript cr.
Transcript show: 'Bonjour le monde'.
Transcript cr
\end{code}

\noindent
Ce code enverra \ct{cr} \`{a} l'objet \glbind{Transcript}, puis
enverra  \ct{show: 'Bonjour le monde'}, et enfin enverra un nouveau \ct{cr}.

Quand une succession de messages doit \^{e}tre envoy\'{e}e \`{a} un \emph{m\^{e}me} receveur, 
ou pour dire les choses plus succinctement en \emphind{cascade}, le receveur est sp\'{e}cifi\'{e} une seule fois et la suite des messages est s\'{e}par\'{e}e par des points-virgules:

\begin{code}{}
Transcript cr;
    show: 'Bonjour le monde';
    cr
\end{code}
Ce code a pr\'{e}cis\'{e}ment le m\^{e}me effet que celui de l'exemple pr\'{e}c\'{e}dent.


%=================================================================
\section{Syntaxe relative aux m\'{e}thodes}
Bien que les expressions peuvent \^{e}tre \'{e}valu\'{e}es n'importe
o\`{u} dans \pharo (par exemple, dans un espace de travail (Workspace),
dans un d\'{e}bogueur (Debugger) ou dans un navigateur de classes), 
les m\'{e}thodes sont en principe d\'{e}finies dans une fen\^{e}tre du
Browser ou du d\'{e}bogueur
% (Methods can also be filed in from an external medium, but this is not the usual way to program in \pharo.)
(\arelire{les m\'{e}thodes peuvent aussi \^{e}tre rentrées} % CHANGE - retrait
                                % de  (par \menu{file in})
depuis une source externe, mais ce n'est pas une fa\c{c}on habituelle de programmer en \pharo).

Les programmes sont d\'{e}velopp\'{e}s, une m\'{e}thode \`{a} la fois,
dans l'environnement d'une classe pr\'{e}cise (une classe est d\'{e}finie en envoyant un message \`{a} une classe existante, en demandant de cr\'{e}er une sous-classe, de sorte qu'il n'y ait pas de syntaxe sp\'{e}cifique pour cr\'{e}er une classe).

Voil\`{a} la m\'{e}thode \mthind{String}{lineCount} (pour compter le
nombre de lignes) dans la classe  \clsind{String}.
La convention habituelle consiste à se ref\'{e}rer aux m\'{e}thodes
comme suit: \ct{ClassName>>>methodName}; ainsi nous nommerons cette
m\'{e}thode \ct{String>>>lineCount}~\footnote{Le commentaire de la
  m\'ethode dit: 
``Retourne le nombre de lignes représentées par le receveur, dans
    lequel chaque cr ajoute une ligne''}.

\needlines{9}
\begin{method}[lineCount]{Compteur de lignes}
String>>>lineCount
   "Answer the number of lines represented by the receiver,
   where every cr adds one line."
   | cr count |
   cr := Character cr.
   count := 1  min: self size.
   self do:
      [:c | c == cr ifTrue: [count := count + 1]].
   ^ count
\end{method}

Sur le plan de la syntaxe, une m\'{e}thode comporte:
\begin{enumerate}
  \item la structure de la m\'{e}thode avec le nom (\ie \ct{lineCount}) et tous les arguments (aucun dans cet exemple);
  \item les commentaires (qui peuvent \^{e}tre plac\'{e}s n'importe
    o\`{u}, mais conventionnellement, un commentaire doit \^{e}tre plac\'{e} au d\'{e}but afin d'expliquer le but de la m\'{e}thode);
  \item les d\'{e}clarations des variables locales (\ie \ct{cr} et
    \ct{count}); 
  \item un nombre quelconque d'expressions separ\'{e}es par des points; dans notre exemple, il y en a quatre.
\end{enumerate}

L'\'{e}valuation de n'importe quelle expression pr\'{e}c\'{e}d\'{e}e
par un \ct{^} (saisi en tapant \verb|^|) provoquera l'arr\^{e}t de la m\'{e}thode \`{a} cet endroit, donnant en retour la valeur de cette expression.
Une m\'{e}thode qui se termine sans retourner explicitement une expression retournera de fa\c{c}on implicite \pvind{self}.
\index{retour!implicite}


Les arguments et les variables locales doivent toujours d\'{e}buter par une lettre minuscule.
Les noms d\'{e}butant par une majuscule sont r\'{e}servés aux variables globales.
Les noms des classes, comme par exemple \ct{Character}, sont tout
simplement des variables globales qui se réfèrent à l'objet repr\'{e}sentant cette classe.

%=================================================================
\section{La syntaxe des blocs}

Les blocs apportent un moyen de diff\'{e}rer l'\'{e}valuation d'une expression.
Un \ind{bloc} est essentiellement une fonction anonyme. Un bloc est \'{e}valu\'{e} en lui envoyant le message \mthind{BlockClosure}{value}.
Le bloc retourne la valeur de la derni\`{e}re expression de son corps,
\`{a} moins qu'il y ait un retour explicite (avec \ct{^}) auquel cas il ne retourne aucune valeur.
\seeindex{value}{BlockClosure}
% Serge : je ne comprends pas pourquoi cela ne retourne rien ...
% Martial : j'ai traduit tel quel et je ne comprends pas non plus

\begin{code}{@TEST}
[ 1 + 2 ] value --> 3
\end{code}

Les blocs peuvent prendre des param\`etres, chacun doit \^etre
d\'eclar\'e en le précédant d'un deux-points.
Une barre verticale s\'{e}pare les d\'{e}clarations des param\`{e}tres
et le corps du bloc.
Pour \'evaluer un bloc avec un param\`{e}tre, vous devez lui envoyer le message 
 \mthind{BlockClosure}{value:} avec un argument.
Un bloc \`{a} deux param\`etres doit recevoir  \mthind{BlockClosure}{value:value:}; et ainsi de suite, jusqu'\`a 4 arguments.

\begin{code}{@TEST}
[ :x | 1 + x ] value: 2 --> 3
[ :x :y | x + y ] value: 1 value: 2 --> 3
\end{code}

Si vous avez un bloc comportant plus de quatre param\`{e}tres, vous devez utiliser
\mthind{BlockClosure}{valueWithArguments:} et passer les arguments à
l'aide d'un tableau (un bloc comportant un grand nombre de param\`{e}tres étant souvent r\'{e}v\'{e}lateur d'un probl\`{e}me au niveau de sa conception).

Des blocs peuvent aussi d\'{e}clarer des variables locales, lesquelles seront entour\'{e}es par des barres verticales, tout comme des d\'{e}clarations de variables locales dans une m\'{e}thode.
Les variables locales sont d\'{e}clar\'{e}es apr\`{e}s les éventuels
arguments:
\index{variable!déclaration}

\begin{code}{@TEST}
[ :x :y | | z | z := x+ y. z ] value: 1 value: 2 --> 3
\end{code}

% Rene : la traduction de ces deux phrases est à controler (merci).
Les blocs sont en fait des \emph{fermetures} lexicales, puisqu'ils
peuvent faire r\'ef\'erence \`a des variables de leur environnement
imm\'ediat. Le bloc suivant fait r\'ef\'erence \`a la variable \ct{x} voisine:
%Les blocs sont en fait des \emph{fermetures} lexicales, d\`{e}s lors
%qu'ils peuvent se r\'{e}f\'{e}rer \`{a} des variables de
%l'environnement qui l'entoure.
%Le bloc suivant concerne la variable \ct{x} de son environnement englobant:
\newpage
\begin{code}{@TEST}
| x |
x := 1.
[ :y | x + y ] value: 2 --> 3
\end{code}

Les blocs sont des instances de la classe \clsind{BlockClosure}; ce
sont donc des objets, de sorte qu'ils peuvent \^{e}tre affect\'{e}s
\`{a} des variables et \^{e}tre pass\'{e}s comme arguments \`{a}
l'instar de tout autre objet.

% CHANGE - retrait de la note de Mise en garde sur squeak 3.9 et les
% fermetures lexicales

%\paragraph{Really important.} \^\ acts as an escaping mechanism. 
%Return expressions inside a nested block expression will terminate the enclosing method.
%In the example 

%\begin{script}[detect]{...} when the expression \ct{^\ x@y} is executed, the method \ct{detect:}
% escapes the current iteration and returns it. 

%TwoLevelSet>>detect: aBlock

%   firstLevel keysAndValuesDo: [ :x :v |
%      v do: [ :y | (aBlock value: x@y) ifTrue: [^x@y]]
%   ].
%   ^nil
%\end{script}


%=================================================================
\section{Conditions et it\'{e}rations}

\st n'offre aucune syntaxe sp\'{e}cifique pour les structures de contr\^{o}le.
Typiquement celles-ci sont obtenues par l'envoi de messages \`{a} des bool\'{e}ens, des nombres ou des collections, avec pour arguments des blocs.

Les clauses conditionnelles sont obtenues par l'envoi des messages
\mthind{Boolean}{ifTrue:}, \mthind{Boolean}{ifFalse:} ou
\mthind{Boolean}{ifTrue:ifFalse:} au r\'{e}sultat d'une expression
bool\'{e}enne. Pour plus de détails sur les booléens, lisez \charef{basic}.

\begin{code}{}
(17 * 13 > 220)
   ifTrue: [ 'plus grand' ]
   ifFalse: [ 'plus petit' ] --> 'plus grand'
\end{code}
% ON: Not a test.
% My regex approach cannot handle multi-line expressions :-(

Les boucles (ou it\'{e}rations) sont obtenues typiquement par l'envoi de messages \`{a} des blocs, des entiers ou des collections.
Comme la condition de sortie d'une boucle peut \^{e}tre \'{e}valu\'{e}e de fa\c{c}on r\'{e}p\'{e}titive, elle se pr\'{e}sentera sous la forme d'un bloc plut\^{o}t que de celle d'une valeur bool\'{e}enne.
Voici pr\'{e}cis\'{e}ment un exemple d'une boucle proc\'{e}durale:
\index{itération}
%\index{itération|voir{Collection, itération}}
\seeindex{Collection, itération}{itération}
%seealso
\seeindex{boucle}{itération}
\seeindex{énumération}{itération}
\index{clause conditionnelle}

\begin{code}{@TEST | n |}
n := 1.
[ n < 1000 ] whileTrue: [ n := n*2 ].
n --> 1024
\end{code}
\cmindex{BlockClosure}{whileTrue:}

\noindent
\mthind{BlockClosure}{whileFalse:} inverse la condition de sortie.

\begin{code}{@TEST | n |}
n := 1.
[ n > 1000 ] whileFalse: [ n := n*2 ].
n --> 1024
\end{code}

\noindent
\mthind{Integer}{timesRepeat:} offre un moyen simple pour impl\'{e}menter un nombre donn\'{e} d'it\'{e}rations:
\begin{code}{@TEST | n |}
n := 1.
10 timesRepeat: [ n := n*2 ].
n --> 1024
\end{code}
% mark
Nous pouvons aussi envoyer le message \mthind{Number}{to:do:} \`{a} un
nombre qui deviendra alors la valeur initiale d'un compteur de boucle.
Le premier argument est la borne sup\'{e}rieure; le second est un bloc qui prend la valeur courante du compteur de boucle comme argument:

\needlines{4}
\begin{code}{@TEST | n |}
n := 0.
1 to: 10 do: [ :counter | n := n + counter ].
n --> 55
\end{code}

\paragraph{It\'erateurs d'ordre sup\'erieur.}

Les collections comprennent un grand nombre de classes diff\'{e}rentes
dont beaucoup acceptent le m\^{e}me protocole.
Les messages les plus importants pour it\'{e}rer sur des collections 
sont 
\mthind{Collection}{do:}, \mthind{Collection}{collect:}, \mthind{Collection}{select:}, \mthind{Collection}{reject:}, \mthind{Collection}{detect:} ainsi que  \mthind{Collection}{inject:into:}.
Ces messages d\'{e}finissent des it\'{e}rateurs d'ordre sup\'erieur qui nous permettent d'\'{e}crire du code tr\`{e}s compact.

Une instance \clsind{Interval} (\ie un intervalle) est une collection qui d\'{e}finit un it\'{e}rateur sur une suite de nombres depuis un d\'{e}but jusqu'\`{a} une fin.
\ct{1 to: 10} repr\'{e}sente l'intervalle de 1 \`{a} 10.
Comme il s'agit d'une collection, nous pouvons lui envoyer le message \ct{do:}.
L'argument est un bloc qui est \'{e}valu\'{e} pour chaque \'{e}l\'{e}ment de la collection.

\begin{code}{@TEST | n |}
n := 0.
(1 to: 10) do: [ :element | n := n + element ].
n --> 55
\end{code}

\ct{collect:} construit une nouvelle collection de la m\^{e}me taille, en transformant chaque \'{e}l\'{e}ment.
\begin{code}{@TEST}
(1 to: 10) collect: [ :each | each * each ] --> #(1 4 9 16 25 36 49 64 81 100)
\end{code}

\ct{select:} et \ct{reject:} construisent des collections nouvelles, contenant un sous-ensemble d'\'{e}l\'{e}ments satisfaisant (ou non) la condition du bloc bool\'{e}en.
\ct{detect:} retourne le premier \'{e}l\'{e}ment satisfaisant la condition.
Ne perdez pas de vue que les chaînes sont aussi des collections, ainsi
vous pouvez it\'{e}rer aussi sur tous les caract\`{e}res.
%Martial: ajout par rapport a l'original (a completer et surement a
%changer de place):
La m\'ethode \mthind{Character}{isVowel} renvoie \ct{true} (\ie vrai)
lorsque le receveur-caract\`ere est une \label{def:isVowel}
voyelle~\footnote{Note du traducteur: les voyelles accentu\'ees ne sont
  pas consid\'er\'ees par d\'efaut comme des voyelles; \st-80 a le
  m\^eme d\'efaut que la plupart des langages de programmation n\'es
  dans la culture anglo-saxonne.}.
%note de martial: cette remarque ne concerne que moi. Elle pourrait
%etre enlevee si elle pose probleme; par contre il est bon de dire les
%limites des manipulations de string en ST. Je l'avais mise dans le
%chapitre Collections.tex

\begin{code}{@TEST}
'bonjour Squeak' select: [ :char | char isVowel ] --> 'oouuea'
'bonjour Squeak' reject: [ :char | char isVowel ] --> 'bnjr Sqk'
'bonjour Squeak' detect: [ :char | char isVowel ] --> $o 
\end{code}
%%%$

Finalement, vous devez garder \`{a} l'esprit que les collections
acceptent aussi l'\'equivalent de l'op\'erateur \emph{fold}
issu de la programmation fonctionnelle au travers de 
la m\'{e}thode \ct{inject:into:}.
Cela vous am\`{e}ne \`{a} g\'{e}n\'{e}rer un r\'{e}sultat cumulatif
utilisant une expression qui accepte une valeur initiale puis 
injecte chaque \'{e}l\'{e}ment de la collection.
Les sommes et les produits sont des exemples typiques.
\seeindex{fold}{\ct{Collection>>>inject:into}}

\begin{code}{@TEST}
(1 to: 10) inject: 0 into: [ :sum :each | sum + each ] --> 55
\end{code}

\noindent
Ce code est \'{e}quivalent \`{a} \ct{0+1+2+3+4+5+6+7+8+9+10}.

Pour plus de d\'{e}tails sur les collections et les flux de donn\'ees,
rendez-vous dans \charefs{collections}{streams}.

%=================================================================
\section{Primitives et Pragmas}

En \st, \mantra et tout se passe par l'envoi de messages.
N\'{e}anmoins, \`{a} certains niveaux, ce mod\`ele a ses limites;
%%points nous ``touchons le fond``.
le fonctionnement de certains objets ne peut \^{e}tre achev\'e qu'en
invoquant la \ind{machine virtuelle} et les \ind{primitive}{}s.

Par exemple, les comportements suivantes sont tous impl\'{e}ment\'{e}s 
sous la forme de primitives:
l'allocation de la m\'{e}moire (\mthind{Behavior}{new} et \mthind{Behavior}{new:}),
la manipulation de bits (\mthind{Integer}{bitAnd:},
\mthind{Integer}{bitOr:} et \mthind{Integer}{bitShift:}),
l'arithm\'{e}tique des pointeurs et des entiers (\ct{+}, \ct{-},  \ct{<},  \ct{>}, \ct{*}, \ct{/ }, \ct{=}, \ct{==}\ldots)
et l'acc\`{e}s des tableaux (\mthind{Object}{at:}, \mthind{Object}{at:put:}).
\seeindex{new@{\ct{new}}}{\ct{Behavior>>>new}}

Les primitives sont invoqu\'{e}es avec la syntaxe  \ct{<primitive: aNumber>} (aNumber \'etant un nombre).
Une m\'{e}thode qui invoque une telle primitive peut aussi embarquer
du code \st qui sera \'{e}valu\'{e}  \emph{seulement} en cas d'\'echec
de la primitive.

Examinons le code pour \cmind{SmallInteger}{+}.
Si la primitive \'{e}choue, l'expression \ct{super + aNumber} sera
\'{e}valu\'{e}e et retourn\'{e}e~\footnote{Le commentaire de la
  m\'ethode dit: ``Ajoute le receveur \`a l'argument et r\'epond le
  r\'esultat s'il s'agit d'un entier de classe SmallInteger. \'Echoue
  si l'argument ou le r\'esultat n'est pas un
  SmallInteger. Essentiel Aucune recherche. Voir la documentation de
  la classe Object: \emph{whatIsPrimitive} (qu'est-ce qu'une primitive).''}.

\needlines{6}
\begin{method}[primitive]{Une m\'ethode primitive}
+ aNumber 
  "Primitive. Add the receiver to the argument and answer with the result
  if it is a SmallInteger. Fail if the argument or the result is not a
  SmallInteger  Essential  No Lookup. See Object documentation whatIsAPrimitive."

  <primitive: 1>
  ^ super + aNumber
\end{method}

%The other use of primitives is to optimize some crucial methods. The idea is that the system could work 
%without the primitive but it would be slow. The following method shows that the method \ct{@} is calling the primitive 18. Here the point creation is clearly expressible in \st therefore the code after the primitive is just the creation of a point illustrating what the primitive is actually doing. Note that such a code will be never called except if the primitive would failed which is extremely rare.  

%\begin{method}[xxx]{xxx}
%Integer>>@ y 
%   "Primitive. Answer a Point whose x value is the receiver and whose y 
%   value is the argument. Optional. No Lookup. See Object documentation 
%   whatIsAPrimitive."

%   <primitive: 18>
%   ^Point x: self y: y
%\end{method}


\arelire{Dans \pharo,} la syntaxe avec <....> est aussi utilis\'{e}e
pour les annotations de m\'{e}thode que l'on appelle des
\emph{pragmas}. % REVOIR dans PBE, il y a toujours 'Since \pharo 3.9' 2009-09-10

\sd{we should give an example}\ab{Please do!  Is don't know about these.}

%=================================================================
\section{R\'{e}sum\'{e} du chapitre}

\begin{itemize}

\item	\pharo a (seulement) six mots r\'{e}serv\'{e}s aussi appel\'{e}s
  \textit{pseudo-variables}: \ct{true}, \ct{false}, \ct{nil},
  \ct{self}, \ct{super} et  \ct{thisContext}.

\item	Il y a cinq types d'objets litt\'{e}raux: les nombres (\ct{5},
  \ct{2.5}, \mbox{\ct{1.9e15},} \ct{2r111}), les caract\`{e}res
  (\ct{$a}), %$
les chaînes (\ct{'bonjour'}), les symboles (\ct{#bonjour}) et les tableaux (\ct{#('bonjour' #bonjour)})

\item	Les chaînes sont d\'{e}limit\'{e}es par des apostrophes et les commentaires par des guillemets. Pour obtenir une apostrophe dans une chaîne, il suffit de la doubler.

\item	Contrairement aux chaînes, les symboles sont par essence globalement uniques.

\item	Employez \ct{#( ... )} pour d\'{e}finir un tableau litt\'{e}ral.
		Employez \ct|{ ... }| pour d\'{e}finir un tableau dynamique.
		Sachez que
		\ct{#( 1 + 2 ) size --> 3}, mais que 
		\ct|{ 1 + 2 } size --> 1|

\item	Il y a trois types de messages:
		%martial: c'est plus propre en sous-itemizant:
  \begin{itemize}
\item \emph{unaire}: \eg \ct{1 asString}, \ct{Array new};
\item 		\emph{binaire}: \eg \ct{3 + 4}, \ct{'salut' , ' Squeak'};
\item 		\emph{\`{a} mots-cl\'{e}s}: \eg \ct{'salue' at: 5 put: $t}%$
      \end{itemize}
\item	Un envoi de messages \emph{en cascade}  est une suite de messages envoy\'{e}s \`{a} la m\^{e}me cible, tous s\'{e}par\'{e}s par des \ct{;}:
\ct{OrderedCollection new add: #albert; add: #einstein; size --> 2}

\item	Les variables locales sont d\'{e}clar\'{e}es à l'aide de barres verticales.
		\arelire{Employez} \ct{:=} \arelire{pour les affectations.} %; \ct{_} ou
   %     \ct{UNDERSCORE} marche aussi; tous deux sont abandonnées
   %     depuis la version 3.9 de \squeak. % CHANGE - martial
		\ct{|x| x:=1}

\item	Les expressions sont les messages envoy\'{e}s, les cascades et
  les affectations; parfois regroup\'{e}es avec des parenth\`{e}ses.
		\emph{Les instructions} sont des expressions s\'{e}par\'{e}es par des points.

\item	Les blocs ou fermetures lexicales sont des expressions limit\'{e}es par des crochets.
		Les blocs peuvent prendre des arguments et peuvent contenir
        des variables locales dites aussi \emph{variables temporaires}.
		Les expressions du bloc ne sont \'{e}valu\'{e}es que lorsque
        vous envoyez un message de la forme \ct{value...} avec le bon nombre d'arguments.\\
		\ct{[:x | x + 2] value: 4 --> 6}.

\item	Il n'y a pas de syntaxe particuli\`{e}re pour les structures
  de contr\^{o}le; ce ne sont que des messages qui, sous certaines conditions, \'{e}valuent des blocs.\\
		\ct{(\st includes: Class) ifTrue: [ Transcript show: Class superclass ]}

\end{itemize}

%=================================================================
\ifx\wholebook\relax\else
\end{document}\fi
%=================================================================
%%% Local Variables:
%%% coding: utf-8
%%% mode: latex
%%% TeX-master: t
%%% TeX-PDF-mode: t
%%% ispell-local-dictionary: "english"
%%% End:


