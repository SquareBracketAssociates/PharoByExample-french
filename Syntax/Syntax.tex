% $Author$
% $Date$
% $Revision: 14633$
% $Id$
% %=================================================================
% translated by Rene Mages squeak@rmages.com start: (Fri, 5 Oct 2007)
% relecture par Rene Mages et Martial Boniou: (Fri, 20 Dec 2007)
% relecture par Rene Mages : (Sun, 13 Jan 2008) de la version #14938
% adaptation pour PBE - martial - Thu Sep 10 22:45:08 CEST 2009 
% relecture par Rene Mages (Fri, 8 Jan 2010) de la version #30178
% relecture par Rene Mages (Sat, 26 Jun 2010) de la version #33750
% $Id: Syntax.tex 28624 2009-08-27 10:59:05Z oscar $
% sync: 29170
%=================================================================
\ifx\wholebook\relax\else
% --------------------------------------------
% Lulu:
	\documentclass[a4paper,10pt,twoside]{book}
	\usepackage[
		papersize={6.13in,9.21in},
		hmargin={.75in,.75in},
		vmargin={.75in,1in},
		ignoreheadfoot
	]{geometry}
	\input{../common.tex}
	\pagestyle{headings}
	\setboolean{lulu}{true}
% --------------------------------------------
% A4:
%	\documentclass[a4paper,11pt,twoside]{book}
%	\input{../common.tex}
%	\usepackage{a4wide}
% --------------------------------------------
    \graphicspath{{figures/} {../figures/}}
	\begin{document}
	\renewcommand{\nnbb}[2]{} % Disable editorial comments
	\sloppy
\fi
%=================================================================
\chapter{Un résumé de la syntaxe}
\chalabel{syntax}

\sd{We should add pragmas.}
\on{Please do so.}

% \sd{It would be good to add link to the chapter where the reader can learn about conditional, exceptions and loops.}
% \on{There are links already.}

\pharo, comme la plupart des dialectes modernes de \st, adopte une syntaxe proche de celle de \st-80.
La \ind{syntaxe} est conçue de telle sorte que le texte d'un
programme lu à haute voix ressemble à de l'\emph{English pidgin} ou ``anglais simplifié'':

\begin{code}{}
(Smalltalk includes: Class) ifTrue: [ Transcript show: Class superclass ]
\end{code}

\noindent
La syntaxe de \pharo (\ie les expressions) est minimaliste; pour l'essentiel, conçue uniquement pour \emph{envoyer des messages}.
%% et  \emph{des déclarations de méthodes}.
Les expressions sont construites à partir d'un nombre très réduit de primitives.
\st dispose seulement de 6 mots-clés et d'aucune syntaxe pour les structures de contrôle, ni pour les déclarations de nouvelles classes.
En revanche, tout ou presque est réalisable en envoyant des messages à des objets.
Par exemple, à la place de la structure de contrôle conditionnelle \emph{si-alors-sinon}, \st envoie des messages comme \ct{ifTrue:} à des objets de la classe \clsind{Boolean}.
Les nouvelles \mbox{(sous-)classes} sont créées en envoyant un message à leur super-classe.

%=================================================================
\section{Les éléments syntaxiques }

Les expressions sont composées des blocs constructeurs suivants:
\begin{enumerate}[label=(\small\itshape\roman{*}), ref=(\small\itshape\roman{*})]
\item six mots-clés réservés ou \emph{pseudo-variables}:
\pvind{self}, \pvind{super}, \pvind{nil}, \pvind{true}, \pvind{false}, and \pvind{thisContext};
\item des expressions constantes pour des \emph{objets littéraux} comprenant les nombres, les caractères, les chaînes de caractères, les symboles et les tableaux;
\item des déclarations de variables;
\item des affectations;
\item des \ind{bloc}{}s ou fermetures lexicales -- \emph{block closures} en anglais -- et;
\item des messages.
\end{enumerate}
\index{littéral!objet}
\seeindex{pseudo-variable}{variable, pseudo}
\seeindex{objet!littéral}{littéral, objet}

% martial: affichage du table forcé
\begin{table}[h]\centering
	\begin{tabular}{ll}
		\toprule
		Syntaxe & ce qu'elle représente \\
		\midrule
		\lct{startPoint}			&	un nom de variable\\
		\lct{Transcript}			&	un nom de variable globale\\
		\lct{self}				&	une pseudo-variable \\
		\midrule
		\lct{1}				 	&	un entier décimal \\
		\lct{2r101}				&	un entier binaire \\
		\lct{1.5}					& un nombre flottant \\
		\lct{2.4e7}				&	une notation exponentielle \\
		\lct{\$a}					& le caractère `a' \\
		\lct{'Bonjour'}				&	la chaîne ``Bonjour'' \\
		\lct{\#Bonjour}				&	le symbole \lct{\#Bonjour} \\
		\lct{\#(1 2 3)}			&	un tableau de littéraux \\
		\lct{\{1. 2. 1+2\}}		&	un tableau dynamique \\
		\midrule
		\lct{"c'est mon commentaire"} 		&	un commentaire  \\
		\midrule
		\lct{| x y |}				&	une déclaration de 2 variables \lct{x} et \lct{y}	\\
		\lct{x := 1}				&	affectation de 1 à \lct{x} \\
		\lct{[ x + y ]}			&	un bloc qui évalue \lct{x+y} \\
		\lct{<primitive: 1>}		&	une primitive de la VM~\footnote{Machine Virtuelle.} ou annotation\\
		\midrule
		\lct{3 factorial}			&	un message unaire \\
		\lct{3 + 4}					&	un message binaire \\
		\lct{2 raisedTo: 6 modulo: 10}		&	un message à mots-clés \\
		\midrule
		\lct{$\uparrow$ true}
 			&	retourne la valeur \lct{true} pour vrai \\
		\lct{Transcript show: 'bonjour'. Transcript cr }		& un
        séparateur d'expression (\lct{.})	\\ 
		\lct{Transcript show: 'bonjour'; cr}	& un message en cascade (\lct{;}) \\
		\bottomrule
	\end{tabular}
\caption{Résumé de la syntaxe de \pharo \tablabel{syntax}}
\end{table}

Dans \tabref{syntax}, nous pouvons voir des exemples divers d'éléments syntaxiques.

\begin{description}
\item[Les variables locales.] \ct{startPoint} est un nom de variable ou identifiant.
Par convention, les identifiants sont composés de mots au format
d'écriture \emph{casse chameau} (``\ind{camelCase}''): chaque mot
excepté le premier débute par une lettre majuscule.
La première lettre d'une variable d'instance, d'une méthode ou d'un bloc argument 
ou d'une variable temporaire doit être en minuscule.
Ce qui indique au lecteur que la portée de la variable est privée .

\item[Les variables partagées.]	Les identifiants qui débutent par
  une lettre majuscule sont des  variables
  \subind{variable}{globale}{s}, des  \subind{classe}{variable}{s} de
  classes, des dictionnaires de \subind{variable}{pool} ou des noms de classes.
\ct{Transcript} est une variable globale, une instance de la classe \ct{TranscriptStream}.
\seeindex{variable globale}{variable, globale}
\seeindex{dictionnaire de pool}{variable, pool}
\seeindex{variable!classe}{classe, variable}

\item[Le receveur.] \pvind{self} est un mot-clé qui pointe vers l'objet sur lequel la méthode courante s'exécute. Nous le nommons  ``le receveur'' car cet objet devra normalement reçevoir le message qui provoque l'exécution de la méthode.
\self est appelé une ``\subind{variable}{pseudo-variable}'' puisque nous ne pouvons rien lui affecter.

\item[Les entiers.] En plus des entiers décimaux habituels comme
  \ct{42}, \pharo propose aussi une \ind{notation en base numérique}
  ou \emph{radix}.
\ct{2r101} est \ct{101} en base 2 (\ie en binaire), qui est égal à l'entier décimal 5.
\index{littéral}
\index{littéral!nombre}

\item[Les nombres flottants.] Ils peuvent être spécifiés avec leur \ind{exposant} en base dix: \mbox{\ct{2.4e7}} est $2.4 \times 10^7$.
\index{nombres flottants}

\item[Les caractères.] Un signe dollar définit un \subind{littéral}{caractère}: \ct{$a}\ignoredollar$ est le littéral pour `a'.
Des instances de caractères non-imprimables peuvent être
obtenues en envoyant des messages ad hoc à la classe
\clsind{Character}, tel que \ct{Character space} \cmindex{Character class}{space} et \ct{Character tab}\cmindex{Character class}{tab}.
		
\item[Les chaînes de caractères.] Les apostrophes sont utilisées pour définir un  littéral \subind{littéral}{chaîne}.
Si vous désirez qu'une chaîne comporte une apostrophe, il suffira de doubler l'apostrophe, comme dans \ct{'aujourd''hui'}.

\item[Les symboles.] Ils ressemblent à des chaînes de caractères, en ce sens qu'ils comportent une suite de caractères.  
Mais contrairement à une chaîne, un \subind{littéral}{symbole} doit être globalement unique.
Il y a seulement un objet symbole \ct{#Bonjour} mais il peut y avoir plusieurs objets chaînes de caractères ayant la valeur \ct{'Bonjour'}.
\seeindex{\#@{\textsf{\#}}}{littéral, symbole}
\seeindex{symbole!littéral}{littéral, symbole}

% martial (25 dec 2007): nous avons dit 'tableaux à la compilation'
% dans d'autres chapitres? 
\item[Les tableaux définis à la compilation.] Ils sont définis par \ct{#( )}, les objets littéraux 
sont séparés par des espaces.
À l'intérieur des parenthèses, tout doit être constant durant la compilation.
Par exemple, \ct{#(27 (true false) abc)}~\footnote{Notez que c'est la même chose 
que \ct{#(27 #(true false) #abc)}.} est un \subind{littéral}{tableau} littéral de 
trois éléments: l'entier \ct{27}, le tableau à la compilation contenant deux 
booléens et le symbole \ct{#abc}.

\seeindex{tableau!littéral}{littéral, tableau}

\item[Les tableaux définis à l'exécution.] Les accolades \ct|{ }|
  définissent un tableau (\subind{tableau}{dynamique}) à l'exécution.
Ses éléments sont des expressions séparées par des points.
Ainsi \ct|{ 1. 2. 1+2 }| définit un tableau dont les éléments sont 1, 2 et le résultat de l'évaluation de 1+2
(\arelire{dans le monde de \st, la notation entre accolades est particulière aux dialectes \pharo et \squeak}. % martial: '...aux dialectes \pharo et à \squeak de \st', c'était lourd
Dans d'autres \st{}s vous devez explicitement construire des tableaux dynamiques).
% CHANGE

\item[Les commentaires.] Ils sont encadrés par des guillemets.
\ct{"Bonjour le commentaire"} est un \ind{commentaire} et non une
chaîne; donc il est ignoré par le compilateur de \pharo.
Les commentaires peuvent se répartir sur plusieurs lignes.
		
\item[Les définitions des variables locales.] Des barres
  verticales \ct{| |} limitent les
  \subind{variable}{déclaration}{}s d'une ou plusieurs variables
  locales dans une méthode (ainsi que dans un bloc).
% \seeindex{\|@{\textsf{\|\|}}}{assignment}
% Can't seem to index or-bars! (special char for index macro)
\seeindex{déclaration de variable}{variable!déclaration}

\item[L'affectation.]	\ct{:=} affecte un objet à une variable.
%Quelquefois vous verrez à la place une $\leftarrow$ .
%Malheureusement, tant qu'elle ne sera pas un caractère
%\textsc{ASCII}, elle apparaîtra sous la forme d'un signe souligné (en
%anglais, \emph{underscore} à moins que vous n'utilisiez une fonte 
%spéciale.
%Ainsi, \ct{x := 1} est identique à \ct{x _ 1} ou \ct{x UNDERSCORE
%1}. Il est préférable d'utiliser  \ct{:=} puisque les autres
%représentations ont été déclarées comme obsolètes depuis
%la version 3.9 de \squeak. % CHANGE
\index{affectation}
\seeindex{:=@{\textsf{:=}}}{affectation}
\seeindex{\_@{\textsf{\_}}}{affectation}
%\seeindex{<-@{$\leftarrow$}}{affectation}

\item[Les blocs.] Des crochets \ct{[ ]} définissent un \ind{bloc},
  aussi connu sous le nom de \emph{block closure} ou fermeture lexicale, laquelle est un 
objet à part entière représentant une fonction.
Comme nous le verrons, les blocs peuvent avoir des arguments et des variables locales.
\seeindex{[ ]@{\textsf{[ ]}}}{bloc}
\seeindex{block closure}{bloc}
\seeindex{fermeture lexicale}{bloc}

\item[Les primitives.]	\ct{<primitive: ...>} marque l'invocation
  d'une \ind{primitive} de la VM ou \ind{machine virtuelle}
(\ct{<primitive: 1>} est la primitive de \ct{SmallInteger>>>+}).
Tout code suivant la primitive est exécuté seulement si la
primitive échoue.
La même syntaxe est aussi employée pour des annotations de
méthode. % CHANGE

\item[Les messages unaires.] Ce sont des simples mots (comme \ct{factorial}) envoyés à un receveur (comme \ct{3}).
\index{message!unaire}
%\seeindex{message unaire}{message, unaire}

\item[Les messages binaires.] Ce sont des opérateurs (comme \ct{+}) envoyés à un receveur et ayant un seul argument. Dans \ct{3+4}, le receveur est \ct{3} et l'argument est \ct{4}.
\index{message!binaire}
%\seeindex{message binaire}{message, binaire}

\item[Les messages à mots-clés.] Ce sont des mots-clés multiples (comme \ct{raisedTo:modulo:}), chacun se terminant par un deux-points (:) et ayant un seul argument. 
Dans l'expression \ct{2 raisedTo: 6 modulo: 10}, le \emph{sélecteur de message} \ct{raisedTo:modulo:} prend les deux arguments \ct{6} et \ct{10}, chacun suivant le \lct{:}. Nous envoyons le message au receveur \ct{2}.
\index{message!sélecteur}
\index{message!à mots-clés}
%\seeindex{message de mot-clé}{message, mot-clé}

\item[Le retour d'une méthode.] \ct{^} est employé pour
  obtenir le \emphind{retour} ou \emph{renvoi} d'une méthode.  Il
  vous faut taper \verb|^| pour obtenir le caractère \ct{^}.
\md{\ct{^} donne toujours un retour de la méthode, même s'il est utilisé 
dans un bloc, il donnera le retour de la méthode inserée dans le bloc.}

\item[Les séquences d'instructions.]	Un point (\ct{.}) est le
  \emphsubind{instruction}{séparateur} \emph{d'instructions}. Placer un point entre 
deux expressions les transforme en deux instructions indépendantes.	
\index{instruction!séquence}
\seeindex{instruction!séparateur}{expression, séparateur}
\seeindex{point}{expression, séparateur}
\seeindex{\ct{.}}{expression, séparateur}

\item[Les cascades.] un point virgule peut être utilisé pour
  envoyer une \emphind{cascade} de messages à un receveur
  unique. Dans \ct{Transcript show: 'bonjour'; cr}, nous envoyons
  d'abord le message à mots-clés \ct{show: 'bonjour'} au receveur 
 \ct{Transcript}, puis nous envoyons au même receveur le message unaire \ct{cr}.
	\seeindex{;}{cascade}

\end{description}

Les classes \ct{Number}, \ct{Character}, \ct{String} et \ct{Boolean} sont décrites avec plus de détails dans \charef{basic}.
\on{Blocks are described in \charef{blocks}. (Control flow and Iterators).}

%=================================================================
\section{Les pseudo-variables}

Dans \st, il y a 6 mots-clés réservés ou  \emph{pseudo-variables}:
\mbox{\pvind{nil},} \mbox{\pvind{true},}  \mbox{\pvind{false},}  \mbox{\pvind{self},} \mbox{\pvind{super}} et \mbox{\pvind{thisContext}}.
Ils sont appelés \subind{variable}{pseudo-variable}{s} car ils sont prédéfinis et ne peuvent pas être l'objet d'une affectation.
\ct{true}, \ct{false} et \ct{nil} sont des constantes tandis que les valeurs de \ct{self}, \ct{super} et de \ct{thisContext} varient de façon dynamique lorsque le code est exécuté. 

\ct{true} et \ct{false} sont les uniques instances des classes
\clsind{Boolean}: \clsind{True} et \clsind{False} (voir \charef{basic} 
pour plus de détails).

\pvind{self} se réfère toujours au receveur de la méthode en cours d'exécution.
\ct{super} se réfère aussi au receveur de la méthode en
cours, mais quand vous envoyez un message à \super, la recherche
de méthode change en démarrant de la super-classe relative à la classe contenant la méthode qui utilise \ct{super}
(pour plus de détails, voyez \charef{model}).

\ct{nil} est l'objet non défini.
C'est l'unique instance de la classe \clsind{UndefinedObject}. 
Les variables d'instance, les variables de classe et les variables locales  sont initialisées à \ct{nil}.

\ct{thisContext} est une pseudo-variable qui représente la structure du sommet de la pile d'exécution.
En d'autres termes, il représente le \clsind{MethodContext} ou le \clsind{BlockClosure} en cours d'exécution.
En temps normal, \ct{thisContext} ne doit pas intéresser la plupart
des programmeurs, mais il est essentiel pour implémenter des
outils de développement tels que le débogueur et il est aussi utilisé pour gérer exceptions et continuations.

%=================================================================
\section{Les envois de messages}

Il y a trois types de messages dans \pharo.
\begin{enumerate}
  \item Les messages \emph{unaires}: messages sans argument.
  \ct{1 factorial} envoie le message  \ct{factorial} à l'objet \ct{1}.
  \item Les messages \emph{binaires}: messages avec un seul argument.
  	\ct{1 + 2} envoie le message \ct{+} avec l'argument \ct{2} à l'objet \ct{1}.
  \item Les messages à \emph{mots-clés}: messages qui comportent un nombre arbitraire d'arguments.
  	\ct{2 raisedTo: 6 modulo: 10} envoie le message comprenant le sélecteur \ct{raisedTo:modulo:} et les arguments \ct{6} et \ct{10} vers l'objet \ct{2}.
\end{enumerate}

Les sélecteurs des messages unaires sont constitués de caractères alphanumériques et débutent par une lettre minuscule. 
\index{message!unaire}

Les sélecteurs des messages binaires sont constitués par un ou plusieurs caractères de l'ensemble suivant:
\index{message!binaire}
\begin{code}{}
+ - / \ * ~ < > = @ % | & ! ? ,
\end{code}
\noindent
% [\~\!\@\%\&\*\-\+\=\\\|\?\/\>\<\,]
\on{Il semble que 3 caractères ou plus fonctionnent bien, mais il n'est pas possible d'avoir plus d'un ``-'' dans un sélecteur binaire. Sans doute à cause d'un conflit avec l'analyseur (parser) des nombres négatifs?}
\ab{Bon; $-$ est étrange .}

Les sélecteurs des messages à mots-clés sont formés d'une suite de mots-clés alphanumériques 
qui commencent par une lettre minuscule et se terminent par \lct{:}.
\index{message!à mots-clés}

Les messages unaires ont la plus haute priorité, puis viennent les messages binaires et, 
pour finir, les messages à mots-clés; ainsi:
\begin{code}{@TEST}
2 raisedTo: 1 + 3 factorial --> 128
\end{code}
D'abord nous envoyons \ct{factorial} à \ct{3}, puis nous envoyons \ct{+ 6} à \ct{1}, et pour finir, nous envoyons \ct{raisedTo: 7} à \ct{2}.  
Rappelons que nous utilisons la notation \lct{\emph{expression}}\ct{-->}\lct{\emph{result}} pour montrer le résultat de l'évaluation d'une expression. 

Priorité mise à part, l'évaluation s'effectue strictement de la gauche vers la droite, donc: 
\begin{code}{@TEST}
1 + 2 * 3 --> 9
\end{code}
et non \ct{7}.
Les parenthèses permettent de modifier l'ordre d'une évaluation:
\begin{code}{@TEST}
1 + (2 * 3) --> 7
\end{code}
Les envois de message peuvent être composés grâce à des points et des points-virgules. Une suite d'expressions séparées par des points provoque  l'évaluation de chaque expression dans la suite comme une \emphind{instruction}, une après l'autre. 
%\index{séparateur!instruction}
\index{expression!séparateur}

\begin{code}{}
Transcript cr.
Transcript show: 'Bonjour le monde'.
Transcript cr
\end{code}

\noindent
Ce code enverra \ct{cr} à l'objet \glbind{Transcript}, puis
enverra  \ct{show: 'Bonjour le monde'}, et enfin enverra un nouveau \ct{cr}.

Quand une succession de messages doit être envoyée à un \emph{même} receveur, 
ou pour dire les choses plus succinctement en \emphind{cascade}, le receveur est 
spécifié une seule fois et la suite des messages est séparée par des points-virgules:

\begin{code}{}
Transcript cr;
    show: 'Bonjour le monde';
    cr
\end{code}
Ce code a précisément le même effet que celui de l'exemple précédent.


%=================================================================
\section{Syntaxe relative aux méthodes}
Bien que les expressions peuvent être évaluées n'importe
où dans \pharo (par exemple, dans un espace de travail (Workspace),
dans un débogueur (Debugger) ou dans un navigateur de classes), 
les méthodes sont en principe définies dans une fenêtre du
Browser ou du débogueur
% (Methods can also be filed in from an external medium, but this is not the usual way to program in \pharo.)
les méthodes peuvent aussi être rentrées % CHANGE - retrait
                                         % de  (par \menu{file in})
depuis une source externe, mais ce n'est pas une façon habituelle de programmer en \pharo).

Les programmes sont développés, une méthode à la fois,
dans l'environnement d'une classe précise (une classe est définie en envoyant 
un message à une classe existante, en demandant de créer une sous-classe, de sorte 
qu'il n'y ait pas de syntaxe spécifique pour créer une classe).

Voilà la méthode \mthind{String}{lineCount} (pour compter le
nombre de lignes) dans la classe  \clsind{String}.
La convention habituelle consiste à se reférer aux méthodes
comme suit: \ct{ClassName>>>methodName}; ainsi nous nommerons cette
méthode \ct{String>>>lineCount}~\footnote{Le commentaire de la
  méthode dit: 
``Retourne le nombre de lignes représentées par le receveur, dans
    lequel chaque \lct{cr} ajoute une ligne''}.

\needlines{9}
\begin{method}[lineCount]{Compteur de lignes}
String>>>lineCount
   "Answer the number of lines represented by the receiver,
   where every cr adds one line."
   | cr count |
   cr := Character cr.
   count := 1  min: self size.
   self do:
      [:c | c == cr ifTrue: [count := count + 1]].
   ^ count
\end{method}

Sur le plan de la syntaxe, une méthode comporte:
\begin{enumerate}
  \item la structure de la méthode avec le nom (\ie \ct{lineCount}) et tous les arguments (aucun dans cet exemple);
  \item les commentaires (qui peuvent être placés n'importe
    où, mais conventionnellement, un commentaire doit être placé au début afin d'expliquer le but de la méthode);
  \item les déclarations des variables locales (\ie \ct{cr} et
    \ct{count}); 
  \item un nombre quelconque d'expressions séparées par des points; dans notre exemple, il y en a \aretirer{quatre}.
% Rene : en fait ce n'est pas quatre mais trois ( l'erreur est presente dans l'edition anglaise ); Martial: bien vu! Je passe \arevoir à \aretirer parce que l'erreur est critique.
\end{enumerate}

L'évaluation de n'importe quelle expression précédée
par un \ct{^} (saisi en tapant \verb|^|) provoquera l'arrêt de la méthode à cet endroit, 
donnant en retour la valeur de cette expression.
Une méthode qui se termine sans retourner explicitement une expression retournera de façon implicite \pvind{self}.
\index{retour!implicite}


Les arguments et les variables locales doivent toujours débuter par une lettre minuscule.
Les noms débutant par une majuscule sont réservés aux variables globales.
Les noms des classes, comme par exemple \ct{Character}, sont tout
simplement des variables globales qui se réfèrent à l'objet représentant cette classe.

%=================================================================
\section{La syntaxe des blocs}

Les blocs apportent un moyen de différer l'évaluation d'une expression.
Un \ind{bloc} est essentiellement une fonction anonyme. Un bloc est évalué 
en lui envoyant le message \mthind{BlockClosure}{value}.
Le bloc retourne la valeur de la dernière expression de son corps,
à moins qu'il y ait un retour explicite (avec \ct{^}) auquel cas il ne retourne aucune valeur.
\seeindex{value}{BlockClosure}
% Serge : je ne comprends pas pourquoi cela ne retourne rien ...
% Martial : j'ai traduit tel quel et je ne comprends pas non plus

\begin{code}{@TEST}
[ 1 + 2 ] value --> 3
\end{code}

Les blocs peuvent prendre des paramètres, chacun doit être
déclaré en le précédant d'un deux-points.
Une barre verticale sépare les déclarations des paramètres
et le corps du bloc.
Pour évaluer un bloc avec un paramètre, vous devez lui envoyer le message 
 \mthind{BlockClosure}{value:} avec un argument.
Un bloc à deux paramètres doit recevoir  \mthind{BlockClosure}{value:value:}; et 
ainsi de suite, jusqu'à 4 arguments.

\begin{code}{@TEST}
[ :x | 1 + x ] value: 2 --> 3
[ :x :y | x + y ] value: 1 value: 2 --> 3
\end{code}

Si vous avez un bloc comportant plus de quatre paramètres, vous devez utiliser
\mthind{BlockClosure}{valueWithArguments:} et passer les arguments à
l'aide d'un tableau (un bloc comportant un grand nombre de paramètres étant 
souvent révélateur d'un problème au niveau de sa conception).

Des blocs peuvent aussi déclarer des variables locales, 
lesquelles seront entourées par des barres verticales, tout 
comme des déclarations de variables locales dans une méthode.
Les variables locales sont déclarées après les éventuels
arguments:
\index{variable!déclaration}

\begin{code}{@TEST}
[ :x :y | | z | z := x+ y. z ] value: 1 value: 2 --> 3
\end{code}

Les blocs sont en fait des \emph{fermetures} lexicales, puisqu'ils
peuvent faire référence à des variables de leur environnement
immédiat. Le bloc suivant fait référence à la variable \ct{x} voisine:
%Les blocs sont en fait des \emph{fermetures} lexicales, dès lors
%qu'ils peuvent se référer à des variables de
%l'environnement qui l'entoure.
%Le bloc suivant concerne la variable \ct{x} de son environnement englobant:

\begin{code}{@TEST}
| x |
x := 1.
[ :y | x + y ] value: 2 --> 3
\end{code}

Les blocs sont des instances de la classe \clsind{BlockClosure}; ce
sont donc des objets, de sorte qu'ils peuvent être affectés à des 
variables et être passés comme arguments à l'instar de tout autre objet.

% CHANGE - retrait de la note de Mise en garde sur squeak 3.9 et les
% fermetures lexicales

%\paragraph{Really important.} \^\ acts as an escaping mechanism. 
%Return expressions inside a nested block expression will terminate the enclosing method.
%In the example 

%\begin{script}[detect]{...} when the expression \ct{^\ x@y} is executed, the method \ct{detect:}
% escapes the current iteration and returns it. 

%TwoLevelSet>>detect: aBlock

%   firstLevel keysAndValuesDo: [ :x :v |
%      v do: [ :y | (aBlock value: x@y) ifTrue: [^ x@y]]
%   ].
%   ^ nil
%\end{script}


%=================================================================
\section{Conditions et itérations}

\st n'offre aucune syntaxe spécifique pour les structures de contrôle.
Typiquement celles-ci sont obtenues par l'envoi de messages à des booléens, 
des nombres ou des collections, avec pour arguments des blocs.

Les clauses conditionnelles sont obtenues par l'envoi des messages
\mthind{Boolean}{ifTrue:}, \mthind{Boolean}{ifFalse:} ou
\mthind{Boolean}{ifTrue:ifFalse:} au résultat d'une expression
booléenne. Pour plus de détails sur les booléens, lisez \charef{basic}.

\begin{code}{}
(17 * 13 > 220)
   ifTrue: [ 'plus grand' ]
   ifFalse: [ 'plus petit' ] --> 'plus grand'
\end{code}
% ON: Not a test.
% My regex approach cannot handle multi-line expressions :-(

Les boucles (ou itérations) sont obtenues typiquement par l'envoi de messages à des blocs, des entiers ou des collections.
Comme la condition de sortie d'une boucle peut être évaluée de façon répétitive, elle se présentera sous la forme d'un bloc plutôt que de celle d'une valeur booléenne.
Voici précisément un exemple d'une boucle procédurale:
\index{itération}
%\index{itération|voir{Collection, itération}}
\seeindex{Collection, itération}{itération}
%seealso
\seeindex{boucle}{itération}
\seeindex{énumération}{itération}
\index{clause conditionnelle}

\begin{code}{@TEST | n |}
n := 1.
[ n < 1000 ] whileTrue: [ n := n*2 ].
n --> 1024
\end{code}
\cmindex{BlockClosure}{whileTrue:}

\noindent
\mthind{BlockClosure}{whileFalse:} inverse la condition de sortie.

\begin{code}{@TEST | n |}
n := 1.
[ n > 1000 ] whileFalse: [ n := n*2 ].
n --> 1024
\end{code}

\noindent
\mthind{Integer}{timesRepeat:} offre un moyen simple pour implémenter un nombre donné d'itérations:
\begin{code}{@TEST | n |}
n := 1.
10 timesRepeat: [ n := n*2 ].
n --> 1024
\end{code}
% mark
Nous pouvons aussi envoyer le message \mthind{Number}{to:do:} à un
nombre qui deviendra alors la valeur initiale d'un compteur de boucle.
Le premier argument est la borne supérieure; le second est un bloc qui 
prend la valeur courante du compteur de boucle comme argument:

\needlines{4}
\begin{code}{@TEST | n |}
n := 0.
1 to: 10 do: [ :counter | n := n + counter ].
n --> 55
\end{code}

\paragraph{Itérateurs d'ordre supérieur.}

Les collections comprennent un grand nombre de classes différentes
dont beaucoup acceptent le même protocole.
Les messages les plus importants pour itérer sur des collections sont 
\mthind{Collection}{do:}, \mthind{Collection}{collect:}, \mthind{Collection}{select:}, \mthind{Collection}{reject:}, \mthind{Collection}{detect:} ainsi que  \mthind{Collection}{inject:into:}.
Ces messages définissent des itérateurs d'ordre supérieur qui nous permettent d'écrire du code très compact.

Une instance \clsind{Interval} (\ie un intervalle) est une collection qui 
définit un itérateur sur une suite de nombres depuis un début jusqu'à une fin.
\ct{1 to: 10} représente l'intervalle de 1 à 10.
Comme il s'agit d'une collection, nous pouvons lui envoyer le message \ct{do:}.
L'argument est un bloc qui est évalué pour chaque élément de la collection.

\begin{code}{@TEST | n |}
n := 0.
(1 to: 10) do: [ :element | n := n + element ].
n --> 55
\end{code}

\ct{collect:} construit une nouvelle collection de la même taille, en transformant chaque élément.
\begin{code}{@TEST}
(1 to: 10) collect: [ :each | each * each ] --> #(1 4 9 16 25 36 49 64 81 100)
\end{code}

\ct{select:} et \ct{reject:} construisent des collections nouvelles, 
contenant un sous-ensemble d'éléments satisfaisant (ou non) la condition du bloc booléen.
\ct{detect:} retourne le premier élément satisfaisant la condition.
Ne perdez pas de vue que les chaînes sont aussi des collections, ainsi
vous pouvez itérer aussi sur tous les caractères.
%Martial: ajout par rapport a l'original (a completer et surement a
%changer de place):
La méthode \mthind{Character}{isVowel} renvoie \ct{true} (\ie vrai)
lorsque le receveur-caractère est une \label{def:isVowel}
voyelle~\footnote{Note du traducteur: les voyelles accentuées ne sont
  pas considérées par défaut comme des voyelles; \st-80 a le
  même défaut que la plupart des langages de programmation nés
  dans la culture anglo-saxonne.}.
%note de martial: cette remarque ne concerne que moi. Elle pourrait
%etre enlevee si elle pose probleme; par contre il est bon de dire les
%limites des manipulations de string en ST. Je l'avais mise dans le
%chapitre Collections.tex

\begin{code}{@TEST}
'bonjour Squeak' select: [ :char | char isVowel ] --> 'oouuea'
'bonjour Squeak' reject: [ :char | char isVowel ] --> 'bnjr Sqk'
'bonjour Squeak' detect: [ :char | char isVowel ] --> $o 
\end{code}
%%%$

Finalement, vous devez garder à l'esprit que les collections
acceptent aussi l'équivalent de l'opérateur \emph{fold}
issu de la programmation fonctionnelle au travers de 
la méthode \ct{inject:into:}.
Cela vous amène à générer un résultat cumulatif
utilisant une expression qui accepte une valeur initiale puis 
injecte chaque élément de la collection.
Les sommes et les produits sont des exemples typiques.
\seeindex{fold}{\ct{Collection>>>inject:into}}

\begin{code}{@TEST}
(1 to: 10) inject: 0 into: [ :sum :each | sum + each ] --> 55
\end{code}

\noindent
Ce code est équivalent à \ct{0+1+2+3+4+5+6+7+8+9+10}.

Pour plus de détails sur les collections et les flux de données,
rendez-vous dans \charefs{collections}{streams}.

%=================================================================
\section{Primitives et Pragmas}

En \st, \mantra et tout se passe par l'envoi de messages.
Néanmoins, à certains niveaux, ce modèle a ses limites;
%%points nous ``touchons le fond``.
le fonctionnement de certains objets ne peut être achevé qu'en
invoquant la \ind{machine virtuelle} et les \ind{primitive}{}s.

Par exemple, les comportements suivantes sont tous implémentés 
sous la forme de primitives:
l'allocation de la mémoire (\mthind{Behavior}{new} et \mthind{Behavior}{new:}),
la manipulation de bits (\mthind{Integer}{bitAnd:},
\mthind{Integer}{bitOr:} et \mthind{Integer}{bitShift:}),
l'arithmétique des pointeurs et des entiers (\ct{+}, \ct{-},  \ct{<},  \ct{>}, \ct{*}, \ct{/ }, \ct{=}, \ct{==}\ldots)
et l'accès des tableaux (\mthind{Object}{at:}, \mthind{Object}{at:put:}).
\seeindex{new@{\ct{new}}}{\ct{Behavior>>>new}}

Les primitives sont invoquées avec la syntaxe  \ct{<primitive: aNumber>} (aNumber étant un nombre).
Une méthode qui invoque une telle primitive peut aussi embarquer
du code \st qui sera évalué  \emph{seulement} en cas d'échec de la primitive.

Examinons le code pour \cmind{SmallInteger}{+}.
Si la primitive échoue, l'expression \ct{super + aNumber} sera
évaluée et retournée~\footnote{Le commentaire de la
  méthode dit: ``Ajoute le receveur à l'argument et répond le
  résultat s'il s'agit d'un entier de classe SmallInteger. Échoue
  si l'argument ou le résultat n'est pas un
  SmallInteger. Essentiel Aucune recherche. Voir la documentation de
  la classe Object: \emph{whatIsPrimitive} (qu'est-ce qu'une primitive).''}.

\needlines{6}
\begin{method}[primitive]{Une méthode primitive}
+ aNumber 
  "Primitive. Add the receiver to the argument and answer with the result
  if it is a SmallInteger. Fail if the argument or the result is not a
  SmallInteger  Essential  No Lookup. See Object documentation whatIsAPrimitive."

  <primitive: 1>
  ^ super + aNumber
\end{method}

%The other use of primitives is to optimize some crucial methods. The idea is that the system could work 
%without the primitive but it would be slow. The following method shows that the method \ct{@} is calling the primitive 18. Here the point creation is clearly expressible in \st therefore the code after the primitive is just the creation of a point illustrating what the primitive is actually doing. Note that such a code will be never called except if the primitive would failed which is extremely rare.  

%\begin{method}[xxx]{xxx}
%Integer>>@ y 
%   "Primitive. Answer a Point whose x value is the receiver and whose y 
%   value is the argument. Optional. No Lookup. See Object documentation 
%   whatIsAPrimitive."

%   <primitive: 18>
%   ^ Point x: self y: y
%\end{method}


Dans \pharo,la syntaxe avec <....> est aussi utilisée
pour les annotations de méthode que l'on appelle des
\emph{pragmas}. % REVOIR dans PBE, il y a toujours 'Since \pharo 3.9' 2009-09-10

\sd{we should give an example}\ab{Please do!  Is don't know about these.}

%=================================================================
\section{Résumé du chapitre}

\begin{itemize}

\item	\pharo a (seulement) six mots réservés aussi appelés
  \textit{pseudo-variables}: \ct{true}, \ct{false}, \ct{nil},
  \ct{self}, \ct{super} et  \ct{thisContext}.

\item	Il y a cinq types d'objets littéraux: les nombres (\ct{5},
  \ct{2.5}, \mbox{\ct{1.9e15},} \ct{2r111}), les caractères
  (\ct{$a}), %$
les chaînes (\ct{'bonjour'}), les symboles (\ct{#bonjour}) et les tableaux (\ct{#('bonjour' #bonjour)})

\item	Les chaînes sont délimitées par des apostrophes et les commentaires par des guillemets. Pour obtenir une apostrophe dans une chaîne, il suffit de la doubler.

\item	Contrairement aux chaînes, les symboles sont par essence globalement uniques.

\item	Employez \ct{#( ... )} pour définir un tableau littéral.
		Employez \ct|{ ... }| pour définir un tableau dynamique.
		Sachez que
		\ct{#( 1 + 2 ) size --> 3}, mais que 
		\ct|{ 1 + 2 } size --> 1|

\item	Il y a trois types de messages:
		%martial: c'est plus propre en sous-itemizant:
  \begin{itemize}
\item \emph{unaire}: \eg \ct{1 asString}, \ct{Array new};
\item 		\emph{binaire}: \eg \ct{3 + 4}, \ct{'salut' , ' Squeak'};
\item 		\emph{à mots-clés}: \eg \ct{'salue' at: 5 put: $t}%$
      \end{itemize}
\item	Un envoi de messages \emph{en cascade}  est une suite de messages envoyés à la même cible, tous séparés par des \ct{;}:
\ct{OrderedCollection new add: #albert; add: #einstein; size --> 2}

\item	Les variables locales sont déclarées à l'aide de barres verticales.
        Employez \ct{:=} pour les affectations. 
   %; \ct{_} ou % \ct{UNDERSCORE} marche aussi; tous deux sont abandonnées
   %     depuis la version 3.9 de \squeak. % CHANGE - martial
		\ct{|x| x:=1}

\item	Les expressions sont les messages envoyés, les cascades et
  les affectations; parfois regroupées avec des parenthèses.
		\emph{Les instructions} sont des expressions séparées par des points.

\item	Les blocs ou fermetures lexicales sont des expressions limitées par des crochets.
		Les blocs peuvent prendre des arguments et peuvent contenir
        des variables locales dites aussi \emph{variables temporaires}.
		Les expressions du bloc ne sont évaluées que lorsque
        vous envoyez un message de la forme \ct{value...} avec le bon nombre d'arguments.\\
		\ct{[:x | x + 2] value: 4 --> 6}.

\item	Il n'y a pas de syntaxe particulière pour les structures
  de contrôle; ce ne sont que des messages qui, sous certaines conditions, évaluent des blocs.\\
		\ct{(\st includes: Class) ifTrue: [ Transcript show: Class superclass ]}

\end{itemize}

%=================================================================
\ifx\wholebook\relax\else
\end{document}\fi
%=================================================================
%%% Local Variables:
%%% coding: utf-8
%%% mode: latex
%%% TeX-master: t
%%% TeX-PDF-mode: t
%%% End:


