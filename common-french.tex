% $Author$ Martial
% $Date$ Wed Oct 10 13:34:55 CEST 2007
% $Revision$ source: SBE 12715 
% Last Changed Date: 2007-10-08 21:32:45 +0200 (Mon, 08 Oct 2007)
%=============================================================
% NB: documentclass must be set in main document.
% Allows book to be generated in multiple formats.
%=============================================================
%:Packages
\usepackage[french]{babel}
\usepackage[T1]{fontenc}  %%%%%% really important to get the code directly in the text!
\usepackage{lmodern}
%\usepackage[scaled=0.85]{bookmanx} % needs another scale factor if used with \renewcommand{\sfdefault}{cmbr}
\usepackage{palatino}
\usepackage[scaled=0.85]{helvet}
\usepackage{microtype}
\usepackage{graphicx}
\usepackage{theorem}
\usepackage[utf8]{inputenc}
% ON: pdfsync breaks the use of p{width} for tabular columns!
\ifdefined\usepdfsync\usepackage{pdfsync}\fi % Requires texlive 2007
%=============================================================
%:More packages
%Stef should check which ones are used!
%\usepackage{picinpar}
%\usepackage{layout}
%\usepackage{color}
%\usepackage{enum}
%\usepackage{a4wide}
% \usepackage{fancyhdr}
\usepackage{ifthen}
\usepackage{float}
\usepackage{longtable}
\usepackage{makeidx}
\usepackage[nottoc]{tocbibind}
\usepackage{multicol}
\usepackage{booktabs}	% book-style tables
\usepackage{topcapt}	% enables \topcaption
\usepackage{multirow}
\usepackage{tabularx}
%\usepackage[bottom]{footmisc}
\usepackage{xspace}
\usepackage{alltt}
\usepackage{amssymb,textcomp}
\usepackage[usenames,dvipsnames]{color}
%\usepackage{colortbl}
\usepackage[hang]{subfigure}\makeatletter\def\p@subfigure{\thefigure\,}\makeatother
\usepackage{rotating}
\usepackage{enumitem}	% apb: allows more control over tags in enumerations
\usepackage{verbatim}     % for comment environment
\usepackage{varioref}	% for page references that work
\labelformat{footnote}{\thechapter--#1} % to distinguish citations from jurabib
\usepackage{needspace}
\usepackage{isodateo} % enable \isodate
\usepackage[newparttoc]{titlesec}
\usepackage{titletoc}
\usepackage{eurosym}
\usepackage{wrapfig}

\usepackage[
	super,
	citefull=first,
	authorformat={allreversed,and},
	titleformat={commasep,italic}
]{jurabib} % citations as footnotes
\usepackage[
	colorlinks=true,
	linkcolor=black,
	urlcolor=black,
	citecolor=black
]{hyperref}   % should come last

%=============================================================
%:URL style
\makeatletter

\def\url@leostyle{%
  \@ifundefined{selectfont}{\def\UrlFont{\sf}}{\def\UrlFont{\sffamily}}}
% ajouter par Martial pour \traduit (met une dague dans les \doublebox
\def\thempfootnote{\fnsymbol{mpfootnote}}

\makeatother
% Now actually use the newly defined style.
\urlstyle{leo}
%=============================================================
%:Booleans
\newboolean{lulu}
\setboolean{lulu}{false}
\newcommand{\ifluluelse}[2]{\ifthenelse{\boolean{lulu}}{#1}{#2}}
%=============================================================
%:Names
\newcommand{\SUnit}{SUnit\xspace}
\newcommand{\sunit}{SUnit\xspace}
\newcommand{\xUnit}{$x$Unit\xspace}
\newcommand{\JUnit}{JUnit\xspace}
%\newcommand{\XP}{eXtreme Programming\xspace}
\newcommand{\st}{Smalltalk\xspace}
\newcommand{\pharo}{Pharo\xspace} % utilisé \pharo et non \Pharo
\newcommand{\sqsrc}{SqueakSource\xspace}
\newcommand{\sqmap}{SqueakMap\xspace}
\newcommand{\squeak}{Squeak\xspace}
%\newcommand{\sbe}{\url{scg.unibe.ch/SBE}\xspace}
%\newcommand{\sbe}{\url{squeakbyexample.org}\xspace}
\newcommand{\sbe}{\url{http://SqueakByExample.org}\xspace}
% pharo
\newcommand{\pharoweb}{\url{http://pharo-project.org}\xspace}
\newcommand{\pbe}{\url{http://pharo-project.org/PharoByExample}\xspace}
% pharo-french
\newcommand{\ppe}{\url{http://pharo-project.org/PharoParLExemple}\xspace}
% squeak-fr: adresse de la version francaise
\newcommand{\spe}{\url{http://SqueakByExample.org/fr}\xspace}
\newcommand{\sba}{\url{http://SquareBracketAssociates.org}\xspace}
% squeak-fr: ajout de la \squeakdev pour eviter les problemes de
% changements d'url rencontres dans la VO:
\newcommand{\squeakdev}{\url{http://www.squeaksource.com/ImageForDevelopers}\xspace} %ou
%\newcommand{\squeakdev}{\url{squeak.ofset.org/squeak-dev}\xspace}
\newcommand{\bam}{\lct{Bounc\-ing\-Atoms\-Morph}\xspace} % REVOIR
%=============================================================
%:Markup macros for proof-reading
\usepackage[normalem]{ulem} % for \sout
\usepackage{xcolor}
\newcommand{\ra}{$\rightarrow$}
\newcommand{\ugh}[1]{\textcolor{red}{\uwave{#1}}} % please rephrase
\newcommand{\ins}[1]{\textcolor{blue}{\uline{#1}}} % please insert
\newcommand{\del}[1]{\textcolor{red}{\sout{#1}}} % please delete
\newcommand{\chg}[2]{\textcolor{red}{\sout{#1}}{\ra}\textcolor{blue}{\uline{#2}}} % please change
%=============================================================
%:Editorial comment macros
%\newcommand{\nnbb}[2]{
%    \fbox{\bfseries\sffamily\scriptsize#1}
%    {\sf\small$\blacktriangleright$\textit{#2}$\blacktriangleleft$}
%   }
\newcommand{\yellowbox}[1]{\fcolorbox{gray}{yellow}{\bfseries\sffamily\scriptsize#1}}
\newcommand{\triangles}[1]{{\sf\small$\blacktriangleright$\textit{#1}$\blacktriangleleft$}}
\newcommand{\nnbb}[2]{\yellowbox{#1} \triangles{#2}}
\newcommand{\fix}{\yellowbox{À CORRIGER!}}
\newcommand{\here}{\yellowbox{CONTINUE ICI!}}

% macros éditeurs/traducteurs
\newcommand{\ab}[1]{\nnbb{Andrew}{#1}}
\newcommand{\sd}[1]{\nnbb{St\'{e}f}{#1}}
\newcommand{\md}[1]{\nnbb{Marcus}{#1}}
\newcommand{\on}[1]{\nnbb{Oscar}{#1}}
\newcommand{\damien}[1]{\nnbb{Damien}{#1}}
\newcommand{\lr}[1]{\nnbb{Lukas}{#1}}
\newcommand{\orla}[1]{\nnbb{Orla}{#1}}
\newcommand{\ja}[1]{\nnbb{Jannik}{#1}}
\newcommand{\jr}[1]{\nnbb{Jorge}{#1}}
\newcommand{\fp}[1]{\nnbb{Fabrizio}{#1}}
%\newcommand{\here}{\nnbb{CONTINUE}{HERE}}
%\newcommand{\here}{\nnbb{CONTINUE}{ICI}}

%=============================================================
%:Abbreviation macros
\newcommand{\ie}{\emph{c-\`a-d.}\xspace}
\newcommand{\cad}{\emph{c-\`a-d.}\xspace}
%\newcommand{\eg}{\emph{e.g.},\xspace}
\newcommand{\eg}{\emph{par ex.},\xspace}
\newcommand{\parex}{\emph{par ex.},\xspace}
\newcommand{\etc}{etc\xspace}
%=============================================================
%:Cross reference macros

% [squeak-fr] martial: remarquez les articles devant les noms
\newcommand{\charef}[1]{le chapitre~\ref{cha:#1}\xspace}
% note de martial: utilise dans chapitre Syntax.tex: a redefinir
\newcommand{\charefs}[2]{les chapitres~\ref{cha:#1} et \ref{cha:#2}\xspace}
\newcommand{\secref}[1]{la section~\ref{sec:#1}\xspace}
\newcommand{\figref}[1]{la figure~\ref{fig:#1}\xspace}
\newcommand{\Figref}[1]{La figure~\ref{fig:#1}\xspace}
\newcommand{\appref}[1]{l'annexe~\ref{app:#1}\xspace}
\newcommand{\tabref}[1]{la table~\ref{tab:#1}\xspace}
% defini pour le chapitre Messages.tex
\newcommand{\Tabref}[1]{La table~\ref{tab:#1}\xspace}
\newcommand{\faqref}[1]{la FAQ~\ref{faq:#1}, p.~\pageref{faq:#1}\xspace}

% [pharo] ajout
\newcommand{\chalabel}[1]{\label{cha:#1}}
\newcommand{\seclabel}[1]{\label{sec:#1}}
\newcommand{\figlabel}[1]{\label{fig:#1}}
\newcommand{\tablabel}[1]{\label{tab:#1}}
\newcommand{\rulelabel}[1]{\label{rule:#1}}
\newcommand{\eglabel}[1]{\label{eg:#1}}
\newcommand{\scrlabel}[1]{\label{scr:#1}}
\newcommand{\mthlabel}[1]{\label{mth:#1}}
\newcommand{\clslabel}[1]{\label{cls:#1}}
\newcommand{\faqlabel}[1]{\label{faq:#1}}

% APB: I removed trailing \xspace commands from these macros because
% \xspace mostly doesn't work.  If you want a space after your
% references, type one!
% ON: xspace has always worked just fine for me!  Please leave them in.
%
\newcommand{\ruleref}[1]{\ref{rule:#1}\xspace}
%
\newcommand{\egref}[1]{exemple~\ref{eg:#1}\xspace}
\newcommand{\Egref}[1]{Exemple~\ref{eg:#1}\xspace}
%
\newcommand{\scrref}[1]{script~\ref{scr:#1}\xspace}
\newcommand{\Scrref}[1]{Script~\ref{scr:#1}\xspace}
% t = the
\newcommand{\tscrref}[1]{le script~\ref{scr:#1}\xspace}
\newcommand{\Tscrref}[1]{Le script~\ref{scr:#1}\xspace}
%
\newcommand{\mthref}[1]{m\'ethode~\ref{mth:#1}\xspace}
\newcommand{\mthsref}[1]{m\'ethodes~\ref{mth:#1}\xspace}
\newcommand{\Mthref}[1]{M\'ethode~\ref{mth:#1}\xspace}
\newcommand{\tmthref}[1]{la m\'ethode~\ref{mth:#1}\xspace}
\newcommand{\Tmthref}[1]{La m\'ethode~\ref{mth:#1}\xspace}
%
\newcommand{\clsref}[1]{classe~\ref{cls:#1}\xspace}
\newcommand{\tclsref}[1]{la classe~\ref{cls:#1}\xspace}
\newcommand{\Tclsref}[1]{La classe~\ref{cls:#1}\xspace}
%=============================================================
%:Menu item macro
% for menu items, so we can change our minds on how to print them! (apb)
\definecolor{lightgray}{gray}{0.89}
\newcommand{\menu}[1]{{%
	\setlength{\fboxsep}{0pt}%
	\colorbox{lightgray}{{{\upshape\sffamily\strut \,#1\,}}}}}
\newcommand{\link}[1]{{%
 \fontfamily{lmr}\selectfont
  \upshape{\stfamily \underline{#1}}}}
% \newcommand{\menu}[1]{{%
% 	\fontfamily{lmr}\selectfont
% 	\upshape\textlangle{\sffamily #1}\textrangle}}
% For submenu items:
\newcommand{\go}{\,$\triangleright$\,}
% \newcommand{\go}{\,$\blacktriangleright$\,}
% For keyboard shortcuts:
%\newcommand{\short}[1]{\mbox{$\langle${\sc CMD}$\rangle$-#1}\xspace}
\newcommand{\short}[1]{\mbox{{\sc cmd}\hspace{0.08em}--\hspace{0.09em}#1}\xspace}
% For buttons:
\newcommand{\button}[1]{{%
	\setlength{\fboxsep}{0pt}%
	\fbox{{\upshape\sffamily\strut \,#1\,}}}}
\newcommand{\toolsflap}{l'onglet \textit{Tools}\xspace}
%=============================================================
%:Mouse clicks % REVOIR * % CHANGE
% [martial: ce sont des verbes] ==BOUTONS==
\newcommand{\btnclick}{clic\xspace} % inutilisé
\newcommand{\btnactclick}{clic d'action\xspace} % inutilisé
\newcommand{\btnmetaclick}{meta-clic\xspace} % inutilisé
\newcommand{\click}{cliquer\xspace} % RED = click
\newcommand{\actclick}{cliquer avec le bouton d'action\xspace} % YELLOW = action-click
\newcommand{\metaclick}{meta-cliquer\xspace} % BLUE = meta-click
\newcommand{\Click}{Cliquer\xspace} % RED = click
\newcommand{\Actclick}{Cliquer avec le bouton d'action\xspace} % YELLOW = action-click
\newcommand{\Metaclick}{Meta-cliquer\xspace} % BLUE = meta-click
\newcommand{\clickant}{cliquant\xspace} % RED = click
\newcommand{\actclickant}{cliquant avec le bouton d'action\xspace} % YELLOW = action-click
\newcommand{\metaclickant}{Meta-cliquer\xspace} % BLUE = meta-click
\newcommand{\clickz}{cliquez\xspace} % RED = click
\newcommand{\actclickz}{cliquez avec le bouton d'action\xspace} % YELLOW = action-click
\newcommand{\metaclickz}{meta-cliquez\xspace} % BLUE = meta-click
\newcommand{\Clickz}{Cliquez\xspace} % RED = click
\newcommand{\Actclickz}{Cliquez avec le bouton d'action\xspace} % YELLOW = action-click
\newcommand{\Metaclickz}{Meta-cliquez\xspace} % BLUE = meta-click
%=============================================================
%:ToSh macros
\newboolean{tosh}
\setboolean{tosh}{false}
\newcommand{\iftoshelse}[2]{\ifthenelse{\boolean{tosh}}{#1}{#2}}
%=============================================================
%:ToSh colors
%\newcommand{\highlightcolor}{\color{blue!65}}
%\newcommand{\boxcolor}{\color{gray!25}}
\newcommand{\highlight}[1]{\textcolor{blue!65}{#1}}
%\newcommand{\codecolor}{\color{blue!65}}
%%\setlength{\fboxrule}{2pt}
%\newcommand{\asPict}[1]{%
%	{\Large\highlight{#1}}}
%=============================================================
%:Reader cues (do this)
%
% Indicate something the reader should try out.
% \newcommand{\dothisicon}{\raisebox{-.5ex}{\includegraphics[width=1.4em]{squeak-logo}}}
\iftoshelse{
	\usepackage{marginnote}
		\renewcommand*{\marginfont}{\footnotesize}
	\newcommand{\vartriangleout}{\ifthenelse{\isodd{\thepage}}{\vartriangleright}{\vartriangleleft}}
	\newcommand{\dothisicon}{\fcolorbox{blue!65}{white}{\highlight{$\vartriangleout$}}}
	\newcommand{\dothis}[1]{%
		\noindent\par\noindent
		{\reversemarginpar
			\marginnote{\fcolorbox{blue!65}{white}{\highlight{$\vartriangleout$}}}}
		%\MarginLabel{do this}
		\noindent\emph{#1}
		\nopagebreak}
}{
	\newcommand{\dothisicon}{\raisebox{-.5ex}{\includegraphics[height=1.2em]{pharo}}}
	\newcommand{\dothis}[1]{%
		\medskip
		\noindent\dothisicon
		\ifx#1\empty\else\quad\emph{#1}\fi
		\par\smallskip\nopagebreak}
}
%===> NEW VERSION <===
% NB: To use this in an individual chapter, you must set:
%\graphicspath{{figures/} {../figures/}}
% at the head of the chapter.  Don't forget the final /
%=============================================================
%:Reader hints (hint)
%
% Indicates a non-obvious consequence 
\newcommand{\hint}[1]{\vspace{1ex}\noindent\fbox{\textsc{Astuce}} \emph{#1}}
%=================================================================
% graphics for Morphic handles
\newcommand{\grabHandle}{\raisebox{-0.2ex}{\includegraphics[width=1em]{blackHandle}}}
\newcommand{\moveHandle}{\raisebox{-0.2ex}{\includegraphics[width=1em]{moveHandle}}}
\newcommand{\debugHandle}{\raisebox{-0.2ex}{\includegraphics[width=1em]{debugHandle}}}
% squeak-fr (added for Morphic handles)
\newcommand{\rotateHandle}{\raisebox{-0.2ex}{\includegraphics[width=1em]{rotateHandle}}}
\newcommand{\viewerHandle}{\raisebox{-0.2ex}{\includegraphics[width=1em]{viewerHandle}}} % A RETIRER (les eToys ne sont plus)
% squeak-fr (add cloverHandle to use \clover in QuickTour.tex as alias
% todo 

%=============================================================
%:Highlighting Important stuff (doublebox)
%
% From Seaside book ...
\newsavebox{\SavedText}
\newlength{\InnerBoxRule}\setlength{\InnerBoxRule}{.75\fboxrule}
\newlength{\OuterBoxRule}\setlength{\OuterBoxRule}{1.5\fboxrule}
\newlength{\BoxSeparation}\setlength{\BoxSeparation}{1.5\fboxrule}
\addtolength{\BoxSeparation}{.5pt}
\newlength{\SaveBoxSep}\setlength{\SaveBoxSep}{2\fboxsep}
%
\newenvironment{doublebox}{\begin{lrbox}{\SavedText}
    \begin{minipage}{.75\textwidth}}
    {\end{minipage}\end{lrbox}\begin{center}
    \setlength{\fboxsep}{\BoxSeparation}\setlength{\fboxrule}{\OuterBoxRule}
    \fbox{\setlength{\fboxsep}{\SaveBoxSep}\setlength{\fboxrule}{\InnerBoxRule}%
      \fbox{\usebox{\SavedText}}}
  \end{center}}
% Use this:
\newcommand{\important}[1]{\begin{doublebox}#1\end{doublebox}}
%=============================================================
%:Section depth
\setcounter{secnumdepth}{2}
%% for this to happen start the file with
%\ifx\wholebook\relax\else
%\input{../common.tex}
%\begin{document}
%\fi
% and terminate by
% \ifx\wholebook\relax\else\end{document}\fi

\DeclareGraphicsExtensions{.pdf, .jpg, .png}
%=============================================================
%:PDF setup
\hypersetup{
%   a4paper,
%   pdfstartview=FitV,
%   colorlinks,
%   linkcolor=darkblue,
%   citecolor=darkblue,
%   pdftitle={Pharo by Example},
pdftitle={Pharo par l'exemple},
   pdfauthor={Andrew P. Black, St\'ephane Ducasse,	Oscar Nierstrasz,
Damien Pollet},
   pdfkeywords={Smalltalk, Squeak, Programmation Orient\'ee Objet},
pdfsubject={Informatique, Computer Science}
}
%=============================================================
%:Page layout and appearance
%
% \renewcommand{\headrulewidth}{0pt}
\renewcommand{\chaptermark}[1]{\markboth{#1}{}}
\renewcommand{\sectionmark}[1]{\markright{\thesection\ #1}}
\renewpagestyle{plain}[\small\itshape]{%
	\setheadrule{0pt}%
	\sethead[][][]{}{}{}%
	\setfoot[][][]{}{}{}}
\renewpagestyle{headings}[\small\itshape]{%
	\setheadrule{0pt}%
	\setmarks{chapter}{section}%
	\sethead[\thepage][][\chaptertitle]{\sectiontitle}{}{\thepage}%
	\setfoot[][][]{}{}{}}
% pagestyle for tableofcontents + index (martial: 2008/04/23)
\newpagestyle{newheadings}[\small\itshape]{%
	\setheadrule{0pt}%
	\setmarks{chapter}{section}%
	\sethead[\thepage][][\chaptertitle]{\chaptertitle}{}{\thepage}%
	\setfoot[][][]{}{}{}}
%=============================================================
%:Title section setup and TOC numbering depth
\setcounter{secnumdepth}{1}
\setcounter{tocdepth}{1}
\titleformat{\part}[display]{\centering}{\huge\partname\ \thepart}{1em}{\Huge\textbf}[]
\titleformat{\chapter}[display]{}{\huge\chaptertitlename\ \thechapter}{1em}{\Huge\raggedright\textbf}[]
\titlecontents{part}[3pc]{%
		\pagebreak[2]\addvspace{1em plus.4em minus.2em}%
		\leavevmode\large\bfseries}
	{\contentslabel{3pc}}{\hspace*{-3pc}}
	{}[\nopagebreak]
\titlecontents{chapter}[3pc]{%
		\pagebreak[0]\addvspace{1em plus.2em minus.2em}%
		\leavevmode\bfseries}
	{\contentslabel{3pc}}{}
	{\hfill\contentspage}[\nopagebreak]
\dottedcontents{section}[3pc]{}{3pc}{1pc}
\dottedcontents{subsection}[3pc]{}{0pc}{1pc}
% \dottedcontents{subsection}[4.5em]{}{0pt}{1pc}
% Make \cleardoublepage insert really blank pages http://www.tex.ac.uk/cgi-bin/texfaq2html?label=reallyblank
\let\origdoublepage\cleardoublepage
\newcommand{\clearemptydoublepage}{%
  \clearpage
  {\pagestyle{empty}\origdoublepage}}
\let\cleardoublepage\clearemptydoublepage % see http://www.tex.ac.uk/cgi-bin/texfaq2html?label=patch
%=============================================================
%:FAQ macros (for FAQ chapter)
\newtheorem{faq}{FAQ}
\newcommand{\answer}{\paragraph{R\'eponse}\ }
%=============================================================
%:Listings package configuration
\usepackage{listings}
%% martial: \caret défini ainsi dans SBE/SPE
%%\newcommand{\caret}{\makebox{\raisebox{0.4ex}{\footnotesize{$\wedge$}}}}
\newcommand{\caret}{\^\,} % dans PharoBook
\newcommand{\escape}{{\sf \textbackslash}}
\definecolor{source}{gray}{0.95}
\lstdefinelanguage{Smalltalk}{
%  morekeywords={self,super,true,false,nil,thisContext}, % This is overkill
  morestring=[d]',
  morecomment=[s]{"}{"},
  alsoletter={\#:},
  escapechar={!},
  escapebegin=\itshape, % comment-like by default (Martial 11/2007)
  literate=
    {BANG}{!}1
    {CARET}{\^}1
    {UNDERSCORE}{\_}1
    {\\st}{Smalltalk}9 % convenience -- in case \st occurs in code
    % {'}{{\textquotesingle}}1 % replaced by upquote=true in \lstset
    {_}{{$\leftarrow$}}1
    {>>>}{{\sep}}1
    {^}{{$\uparrow$}}1
    {~}{{$\sim$}}1
    {-}{{\sf -\hspace{-0.13em}-}}1  % the goal is to make - the same width as +
    {+}{\raisebox{0.08ex}{+}}1		% and to raise + off the baseline to match -
    {-->}{{\quad$\longrightarrow$\quad}}3
	, % Don't forget the comma at the end!
  tabsize=4
}[keywords,comments,strings]
% ajout pour les échappements dans les codes
% indispensable pour mettre le code en emphase (cf. Model.tex) 
\newcommand{\codeify}[1]{\NoAutoSpaceBeforeFDP#1\AutoSpaceBeforeFDP}
%\renewcommand{\codeify}[1]{#1} % TEST
\newcommand{\normcomment}[1]{\emph{#1}} %cf. Streams
\newcommand{\normcode}[1]{\emph{\codeify{#1}}} %cf. Streams
\newcommand{\emcode}[1]{\textbf{\normcode{#1}}} % Martial 11/2007
\lstset{language=Smalltalk,
	basicstyle=\sffamily,
	keywordstyle=\color{black}\bfseries,
	% stringstyle=\ttfamily, % Ugly! do we really want this? -- on
	mathescape=true,
	showstringspaces=false,
	keepspaces=true,
	breaklines=true,
	breakautoindent=true,
    backgroundcolor=\color{source},
    lineskip={-1pt}, % Ugly hack
	upquote=true, % straight quote; requires textcomp package
	columns=fullflexible} % no fixed width fonts
% In-line code (literal)
% Normally use this for all in-line code:
\newcommand{\ct}{\lstinline[mathescape=false,backgroundcolor=\color{white},basicstyle={\sffamily\upshape}]}
% apb 2007.8.28 added the \upshape declaration to avoid getting italicized code in \dothis{ } sections.
% In-line code (latex enabled)
% Use this only in special situations where \ct does not work
% (within section headings ...):

% [squeak-fr] Modification de \lct suivant les indications de Martial Boniou
\newcommand{\lct}[1]{\textsf{\textup{\NoAutoSpaceBeforeFDP#1\AutoSpaceBeforeFDP}}} %\xspace
%\renewcommand{\lct}[1]{\textsf{\textup{#1}}} % TEST
% Use these for system categories and protocols:
\newcommand{\scat}[1]{\emph{\textsf{#1}}\xspace}
\newcommand{\pkg}[1]{\emph{\textsf{#1}}\xspace}
\newcommand{\prot}[1]{\emph{\textsf{#1}}\xspace}
% Code environments
% NB: the arg is for tests
% Only code and example environments may be tests
\lstnewenvironment{code}[1]{%
	\lstset{%
		%frame=lines,
      frame=single,
      framerule=0pt,
		mathescape=false
	}
}{}
\def\ignoredollar#1{}
%=============================================================
%:Code environments (method, script ...)
% NB: the third arg is for tests
% Only code and example environments may be tests
\lstnewenvironment{example}[3][defaultlabel]{%
	\renewcommand{\lstlistingname}{Exemple}%
	\lstset{
%		frame=lines,
      frame=single,
      framerule=0pt,
		mathescape=false,
		caption={\emph{#2}},
		label={eg:#1}
	}
}{}
\lstnewenvironment{script}[2][defaultlabel]{%
\renewcommand{\lstlistingname}{Script}%
	\lstset{
		%frame=lines,
      frame=single,
      framerule=0pt,
      mathescape=false,
		name={Script},
		caption={\emph{#2}},
		label={scr:#1}
	}
}{}
\lstnewenvironment{method}[2][defaultlabel]{%
	\renewcommand{\lstlistingname}{M\'ethode}%
	\lstset{
%		frame=lines,
      frame=single,
      framerule=0pt,
		mathescape=false,
		name={M\'ethode},
		caption={\emph{#2}},
		label={mth:#1}
	}
}{}
\lstnewenvironment{methods}[2][defaultlabel]{% just for multiple methods at once
	\renewcommand{\lstlistingname}{M\'ethodes}%
	\lstset{
	%	frame=lines,
      frame=single,
      framerule=0pt,
		mathescape=false,
		name={M\'ethode},
		caption={\emph{#2}},
		label={mth:#1}
	}
}{}
\lstnewenvironment{numMethod}[2][defaultlabel]{%
	\renewcommand{\lstlistingname}{M\'ethode}%
	\lstset{
		numbers=left,
		numberstyle={\tiny\sffamily},
		frame=single,
        framerule=0pt,
		mathescape=false,
		name={M\'ethode},
		caption={\emph{#2}},
		label={mth:#1}
	}
}{}
\lstnewenvironment{classdef}[2][defaultlabel]{%
	\renewcommand{\lstlistingname}{Classe}%
	\lstset{
		frame=single,
framerule=0pt,
		mathescape=false,
		name={Classe},
		caption={\emph{#2}},
		label={cls:#1}
	}
}{}
%=============================================================
%:Reserving space
% Usually need one more line than the actual lines of code
\newcommand{\needlines}[1]{\Needspace{#1\baselineskip}}
%=============================================================
%:Indexing macros
% Macros ending with "ind" generate text as well as an index entry
% Macros ending with "index" *only* generate an index entry
\newcommand{\ind}[1]{\index{#1}#1\xspace} % plain text
\newcommand{\subind}[2]{\index{#1!#2}#2\xspace} % show #2, subindex inder #1
\newcommand{\emphind}[1]{\index{#1}\emph{#1}\xspace} % emph #1
\newcommand{\emphsubind}[2]{\index{#1!#2}\emph{#2}\xspace} % show emph #2, subindex inder #1
\newcommand{\scatind}[1]{\index{#1@\textsf{#1} (cat\'egorie)}\scat{#1}} % category
\newcommand{\pkgind}[1]{\index{#1@\textsf{#1} (paquetage)}\pkg{#1}} % package
\newcommand{\protind}[1]{\index{#1@\textsf{#1} (protocole)}\prot{#1}} % protocol
% \newcommand{\clsind}[1]{\index{#1@\textsf{#1} (class)}\ct{#1}\xspace}
\newcommand{\clsind}[1]{\index{#1!\#@(classe)}\ct{#1}\xspace} % class
\newcommand{\cvind}[1]{\index{#1@\textsf{#1} (variable de classe)}\ct{#1}\xspace} % class var
\newcommand{\glbind}[1]{\index{#1@\textsf{#1} (globale)}\ct{#1}\xspace} % global
\newcommand{\patind}[1]{\index{#1@#1 (patron)}\ct{#1}\xspace} % pattern
\newcommand{\pvind}[1]{\index{#1@\textsf{#1} (pseudo-variable)}\ct{#1}\xspace} % pseudo variable
\newcommand{\clsmthind}[2]{\index{#1!#2@\ct{#2}}\ct{#1>>>#2}\xspace} % class + method name
% [squeak - fr]Martial: I found the following cleaner (should be
% merged in SBE for self and super)
\newcommand{\subpvindex}[2]{\index{#1@\textsf{#1} (pseudo-variable)!#2}}
\newcommand{\subpvind}[2]{\index{#1@\textsf{#1} (pseudo-variable)!#2}#2\xspace}
% used in Model.tex
\newcommand{\mthind}[2]{\index{#1!#2@\ct{#2}}\ct{#2}\xspace} % show method name only
\newcommand{\lmthind}[2]{\index{#1!#2@\ct{#2}}\lct{#2}\xspace} % show method name only
\newcommand{\cmind}[2]{\index{#1!#2@\ct{#2}}\ct{#1>>>#2}\xspace} % show class>>method
\newcommand{\lcmind}[2]{\index{#1!#2@\ct{#2}}\lct{#1>>>#2}\xspace} % show class>>method
\newcommand{\toolsflapind}{\index{onglet Tools}\toolsflap} % index tools flap
% The following only generate an index entry:
\newcommand{\clsindex}[1]{\index{#1@\textsf{#1} (classe)}\ct{#1}\xspace}
%\newcommand{\clsindex}[1]{\index{#1!\#@(classe)}} % class
\newcommand{\mthindex}[2]{\index{#1!#2@\ct{#2}}} % method
\newcommand{\cmindex}[2]{\index{#1!#2@\ct{#2}}} % class>>method
\newcommand{\cvindex}[1]{\index{#1@\textsf{#1} (variable de classe)}} % class var
\newcommand{\glbindex}[1]{\index{#1@\textsf{#1} (globale)}}% global
\newcommand{\pvindex}[1]{\index{#1@\textsf{#1} (pseudo-variable)}}% pseudo var
\newcommand{\seeindex}[2]{\index{#1|see{#2}}} % #1, see #2
\newcommand{\scatindex}[1]{\index{#1@\textsf{#1} (cat\'egorie)}} % category
\newcommand{\pkgindex}[1]{\index{#1@\textsf{#1} (paquetage)}} % package
\newcommand{\protindex}[1]{\index{#1@\textsf{#1} (protocole)}} % protocol
% How can we have the main entry page numbers in bold yet not break the hyperlink?
\newcommand{\boldidx}[1]{{\bf #1}} % breaks hyperlink
%\newcommand{\indmain}[1]{\index{#1|boldidx}#1\xspace} % plain text, main entry
%\newcommand{\emphsubindmain}[2]{\index{#1!#2|boldidx}\emph{#2}\xspace} % subindex, main entry
%\newcommand{\subindmain}[2]{\index{#1!#2|boldidx}#2\xspace} % subindex, main entry
%\newcommand{\clsindmain}[1]{\index{#1@\textsf{#1} (class)|boldidx}\ct{#1}\xspace}
%\newcommand{\clsindmain}[1]{\index{#1!\#@(class)|boldidx}\ct{#1}\xspace} % class main
%\newcommand{\indexmain}[1]{\index{#1|boldidx}} % main index entry only
\newcommand{\indmain}[1]{\index{#1}#1\xspace} % the main index entry
                                % for this item 
\newcommand{\emphsubindmain}[2]{\index{#1!#2}\emph{#2}\xspace} % subindex, main entry
\newcommand{\subindmain}[2]{\index{#1!#2}#2\xspace} % subindex, main entry
%\newcommand{\clsindmain}[1]{\index{#1@\textsf{#1} (class)}\ct{#1}\xspace}
\newcommand{\clsindmain}[1]{\index{#1!\#@(classe)}\ct{#1}\xspace} % class main
\newcommand{\indexmain}[1]{\index{#1}} 
%=============================================================
%:Code macros
% some constants
\newcommand{\codesize}{\small}
\newcommand{\codefont}{\sffamily}
%\newcommand{\cat}[1]{\textit{Dans la cat\'egorie #1}}%%To remove later
\newlength{\scriptindent}
\setlength{\scriptindent}{.3cm}
%% Method presentation constants
\newlength{\methodindent}
\newlength{\methodwordlength}
\newlength{\aftermethod}
\setlength{\methodindent}{0.2cm}
\settowidth{\methodwordlength}{\ M\'ethode\ }
%=============================================================
%:Smalltalk macros
%\newcommand{\sep}{{$\gg$}}
\newcommand{\sep}{\mbox{>>}}
\newcommand{\self}{\ct{self}\xspace}
\newcommand{\super}{\ct{super}\xspace}
\newcommand{\nil}{\ct{nil}\xspace}
%=============================================================
% be less conservative about float placement
% these commands are from http://www.tex.ac.uk/cgi-bin/texfaq2html?label=floats
\renewcommand{\topfraction}{.9}
\renewcommand{\bottomfraction}{.9}
\renewcommand{\textfraction}{.1}
\renewcommand{\floatpagefraction}{.85}
\renewcommand{\dbltopfraction}{.66}
\renewcommand{\dblfloatpagefraction}{.85}
\setcounter{topnumber}{9}
\setcounter{bottomnumber}{9}
\setcounter{totalnumber}{20}
\setcounter{dbltopnumber}{9}
%=============================================================
%% [Squeak-fr]
% pour identifier les zones de texte à corriger d'urgence!
\newcommand{\arevoir}[1]{\ugh{#1}}
\newcommand{\arelire}[1]{\textcolor{blue}{#1}}
\newcommand{\aretirer}[1]{\del{#1}}
% \traduit utilisé dans Model.tex
\newcommand{\traduit}[1]{\footnote[2]{#1}}
% changeset alias
\newcommand{\changeset}{\emph{change set}\xspace}
\newcommand{\changesets}{\emph{change sets}\xspace}
% callback alias
\newcommand{\callback}{\emph{callback}\xspace}
% blobmorph alias (QuickTour->blob)
\newcommand{\blobmorph}{\emph{blob}\xspace}
% repository
\newcommand{\squeaksource}{\textsf{SqueakSource}\xspace}
\newcommand{\sourceforge}{\textsf{SourceForge}\xspace}
% L'onglet Tools
\newcommand{\Toolsflap}{L'onglet \textit{Tools}\xspace}
% Mac OS X
\newcommand{\macosx}{\mbox{Mac OS X}\xspace}
% code en francais (uniquement dans le chapitre BasicClasses)
\newcommand{\codefrench}[1]{\NoAutoSpaceBeforeFDP\texttt{#1}\AutoSpaceBeforeFDP\xspace}
%\renewcommand{\codefrench}[1]{\texttt{#1}} % TEST
% mantra du modele objet
\newcommand{\Mantra}{Tout est objet\xspace}
\newcommand{\mantra}{\MakeLowercase{\Mantra}\xspace}
%============================================================
%% spécial PBE (Pharo By Example - vf)
\newcommand{\senders}{\emph{senders}\xspace}
\newcommand{\sender}{\emph{sender}\xspace}
\newcommand{\implementors}{\emph{implementors}\xspace}
\newcommand{\implementor}{\emph{implementor}\xspace}
\newcommand{\truetype}{\textsf{TrueType}\xspace}
% césure (pour forcer les coupures de mots)
\hyphenation{Omni-Brow-ser}
\hyphenation{m\'e-tho-de} % erreur de cesure commune
\hyphenation{m\'e-tho-des}
\hyphenation{e-xem-ple}
\hyphenation{en-re-gi-stre}
\hyphenation{a-na-ly-seur}
\hyphenation{glo-ba-le}
\hyphenation{fi-gu-re}
\hyphenation{vi-si-bles}
\hyphenation{cor-res-pon-dan-te}
\hyphenation{Work-space}
%=============================================================
% apb doesn't like paragraphs to run in to each other without a break
\parskip 1ex
%=============================================================
%:Stuff to check, merge or deprecate
%\setlength{\marginparsep}{2mm}
%\renewcommand{\baselinestretch}{1.1}
%=============================================================
