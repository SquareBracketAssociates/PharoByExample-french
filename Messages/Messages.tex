% $Author$ traducteur Serge
% $Date$ 
% $Revision$
% relecture par Martial Boniou: Thu Dec 13 22:40:00 CET 2007
% Une instance de Pen est appele 'crayon' (et non plus, crayon ou
% stylo...)
% relecture par Rene Mages : Thu Dec 20 12:13:40 2007
% relecture par Rene Mages : Thr Jan 10 12:11:33 2008
% adaptation pour PBE - martial - Fri Sep 11 17:52:08 CEST 2009 from
% $Author: oscar $ % $Date: 2009-08-27 12:59:05 +0200 (Thu, 27 Aug 2009) $ % $Revision: 28624 $
%=================================================================
\ifx\wholebook\relax\else
% --------------------------------------------
% Lulu:
	\documentclass[a4paper,10pt,twoside]{book}
	\usepackage[
		papersize={6.13in,9.21in},
		hmargin={.75in,.75in},
		vmargin={.75in,1in},
		ignoreheadfoot
	]{geometry}
	\input{../common.tex}
	\pagestyle{headings}
	\setboolean{lulu}{true}
% --------------------------------------------
% A4:
%	\documentclass[a4paper,11pt,twoside]{book}
%	\input{../common.tex}
%	\usepackage{a4wide}
% --------------------------------------------
    \graphicspath{{figures/} {../figures/}}
	\begin{document}
	\renewcommand{\nnbb}[2]{} % Disable editorial comments
	\sloppy
\fi
%=================================================================
\chapter{Comprendre la syntaxe des messages}
\chalabel{understanding}

Bien que la syntaxe des messages \st soit extr\^emement simple, elle n'est pas habituelle et cela peut prendre un certain temps pour s'y habituer. Ce chapitre offre quelques conseils pour vous aider \`a mieux appr\'ehender la syntaxe sp\'eciale des envois de messages.
Si vous vous sentez en confiance avec la syntaxe, vous pouvez choisir de sauter ce chapitre ou bien d'y revenir un peu plus tard.

\on{I still feel this chapter contains too much repetition.
I would like to get feedback from students.}

%=============================================================
\section{Identifier les messages}

En \st, exception faite des \'el\'ements syntaxiques rencontr\'es dans
\charef{syntax} (\ct+:= ^ . ; # () {} [ : | ]+), tout se passe par envoi de messages.
Comme en \ind{C++}, vous pouvez d\'efinir vos op\'erateurs comme \ct{+} pour vos propres classes, mais tous les op\'erateurs ont la m\^eme pr\'ec\'edence.
De plus, il n'est pas possible de changer l'arit\'e d'une m\'ethode:
\ct{-} est toujours un message binaire, et il n'est pas possible
d'avoir une forme unaire avec une surcharge diff\'erente.

Avec \st, l'ordre dans lequel les messages sont envoy\'es est
d\'etermin\'e par le type de message. Il n'y a que trois formes de
messages: les messages \emphsubind{message}{unaire},
\emphsubind{message}{binaire} et \emphsubind{message}{à mots-clés}.
 Les messages unaires sont toujours envoy\'es en premier, puis
les messages binaires et enfin ceux \`a mots-cl\'es. Comme dans la
plupart des langages,  les \ind{parenthèses} peuvent \^etre utilis\'ees pour changer l'ordre d'\'evaluation. Ces r\`egles rendent le code \st aussi facile \`a lire que possible. La plupart du temps, il n'est pas n\'ecessaire de r\'efl\'echir \`a ces r\`egles.

Comme la plupart des calculs en \st sont effectu\'es par des envois de messages, identifier correctement les messages est crucial. La terminologie suivante va nous \^etre utile:

\begin{itemize}
  \item Un message est compos\'e d'un \emphsubind{message}{sélecteur} et d'arguments optionnels.
  \item Un message est envoy\'e au \emphsubind{message}{receveur}.
  \item La combinaison d'un message et de son receveur est appel\'e un \emphsubind{message}{envoi} \emph{de message}  comme il est montr\'e dans \figref{firstScriptMessage}.
\end{itemize}

\begin{figure}[htb]
\begin{minipage}{0.53\textwidth}
	\begin{center}
	\includegraphics[width=0.95\textwidth]{message}
	\caption{Deux messages compos\'es d'un receveur, d'un s\'electeur de m\'ethode et d'un ensemble d'arguments.\figlabel{firstScriptMessage}}\end{center}
\end{minipage}
\hfill
\begin{minipage}{0.43\textwidth}
	\begin{center}
	\ifluluelse
		{\includegraphics[width=0.9\textwidth]{uKeyUnOne}}
		{\includegraphics[width=6cm]{uKeyUnOne}}
	\caption{\ct{aMorph color: Color yellow} est compos\'e de deux expressions : \ct{Color yellow} et \ct{aMorph color: Color yellow}.\figlabel{ellipse}}
	\end{center}
\end{minipage}
\end{figure}


\important{Un message est toujours envoy\'e \`a un receveur qui peut \^etre un simple litt\'eral, une variable ou le r\'esultat de l'\'evaluation d'une autre expression.}

Nous vous proposons de vous faciliter la lecture au moyen d'une
notation graphique: nous soulignerons le receveur afin de vous aider
\`a l'identifier. Nous entourerons \'egalement chaque expression dans
une ellipse et num\'eroterons les expressions \`a partir de la
premi\`ere \`a \^etre \'evalu\'ee afin de voir l'ordre d'envoi des messages.

%\begin{figure}[!ht]
%\begin{center}
%\includegraphics[width=6cm]{uKeyUnOne}
%\end{center}
%\caption{\ct{aMorph color: Color yellow} is composed of two expressions: \ct{Color yellow} and \ct{aMorph color: Color yellow}.\figlabel{ellipse}}
%\end{figure}

\Figref{ellipse} repr\'esente deux envois de messages, \ct{Color yellow} et \ct{aMorph color: Color yellow}, de telle sorte qu'il y a
deux ellipses. L'expression \ct{Color yellow} est d'abord \'evalu\'e
en premier, ainsi son ellipse est num\'erot\'ee \`a \ct{1}. Il y a
deux receveurs: \ct{aMorph} qui re\c{c}oit le message \ct{color: ...}
et \ct{Color} qui re\c{c}oit le message \ct{yellow} 
%ajout
(\emph{yellow} correspond \`a la couleur jaune en anglais). 
Chacun des receveurs est soulign\'e.

Un receveur peut \^etre le premier \'el\'ement d'un message, comme
\ct{100} dans l'expression \ct{100 + 200} ou \ct{Color} 
%ajout
(la classe des couleurs)
dans l'expression \ct{Color yellow}. Un objet receveur peut
\'egalement \^etre le r\'esultat de l'\'evaluation d'autres
messages. Par exemple, dans le message \ct{Pen new go: 100}, le
receveur de ce message \ct{go: 100} 
%ajout
(litt\'eralement, aller \`a 100)
est l'objet retourn\'e par cette expression \ct{Pen new} 
%ajout
(soit une instance de \ct{Pen}, la classe crayon). Dans tous les cas,
le message est envoy\'e \`a un objet appel\'e le \emph{receveur} qui a
pu \^etre cr\'e\'e par un autre envoi de message.

\begin{table}\centering
	\begin{tabularx}{\linewidth}{llX}
		\toprule
		Expression & Type de messages & R\'esultat \\
		\midrule
		\lct{Color yellow}
			& unaire
			& Cr\'ee une couleur.
		\\
		\lct{aPen  go: 100}
			& \`a mots-cl\'es
			& Le crayon receveur se d\'eplace en avant de 100 pixels.
		\\
		\lct{100 + 20}
			& binaire
			& Le nombre 100 re\c{c}oit le message + avec le param\`etre 20.
		\\
		\lct{Browser open}
			& unaire
			& Ouvre un nouveau navigateur de classes.
		\\
		\lct{Pen new  go: 100}
			& unaire et \`a mots-cl\'es
			& Un crayon est cr\'e\'e puis d\'eplac\'e de 100 pixels.
		\\
		\lct{aPen go: 100 + 20}
			& \`a mots-cl\'es et binaire
			& Le crayon receveur se d\'eplace vers l'avant de 120 pixels.
		\\
		\bottomrule
	\end{tabularx}
	\caption{Exemples de messages}\tablabel{messageExamples}
\end{table}

\Tabref{messageExamples} montre diff\'erents exemples de messages.
Vous devez remarquer que tous les messages n'ont pas obligatoirement
d'arguments. Un message unaire comme \ct{open} (pour ouvrir) ne n\'ecessite pas d'arguments. Les messages \`a mots-cl\'es simples ou les messages binaires comme \ct{go: 100} et \ct{+ 20} ont chacun un argument. 
Il y a aussi des messages simples et des messages
compos\'es. \ct{Color yellow} et \ct{100 + 20} sont simples: un
message est envoy\'e \`a un objet, tandis que l'expression \ct{aPen go: 100 + 20} est compos\'ee de deux messages: \ct{+ 20} est
envoy\'e \`a \ct{100} et \ct{go:} est envoy\'e \`a \ct{aPen} avec pour
argument le r\'esultat du premier message.
Un receveur peut \^etre une expression qui peut retourner un
objet. Dans \ct{Pen new go: 100}, le message \ct{go: 100} est envoy\'e
\`a l'objet qui r\'esulte de l'\'evaluation de l'expression \ct{Pen new}.

% ON: An enumerated list here is overkill!
%=============================================================
\section{Trois sortes de messages}

\st d\'efinit quelques r\`egles simples pour d\'eterminer l'ordre dans lequel les messages sont envoy\'es. Ces r\`egles sont bas\'ees sur la distinction \'etablie entre les 3 formes d'envoi de messages: 
\begin{itemize}
\item \emph{Les messages unaires} sont des messages qui sont envoy\'es
  \`a un objet sans autre information. Par exemple dans \ct{3 factorial}, \ct{factorial} (pour factorielle) est un message unaire. 
\item  \emph{Les messages binaires} sont des messages form\'es avec
  des op\'erateurs (souvent arithm\'etiques). Ils sont binaires car
  ils ne concernent que deux objets: le receveur et l'objet
  argument. Par exemple, dans \ct{10 + 20}, \ct{+} est un message
  binaire qui est envoy\'e au receveur \ct{10} avec l'argument \ct{20}. 
\item  \emph{Les messages \`a mots-cl\'es} sont des messages form\'es avec plusieurs mots-cl\'es, chacun d'entre eux se finissant par deux points (\ct{:}) et prenant un param\`etre.
Par exemple, dans \ct{anArray at: 1 put: 10}, le mot-cl\'e \ct{at:}
prend un argument \ct{1} et le mot-cl\'e \ct{put:} prend l'argument \ct{10}.
\end{itemize}

%-------------------------------------------------------------
\subsection{Messages unaires}
Les messages unaires sont des messages qui ne n\'ecessitent aucun
argument. Ils suivent le mod\`ele syntaxique suivant: \ct{receveur nomMessage}. Le s\'electeur est constitu\'e d'une s\'erie de
caract\`eres ne contenant pas de deux points (\ct{:}) (\eg
\ct{factorial}, \ct{open}, \ct{class}).
\needlines{4}
\begin{code}{@TEST}
89 sin           --> 0.860069405812453
3 sqrt           --> 1.732050807568877
Float pi         --> 3.141592653589793
'blop' size     --> 4
true not        --> false
Object class --> Object class  "La classe de Object est Object class (BANG)"
\end{code}
% ON: I changed the examples to things we can test

\important{Les messages unaires sont des messages qui ne n\'ecessitent pas d'argument.\\
Ils suivent le moule syntaxique: \lct{receveur \textbf{s\'electeur}}}

%-------------------------------------------------------------
\subsection{Messages binaires} 
Les messages binaires sont des messages qui n\'ecessitent exactement un argument \emph{et} dont le s\'electeur consiste en une s\'equence de un ou plusieurs caract\`eres de l'ensemble: \ct{+}, \ct{-}, \ct{*}, \ct{/}, \ct{&}, \ct{=}, \ct{>}, \ct{|}, \ct{<}, \ct{~}, et \ct{@}. Notez que \ct{--} n'est pas autoris\'e.

\begin{code}{@TEST}
100@100      --> 100@100  "!cr\'ee! un objet Point"
3 + 4              --> 7
10 - 1            --> 9
4 <= 3            --> false
(4/3) * 3 = 4   --> true  "!l'\'egalit\'e! est juste un message binaire et les fractions sont exactes"
(3/4) == (3/4) --> false  "!deux fractions \'egales ne sont pas le m\^eme objet!"
\end{code}

\important{Les messages binaires sont des messages qui n\'ecessitent exactement un argument \emph{et} dont le s\'electeur est compos\'e d'une s\'equence de caract\`eres parmi : \ct{+}, \ct{-}, \ct{*}, \ct{/}, \ct{\&}, \ct{=}, \ct{>}, \ct{|}, \ct{<}, \ct{\~}, et \ct{@}. \ct{--} n'est pas possible.\\
Ils suivent le moule syntaxique: \lct{receveur \textbf{s\'electeur} argument}}

%-------------------------------------------------------------
\subsection{Messages \`a mots-cl\'es}

Les messages \`a mots-cl\'es sont des messages qui n\'ecessitent un ou plusieurs arguments et dont le s\'electeur consiste en un ou plusieurs mots-cl\'es se finissant par deux points \ct{:}.  Les messages \`a mots-cl\'es suivent le moule syntaxique: 
\lct{receveur \textbf{selecteurMotUn:} argument\-Un \textbf{motDeux:} argumentDeux}

Chaque mot-cl\'e utilise un argument. Ainsi \ct{r:g:b:} est une
m\'ethode avec 3 arguments, \ct{playFileNamed:} et \ct{at:} sont des
m\'ethodes avec un argument, et \ct{at:put:} est une m\'ethode avec
deux arguments. Pour cr\'eer une instance de la classe \ct{Color} on
peut utiliser la m\'ethode \ct{r:g:b:} comme dans \ct{Color r: 1 g: 0 b: 0} cr\'eant ainsi la couleur rouge. Notez que les deux points ne font pas partie du s\'electeur.

\important{En \ind{Java} ou \ind{C++}, l'invocation de m\'ethode \st \ct{Color r: 1 g: 0 b: 0} serait \'ecrite \ct{Color.rgb(1,0,0)}.}

\begin{code}{@TEST | nums |}
1 to: 10                        --> (1 to: 10)  "!cr\'eation! d'un intervalle"
Color r: 1 g: 0 b: 0       --> Color red  "!cr\'eation! d'une nouvelle
couleur (rouge)"
12 between: 8 and: 15 --> true

nums := Array newFrom: (1 to: 5).
nums at: 1 put: 6.
nums --> #(6 2 3 4 5)
\end{code}
% ON: Changed to real examples that we can test

\important{Les messages bas\'es sur les mots-cl\'es sont des messages qui n\'ecessitent un ou plusieurs arguments. Leurs s\'electeurs consistent en un ou plusieurs mots-cl\'es chacun se terminant par deux points (\ct{\:}). Ils suivent le moule syntaxique:\\
\lct{receveur \textbf{selecteurMotUn:} argumentUn \textbf{motDeux:} argumentDeux}}

%=============================================================
\section{Composition de messages}
Les trois formes d'envoi de messages ont chacune des priorit\'es diff\'erentes, ce qui permet de les composer de mani\`ere \'el\'egante.

\begin{enumerate}
\item Les messages unaires sont envoy\'es en premier, puis les messages binaires et enfin les messages \`a mots-cl\'es.
\item Les messages entre \ind{parenthèses} sont envoy\'es avant tout autre type de messages. 
\item Les messages de m\^eme type sont envoy\'es de gauche \`a droite. 
\end{enumerate}
\index{message!ordre d'évaluation}

Ces r\`egles ont un ordre de lecture tr\`es naturel. Maintenant si
vous voulez \^etre s\^ur que vos messages sont envoy\'es dans l'ordre
que vous souhaitez, vous pouvez toujours mettre des parenth\`eses
suppl\'ementaires comme dans \figref{uKeyUn}. Dans cet exemple, le
message \ct{yellow} est un message unaire et le message \ct{color:}
est un message \`a mots-cl\'es; ainsi l'expression \ct{Color yellow}
est envoy\'e en premier. N\'eanmoins comme les expressions entre
parenth\`eses sont envoy\'ees en premier, mettre des parenth\`eses
(normalement inutiles) autour de \ct{Color yellow} permet d'accentuer
le fait qu'elle
%l'expression
doit \^etre envoy\'ee en premier. Le reste de cette section illustre
chacun de ces diff\'erents points.

\begin{figure}[ht]
\ifluluelse
	{\centerline{\includegraphics[width=0.9\textwidth]{uKeyUn}} }
	{\centerline{\includegraphics[width=10cm]{uKeyUn}} }
\caption{Les messages unaires sont envoy\'es en premier; donc ici le
  premier message est \ct{Color yellow}. Il retourne un objet de
  couleur jaune qui est pass\'e comme argument du message \ct{aPen color:}.\figlabel{uKeyUn}}
\end{figure}

%---------------------------------------------------------
\subsection*{Unaire > Binaire > Mots-cl\'es}
Les messages unaires sont d'abord envoy\'es, puis les messages
binaires et enfin les messages \`a mots-cl\'es. Nous pouvons
\'egalement dire que les messages unaires ont une priorit\'e plus
importante que les autres types de messages.

\important{\textbf{R\`egle  une.} Les messages unaires sont envoy\'es en premier, puis les messages binaires et finalement les messages \`a mots-cl\'es.\\
\centerline{\lct{Unaire > Binaire > Mots-cl\'es}}
}

Comme ces exemples suivants le montrent, les r\`egles de syntaxe de
\st permettent d'assurer une certaine lisibilit\'e des expressions:
\begin{code}{@TEST}
1000 factorial / 999 factorial --> 1000
2 raisedTo: 1 + 3 factorial     --> 128
\end{code}
\noindent

Malheureusement, les r\`egles sont un peu trop simplistes pour les
expressions arithm\'etiques. D\`es lors, des parenth\`eses doivent
\^etre introduites chaque fois que l'on veut imposer un ordre de
priorit\'e entre deux op\'erateurs binaires:
\begin{code}{@TEST}
1 + 2 * 3   --> 9
1 + (2 * 3) --> 7
\end{code}

L'exemple suivant qui est un peu plus complexe (!) est l'illustration que m\^eme des expressions \st compliqu\'ees peuvent \^etre lues de mani\`ere assez naturelle: 
\begin{code}{@TEST}
[:aClass | aClass methodDict keys select: [:aMethod | (aClass>>aMethod) isAbstract ]] value: Boolean --> an IdentitySet(#or: #| #and: #& #ifTrue: #ifTrue:ifFalse: #ifFalse: #not #ifFalse:ifTrue:)
\end{code}
% note de martial: j'ai ajoute entre parentheses le nom des messages
% pour plus de clarte
\noindent
Ici nous voulons savoir quelles m\'ethodes de la classe \ct{Boolean}
(classe des bool\'eens) sont abstraites.
Nous interrogeons la classe argument \ct{aClass} pour r\'ecup\'erer
les cl\'es (via le message unaire \ct{keys}) de son dictionnaire de
m\'ethodes (via le message unaire \ct{methodDict}), puis nous en
s\'electionnons (via le message \`a mots-cl\'es \ct{select:}) les
m\'ethodes de la classe qui sont abstraites.
Ensuite nous lions (par \ct{value:}) l'argument \ct{aClass} \`a la
valeur concr\`ete \ct{Boolean}.
Nous avons besoin des parenth\`eses uniquement pour le message binaire
\ct{>>}, qui s\'electionne une m\'ethode d'une classe, avant d'envoyer
le message unaire \mbox{\ct{isAbstract}} \`a cette m\'ethode. Le
r\'esultat (sous la forme d'un ensemble de classe \ct{IdentifySet})
nous montre quelles m\'ethodes doivent \^etre impl\'ement\'ees par les
sous-classes concr\`etes de \ct{Boolean}: \ct{True} et \ct{False}.


\paragraph{Exemple.}
Dans le message \ct{aPen color: Color yellow}, il y a un message \emph{unaire} \ct{yellow} envoy\'e \`a la classe \ct{Color} et un message \`a \emph{mots-cl\'es} \ct{color:} envoy\'e \`a \ct{aPen}. Les messages unaires sont d'abord envoy\'es, de telle sorte que l'expression \ct{Color yellow} soit d'abord ex\'ecut\'ee (1). Celle-ci retourne un objet couleur qui est pass\'e en argument du message \ct{aPen color: aColor} (2) comme indiqu\'e dans l'\egref{decColor}.
\Figref{uKeyUn} montre graphiquement comment les messages sont envoy\'es.

\needlines{5}
\begin{example}[decColor]{D\'ecomposition de l'\'evaluation de \ct{aPen color: Color yellow}}{}
        aPen color: Color yellow
(1)                       Color yellow        "message unaire !envoy\'e! en premier"
                        --> aColor
(2)   aPen color: aColor                 "puis le message !\`a mots-cl\'es!"
\end{example}

\paragraph{Exemple.} Dans le message \ct{aPen go: 100 + 20}, il y a le message \emph{binaire} \ct{+ 20} et un message \`a \emph{mots-cl\'es} \ct{go:}. Les messages binaires sont d'abord envoy\'es avant les messages \`a mots-cl\'es, ainsi \ct{100 + 20} est envoy\'e en premier (1): le message \ct{+ 20} est envoy\'e \`a l'objet \ct{100} et retourne le nombre \ct{120}. Ensuite le message \ct{aPen go: 120} est envoy\'e avec comme argument \ct{120} (2).
L'\egref{decGo} nous montre comment l'expression est \'evalu\'e. 

\begin{example}[decGo]{D\'ecomposition de \ct{aPen go: 100 + 20}}{}
      aPen go: 100 + 20   
(1)                 100 + 20           "le message binaire en premier"
                   -->   120
(2)  aPen go: 120                   "puis le message !\`a mots-cl\'es!"
\end{example}

\begin{figure}[htb]
\begin{minipage}{0.48\textwidth}
	\ifluluelse
		{\centerline{\includegraphics[width=0.9\textwidth]{uKeyBin}}}
		{\centerline{\includegraphics[width=6cm]{uKeyBin}}}
	\caption{Les messages unaires sont envoy\'es en premier, ainsi
      \ct{Color yellow} est d'abord envoy\'e. Il retourne un objet de
      couleur jaune qui est pass\'e en argument du message \ct{aPen color:}.\figlabel{uKeyBin}}
\end{minipage}
\hfill
\begin{minipage}{0.48\textwidth}
	\begin{center}
	\ifluluelse
		{\includegraphics[width=0.9\textwidth]{uunKeyBin}}
		{\includegraphics[width=6cm]{uunKeyBin}}
\caption{D\'ecomposition de \ct{Pen new go: 100 + 20}.\figlabel{unKeyBin}}
\end{center}
\end{minipage}
\end{figure}

\paragraph{Exemple.} Comme exercice, nous vous laissons d\'ecomposer
l'\'evaluation du message \ct{Pen new go: 100 + 20} qui est compos\'e
d'un message unaire, d'un message \`a mots-cl\'es et d'un message
binaire (voir \figref{unKeyBin}).

%-------------------------------------------------------------
\subsection{Les parenth\`eses en premier}

\important{\textbf{R\`egle deux.} Les messages parenth\'es\'es sont envoy\'es avant tout autre message.\\
\centerline{\lct{(Msg) > Unaire > Binaire > Mots-cl\'es}}}

\begin{code}{@TEST}
1.5 tan rounded asString = (((1.5 tan) rounded) asString) --> true    "les !parenth\`eses! sont !n\'ecessaires! ici"
3 + 4 factorial   --> 27    "(et pas 5040)"
(3 + 4) factorial --> 5040
\end{code}

Ici nous avons besoin des \ind{parenthèses} pour forcer l'envoi de \ct{lowMajorScaleOn:} avant \ct{play}.
\begin{code}{}
(FMSound lowMajorScaleOn: FMSound clarinet) play 
"(1) envoie le message clarinet !\`a! la classe FMSound pour !cr\'eer! le son de clarinette.
 (2) envoie le son !\`a! FMSound comme argument du message !\`a! !mots-cl\'es! lowMajorScaleOn:.
 (3) joue le son !r\'esultant!."
\end{code}

% ON: This has nothing to do with parentheses!
%RecordingControlsMorph new openInWorld
%"An instance of the digitizer is created then visualized. If your microphone is plugged in try a sampleBANG"

% ON: This link is broken, and the result does not understand display!
%(HTTPSocket httpShowGif:
%   'www.altavista.digital.com/av/pix/default/av-adv.gif') display

\paragraph{Exemple.}
Le message \ct{(65@325 extent: 134@100) center} retourne le centre
du rectangle dont le point sup\'erieur gauche est $(65, 325)$ et dont
la taille est $134{\times}100$. L'\egref{decExtent} montre comment le
message est d\'ecompos\'e et envoy\'e. Le message entre parenth\`eses
est d'abord envoy\'e: il contient deux messages binaires \ct{65@325}
et \ct{134@100} qui sont d'abord envoy\'es et qui retournent des
points, et un message \`a mots-cl\'es \ct{extent:} qui est ensuite
envoy\'e et qui retourne un rectangle. Finalement le message unaire
\ct{center} est envoy\'e au rectangle et le point central est retourn\'e.

\'Evaluer ce message sans parenth\`eses d\'eclencherait une erreur car
l'objet \ct{100} ne comprend pas le message \ct{center}.

\needlines{9} % CHANGE REVOIR
\begin{example}[decExtent]{Exemple avec des parenth\`eses.}{}
      (65 @ 325 extent: 134 @ 100) center
(1)   65@325                                                    "binaire"
    --> aPoint
(2)                                134@100                     "binaire"
                                 --> anotherPoint
(3)   aPoint extent: anotherPoint                       "!\`a mots-cl\'es!"
      --> aRectangle
(4)   aRectangle center                                     "unaire"
      --> 132@375
\end{example}

\subsection{De gauche \`a droite}
Maintenant nous savons comment les messages de diff\'erentes natures
ou priorit\'es sont trait\'es. Il reste une question \`a traiter:
comment les messages de m\^eme priorit\'e sont envoy\'es? Ils sont
envoy\'es de gauche \`a droite. Notez que vous avez d\'ej\`a vu ce
comportement dans l'\egref{decExtent} dans lequel les deux messages de
cr\'eation de points (\ct{@}) sont envoy\'es en premier.

\important{{\textbf{R\`egle trois.} Lorsque les messages sont de m\^eme nature, l'ordre d'\'evaluation est de gauche \`a droite.}}

%\begin{figure}
%\centerline{\includegraphics[width=8cm]{ucompoUn}} 
%\caption{The message \ct{Pen new east} is composed of two unary messages. Therefore the leftmost one, \ct{new},  is sent and it returns a new robot to which the second message \ct{east} is sent. \figlabel{compoUn}}
%\end{figure}

\paragraph{Exemple.} Dans l'expression \ct{Pen new down}, tous les
messages sont des messages unaires, donc celui qui est le plus \`a
gauche \ct{Pen new} est envoy\'e en premier. Il retourne un nouveau
crayon auquel le deuxi\`eme message \ct{down} 
%ajout
(pour poser la pointe du crayon et dessiner)
est envoy\'e comme il est montr\'e dans \figref{unaryMessages}.

\begin{figure}
	\centering
	\includegraphics[width=8cm]{ucompoUn}
	\caption{D\'ecomposition de \ct{Pen new down}.\figlabel{unaryMessages}}
\end{figure}

%-------------------------------------------------------------
%\subsection{Inconsistances arithm\'etiques}
% note de martial: j'ai fait des recherches; c'est plus correct
% qu'inconsistence; le vrai terme est irrationnalite
\subsection{Incoh\'erences arithm\'etiques}
Les r\`egles de composition des messages sont simples mais peuvent
engendrer des incoh\'erences dans l'\'evaluation des expressions
arithm\'etiques qui sont exprim\'ees sous forme de messages binaires
%ajout
(nous parlons aussi d'irrationnalit\'e arithm\'etique).
Voici des situations habituelles o\`u des parenth\`eses suppl\'ementaires sont n\'ecessaires.

\begin{code}{@TEST}
3 + 4 * 5      --> 35    "(pas 23)  les messages binaires sont !envoy\'es! de gauche !\`a! droite"
3 + (4 * 5)    --> 23
1 + 1/3         --> (2/3)    "et pas 4/3"
1 + (1/3)       --> (4/3)
1/3 + 2/3       --> (7/9)    "et pas 1"
(1/3) + (2/3)  --> 1
\end{code}

\paragraph{Exemple.} 
Dans l'expression \ct{20 + 2 * 5}, il y a seulement les messages
binaires \ct{+} et \ct{*}. En \st, il n'y a pas de priorit\'e
sp\'ecifique pour les op\'erations \ct{+} et \ct{*}. Ce ne sont que
des messages binaires, ainsi \ct{*} n'a pas priorit\'e sur \ct{+}. Ici
le message le plus \`a gauche \ct{+} est envoy\'e en premier (1) et
ensuite \ct{*} est envoy\'e au r\'esultat comme nous le voyons dans l'\egref{binaryMessages1}.  

\begin{example}[binaryMessages1]{D\'ecomposer \ct{20 + 2 * 5}}{}
"Comme il n'y a pas de !priorit\'e! entre les messages binaires, le message le plus !\`a! gauche, + est !\'evalu\'e! en premier !m\^eme! si !d'apr\`es! les !r\`egles! de !l'arithm\'etique! le * devrait d'abord !\^etre! !envoy\'e.!"

      20 + 2 * 5 
(1)  20 + 2 --> 22
(2)  22       * 5 --> 110
\end{example}

\begin{figure}
\begin{center}\includegraphics[width=8cm]{ucompoNoBracketPar}\end{center}
\end{figure}
\noindent
Comme il est montr\'e dans l'\egref{binaryMessages1} le r\'esultat de
cette expression n'est pas \ct{30} mais \ct{110}. Ce r\'esultat est
peut-\^etre inattendu mais r\'esulte directement des r\`egles
utilis\'ees pour envoyer des messages. Ceci est le prix \`a payer pour
la simplicit\'e du mod\`ele de \st. Afin d'avoir un r\'esultat
correct, nous devons utiliser des parenth\`eses. Lorsque les messages
sont entour\'es par des parenth\`eses, ils sont \'evalu\'es en
premier. Ainsi l'expression \ct{20 + (2 * 5)} retourne le r\'esultat
comme nous le voyons dans l'\egref{mathcorrect}.

\needlines{4}
\begin{example}[mathcorrect]{D\'ecomposition de \ct{20 + (2 * 5)}}{}
"Les messages !entour\'es! de !parenth\`eses! sont !\'evalu\'es! en premier ainsi * est !envoy\'e! avant + afin de produire le comportement !souhait\'e.!"

    20 + (2 * 5) 
(1)        (2 * 5) --> 10
(2) 20 + 10      --> 30
\end{example}

\begin{figure}
\begin{center}
\includegraphics[width=8cm]{ucompoNumberBracket}
\end{center}
\end{figure}

\important{En \st, les op\'erateurs arithm\'etiques comme + et * n'ont
  pas des priorit\'es diff\'erentes. \ct{+} et \ct{*} ne sont que des
  messages binaires; donc \ct{*} n'a pas priorit\'e sur
  \ct{+}. Utiliser des parenth\`eses pour obtenir le r\'esultat d\'esir\'e.}

%  At the beginning put parenthesis when you have multiple binary messages.}  HUH?  At the beginning of what?!

\begin{figure}
\begin{center}
\ifluluelse
	{\includegraphics[width=\textwidth]{uKeyUnBinPar}}
	{\includegraphics[width=0.8\textwidth]{uKeyUnBinPar}}
\ifluluelse
	{\includegraphics[width=\textwidth]{uunKeyBinPar}}
	{\includegraphics[width=10cm]{uunKeyBinPar}}
\end{center}
\caption{Messages \'equivalents en utilisant des parenth\`eses.\figlabel{uKeyUnBinPar}}
\end{figure}

Notez que la premi\`ere r\`egle, disant que les messages unaires sont
envoy\'es avant les messages binaires ou \`a mots-cl\'es, ne nous force
pas \`a mettre explicitement des parenth\`eses autour
d'eux. \Tabref{expressions} montre des expressions \'ecrites en
respectant les r\`egles et les expressions \'equivalentes si les
r\`egles n'existaient pas. Les deux versions engendrent le m\^eme
effet et retournent les m\^emes valeurs.

\begin{figure}\centering
	\begin{tabular}{l@{\qquad}l}
	\toprule
	Priorit\'e implicite & \'Equivalent explicite parenth\'es\'e\\
	\midrule
	\lct{aPen color: Color yellow}
		& \lct{aPen color: (Color yellow)}
		\\
	\lct{aPen go: 100 + 20}
		& \lct{aPen go: (100 + 20)}
		\\
	\lct{aPen penSize: aPen penSize + 2}
		& \lct{aPen penSize: ((aPen penSize) + 2)}
		\\
	\lct{2 factorial + 4}
		& \lct{(2 factorial) + 4}
		\\
	\bottomrule
	\end{tabular}
	\caption{Des expressions et leurs versions \'equivalentes compl\`etement parenth\'es\'ees.\tablabel{expressions}}
\end{figure}

%=============================================================
\section{Quelques astuces pour identifier les messages \`a mots-cl\'es}
Souvent les d\'ebutants ont des probl\`emes pour comprendre quand ils doivent ajouter des parenth\`eses. Voyons comment les messages \`a mots-cl\'es sont reconnus par le compilateur.

%-------------------------------------------------------------
\subsection{Des parenth\`eses ou pas ?}
Les caract\`eres \ct{[}, \ct{]}, and \ct{(}, \ct{)} 
d\'elimitent des zones distinctes. Dans ces zones, un message \`a mots-cl\'es est la plus longue s\'equence de mots termin\'es par (\ct{:}) qui n'est pas coup\'e par les caract\`eres (\ct{.}), ou (\ct{;}). 
Lorsque les caract\`eres \ct{[}, \ct{]}, et \ct{(}, \ct{)} entourent des mots avec des deux points, ces mots participent au message \`a mots-cl\'es \emph{local} \`a la zone d\'efinie.

Dans cet exemple, il y a deux mots-cl\'es distincts: \ct{rotatedBy:magnify:smoothing:} et \ct{at:put:}.

\begin{code}{}
aDict
   at: (rotatingForm 
          rotateBy: angle	
          magnify: 2 
          smoothing: 1)
   put: 3
\end{code}

\important{
Les caract\`eres \ct{[}, \ct{]}, et \ct{(}, \ct{)} d\'elimitent des zones distinctes. Dans ces zones, un message \`a mots-cl\'es est la plus longue s\'equence de mots qui se termine par (\ct{\:}) qui n'est pas coup\'e par les carac\`eres (\ct{.}),  ou \ct{\;}. 
Lorsque les caract\`eres \ct{[}, \ct{]}, et \ct{(}, \ct{)} entourent des mots avec des deux points, ces mots participent au message \`a mots-cl\'es local \`a cette zone.}

\on{Sounds terribly complicated.} %martial: d'accord

\hint{Si vous avez des probl\`emes avec ces r\`egles de priorit\'e, vous pouvez commencer simplement en entourant avec des parenth\`eses chaque fois que vous voulez distinguer deux messages avec la m\^eme priorit\'e.}

L'expression qui suit ne n\'ecessite pas de parenth\`eses car l'expression \ct{x isNil} est unaire donc envoy\'ee avant le message \`a mots-cl\'es \mbox{\lct{ifTrue:}.}
\begin{code}{}
(x isNil)
   ifTrue:[...]
\end{code}

L'expression qui suit n\'ecessite des parenth\`eses car les messages \ct{includes:} et \ct{ifTrue:} sont chacun des messages \`a mots-cl\'es. 
\begin{code}{}
ord := OrderedCollection new.
(ord includes: $a)
   ifTrue:[...]
\end{code}%$

\noindent
Sans les parenth\`eses le message inconnu \ct{includes:ifTrue:} serait envoy\'e \`a la collection!

%-------------------------------------------------------------
\subsection{Quand utiliser les \lct{[ ]} ou les \lct{( )} ?}

Vous pouvez avoir des difficult\'es \`a comprendre quand utiliser des crochets plut\^ot que des parenth\`eses.
Le principe de base est que vous devez utiliser des \ct{[ ]} lorsque vous ne savez pas combien de fois une expression peut \^etre \'evalu\'ee (peut-\^etre m\^eme jamais).
\lct{[\emph{expression}]} va cr\'eer une fermeture lexicale ou
\ind{bloc} (\ie un objet) \`a partir de
\mbox{\lct{\emph{expression}},} qui peut \^etre \'evalu\'ee autant de
fois qu'il le faut (voire jamais) en fonction du contexte.

Ainsi les clauses conditionnelles de \ct{ifTrue:} ou \ct{ifTrue:ifFalse:} n\'ecessitent des blocs. Suivant le m\^eme principe, \`a la fois le receveur et l'argument du message \ct{whileTrue:} n\'ecessitent l'utilisation des crochets car nous ne savons pas combien de fois le receveur ou l'argument seront ex\'ecut\'es.

Les parenth\`eses quant \`a elles n'affectent que l'ordre d'envoi des messages.
Aucun objet n'est cr\'e\'e, ainsi dans \lct{(\emph{expression})},
\lct{\emph{expression}} sera \emph{toujours} \'evalu\'e exactement une
fois 
%martial: erreur dans l'original: (en supposant que le code du son est
%\'evalu\'e une fois). En fait, il ne s'agit pas de 'sounding' mais
%'surrounding'
(en supposant que le code englobant l'expression soit \'evalu\'e une
fois).

\begin{code}{}
[ x isReady ] whileTrue: [ y doSomething ]   "!\`a! la fois le receveur et l'argument doivent !\^etre! des blocs"
4 timesRepeat: [ Beeper beep ]                   "l'argument est !\'evalu\'e! plus d'une fois, donc doit !\^etre! un bloc"
(x isReady) ifTrue: [ y doSomething ]           "le receveur est !\'evalu\'e! qu'une fois, donc n'est pas un bloc!"
\end{code}

%=============================================================
\section{S\'equences d'expression}
Les expressions (\ie envois de message, affectations\ldots) s\'epar\'ees par des points sont \'evalu\'ees en s\'equence.
Notez qu'il n'y a pas de point entre la d\'efinition d'un variable et l'expression qui suit.
La valeur d'une s\'equence est la valeur de la derni\`ere
expression. Les valeurs retourn\'ees par toutes les expressions
except\'ee la derni\`ere sont ignor\'ees. Notez que le point est un 
%\subind{statement}{s\'eparateur}
\subind{expression}{séparateur}
et non un terminateur d'expression. Le point final est donc optionnel.
\seeindex{séparateur}{expression, séparateur}

\begin{code}{@TEST}
| box |
box := 20@30 corner: 60@90.
box containsPoint: 40@50 --> true
\end{code}

%=============================================================
\section{Cascades de messages}
\st offre la possibilit\'e d'envoyer plusieurs messages au m\^eme
receveur en utilisant le point-virgule (\ct{;}). Dans le jargon \st,
nous parlons de \emphind{cascade}.
\seeindex{message!cascade}{cascade}

\important{Expression Msg1 ; Msg2}

\begin{minipage}{0.35\textwidth}
\begin{code}{}
Transcript show: 'Pharo est '.
Transcript show: 'extra '.
Transcript cr.
\end{code}
\end{minipage}
\emph{~est \'equivalent \`a :~}
\begin{minipage}{0.35\textwidth}
\begin{code}{}
Transcript        
   show: 'Pharo est';
   show: 'extra ';
   cr
\end{code}
\end{minipage}

Notez que l'objet qui re\c{c}oit la cascade de messages peut \'egalement \^etre le r\'esultat d'un envoi de message.
En fait, le receveur de la cascade est le receveur du premier message
de la cascade. Dans l'exemple qui suit, le premier message en cascade
est \ct{setX:setY} puisqu'il est suivi du point-virgule. Le receveur
du message cascad\'e \ct{setX:setY:} est le nouveau point r\'esultant
de l'\'evaluation de \ct{Point new}, et \emph{non pas} \ct{Point}. Le
message qui suit \ct{isZero} (pour tester s'il s'agit de z\'ero) est
envoy\'e au m\^eme receveur. 

\begin{code}{}
Point new setX: 25 setY: 35; isZero --> false
\end{code}

%=============================================================
\section{R\'esum\'e du chapitre}

\begin{itemize}
\item Un message est toujours envoy\'e \`a un objet nomm\'e le \emph{receveur} qui peut \^etre le r\'esultat d'autres envois de messages.

\item Les messages unaires sont des messages qui ne n\'ecessitent pas d'arguments.\\
Ils sont de la forme \lct{receveur \textbf{s\'electeur}}.

\item Les messages binaires sont des messages qui concernent deux objets, le receveur et un autre objet \emph{et} dont le s\'electeur est compos\'e de un ou deux caract\`eres de la liste suivante: \ct{+}, \ct{-}, \ct{*}, \ct{/}, \ct{|}, \texttt{\&}, \ct{=}, \ct{>}, \ct{<}, \texttt{\~}, et \ct{@}.\\
Ils sont de la forme: \lct{receveur \textbf{s\'electeur} argument}.
\item Les messages \`a mots-cl\'es sont des messages qui concernent plus d'un objet et qui contiennent au moins un caract\`ere deux points (\ct{:}).\\
Ils sont de la forme: 
\lct{receveur \textbf{s\'electeurMotUn:} argumentUn \textbf{motDeux:} argumentDeux}.

\item \textbf{R\`egle un.} Les messages unaires sont d'abord envoy\'es, puis les messages binaires et finalement les messages \`a mots-cl\'es.
\item \textbf{R\`egle deux.} Les messages entre parenth\`eses sont envoy\'es avant tous les autres.
\item \textbf{R\`egle trois.} Lorsque les messages sont de m\^eme nature, l'ordre d'\'evaluation est de gauche \`a droite.
\item En \st, les op\'erateurs arithm\'etiques traditionnels comme +
  ou * ont la m\^eme priorit\'e. \ct{+} et \ct{*} ne sont que des
  messages binaires; donc \ct{*} n'a aucune priorit\'e sur
  \ct{+}. Vous devez utiliser les parenth\`eses pour obtenir un
  r\'esultat diff\'erent.
\end{itemize}

%\end{document}
% ON: Don't ever put an \end{document} in a chapter
% It will make the book stop there!
%=================================================================
\ifx\wholebook\relax\else\end{document}\fi
%=================================================================

%---------------------------------------------------------
