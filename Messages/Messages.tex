% $Author$ traducteur Serge
% $Date$ 
% $Revision$
% relecture par Martial Boniou: Thu Dec 13 22:40:00 CET 2007
% Une instance de Pen est appele 'crayon' (et non plus, crayon ou
% stylo...)
% relecture par Rene Mages : Thu Dec 20 12:13:40 2007
% relecture par Rene Mages : Thu Jan 10 12:11:33 2008
% adaptation pour PBE - martial - Fri Sep 11 17:52:08 CEST 2009 
% relecture par Rene Mages : Sun Jan 10 12:11:33 2010
% relecture par Rene Mages : Mon Apr 11 12:11:33 2011
% relecture par Rene Mages : Tue May  3 12:11:33 2011
% $Author: oscar $ % $Date: 2009-08-27 12:59:05 +0200 (Thu, 27 Aug 2009) $ 
%=================================================================
\ifx\wholebook\relax\else
% --------------------------------------------
% Lulu:
	\documentclass[a4paper,10pt,twoside]{book}
	\usepackage[
		papersize={6.13in,9.21in},
		hmargin={.75in,.75in},
		vmargin={.75in,1in},
		ignoreheadfoot
	]{geometry}
	\input{../common.tex}
	\pagestyle{headings}
	\setboolean{lulu}{true}
% --------------------------------------------
% A4:
%	\documentclass[a4paper,11pt,twoside]{book}
%	\input{../common.tex}
%	\usepackage{a4wide}
% --------------------------------------------
    \graphicspath{{figures/} {../figures/}}
	\begin{document}
	\renewcommand{\nnbb}[2]{} % Disable editorial comments
	\sloppy
\fi
%=================================================================
\chapter{Comprendre la syntaxe des messages}
\chalabel{understanding}

Bien que la syntaxe des messages \st soit extrêmement simple, elle n'est pas 
habituelle pour un développeur qui viendrait du monde C/Java. Un certain temps d'adaptation peut être nécessaire. L'objectif de ce chapitre est de donner quelques conseils pour vous aider à mieux appréhender la syntaxe particulière  des envois de messages.
Si vous vous sentez suffisament en confiance avec la syntaxe, vous pouvez choisir de sauter 
ce chapitre ou bien d'y revenir un peu plus tard.

%=============================================================
\section{Identifier les messages}

Avec \st, exception faite des éléments syntaxiques rencontrés dans
\charef{syntax} (\ct+:= ^ . ; # () {} [ : | ]+), tout se passe par envoi de messages.
Comme en \ind{C++}, vous pouvez définir vos opérateurs comme \ct{+} pour vos propres 
classes, mais tous les opérateurs ont la même précédence.
De plus, il n'est pas possible de changer l'arité d'une méthode:
\ct{-} est toujours un message binaire, et il n'est pas possible
d'avoir une forme unaire avec une surcharge différente.

Avec \st, l'ordre dans lequel les messages sont envoyés est
déterminé par le type de message. Il n'y a que trois formes de
messages : les messages \emphsubind{message}{unaire},
\emphsubind{message}{binaire} et \emphsubind{message}{à mots-clés}.
 Les messages unaires sont toujours envoyés en premier, puis
les messages binaires et enfin ceux à mots-clés. Comme dans la
plupart des langages,  les \ind{parenthèses} peuvent être utilisées 
pour changer l'ordre d'évaluation. Ces règles rendent le code \st 
aussi facile à lire que possible. La plupart du temps, il n'est pas 
nécessaire de réfléchir à ces règles.

Comme la plupart des calculs en \st sont effectués par des envois de 
messages, identifier correctement les messages est crucial. La terminologie 
suivante va nous être utile :

\begin{itemize}
  \item Un message est composé d'un \emphsubind{message}{sélecteur} et d'arguments optionnels,
  \item Un message est envoyé au \emphsubind{message}{receveur},
  \item La combinaison d'un message et de son receveur est appelé un \emphsubind{message}{envoi} \emph{de message}  comme il est montré dans \figref{firstScriptMessage}.
\end{itemize}

\begin{figure}[htb]
\begin{minipage}{0.53\textwidth}
	\begin{center}
	\includegraphics[width=0.95\textwidth]{message}
	\caption{Deux messages composés d'un receveur, d'un sélecteur de méthode et d'un ensemble d'arguments.\figlabel{firstScriptMessage}}\end{center}
\end{minipage}
\hfill
\begin{minipage}{0.43\textwidth}
	\begin{center}
	\ifluluelse
		{\includegraphics[width=0.9\textwidth]{uKeyUnOne}}
		{\includegraphics[width=6cm]{uKeyUnOne}}
	\caption{\ct{aMorph color: Color yellow} est composé de deux expressions : \ct{Color yellow} et \ct{aMorph color: Color yellow}.\figlabel{ellipse}}
	\end{center}
\end{minipage}
\end{figure}


\important{Un message est toujours envoyé à un receveur qui peut être un simple littéral, une variable ou le résultat de l'évaluation d'une autre expression.}

Nous vous proposons de vous faciliter la lecture au moyen d'une
notation graphique: nous soulignerons le receveur afin de vous aider
à l'identifier. Nous entourerons également chaque expression dans
une ellipse et numéroterons les expressions à partir de la
première à être évaluée afin de voir l'ordre d'envoi des messages.

%\begin{figure}[!ht]
%\begin{center}
%\includegraphics[width=6cm]{uKeyUnOne}
%\end{center}
%\caption{\ct{aMorph color: Color yellow} is composed of two expressions: \ct{Color yellow} and \ct{aMorph color: Color yellow}.\figlabel{ellipse}}
%\end{figure}

\Figref{ellipse} représente deux envois de messages, \ct{Color yellow} et \ct{aMorph color: Color yellow}, 
de telle sorte qu'il y a deux ellipses. 
L'expression \ct{Color yellow} est d'abord évaluée
en premier, ainsi son ellipse est numérotée à \ct{1}. Il y a
deux receveurs: \ct{aMorph} qui reçoit le message \ct{color: ...}
et \ct{Color} qui reçoit le message \ct{yellow} 
%ajout
(\emph{yellow} correspond à la couleur jaune en anglais). 
Chacun des receveurs est souligné.

Un receveur peut être le premier élément d'un message, comme
\ct{100} dans l'expression \ct{100 + 200} ou \ct{Color} 
%ajout
(la classe des couleurs)
dans l'expression \ct{Color yellow}. Un objet receveur peut
également être le résultat de l'évaluation d'autres
messages. Par exemple, dans le message \ct{Pen new go: 100}, le
receveur de ce message \ct{go: 100} 
%ajout
(littéralement, aller à 100)
est l'objet retourné par cette expression \ct{Pen new} 
%ajout
(soit une instance de \ct{Pen}, la classe crayon). Dans tous les cas,
le message est envoyé à un objet appelé le \emph{receveur} qui a
pu être créé par un autre envoi de message.

\begin{table}\centering
	\begin{tabularx}{\linewidth}{llX}
		\toprule
		Expression & Type de messages & Résultat \\
		\midrule
		\lct{Color yellow}
			& unaire
			& Crée une couleur.
		\\
		\lct{aPen  go: 100}
			& à mots-clés
			& Le crayon receveur se déplace en avant de 100 pixels.
		\\
		\lct{100 + 20}
			& binaire
			& Le nombre 100 reçoit le message + avec le paramètre 20.
		\\
		\lct{Browser open}
			& unaire
			& Ouvre un nouveau navigateur de classes.
		\\
		\lct{Pen new  go: 100}
			& unaire et à mots-clés
			& Un crayon est créé puis déplacé de 100 pixels.
		\\
		\lct{aPen go: 100 + 20}
			& à mots-clés et binaire
			& Le crayon receveur se déplace vers l'avant de 120 pixels.
		\\
		\bottomrule
	\end{tabularx}
	\caption{Exemples de messages}\tablabel{messageExamples}
\end{table}

\Tabref{messageExamples} montre différents exemples de messages.
Vous devez remarquer que tous les messages n'ont pas obligatoirement
d'arguments. Un message unaire comme \ct{open} (pour ouvrir) ne nécessite pas d'arguments. Les messages à mots-clés simples ou les messages binaires comme \ct{go: 100} et \ct{+ 20} ont chacun un argument. 
Il y a aussi des messages simples et des messages
composés. \ct{Color yellow} et \ct{100 + 20} sont simples: un
message est envoyé à un objet, tandis que l'expression \ct{aPen go: 100 + 20} est composée de deux messages: \ct{+ 20} est
envoyé à \ct{100} et \ct{go:} est envoyé à \ct{aPen} avec pour
argument le résultat du premier message.
Un receveur peut être une expression qui peut retourner un
objet. Dans \ct{Pen new go: 100}, le message \ct{go: 100} est envoyé
à l'objet qui résulte de l'évaluation de l'expression \ct{Pen new}.

% ON: An enumerated list here is overkill!
%=============================================================
\section{Trois sortes de messages}

\st utilise quelques règles simples pour déterminer l'ordre dans lequel 
les messages sont envoyés. Ces règles sont basées sur la distinction établie 
entre les 3 formes d'envoi de messages: 
\begin{itemize}
\item \emph{Les messages unaires} sont des messages qui sont envoyés
  à un objet sans autre information. Par exemple dans \ct{3 factorial}, \ct{factorial} (pour factorielle) est un message    
  unaire. 
\item  \emph{Les messages binaires} sont des messages formés avec
  des opérateurs (souvent arithmétiques). Ils sont binaires car
  ils ne concernent que deux objets: le receveur et l'objet
  argument. Par exemple, dans \ct{10 + 20}, \ct{+} est un message
  binaire qui est envoyé au receveur \ct{10} avec l'argument \ct{20}. 
\item  \emph{Les messages à mots-clés} sont des messages formés avec plusieurs mots-clés, chacun d'entre eux se finissant par deux points (\ct{:}) et prenant un paramètre.
Par exemple, dans \ct{anArray at: 1 put: 10}, le mot-clé \ct{at:}
prend un argument \ct{1} et le mot-clé \ct{put:} prend l'argument \ct{10}.
\end{itemize}

%-------------------------------------------------------------
\subsection{Messages unaires}
Les messages unaires sont des messages qui ne nécessitent aucun
argument. Ils suivent le modèle syntaxique suivant: \ct{receveur nomMessage}. Le sélecteur est constitué d'une série de
caractères ne contenant pas de deux points (\ct{:}) (\eg
\ct{factorial}, \ct{open}, \ct{class}).
\needlines{4}
\begin{code}{@TEST}
89 sin           --> 0.860069405812453
3 sqrt           --> 1.732050807568877
Float pi         --> 3.141592653589793
'blop' size     --> 4
true not        --> false
Object class --> Object class  "La classe de Object est Object class (BANG)"
\end{code}
% ON: I changed the examples to things we can test

\important{Les messages unaires sont des messages qui ne nécessitent pas d'argument.
Ils suivent le moule syntaxique: \lct{receveur \textbf{sélecteur}}}

%-------------------------------------------------------------
\subsection{Messages binaires} 
Les messages binaires sont des messages qui nécessitent exactement un argument \emph{et} dont le sélecteur consiste en une séquence de un ou plusieurs caractères de l'ensemble: \ct{+}, \ct{-}, \ct{*}, \ct{/}, \ct{&}, \ct{=}, \ct{>}, \ct{|}, \ct{<}, \ct{~}, et \ct{@}. Notez que \ct{--} n'est pas autorisé.

\begin{code}{@TEST}
100@100      --> 100@100  "!crée! un objet Point"
3 + 4              --> 7
10 - 1            --> 9
4 <= 3            --> false
(4/3) * 3 = 4   --> true  "!l'égalité! est juste un message binaire et les fractions sont exactes"
(3/4) == (3/4) --> false  "!deux fractions égales ne sont pas le même objet!"
\end{code}

\important{Les messages binaires sont des messages qui nécessitent exactement un argument \emph{et} dont le sélecteur est composé d'une séquence de caractères parmi : \ct{+}, \ct{-}, \ct{*}, \ct{/}, \ct{\&}, \ct{=}, \ct{>}, \ct{|}, \ct{<}, \ct{\~}, et \ct{@}. \ct{--} n'est pas possible.\\
Ils suivent le moule syntaxique: \lct{receveur \textbf{sélecteur} argument}}

%-------------------------------------------------------------
\subsection{Messages à mots-clés}

Les messages à mots-clés sont des messages qui nécessitent un ou plusieurs arguments et 
dont le sélecteur consiste en un ou plusieurs mots-clés se finissant par deux points \ct{:}.  
Les messages à mots-clés suivent le moule syntaxique: 
\lct{receveur \textbf{selecteurUn:} argumentUn \textbf{selecteurDeux:} argumentDeux}

Chaque mot-clé utilise un argument. Ainsi \ct{r:g:b:} est une
méthode avec 3 arguments, \ct{playFileNamed:} et \ct{at:} sont des
méthodes avec un argument, et \ct{at:put:} est une méthode avec
deux arguments. Pour créer une instance de la classe \ct{Color} on
peut utiliser la méthode \ct{r:g:b:} comme dans \ct{Color r: 1 g: 0 b: 0} 
créant ainsi la couleur rouge. Notez que les deux points ne font pas partie du sélecteur.

\important{En \ind{Java} ou \ind{C++}, l'invocation de méthode \st \ct{Color r: 1 g: 0 b: 0} serait écrite \ct{Color.rgb(1,0,0)}.}

\begin{code}{@TEST | nums |}
1 to: 10                        --> (1 to: 10)  "!création! d'un intervalle"
Color r: 1 g: 0 b: 0       --> Color red  "!création! d'une nouvelle couleur (rouge)"
12 between: 8 and: 15 --> true

nums := Array newFrom: (1 to: 5).
nums at: 1 put: 6.
nums --> #(6 2 3 4 5)
\end{code}
% ON: Changed to real examples that we can test

\important{Les messages basés sur les mots-clés sont des messages qui nécessitent un ou plusieurs arguments. Leurs sélecteurs consistent en un ou plusieurs mots-clés chacun se terminant par deux points (\ct{\:}). Ils suivent le moule syntaxique:\\
\lct{receveur \textbf{selecteurUn:} argumentUn \textbf{selecteurDeux:} argumentDeux}}

%=============================================================
\section{Composition de messages}
Les trois formes d'envoi de messages ont chacune des priorités différentes, ce qui permet de les composer de manière élégante.

\begin{enumerate}
\item Les messages unaires sont envoyés en premier, puis les messages binaires et enfin les messages à mots-clés.
\item Les messages entre \ind{parenthèses} sont envoyés avant tout autre type de messages. 
\item Les messages de même type sont envoyés de gauche à droite. 
\end{enumerate}
\index{message!ordre d'évaluation}

Ces règles ont un ordre de lecture très naturel. Maintenant si
vous voulez être sûr que vos messages sont envoyés dans l'ordre
que vous souhaitez, vous pouvez toujours mettre des parenthèses
supplémentaires comme dans \figref{uKeyUn}. Dans cet exemple, le
message \ct{yellow} est un message unaire et le message \ct{color:}
est un message à mots-clés; ainsi l'expression \ct{Color yellow}
est envoyé en premier. Néanmoins comme les expressions entre
parenthèses sont envoyées en premier, mettre des parenthèses
(normalement inutiles) autour de \ct{Color yellow} permet d'accentuer
le fait qu'elle
%l'expression
doit être envoyée en premier. Le reste de cette section illustre
chacun de ces différents points.

\begin{figure}[ht]
\ifluluelse
	{\centerline{\includegraphics[width=0.9\textwidth]{uKeyUn}} }
	{\centerline{\includegraphics[width=10cm]{uKeyUn}} }
\caption{Les messages unaires sont envoyés en premier; donc ici le
  premier message est \ct{Color yellow}. Il retourne un objet de
  couleur jaune qui est passé comme argument du message \ct{aPen color:}.\figlabel{uKeyUn}}
\end{figure}

%---------------------------------------------------------
\subsection*{Unaire > Binaire > Mots-clés}
Les messages unaires sont d'abord envoyés, puis les messages
binaires et enfin les messages à mots-clés. Nous pouvons
également dire que les messages unaires ont une priorité plus
importante que les autres types de messages.

\important{\textbf{Règle  une.} Les messages unaires sont envoyés en premier, puis les messages binaires et finalement les messages à mots-clés.\\
\centerline{\lct{Unaire > Binaire > Mots-clés}}
}

Comme ces exemples suivants le montrent, les règles de syntaxe de
\st permettent d'assurer une certaine lisibilité des expressions:
\begin{code}{@TEST}
1000 factorial / 999 factorial --> 1000
2 raisedTo: 1 + 3 factorial     --> 128
\end{code}
\noindent

Malheureusement, les règles sont un peu trop simplistes pour les
expressions arithmétiques. Dès lors, des parenthèses doivent
être introduites chaque fois que l'on veut imposer un ordre de
priorité entre deux opérateurs binaires:
\begin{code}{@TEST}
1 + 2 * 3   --> 9
1 + (2 * 3) --> 7
\end{code}

L'exemple suivant qui est un peu plus complexe (!) est l'illustration que même des expressions \st compliquées peuvent être lues de manière assez naturelle: 
\begin{code}{@TEST}
[:aClass | aClass methodDict keys select: [:aMethod | (aClass>>aMethod) isAbstract ]] value: Boolean --> an IdentitySet(#or: #| #and: #& #ifTrue: #ifTrue:ifFalse: #ifFalse: #not #ifFalse:ifTrue:)
\end{code}
% note de martial: j'ai ajoute entre parentheses le nom des messages
% pour plus de clarte
\noindent
Ici nous voulons savoir quelles méthodes de la classe \ct{Boolean}
(classe des booléens) sont abstraites.
Nous interrogeons la classe argument \ct{aClass} pour récupérer
les clés (via le message unaire \ct{keys}) de son dictionnaire de
méthodes (via le message unaire \ct{methodDict}), puis nous en
sélectionnons (via le message à mots-clés \ct{select:}) les
méthodes de la classe qui sont abstraites.
Ensuite nous lions (par \ct{value:}) l'argument \ct{aClass} à la
valeur concrète \ct{Boolean}.
Nous avons besoin des parenthèses uniquement pour le message binaire
\ct{>>}, qui sélectionne une méthode d'une classe, avant d'envoyer
le message unaire \mbox{\ct{isAbstract}} à cette méthode. Le
résultat (sous la forme d'un ensemble de classe \ct{IdentifySet})
nous montre quelles méthodes doivent être implémentées par les
sous-classes concrètes de \ct{Boolean}: \ct{True} et \ct{False}.


\paragraph{Exemple.}
Dans le message \ct{aPen color: Color yellow}, il y a un message \emph{unaire} \ct{yellow} 
envoyé à la classe \ct{Color} et un message à \emph{mots-clés} \ct{color:} envoyé à \ct{aPen}. 
Les messages unaires sont d'abord envoyés, de telle sorte que l'expression \ct{Color yellow} 
soit d'abord exécutée (1). Celle-ci retourne un objet couleur qui est passé en argument du 
message \ct{aPen color: aColor} (2) comme indiqué dans l'\egref{decColor}.
\Figref{uKeyUn} montre graphiquement comment les messages sont envoyés.

\needlines{5}
\begin{example}[decColor]{Décomposition de l'évaluation de \ct{aPen color: Color yellow}}{}
        aPen color: Color yellow
(1)                       Color yellow        "message unaire !envoyé! en premier"
                        --> aColor
(2)   aPen color: aColor                 "puis le message !à mots-clés!"
\end{example}

\paragraph{Exemple.} Dans le message \ct{aPen go: 100 + 20}, il y a le message \emph{binaire} \ct{+ 20} et un message à \emph{mots-clés} \ct{go:}. Les messages binaires sont d'abord envoyés avant les messages à mots-clés, ainsi \ct{100 + 20} est envoyé en premier (1): le message \ct{+ 20} est envoyé à l'objet \ct{100} et retourne le nombre \ct{120}. Ensuite le message \ct{aPen go: 120} est envoyé avec comme argument \ct{120} (2).
L'\egref{decGo} nous montre comment l'expression est évaluée. 

\begin{example}[decGo]{Décomposition de \ct{aPen go: 100 + 20}}{}
      aPen go: 100 + 20   
(1)                 100 + 20           "le message binaire en premier"
                   -->   120
(2)  aPen go: 120                   "puis le message !à mots-clés!"
\end{example}

\begin{figure}[htb]
\begin{minipage}{0.48\textwidth}
	\ifluluelse
		{\centerline{\includegraphics[width=0.9\textwidth]{uKeyBin}}}
		{\centerline{\includegraphics[width=6cm]{uKeyBin}}}
	\caption{Les messages unaires sont envoyés en premier, ainsi
      \ct{Color yellow} est d'abord envoyé. Il retourne un objet de
      couleur jaune qui est passé en argument du message \ct{aPen color:}.\figlabel{uKeyBin}}
\end{minipage}
\hfill
\begin{minipage}{0.48\textwidth}
	\begin{center}
	\ifluluelse
		{\includegraphics[width=0.9\textwidth]{uunKeyBin}}
		{\includegraphics[width=6cm]{uunKeyBin}}
\caption{Décomposition de \ct{Pen new go: 100 + 20}.\figlabel{unKeyBin}}
\end{center}
\end{minipage}
\end{figure}

\paragraph{Exemple.} Comme exercice, nous vous laissons décomposer
l'évaluation du message \ct{Pen new go: 100 + 20} qui est composé
d'un message unaire, d'un message à mots-clés et d'un message
binaire (voir \figref{unKeyBin}).

%-------------------------------------------------------------
\subsection{Les parenthèses en premier}

\important{\textbf{Règle deux.} Les messages parenthésés sont envoyés avant tout autre message.\\
\centerline{\lct{(Msg) > Unaire > Binaire > Mots-clés}}}

\begin{code}{@TEST}
1.5 tan rounded asString = (((1.5 tan) rounded) asString) --> true    "les !parenthèses! sont !nécessaires! ici"
3 + 4 factorial   --> 27    "(et pas 5040)"
(3 + 4) factorial --> 5040
\end{code}

Ici nous avons besoin des \ind{parenthèses} pour forcer l'envoi de \ct{lowMajorScaleOn:} avant \ct{play}.
\begin{code}{}
(FMSound lowMajorScaleOn: FMSound clarinet) play 
"(1) envoie le message clarinet !à! la classe FMSound pour !créer! le son de clarinette.
 (2) envoie le son !à! FMSound comme argument du message !à! !mots-clés! lowMajorScaleOn:.
 (3) joue le son !résultant!."
\end{code}

% ON: This has nothing to do with parentheses!
%RecordingControlsMorph new openInWorld
%"An instance of the digitizer is created then visualized. If your microphone is plugged in try a sampleBANG"

% ON: This link is broken, and the result does not understand display!
%(HTTPSocket httpShowGif:
%   'www.altavista.digital.com/av/pix/default/av-adv.gif') display

\paragraph{Exemple.}
Le message \ct{(65@325 extent: 134@100) center} retourne le centre
du rectangle dont le point supérieur gauche est $(65, 325)$ et dont
la taille est $134{\times}100$. L'\egref{decExtent} montre comment le
message est décomposé et envoyé. Le message entre parenthèses
est d'abord envoyé: il contient deux messages binaires \ct{65@325}
et \ct{134@100} qui sont d'abord envoyés et qui retournent des
points, et un message à mots-clés \ct{extent:} qui est ensuite
envoyé et qui retourne un rectangle. Finalement le message unaire
\ct{center} est envoyé au rectangle et le point central est retourné.

Évaluer ce message sans parenthèses déclencherait une erreur car
l'objet \ct{100} ne comprend pas le message \ct{center}.

\needlines{9} % CHANGE REVOIR
\begin{example}[decExtent]{Exemple avec des parenthèses.}{}
      (65@325 extent: 134@100) center
(1)   65@325                                                    "binaire"
    --> aPoint
(2)                                134@100                     "binaire"
                                 --> anotherPoint
(3)   aPoint extent: anotherPoint                       "!à mots-clés!"
      --> aRectangle
(4)   aRectangle center                                     "unaire"
      --> 132@375
\end{example}

\subsection{De gauche à droite}
Maintenant nous savons comment les messages de différentes natures
ou priorités sont traités. Il reste une question à traiter:
comment les messages de même priorité sont envoyés? Ils sont
envoyés de gauche à droite. Notez que vous avez déjà vu ce
comportement dans l'\egref{decExtent} dans lequel les deux messages de
création de points (\ct{@}) sont envoyés en premier.

\important{{\textbf{Règle trois.} Lorsque les messages sont de même nature, l'ordre d'évaluation est de gauche à droite.}}

%\begin{figure}
%\centerline{\includegraphics[width=8cm]{ucompoUn}} 
%\caption{The message \ct{Pen new east} is composed of two unary messages. Therefore the leftmost one, \ct{new},  is sent and it returns a new robot to which the second message \ct{east} is sent. \figlabel{compoUn}}
%\end{figure}

\paragraph{Exemple.} Dans l'expression \ct{Pen new down}, tous les
messages sont des messages unaires, donc celui qui est le plus à
gauche \ct{Pen new} est envoyé en premier. Il retourne un nouveau
crayon auquel le deuxième message \ct{down} 
%ajout
(pour poser la pointe du crayon et dessiner)
est envoyé comme il est montré dans \figref{unaryMessages}.

\begin{figure}
	\centering
	\includegraphics[width=8cm]{ucompoUn}
	\caption{Décomposition de \ct{Pen new down}.\figlabel{unaryMessages}}
\end{figure}

%-------------------------------------------------------------
%\subsection{Inconsistances arithmétiques}
% note de martial: j'ai fait des recherches; c'est plus correct
% qu'inconsistence; le vrai terme est irrationnalite
\subsection{Incohérences arithmétiques}
Les règles de composition des messages sont simples mais peuvent
engendrer des incohérences dans l'évaluation des expressions
arithmétiques qui sont exprimées sous forme de messages binaires
%ajout
(nous parlons aussi d'irrationnalité arithmétique).
Voici des situations habituelles où des parenthèses supplémentaires sont nécessaires.

\begin{code}{@TEST}
3 + 4 * 5      --> 35    "(pas 23)  les messages binaires sont !envoyés! de gauche !à! droite"
3 + (4 * 5)    --> 23
1 + 1/3         --> (2/3)    "et pas 4/3"
1 + (1/3)       --> (4/3)
1/3 + 2/3       --> (7/9)    "et pas 1"
(1/3) + (2/3)  --> 1
\end{code}

\paragraph{Exemple.} 
Dans l'expression \ct{20 + 2 * 5}, il y a seulement les messages
binaires \ct{+} et \ct{*}. En \st, il n'y a pas de priorité
spécifique pour les opérations \ct{+} et \ct{*}. Ce ne sont que
des messages binaires, ainsi \ct{*} n'a pas priorité sur \ct{+}. Ici
le message le plus à gauche \ct{+} est envoyé en premier (1) et
ensuite \ct{*} est envoyé au résultat comme nous le voyons dans l'\egref{binaryMessages1}.  

\begin{example}[binaryMessages1]{Décomposer \ct{20 + 2 * 5}}{}
"Comme il n'y a pas de !priorité! entre les messages binaires, le message le plus !à! gauche, + est !évalué! en premier !même! si !d'après! les !règles! de !l'arithmétique! le * devrait d'abord !être! !envoyé.!"

      20 + 2 * 5 
(1)  20 + 2 --> 22
(2)  22       * 5 --> 110
\end{example}

\begin{figure}
\begin{center}\includegraphics[width=8cm]{ucompoNoBracketPar}\end{center}
\end{figure}
\noindent
Comme il est montré dans l'\egref{binaryMessages1} le résultat de
cette expression n'est pas \ct{30} mais \ct{110}. Ce résultat est
peut-être inattendu mais résulte directement des règles
utilisées pour envoyer des messages. Ceci est le prix à payer pour
la simplicité du modèle de \st. Afin d'avoir un résultat
correct, nous devons utiliser des parenthèses. Lorsque les messages
sont entourés par des parenthèses, ils sont évalués en
premier. Ainsi l'expression \ct{20 + (2 * 5)} retourne le résultat
comme nous le voyons dans l'\egref{mathcorrect}.

\needlines{4}
\begin{example}[mathcorrect]{Décomposition de \ct{20 + (2 * 5)}}{}
"Les messages !entourés! de !parenthèses! sont !évalués! en premier ainsi * est !envoyé! avant + afin de produire le comportement !souhaité.!"

    20 + (2 * 5) 
(1)        (2 * 5) --> 10
(2) 20 + 10      --> 30
\end{example}

\begin{figure}
\begin{center}
\includegraphics[width=8cm]{ucompoNumberBracket}
\end{center}
\end{figure}

\important{En \st, les opérateurs arithmétiques comme \ct{+} et \ct{*} n'ont
  pas des priorités différentes. \ct{+} et \ct{*} ne sont que des
  messages binaires; donc \ct{*} n'a pas priorité sur
  \ct{+}. Utiliser des parenthèses pour obtenir le résultat désiré.}

%  At the beginning put parenthesis when you have multiple binary messages.}  HUH?  At the beginning of what?!

\begin{figure}
\begin{center}
\ifluluelse
	{\includegraphics[width=\textwidth]{uKeyUnBinPar}}
	{\includegraphics[width=0.8\textwidth]{uKeyUnBinPar}}
\ifluluelse
	{\includegraphics[width=\textwidth]{uunKeyBinPar}}
	{\includegraphics[width=10cm]{uunKeyBinPar}}
\end{center}
\caption{Messages équivalents en utilisant des parenthèses.\figlabel{uKeyUnBinPar}}
\end{figure}

Notez que la première règle, disant que les messages unaires sont
envoyés avant les messages binaires ou à mots-clés, ne nous force
pas à mettre explicitement des parenthèses autour
d'eux. \Tabref{expressions} montre des expressions écrites en
respectant les règles et les expressions équivalentes si les
règles n'existaient pas. Les deux versions engendrent le même
effet et retournent les mêmes valeurs.

\begin{figure}\centering
	\begin{tabular}{l@{\qquad}l}
	\toprule
	Priorité implicite & Équivalent explicite parenthésé\\
	\midrule
	\lct{aPen color: Color yellow}
		& \lct{aPen color: (Color yellow)}
		\\
	\lct{aPen go: 100 + 20}
		& \lct{aPen go: (100 + 20)}
		\\
	\lct{aPen penSize: aPen penSize + 2}
		& \lct{aPen penSize: ((aPen penSize) + 2)}
		\\
	\lct{2 factorial + 4}
		& \lct{(2 factorial) + 4}
		\\
	\bottomrule
	\end{tabular}
	\caption{Des expressions et leurs versions équivalentes complètement parenthésées.\tablabel{expressions}}
\end{figure}

%=============================================================
\section{Quelques astuces pour identifier les messages à mots-clés}
Souvent les débutants ont des problèmes pour comprendre quand ils doivent ajouter des parenthèses. Voyons comment les messages à mots-clés sont reconnus par le compilateur.

%-------------------------------------------------------------
\subsection{Des parenthèses ou pas ?}
Les caractères \ct{[}, \ct{]}, and \ct{(}, \ct{)} 
délimitent des zones distinctes. Dans ces zones, un message à mots-clés est la plus longue séquence de mots terminés par (\ct{:}) qui n'est pas coupé par les caractères (\ct{.}), ou (\ct{;}). 
Lorsque les caractères \ct{[}, \ct{]}, et \ct{(}, \ct{)} entourent des mots avec des deux points, ces mots participent au message à mots-clés \emph{local} à la zone définie.

Dans cet exemple, il y a deux mots-clés distincts: \ct{rotatedBy:magnify:smoothing:} et \ct{at:put:}.

\begin{code}{}
aDict
   at: (rotatingForm 
          rotateBy: angle	
          magnify: 2 
          smoothing: 1)
   put: 3
\end{code}

\important{
Les caractères \ct{[}, \ct{]}, et \ct{(}, \ct{)} délimitent des zones distinctes. Dans ces zones, un message à mots-clés est la plus longue séquence de mots qui se termine par (\ct{\:}) qui n'est pas coupé par les caractères (\ct{.}),  ou \ct{\;}. 
Lorsque les caractères \ct{[}, \ct{]}, et \ct{(}, \ct{)} entourent des mots avec des deux points, ces mots participent au message à mots-clés local à cette zone.}

\on{Sounds terribly complicated.} %martial: d'accord

\hint{Si vous avez des problèmes avec ces règles de priorité, vous pouvez commencer simplement en entourant avec des parenthèses chaque fois que vous voulez distinguer deux messages avec la même priorité.}

L'expression qui suit ne nécessite pas de parenthèses car l'expression \ct{x isNil} est unaire donc envoyée avant le message à mots-clés \mbox{\lct{ifTrue:}.}
\begin{code}{}
(x isNil)
   ifTrue:[...]
\end{code}

L'expression qui suit nécessite des parenthèses car les messages \ct{includes:} et \ct{ifTrue:} sont chacun des messages à mots-clés. 
\begin{code}{}
ord := OrderedCollection new.
(ord includes: $a)
   ifTrue:[...]
\end{code}%$

\noindent
Sans les parenthèses le message inconnu \ct{includes:ifTrue:} serait envoyé à la collection!

%-------------------------------------------------------------
\subsection{Quand utiliser les \ct{[ ]} ou les \ct{( )} ?}

Vous pouvez avoir des difficultés à comprendre quand utiliser des crochets plutôt que des parenthèses.
Le principe de base est que vous devez utiliser des \ct{[ ]} lorsque vous ne savez pas combien de fois une expression peut être évaluée (peut-être même jamais).
\lct{[\emph{expression}]} va créer une fermeture lexicale ou
\ind{bloc} (\ie un objet) à partir de
\mbox{\lct{\emph{expression}},} qui peut être évaluée autant de
fois qu'il le faut (voire jamais) en fonction du contexte.

Ainsi les clauses conditionnelles de \ct{ifTrue:} ou \ct{ifTrue:ifFalse:} nécessitent des blocs. Suivant le même principe, à la fois le receveur et l'argument du message \ct{whileTrue:} nécessitent l'utilisation des crochets car nous ne savons pas combien de fois le receveur ou l'argument seront exécutés.

Les parenthèses quant à elles n'affectent que l'ordre d'envoi des messages.
Aucun objet n'est créé, ainsi dans \lct{(\emph{expression})},
\lct{\emph{expression}} sera \emph{toujours} évalué exactement une
fois 
%martial: erreur dans l'original: (en supposant que le code du son est
%évalué une fois). En fait, il ne s'agit pas de 'sounding' mais
%'surrounding'
(en supposant que le code englobant l'expression soit évalué une
fois).

\begin{code}{}
[ x isReady ] whileTrue: [ y doSomething ]   "!à! la fois le receveur et l'argument doivent !être! des blocs"
4 timesRepeat: [ Beeper beep ]                   "l'argument est !évalué! plus d'une fois, donc doit !être! un bloc"
(x isReady) ifTrue: [ y doSomething ]           "le receveur est !évalué! qu'une fois, donc n'est pas un bloc!"
\end{code}

%=============================================================
\section{Séquences d'expression}
Les expressions (\ie envois de message, affectations, \ldots) séparées par des points sont évaluées en séquence.
Notez qu'il n'y a pas de point entre la définition d'un variable et l'expression qui suit.
La valeur d'une séquence est la valeur de la dernière
expression. Les valeurs retournées par toutes les expressions
exceptée la dernière sont ignorées. Notez que le point est un 
%\subind{statement}{séparateur}
\subind{expression}{séparateur}
et non un terminateur d'expression. Le point final est donc optionnel.
\seeindex{séparateur}{expression, séparateur}

\begin{code}{@TEST}
| box |
box := 20@30 corner: 60@90.
box containsPoint: 40@50 --> true
\end{code}

%=============================================================
\section{Cascades de messages}
\st offre la possibilité d'envoyer plusieurs messages au même
receveur en utilisant le point-virgule (\ct{;}). Dans le jargon \st,
nous parlons de \emphind{cascade}.
\seeindex{message!cascade}{cascade}

\important{Expression Msg1 ; Msg2}

\begin{minipage}{0.35\textwidth}
\begin{code}{}
Transcript show: 'Pharo est '.
Transcript show: 'extra '.
Transcript cr.
\end{code}
\end{minipage}
\emph{~est équivalent à :~}
\begin{minipage}{0.35\textwidth}
\begin{code}{}
Transcript        
   show: 'Pharo est';
   show: 'extra ';
   cr
\end{code}
\end{minipage}

Notez que l'objet qui reçoit la cascade de messages peut également être le résultat d'un envoi de message.
En fait, le receveur de la cascade est le receveur du premier message
de la cascade. Dans l'exemple qui suit, le premier message en cascade
est \ct{setX:setY} puisqu'il est suivi du point-virgule. Le receveur
du message cascadé \ct{setX:setY:} est le nouveau point résultant
de l'évaluation de \ct{Point new}, et \emph{non pas} \ct{Point}. Le
message qui suit \ct{isZero} (pour tester s'il s'agit de zéro) est
envoyé au même receveur. 

\begin{code}{}
Point new setX: 25 setY: 35; isZero --> false
\end{code}
%=============================================================
\section{Résumé du chapitre}

\begin{itemize}
\item Un message est toujours envoyé à un objet nommé le \emph{receveur} qui peut être le résultat d'autres envois de messages.

\item Les messages unaires sont des messages qui ne nécessitent pas d'arguments. Ils sont de la forme \lct{receveur \textbf{sélecteur}}.

\item Les messages binaires sont des messages qui concernent deux objets, le receveur et un autre objet \emph{et} dont le sélecteur est composé de un ou deux caractères de la liste suivante: \ct{+}, \ct{-}, \ct{*}, \ct{/}, \ct{|}, \texttt{\&}, \ct{=}, \ct{>}, \ct{<}, \texttt{\~}, et \ct{@}.\\
Ils sont de la forme: \lct{receveur \textbf{sélecteur} argument}.

\item Les messages à mots-clés sont des messages qui concernent plus d'un objet et qui contiennent au moins un caractère deux points (\ct{:}).\\
Ils sont de la forme: 
\lct{receveur \textbf{selecteurUn:} argumentUn \textbf{selecteurDeux:} argumentDeux}


\item \textbf{Règle un.} Les messages unaires sont d'abord envoyés, puis les messages binaires et finalement les messages à mots-clés.

\item \textbf{Règle deux.} Les messages entre parenthèses sont envoyés avant tous les autres.

\item \textbf{Règle trois.} Lorsque les messages sont de même nature, l'ordre d'évaluation est de gauche à droite.

\item En \st, les opérateurs arithmétiques traditionnels comme \ct{+}
  ou \ct{*} ont la même priorité. \ct{+} et \ct{*} ne sont que des
  messages binaires; donc \ct{*} n'a aucune priorité sur
  \ct{+}. Vous devez utiliser les parenthèses pour obtenir un
  résultat différent.
\end{itemize}

%\end{document}
% ON: Don't ever put an \end{document} in a chapter
% It will make the book stop there!
%=================================================================
\ifx\wholebook\relax\else\end{document}\fi
%=================================================================

%---------------------------------------------------------
