% $Author: oscar $
% $Date: 2007-06-06 15:10:38 +0200 (Wed, 06 Jun 2007) $
% $Revision: 23297 $
% martial french touch
%=============================================================
% Demo and test the common macros
%=================================================================
\documentclass[a4paper,10pt,twoside]{book}
%=============================================================
\input{common.tex} % doit contenir babel-french
\graphicspath{{figures/}} % for \dothis
%=============================================================
\begin{document}
\sloppy
\mainmatter
%=============================================================
\section*{Environnements de code basés sur le package listings}
%=============================================================
\paragraph{Exemple d'environnement de code}
Utilisez les environnements script, method, classdef et example.
Chacun a un nom et un label optionnel.
Avec les labels, nous pouvons avoir des références croisées de 
With labels we can have cross references to
\verb|\egref{history}| \egref{history},
\verb|\scrref{helloworld}| \scrref{helloworld},
\verb|\clsref{myclass}| \clsref{myclass},
et \verb|\mthref{doit}| \mthref{doit}.

\begin{verbatim}
\begin{example}[history]{La première chose que Smalltalk peut faire}{}
3 + 4 --> 7
\end{example}
\end{verbatim}
\begin{example}[history]{La première chose que Smalltalk peut faire}{}
3 + 4 --> 7
\end{example}

\begin{verbatim}
\begin{script}[helloworld]{La première chose que vous pouvez tester}
Transcript show: 'hello world'
\end{script}
\end{verbatim}
\begin{script}[helloworld]{La première chose que vous pouvez tester}
Transcript show: 'hello world'
\end{script}

\begin{verbatim}
\begin{classdef}[myclass]{définition de MyClass}
Object subclass: #MyClass
	instancevariables: '...'
	...
\end{classdef}
\end{verbatim}
\begin{classdef}[myclass]{définition de MyClass}
Object subclass: #MyClass
	instancevariables: '...'
	...
\end{classdef}

\begin{verbatim}
\begin{method}[doit]{Une méthode doit}
MyClass>>>doit
<tab>^ super doit
\end{method}
\end{verbatim}
\begin{method}[doit]{Une méthode doit}
MyClass>>>doit
	^ super doit
\end{method}

L'environnement de code ordinaire n'a ni titre, ni label, ni
numérotation:

\begin{verbatim}
\begin{code}{}
Du code tout court
\end{code}
\end{verbatim}
\begin{code}{}
Du code tout court
\end{code}

%=============================================================
\paragraph{Les environnements Listings et ses macros}
L'environnement de code

\begin{verbatim}
\begin{code}{}
    ...
\end{code}
\end{verbatim}
accepte du code ordinaire ou \emph{verbatim} et traduit certains
caractères spéciaux comme  $\wedge$ en \ct{^}. 
Même les tabulations sont maintenus (ce qui n'est pas vrai pour le \emph{verbatim}).
\begin{code}{}
"Tout les !caractères! correctes: ^ $ ' % \\ << >> _ { }"
"Si vous voulez !{\bf vraiment}! un point d'exclamation BANG, vous devez !écrire! explicitement !{\bf BANG}!"
"Si vous voulez vraiment un chapeau de renvoi ainsi: CARET au lieu de ^, vous devez !écrire! !{\bf CARET}!." 
| y |
true & false not & (nil isNil) ifFalse: [self halt].
y _ self size + super size.
#($a #a 'a' 1 1.0)
	do: [:each | Transcript 
			show: (each class name); 
                     show: ' ';
                     show: (each printString).
{ 1 + 2 . 3 \\ 4 . 1 << 3. 2 >> 5 . 1 % 2 }.
^ x < y 
\end{code}

La table QWERTY:
\begin{code}{}
BANG @ # $ % ^ & * ( ) UNDERSCORE +
1 2 3 4 5 6 7 8 9 0 - =
 Q W E R T Y U I O P { }
 q w e r t y u i o p [ ]
  A S D F G H J K L : "" |   		 (deux fois " pour annuler l'italique")
  a s d f g h j k l ; ' \
   Z X C V B N M < > ?
   z x c v b n m , . /
\end{code}
%$
Échappement LaTeX dans le code:
\begin{verbatim}
\begin{code}{}
code ordinaire et !\textbf{gras}!
\end{code}
\end{verbatim}

\begin{code}{}
code ordinaire et !\textbf{gras}!
\end{code}

% ^ $ \\ % # ' 

Dans le code \emph{inline} avec  \verb|\ct| comme, \parex 
\verb|\ct{1 + 2 --> 3}| imprimé ainsi:
\ct{1 + 2 --> 3}, le texte peut suivre immédiatement.
Les ``caractères-bloc'' autour de \verb|\ct| peuvent être n'importe quelle
paire de caractères; utile si vous voulez  \ct${ et }$ dans le code.

\subsection{Les caractères spéciaux avec  $\backslash$ct}
\ct=^ ~ # $ ' % \\ << >> _ {  } ! -- --> =\\
\verb|\ct=^ ~ # $ ' % \\ << >> _ {  } ! -- --> =|

\subsection{Conventions spéciales}

\verb$\ct{Class>>>method}$ s'imprime  \ct{Class>>>method}.\\
\verb$\ct{3 + 4 - 5 --> 2}$ s'imprime  \ct{3 + 4 - 5 --> 2}.

Utiliser \verb$@TEST$ pour inclure le code dans des tests automatiques
et utiliser  \verb$-->$ pour représenter leur résultat respectif.
\begin{verbatim}
\begin{code}{@TEST}
true       --> true
3@4     --> 3@4
$a         --> $a
#(1 2 3) --> #(1 2 3)
\end{code}
\end{verbatim}

%=============================================================
%\paragraph{New code environment}

%Experiment:

%\begin{testenv}{testing}
%self doit
%\end{testenv}

%=============================================================
\section*{Autres macros}
%=============================================================
\paragraph{url}
Ne pas oublier le \texttt{http://}:
\url{http://SqueakByExample.org}, \pharoweb ou
\ppe
%=============================================================
\paragraph{noms}
\SUnit
\xUnit
\st
\pharo
\squeak
\sqsrc
%=============================================================
\paragraph{annotations originales}
\fix \ugh{please rephrase this}
\ins{please insert this text}
\del{delete this}
ou encore
\chg{change this}{to this}
\ab{...}
\paragraph{annotations de l'édition française} % ajout vf
~\\
\tradalert{martial}{le reste n'est pas synchrone}
\begin{itemize}
\item \verb|\arevoir| \arevoir{Texte à revoir soit parce que c'est
    incertain, soit parce que c'est une formulation ``pas très compréhensible''};
\item \verb|\arelire| \arelire{Changement par rapport à la version de
    Squeak Par l'Exemple ou raffinement par rapport à la version
    originale Pharo By Example};
\item \verb|\arevoir| \aretirer{Totalement faux mais à garder car
    existe toujours dans la version originale}.
\end{itemize}
%=============================================================
\paragraph{abbréviations}
\ie
\cad
\eg
\parex
\etc
%=============================================================
\paragraph{Macros Smalltalk}
sep: \sep
%=============================================================
\paragraph{scat et prot}
Catégorie \scat{Kernel-Objects} et protocoles \prot{accessing}.
Les paquetages (ou \emph{packages}) s'écrivent comme les catégories
avec \verb|\scat|.
%=============================================================
\paragraph{menu}
Le menu \menu{World \go Tools}
%=============================================================
\paragraph{button}
Le bouton \button{Create}
%=============================================================
\paragraph{do this}
\verb|\dothis|
\ct{-->}
\dothis{Télécharger immédiatemment \pharo.}
%=============================================================
\paragraph{Raccourcis-clavier}
\verb|\short{d}|
\ct{-->}
\short{d}
% ajout vf
\paragraph{Les clics de souris}
Pour l'édition \pharo, les clics sont écrits:
\begin{itemize}
\item \verb|\click| pour \click{}
\item \verb|\clickant| pour \clickant{}
\item \verb|\clickz| pour \clickz{}
\end{itemize}

Le $c$ de \verb|\click| peut être majuscule. Les alternatives de
\verb|\click| sont \verb|\actclick| et \verb|\metaclick| pour
``\actclick'' et ``\metaclick'' respectivement. La règle de la
capitalisation (première lettre en majuscule) existe aussi.
%=============================================================
\section*{Citations de bas de page}
Il y a un grand livre sur \squeak par Ducasse\cite{Duca05j}.
%=============================================================
\section*{Les points importants}
\important{L'heure est grave}
%=============================================================
\section*{Remarques de l'édition française}
\paragraph{les termes et phrases toute faites}
\verb|\mantra| pour ``\mantra''

\begin{tabular}{|l|l|l|}
  \hline
  mot anglais & commande & imprimé \\
  \hline
  callback & \verb|\callback| & \callback\\
  changeset & \verb|\changeset| & \changeset\\
  sender & \verb|\sender| & \sender\\
  sender & \verb|\sender| & \sender\\
  senders & \verb|\senders| & \senders\\
  Senders & \verb|\Senders| & \Senders\\
  implementors & \verb|\implementors| & \implementors\\
  implementor & \verb|\implementor| & \implementor\\
  truetype & \verb|\truetype| & \truetype\\
  \verb|\bam|\footnote{Dans le texte uniquement.} & \verb|\bamfr| & \bamfr\\
  \hline
\end{tabular}

\paragraph{accents dans les environnements de code}

\subparagraph{cas des commentaires}
Il suffit de baliser chaque mot~\footnote{Nous ne pouvons pas baliser
  toute une phrase en raison d'une implémentation \emph{cra-cra} du
  package \texttt{listings}.} accentué avec des points d'exclamation
\texttt{!}.

\begin{verbatim}
\begin{code}{}
"Des !caractères! !accentués!"
\end{code}
\end{verbatim}
\begin{code}{}
"Des !caractères! !accentués!"
\end{code}

\subparagraph{autres éléments}

\verb|\normcode| et \verb|emcode| permettent d'écrire dans le code
entre \texttt{!} un
mot accentué sans qu'il soit en italique comme dans tous commentaires:
il sera ainsi respectivement de typographie normal et grasse.

\begin{code}{}
BladeRunner>>run
  self mayRun ifFalse: [self error: 'Reste !\normcode{caché}! alors BANG'].
  self runningFree
\end{code}

\paragraph{convention}

\begin{itemize}
\item Nous parlons à la première personne du pluriel et vous lisez à
  la seconde personne du pluriel\,---\,pas de ``on'';
\item Les notes de bas de pages commencent par une majuscule et
  finissent par un point;
\item Les légendes (\emph{caption}) des figures commencent normalement
  par un verbe à l'impératif et finissent par un point;
\item les articles de \verb|itemize| finissent par \verb|;| sauf le
  dernier qui se termine par un point.
\end{itemize}

\paragraph{méthodologie}

Utilisez les \verb|hyphenation| dans \texttt{common.tex} pour forcer
les césures. En cas de doutes sur les index (\verb|\index| et
\verb|\seeindex|) durant la traduction, laissez-les tels quels: il
sera plus facile pour le responsable de l'indexation de faire des
corrections d'une version orginale cohérente que d'une version
anglaise incohérente.

Les relecteurs devraient explorer les commentaires suivants dans le
code \LaTeX:
\begin{itemize}
\item \verb|\% ATTENDRE|  signifie ``en attente de
  corrections\/confirmations dans la version originale'';
\item \verb|\% CHANGE|  signifie ``changement par rapport à Squeak Par
  l'Exemple''. Souvent précédé d'une balise \verb|\arelire| dans les
  parties I, II, III et IV du document;
\item \verb|\% REVOIR|  signifie ``phrase lourde, terme francisé
  incorrect, terme à franciser, inexactitude par rapport à la version
  actuelle de \pharo''.
\end{itemize}

Ce document est une base technique; pour les
conventions sur le vocabulaire ou les mises à jours des outils
(\parex{} les paquetages MorphicExtras et consorts), référez-vous aux fichiers README.txt,
CHANGES.txt, mailing-lists et
\url{http://community.ofset.org/index.php/Squeak_par_l\%27exemple}.

%=============================================================
% \bibliographystyle{plain}
%\bibliographystyle{jurabib}
%\nobibliography{scg}

%
% This bibliography was produced by using jurabib.bst
%
\begin{thebibliography}{}

\bibitem[{Ducasse\jbdy {2005}}%
         {}%
         {{0}{}{book}{2005}{}{}{}{}%
          {APress\bibbdsep {} 2005}}%
         {{Squeak: Learn Programming with Robots}%
          {}{}{2}{}{}{}{}{}}%
        ]{Duca05j}
 \jbbibargs {\bibnf {Ducasse} {St\'ephane} {S.} {} {}} {St\'ephane Ducasse}
  {au} {\bibtfont {Squeak: Learn Programming with Robots}\bibatsep\  \apyformat
  {APress\bibbdsep {} 2005} \jbnote {1} {ISBN: 1-59059-491-6} } {\bibhowcited}
  \jbendnote {ISBN: 1-59059-491-6} \jbdoitem {{Ducasse}{St\'ephane}{S.}{}{}} {}
  {} \bibAnnote {book}

\end{thebibliography}

%=============================================================
\end{document}
%=============================================================
