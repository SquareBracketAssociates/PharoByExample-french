% $Author$ serge
% $Date$
% $Revision$
% relecture et synchro avec la version originale: martial boniou
% Fri Dec 14 14:05:59 CET 2007
% note: j'ai corrige l'url de telechargement/commande de SBE
% relecture: rene mages : Mon Dec 24 11:47:29 CET 2007
% relecture: rene mages : Sat Jan 12 18:47:29 CET 2008
% relecture: martial boniou: Mon Feb 11 11:58:24 CET 2008
% adaptation pour Pharo: martial - Thu Sep 10 23:36:52 CEST 2009 
% relecture: rene mages : Sat Jan 9 18:18:18 CET 2010
% relecture: rene mages : Thu Jun 24 18:18:18 CET 2010
% relecture: rene mages : Sun Apr 10 18:18:18 CET 2011
% $Author: mars $ $Date: 2009-09-09 12:48:45 +0200 (Wed, 09 Sep 2009)
% $ $Revision: 29014 $
% sync: 29170
%=================================================================
\ifx\wholebook\relax\else
% --------------------------------------------
% Lulu:
	\documentclass[a4paper,10pt,twoside]{book}
	\usepackage[
		papersize={6.13in,9.21in},
		hmargin={.75in,.75in},
		vmargin={.75in,1in},
		ignoreheadfoot
	]{geometry}
	\input{../common.tex}
  \pagestyle{headings}
	\setboolean{lulu}{true}
% --------------------------------------------
% A4:
%	\documentclass[a4paper,11pt,twoside]{book}
%	\input{../common.tex}
%	\usepackage{a4wide}
% --------------------------------------------
    \graphicspath{{figures/} {../figures/}}
	\begin{document}
    \sloppy
    \frontmatter
\fi
%=================================================================
\renewcommand{\nnbb}[2]{} % Disable editorial comments
\sloppy
%=================================================================
\chapter{Préface}\chalabel{intro}

%=================================================================
\section*{Qu'est ce que \pharo?}

\pharo est une implémentation moderne, libre et complète du
langage de programmation \st et de son environnement. \pharo est
un \emph{fork}\footnote{Un fork, ou embranchement, est un nouveau logiciel créé à partir du code source d'un logiciel existant. Cela suppose que les droits accordés par les auteurs le permettent : ils doivent autoriser l'utilisation, la modification et la redistribution du code source. C'est pour cette raison que les forks se produisent facilement dans le domaine des logiciels libres. (Extrait de Wikipedia)} de \squeak \cite{Inga97a}, 
une réécriture de l'environnement \st-80 original. 
Alors que \squeak fut développé principalement en tant que
plateforme pour le développement de logiciels éducatifs
expérimentaux, \pharo tend à offrir une plateforme,
à la fois, \emph{open-source} et épurée pour le développement de
logiciels professionnels et aussi, stable et robuste pour la recherche 
et le développement dans le domaine des langages et environnement dynamiques. 
\pharo est l'implémentation Smalltalk de référence de Seaside: le \framework 
destiné au développement web% if no seaside chapter => book ad
\ifseaside{.}{ et qui fait l'objet d'un livre disponible en version anglaise sur \dwdseaside.}

\pharo résout les problèmes de licence inhérent à \squeak. 
Contrairement aux versions précédentes de \squeak, le noyau
de \pharo ne contient que du code sous licence MIT. Le projet \pharo
a débuté en mars 2008 depuis un \emph{fork} de la version 3.9 de \squeak et la première version 1.0 \emph{beta} a été publiée le
31 juillet 2009. % CHANGE

Bien que dépourvu de nombreux paquetages présents dans \squeak, \pharo
est fourni avec beaucoup de fonctionalités optionnelles dans \squeak.
Par exemple, les fontes \truetype sont inclues dans \pharo. \pharo
dispose aussi du support pour de véritables fermetures lexicales ou
\emph{block closures}. Les élements d'interface utilisateurs ont été
revus et simplifiés. % CHANGE

\pharo est extrêmement portable --- même sa machine virtuelle est
entièrement écrite en \st, ce qui facilite son débogage, son
analyse et les modifications à venir. \pharo est le véhicule de tout
un ensemble de projets innovants, des applications multimédias et
éducatives aux environnements de développement pour le web.
% There is an important aspect behind \pharo: \pharo should not just be a copy of the past but really \emph{reinvent} Smalltalk. Big-bang approaches rarely succeed. \pharo will really favor evolutionary and incremental changes. We want to be able to experiment with important new features or libraries. Evolution means that \pharo accepts mistakes and is not not aiming for the next perfect solution in one big step\,---\,even if we would love it. \pharo will favor small incremental changes but a multitude of them. The success of \pharo depends on the contributions of its community.
Il est important de préciser le fait suivant concernant \pharo: 
\pharo ne devrait pas être qu'une simple copie du passé mais véritablement
une \emph{réinvention} de Smalltalk. Pour autant, les approches où l'on fait table rase du passé fonctionnent rarement.
Au contraire, \pharo encourage les changements évolutifs et
incrémentaux. Nous voulons qu'il soit possible d'expérimenter via de nouvelles fonctionalités et bibliothèques. Par \emph{évolution},
nous disons que \pharo tolère les erreurs et n'a pas pour objectif
de devenir la prochaine solution de rêve d'un bond\,---\,même si nous
le désirons.
\pharo favorisera toutes les évolutions à caractère incrémental. 
Le succès de \pharo dépend des contributions de sa communauté.

%=================================================================
\section*{Qui devrait lire ce livre?}

%This book is based on \emph{Squeak by Example}\footnote{\sbe}, an open-source introduction to \squeak.
%The book has been liberally adapted and revised to reflect the differences between \pharo and \squeak.
%This book presents the various aspects of \pharo, starting with the basics, and proceeding to more advanced topics.
Ce livre est dérivé du livre \emph{Squeak Par l'Exemple}\footnote{\spe;
  traduction française de Squeak By Example \mbox{(\sbe).}}, une introduction
à \squeak éditée sous \emph{licence libre}. Il a néanmoins été librement
adapté pour refléter les différences qui existent entre \pharo et \squeak. Ce livre 
présente différents aspects de \pharo, en commençant par les concepts 
de base et en poursuivant vers des sujets plus avancés.

Ce livre ne vous apprendra pas à programmer. Le lecteur doit avoir quelques 
notions concernant les langages de programmation. Quelques connaissances 
sur la programmation orientée objet seront utiles.

Ce livre introduit l'environnement de programmation, le langage et
les outils de \pharo. Vous serez confronté à de nombreuses bonnes
pratiques de Smalltalk, mais l'accent sera mis plus particulièrement
sur les aspects techniques et non sur la conception orientée
objet. Nous vous présenterons, autant que possible, un grand nombre d'exemples (nous avons été inspiré par l'excellent livre de Alec
Sharp sur Smalltalk\cite{Shar97a}).
\index{Sharp, Alex}

Il y a plusieurs autres livres sur \st disponibles gratuitement 
sur le web mais aucun d'entre eux ne se concentrent sur \pharo. 
Voyez par exemple: \url{http://stephane.ducasse.free.fr/FreeBooks.html}

\ifluluelse{}{\newpage} % layout hint
%=================================================================
\section*{Un petit conseil}

% http://www.surfscranton.com/architecture/KnightsPrinciples.htm

Ne soyez pas frustré par des éléments de \st que vous ne comprenez pas immédiatement.
Vous n'avez pas tout à connaître!
Alan Knight exprime ce principe comme suit\footnote{\url{http://www.surfscranton.com/architecture/KnightsPrinciples.htm}}:
\index{Knight, Alan}
\important{{\bf Ne vous en préoccupez pas!}%
%\important{{\bf Moquez-vous en!}%
\footnote{Dans sa version originale: ``Try not to care''.}
Les développeurs \st débutants ont souvent beaucoup de
difficultés car ils pensent qu'il est nécessaire de connaître
tous les détails d'une chose avant de l'utiliser. Cela signifie
qu'il leur faut un moment avant de maîtriser un 
simple: \ct{Transcript show: 'Hello World'}. 
Une des grandes avancées de la programmation par objets est de 
pouvoir répondre à la question ``Comment ceci marche?'' avec  ``Je ne m'en préoccupe pas''.}

%=================================================================
\section*{Un livre ouvert}

Ce livre est ouvert dans plusieurs sens:

\begin{itemize}

\item	Le contenu de ce livre est diffusé sous la licence Creative Commons Paternité - Partage des Conditions Initiales à l'Identique.
		En résumé, vous êtes autorisé à partager librement et à adapter ce livre, tant que vous respectez les conditions de la licence disponible à l'adresse suivante: 
		\url{http://creativecommons.org/licenses/by-sa/3.0/}.

\item	Ce livre décrit simplement les concepts de base de \pharo.
		Idéalement, nous voulons encourager de nouvelles personnes à contribuer à des chapitres sur des parties de \pharo qui ne sont pas encore décrites.
		Si vous voulez participer à ce travail, merci de nous contacter. Nous voulons voir ce livre se développer!
\end{itemize}

Plus de détails concernant ce livre sont disponibles sur le site web \mbox{\ppe.}
%\spe, hébergé par l'\emph{Institute of Computer Science and Applied Mathematics} de l'Université de Berne en Suisse.

%=================================================================
\section*{La communauté \pharo}

La communauté \pharo est amicale et active.
Voici une courte liste de ressources que vous pourrez trouver utiles:

\begin{itemize}
\item \url{http://www.pharo-project.org} est le site web principal de \pharo.
\item \url{http://www.squeaksource.com}: \squeaksource est l'équivalent de
  \sourceforge pour les projets \pharo. De nombreux paquetages optionnels se trouvent ici.
\item \url{https://groups.google.com/group/smalltalk-fr?hl=fr} est un groupe de discussions francophone généraliste sur \st{}. En raison de la présence de programmeurs \pharo{} (dont l'équipe de traducteurs), vous pouvez poster des messages relatifs à \pharo{} ou à l'édition française du livre ``Pharo par l'Exemple''.
\item \arelire{\url{http://www.pharocasts.com} est un \emph{blog} \emph{en anglais} dirigé par Laurent Laffont et diffusant des vidéos démonstratives de \pharo{}. Un bon moyen d'apprendre autrement! Une vidéo dédiée au jeu Lights Out est aussi disponible. Vous pourrez vous y référer lorsque vous aborderez \charef{firstApp}.}
\end{itemize}

% REVOIR : listes de diffusion parties peut être attendre la version
% originale définitive sinon mettre au moins squeak-fr

% IRC et 'Autres sites' partie

%\paragraph{La communauté francophone de Squeak dispose également de plusieurs sites web :}
% \begin{itemize}
% \item \url{community.ofset.org/wiki/Squeak} est un Wiki qui regroupe la plupart des ressources en français concernant \pharo et \st. On y trouve notamment les actualités de la communauté, des tutoriels sur la programmation avec Squeak, des contenus pédagogiques utilisant les EToys.
% \item \url{planet-fr.squeak.org} est un agrégateur de différents blogs francophones qui s'intéressent à \pharo.
% \end{itemize}

%=================================================================
\section*{Exemples et exercices}

Nous utilisons deux conventions typographiques dans ce livre.

Nous avons essayé de fournir autant d'exemples que possible.
Il y a notamment plusieurs exemples avec des fragments de code qui
peuvent être évalués. Nous utilisons le symbole \ct{-->} afin
d'indiquer le résultat qui peut être obtenu en sélectionnant
l'expression et en utilisant l'option \menu{print it} du menu contextuel:

\begin{code}{@TEST}
3 + 4 --> 7    "Si vous !sélectionner! 3+4 et 'print it', 7 s'affichera"
\end{code}

% mise a jour (12/2007)
Si vous voulez découvrir \pharo en vous amusant avec ces morceaux de
code, sachez que vous pouvez charger un fichier texte avec la
totalité des codes d'exemples via le site web du livre: \ppe. 

La deuxième convention que nous utilisons est l'icône
\dothisicon{} pour vous indiquer que vous avez quelque chose à faire: 

\dothis{Avancez et lisez le prochain chapitre!}

%=================================================================
%\section*{Typographic convention}

%\on{This is repeated in the First Application chapter.  I suggest we remove it from the Preface.}

%Programming in \st means defining classes and methods.
%Unlike most programming languages where programs sit in files, in \st classes and methods are objects too, and they are edited using a dedicated code browser.
%The browser will show you the code of a method in the context of the class it belongs to.

%Unfortunately this book is not (yet) interactive, so when we show you the code of a method, it is not always immediately clear for which class it is defined.
%For example, we cannot immediately tell which class the method \ct{cellsPerSide} belongs to:

%\begin{code}{}
%cellsPerSide
%   "The number of cells along each side of the game"
%   ^ 10
%\end{code}

%The \st convention to indicate that a method \ct{aMethod} belongs to a class \ct{aClass} is to write its name as \ct{aClass>>>aMethod}.
%So, if it is not immediately clear from the context which class a method belongs to, we will show it explicitly like this:

%\begin{code}{}
%SBEGame>>>cellsPerSide
%   "The number of cells along each side of the game"
%   ^ 10
%\end{code}

%Of course, when you actually type the code of the method into the browser, you don't have to type the class name or the \ct{>>>}; instead, you just make sure that the appropriate class is selected in the browser.

%=================================================================
\section*{Remerciements pour l'édition anglaise}

% We would like to thank various people who have contributed to this book.
% In particular, we thank

%martial: a reformuler - mettre 'les auteurs' a la place de nous
Nous voulons remercier Hilaire Fernandes et Serge Stinckwich qui nous
ont autorisé à traduire des parties de leurs articles sur \st et
Damien Cassou pour sa contribution au chapitre sur les flots de
données ou \emph{streams}.

%  update (12/2007) - a relire
Nous remercions particulièrement Alexandre Bergel, Orla Greevy,
Fabrizio Perin, Lukas Renggli, Jorge Ressia et Erwann Wernli pour leurs
corrections détaillées % REVOIR en final (encore un ajout)
% ajout - vf
de l'édition originale. % REVOIR

Nous remercions l'Université de Berne en Suisse pour le soutien
gracieusement offert à cette entreprise \emph{Open Source} et pour
les facilités d'hébergement web de ce livre.

Nous remercions aussi la communauté \squeak pour son soutien et son
enthousiasme sur ce projet et pour sa communication quant à l'aide à
la correction de la première édition de la version originale 
de ce livre.

\section*{Remerciements pour l'édition française}

L'édition française de ce livre a été réalisée par l'équipe de traducteurs: 
Martial Boniou, René Mages et Serge Stinckwich. Nous remercions également Karine Mordal-Manet pour sa relecture de certaines parties du livre et Mathieu Chappuis, Luc Fabresse, Nicolas Petton, Alain Plantec et Benoît Tuduri pour leur participation à la traduction de la version \squeak de cet ouvrage.

%% REVOIR
%Martial Boniou,  Mathieu Chappuis, Luc Fabresse, René Mages, Nicolas Petton, Alain Plantec, Serge Stinckwich et Benoît Tuduri.
%Cette équipe remercie l'association OFSET\footnote{OFSET est une organisation française à but non lucratif de type association loi 1901. Elle a été créée pour répondre à la faiblesse du développement de logiciels libres éducatifs pour le système GNU. Elle fait la promotion de toutes les sortes de développements et de localisations nécessaires aux systèmes éducatifs à travers le monde.} (\url{www.ofset.org}) qui héberge notamment le Wiki de la communauté francophone de Squeak, ainsi que le magazine Gnu/Linux Magazine France (\url{www.gnulinuxmag.com}) qui nous a autorisé en reprendre en partie certains articles sur Smalltalk parus dans ses colonnes.

% note sur de la bibliographie en francais (Smalltalk + Squeak (Briffault, Bots...))

%=============================================================
\ifx\wholebook\relax\else
   \bibliographystyle{jurabib}
   \nobibliography{scg}
   \end{document}
\fi
%=============================================================

