% $Author: oscar $
% traduction Martial
% $Date: 2009-07-06 10:21:47 +0200 (Mon, 06 Jul 2009) $
% $Revision: 27886 $

% HISTORY:
% 2009-07-06 - Oscar opened new template (no contents)

%=================================================================
\ifx\wholebook\relax\else
% --------------------------------------------
% Lulu:
	\documentclass[a4paper,10pt,twoside]{book}
	\usepackage[
		papersize={6.13in,9.21in},
		hmargin={.75in,.75in},
		vmargin={.75in,1in},
		ignoreheadfoot
	]{geometry}
	\input{../common.tex}
	\pagestyle{headings}
	\setboolean{lulu}{true}
% --------------------------------------------
% A4:
%	\documentclass[a4paper,11pt,twoside]{book}
%	\input{../common.tex}
%	\usepackage{a4wide}
% --------------------------------------------
    \graphicspath{{figures/} {../figures/}}
	\begin{document}
	% \renewcommand{\nnbb}[2]{} % Disable editorial comments
	\sloppy
	\frontmatter
\fi
%=================================================================
\chapter{\arelire{Préface à l'Édition Omnibus}}\chalabel{omnibus}
\newcommand{\web}{web\xspace}
\newcommand{\framework}{\emph{framework}\xspace}

Cette édition spéciale de \emph{Pharo par l'Exemple} comprend de
nombreux chapitres sur des sujets plus avancés
dont l'édition pour un second volume est prévue pour la fin de l'année
2009 ou le début de l'année 2010.
Cette édition a été spécialement préparée pour le 
\emph{Masters course} intitulé \emph{Dynamic Object-Oriented
  Programming with
  Smalltalk}\footnote{\url{http://scg.unibe.ch/teaching/smalltalk}}
donné à l'Université de Berne au second semestre 2009.
Les chapitres complémentaires couvrent des sujets tels que le système
de dépôt pour le versionage de projets Smalltalk partagés sur le \web
nommé Monticello,
l'environnement de développement \web avancé Seaside,
les expressions régulières (\emph{Regex})en Smalltalk,
le \framework Omnibrowser destiné aux développement de navigateurs
(\emph{browsers}),
la gestion des exceptions en Smalltalk ainsi que le support de
l'introspection et de \arevoir{la réflexivité} en Smalltalk.

%=================================================================
\section*{Remerciements}

Nous remercions Vassili Bykov pour sa permission d'adapter librement
la documentation \emph{Regex} dans le chapitre sur les expressions
régulières en Smalltalk.

Nous tenons à remercier aussi
Orla Greevy,
Fabrizio Perin,
Lukas Renggli,
Jorge Ressia,
et
David Roethlisberger
pour leurs corrections détaillées sur ce matériel additionnel
ainsi que Frederica Nierstrasz pour sa contribution 
artistique à la couverture.

%\on{Add ACKs for new chapters!}
%\on{pharo cover art: Samuel Morello}

%Thanks to the following reviewers:
%Orla Greevy,
%Lukas Renggli.

%Thanks to Vassili Bykov for permission to adapt his Regex documentation.

%=============================================================
\ifx\wholebook\relax\else
   \bibliographystyle{jurabib}
   \nobibliography{scg}
   \end{document}
\fi
%=============================================================
