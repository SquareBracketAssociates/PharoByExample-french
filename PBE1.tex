% $Author$
% $Date$
% $Revision$
% relecture: martial + rene Tue Dec 25 12:46:57 CET 2007
% relecture: martial + rene Sun Jan 13 12:46:57 CET 2008
% relecture; martial Mon Feb 11 11:39:32 CET 2008 (ajout: titre
% revision pour Pharo (SBE.tex devient PBE1.tex) Mon Sep  7 17:40:49
% CEST 2009
% $Author: oscar $
% $Date: 2009-06-30 15:08:59 +0200 (Tue, 30 Jun 2009) $
% $Revision: 27772 $

%=================================================================
% This is the main file for the Squeak By Example book.
% The individual chapters can also be latexed by themselves.
%=================================================================
\documentclass[a4paper,10pt,twoside]{book}
\usepackage[
	papersize={6.13in,9.21in},
	hmargin={.75in,.75in},
	vmargin={.75in,1in},
	ignoreheadfoot
]{geometry}
\input{common.tex}
\setboolean{lulu}{true}

%=================================================================
% A4
%\documentclass[a4paper,11pt,twoside]{book}
%\input{common.tex}
%\usepackage{a4wide}
%=================================================================
% Add the path for the figures of each chapter here:
\graphicspath{
	{figures/}
% BOOK 1
	{BasicClasses/figures/}
	{Collections/figures/}
	{Environment/figures/}
	{FirstApp/figures/}
	{Messages/figures/}
	{Metaclasses/figures/}
	{Metaprogramming/figures/}
	{Model/figures/}
	{Morphic/figures/}
	{Preface/figures/}
	{Profiling/figures/}
	{QuickTour/figures/}
	{SUnit/figures/}
	{Streams/figures/}
	{Syntax/figures/}
   {Reflection/figures/}
   {Seaside/figures/}
}
%=================================================================
\let\wholebook=\relax
\makeindex
\makeglossary
%=================================================================
\renewcommand{\nnbb}[2]{} % Disable editorial comments
%=================================================================
\begin{document}
\frontmatter
%=================================================================
\setcounter{page}{1}
\pagestyle{headings}
%=================================================================
%:Inside cover
\author{
	Andrew Black\quad
	St\'ephane Ducasse\\[1ex]
	Oscar Nierstrasz\quad
	Damien Pollet
	\\[4ex]
	avec l'aide de Damien Cassou et Marcus Denker
%Ajout des traducteurs
	\\[4ex]
Traduit en fran\c{c}ais par :\\[2ex]
	\arelire{A VENIR} % REVOIR
    %	Martial Boniou\quad
	%	Mathieu Chappuis\\[1ex]
	%	Luc Fabresse\quad
	%	Ren\'e Mages\\[1ex]
	%	Nicolas Petton\quad
	%	Alain Plantec\\[1ex]
	%	Serge Stinckwich\quad
    %    Beno\^it Tuduri\\[1ex]
	}
% (to be updated at the end)
\title{\Huge\bf Pharo par l'Exemple}
%\isodate
\numdate
\selectlanguage{french}
\date{\emph{Version du \today}}
\maketitle
%=================================================================
%:Copyright notice
~ % force the vfill
\vfill
\begin{footnotesize}
\setlength{\parindent}{0pt}
Ce livre est disponible en libre t\'el\'echargement depuis
\ppe sous le titre
\emph{Pharo par l'exemple}, \arelire{ISBN XXX-X-XXXXXXX-X-X. La première édition
a été publié en Décembre 2009. Couverture par Samuel Morello.} % REVOIR
                                % ou bien Nierstrasz selon la version
                                % final ( % ATTENDRE )
%par \emph{Square Bracket Associates}, Suisse
% (\sba).
Vous pouvez vous procurer une
copie à l'adresse: \ppe.
\\[1cm]

Copyright \copyright~2007, 2008, 2009 by Andrew P.~Black, St\'ephane
Ducasse, Oscar Nierstrasz and Damien Pollet.\\[1cm] % REVOIR

Le contenu de ce livre est prot\'eg\'e par la licence Creative Commons
Paternit\'e Version 3.0 de la licence g\'en\'erique - Partage des Conditions Initiales \`a l'Identique.

\emph{Vous \^etes libres :}
\begin{description}
  \item de reproduire, distribuer et communiquer cette cr\'eation au public
  \item de modifier cette cr\'eation
\end{description}
\emph{Selon les conditions suivantes :}
\begin{description}
  \item[Paternit\'e.] Vous devez citer le nom de l'auteur original de la mani\`ere indiqu\'ee par l'auteur de l'\oe{}uvre ou le titulaire des droits qui vous conf\`ere cette autorisation (mais pas d'une mani\`ere qui sugg\'ererait qu'ils vous soutiennent ou approuvent votre utilisation de l'\oe{}uvre).
  \item[Partage des Conditions Initiales \`a l'Identique.] Si vous transformez ou modifiez cette \oe{}uvre pour en cr\'eer une nouvelle, vous devez la distribuer selon les termes du m\^eme contrat ou avec une licence similaire ou compatible.
\end{description}
\begin{itemize}
  \item \`A chaque r\'eutilisation ou distribution de cette cr\'eation, vous devez faire appara\^itre clairement au public les conditions contractuelles de sa mise \`a disposition. La meilleure mani\`ere de les indiquer est un lien vers cette page web:\\
  \url{http://creativecommons.org/licenses/by-sa/3.0/deed.fr}
  \item Chacune de ces conditions peut \^etre lev\'ee si vous obtenez l'autorisation du titulaire des droits sur cette \oe{}uvre.
  \item Rien dans ce contrat ne diminue ou ne restreint le droit moral de l'auteur ou des auteurs.
\end{itemize}
\raisebox{-0.25cm}{\includegraphics[width=2cm]{CreativeCommons-BY-SA}}\quad
\parbox{\textwidth-2cm-1em}{
	Ce qui pr\'ec\`ede n'affecte en rien vos droits en tant qu'utilisateur (exceptions au droit d'auteur: copies r\'eserv\'ees \`a l'usage priv\'e du copiste, courtes citations, parodie, \ldots).
	Ceci est le R\'esum\'e Explicatif du Code Juridique (la version int\'egrale du contrat):\\
\url{http://creativecommons.org/licenses/by-sa/3.0/legalcode}}
\end{footnotesize}
\vfill
%=================================================================
%:TOC
\pagestyle{newheadings}
\tableofcontents
% \listoffigures
% \listoftables
% \lstlistoflistings
\sloppy % To avoid LaTeX's annoying habit of letting lines stick over the margins!
%=================================================================
%:Preface
% $Author$ serge
% $Date$
% $Revision$
% relecture et synchro avec la version originale: martial boniou
% Fri Dec 14 14:05:59 CET 2007
% note: j'ai corrige l'url de telechargement/commande de SBE
% relecture: rene mages : Mon Dec 24 11:47:29 CET 2007
% relecture: rene mages : Sat Jan 12 18:47:29 CET 2008
% relecture: martial boniou: Mon Feb 11 11:58:24 CET 2008
% adaptation pour Pharo: martial - Thu Sep 10 23:36:52 CEST 2009 from
% $Author: mars $ $Date: 2009-09-09 12:48:45 +0200 (Wed, 09 Sep 2009)
% $ $Revision: 29014 $
% sync: 29170
%=================================================================
\ifx\wholebook\relax\else
% --------------------------------------------
% Lulu:
	\documentclass[a4paper,10pt,twoside]{book}
	\usepackage[
		papersize={6.13in,9.21in},
		hmargin={.75in,.75in},
		vmargin={.75in,1in},
		ignoreheadfoot
	]{geometry}
	\input{../common.tex}
  \pagestyle{headings}
	\setboolean{lulu}{true}
% --------------------------------------------
% A4:
%	\documentclass[a4paper,11pt,twoside]{book}
%	\input{../common.tex}
%	\usepackage{a4wide}
% --------------------------------------------
    \graphicspath{{figures/} {../figures/}}
	\begin{document}
    \sloppy
    \frontmatter
\fi
%=================================================================
\renewcommand{\nnbb}[2]{} % Disable editorial comments
\sloppy
%=================================================================
\chapter{Pr\'eface}\chalabel{intro}

%=================================================================
\section*{Qu'est ce que \pharo?}

\arelire{\pharo est une impl\'ementation moderne, libre et compl\`ete du
langage de programmation \st et de son environnement. \pharo est
dérivé de \squeak\cite{Inga97a}, une ré-programmation du classique
système \st-80. Alors que \squeak fut développé principalement en tant que
 plateforme pour le développement de logiciels éducatifs
 expérimentaux, \pharo tend à offrir une plateforme,
 à la fois, \emph{open-source} et épurée pour le développement de
 logiciels professionnels
et aussi, 
 stable et robuste pour la recherche et le développement dans le
 domaine des langages et environnement dynamiques. \pharo est le
 système de référence de la bibliothèque de développement web
 Seaside.} % CHANGE

\arelire{\pharo résout les problèmes de licence inérrant à
\squeak. Contrairement aux versions précédentes de \squeak, le noyau
de \pharo ne contient que du code sous licence MIT. Le projet \pharo
a débuté en mars 2008 depuis un \emph{fork}\footnote{Embranchement à
  partir duquel le code d'un logiciel sert de base à un nouveau produit.}
de \squeak 3.9 et la première version 1.0 \emph{beta} a été publiée le
31 juillet 2009.} % CHANGE

\arelire{Bien que dépourvu de nombreux paquetages présents dans \squeak, \pharo
est fourni avec beaucoup de fonctionalités optionnelles dans \squeak.
Par exemple, les fontes \truetype sont inclus dans \pharo. \pharo
dispose aussi du support pour de véritables fermetures lexicales ou
\emph{block closures}. Les élements d'interface utilisateurs ont été
revus et simplifiés.} % CHANGE

\pharo est extr\^emement portable --- m\^eme sa machine virtuelle est
entièrement \'ecrite en \st, ce qui facilite son d\'ebogage, son
analyse et les modifications à venir. \pharo est le v\'ehicule de tout
un ensemble de projets innovants, des applications multim\'edias et
\'educatives aux environnements de d\'eveloppement pour le web.
% There is an important aspect behind \pharo: \pharo should not just be a copy of the past but really \emph{reinvent} Smalltalk. Big-bang approaches rarely succeed. \pharo will really favor evolutionary and incremental changes. We want to be able to experiment with important new features or libraries. Evolution means that \pharo accepts mistakes and is not not aiming for the next perfect solution in one big step\,---\,even if we would love it. \pharo will favor small incremental changes but a multitude of them. The success of \pharo depends on the contributions of its community.
Il est important de préciser le fait suivant concernant \pharo: \arelire{\pharo
ne devrait pas être qu'une simple copie du passé mais véritablement
une \emph{réinvention} de Smalltalk. \arevoir{Les approches en \emph{big bang}
fonctionnent rarement.} \pharo encourage les changements évolutifs et
incrémentaux. Nous voulons être capable d'expérimenter via les
nouvelles fonctionalités et autre bibliothèques. Par \emph{évolution},
nous disons que \pharo \arevoir{tolèrent les erreurs et n'a pas pour objectif
de devenir la prochaine solution de rêve d'un bond\,---\,même si nous
le désirons.}
\arevoir{\pharo favorisera de multiples évolutions.}} % REVOIR
Le succès de \pharo dépend des contributions de sa communauté.

%=================================================================
\section*{Qui devrait lire ce livre?}

%This book is based on \emph{Squeak by Example}\footnote{\sbe}, an open-source introduction to \squeak.
%The book has been liberally adapted and revised to reflect the differences between \pharo and \squeak.
%This book presents the various aspects of \pharo, starting with the basics, and proceeding to more advanced topics.
Ce livre est basé sur \emph{Squeak Par l'Exemple}\footnote{\spe;
  traduction française de Squeak By Example (\sbe).}, une introduction
à \squeak éditée en \emph{open-source}. \arelire{Il a néanmoins été librement
adapté pour refléter les différences entre \pharo et \squeak. Ce livre pr\'esente diff\'erents aspects de \pharo, en commen\c{c}ant par les concepts de base et en poursuivant vers des sujets plus avanc\'es.}

Ce livre ne vous apprendra pas \`a programmer. Le lecteur doit avoir quelques notions concernant les langages de programmation. Quelques connaissances sur la programmation objet seront utiles.

Ce livre introduit l'environnement de programmation, le langage et
les outils de \pharo. Vous serez confront\'e \`a de nombreuses bonnes
pratiques de Smalltalk, mais l'accent sera mis plus particuli\`erement
sur les aspects techniques et non sur la conception orient\'ee
objet. Nous vous pr\'esenterons, autant que possible, une foule 
d'exemples (nous avons \'et\'e inspir\'e par l'excellent livre de Alec
% rene : c'est bien Alec et non Alex (verification OK)
Sharp sur Smalltalk\cite{Shar97a}).
\index{Sharp, Alex}

Il y a plusieurs autres livres sur \st disponibles gratuitement sur le web mais aucun d'entre eux ne se concentrent sur \pharo. Voyez par exemple:
\url{http://stephane.ducasse.free.fr/FreeBooks.html}

\ifluluelse{}{\newpage} % layout hint
%=================================================================
\section*{Un petit conseil}

% http://www.surfscranton.com/architecture/KnightsPrinciples.htm

Ne soyez pas frustr\'e par des \'el\'ements de \st que vous ne comprenez pas imm\'ediatement.
Vous n'avez pas tout \`a conna\^itre!
Alan Knight exprime ce principe comme suit\footnote{\url{http://www.surfscranton.com/architecture/KnightsPrinciples.htm}}:
\index{Knight, Alan}
\important{{\bf Ne vous en pr\'eoccupez pas!}%
%\important{{\bf Moquez-vous en!}%
\footnote{Dans sa version originale: ``Try not to care''.}
Les d\'eveloppeurs \st d\'ebutants ont souvent beaucoup de
difficult\'es car ils pensent qu'il est n\'ecessaire de conna\^itre
tous les d\'etails d'une chose avant de l'utiliser. Cela signifie
qu'il leur faut un moment avant de ma\^{\i}triser un simple: \ct{Transcript show: 'Hello World'}. Une des grandes avanc\'ees de la programmation par objets est de pouvoir r\'epondre \`a la question ``Comment ceci marche?'' avec  ``Je ne m'en pr\'eoccupe pas''.}

%=================================================================
\section*{Un livre ouvert}

Ce livre est ouvert dans plusieurs sens:

\begin{itemize}

\item	Le contenu de ce livre est diffus\'e sous la licence Creative Commons Paternit\'e - Partage des Conditions Initiales \`a l'Identique.
		En r\'esum\'e, vous \^etes autoris\'e \`a partager librement et \`a adapter ce livre, tant que vous respectez les conditions de la licence disponible \`a l'adresse suivante: 
		\url{http://creativecommons.org/licenses/by-sa/3.0/}.

\item	Ce livre d\'ecrit simplement les concepts de base de \pharo.
		Id\'ealement, nous voulons encourager de nouvelles personnes \`a contribuer \`a des chapitres sur des parties de \pharo qui ne sont pas encore d\'ecrites.
		Si vous voulez participer \`a ce travail, merci de nous contacter. Nous voulons voir ce livre se d\'evelopper!
\end{itemize}

\arelire{Plus de d\'etails concernant ce livre sont disponibles sur le site
web \ppe.}
%\spe, h\'eberg\'e par l'\emph{Institute of Computer Science and Applied Mathematics} de l'Universit\'e de Berne en Suisse.

%=================================================================
\section*{La communaut\'e \pharo}

La communaut\'e \pharo est amicale et active.
Voici une courte liste de ressources que vous pourrez trouver utiles:

\begin{itemize}
\item \url{http://www.pharo-project.org} est le site web principale de \pharo.
\item \url{http://www.squeaksource.com}: \squeaksource est l'\'equivalent de
  \sourceforge pour les projets \pharo. De nombreux paquetages
  optionnels se trouvent ici.
\end{itemize}

% REVOIR : listes de diffusion parties peut être attendre la version
% originale définitive sinon mettre au moins squeak-fr

% IRC et 'Autres sites' partie

%\paragraph{La communaut\'e francophone de Squeak dispose \'egalement de plusieurs sites web :}
% \begin{itemize}
% \item \url{community.ofset.org/wiki/Squeak} est un Wiki qui regroupe la plupart des ressources en fran\c{c}ais concernant \pharo et \st. On y trouve notamment les actualit\'es de la communaut\'e, des tutoriels sur la programmation avec Squeak, des contenus p\'edagogiques utilisant les EToys.
% \item \url{planet-fr.squeak.org} est un agr\'egateur de diff\'erents blogs francophones qui s'int\'eressent \`a \pharo.
% \end{itemize}

%=================================================================
\section*{Exemples et exercices}

Nous utilisons deux conventions typographiques dans ce livre.

Nous avons essay\'e de fournir autant d'exemples que possible.
Il y a notamment plusieurs exemples avec des fragments de code qui
peuvent \^etre \'evalu\'es. Nous utilisons le symbole \ct{-->} afin
d'indiquer le r\'esultat qui peut \^etre obtenu en s\'electionnant
l'expression et en utilisant l'option \menu{print it} du menu contextuel:

\begin{code}{@TEST}
3 + 4 --> 7    "Si vous !s\'electionner! 3+4 et 'print it', 7 s'affichera"
\end{code}

% mise a jour (12/2007)
Si vous voulez d\'ecouvrir \pharo en vous amusant avec ces morceaux de
code, sachez que vous pouvez charger un fichier texte avec la
totalit\'e des codes d'exemple via le site web du livre: \ppe. 

La deuxi\`eme convention que nous utilisons est l'ic\^one
\dothisicon{} pour vous indiquer que vous avez quelque chose \`a faire: 

\dothis{Avancez et lisez le prochain chapitre!}

%=================================================================
%\section*{Typographic convention}

%\on{This is repeated in the First Application chapter.  I suggest we remove it from the Preface.}

%Programming in \st means defining classes and methods.
%Unlike most programming languages where programs sit in files, in \st classes and methods are objects too, and they are edited using a dedicated code browser.
%The browser will show you the code of a method in the context of the class it belongs to.

%Unfortunately this book is not (yet) interactive, so when we show you the code of a method, it is not always immediately clear for which class it is defined.
%For example, we cannot immediately tell which class the method \ct{cellsPerSide} belongs to:

%\begin{code}{}
%cellsPerSide
%   "The number of cells along each side of the game"
%   ^ 10
%\end{code}

%The \st convention to indicate that a method \ct{aMethod} belongs to a class \ct{aClass} is to write its name as \ct{aClass>>>aMethod}.
%So, if it is not immediately clear from the context which class a method belongs to, we will show it explicitly like this:

%\begin{code}{}
%SBEGame>>>cellsPerSide
%   "The number of cells along each side of the game"
%   ^ 10
%\end{code}

%Of course, when you actually type the code of the method into the browser, you don't have to type the class name or the \ct{>>>}; instead, you just make sure that the appropriate class is selected in the browser.

%=================================================================
\section*{Remerciements pour l'\'edition anglaise}

% We would like to thank various people who have contributed to this book.
% In particular, we thank

%martial: a reformuler - mettre 'les auteurs' a la place de nous
Nous voulons remercier Hilaire Fernandes et Serge Stinckwich qui nous
ont autoris\'e \`a traduire des parties de leurs articles sur \st et
Damien Cassou pour sa contribution au chapitre sur les flots de
donn\'ees ou \emph{streams}.

%  update (12/2007) - a relire
Nous remercions particuli\`erement Alexandre Bergel, Orla Greevy,
Fabrizio Perin, Lukas Renggli, Jorge Ressia \arelire{et Erwann Wernli} pour leurs
corrections détaillées % REVOIR en final (encore un ajout)
% ajout - vf
de l'édition originale. % REVOIR

Nous remercions l'Universit\'e de Berne en Suisse pour le soutien
gracieusement offert \`a cette entreprise \emph{Open Source} et pour
les facilit\'es d'h\'ebergement web de ce livre.

Nous remercions aussi la communauté \squeak pour leur soutien et leur
enthousiasme sur ce projet et pour leur communication quant à l'aide à
la correction de la première édition 
de la version originale %ajout
de ce livre. % REVOIR

\section*{Remerciements pour l'\'edition fran\c{c}aise}

L'\'edition fran\c{c}aise de ce livre a \'et\'e r\'ealis\'ee par
l'\'equipe de traducteurs et de relecteurs suivantes: \arelire{A VENIR.}
%% REVOIR
%Martial Boniou,  Mathieu Chappuis, Luc Fabresse, Ren\'e Mages, Nicolas Petton, Alain Plantec, Serge Stinckwich et Beno\^it Tuduri.
%Cette \'equipe remercie l'association OFSET\footnote{OFSET est une organisation fran\c{c}aise \`a but non lucratif de type association loi 1901. Elle a \'et\'e cr\'e\'ee pour r\'epondre \`a la faiblesse du d\'eveloppement de logiciels libres \'educatifs pour le syst\`eme GNU. Elle fait la promotion de toutes les sortes de d\'eveloppements et de localisations n\'ecessaires aux syst\`emes \'educatifs \`a travers le monde.} (\url{www.ofset.org}) qui h\'eberge notamment le Wiki de la communaut\'e francophone de Squeak, ainsi que le magazine Gnu/Linux Magazine France (\url{www.gnulinuxmag.com}) qui nous a autoris\'e en reprendre en partie certains articles sur Smalltalk parus dans ses colonnes.

% note sur de la bibliographie en francais (Smalltalk + Squeak (Briffault, Bots...))

%=============================================================
\ifx\wholebook\relax\else
   \bibliographystyle{jurabib}
   \nobibliography{scg}
   \end{document}
\fi
%=============================================================

\mainmatter
%=================================================================
%:PART 1 -- Getting Started
\part{Comment démarrer}
\pagestyle{headings}
%:Quick Tour
% $Author: oscar$
% $Translate: mathieu chappuis + martial boniou $
% $Date: 2007-12-13 15:59:51 +0100 (Thu, 13 Dec 2007) $
% $Revision$
% $french: Sun Dec 16 14:25:37 CET 2007$ 
%%%%%%%%%%%%%%%%%%%%%%
% note temporaire de Martial destine aux relectures:
% collapse a window --> ranger la fenetre (et pas reduire, si vous
% voyez cette erreur, SVP merci de corriger)
%%%%%%%%%%%%%%%%%%%%%%
% relecture: Rene Mages (fusion par martial: Wed Dec 26 17:28:17 CET 2007)
% relecture: Rene Mages (Sat Jan 12 17:28:17 CET 2007)
% note de martial: revoir les index pour les termes liés à Morphic
% adaptation pour PBE - 28729
% sync avec la version: 29170
%=================================================================
\ifx\wholebook\relax\else
% --------------------------------------------
% Lulu:
	\documentclass[a4paper,10pt,twoside]{book}
	\usepackage[
		papersize={6.13in,9.21in},%,
		%% Martial: j'ai enlevé les lignes pour tester la mise en page "manuelle" de la figure colouredMouse
		hmargin={.75in,.75in},
		vmargin={.75in,1in}
%		ignoreheadfoot
	]{geometry}
	\input{../common.tex}
	\pagestyle{headings}
	\setboolean{lulu}{true}
% --------------------------------------------
% A4:
%	\documentclass[a4paper,11pt,twoside]{book}
%	\input{../common.tex}
%	\usepackage{a4wide}
% --------------------------------------------
    \graphicspath{{figures/} {../figures/}}
	\begin{document}
	%\renewcommand{\nnbb}[2]{} % Disable editorial comments
	\sloppy
\fi
%=================================================================
\newcommand{\clover}{%
	\raisebox{-0.8ex}[0pt][0pt]{%
		\includegraphics[width=1em]{cloverleafKey}}}
%=================================================================

\chapter{Une visite de \pharo}
\chalabel{quick}

Nous vous proposons dans ce chapitre une premi\`ere visite de \pharo afin de vous familiariser avec son environnement.
De nombreux aspects seront abordés; il est conseillé d'avoir une
machine pr\^ete \`a l'emploi pour suivre ce chapitre. 

Cette icône \dothisicon{} dans le texte signalera les étapes où vous devrez essayer quelque chose vous-même.
Vous apprendrez à lancer \pharo et les différentes manières d'utiliser l'environnement et les outils de base.
La création des méthodes, des objets et les envois de messages seront également abordés.

%=================================================================
\section{Premiers pas}

\pharo est librement disponible au \ind{téléchargement} depuis le site
web: \pharoweb. % CHANGE
Vous devez y télécharger 3 archives (pour 4 fichiers principaux qui
constituent une installation courante de \pharo; voir \figref{download}) 

\begin{figure}[htb]
\centerline {\includegraphics[width=\textwidth]{annotatedDownload}}
\caption{Les fichiers à télécharger de \pharo. \figlabel{download}}
\end{figure}

\begin{enumerate}

\item La \emphind{machine virtuelle} (abr\'eg\'ee en VM pour
  \emph{virtual machine}) est la seule partie de l'environnement qui
  est particulière à chaque système d'exploitation. Des machines
  virtuelles pré-compilées sont disponibles pour la plupart des
  systèmes (Linux, \macosx, Win32). Dans \figref{download}, vous
  avez par exemple l'ic\^one de la machine virtuelle pour le syst\`eme
  \macosx: \textit{Squeak 4.0.1beta1U.app}~\footnote{\pharo est dérivé
  de \squeak{} 3.9 et il partage actuellement la machine virtuelle
  avec \squeak.}.
%martial: ajout des index dans la vf
\index{machine virtuelle}
\seeindex{VM}{machine virtuelle}

  \item Le fichier \emphind{source} contient le code source du système
    \pharo. Ce fichier ne change pas tr\`es fréquement. Dans \figref{download}, il
    correspond au fichier \emph{SqueakV39.sources}.
%ajout fr index
\index{fichier!source}
\seeindex{fichier-source}{fichier, source}
\seeindex{SqueakV39.sources}{fichier, source}

\item Le \emph{système} \emphind{image} est un cliché d'un système
  \pharo{} en fonctionnement, figé à un instant donné. 
Il est composé de deux fichiers: le premier nommé avec l'extension
\emph{.}\emphind{image} contient l'état de tous les objets du système
dont les classes et les méthodes (qui sont aussi des
objets). Le second avec l'extension \emph{.}\emphind{changes} contient
le journal de toutes les modifications apportées au code source du
système (contenu dans le fichier source).
\arelire{Dans \figref{download}, nous voyons que nous utilisons les images et
fichiers \emph{changes} de \textit{PBE}.
\arevoir{Nous utiliserons en fait une image légérement différente dans
ce livre.}} \on{fix the figure to use Damien's PBE image} % CHANGE
%ajout fr index
\index{fichier!image}
\index{fichier!changes}

\end{enumerate}

\dothis{Téléchargez et installez \pharo sur votre ordinateur.}
\arelire{Nous vous recommandons d'utiliser l'image fournie sur la page web du
livre\footnote{\ppe}.}
\index{téléchargement}
\seclabel{sbeImage} %????


Sachez que si vous avez déjà une autre version de \pharo qui fonctionne sur votre
machine, la plupart des exemples d'introduction de ce livre
fonctionneront. Il n'est donc pas nécessaire de mettre à jour \pharo.
D\`es lors, ne soyez pas surpris de constater parfois des différences dans l'apparence ou le comportement que nous décrirons.
% D'un autre c\^ot\'e, si vous téléchargez \pharo pour la première fois,
% vous devriez rapatrier et utiliser l'image \emph{\pharo par l'Exemple}.

Pendant que vous travaillez avec \pharo les fichiers \emph{.image} et \emph{.changes} sont modifiés, vous devez vous assurer qu'ils sont accessibles en écriture.
Conservez toujours ces deux fichiers ensemble, \cad dans le même dossier.
Et surtout, ne tentez pas de les modifier avec un éditeur de texte, \pharo les utilise pour stocker vos objets de travail et vos changements dans le code source.
Faire une copie de sauvegarde de vos images téléchargées et de vos
fichiers \emph{changes} est une bonne id\'ee; vous pourrez ainsi
toujours démarrer à partir d'une image propre et y recharger votre code.

Les fichiers \emphind{source}{}\emph{s} et l'exécutable de la VM peuvent être
en lecture seule\,---\,il est donc possible de les partager entre plusieurs utilisateurs.
Ces quatre fichiers peuvent résider dans le même dossier, mais vous pouvez également placer la machine virtuelle et les fichiers sources dans un dossier partagé distinct.
Vous pouvez adapter l'installation de \pharo à vos habitudes de travail
et \`a votre système d'exploitation.

%-----------------------------------------------------------------
\begin{figure}[htb]
%\centerline {\includegraphics[width=0.6\textwidth]{download}}
\centerline {\includegraphics[width=0.95\textwidth]{startup}}
\caption{Une image \arevoir{\ppe} fra\^{\i}chement d\'emarr\'ee.\figlabel{startup}}
\end{figure}

\index{lancement de \pharo}

\paragraph{Lancement.} Pour lancer \pharo, selon votre système: glissez
le fichier \emph{.}\emphind{image} sur l'icône de l'exécutable de la
machine virtuelle, ou double-cliquez sur le fichier
\emph{.}\emphind{image}, ou encore, depuis une ligne de commande,
tapez le nom du fichier binaire correspondant \`a la machine virtuelle
suivi du chemin d'accès au fichier \emph{.}\emphind{image} (si vous
avez installé plusieurs machines virtuelles, le système ne choisira
pas forcément celle qui convient, il sera préférable de
glisser-déposer l'image sur la VM ou d'utiliser la ligne de commande).

Une fois lancé, \pharo vous présente une large fenêtre qui contient des
espaces de travail nommés \emphind{Workspace} (voir \figref{startup}).
\arelire{Vous pourriez remarquer un barre de menus mais \pharo{}
  emploie principalement des menus contextuels.}


\dothis{Lancez \pharo. Vous pouvez fermer les fen\^etres d\'ej\`a
  ouvertes en \clickant sur \arevoir{l'icône {\sf X}} situ\'e sur le coin
  supérieur gauche des fenêtres ou les ranger (ce qui normalement les
  r\'eduit \`a  leur barre de titre) en \clickant sur le symbole
  \raisebox{-0.2ex}{{\Large $\circ$}} au coin supérieur droit.} % REVOIR dans la VO PBE x devrait être le bouton rouge si le style par défaut est OS X


%-----------------------------------------------------------------
\paragraph{Première interaction.}

Les options du menu World (``Monde'' en anglais) pr\'esent\'ees dans
\figref{threeButtons:click} sont un bon point de d\'epart.

\dothis{Cliquez à l'aide de la souris dans l'arrière plan de la
  fenêtre principale pour afficher le menu World, puis sélectionnez
  \menu{Workspace} pour créer un nouvel espace de
  travail ou Workspace.}

\begin{figure}[tbh]
	\centering
	\subfigure[Le menu World]{\figlabel{threeButtons:click}% click
		\includegraphics[width=0.40\linewidth]{worldMenu}}\hfill
	\subfigure[le menu contextuel]{\figlabel{threeButtons:actclick}% action click
		\includegraphics[width=0.55\linewidth]{yellowButtonMenuOnWorkspace}}\hfill
	\subfigure[Le halo Morphic]{\figlabel{threeButtons:metaclick}% meta click
		\includegraphics[width=0.60\linewidth]{morphicHaloOnWorkspace}}% these braces needed (else no whitespace at end of line)
	\caption{\arelire{Le menu World (affiché en \clickant{} avec la
        souris), un menu contextuel (affiché en \actclickant{}) et un
      \subind{Morphic}{halo} Morphic (affiché en \metaclickant).}\figlabel{threeButtons}}
\end{figure} % CHANGE
%\seeindex{morphic halo}{Morphic}
\seeindex{halo}{Morphic} % REVOIR

% ON: I had to shrink this and move it up to avoid
% it running over the end of the page.
% position d'origine de la figure colouredMouse

\st a été conçu à l'origine pour être utilisé avec une souris à trois
boutons. Si votre souris en a moins, vous pourrez utiliser des touches
du clavier en complément de la souris pour simuler les boutons
manquants. Une souris à deux boutons fonctionne bien avec \pharo, mais si
la v\^otre n'a qu'un seul bouton vous devriez songer à adopter un
modèle récent avec une molette qui fera office de troisième bouton: votre travail avec \pharo n'en sera que plus agréable.

\pharo évite les termes ``clic gauche'' ou ``clic droit'' car leurs
effets peuvent varier selon les systèmes, le mat\'eriel ou les
réglages utilisateur.
\arelire{Originellement, \st{} introduit des couleurs pour définir les
différents boutons de souris~\footnote{Les couleurs de boutons sont
  \emph{rouge}, \emph{jaune} et \emph{bleu}. Les auteurs de ce livre
  n'ont jamais pu se souvenir à quelle couleur se réfère chaque
  bouton.}.}
\index{bouton!rouge}
\index{bouton!jaune}
\index{bouton!bleu}
\arelire{Puisque de nombreux utilisateurs utiliseront diverses touches de
  modifications (\emph{Ctrl}, \emph{Alt}, \emph{Meta} \etc) pour
  réaliser les mêmes actions, nous utiliserons plutôt les termes
  suivants:}
\begin{description}
\item[\clickbtn:] \arelire{il s'agit du bouton de la souris le plus fréquemment
  utilisé et correspond au fait de \click{} avec une souris à un seul
  bouton sans aucun touche de modifications; \click{} sur
  l'arrière-plan de l'image fait apparaître le menu ``World'' (voir
  \figref{threeButtons:click}); 
nous utiliserons le terme \emph{\click} pour définir cette action;} % vf-only
\item[\actclickbtn:] \arelire{c'est le second bouton le plus utilisé; il est
  utilisé pour afficher un menu contextuel \ie{} un menu qui fournit
  différentes actions dépendant de où se trouve la souris comme le
  montre \figref{threeButtons:actclick}. Si vous n'avez pas de souris
  à multiples boutons, vous configurerez normalement la touche de
  modifications \emph{Ctrl} pour effectuer cette même action avec
  votre unique bouton de souris;
nous utiliserons l'expression ``\emph{\actclick}\footnote{En anglais,
  le terme utilisé est ``to actclick''.}''.} % vf-only: 2 dernières
                                % lignes
\item[\metaclickbtn:]  \arelire{vous pouvez finalement \emph{\metaclick{}} sur
  un objet affiché dans l'image pour activer le
  ``\subind{Morphic}{halo} Morphic'' qui est une consellation d'icônes
  autour de l'objet actif à l'écran; chaque icône repr\'esentant une poign\'ee de contr\^ole
permettant des actions telles que \emph{changer la taille} ou
\emph{faire pivoter l'objet}, comme vous pouvez le voir sur
\figref{threeButtons:metaclick}~\footnote{Notez que les icônes
    Morphic sont habituellement actives dans \pharo, mais vous pouvez
    les désactiver via le Preferences Browser que nous verrons plus
    loin.}.
En survolant lentement une icône avec le pointeur de votre souris,
une bulle d'aide en affichera un descriptif de sa fonction.
Dans \pharo, \metaclick dépend de votre système d'exploitation:
Soit vous devez maintenir {\sc Shift} \emph{Ctrl} ou {\sc Shift}
\emph{Alt} tout en \clickant.}
% \ab{This makes it sound like either {\sc shift} \emph{ctrl} or {\sc shift} \emph{alt} will work.  On my (Mac OS) system, only the latter works.  Perhaps we want to say: In \pharo, how you meta-click depends on your operating system. On Linux \ldots}
% Typically you will use a third modifier key, such as \emph{command} or \emph{meta} to \metaclick.
\end{description}


% %martial: il faut regler a la main la position et la hauteur de
% %l'encart avec la souris 
% \begin{wrapfigure}[15]{r}{0.25\linewidth}
% % The parameters are the number of narrow lines to the right of the figure [19],
% % the placement {r} for right, and the width of the figure. Capital R will allow some float.
% % Inside the wrapfig environment, linewidth is special --- the width of the figure.
% \includegraphics[width=0.95\linewidth]{colouredMouse}
% \caption{La souris de l'auteur. Le clic avec la molette correspond au bouton bleu.\figlabel{colouredMouse}}
% \end{wrapfigure}
% \newpage

\dothis{Saisissez \ct{Time now}
%ajout 
(expression retournant l'heure actuelle) dans le Workspace.
Puis \actclickz{} dans le Workspace et sélectionnez
\menu{print it} 
%ajout
(en fran\c{c}ais, ``imprimez-le'')
dans le menu qui apparaît.}

% \dothis{\Metaclickz{} sur le Workspace.
% Déplacez la poignée 
% %martial: j'ai decide de mettre tout les Morphic handles dans le
% %repertoire 'figures' de la racine 
% \rotateHandle{}
% %\raisebox{-0.4ex}{\includegraphics[width=1em]{morphicRotate}}
% située à proximité du coin inférieur gauche pour faire pivoter le Workspace.}

\arelire{Nous recommandons aux droitiers de configurer leur souris pour
\click{} avec le bouton gauche
% ajout vf
(qui devient donc le bouton de \clickbtn),
\actclick{} avec le bouton droit et \metaclick{} avec la
\arevoir{molette de défilement cliquable}, si elle est disponible.}
% Avec une souris sans molette il est possible d'invoquer le menu halo
% en maintenant \ct{alt}, \ct{ctrl}
% ou \ct{option} pendant que vous cliquez sur le \ind{bouton rouge}.
% \ab{This doesn't work any more.  This sentence either repeats or
% contradicts the meta-click item above; neither is a good idea.}
Si vous utilisez un Macintosh avec une souris à un bouton, vous pouvez
simuler le second bouton en maintenant la touche \clover{} enfoncée 
en \clickant. Cependant, si vous prévoyez d'utiliser \pharo souvent, nous
vous recommandons d'investir dans un modèle à deux boutons au minimum.

%  j'ai ajouté CTRL car sur mon linux ni alt ni fn.. ne marchent pour
%  ça. seul ctrl le fait..
% note de martial: ca depend aussi du windowmanager; c'est une bonne
% idee de le mettre en tout cas 
Vous pouvez configurer votre souris selon vos souhaits en utilisant
les préférences de votre système ou le pilote de votre dispostif de
pointage.
\ab{How can I get meta-click without a three-finger salute?  Is this a secret?}
\pharo vous propose des réglages pour adapter votre souris et les
touches spéciales de votre clavier. 
Dans l'outil de r\'eglage des pr\'ef\'erences nomm\'e \ind{Preference
  Browser} (\menu{System \go Preferences {\ldots\go}
  Preference Browser\ldots} dans le menu \menu{World}), la catégorie
\menu{keyboard} contient une option \emph{swapControlAndAltKeys}
permettant de \arelire{permuter les fonctions ``\actclick{}'' et
  ``\metaclick''}.
Cette catégorie
propose aussi des options afin de dupliquer les touches de modifications.%  et 
% rendre une pression sur \ct{alt} \'equivalente à une pression sur \ct{ctrl}.

\begin{figure}[htb]
	{\centerline {\includegraphics[width=\textwidth]{PreferenceBrowser}}}
\caption{Le Preference Browser.\figlabel{prefBrowser}}
\end{figure}

%=================================================================
\section{Le menu World}
\index{menu World}

\dothis{\Clickz dans l'arrière plan de \pharo.}
Le menu \menu{World} apparaît à nouveau.
La plupart des menus de \pharo ne sont pas modaux; ils ne bloquent pas
le système dans l'attente d'une réponse.
Avec \pharo vous pouvez maintenir ces menus sur l'écran en \clickant{} sur
l'icône en forme d'épingle au coin supérieur droit. Essayez!%  Vous
% remarquerez que les menus apparaissent  quand  vous cliquez  mais ne
% disparaissent pas quand vous relâchez votre bouton, ils restent
% visibles jusqu'à que vous ayez fait une sélection ou que vous ayez
% cliqué en dehors du menu. Tous les menus affichés à l'écran peuvent se déplacer en glissant leur barre de titre, comme n'importe quelle fenêtre.

Le menu World vous offre un moyen simple d'accéder à la plupart des
outils disponibles dans \pharo.

\dothis{\'Etudiez attentivement \arelire{le menu \menu{World} et, en
    particulier, son sous-menu \menu{Tools}} (voir \figref{threeButtons:click}).}

Vous y trouverez une liste des principaux outils de \pharo.
Nous aurons affaire à eux dans les prochains chapitres.

%=================================================================
\section{\arelire{Envoyer des messages}} % CHANGE spécial Pharo

\dothis{Ouvrez un espace de travail Workspace et saisissez-y
    le texte suivant:}

\begin{code}{}
BouncingAtomsMorph new openInWorld
\end{code}

\dothis{Maintenant \actclickz. Un menu devrait
  apparaître. Sélectionnez l'option \menu{do it (d)} (en français,
  ``faîtes-le!'') comme le montre la \figref{doit}.}

\begin{figure}[htb]
\centerline {\includegraphics[width=0.8\textwidth]{Doit}}
\caption{Évaluer une expression avec \menu{do it}.\figlabel{doit}}
\end{figure}

Une fenêtre contenant un grand nombre d'\bamfr
(en anglais, ``\emph{bouncing atoms}'') s'ouvre dans le coin supérieur
gauche de votre image \pharo.
%A window containing a large number of bouncing atoms should open in the top left of the \pharo image.

Vous venez tout simplement d'évaluer votre première expression \st. 
%You have just evaluated your first \st expression!
Vous avez juste envoyé le message 
 \ct{new} à la classe \bam ce qui résulte de la création d'une
 nouvelle instance qui à son tour \arevoir{reçoit le message}
 \ct{openInWorld}. % REVOIR dans le texte orig. pas de notion de
                   % receveur mais on parle uniquement d'envoyer (dur
                   % à reformuler) - martial
% You just sent the message \ct{new} to the \bam class, resulting in a new \bam instance, followed by the message \ct{openInWorld} to this instance.
La classe \bam{} a décidé de ce qu'il fallait faire avec le message
\ct{new}: elle recherche dans ces \emph{méthodes} pour répondre de
façon appropriée au message \ct{new}
% vf
\arevoir{(\ie{} ``nouveau'' en français; ce que nous traduirons par
  \emph{nouvelle instance})}.
% The \bam class decided what to do with the \ct{new} message, that is, it looked up its \emph{methods} for handling \ct{new} message and reacted appropriately.
De même, l'instance \bam recherchera dans ces méthodes comment
répondre à \ct{openInWorld}.
% Similarly the \bam instance looked up its method for responding to \ct{openInWorld} and took appropriate action.

Si vous discutez avec des habitués de \st, vous constaterez rapidement
qu'ils n'emploient généralement pas les expressions comme ``faire appel
à une opération'' ou ``invoquer une méthode'': ils diront ``envoyer
un message''.
% If you talk to Smalltalkers for a while, you will quickly notice that they generally do not use expressions like ``call an operation'' or ``invoke a method'', but instead they will say ``send a message''.
Ceci reflète l'idée que les objets sont responsables de leurs propres
actions.
% This reflects the idea that objects are responsible for their own actions. 
Vous ne \emph{direz} jamais à un objet quoi faire\,---\,vous lui
\emph{demanderez} polimment de faire quelque chose en lui envoyant un
message.
% You never \emph{tell} an object what to do\,---\,instead you politely \emph{ask} it to do something by sending it a message. 
C'est l'objet, et non pas vous, qui choisit la méthode appropriée pour
répondre à votre message.
% The object, not you, selects the appropriate method for responding to your message.

%=================================================================
\section{Enregistrer, quitter et redémarrer une session \pharo.}

\dothis{\arelire{\Clickz{} sur la fenêtre de démo des \bamfr{} et
    déplacez-la où vous voulez. Vous avons maintenant la démo ``dans
    la main''. Posez-la en \clickant.}}

\begin{figure}[htb]
\begin{minipage}[b]{0.48\textwidth}
	{\centerline{\includegraphics[width=0.7\textwidth]{atoms}}}
	\caption{Une instance de \bam.\figlabel{atoms}}
\end{minipage}
\hfill
\begin{minipage}[b]{0.48\textwidth}
	{\centerline{\includegraphics[width=0.7\textwidth]{saveAs}}}
	\caption{La bo\^{\i}te de dialogue \menu{save as\ldots}.\figlabel{saveas}}
\end{minipage}
\end{figure}

\dothis{Sélectionnez \menu{World\go{}Save as\ldots} et entrez le nom
  ``myPharo'', puis \clickz sur le bouton \button{OK}.
%ajout
pour sauvegarder sous un nouveau nom d'image.
Pour quitter, sélectionnez \menu{World\go{}Save and quit}.} % CHANGE

Le dossier qui contenait les fichiers image et \emph{changes} lorsque
vous avez lancé cette session de travail avec \pharo contient d\'esormais
deux nouveaux fichiers: ``myPharo.\ind{image}'' et ``myPharo.\ind{changes}''. 
%image vivante = working state of the pharo image
Ils repr\'esentent l'image ``vivante'' de votre session \pharo au moment qui précédait votre enregistrement avec \menu{Save and quit}.
Ces deux fichiers peuvent être copiés à votre convenance dans les
dossiers  de votre disque pour y être utilisés plus tard. \`A vous de
les invoquer en prenant soin (selon votre syst\`eme de fichiers) de
%note de martial: j'ai transforme pour plus de logique mais c'est
%lourd: a revoir
d\'eplacer, copier ou lier le fichier \emph{.source} correspondant,
tout en veillant \`a ex\'ecuter la bonne machine virtuelle.

\dothis{Lancez \pharo avec l'image que vous venez de créer \cad le
  fichier ``myPharo.image''.}

Vous retrouvez l'état de votre session exactement tel qu'il était
avant que vous quittiez \pharo. La démo des \bamfr{} est toujours sur votre
fenêtre de travail, en train de se déplacer d'o\`u vous l'aviez
abandonné.

En lançant pour la première fois \pharo, la \ind{machine virtuelle}
charge le fichier image que vous spécifiez. Ce fichier contient
l'instantané d'un grand nombre d'objets et surtout le code
pré-existant accompagné des outils de développement qui sont
d'ailleurs des objets comme les autres. En travaillant dans \pharo, vous
allez envoyer des messages à ces objets, en créer de nouveaux, et
certains seront supprimés et l'espace-mémoire utilisé sera récupéré
(\ie pass\'e au ramasse-miettes ou \emph{garbage collector}).

En quittant \pharo vous sauvegardez un instantané de tous vos objets. En sauvegardant (par ``Save''), vous remplacerez l'image courante par l'instantané de votre session. Pour préserver l'image courante, vous devez enregister sous un nouveau nom comme nous venons de le faire.

Chaque fichier \emph{.image} est accompagné d'un fichier \emph{.changes}.
Ce fichier contient un journal de toutes les modifications que vous avez faites en utilisant l'environnement de développement.
Vous n'avez pas à vous soucier de ce fichier la plupart du temps.
Mais comme nous allons le voir plus tard, le fichier \emph{.changes} pourra être utilisé pour rétablir votre système \pharo à la suite d'erreurs.

L'image sur laquelle vous travaillez provient d'une image de \st-80 créée à la fin des années 1970.
Beaucoup des objets qu'elle contient sont là depuis des décennies!

Vous pourriez penser que l'utilisation d'une image est incontournable pour stocker et gérer des projets, mais comme nous le verrons bientôt il existe des outils plus adaptés pour gérer le code et travailler en équipe sur des projets.
Les images sont très utiles mais nous consid\'erons comme une pratique un peu dépassée et fragile pour diffuser et partager vos projets alors qu'il existe des outils tels que Monticello qui proposent de biens meilleurs moyens de suivre les évolutions du code et de le partager entre plusieurs développeurs.

\dothis{\arelire{\Metaclickz{} sur la fenêtre
  d'\bamfr\footnote{Souvenez-vous que vous pourriez avoir besoin
    d'activer l'option \ct{halosEnabled} dans le Preference Browser.}.}}

%martial: le choix des noms 'poignee' 'icone' ... pourra etre change
Vous verrez tout autour une collection d'icônes colorées nomm\'ee
\subind{Morphic}{halo} de \bam; l'ic\^one
% \emphsubind{halo}{ic\^one}
\subind{Morphic}{halo} est aussi appel\'ee \emphsubind{halo}{poignée}.
Cliquez sur la poignée rose p\^ale qui contient une croix; la fenêtre
de démo disparaît. % CHANGE
%(index Morphic vs halo) a revoir?
\seeindex{poignée}{halo, poignée}
\seeindex{Morphic!poignée}{halo, poignée}
\seeindex{icône}{halo, icône}
\seeindex{halo!icône}{halo, poignée}
\seeindex{Morphic!icône}{halo, icône}
\seeindex{halo}{Morphic, halo}

%=================================================================
\section{Les fenêtres Workspace et Transcript}
\seclabel{transcript}

% martial - REVOIR - j'ai choisi de merger les deux dothis
\dothis{Fermez toutes fenêtres actuellement ouvertes. 
Ouvrez un \ind{Transcript} (via le menu \menu{World \go{} Tools}) et un \ind{Workspace}.
Positionnez et redimensionnez le Transcript et le Workspace
  pour que ce dernier recouvre le Transcript.} % CHANGE
Vous pouvez redimensionner les fenêtres en glissant l'un de leurs
coins ou en \metaclickant qui affiche les poignées
\emph{halo}: utilisez alors l'icône jaune située en bas à droite.

Une seule fenêtre est active à la fois; elle s'affiche au premier plan
et \arelire{son contour est alors mis en relief}. % CHANGE

Le Transcript est un objet qui est couramment utilisé pour afficher
des messages du système. C'est un genre de ``console''.
% Sachez que l'affichage dans la fenêtre Transcript est extr\^ement
% lent, donc si vous la conservez ouverte et que vous y affichez des
% résultats, certaines opérations peuvent \^etre 10 fois plus lentes.
% De plus, le Transcript n'est pas conçu pour recevoir
% simultanément des messages à afficher provenant de plusieurs objets:
% il n'est pas prot\'eg\'e contre les acc\`es concourrants (en anglais,
% \emph{thread-safe}), donc vous pourriez \^etre t\'emoin de
% comportements \'etranges si plusieurs objets tentent d'\'ecrire de
% mani\`ere concourrante dans le Transcript. 
% ON: I think the transcript has been made thread-safe now, right?

%%%% martial: Sat Dec 15 14:13:47 CET 2007
Les fen\^etres de Workspace ou espace de travail sont destin\'ees \`a
y saisir vos expressions de code \st \`a exp\'erimenter.
Vous pouvez aussi les utiliser simplement pour taper une quelconque
note de texte \`a retenir, comme une liste de choses \`a faire (en
anglais, \emph{todo-list}) ou des instructions pour quiconque est
amen\'e \`a utiliser votre image.
Les Workspaces sont souvent employ\'es pour maintenir une
documentation \`a propos de l'image courante, comme c'est le cas
dans l'image standard pr\'ec\'edemment charg\'ee (voir
\figref{startup}).

% originellement 'hello world'
\dothis{Saisissez le texte suivant dans l'espace de travail Workspace:}
\begin{code}{}
Transcript show: 'hello world'; cr.
\end{code}

%ajout
Exp\'erimentez la s\'election
en double-\clickant{} dans l'espace de travail \`a diff\'erents points dans
le texte que vous venez de saisir.
% entire word, entire string, or the whole text ((diff: string and word?))
Remarquez comment un mot entier ou tout un texte est
s\'electionn\'e %selon l'endroit o\`u vous cliquez.
\arelire{selon que vous \clickz{} sur un mot, à la fin d'une chaîne de
caractères ou à la fin d'une expression entière.}

\dothis{S\'electionnez le texte que vous avez saisi puis \actclickz{}.
Choisissez \menu{do it (d)} 
%ajout
(dans le sens ``faites-le!'', \cad \emph{\'evaluer le code
  s\'electionn\'e})
dans le menu contextuel.}
Notez que le texte ``hello world''~\footnote{\arelire{NdT: C'est une tradition de la
  programmation: tout premier programme dans un nouveau langage de
  programmation consiste à afficher la phrase en anglais ``hello world''
  signifiant ``bonjour le monde''.}}
 appara\^{\i}t dans la
fen\^etre Transcript (voir \figref{helloworld}).
Refaites encore un \menu{do it (d)}
(Le \menu{(d)} dans l'option de menu \menu{do it (d)} vous indique que
le raccourci-clavier correspondant est \short{d}. Pour plus
d'informations, rendez-vous dans la prochaine section!).

\begin{figure}[htb]
\ifluluelse
	{\centerline {\includegraphics[width=\textwidth]{HelloWorld}}}
\caption{\arelire{Les fenêtres sont superposées. Le Workspace est actif.}\figlabel{helloworld}}
\end{figure}

% Vous venez d'\'evaluer votre premi\`ere
% expression \st!
% Vous avez seulement envoyé le message \ct{show: hello world} \`a
% l'objet \ct{Transcript} (\ct{show:} veut dire: afficher), suivi du
% message \ct{cr} 
% %ajout
% (qui a le sens de \emph{carriage return}, \cad retour-chariot
% permettant de forcer le passage \`a la ligne suivante).
% Le \ct{Transcript} d\'ecide ensuite de quoi faire avec ce message; il
% cherche parmi ses \emph{m\'ethodes} celles qui g\`erent une r\'eponse
% aux messages \ct{show:} et \ct{cr} et qui r\'eagissent de fa\c{c}on
% appropri\'ee.

%=================================================================
\section{Les raccourcis-clavier}

Si vous voulez \'evaluer une expression, vous n'avez pas besoin de
toujours passer par le menu accessible en \actclickant: les
raccourcis-clavier sont l\`a pour vous. Ils sont mentionn\'es
dans les expressions parenth\'es\'ees des options des menus. Selon
votre plateforme, vous pouvez \^etre amen\'e \`a presser l'une des
touches de modifications soit \texttt{Control}, \texttt{Alt},
\texttt{Command} ou \texttt{Meta} (nous les indiquerons de mani\`ere
g\'en\'erique par \short{\emph{touche}}).
\index{raccourci-clavier}
\seeindex{clavier!raccourci-clavier}{raccourci-clavier}
\seeindex{clavier!événement}{événement, clavier} % martial: pour
                                % Morphic surtout

\dothis{R\'e\'evaluez l'expression dans le Workspace en utilisant
  cette fois-ci le raccourci-clavier: \short{d}.}
\index{raccourci-clavier!do it}

En plus de \menu{do it}, vous aurez not\'e la pr\'esence de
\menu{print it} 
%ajout
(pour \'evaluer et afficher le r\'esultat dans le m\^eme espace de travail), 
de \menu{inspect it} (pour inspecter) et de \menu{explore it} (pour
explorer). 
Jetons un coup d'\oe il \`a ceux-ci.

\dothis{Entrez l'expression \ct{3 + 4} dans le Workspace. Maintenant
  \'evaluez en faisant un \menu{do it} avec le raccourci-clavier.}

Ne soyez pas surpris que rien ne se passe!
Ce que vous venez de faire, c'est d'envoyer le message \ct{+} avec
l'argument \ct{4} au nombre \ct{3}. Le r\'esultat \ct{7} aura
normalement \'et\'e calcul\'e et retourn\'e, mais puisque votre espace de
travail Workspace ne savait que faire de ce r\'esultat, la r\'eponse a
simplement \'et\'e jet\'ee dans le vide. Si vous voulez voir le
r\'esultat, vous devriez faire \menu{print it} au lieu
de \menu{do it}. En fait, \menu{print it} compile l'expression,
l'ex\'ecute et envoie le message \ct{printString} au r\'esultat puis
affiche la cha\^{\i}ne de caract\`ere r\'esultante.

\dothis{S\'electionnez \ct{3+4} et faites \menu{print it} (\short{p}).}
Cette fois, nous pouvons lire le r\'esultat que nous attendions (voir
\figref{printit}).
\index{raccourci-clavier!print it}

\begin{figure}[htb]
% \centerline {\includegraphics[width=0.4\textwidth]{PrintIt}}
\centerline {\includegraphics[width=0.8\textwidth]{PrintIt}}
\caption{Afficher le r\'esultat sous forme de cha\^{\i}ne de
  caract\`eres avec \menu{print it} plut\^ot que de simplement
  \'evaluer avec \menu{do it}.\figlabel{printit}}
\end{figure}

\needlines{3}
\begin{code}{@TEST}
3 + 4 --> 7
\end{code}
\noindent
Nous utilisons la notation \ct{-->} comme convention dans tout le
livre pour indiquer qu'une expression particuli\`ere donne un certain
r\'esultat quand vous l'\'evaluez avec \menu{print it}.

\dothis{Effacez le texte surlign\'e ``\ct{7}''; comme \pharo devrait l'avoir
  s\'electionn\'e pour vous, vous n'avez qu'\`a presser sur la touche
  de suppression (suivant votre type de clavier \texttt{Suppr.} ou
  \texttt{Del.}). S\'electionnez \ct{3+4} \`a nouveau et, cette fois,
  faites une inspection avec \menu{inspect it} (\short{i}).}
%ajout
\index{raccourci-clavier!inspect it}
\index{inspecteur}
\seeindex{Inspector}{inspecteur}

\noindent
Vous devriez maintenant voir une nouvelle fen\^etre appel\'ee
\emphind{inspecteur} avec pour titre 
 \ct{SmallInteger: 7} (voir \figref{inspectit}).
L'inspecteur ou (sous son nom de classe) Inspector est un outil
extr\^emement utile: il vous permet de naviguer et d'interagir avec
n'importe quel objet du syst\`eme.
Le titre nous dit que \ct{7} est une instance de la classe
\clsind{SmallInteger} 
%ajout
(classe des entiers sur 31 bits).
Le panneau de gauche nous offre une vue des variables d'instance de
l'objet en cours d'inspection. Nous pouvons naviguer entre ces
variables et le panneau de droite nous affiche leur valeur.
Le panneau inf\'erieur peut \^etre utilis\'e pour \'ecrire des
expressions envoyant des messages \`a l'objet.

\begin{figure}[htb]
\centerline {\includegraphics[width=\textwidth]{InspectIt}}
\caption{Inspecter un objet.\figlabel{inspectit}}
\end{figure}

\dothis{Saisissez \ct{self squared} dans le panneau inf\'erieur de
  l'inspecteur que vous aviez ouvert sur l'entier \ct{7} et faites un
  \menu{print it}.
%ajout
Le message \ct{squared} (carr\'e) va \'elever le nombre \ct{7} lui-m\^eme (\ct{self}).}

\needlines{2}
\dothis{Fermez l'inspecteur. Saisissez dans un Workspace le
  mot-expression \ct{Object} et explorez-le via \menu{explore it}
  (\short{I}, i majuscule).}
\index{raccourci-clavier!explore it}
\index{explorateur}
\seeindex{Explorer}{explorateur}

Vous devriez voir maintenant une fen\^etre intitul\'ee \clsind{Object}
contenant le texte \mbox{$\triangleright$ \ct{root: Object}}.
Cliquez sur le triangle pour l'ouvrir (voir \figref{exploreit}).

\begin{figure}[htb]
\centerline {\includegraphics[width=0.7\textwidth]{ExploreIt}}
\caption{Explorer \ct{Object}.\figlabel{exploreit}}
\end{figure}

Cet explorateur (ou Explorer) est similaire \`a l'inspecteur mais il
offre une vue arborescente d'un objet complexe.
Dans notre cas, l'objet que nous observons est la classe \ct{Object}.
Nous pouvons voir directement toutes les informations stock\'ees dans
cette classe et naviguer facilement dans toutes ses parties.

%=================================================================
% \section{\sqmap}
% \index{SqueakMap}

% %web-based catalogue
% \sqmap est un catalogue web des ``packages'' ou
% \ind{paquetage}{}s\,---\,applications et biblioth\`eques de programmes (dites
% aussi librairies)\,---\,que vous pouvez t\'el\'echarger dans votre
% image.
% Les paquetages sont h\'eberg\'es sur de nombreux serveurs de
% par le monde et sont maintenus par un grand nombre de personnes. Certains de ces paquetages ne fonctionnent que sur une version spécifique de \pharo.
% \lr{Maybe mention Package Universes (SqueakMap is not maintained anymore)}

% \dothis{Ouvrez \menu{World \go open\ldots \go \sqmap Package Loader}.}
% Vous aurez besoin d'une connection Internet pour que cela
% fonctionne. Au bout d'un certain temps, la fen\^etre du gestionnaire
% de chargement \sqmap devrait appara\^{\i}tre (voir \figref{sokoban}).
% Sur le c\^ot\'e gauche, vous pouvez voir une longue liste de
% paquetages. Le champ de saisie situ\'e dans le coin sup\'erieur gauche
% est un panneau de recherche pour vous aider \`a trouver ce que vous
% cherchez dans la liste.

% Saisissez ``\ind{Sokoban}'' dans ce champ de recherche et
%   tapez sur la touche \textsc{Entr\'ee}.
% Cliquer sur le triangle dirig\'e vers le nom du paquetage vous
% r\'ev\`ele une liste des versions disponibles. Quand un paquetage ou
% une version est s\'electionn\'e, des informations \`a leur sujet sont
% affich\'ees dans le panneau de droite.
% Naviguez dans la derni\`ere version du jeu \ct{Sokoban}.
% Activez le menu contextuel du panneau de liste en cliquant dans cet
% espace avec le \ind{bouton jaune} et choisissez \menu{install} pour
% installer le paquetage s\'electionn\'e
% (si \pharo se plaint qu'il n'est pas s\^ur que cette version du jeu
% fonctionne dans votre image, r\'epondez aux questions par ``yes'' 
% %ajout
% pour confirmer l'installation).
% Remarquez qu'une fois que le paquetage a \'et\'e install\'e, il est
% marqu\'e d'une ast\'erisque dans la liste du \sqmap Package Loader.

% \begin{figure}[htb]
% \ifluluelse
% 	{\centerline {\includegraphics[width=\textwidth]{SqueakMap}}}
% 	{\centerline {\includegraphics[width=0.8\textwidth]{SqueakMap}}}
% \caption{Utiliser \sqmap pour installer le jeu Sokoban.\figlabel{sokoban}}
% \end{figure}

% \dothis{Apr\`es avoir installé ce paquetage, d\'emarrez \ct{Sokoban}
%   en \'evaluant \ct{SokobanMorph random openInWorld} dans un Workspace
% %ajout
% (souvenez-vous de faire \menu{do it} sur toute la s\'election).}

% % You can also try the \ct{NsGame}; execute it using \ct{NsGame new openInWorld}.
% % ON: I could not find NsGame anywhere!

% Le panneau inf\'erieur gauche du \sqmap Package Loader fournit
% plusieurs possibilit\'es pour filtrer la liste des paquetages. Vous
% pouvez choisir de ne voir que les paquetages qui sont compatibles avec
% une version particuli\`ere de \pharo
% %ajout
% (\emph{Squeak versions}), 
% ou qui sont de la famille des jeux
% %ajout
% (\emph{Entertainment\go{}Games}), 
% \etc.

%=================================================================
\section{Le navigateur de classes Class Browser}

Le navigateur de classes nomm\'e
\ind{Class Browser}~\footnote{\arevoir{Ce navigateur est confusément référé sous les noms ``System Browser''
  ou ``Code Browser''. \pharo{} utilise l'implémentation
  \ind{OmniBrowser} du navigateur connue aussi comme ``OB'' ou
  ``Package Browser''. Dans ce livre, nous utiliserons simplement le
  terme de Browser ou, s'il y a ambiguïté, nous parlerons de
  navigateur de classes}.} est un des
outils-cl\'e pour programmer. % CHANGE
Comme nous le verrons bient\^ot, il y a plusieurs navigateurs ou
\emph{browsers} int\'eressants disponibles pour \pharo, mais c'est le
plus simple que vous pourrez trouver dans n'importe quelle image, que
nous allons utiliser ici. % REVOIR (toujours vrai?)
\seeindex{navigateur de classes}{Browser}
\seeindex{Class Browser}{Browser}

\dothis{Ouvrez un navigateur de classes en s\'electionnant \menu{World
    \go{} Class Browser}~\footnote{\arelire{Si votre Browser ne
      ressemble pas à celui visible sur \figref{classBrowser}, vous
      pourriez avoir besoin de changer le navigateur par défaut. Voyez
    \faqref{packagebrowser}}}.} 

\begin{figure}[htb]
\ifluluelse
	{\centerline {\includegraphics[width=\textwidth]{ClassBrowser1}}}
	{\centerline {\includegraphics[width=0.7\textwidth]{ClassBrowser1}}}
\caption{Le navigateur de classes (ou Browser) affichant la
  m\'ethode \ct{printString} de la classe Object.
\figlabel{classBrowser}}
\end{figure}

Nous pouvons voir un navigateur de classes sur \figref{classBrowser}.
La barre de titre indique que nous sommes en train de parcourir la
classe \clsind{Object}.

\`A l'ouverture du Browser, tous les panneaux sont vides except\'e
le premier \`a gauche.
Ce premier panneau liste tous les \emph{paquetages} 
% vf
(en anglais, \emph{packages})
connus; \arelire{elles contiennent des groupes de classes apparentées}.
%\index{catégorie}

\dothis{\clickz{} sur le paquetage \scatind{Kernel}.}
Cette manipulation permet l'affichage dans le second panneau de toutes les
classes du paquetage s\'electionn\'e.

\dothis{S\'electionnez la classe \clsind{Object}.}
D\'esormais les deux panneaux restants se remplissent.
Le troisi\`eme panneau affiche les \emph{protocoles} de la classe
s\'electionn\'ee.
Ce sont des regroupements commodes pour relier des m\'ethodes
connexes. Si aucun \ind{protocole} n'est s\'electionn\'e, vous devriez
voir toutes les m\'ethodes disponibles de la classe dans le
quatri\`eme panneau.

\dothis{S\'electionnez le protocole \protind{printing}, 
%ajout
protocole de l'affichage.}
Vous pourriez avoir besoin de faire d\'efiler (avec la barre de
d\'efilement) la liste des protocoles pour le trouver.
Vous ne voyez maintenant que les m\'ethodes relatives \`a
l'affichage.

\dothis{S\'electionnez la m\'ethode \mthind{Object}{printString}.}
D\`es lors, vous voyez dans la partie inf\'erieure du Browser
le code source de la m\'ethode \ct{printString} partag\'e par tous
les objets 
%ajout
(tous dérivés de la classe Object,
exception faite de ceux qui la surcharge).

%=================================================================
\section{Trouver les classes}

Il existe plusieurs moyens de trouver une classe dans \pharo.
Tout d'abord, comme nous l'avons vu plus haut, nous pouvons savoir (ou
deviner) dans quelle cat\'egorie elle se trouve et, de l\`a, naviguer
jusqu'\`a elle via le navigateur de classes.
\index{Browser}
\seeindex{Browser!trouver une classe}{classe, recherche}
\index{classe!recherche}
\seeindex{classe!trouver}{classe, recherche}

Une seconde technique consiste \`a envoyer le message \ct{browse}
(ce mot a le sens de ``naviguer'') \`a la classe, ce qui a pour effet
d'ouvrir un navigateur de classes sur celle-ci
%ajout
(si elle existe bien s\^ur).
Supposons que nous voulions naviguer dans la classe \clsind{Boolean}
(la classe des bool\'eens).

\dothis{Saisissez \ct{Boolean browse} dans un Workspace et faites un \menu{do it}.}
Un navigateur s'ouvrira sur la classe \ct{Boolean} (voir \figref{browseBoolean}).
Il existe aussi un \ind{raccourci-clavier} \short{b} (browse) que vous
pouvez utiliser dans n'importe quel outil o\`u vous trouvez un nom de
classe;
\index{raccourci-clavier!browse it}
s\'electionnez le nom de la classe 
%ajout
(\parex \ct{Boolean})
puis tapez \short{b}.

\dothis{Utilisez le raccourci-clavier pour naviguer dans la classe \ct{Boolean}.}

\begin{figure}[hbt]
	{\centerline {\includegraphics[width=\textwidth]{Kernel-objects-boolean}}}
\caption{Le navigateur de classes affichant la d\'efinition de la
  classe \ct{Boolean}.\figlabel{browseBoolean}}
\end{figure}

Remarquez \arelire{que nous voyons une \emph{définition de classe}
  quand la classe \ct{Boolean} est sélectionnée mais sans qu'aucun
  protocole ni aucune méthode ne le soit}
% martial: on utilise le PLURIEL lorsque l’action ou l’état peut être
% rapporté aux deux sujets ; et le SINGULIER, lorsqu’il ne peut être
% rapporté qu’à un seul à la fois.
(voir \figref{browseBoolean}).
Ce n'est rien de plus qu'un message \st ordinaire qui est envoy\'e \`a
la classe parente lui r\'eclamant de cr\'eer une sous-classe.
Ici nous voyons qu'il est demand\'e \`a la classe \ct{Object} de
cr\'eer une sous-classe nomm\'ee \ct{Boolean} sans aucune variables
d'instance, ni variables de classe ou ``pool dictionaries'' et de 
mettre la classe \ct{Boolean} dans la cat\'egorie \scatind{Kernel-Objects}.
% The lower pane shows the \emph{class comment} --- a piece of plain text describing the class.
\arelire{Si vous \clickz{} sur le bouton \button{?} en bas du panneau
  de classes, vous verrez le \subind{classe}{commentaire} de classe
  dans un panneau dédié comme le montre \figref{classComment}.} % CHANGE

% Le nouveau panneau en dessous nous montre le \emph{commentaire de
%   classe}\,---\,quelques paragraphes de texte d\'ecrivant la classe.
% Si vous cliquez sur le bouton \button{?} \`a la base du panneau des
% classes 
% %ajout
% (\cad le second),
% vous pouvez voir le \subind{classe}{commentaire} de classe dans un
% panneau d\'edi\'e.

% \ab{I thought that this was supposed to be a \emph{Quick} tour!  And here we are describing a tool that I have used maybe twice in 10 years!   In any case, this description should be deferred to the \textbf{Environment} chapter}
% \on{I don't see why.  I use the hierarchy browser a lot!  I think it is really useful to know from the beginning, to help you find your through the hierarchy.}

% Si vous souhaitez explorer la hi\'erarchie des h\'eritages de \pharo, le
% navigateur nomm\'e \emphind{Hierarchy Browser} vous y aidera.
% Ça peut \^etre utile si vous \^etes en train de chercher une
% sous-classe ou une super-classe inconnue d'une classe connue.
% Le Hierarchy Browser ou navigateur hi\'erarchique est similaire au Browser except\'e que la liste des classes est arrang\'ee comme
% une arborescence indent\'ee refl\'etant l'h\'eritage.

% \dothis{Cliquez sur le bouton \button{hierarchy} dans le navigateur de
%   classes lorsque la classe \ct{Boolean} est s\'electionn\'ee.}
% \noindent
% Il est r\'esulte l'ouverture d'un Hierarchy Browser affichant les
% super-classes et les sous-classes de \clsind{Boolean}.
% % (\figref{booleanhierarchybrowser}).
% Naviguez un peu dans la super-classe et les sous-classes imm\'ediates
% de \ct{Boolean}.

\begin{figure}[hbt]
\centerline {\includegraphics[width=\textwidth]{classComment}}
\caption{Le commentaire de classe de \ct{Boolean}.
\figlabel{classComment}}
\end{figure}

Souvent, la m\'ethode la plus rapide de trouver une classe consiste
\`a la rechercher par son nom. Par exemple, supposons que vous \^etes
\`a la recherche d'une classe inconnue qui repr\'esente les jours et
les heures.% dates and times.

\dothis{Placez la souris dans le panneau des paquetages
  du Browser et tapez \short{f} ou s\'electionnez \menu{find
    class\ldots (f)} dans le menu contextuel accessible en
  \actclickant.
Saisissez ``time'' 
%ajout
(\cad le temps, puisque c'est l'objet de notre qu\^ete) 
dans la bo\^{\i}te de dialogue et acceptez cette entr\'ee.} 
\noindent
Une liste de classes dont le nom contient ``time'' vous sera
pr\'esent\'ee (voir \figref{findit}). Choisissez-en une, disons,
\ct{Time}; 
%martial: ca fait longtemps qu'il n'y a plus ce comportement
un navigateur l'affichera avec un commentaire de classe
sugg\'erant d'autres classes pouvant \^etre utiles. Si vous voulez
naviguer dans l'une des autres classes, s\'electionnez son nom (dans
n'importe quelle zone de texte) et tapez \short{b}.
\index{raccourci-clavier!find\ldots}
\index{raccourci-clavier!browse it}

\begin{figure}[hbt]
\centerline{
	\includegraphics[width=0.45\textwidth]{FindIt}
	\hspace{1cm}
	\includegraphics[width=0.45\textwidth]{TimeClasses}
}
\caption{Rechercher une classe d'apr\`es son nom.\figlabel{findit}}
\end{figure}

Notez que si vous tapez le nom complet (et correctement capitalis\'e 
%ajout
\cad en respectant la casse)
de la classe dans la bo\^{\i}te de dialogue de recherche (find), le
navigateur ira directement \`a cette classe sans montrer aucune liste
de classes \`a choisir.

%=================================================================
\section{Trouver les m\'ethodes}
\seclabel{quick:methodFinder}

Vous pouvez parfois deviner le nom de la m\'ethode ou, tout au moins,
une partie de son nom plus facilement que le nom d'une classe.
Par exemple, si vous \^etes int\'eress\'e par la connaissance du temps
actuel, vous pouvez vous attendre \`a ce qu'il y ait 
%martial: phrase differente pour le sens en francais
une m\'ethode affichant le temps \emph{maintenant}: comme la langue de \st
est l'anglais et que \emph{maintenant} se dit ``now'', une m\'ethode
contenant le mot ``now'' a de forte chance d'exister.
Mais o\`u pourrait-elle \^etre?
L'outil \emphind{Method Finder} peut vous aider \`a la trouver.
\seeindex{Browser!trouver une méthode}{méthode, recherche}
\index{méthode!recherche}
\seeindex{méthode!trouver}{méthode, recherche}

\dothis{\arelire{Sélectionnez \menu{World \go{} Tools \go{} Method Finder}.}
Saisissez ``now'' dans le panneau sup\'erieur gauche et cliquez sur
\menu{accept} (ou tapez simplement la touche \textsc{Entr\'ee}).}
Le chercheur de m\'ethodes Method Finder affichera une liste de tous
les noms de m\'ethodes contenant la sous-cha\^{\i}ne de caract\`eres ``now''.  

Pour d\'efiler jusqu'\`a \ct{now} lui-m\^eme, tapez ``\ct{n}''; cette
astuce fonctionne sur toutes les zones \`a d\'efilement de n'importe
quelle fen\^etre. En s\'electionnant ``now'', le panneau de droite
vous pr\'esentera les classes qui d\'efinissent une m\'ethode
avec ce nom, comme le montre \figref{MethodFinder}.
S\'electionner une de ces classes vous ouvrira un navigateur sur
celle-ci.

\begin{figure}[hbt]
\centerline {\includegraphics[width=0.7\textwidth]{methodFinder-now}}
\caption{Le Method Finder affichant toutes les classes qui
  d\'efinissent une m\'ethode appel\'ee \ct{now}.
\figlabel{MethodFinder}}
\end{figure}

\`A d'autres moments, vous pourriez avoir en t\^ete qu'une m\'ethode
existe bien sans savoir comment elle s'appelle.
Le Method Finder peut encore vous aider! Par exemple, partons de la
situation suivante: vous voulez trouvez une m\'ethode qui transforme
une cha\^{\i}ne de caract\`eres en sa version majuscule, \cad qui
transforme \ct{'eureka'} en \ct{'EUREKA'}.

\dothis{Saisissez \ct{'eureka' . 'EUREKA'} dans le Method Finder,
  comme le montre \figref{methodFinder-example1}.}
\noindent
Le Method Finder vous sugg\`ere une m\'ethode qui fait ce
que vous voulez~\footnote{\arelire{Si une fenêtre s'ouvre soudain avec un
  message d'alerte à propos d'une méthode obsolète\,---\, le terme
  anglais est \emph{deprecated method}\,---\, ne paniquez pas: le
  Method Finder est simplement en train d'essayer de chercher parmi
  tous les candidats incluant ainsi les méthodes obsolètes. \Clickz{}
  alors sur le bouton~\button{Proceed}.}}.

Un ast\'erisque au d\'ebut d'une ligne dans le panneau de droite du
Method Finder vous indique que cette m\'ethode est celle qui a \'et\'e
effectivement utilis\'ee pour obtenir le r\'esultat requis.
Ainsi, l'ast\'erisque devant \ct{String asUppercase} vous fait savoir
que la m\'ethode \mthind{String}{asUppercase} 
%ajout
(traduisible par ``en tant que majuscule'')
d\'efinie dans la classe \clsind{String} 
%ajout
(la classe des cha\^{\i}nes de caract\`eres)
a \'et\'e ex\'ecut\'ee et a renvoy\'e le r\'esultat voulu.
Les m\'ethodes qui n'ont pas d'ast\'erisque ne sont que d'autres
m\'ethodes que celles qui retournent le r\'esultat attendu.
\cmind{Character}{asUppercase} n'a pas \'et\'e ex\'ecut\'ee dans notre
exemple, parce que \ct{'eureka'} n'est pas un caract\`ere de classe \clsind{Character}.

\begin{figure}[hbt]
	{\centerline {\includegraphics[width=\textwidth]{MethodFinder-example1}}}
\caption{Trouver une méthode par l'exemple.
\figlabel{methodFinder-example1}}
\end{figure}

Vous pouvez aussi utiliser le Method Finder pour trouver des
m\'ethodes avec plusieurs arguments; par exemple, si vous recherchez
une m\'ethode qui trouve le plus grand commun diviseur de deux
entiers, vous pouvez essayer de saisir \ct{25. 35. 5} comme exemple.
Vous pouvez aussi donner au Method Finder de multiples exemples pour
restreindre le champ des recherches; le texte d'aide situ\'e dans le
panneau inf\'erieure vous apprendra \arelire{comment faire}. % CHANGE

%=================================================================
\section{D\'efinir une nouvelle m\'ethode}
\seeindex{développement orienté tests}{Test Driven Development}
L'av\`enement de la m\'ethodologie de d\'eveloppement orient\'e tests
ou \emphind{Test Driven Development}\cite{Beck03a} a chang\'e la
fa\c{c}on d'\'ecrire du code.
L'id\'ee derri\`ere cette technique aussi appel\'ee TDD se r\'esume par l'\'ecriture
du test qui d\'efinit le comportement d\'esir\'e de notre
code \emph{avant} celle du code proprement dit.
\`A partir de l\`a seulement, nous \'ecrivons le code qui satisfait au
test.
\seeindex{développement dirigé par le comportement}{Test Driven Development} 
\seeindex{Behavior Driven Development}{Test Driven Development}


%says something loudly and with emphasis
Supposons que nous voulions \'ecrire une m\'ethode qui ``hurle quelque
chose''. Qu'est-ce que cela veut dire au juste? Quelle serait le nom
le plus convenable pour une telle m\'ethode? Comment pourrions-nous
\^etre s\^urs que les programmeurs en charge de la maintenance future
du code auront une description sans ambigu\"{\i}t\'e de ce que ce code
est cens\'e faire?
Nous pouvons r\'epondre \`a toutes ces questions en proposant
l'exemple suivant:

%martial: j'ai change figures/testShoutConfirm.png et j'ai remplace
%"Don't panic" par "Pas de panique"
\begin{quote}
Quand nous envoyons le message \ct{shout} (qui veut dire ``crier'' en anglais)
\`a la cha\^{\i}ne de caract\`eres ``Pas de panique'', le r\'esultat
devrait \^etre ``PAS DE PANIQUE!''.
\end{quote}

\noindent
Pour faire de cet exemple quelque chose que le syst\`eme peut
utiliser, nous le transformons en m\'ethode de test:
\index{test}
%\seeindex{testing}{test}
\index{SUnit}

\needlines{3}
\begin{method}[testShout]{Un test pour la m\'ethode shout}
testShout
	self assert: ('Pas de panique' shout = 'PAS DE PANIQUEBANG')
\end{method} % BANG is the escape for !

Comment cr\'eons-nous une nouvelle m\'ethode dans \pharo? Premi\`erement,
nous devons d\'ecider quelle classe va accueillir la m\'ethode.
Dans ce cas, la m\'ethode \ct{shout} que nous testons ira dans la
classe \clsind{String}
%ajout
car c'est la classe des cha\^{\i}nes de caract\`eres et ``Pas de panique'' en est une.
Donc, par convention, le test correspondant ira dans une classe
nomm\'ee \clsind{StringTest}.

\begin{figure}[hbt]
	{\centerline {\includegraphics[width=\textwidth]{StringTest-newMethodTemplate}}}
\caption{Le patron de la nouvelle m\'ethode dans la classe \ct{StringTest}.
\figlabel{newMethodTemplate}}
\end{figure}

\dothis{Ouvrez un navigateur de classes sur la classe
  \ct{StringTest}. S\'electionnez un protocole appropri\'e pour notre
  m\'ethode; dans notre cas, \menu{tests - converting} 
%ajout
(signifiant tests de conversion, puisque notre m\'ethode modifiera le texte en retour),
comme nous pouvons le voir sur \figref{newMethodTemplate}.
Le texte surlign\'e dans le panneau inf\'erieur est un patron de
m\'ethode qui vous rappelle ce \`a quoi ressemble une m\'ethode.
Effacez-le et saisissez le code de  \tmthref{testShout}.}
Une fois que vous avez commenc\'e \`a entrer le texte dans le
navigateur, l'espace de saisie est entour\'e de rouge pour vous
rappeler que ce panneau contient des changements non-sauvegard\'es.
%ajout
Lorsque vous avez fini de saisir le texte de la m\'ethode de test,
s\'electionnez \menu{accept (s)} via le menu activ\'e en \actclickant{} dans ce panneau ou utilisez le raccourci-clavier
\short{s}: ainsi, vous compilerez et sauvegarderez votre m\'ethode.
\index{raccourci-clavier}
\index{raccourci-clavier!accept}
\seeindex{accept it}{raccourci-clavier, accept}
%ajout
\seeindex{méthode!accepter}{raccourci-clavier, accept}

\arelire{Si c'est la première fois que vous acceptez du code dans
  votre image, vous serez invité à saisir votre nom dans une
  fenêtre spécifique. Beaucoup de personnes ont contribué au code de
  l'image; c'est important de garder une trace de tous ceux qui créent
  ou modifient les méthodes. Entrez simplement votre prénom suivi de
  votre nom sans espaces ou de point de séparation.}
% If this is the first time you have accepted any code in your image, you will likely be prompted to enter your name. Since many people have contributed code to the image, it is important to keep track of everyone who creates or modifies methods. Simply enter your first and last names, without any spaces, or separated by a dot.

%\begin{figure}[hbt]
%\centerline {\includegraphics[width=0.35\textwidth]{initials}}
%\caption{Saisir ses initiales.
%\figlabel{initials}}
%\end{figure}

Puisqu'il n'y a pas encore de m\'ethode nomm\'ee \ct{shout}, le 
Browser vous demandera confirmation que c'est bien le nom que vous
d\'esirez\,---\,il vous sugg\`erera d'ailleurs d'autres noms de
m\'ethodes existantes dans le syst\`eme (voir \figref{testShoutConfirm}).
Ce comportement du navigateur est utile si vous aviez effectivement
fait une erreur de frappe. Mais ici, nous voulons \emph{vraiment}
\'ecrire \ct{shout} puisque c'est la m\'ethode que nous voulons
cr\'eer. D\`es lors, nous n'avons qu'\`a confirmer cela en
s\'electionnant la premi\`ere option parmi celles du menu, comme vous
le voyez sur \figref{testShoutConfirm}. 

%\begin{figure}[htb]
%\begin{minipage}[b]{0.48\textwidth}
%\centerline {\includegraphics[width=0.9\textwidth]{name}}
%\caption{Saisir son nom.\figlabel{name}}
%\end{minipage}
%\hfill
% \begin{figure}[hbt]
% \ifluluelse
% 	{\centerline {\includegraphics[width=\textwidth]{testShoutConfirm}}}
% 	{\centerline {\includegraphics[width=0.8\textwidth]{testShoutConfirm}}}
% \caption{Accepter la m\'ethode testShout dans la classe \ct{StringTest}.
% \figlabel{testShoutConfirm}}
% \end{figure}

\begin{figure}[htb]
 \centerline {\includegraphics[width=0.6\textwidth]{name}}
 \caption{Saisir son nom.\figlabel{name}}
 \end{figure}

\begin{figure}[htb]
 	{\centerline {\includegraphics[width=\textwidth]{testShoutConfirm}}}
 \caption{Accepter la m\'ethode \ct{testShout} dans la classe \ct{StringTest}.
 \figlabel{testShoutConfirm}}
 \end{figure}

 \dothis{Lancez votre test nouvellement cr\'e\'e: ouvrez le programme
   \ind{SUnit} nomm\'e \emphind{TestRunner} depuis le menu \menu{World}.}

 Les deux panneaux les plus \`a gauche se pr\'esentent un peu comme les
 panneaux sup\'erieurs du Browser. Le panneau de gauche contient
 une liste de cat\'egories restreintes aux cat\'egories qui
 contiennent des classes de test.

\dothis{S\'electionnez \scat{CollectionsTests-Text} et
le panneau juste à droite vous affichera alors toutes les classes de test
de cette cat\'egorie dont la classe \ct{StringTest}.
Les classes sont déjà séléctionnées dans cette catégorie; \clickz{}
alors sur \menu{Run Selected} pour lancer tous ces tests.} % CHANGE

\begin{figure}[hbt]
\centerline {\includegraphics[width=\textwidth]{testRunnerOnStringTest}}
\caption{Lancer les tests de \ct{String}.
\figlabel{testRunnerTestShout}}
\end{figure}

Vous devriez voir un message comme celui de
\figref{testRunnerTestShout}, vous indiquant qu'il y a eu une erreur
lors de l'ex\'ecution des tests. La liste des tests qui donne
naissance \`a une erreur est affich\'ee dans le panneau inf\'erieur de
droite; comme vous pouvez le voir, c'est bien
\ct{StringTest>>>#testShout} le coupable
(remarquez que la notation \ct{StringTest>>#testShout} est la fa\c{c}on dont \st
identifie la m\'ethode de la classe \ct{StringTest}).
Si vous \clickz{} sur cette ligne de texte, le test erron\'e sera
lanc\'e \`a nouveau mais, cette fois-ci, de telle fa\c{c}on que vous
voyez l'erreur surgir:
``\ct{MessageNotUnderstood: ByteString>>>shout}''.
\seeindex{\ct{>>}}{Behavior, \ct{>>}}
\cmindex{Behavior}{>>}

La fen\^etre qui s'ouvre avec le message d'erreur est le d\'ebogueur \st (voir \figref{predebugger}).
Nous verrons le d\'ebogueur nomm\'e \ind{Debugger} et ses
fonctionnalit\'es dans \charef{env}.
\seeindex{Debugger}{débogueur}

\begin{figure}[hbt]

	\centerline {\includegraphics[width=\textwidth]{Predebugger}}
\caption{La fenêtre de démarrage du d\'ebogueur.}
\figlabel{predebugger}
\end{figure}

L'erreur \'etait bien s\^ur attendue; lancer le test g\'en\`ere une
erreur parce que nous n'avons pas encore \'ecrit la m\'ethode qui dit
aux cha\^{\i}nes de caract\`eres comment hurler 
%ajout pour le francais
\cad comment r\'epondre au message \ct{shout}.
De toutes fa\c{c}ons, c'est une bonne pratique de s'assurer que le test
\'echoue; cela confirme que nous avons correctement
configur\'e notre machine \`a tests % testing machinery
et que le nouveau test est actuellement en cours d'ex\'ecution.
Une fois que vous avez vu l'erreur, vous pouvez cliquer sur le bouton
\button{Abandon} pour abandonner le test en cours, ce qui fermera la
fen\^etre du d\'ebogueur.
Sachez qu'en \st vous pouvez souvent d\'efinir la m\'ethode manquante
%ajout
directement depuis le d\'ebogueur 
en utilisant le bouton \button{Create}, en y \'editant la m\'ethode
nouvellement cr\'e\'ee puis, \emph{in fine}, en appuyant sur le bouton
\button{Proceed} pour poursuivre le test.

D\'efinissons maintenant la m\'ethode qui fera du test un succ\`es!

\dothis{S\'electionnez la classe \clsind{String} dans le 
  Browser et rendez-vous dans le protocole 
%ajout
d\'ej\`a existant des m\'ethodes de conversion et appel\'e
\menu{converting}. \`A la place du patron de cr\'eation de m\'ethode,
saisissez le texte de \tmthref{shout} et faites \menu{accept}
(saisissez \caret pour obtenir un \mbox{\ct{^}})}
\begin{method}[shout]{La m\'ethode shout}
shout
	^ self asUppercase, 'BANG'
\end{method}

La virgule est un op\'erateur de concat\'enation de cha\^{\i}nes de
caract\`eres, donc, le corps de cette m\'ethode ajoute un point
d'exclamation \`a la version majuscule
%martial: ajout pour rappeler au francais que uppercase eleve au majuscule
(obtenue avec la m\'ethode \mthind{String}{asUppercase})
de l'objet \ct{String} auquel le message \ct{shout} a \'et\'e
envoy\'e.
Le $\uparrow$ dit \`a \pharo que l'expression qui suit est la r\'eponse
que la m\'ethode doit retourner; dans notre cas, il s'agit de la
nouvelle cha\^{\i}ne concat\'en\'ee.
\seeindex{virgule}{Collection, opérateur virgule}
\index{Collection!opérateur virgule}

Est-ce que cette m\'ethode fonctionne? Lan\c{c}ons tout simplement
notre test afin de le savoir.

\dothis{Cliquez encore sur le bouton \menu{Run Selected} du Test
  Runner. Cette fois vous devriez obtenir une barre de signalisation
  verte (et non plus rouge) et son texte vous confirmera que tous les
  tests lanc\'es se feront sans aucun \'echec (ni \emph{failures}, ni
  \emph{errors}).}
Vous voyez une barre verte~\footnotemark\ dans le Test Runner? Bravo!
Sauvegardez votre image et faites une pause. 
%martial: ajout (ca fait toujours plaisir!)
Vous l'avez bien m\'erit\'e. 
% \footnotetext{En r\'ealit\'e, vous pourriez ne pas obtenir de barre
%   verte car certaines images contiennent des tests pour des
%   \emph{bugs} \`a corriger. Ne vous inqui\'etez pas!
% \pharo est en perp\'etuelle \'evolution.
% }

\begin{figure}[hbt]
	{\centerline{\includegraphics[width=0.7\textwidth]{String-Shout}}}
\caption{La m\'ethode \ct{shout} dans la classe \ct{String}.\figlabel{String-shout}}
\end{figure}

%=================================================================
\section{R\'esum\'e du chapitre}
%martial: j'ai mis 'intronise' si trop solennel, corriger par 'presente'
Dans ce chapitre, nous vous avons introduit \`a l'environnement de
\pharo et nous vous avons montr\'e comment utiliser certains de ses
principaux outils comme le   Browser, le
Method Finder et le Test Runner. Vous avez pu avoir un aper\c{c}u de la
syntaxe sans que vous puissiez encore la comprendre suffisamment \`a ce stade.

\begin{itemize}
  \item Un syst\`eme \pharo fonctionnel comprend une \emph{machine
      virtuelle} (souvent abr\'eg\'ee par VM), un fichier
    \emph{sources} et un couple de fichiers: une \emph{image} et un
    fichier \emph{changes}. Ces deux derniers sont les seuls \`a
    \^etre susceptibles de changer, puisqu'ils sauvegardent un clich\'e
    du syst\`eme actif.
  \item Quand vous restorez une image \pharo, vous vous retrouvez
    exactement dans le m\^eme \'etat\,---\,avec les m\^emes objets
    lanc\'es\,---\,que lorsque vous l'avez laiss\'ee au moment de votre derni\`ere
    sauvegarde de cette image.
  \item \pharo est destin\'e \`a fonctionner avec une souris \`a trois
    boutons \arelire{pour \click, \actclick ou \metaclick}.
 Si vous n'avez pas de souris \`a trois boutons, vous pouvez utiliser
 des touches de modifications au clavier pour obtenir le m\^eme effet.
  \item Vous \clickz sur l'arri\`ere-plan de
    \pharo pour faire appara\^{\i}tre le \emph{menu World} et pouvoir
    lancer depuis celui-ci divers outils.
  \item Un \emph{Workspace} ou espace de travail est un outil
    destin\'e \`a \'ecrire et \'evaluer des fragments de code. Vous
    pouvez aussi l'utiliser pour y stocker un texte quelconque.
  \item Vous pouvez utiliser des raccourcis-clavier sur du texte
    dans un Workspace ou tout autre outil pour en
    \'evaluer le code. Les plus importants sont \menu{do it}
    (\short{d}), \menu{print it} (\short{p}), \menu{inspect it}
    (\short{i}) et \menu{explore it} (\short{I}).
\index{raccourci-clavier}
  \item \sqmap est un outil pour t\'el\'echarger des paquetages utiles
    depuis Internet.
  \item Le navigateur de classes \emph{Browser} est le
    principal outil pour naviguer dans le code \pharo et pour
    d\'evelopper du nouveau code.
  \item Le \emph{Test Runner} permet d'effectuer des tests
    unitaires. Il supporte pleinement la m\'ethodologie de
    programmation orient\'ee tests connue sous le nom de \emph{Test
      Driven Development}.
\end{itemize}

%=================================================================
\ifx\wholebook\relax\else 
   \bibliographystyle{jurabib}
   \nobibliography{scg}
   \end{document}
\fi
%=================================================================

%%% Local Variables:
%%% coding: utf-8
%%% mode: latex
%%% TeX-master: t
%%% TeX-PDF-mode: t
%%% ispell-local-dictionary: "english"
%%% End:


%:First Application
% $Author$ traduit par Serge
% $Date$ 12/12/2007
% $Revision$
% relecture par Martial. Remarque generale:
% J'ai pris le parti de traduire les commentaires car il ne s'agit pas
% de methodes internes a Squeak mais des methodes creees par des
% lecteurs francophones
% j'ai fait aussi des ajouts destines pour l'essentiel a traduire les
% methodes ou certains menus (inutile de les reprendre dans la VO)
% relecture: Rene Mages (12/19/2007) + Martial -> Mon Dec 24 18:27:55 CET 2007
% relecture: Rene Mages (01/10/2008) version 14902
% relecture: Martial Boniou (01/30/2008) version 15374
% adaptation pour PBE - martial - (09/20/2009) version 29552
% relecture: Rene Mages (10/01/2010) version 30222
% relecture: Rene Mages  (25/06/2010) version 33735

\ifx\wholebook\relax\else
% --------------------------------------------
% Lulu:
	\documentclass[a4paper,10pt,twoside]{book}
	\usepackage[
		papersize={6.13in,9.21in},
		hmargin={.75in,.75in},
		vmargin={.75in,1in},
		ignoreheadfoot
	]{geometry}
	\input{../common.tex}
	\pagestyle{headings}
	\setboolean{lulu}{true}
% --------------------------------------------
% A4:
%	\documentclass[a4paper,11pt,twoside]{book}
%	\input{../common.tex}
%	\usepackage{a4wide}
% --------------------------------------------
    \graphicspath{{figures/} {../figures/}}
	\begin{document}
	% \renewcommand{\nnbb}[2]{} % Disable editorial comments
	\sloppy
\fi
%=================================================================
\chapter{\titreFirstapp}
\chalabel{firstApp}

Dans ce chapitre, nous allons développer un jeu simple de
réflexion, le jeu \ind{Lights Out}~\footnote{En anglais,
  \url{http://en.wikipedia.org/wiki/Lights_Out_(game)}.
% martial - TODO ajouter une définition en français
}. 
En cours de route, nous allons faire la démonstration de la plupart des outils que \arelire{les développeurs \pharo utilisent pour}
% Rene propose de supprimer pharo à la ligne précedente.
% Martial: comme tu veux, ça ne me dérange pas; ça rappelle que les méthodologies et les outils sont associés aux dév. Pharo et pas au dév. Haskell.
construire et déboguer leurs programmes et comment les programmes sont échangés entre les développeurs. Nous verrons notamment le navigateur de classes, l'inspecteur d'objet, le débogueur et le navigateur de \ind{paquetage}{}s \ind{Monticello}. 
Le développement avec \st est efficace: vous découvrirez que vous passerez beaucoup plus de temps à écrire du code et beaucoup moins à gérer le processus de développement. 
Ceci est en partie du au fait que \st est un langage très simple, et d'autre part que les outils qui forment l'environnement de programmation sont très intégrés avec le langage.

%=================================================================
\section{Le jeu Lights Out}

\begin{figure}[ht]
	\vskip -\baselineskip
	\centerline{\includegraphics[width=.3\linewidth]{GameBoard}}
	\caption{Le plateau de jeu Lights Out. L'utilisateur vient de cliquer sur une case avec la souris comme le montre le curseur.
	\figlabel{gameBoard}}
\end{figure}

Pour vous montrer comment utiliser les outils de développement de
\pharo, nous allons construire un jeu très simple nommé
\emph{Lights Out}.  Le plateau de jeu est montré dans
\figref{gameBoard}; il consiste en un tableau rectangulaire de
\emph{cellules} jaunes claires.  Lorsque l'on clique sur l'une de ces
cellules avec la souris, les quatre qui l'entourent deviennent
bleues. Cliquez de nouveau et elles repassent au jaune pâle. Le but du 
jeu est de passer au bleu autant de cellules que possible.

Le jeu Lights Out montré dans \figref{gameBoard} est fait de deux types d'objets: le plateau de jeu lui-même et une centaine de cellule-objets individuelles. Le code \pharo pour réaliser ce jeu va contenir deux classes: une pour le jeu et une autre pour les cellules.
Nous allons voir maintenant comment définir ces deux classes en utilisant les outils de programmation de \pharo.

%=================================================================
\section{Créer un nouveau paquetage}

Nous avons déjà vu le \ind{Browser}
dans \charef{quick}, où nous avons appris à naviguer dans les classes
et méthodes, et à définir de nouvelles méthodes.
Nous allons maintenant voir comment créer des paquetages (ou \emph{packages}), des catégories et des classes.
\index{catégorie!création}
\index{package!création}

\dothis{Ouvrez un Browser et \actclickz{} sur le panneau des paquetages.
Sélectionnez \menu{create package}\footnote{Nous supposons que le 
\aretirer{Package} Browser est installé en tant que navigateur de classes par défaut. 
Si le Browser ne ressemble pas à celui de la
\figref{addPackage}, vous aurez besoin de changer le navigateur par
défaut. Voyez \faqref{packageBrowser}.}.}

\begin{figure}[htb]
\begin{minipage}[b]{0.48\textwidth}
\ifluluelse
	{\centerline {\includegraphics[width=0.9\textwidth]{AddPackage}}}
	{\centerline {\includegraphics[scale=0.7]{AddPackage}}}
	\caption{Ajouter un paquetage.
	\figlabel{addPackage}}
\end{minipage}
\hfill
\begin{minipage}[b]{0.48\textwidth}
\ifluluelse
	{\centerline {\includegraphics[width=0.8\textwidth]{ClassTemplate}}}
	{\centerline {\includegraphics[scale=0.6]{ClassTemplate}}}
	\caption{Le \arelire{patron} de création d'une classe.
	\figlabel{classTemplate}}
\end{minipage}
\end{figure}

Tapez le nom du nouveau paquetage (nous allons utiliser
\scat{PBE-LightsOut}) dans la boîte de dialogue et \clickz{} sur
\menu{accept} (ou appuyez simplement sur la touche entrée); le nouveau
paquetage est créé et s'affiche dans la liste des paquetages en
respectant l'ordre alphabétique.

%=================================================================
\section{Définir la classe LOCell}

Pour l'instant, il n'y a aucune classe dans le nouveau paquetage. Cependant le
panneau de code inférieur\,---\,qui est la zone principale
d'édition\,---\,affiche un patron pour faciliter la création d'une
nouvelle classe (voir \figref{classTemplate}).

Ce modèle nous montre une expression \st qui envoie un message à la
classe appelée \ct{Object}, lui demandant de créer une sous-classe
appelée \ct{NameOfSubClass}.  La nouvelle classe n'a pas de variables
et devrait appartenir à la catégorie \scat{PBE-LightsOut}.

\subsection{\arelire{À propos des catégories et des paquetages}}
\seclabel{categoriesPackages}

Historiquement, \st{} ne connaît que les \emph{catégories}. Vous
pouvez vous interroger sur la différence qui peut exister entre
catégories et paquetages.
Une catégorie est simplement une collection de classes apparentées
dans une image \st. Un \emph{paquetage} (ou \emph{package})
est une collection de classes apparentées \emph{et de méthodes
  d'extension} qui peuvent être versionnées via l'outil de versionnage
Monticello.
Par convention, les noms de paquetages et les noms de catégories sont
les mêmes.
La plupart du temps, nous n'accordons pas de différence mais, dans ce
livre, nous serons attentifs à utiliser la terminologie exacte car il
y a des cas où la différence est cruciale.
Vous en apprendrez plus lorsque nous aborderons le travail avec
Monticello.
\index{paquetage}
\seeindex{package}{paquetage}
\index{catégorie}

% \subsection{On Categories and Packages}
% \seclabel{categoriesPackages}

% Historically, \st only knows about \emph{categories}, not packages.
% You may well ask, what is the difference?
% A category is simply a collection of related classes in a \st image.
% A \emph{package} is a collection of related classes \emph{and extension methods} that may be versioned using the Monticello versioning tool.
% By convention, package names and category names are the same.
% For most purposes we do not care about the difference, but we will be careful to use the correct terminology in this book since there are points where the difference is crucial.
% We will learn more when we start working with Monticello.
% \index{package}
% \index{category}

\subsection{Créer une nouvelle classe}

Nous modifions simplement le modèle afin de créer la classe que nous souhaitons.

\dothis{Modifiez le modèle de création d'une classe comme suit:}
\begin{itemize}
  \item remplacez \clsind{Object} par \clsind{SimpleSwitchMorph};
  \item remplacez \ct{NameOfSubClass} par \clsind{LOCell};
  \item ajoutez \ct{mouseAction} dans la liste de variables d'instances.
\end{itemize}
Le résultat doit ressembler à \tclsref{firstClassDef}.

\needlines{5}
\begin{classdef}[firstClassDef]{Définition de la classe \ct| LOCell|}
SimpleSwitchMorph subclass: #LOCell
   instanceVariableNames: 'mouseAction'
   classVariableNames: ''
   poolDictionaries: ''
   category: 'PBE-LightsOut'
\end{classdef}
\index{Browser!définir une classe}
\index{classe!création}
\index{Morphic}

Cette nouvelle définition consiste en une expression \st qui envoie un message à une classe existante \ct{SimpleSwitchMorph}, lui demandant de créer une sous-classe appelée \ct{LOCell}
(en fait, comme \ct{LOCell} n'existe pas encore, nous passons comme argument le \emphind{symbole} \ct{#LOCell} qui correspond au nom de la classe à créer).
Nous indiquons également que les instances de cette nouvelle classe doivent avoir une variable d'instance \ct{mouseAction}, que nous utiliserons pour définir l'action que la cellule doit effectuer lorsque l'utilisateur clique dessus avec la souris.

\emph{À ce point, nous n'avons encore rien construit.}
Notez que le bord du panneau du modèle de la classe est passé en rouge
(voir \figref{acceptClassDef}).
Cela signifie qu'il y a des \emph{modifications non sauvegardées}.
Pour effectivement envoyer ce message, vous devez faire \menu{accept}.

\begin{figure}[h!t]
\ifluluelse
	{\centerline {\includegraphics[width=\textwidth]{AcceptClassDef}}}
	{\centerline {\includegraphics[scale=0.7]{AcceptClassDef}}}
\caption{Le modèle de création d'une classe.
\figlabel{acceptClassDef}}
\end{figure}

\dothis{Acceptez la nouvelle définition de classe.}
\Actclickz{} et sélectionnez \menu{accept} ou encore utilisez le
raccourci-clavier \short{s} (pour ``save'' 
%ajout
\cad sauvegarder).
Ce message sera envoyé à \ct{SimpleSwitchMorph}, ce qui aura pour
effet de compiler la nouvelle classe.
\index{raccourci-clavier!accept}

Une fois la définition de classe acceptée, la classe va être créée et
apparaîtra dans le panneau des classes du navigateur (voir \figref{LOCell}).
Le panneau d'édition montre maintenant la définition de la classe et
un petit panneau dessous vous invite à écrire quelques mots décrivant
l'objectif de la classe. Nous appelons cela un \emph{commentaire de
  classe}; il est assez important d'en écrire un qui donnera aux
autres développeurs une vision 
%de haut niveau 
globale de votre classe.
Les Smalltalkiens accordent une grande valeur à la lisibilité de leur
code et il n'est pas habituel de trouver des commentaires détaillés
dans leurs méthodes; la philosophie est plutôt d'avoir un code qui
parle de lui-même (si cela n'est pas le cas, vous devrez le
refactoriser jusqu'à ce que ça le soit!). 
Un \subind{classe}{commentaire} de classe ne nécessite pas une
description détaillée de la classe, mais quelques mots la décrivant
sont vitaux si les développeurs qui viennent après vous souhaitent
passer un peu de temps sur votre classe.
\index{refactoring}

\dothis{Tapez un commentaire de classe pour \ct{LOCell} et
  acceptez-le; vous aurez tout le loisir de l'améliorer par la suite.}

\begin{figure}[h!t]
%\ifluluelse
%	{\centerline {\includegraphics[width=\textwidth]{LOCell}}}
%	{\centerline {\includegraphics[scale=0.7]{LOCell}}}
%martial: description de la frame de commentaire ajoute pour les non-anglophones
\caption{La classe nouvellement créée \ct{LOCell}. Le panneau
  inférieur est le panneau de commentaires; par défaut, il dit:
  ``CETTE CLASSE N'A PAS DE COMMENTAIRE!''.
\figlabel{LOCell}}
\end{figure}

%=================================================================
\section{Ajouter des méthodes à la classe}

Ajoutons maintenant quelques méthodes à notre classe.

\dothis{Sélectionnez le protocole \prot{-{}-all-{}-} dans le panneau 
%des contrôleurs ??
des protocoles.}
Vous voyez maintenant un modèle pour la création d'une méthode dans le panneau d'édition.
Sélectionnez-le et remplacez-le par le texte de \tmthref{scbecellinitialize}.
\protindex{all}
\index{méthode!création}
\index{Browser!définir une méthode}

\needlines{10}
\begin{numMethod}[scbecellinitialize]{Initialiser les instances de \ct{LOCell}.}
initialize
   super initialize.
   self label: ''.
   self borderWidth: 2.
   bounds := 0@0 corner: 16@16.
   offColor := Color paleYellow.
   onColor := Color paleBlue darker.
   self useSquareCorners.
   self turnOff
\end{numMethod}
\seeindex{méthode!d'initialisation}{initialisation}
\index{initialisation}

\noindent
Notez que les caractères \ct{''} de la ligne 3 sont deux
%quotes séparées avec rien entre les deux, et pas un guillemet ! 
apostrophes~\footnote{Nous utilisons le terme ``quote'' en anglais.} 
sans espace entre elles, et non un guillemet (")!
\ct{''} représente la chaîne de caractères vide.

\dothis{Faites un \menu{accept} de cette définition de méthode.}

Que fait le code ci-dessus?  Nous n'allons pas rentrer dans tous les
détails maintenant (ce sera l'objet du reste de ce livre!), mais nous
allons vous en donner un bref aperçu. Reprenons le code ligne par ligne.

Notons que la méthode s'appelle \mthind{LOCell}{initialize}.
Ce nom dit bien ce qu'il veut dire~\footnote{En anglais, puisque c'est
  la langue conventionnelle en \st.}!
Par convention, si une classe définit une méthode nommée
\ct{initialize}, cette méthode sera appelée dès que l'objet aura été créé.
Ainsi dès que nous évaluons \ct{LOCell new}, le message \ct{initialize} sera envoyé automatiquement à cet objet nouvellement créé.
Les méthodes d'initialisation sont utilisées pour définir l'état des objets, généralement pour donner une valeur à leurs variables d'instances; c'est exactement ce que nous faisons ici.
%\seeindex{Object!initialization}{initialization}
\seeindex{objet!initialisation}{initialisation}
\seeindex{Object!initialize}{initialisation}
\index{initialisation}

La première action de cette méthode (ligne 2) est d'exécuter la méthode \ct{initialize} de sa super-classe, \ct{SimpleSwitchMorph}.
L'idée est que tout état hérité sera initialisé correctement par la méthode \ct{initialize} de la super-classe.
C'est toujours une bonne idée d'initialiser l'état hérité en envoyant
\ct{super initialize} avant de faire tout autre chose; nous ne savons
pas exactement ce que la méthode \ct{initialize} de
\ct{SimpleSwitchMorph} va faire, et nous ne nous en soucions pas, mais
il est raisonnable de penser que cette méthode va initialiser quelques
variables d'instance avec des valeurs par défaut, et qu'il vaut mieux
le faire au risque de se retrouver dans un état incorrect.

Le reste de la méthode donne un état à cet objet.
Par exemple, envoyer \ct{self label: ''} affecte le label de cet objet avec la chaîne de caractères vide.
\pvindex{self}

L'expression \ct{0@0 corner: 16@16} nécessite probablement plus d'explications.
\ct{0@0} représente un objet \clsind{Point} dont les coordonnées $x$ et $y$ ont été fixées à 0.
En fait, \ct{0@0} envoie le message \ct{@} au
% Yuck... the following should be \mthind{Number}{@} 
%%% THIS IS BROKEN -- don't do it! (on)
%\def\atsign{\textsf{@}}%
%{\makeatletter
%	\protected@write\@indexfile{}%
%    {\string\indexentry{\string\atsign|see{Number, \string\atsign}}{\thepage}}%
%	\protected@write\@indexfile{}%
%    {\string\indexentry{Number!\string\atsign|hyperpage}{\thepage}}%
%	\makeatother}
nombre \ct{0} avec l'argument \ct{0}.
L'effet produit sera que le nombre \ct{0} va demander à la classe \ct{Point} de créer une nouvelle instance de coordonnées (0,0).
Puis, nous envoyons à ce nouveau point le message \ct{corner: 16@16}, ce qui cause la création d'un \clsind{Rectangle} de coins \ct{0@0} et \ct{16@16}.
Ce nouveau rectangle va être affecté à la variable \ct{bounds} héritée de la super-classe.

Notez que l'origine de l'écran \pharo est en \emph{haut à gauche} et que les coordonnées en $y$ augmentent \emph{vers le bas}.

Le reste de la méthode doit être compréhensible de lui-même.
Une partie de l'art d'écrire du bon code \st est de choisir les bons
noms de méthodes de telle sorte que le code \st peut être lu comme
de l'anglais simplifié (\emph{English pidgin}).  
% Rene prefere ne plus utiliser : (ou parler \textit{petit-nègre}).
Vous devriez être capable d'imaginer l'objet se parlant à lui-même et
dire:  ``Utilise des bords carrés!'' (d'où \ct{useSquareCorners}),
``Éteins les cellules!'' (en anglais, \ct{turnOff}).
% martial: note provisoire pour Serge: les phrases entre guillemets
% etaient ecrites avec des \ct{} or ces balises sont pour le
% code. Donc j'ai prefere laisser ta traduction mais mettre les noms
% des messages entre \ct{} a cote entre parentheses

%=================================================================
\section{Inspecter un objet}

Vous pouvez tester l'effet du code que vous avez écrit en créant un
nouvel objet \ct{LOCell} et en l'inspectant avec l'inspecteur nommé \arelire{Inspector}.
% ajout -vf avec l'inspecteur nomme Inspector
\dothis{Ouvrez un espace de travail (Workspace). Tapez l'expression \ct{LOCell new} et choisissez \menu{inspect it}.}

\begin{figure}[htbp]
   \centering
%   \includegraphics[width=\textwidth]{LOCellInspector} 
   \caption{L'inspecteur utilisé pour examiner l'objet LOCell.\figlabel{LOCellInspector}}
\end{figure}

Le panneau gauche de l'\ind{inspecteur} montre une liste de variables
d'instances; si vous en sélectionnez une (par exemple
\mbox{\ct{bounds}),} la valeur de la \ind{variable d'instance} est
affichée dans le panneau droit. % CHANGE

%  You can also use the inspector to change the value of an instance
%  variable.
% \dothis{Changez la valeur de \ct{bounds} à \ct{0@0 corner: 50@50} et
%   faites un \menu{accept}.}
% \on{This does not work any more. I get:}
% \ct{OTNamedVariableNode(Object)>>doesNotUnderstand: #selectedClass}
% \on{should use the mini workspace instead to send bounds: ?}

Le panneau en bas d'un inspecteur est un mini-espace de
travail. C'est très utile car, dans cet espace de travail, la
  pseudo-variable \self{} est liée à l'objet sélectionné.

% \dothis{Tapez le texte \ct{self openInWorld} dans la zone du bas et
%  choisissez \menu{do it} via le menu.}
\dothis{Sélectionnez LOCell à la racine de la fenêtre de
    l'inspecteur. Saisissez l'expression 
\ct{self bounds: (200@200 corner: 250@250)} dans le panneau inférieur et faîtes un
    \menu{do it} (via le menu contextuel ou le raccourci-clavier).}

La variable \ct{bounds} devrait changer dans l'inspecteur. 
Saisissez maintenant \ct{self openInWorld} dans
  ce même panneau et évaluez le code avec \menu{do it}. La cellule
  doit apparaître près du coin supérieur gauche, là où les coordonnées
  \ct{bounds} doivent le faire apparaître.
\Metaclickz{} sur la cellule afin de faire apparaître son
\subind{Morphic}{halo} Morphic.

Déplacez la cellule avec la poignée marron (à 
%côte de celle en haut à droite
gauche de l'icône du coin supérieur droit) et redimensionnez-la avec la poignée jaune (en bas à droite).
Vérifiez que les limites indiquées par l'inspecteur sont modifiées en
conséquence (il faudra peut-être \actclick{} sur \menu{refresh} pour voir les nouvelles valeurs). % CHANGE

\begin{figure}[htbp]
\centering
%\ifluluelse
%	{\includegraphics[width=\textwidth]{LOCellResize} }
%	{\includegraphics[scale=0.7]{LOCellResize} }
\caption{Redimensionner la cellule.\figlabel{cellresize}}
\end{figure}

\dothis{Détruisez la cellule en cliquant sur le \ct{x} de la poignée 
%mauve.
rose pâle (en haut à gauche).}

%=================================================================
\section{Définir la classe LOGame}

Créons maintenant l'autre classe dont nous avons besoin dans le jeu; nous l'appellerons \clsind{LOGame}.

\dothis{Faites apparaître le modèle de définition de classe dans la fenêtre principale du navigateur.}
Pour cela, \clickz{} sur le nom du paquetage.
Éditez le code de telle sorte qu'il puisse être lu comme suit puis faites \menu{accept}.

\needlines{6}
\begin{classdef}[sbegame]{Définition de la classe \ct{LOGame}}
BorderedMorph subclass: #LOGame
   instanceVariableNames: ''
   classVariableNames: ''
   poolDictionaries: ''
   category: 'PBE-LightsOut'
\end{classdef}

Ici nous sous-classons \clsind{BorderedMorph}; \clsind{Morph} est la
super-classe de toutes les formes graphiques de \pharo, et (surprise!)
un \ct{BorderedMorph} est un \ct{Morph} avec un bord.  
Nous pourrions également insérer les noms des variables d'instances
entre apostrophes sur la seconde ligne, mais pour l'instant 
laissons cette liste vide.

Définissons maintenant une méthode \mthind{LOGame}{initialize} pour \ct{LOGame}.

\dothis{Tapez ce qui suit dans le navigateur comme une méthode de \ct{LOGame} et faites ensuite \menu{accept}:}

\begin{numMethod}[sbegameinitialize]{Initialisation du jeu}
initialize
   | sampleCell width height n |
   super initialize.
   n := self cellsPerSide.
   sampleCell := LOCell new.
   width := sampleCell width.
   height := sampleCell height.
   self bounds: (5@5 extent: ((width*n) @(height*n)) + (2 * self borderWidth)).
   cells := Matrix new: n tabulate: [ :i :j | self newCellAt: i at: j ]
\end{numMethod}

%\sd{it would be nicer if we would not have to create an instance of LOCell for nothing}
%\on{yes}

\pharo va se plaindre qu'il ne connaît pas la signification de
certains termes.
Il vous indique alors qu'il ne connaît pas le message
\ct{cellsPerSide} (en français, ``cellules par côté'') et
vous suggère un certain nombre de propositions, dans le cas où il
s'agirait d'une erreur de frappe.

\begin{figure}[htb]
\begin{minipage}{0.48\textwidth}
	\centering
	\ifluluelse
		{\includegraphics[width=\textwidth]{UnknownSelector}}
		{\includegraphics[scale=0.7]{UnknownSelector}}
	\caption{\pharo détecte un sélecteur inconnu.\figlabel{unknownSelector}}
\end{minipage}
\hfill
\begin{minipage}{0.48\textwidth}
	\centering
	\ifluluelse
		{\includegraphics[width=\textwidth]{DeclareInstanceVar}}
		{\includegraphics[scale=0.7]{DeclareInstanceVar}}
	\caption{Déclaration d'une nouvelle variable d'instance.\figlabel{declareInstance}}
\end{minipage}
\end{figure}

Mais \ct{cellsPerSide} n'est pas une erreur \,---\, c'est juste le nom d'une méthode que nous n'avons pas encore définie\,---\,que nous allons écrire dans une minute ou deux.

\dothis{Sélectionnez la première option du menu, afin de confirmer que nous parlons bien de \ct{cellsPerSide}.}

Puis, \pharo va se plaindre de ne pas connaître la signification de \ct{cells}. Il vous offre plusieurs possibilités de correction.

\dothis{Choisissez \menu{declare instance} parce que nous souhaitons que \ct{cells} soit une variable d'instance.}
Enfin, \pharo va se plaindre à propos du message \ct{newCellAt:at:}
envoyé à la dernière ligne; ce n'est pas non plus une erreur,
confirmez donc ce message aussi.
%\index{on the fly variable definition}
%\index{instance variable definition} 
\index{définition de variable à la volée}
\index{définition de variable d'instance}

Si vous regardez maintenant de nouveau la définition de classe (en cliquant sur le bouton \button{instance}), vous allez voir que la définition a été modifiée pour inclure la variable d'instance \ct{cells}.

Examinons plus précisemment cette méthode \ct{initialize}.
La ligne \mbox{\ct{| sampleCell width height n |}} déclare 4 variables temporaires. Elles sont appelées variables temporaires car leur portée et leur durée de vie sont limitées à cette méthode. Des variables temporaires avec des noms explicites sont utiles afin de rendre le code plus lisible. \st n'a pas de syntaxe spéciale pour distinguer les constantes et les variables et en fait, ces 4 ``variables'' sont ici des constantes. Les lignes 4 à 7 définissent ces constantes.

Quelle doit être la taille de notre plateau de jeu? Assez grande pour pouvoir contenir un certain nombre de cellules et pour pouvoir dessiner un bord autour d'elles.
Quel est le bon nombre de cellules? 5? 10? 100? Nous ne le savons pas
pour l'instant et si nous le savions, il y aurait des chances pour que
nous changions d'idée par la suite. Nous déléguons donc la
responsabilité de connaître ce nombre à une autre méthode, que nous
appelons \ct{cellsPerSide} et que nous écrirons bientôt.
%martial: 'bientot' remplace 'dans une minute ou deux'. Trop repetitif!
C'est parce que nous envoyons le message \ct{cellsPerSide} avant de
définir une méthode avec ce nom que \pharo nous demande ``confirm,
correct, or cancel'' (\cad ``confirmez, corrigez ou annulez'') lorsque
nous acceptons le corps de la méthode \mbox{\ct{initialize}.}
Que cela ne vous inquiète pas:
c'est en fait une bonne pratique d'écrire en fonction d'autres méthodes qui ne sont pas encore définies.
Pourquoi? En fait, ce n'est que quand nous avons commencé à écrire la
méthode \ct{initialize} que nous nous sommes rendu compte que nous en
avions besoin, et à ce point, nous lui avons donné un nom 
%qui fait sens 
significatif et nous avons poursuivi, sans nous interrompre.
 
La quatrième ligne utilise cette méthode: le code \st \ct{self cellsPerSide} envoie le message \ct{cellsPerSide} à \pvind{self}, \cad à l'objet lui-même. La réponse, qui sera le nombre de cellules par côté du plateau de jeu, est affectée à \ct{n}.

Les trois lignes suivantes créent un nouvel objet \ct{LOCell} et assignent sa largeur et sa hauteur aux variables temporaires appropriées.

%The eighth line sends the message \ct{bounds:} to \self.
%\ct{bounds:} is a method that we inherit from our superclass; it is
%used to define the space on the screen that this Morph will occupy.  
%The single colon (\ct{:}) at the end of the name says that \ct{bounds:} expects a single parameter, which should be a rectangle object.
La ligne 8 fixe la valeur de \ct{bounds} (définissant les limites) du nouvel objet. Ne vous inquiétez pas trop sur les détails pour l'instant. Croyez-nous: l'expression entre parenthèses crée un carré avec comme origine (\ie son coin haut à gauche) le point (5,5) et son coin bas droit suffisamment loin afin d'avoir de l'espace pour le bon nombre de cellules.

La dernière ligne affecte la variable d'instance \ct{cells} de l'objet
\ct{LOGame} à un nouvel objet \clsind{Matrix} avec le bon nombre de lignes et de colonnes.
Nous réalisons cela en envoyant le message \ct{new:tabulate:} à la classe \ct{Matrix} (les classes sont des objets aussi, nous pouvons leur envoyer des messages).
Nous savons que \mthind{Matrix class}{new:tabulate:} prend deux arguments parce qu'il y a deux fois deux points (\ct{:}) dans son nom. Les arguments arrivent à droite après les deux points.
Si vous êtes habitué à des langages de programmation où les arguments sont tous mis à l'intérieur de parenthèses, ceci peut sembler surprenant dans un premier temps. Ne vous inquiétez pas, c'est juste de la syntaxe!
Cela s'avère être une excellente syntaxe car le nom de la méthode peut être utiliser pour expliquer le rôle des arguments. Par exemple, il est très clair que \ct{Matrix rows:5 columns:2} a 5 lignes et 2 colonnes et non pas 2 lignes et 5 colonnes.
\cmindex{Matrix class}{rows:columns:}

\ct{Matrix new: n tabulate: [ :i :j | self newCellAt: i at: j ]} crée une nouvelle matrice de taille \ct{n}{$\times$}\ct{n} et initialise ses éléments. La valeur initiale de chaque élément dépend de ses coordonnées. L'élément \ct{(i,j)} sera initialisé avec le résultat de l'évaluation de \ct{self newCellAt: i at: j}.  

%:===> Pretty-print is broken! (how to pretty-print?)

% \on{I think it is silly to copy paste from the pretty-print view to the normal view}

% Voilà pour \ct{initialize}!  Lorsque vous acceptez cette méthode, vous
% pouvez également simultanément en profiter pour formater proprement
% votre code. Vous n'avez pas besoin de faire cela à la main: à partir
% du menu déclenché en \actclickant, sélectionnez
% \menu{more \ldots \go prettyprint}, et le navigateur vous formatera
% le code pour vous. Vous n'avez qu'à faire \menu{accept} après avoir 
% choisi ce nouveau formatage en \subind{méthode}{pretty-print} ou 
% %martial: j'ai reformule les deux phrases suivantes (lourdes)
% bien, si le résultat ne vous plaît pas, vous pouvez utiliser 
% le raccourci-clavier \subind{raccourci-clavier}{cancel}
% (\short{l}\,---\, ceci est un \emph{L} en minuscule) pour annuler.
% Vous pouvez également configurer votre navigateur de code pour
% toujours formater élégamment le code chaque fois qu'il vous
% l'affiche:
% utilisez pour ce faire le bouton le plus à droite dans la barre de
% boutons de votre navigateur de classe afin d'ajuster la vue.
% \seeindex{pretty-print}{méthode}

% Si vous utilisez beaucoup le menu \menu{more\, \ldots}, sachez que
% vous pouvez appuyer sur la touche {\sc shift} lorsque vous cliquez
% afin de le faire apparaître directement.

%=================================================================
\section{Organiser les méthodes en protocoles}

Avant de définir de nouvelles méthodes, attardons-nous un peu sur le 
troisième panneau en haut du navigateur.
De la même façon que le premier panneau du navigateur nous permet de
catégoriser les classes dans des paquetages de telle sorte
que nous ne soyons pas submergés par une liste de noms de classes trop
longue dans le second panneau, le troisième panneau nous permet de
catégoriser les méthodes de telle sorte que n'ayons pas une liste de
méthodes trop longue dans le quatrième panneau.
Ces catégories de méthodes sont appelées ``protocoles''.
\index{protocole}

S'il n'y avait que quelques méthodes par classe, ce niveau hiérarchique supplémentaire ne serait pas vraiment nécessaire.
C'est pour cela que le navigateur offre un protocole virtuel
\prot{-{}-all-{}-} (\cad ``tout'' en français) qui, vous ne serez pas surpris de l'apprendre, contient toutes les méthodes de la classe.
\protindex{all}

\begin{figure}[htbp]
   \centering
   \includegraphics[width=\textwidth]{Categorize} 
   \caption{Catégoriser \arelire{de façon automatique} toutes les méthodes
     non catégorisées.\figlabel{categorize}}
\end{figure}

Si vous avez suivi l'exemple jusqu'à présent, le troisième panneau doit contenir le protocole \protind{as yet unclassified}~\footnote{NdT: non encore classifié.}.

\dothis{\Actclickz{} dans le panneau des protocoles et sélectionnez
  \menu{various \go{} categorize automatically} afin de régler ce
  problème et déplacer les méthodes \ct{initialize} vers un nouveau
  protocole appelé \protind{initialization}.} % CHANGE
Comment \pharo sait que c'est le bon protocole? En général,
\pharo ne peut pas le savoir mais dans notre cas, il y a aussi une méthode \ct{initialize} dans la super-classe et \pharo suppose que notre méthode \ct{initialize} doit être rangée dans la même catégorie que celle qu'elle surcharge.
%\index{method!categorize}
\index{méthode!catégorisation}

% Vous pouvez également vous rendre compte que \pharo a déjà rangé votre
% méthode \ct{initialize} dans le protocole \protind{initialization}. Si
% c'est le cas, c'est probablement que vous avez chargé un paquetage nommé \ct{AutomaticMethodCategorizer} dans votre image.

\paragraph{Une convention typographique.} Les Smalltalkiens utilisent fréquemment la notation ``\verb|>>|'' afin d'identifier la classe à laquelle la méthode appartient, ainsi par exemple, la méthode \ct{cellsPerSide} de la classe \ct{LOGame} sera référencée par \ct{LOGame>>cellsPerSide}.
Afin d'indiquer que cela ne fait pas partie de la syntaxe de \st, nous utiliserons plutôt le symbole spécial \ct{>>>} de telle sorte que cette méthode apparaîtra dans le texte comme \ct{LOGame>>>cellsPerSide}
\cmindex{Behavior}{>>}

À partir de maintenant, lorsque nous voudrons montrer une méthode dans ce livre, nous écrirons le nom de cette méthode sous cette forme. Bien sûr, lorsque vous tapez le code dans un navigateur, vous n'avez pas à taper le nom de la classe ou le \ct{>>>}; vous devrez juste vous assurez que la classe appropriée est sélectionnée dans le panneau des classes.

Définissons maintenant les autres méthodes qui sont utilisées par la méthode \ct{LOGame>>>initialize}. Les deux peuvent être mises dans le protocole \prot{initialization}.

\begin{method}[sbegamecellsperside]{Une méthode constante}
LOGame>>>cellsPerSide
   "Le nombre de cellules le long de chaque !côté! du jeu"
   ^ 10
\end{method}
\cmindex{LOGame}{cellsPerSide}
% le commentaire "The number of cells along each side of the game"
\index{méthode!constante}

Cette méthode ne peut pas être plus simple: elle retourne la constante
10. Représenter les constantes comme des méthodes a comme avantage que 
si le programme évolue de telle sorte que la constante dépende d'autres 
propriétés, la méthode peut être modifiée pour calculer la valeur.

\needlines{10}
\begin{method}[newCellAt:at:]{Une méthode d'aide à l'initialisation}
LOGame>>>newCellAt: i at: j
   "!Crée! une cellule !à! la position (i,j) et l'ajoute dans ma !représentation! graphique !à! la position correcte. Retourne une nouvelle cellule"
   | c origin |
   c := LOCell new.
   origin := self innerBounds origin.
   self addMorph: c.
   c position: ((i - 1) * c width) @ ((j - 1) * c height) + origin.
   c mouseAction: [self toggleNeighboursOfCellAt: i at: j]
\end{method}
% le commentaire "Create a cell for position (i,j) and add it to my on-screen
%   representation at the appropriate screen position.  Answer the new cell"
\cmindex{LOGame}{newCellAt:at:}
%   ^ c      "omit this final line to create a bug"

\dothis{Ajoutez les méthodes \ct{LOGame>>>cellsPerSide} et \ct{LOGame>>>newCellAt:at:}.}
Confirmez que les sélecteurs \ct{toggleNeighboursOfCellAt:at:} et \ct{mouseAction:} s'épellent correctement.

\Tmthref{newCellAt:at:} retourne une nouvelle cellule \ct{LOCell} à la position \ct{(i,j)} dans la matrice (\clsind{Matrix}) de cellules.
La dernière ligne définit l'action de la souris (\ct{mouseAction}) associée à la cellule comme le \emph{bloc}
\mbox{\ct{[self toggleNeighboursOfCellAt:i at:j]}.}
%martial: callback = fonction de rappel ? (plus francais)
En effet, ceci définit le comportement de rappel ou \callback à effectuer lorsque nous cliquons à la souris.
La méthode correspondante doit être aussi définie.

\begin{method}[toggleNeighboursOfCellAt:at:]{La méthode \callback}
LOGame>>>toggleNeighboursOfCellAt: i at: j
   (i > 1) ifTrue: [ (cells at: i - 1 at: j ) toggleState].
   (i < self cellsPerSide) ifTrue: [ (cells at: i + 1 at: j) toggleState].
   (j > 1) ifTrue: [ (cells at: i  at: j - 1) toggleState].
   (j < self cellsPerSide) ifTrue: [ (cells at: i at: j + 1) toggleState]
\end{method}
\cmindex{LOGame}{toggleNeighboursOfCellAt:at:}

\Tmthref{toggleNeighboursOfCellAt:at:} (traduisible par ``change les
voisins de la cellule\ldots'') change l'état des 4 cellules au nord, sud, ouest et est de la cellule (\ct{i}, \ct{j}). La seule complication est que le plateau de jeu est fini. Il faut donc s'assurer qu'une cellule voisine existe avant de changer son état.

\dothis{Placez cette méthode dans un nouveau protocole appelé
  \prot{game logic} (pour ``logique du jeu'') et créé
en \actclickant{} dans le panneau des protocoles.}
Pour déplacer cette méthode, vous devez simplement cliquer sur son nom
puis la glisser-déposer sur le nouveau protocole (voir \figref{dragMethod}).

\begin{figure}[htbp]
   \centering
   \ifluluelse
		{\includegraphics[width=\textwidth]{DragMethod} }
		{\includegraphics[scale=0.7]{DragMethod} }
   \caption{Faire un glisser-déposer de la méthode dans un protocole.\figlabel{dragMethod}}
\end{figure}

Afin de compléter le jeu Lights Out, nous avons besoin de définir encore deux méthodes dans la classe \ct{LOCell} pour gérer les événements souris.
\begin{method}[mouseAction:]{Un mutateur typique}
LOCell>>>mouseAction: aBlock
   ^ mouseAction := aBlock
\end{method}
\cmindex{LOCell}{mouseAction:}

La seule action de \tmthref{mouseAction:} consiste à donner comme
valeur à la variable \ct{mouseAction} celle de l'argument puis, à en
retourner la nouvelle valeur. Toute méthode qui \emph{change} la
valeur d'une variable d'instance de cette façon est appelée une
\emph{méthode d'accès en écriture} ou \emph{mutateur} (vous pourrez
trouver dans la littérature le terme anglais \emph{setter}); une
méthode qui \emph{retourne} la valeur courante d'une variable
d'instance est appelée une \emph{méthode d'accès en lecture} ou
\emph{accesseur} (le mot anglais équivalent est \emph{getter}).
%\seeindex{setter method}{accessor}
%\seeindex{getter method}{accessor}
\seeindex{méthode d'accès en lecture}{méthode d'accès}
\seeindex{méthode d'accès en écriture}{méthode d'accès}
\seeindex{accesseur}{méthode d'accès}
\seeindex{mutateur}{méthode d'accès}
\seeindex{méthode!accès}{méthode d'accès}
\seeindex{méthode!getter}{méthode d'accès}
\seeindex{méthode!setter}{méthode d'accès}

Si vous êtes habitués aux méthodes d'accès en lecture (\emph{getter})
et écriture (\emph{setter}) dans d'autres langages de programmation,
vous vous attendez à avoir deux méthodes nommées \ct{getMouseAction}
et \ct{setMouseAction}.
La convention en \st est différente.
Une méthode d'accès en lecture a toujours le même nom que la variable
correspondante et la méthode d'accès en écriture est nommée de la même
manière avec un ``\ct{:}'' à la fin; ici nous avons donc
\ct{mouseAction} et \ct{mouseAction:}.

%martial a serge (12/13/2007): il faudra discuter de cela:
%les accesseurs sont des methodes d'acces en lecture et non pas des
%equivalents de l'anglais 'accessors' d'apres toutes mes docs
%j'ai reformule 
%%Les méthodes d'accès en lecture et écriture sont appelés des méthodes \emphind{accesseurs} et par convention elles doivent être placées dans le protocole \protind{accessing}.
Une méthode d'accès (en lecture ou en écriture) est appelée
en anglais \emphind{accessor} et par convention, elle doit être
placée dans le protocole \protind{accessing}.
En \st, \emph{toutes} les variables d'instances sont privées à
l'objet qui les possède, ainsi la seule façon pour un autre objet de
lire ou de modifier ces variables en \st se fait au travers de
ces méthodes d'accès comme ici~\footnote{En fait, les variables
  d'instances peuvent être accédées également dans les sous-classes.}.

\dothis{Allez à la classe \ct{LOCell}, définissez \ct{LOCell>>>mouseAction:} et mettez-la dans le protocole \prot{accessing}.}

Finalement, vous avez besoin de définir la méthode \ct{mouseUp:}; elle
sera appelée automatiquement par l'infrastructure (ou \emph{framework})
graphique si le bouton de la souris est pressé lorsque le pointeur de
celle-ci est au-dessus d'une cellule sur l'écran.

\begin{method}[sbecellmouseup]{Un gestionnaire d'événement}
LOCell>>>mouseUp: anEvent
   mouseAction value
\end{method}
\cmindex{LOCell}{mouseUp:}

\dothis{Ajoutez la méthode \ct{LOCell>>>mouseUp:} 
%ajout
définissant l'action lorsque le bouton de la souris est relaché
puis, faites \menu{categorize automatically}.}
%\index{method!categorize}
\index{méthode!catégorisation}

Que fait cette méthode? Elle envoie le message \ct{value} à l'objet
stocké dans la variable d'instance \ct{mouseAction}. 
Rappelez-vous que dans la méthode \ct{LOGame>>>newCellAt: i at: j}
nous avons affecté le fragment de code qui suit à \ct{mouseAction}:

\ct{[self toggleNeighboursOfCellAt: i at: j ]} 

% martial: j'ai remplace 'fragment de code' ici par 'bloc' et j'ai
% ajoute la reference vers le chapitre syntax.tex car c'est plus clair
% pour le debutant qui a ce stade ne maitrise pas bien les elements
% fondamentaux de squeak 
\noindent
Envoyer le message \ct{value} provoque l'évaluation de ce bloc
(toujours entre crochets, voir \charef{syntax}) et, par voie de
conséquence, est responsable du changement d'état des cellules.

%=================================================================
\section{Essayons notre code}

Voilà, le jeu Lights Out est complet!

Si vous avez suivi toutes les étapes, vous devriez pouvoir jouer au jeu qui comprend 2 classes et 7 méthodes.

\dothis{Dans un espace de travail, tapez \ct{LOGame new openInWorld} et faites \menu{do it}.}

Le jeu devrait s'ouvrir et vous devriez pouvoir cliquer sur les cellules et vérifier si le jeu fonctionne.

Du moins en théorie\ldots{}
Lorsque vous cliquez sur une cellule une fenêtre de \emphind{notification} appelée la fenêtre \clsind{PreDebugWindow} devrait apparaître avec un message d'erreur!
Comme nous pouvons le voir sur \figref{lightsOutError}, elle dit \ct{MessageNotUnderstood: LOGame>>>toggleState}.

\begin{figure}[ht]
\ifluluelse
	{\centerline{\includegraphics[width=\textwidth]{Error}}}
	{\centerline{\includegraphics[scale=0.7]{Error}}}
\caption{Il y a une erreur dans notre jeu lorsqu'une cellule est sélectionnée!
  \figlabel{lightsOutError}}
\end{figure}

\noindent
Que se passe-t-il? Afin de le découvrir, utilisons l'un des outils les plus puissants de \st, le \ind{débogueur}.

\dothis{Cliquez sur le bouton \menu{debug} de la fenêtre de notification.}
Le débogueur nommé Debugger devrait apparaître.
Dans la partie supérieure de la fenêtre du débogueur, nous pouvons
voir la pile d'exécution, affichant toutes les méthodes actives; en
sélectionnant l'une d'entre elles, nous voyons dans le panneau du
milieu le code \st en cours d'exécution dans cette méthode, avec
la partie qui a déclenchée l'erreur en caractère gras.

\dothis{Cliquez sur la ligne nommée
\ct{LOGame>>>toggleNeighboursOfCellAt:at:} (près du haut).}
Le débogueur vous montrera le \ind{contexte d'exécution} à l'intérieur
de la méthode où l'erreur s'est déclenchée (voir \figref{debugToggle}).

\begin{figure}[ht]
\ifluluelse
	{\centerline {\includegraphics[width=\textwidth]{Debugger}}}
	{\centerline {\includegraphics[scale=0.7]{Debugger}}}
\caption{Le débogueur avec la méthode \ct{toggleNeighboursOfCell:at:} sélectionnée.
\figlabel{debugToggle}}
\end{figure}

Dans la partie inférieure du débogueur, il y a deux petites fenêtres
d'inspection. Sur la gauche, vous pouvez inspecter l'objet-receveur du 
message qui cause l'exécution de la méthode sélectionnée. Vous pouvez 
voir ici les valeurs des variables d'instance.
Sur la droite, vous pouvez inspecter l'objet qui représente la méthode 
en cours d'exécution. Il est possible d'examiner ici les valeurs des 
paramètres et les variables temporaires.

En utilisant le débogueur, vous pouvez exécuter du code pas à pas,
inspecter les objets dans les paramètres et les variables locales,
évaluer du code comme vous le faites dans le Workspace et, de manière
surprenante pour ceux qui sont déjà habitués à d'autres débogueurs, il
est possible de modifier le code en cours de débogage! 
Certains Smalltalkiens programment la plupart du temps dans le
débogueur, plutôt que dans le navigateur de classes.
%martial: Tue Dec 25 10:30:52 CET 2007 modif' car trop lourd
L'avantage est certain: la méthode que vous écrivez est telle
qu'elle sera exécutée \ie avec ses paramètres dans son contexte
actuel d'exécution.
%: l'avantage en est
%que vous voyez la méthode que vous écrivez telle qu'elle sera
%exécutée, avec de paramètres dans son contexte actuel d'exécution.

Dans notre cas, vous pouvez voir dans la première ligne du panneau du 
haut que le message \ct{toggleState} a été envoyé à une instance de \ct{LOGame}, 
alors qu'il était clairement destiné à une instance de \ct{LOCell}.
Le problème se situe vraisemblablement dans l'initialisation de la matrice \ct{cells}.
En parcourant le code de \cmind{LOGame}{initialize}, nous pouvons
voir que \ct{cells} est rempli avec les valeurs retournées par
\ct{newCellAt:at:}, mais lorsque nous regardons cette méthode, nous
constatons qu'il n'y a pas de valeur retournée ici!
Par défaut, une méthode retourne \ct{self}, ce qui dans le cas 
de \ct{newCellAt:at:} est effectivement une instance de \ct{LOGame}.
\index{méthode!renvoi de self}

\dothis{Fermez la fenêtre du débogueur.
Ajoutez l'expression ``\ct{^ c}'' à la fin de la méthode \ct{LOGame>>>newCellAt:at:} de telle sorte qu'elle retourne \ct{c}
% It should now look as shown in \mthref{newCellAt:at:nobug}.
(voir \tmthref{newCellAt:at:nobug}).}

% \needlines{6}
\begin{method}[newCellAt:at:nobug]{Corriger l'erreur}
LOGame>>>newCellAt: i at: j
    "!Crée! une cellule !à! la position (i,j) et l'ajoute dans ma !représentation! graphique !à! la position correcte. Retourne une nouvelle cellule"
   | c origin |
   c := LOCell new.
   origin := self innerBounds origin.
   self addMorph: c.
   c position: ((i - 1) * c width) @ ((j - 1) * c height) + origin.
   c mouseAction: [self toggleNeighboursOfCellAt: i at: j].
   ^ c
\end{method}
\cmindex{LOGame}{newCellAt:at:}
% le commentaire
%"Create a cell for position (i,j) and add it to my on-screen
%   representation at the appropriate screen position.  Answer the new cell"

\noindent
Rappelez-vous ce que nous avons vu dans \charef{quick}:
pour renvoyer une valeur d'une méthode en \st, nous utilisons 
 \ct{^}, que nous pouvons obtenir en tapant \verb|^|.
% \index{^@\verb|^|}
\index{^@{$\uparrow$}|see{renvoi}}
%%?\index{retour|see{renvoi}}

%martial: j'ai retourne les prepositions car autrement c'est trop lourd
Il est souvent possible de corriger le code directement dans la
fenêtre du débogueur et de poursuivre l'application en cliquant sur
\menu{Proceed}.
Dans notre cas, la chose la plus simple à faire est de fermer la
fenêtre du débogueur, détruire l'instance en cours d'exécution (avec
le \subind{Morphic}{halo} Morphic) et d'en créer une nouvelle, parce que le bug
ne se situe pas dans une méthode erronée mais dans l'initialisation de l'objet.

%Indeed, even in this case it would be possible to \menu{do} \ct{self initialize} and then \menu{Proceed} the \ct{toggleNeighboursOfCellAt:at:} method.
%\ab{Stéph, did you try this?  It seems to me that it ought to work, but when I tried it, it messed up my image.}
% ON : It messed me up too!  Better not propose this.

\dothis{Exécutez \ct{LOGame new openInWorld} de nouveau.}
Le jeu doit maintenant se dérouler sans problèmes
\ldots{} ou presque! S'il vous arrive de bouger la souris
  entre le moment où vous \clickz{} et le moment où vous relâchez le
  bouton de la souris, la cellule sur laquelle se trouve la souris
  sera aussi changée. Ceci résulte du comportement hérité de
  \ct{SimpleSwitchMorph}. Nous pouvons simplement corriger celà en
  surchargeant \ct{mouseMove:} pour lui dire de ne rien faire: % CHANGE

% \needlines{6}
\begin{method}[mouseMove:]{Surcharger les actions associées aux déplacements de la souris}
LOGame>>>mouseMove: anEvent
\end{method}

Et voilà! % CHANGE

%\sd{It would be good to have a word about the debugger buttons into, step.... Or to have a separate chapter, we would use the material I wrote for my turtle book, please check it.}
%\on{I think that is too much for this chapter. It will come soon enough.}

%=================================================================
\section{Sauvegarder et partager le code \st}
\seclabel{Monticello}

Maintenant que nous avons un jeu Lights Out fonctionnel, vous avez
probablement envie de le sauvegarder quelque part de telle sorte à
pouvoir le partager avec des amis. Bien sûr, vous pouvez sauvegarder
l'ensemble de votre image \pharo et montrer votre premier programme
en l'exécutant, mais vos amis ont probablement leur propre code dans
leurs images et ne veulent pas s'en passer pour utiliser votre image.
Nous avons donc besoin de pouvoir extraire le code source d'une image
\pharo afin que d'autres développeurs puissent le charger dans leurs images.

La façon la plus simple de le faire est d'effectuer une exportation ou
sortie-fichier (\emph{filing out}) de votre code. 
Le menu activé en \actclickant{} dans le panneau des paquetages vous permet de
générer un fichier correspondant au paquetage \scat{PBE-LightsOut} tout entier 
via l'option \menu{various \go{} file out}.
% CHANGE ATTENDRE REVOIR : devrait être des packages
Le fichier résultant est plus lisible par tout un chacun, même si son
contenu est plutôt destiné aux machines qu'aux hommes.
Vous pouvez envoyer par email ce fichier à vos amis et ils peuvent le
charger dans leurs propres images \pharo en utilisant le navigateur
de fichiers File List Browser.
\seeindex{sauvegarde du code}{catégorie}
\seeindex{catégorie!exportation de fichier}{fichier, exportation}
\seeindex{classe!exportation de fichier}{fichier, exportation}
\seeindex{méthode!exportation de fichier}{fichier, exportation}
% pour le sortie fichier utilise par Serge
\seeindex{sortie-fichier}{fichier, exportation}
\seeindex{fichier!filing-out}{fichier, exportation}
\index{fichier!exportation}

\dothis{%
\Actclickz{} sur le paquetage
  \scat{PBE-LightsOut} et choisissez \menu{various \go{} file out} pour exporter le contenu.}
Vous devriez trouver maintenant un fichier PBE-LightsOut.st dans le même
répertoire où votre image a été sauvegardée.
Jetez un coup d'\oe il à ce fichier avec un éditeur de texte.

\dothis{Ouvrez une nouvelle image \pharo et utilisez l'outil File
  Browser (\menu{Tools \go{} File Browser}) pour faire une importation
  de fichier via l'option de menu  \menu{file in} dans le fichier
  PBE-LightsOut.st. Vérifiez que le jeu fonctionne maintenant dans une
  nouvelle image.}


\seeindex{catégorie!importation de fichier}{fichier, importation}
\seeindex{classe!importation de fichier}{fichier, importation}
\seeindex{méthode!importation de fichier}{fichier, importation}
\seeindex{fichier!filing-in}{fichier, importation}
\index{fichier!importation}

\begin{figure}[ht]
\centerline {\includegraphics[width=\textwidth]{FileIn}}
\caption{Charger le code source dans \pharo.
\figlabel{filein}}
\end{figure}

\subsection{Les paquetages Monticello}
Bien que les exportations de fichiers soient une façon convenable de
faire des sauvegardes du code que vous avez écrit, elles font
maintenant partie du passé.
Tout comme la plupart des développeurs de projets libres
\emph{Open-Source} qui trouvent plus utile de maintenir leur code dans
des dépôts en utilisant \ind{CVS}~\footnote{\url{http://www.nongnu.org/cvs}}
ou \ind{Subversion}~\footnote{\url{http://subversion.tigris.org}}, les
programmeurs sur \pharo gèrent maintenant leur code au moyen de
paquetages \ind{Monticello} (dit, en anglais, \emph{packages}): 
ces paquetages sont représentés comme des fichiers dont le nom se
termine en \ct{.mcz}; ce sont en fait des fichiers compressés en
\emph{zip} qui contiennent le code complet de votre paquetage.

En utilisant le navigateur de paquetages Monticello, vous pouvez sauver les 
paquetages dans des dépôts en utilisant de nombreux types de serveurs, notamment 
des serveurs FTP et HTTP; vous pouvez également écrire vos paquetages dans un 
dépôt qui se trouve dans un répertoire de votre système local de fichiers.
Une copie de votre paquetage est toujours \emph{en cache} sur disque local 
dans le répertoire \emph{package-cache}. 
Monticello vous permet de sauver de multiples versions de votre programme, 
fusionner des versions, revenir à une ancienne version et voir les différences 
entre plusieurs versions.
En fait, nous retrouvons les mêmes types d'opérations auxquelles vous
pourriez être habitués en utilisant CVS ou Subversion pour
partager votre travail.
\seeindex{Package Browser}{Monticello}
\seeindex{Monticello Browser}{Monticello}
\seeindex{navigateur Monticello}{Monticello}

%martial: IMPORTANT -- semble etre retire de l'original
%% Une bonne astuce est de toujours développer dans le même
%% répertoire. De cette façon vous pouvez obtenir une copie de tout le
%% code que vous avez publié sur \sqsrc sur votre machine
%% locale. Vous pouvez  alors faire une sauvegarde et naviguer dans le code à votre convenance.

Vous pouvez également envoyer un fichier \ct{.mcz} par email.
Le destinataire devra le placer dans son répertoire \emph{package-cache}; il sera alors capable d'utiliser Monticello pour le parcourir et le charger. 
%(It is also possible to load it using the file list, but there is a difference between loading a \ct{.mcz} file using a file list and using Monticello \sd{check}.)

\dothis{Ouvrez le navigateur Monticello ou Monticello Browser depuis
  le menu \menu{World}.} % CHANGE
Dans la partie droite du navigateur (voir \figref{monticello1}), il y a une liste des dépôts Monticello incluant tous les dépôts dans lesquels du code a été chargé dans l'image que vous utilisez. 
%In addition to \sqsrc servers, Monticello repositories can live in a variety of other places, the simplest being a directory on your local disk.

\begin{figure}[hbt]
\ifluluelse
	{\centerline {\includegraphics[width=\textwidth]{MonticelloBrowser}}}
	{\centerline {\includegraphics[scale=0.7]{MonticelloBrowser}}}
\caption{Le navigateur Monticello.
\figlabel{monticello1}}
\end{figure}

En haut de la liste dans le navigateur Monticello, il y a un dépôt
dans un répertoire local appelé \emphind{package cache}: il s'agit
d'un répertoire-cache pour des copies de paquetages que vous avez
chargées ou publiées sur le réseau. Ce cache est vraiment utile car il
vous permet de garder votre historique local. Il vous permet également
de travailler là où vous n'avez pas d'accès Internet ou lorsque 
l'accès est si lent que vous n'avez pas envie de sauver fréquemment 
sur un dépôt distant.

\subsection{Sauvegarder et charger du code avec Monticello}
Dans la partie gauche du navigateur Monticello, il y a une liste de
paquetages dont vous avez une version chargée dans votre image; les
paquetages qui ont été modifiés depuis qu'ils ont été chargés sont
marqués d'une 
%étoile; martial:  http://www.iam.unibe.ch/pipermail/sbe-discussion/2007-December/000102.html
astérisque
(ils sont parfois appelés des \emphsubind{paquetage}{dirty package}{}\emph{s}). 
Si vous sélectionnez un paquetage, la liste des dépôts est restreinte à ceux qui 
contiennent une copie du paquetage sélectionné.
\seeindex{*}{paquetage, dirty package}
\seeindex{dirty package}{paquetage, dirty package}

% \aretirer{%
% Qu'est-ce qu'un paquetage? Pour l'instant, vous pouvez penser le
% paquetage comme un groupe de classes et de catégories de méthodes qui
% partagent le même préfixe. Comme nous avons mis tout le code du jeu
% Lights Out dans la catégorie de classes appelée \scat{PBE-LightsOut},
% nous pouvons le désigner comme le paquetage \ct{PBE-LightsOut}.} % bizarre avec la sous-section 'Des categories et des paquetages'

\dothis{Ajoutez le paquetage \ct{PBE-LightsOut} à votre navigateur Monticello en utilisant le bouton \button{+Package}.}

\subsection{\ind{\sqsrc}: un \ind{SourceForge} pour \pharo} 
Nous pensons que la meilleure façon de sauvegarder votre code et de le
partager est de créer un compte sur un serveur \sqsrc. \sqsrc est similaire à
\sourceforge~\footnote{\url{http://www.sourceforge.net}}: il s'agit d'un
\emph{portail web} à un serveur Monticello HTTP qui vous permet de gérer vos projets.
Il y a un serveur public \sqsrc à l'adresse
\url{http://www.squeaksource.com} et une copie du code concernant ce
livre est enregistrée sur
\url{http://www.squeaksource.com/PharoByExample.html}. Vous pouvez
consulter ce projet à l'aide d'un navigateur internet, mais il est
beaucoup plus productif de le faire depuis \pharo en utilisant
l'outil \emph{ad hoc}, le navigateur Monticello, qui vous permet de
gérer vos paquetages.

\dothis{Ouvrez un navigateur web à l'adresse \url{http://www.squeaksource.com}.
Ouvrer un compte et ensuite, créez un projet (\ie via ``register'')
pour le jeu Lights Out.}
\sqsrc va vous montrer l'information que vous devez utiliser
lorsque nous ajoutons un dépôt au moyen de Monticello.

Une fois que votre projet a été créé sur \sqsrc, vous devez indiquer au système \pharo de l'utiliser.

\dothis{Avec le paquetage \ct{PBE-LightsOut} sélectionné, cliquez sur le
  bouton \button{+Repository} dans le navigateur Monticello.}  Vous
verrez une liste des différents types de dépôts disponibles; pour
ajouter un dépôt \sqsrc, sélectionner le menu \menu{HTTP}. Une
boîte de dialogue vous permettra de rentrer les informations
nécessaires pour le serveur.
Vous devez copier le modèle ci-dessous pour identifier votre projet
\sqsrc, copiez-le dans Monticello en y ajoutant vos initiales
et votre mot de passe:

\needlines{5}
\begin{code}{}
MCHttpRepository 
    location: 'http://www.squeaksource.com/!\emph{VotreProjet}!'
    user: '!\emph{vosInitiales}!' 
    password: '!\emph{votreMotDePasse}!'
\end{code}   

\noindent
Si vous passez en paramètres des initiales et un mot de passe vide,
vous pouvez toujours charger le projet, mais vous ne serez pas
autorisé à le mettre à jour:

\needlines{5}
\begin{code}{}
MCHttpRepository 
    location: 'http://www.squeaksource.com/SqueakByExample'
    user: '' 
    password: ''
\end{code}   

%You can then load the code in your image by selecting the version you want. You can browse the code without loading it, using the \button{Browse} button.
Une fois que vous avez accepté ce modèle, un nouveau dépôt doit
apparaître dans la partie droite du navigateur Monticello.

\begin{figure}[hbt]
\ifluluelse
	{\centerline {\includegraphics[width=\textwidth]{BrowseRepository}}}
	{\centerline {\includegraphics[scale=0.7]{BrowseRepository}}}
\caption{Parcourir un dépôt Monticello.
\figlabel{monticello3}}
\end{figure}

\dothis{Cliquez sur le bouton \button{Save} pour faire une première
  sauvegarde du jeu Lights Out sur \sqsrc.}

Pour charger un paquetage dans votre image, vous devez d'abord
sélectionner une version particulière. Vous pouvez faire cela dans le
navigateur de dépôts \emph{Repository Browser}, que vous pouvez ouvrir
avec le bouton \button{Open} ou en \actclickant{}
% ajout - vf
pour choisir \menu{open repository} dans le menu contextuel. % CHANGE
Une fois que vous avez sélectionné une version, vous pouvez la
charger dans votre image.

\dothis{Ouvrez le dépôt \ct{PBE-LightsOut} que vous venez de sauvegarder.}

Monticello a beaucoup d'autres fonctionnalités qui seront discutées
plus en détail dans \charef{env}.
Vous pouvez également consulter la documentation en ligne de
Monticello à l'adresse \url{http://www.wiresong.ca/Monticello/}.

%=================================================================
\section{Résumé du chapitre}
Dans ce chapitre, nous avons vu comment créer des catégories, des classes
et des méthodes. Nous avons vu aussi comment utiliser le navigateur de
classes (Browser), l'inspecteur (Inspector), le débogueur (Debugger)
et le navigateur Monticello.

\begin{itemize}
  \item Les catégories sont des groupes de classes connexes.
%qui sont reliées entre-elles.
  \item Une nouvelle classe est créée en envoyant un message à sa super-classe.
  \item Les protocoles sont des groupes de méthodes apparentées.
  \item Une nouvelle méthode est créée ou modifiée en éditant la définition dans le navigateur de classes et en \emph{acceptant} les modifications.
  \item L'inspecteur offre une manière simple et générale pour inspecter et interagir avec des objets arbitraires.
  \item Le navigateur de classes détecte l'utilisation de méthodes et de variables non déclarées et propose d'éventuelles corrections.
  \item La méthode \ct{initialize} est automatiquement exécutée après
    la création d'un objet dans \pharo. Vous pouvez y mettre
    le code d'initialisation que vous voulez.
  \item Le débogueur est une interface de haut niveau pour inspecter et modifier l'état d'un programme en cours d'exécution.
  \item Vous pouvez partager le code source en sauvegardant une
    catégorie sous forme d'un fichier d'exportation.
  \item Une meilleure façon de partager le code consiste à faire
    appel à Monticello afin de gérer un dépôt externe défini, par
    exemple, comme un projet \sqsrc.
\end{itemize}

%=================================================================
\ifx\wholebook\relax\else\end{document}\fi
%=================================================================
%=================================================================
%%% Local Variables:
%%% coding: utf-8
%%% mode: latex
%%% TeX-master: t
%%% TeX-PDF-mode: t
%%% End:

%:Syntax
% $Author$
% $Date$
% $Revision: 14633$
% $Id$
% %=================================================================
% translated by Rene Mages squeak@rmages.com start: (Fri, 5 Oct 2007)
% relecture par Rene Mages et Martial Boniou: (Fri, 20 Dec 2007)
% relecture par Rene Mages : (Sun, 13 Jan 2008) de la version #14938
% adaptation pour PBE - martial - Thu Sep 10 22:45:08 CEST 2009 from
% $Id: Syntax.tex 28624 2009-08-27 10:59:05Z oscar $
%=================================================================
\ifx\wholebook\relax\else
% --------------------------------------------
% Lulu:
	\documentclass[a4paper,10pt,twoside]{book}
	\usepackage[
		papersize={6.13in,9.21in},
		hmargin={.75in,.75in},
		vmargin={.75in,1in},
		ignoreheadfoot
	]{geometry}
	\input{../common.tex}
	\pagestyle{headings}
	\setboolean{lulu}{true}
% --------------------------------------------
% A4:
%	\documentclass[a4paper,11pt,twoside]{book}
%	\input{../common.tex}
%	\usepackage{a4wide}
% --------------------------------------------
    \graphicspath{{figures/} {../figures/}}
	\begin{document}
	\renewcommand{\nnbb}[2]{} % Disable editorial comments
	\sloppy
\fi
%=================================================================
\chapter{Un r\'esum\'e de la syntaxe}
\chalabel{syntax}

\sd{We should add pragmas.}
\on{Please do so.}

% \sd{It would be good to add link to the chapter where the reader can learn about conditional, exceptions and loops.}
% \on{There are links already.}

\pharo, comme la plupart des dialectes modernes de \st, adopte une syntaxe proche de celle de \st-80.
La \ind{syntaxe} est con\c{c}ue de telle sorte que le texte d'un
programme lu \`{a} haute voix ressemble \`{a} de l'\emph{English pidgin} ou ``anglais simplifié'':

\begin{code}{}
(Smalltalk includes: Class) ifTrue: [ Transcript show: Class superclass ]
\end{code}

\noindent
La syntaxe de \pharo (\ie les expressions) est minimaliste; pour l'essentiel, con\c{c}ue uniquement pour \emph{envoyer des messages}.
%% et  \emph{des d\'{e}clarations de m\'{e}thodes}.
Les expressions sont construites \`{a} partir d'un nombre tr\`{e}s r\'{e}duit de primitives.
\st dispose seulement de 6 mots-cl\'{e}s et d'aucune syntaxe pour les structures de contr\^{o}le, ni pour les d\'{e}clarations de nouvelles classes.
En revanche, tout ou presque est r\'{e}alisable en envoyant des messages \`{a} des objets.
Par exemple, \`{a} la place de la structure de contr\^{o}le conditionnelle \emph{si-alors-sinon}, \st envoie des messages comme \ct{ifTrue:} \`{a} des \arevoir{objets bool\'eens}.
% rene propose la suppresion de :  de la classe \clsind{Boolean} (ne figure pas dans le texte original).
Les nouvelles \mbox{(sous-)classes} sont cr\'{e}\'{e}es en envoyant un message \`{a} leur super-classe.

%=================================================================
\section{Les \'{e}l\'{e}ments syntaxiques }

Les expressions sont compos\'{e}es des blocs constructeurs suivants:
\begin{enumerate}[label=(\small\itshape\roman{*}), ref=(\small\itshape\roman{*})]
\item six mots-cl\'{e}s r\'{e}serv\'{e}s ou \emph{pseudo-variables}:
\pvind{self}, \pvind{super}, \pvind{nil}, \pvind{true}, \pvind{false}, and \pvind{thisContext};
\item des expressions constantes pour des \emph{objets littéraux} comprenant les nombres, les caract\`{e}res, les chaînes de caract\`{e}res, les symboles et les tableaux;
\item des d\'{e}clarations de variables;
\item des affectations;
\item des \ind{bloc}{}s ou fermetures lexicales -- \emph{block closures} en anglais -- et;
\item des messages.
\end{enumerate}
\index{littéral!objet}
\seeindex{pseudo-variable}{variable, pseudo}
\seeindex{objet!littéral}{littéral, objet}

\begin{table}\centering
	\begin{tabular}{ll}
		\toprule
		Syntaxe & ce qu'elle repr\'{e}sente \\
		\midrule
		\lct{startPoint}			&	un nom de variable\\
		\lct{Transcript}			&	un nom de variable globale\\
		\lct{self}				&	une pseudo-variable \\
		\midrule
		\lct{1}				 	&	un entier decimal \\
		\lct{2r101}				&	un entier binaire \\
		\lct{1.5}					& un nombre flottant \\
		\lct{2.4e7}				&	une notation exponentielle \\
		\lct{\$a}					& le caract\`{e}re `a' \\
		\lct{'Bonjour'}				&	la cha\^{\i}ne ``Bonjour'' \\
		\lct{\#Bonjour}				&	le symbole \lct{\#Bonjour} \\
		\lct{\#(1 2 3)}			&	un tableau de litt\'{e}raux \\
		\lct{\{1. 2. 1+2\}}		&	un tableau dynamique \\
		\midrule
		\lct{"c'est mon commentaire"} 		&	un commentaire  \\
		\midrule
		\lct{| x y |}				&	une d\'eclaration de 2 variables \lct{x} et \lct{y}	\\
		\lct{x := 1}				&	affectation de 1 \`a \lct{x} \\
		\lct{[ x + y ]}			&	un bloc qui \'evalue \lct{x+y} \\
		\lct{<primitive: 1>}		&	une primitive de la VM~\footnote{Machine Virtuelle.} ou annotation\\
		\midrule
		\lct{3 factorial}			&	un message unaire \\
		\lct{3 + 4}					&	un message binaire \\
		\lct{2 raisedTo: 6 modulo: 10}		&	un message \`a mots-cl\'es \\
		\midrule
		\lct{$\uparrow$ true}
 			&	retourne la valeur \lct{true} pour vrai \\
		\lct{Transcript show: 'bonjour'. Transcript cr }		& un
        s\'eparateur d'expression (\lct{.})	\\ 
		\lct{Transcript show: 'bonjour'; cr}	& un message en cascade (\lct{;}) \\
		\bottomrule
	\end{tabular}
\caption{R\'esum\'e de la syntaxe de \pharo \tablabel{syntax}}
\end{table}


Dans \tabref{syntax}, nous pouvons voir des exemples divers d'\'{e}l\'{e}ments syntaxiques.
\begin{description}
\item[Les variables locales.] \ct{startPoint} est un nom de variable ou identifiant.
Par convention, les identifiants sont compos\'{e}s de mots au format
d'écriture \emph{casse chameau} (``\ind{camelCase}''): chaque mot
except\'{e} le premier d\'{e}bute par une lettre majuscule.
La premi\`{e}re lettre d'une variable d'instance, d'une m\'{e}thode ou d'un bloc argument ou d'une variable temporaire doit \^{e}tre en minuscule.
Ce qui indique au lecteur que la port\'{e}e de la variable est priv\'{e}e .

\item[Les variables partag\'{e}es.]	Les identifiants qui d\'{e}butent par
  une lettre majuscule sont des  variables
  \subind{variable}{globale}{s}, des  \subind{classe}{variable}{s} de
  classes, des dictionnaires de \subind{variable}{pool} ou des noms de classes.
\ct{Transcript} est une variable globale, une instance de la classe \ct{TranscriptStream}.
\seeindex{variable globale}{variable, globale}
\seeindex{dictionnaire de pool}{variable, pool}
\seeindex{variable!classe}{classe, variable}

\item[Le receveur.] \pvind{self} est un mot-cl\'{e} qui pointe vers l'objet sur lequel la m\'{e}thode courante s'ex\'{e}cute. Nous le nommons  ``le receveur'' car cet objet devra normalement re\c{c}evoir le message qui provoque l'ex\'{e}cution de la m\'{e}thode.
\self est appel\'{e} une ``\subind{variable}{pseudo-variable}'' puisque nous ne pouvons rien lui affecter.

\item[Les entiers.] En plus des entiers d\'{e}cimaux habituels comme
  \ct{42}, \pharo propose aussi une \ind{notation en base num\'{e}rique}
  ou \emph{radix}.
\ct{2r101} est \ct{101} en base 2 (\ie en binaire), qui est \'{e}gal \`{a} l'entier d\'{e}cimal 5.
\index{littéral}
\index{littéral!nombre}

\item[Les nombres flottants.] Ils peuvent \^{e}tre sp\'{e}cifi\'{e}s avec leur \ind{exposant} en base dix: \mbox{\ct{2.4e7}} est $2.4 \times 10^7$.
\index{nombres flottants}

\item[Les caract\`{e}res.] Un signe dollar d\'{e}finit un \subind{littéral}{caractère}: \ct{$a}\ignoredollar$ est le litt\'{e}ral pour `a'.
Des instances de caract\`{e}res non-imprimables peuvent \^{e}tre
obtenues en envoyant des messages ad hoc \`{a} la classe
\clsind{Character}, tel que \ct{Character space} \cmindex{Character class}{space} et \ct{Character tab}\cmindex{Character class}{tab}.
		
\item[Les chaînes de caract\`{e}res.] Les apostrophes sont utilis\'{e}es pour d\'{e}finir un  litt\'{e}ral \subind{littéral}{chaîne}.
Si vous d\'{e}sirez une chaine comportant une apostrophe, il suffira de doubler l'apostrophe, comme dans \ct{'aujourd''hui'}.

\item[Les symboles.] Ils ressemblent \`a des chaînes de caract\`{e}res, en ce sens qu'ils comportent une suite de caract\`{e}res.  
Mais contrairement \`{a} une chaîne, un \subind{littéral}{symbole} doit \^{e}tre globalement unique.
Il y a seulement un objet symbole \ct{#Bonjour} mais il peut y avoir plusieurs objets chaînes de caract\`{e}res ayant la valeur \ct{'Bonjour'}.
\seeindex{\#@{\textsf{\#}}}{littéral, symbole}
\seeindex{symbole!littéral}{littéral, symbole}

% martial (25 dec 2007): nous avons dit 'tableaux \`a la compilation'
% dans d'autres chapitres? 
\item[Les tableaux définis \`{a} la compilation.] Ils sont d\'{e}finis par \ct{#( )}, les objets litt\'{e}raux sont s\'{e}par\'{e}s par des espaces.
À l'int\'{e}rieur des parenth\`{e}ses, tout doit \^{e}tre constant durant la compilation.
Par exemple,  \ct{#(27 #(true false) abc)} est un
\subind{littéral}{tableau} litt\'{e}ral de trois \'{e}l\'{e}ments: l'entier \ct{27}, le tableau \`{a} la compilation contenant deux bool\'{e}ens et le symbole \ct{#abc}.
\seeindex{tableau!littéral}{littéral, tableau}

\item[Les tableaux définis \`{a} l'ex\'{e}cution.] Les accolades \ct|{ }|
  d\'{e}finissent un tableau (\subind{tableau}{dynamique}) \`{a} l'ex\'{e}cution.
Ses \'{e}l\'{e}ments sont des expressions s\'epar\'{e}es par des points.
Ainsi \ct|{ 1. 2. 1+2 }| d\'{e}finit un tableau dont les \'{e}l\'{e}ments sont 1, 2 et le r\'{e}sultat de l'\'{e}valuation de 1+2
(la notation entre accolades est \arelire{particuli\`{e}re \`{a} \pharo et à
\squeak}. % CHANGE
Dans d'autres \st{}s vous devez explicitement construire des tableaux dynamiques).

\item[Les commentaires.] Ils sont encadr\'{e}s par des guillemets.
\ct{"Bonjour le commentaire"} est un \ind{commentaire} et non une
chaîne; donc il est ignor\'{e} par le compilateur de \pharo.
Les commentaires peuvent se r\'{e}partir sur plusieurs lignes.
		
\item[Les d\'{e}finitions des variables locales.] Des barres
  verticales \ct{| |} limitent les
  \subind{variable}{déclaration}{}s d'une ou plusieurs variables
  locales dans une m\'{e}thode (ainsi que dans un bloc).
% \seeindex{\|@{\textsf{\|\|}}}{assignment}
% Can't seem to index or-bars! (special char for index macro)
\seeindex{déclaration de variable}{variable!déclaration}

\item[L'affectation.]	\ct{:=} affecte un objet \`{a} une variable.
%Quelquefois vous verrez \`{a} la place une $\leftarrow$ .
%Malheureusement, tant qu'elle ne sera pas un caract\`{e}re
%\textsc{ASCII}, elle apparaîtra sous la forme d'un signe souligné (en
%anglais, \emph{underscore} \`{a} moins que vous n'utilisiez une fonte 
%sp\'{e}ciale.
%Ainsi, \ct{x := 1} est identique \`{a} \ct{x _ 1} ou \ct{x UNDERSCORE
%1}. Il est préférable d'utiliser  \ct{:=} puisque les autres
%repr\'{e}sentations ont \'{e}t\'{e} déclarées comme obsolètes depuis
%la version 3.9 de \squeak. % CHANGE
\index{affectation}
\seeindex{:=@{\textsf{:=}}}{affectation}
\seeindex{\_@{\textsf{\_}}}{affectation}
%\seeindex{<-@{$\leftarrow$}}{affectation}

\item[Les blocs.] Des crochets \ct{[ ]} définissent un \ind{bloc},
  aussi connu sous le nom de \emph{block closure} ou fermeture lexicale, laquelle est un objet \`{a} part enti\`{e}re repr\'{e}sentant une fonction.
Comme nous le verrons, les blocs peuvent avoir des arguments et des variables locales.
\seeindex{[ ]@{\textsf{[ ]}}}{bloc}
\seeindex{block closure}{bloc}
\seeindex{fermeture lexicale}{bloc}

\item[Les primitives.]	\ct{<primitive: ...>} marque l'invocation
  d'une \ind{primitive} de la VM ou \ind{machine virtuelle}
(\ct{<primitive: 1>} est la primitive de \ct{SmallInteger>>>+}).
Tout code suivant la primitive est ex\'{e}cut\'{e} seulement si la
primitive \'{e}choue.
La m\^{e}me syntaxe est aussi employ\'{e}e pour des annotations de
m\'{e}thode. % CHANGE

\item[Les messages unaires.] Ce sont des simples mots (comme \ct{factorial}) envoy\'{e}s \`{a} un receveur (comme \ct{3}).
\index{message!unaire}
%\seeindex{message unaire}{message, unaire}

\item[Les messages binaires.] Ce sont des op\'{e}rateurs (comme \ct{+}) envoy\'{e}s \`{a} un receveur et ayant un seul argument. Dans \ct{3+4}, le receveur est \ct{3} et l'argument est \ct{4}.
\index{message!binaire}
%\seeindex{message binaire}{message, binaire}

\item[Les messages \`{a} mots-cl\'{e}s.] Ce sont des mots-cl\'{e}s multiples (comme \ct{raisedTo:modulo:}), chacun se terminant par un deux-points (:) et ayant un seul argument. 
Dans l'expression \ct{2 raisedTo: 6 modulo: 10}, le \emph{sélecteur de message} \ct{raisedTo:modulo:} prend les deux arguments \ct{6} et \ct{10}, chacun suivant le \lct{:}. Nous envoyons le message au receveur \ct{2}.
\index{message!sélecteur}
\index{message!à mots-clés}
%\seeindex{message de mot-cl\'{e}}{message, mot-cl\'{e}}

\item[Le retour d'une m\'{e}thode.] \ct{^} est employ\'{e} pour
  obtenir le \emphind{retour} ou \emph{renvoi} d'une m\'{e}thode.  Il
  vous faut taper \verb|^| pour obtenir le caract\`{e}re \ct{^}.
\md{\ct{^} donne toujours un retour de la m\'{e}thode, m\^{e}me s'il est utilis\'{e} dans un bloc, il donnera le retour de la m\'{e}thode inser\'{e}e dans le bloc.}

\item[Les séquences d'instructions.]	Un point (\ct{.}) est le
  \emphsubind{instruction}{séparateur} \emph{d'instructions}. Placer un point entre deux expressions les transforme en deux instructions ind\'{e}pendantes.	
\index{instruction!séquence}
\seeindex{instruction!séparateur}{expression, séparateur}
\seeindex{point}{expression, séparateur}
\seeindex{\ct{.}}{expression, séparateur}

\item[Les cascades.] un point virgule peut \^{e}tre utilis\'{e} pour
  envoyer une \emphind{cascade} de messages \`{a} un receveur
  unique. Dans \ct{Transcript show: 'bonjour'; cr}, nous envoyons
  d'abord le message à mots-cl\'{e}s \ct{show: 'bonjour'} au receveur  \ct{Transcript}, puis nous envoyons au m\^{e}me receveur le message unaire \ct{cr}.
	\seeindex{;}{cascade}

\end{description}

Les classes \ct{Number}, \ct{Character}, \ct{String} et \ct{Boolean} sont d\'{e}crites avec plus de d\'{e}tails dans \charef{basic}.
\on{Blocks are described in \charef{blocks}. (Control flow and Iterators).}

%=================================================================
\section{Les pseudo-variables}

Dans \st, il y a 6 mots-cl\'{e}s r\'{e}serv\'{e}s ou  \emph{pseudo-variables}:
\pvind{nil}, \pvind{true},  \pvind{false},  \pvind{self},
\pvind{super} et \pvind{thisContext}.
Ils sont appel\'{e}s \subind{variable}{pseudo-variable}{s} car ils sont pr\'{e}d\'{e}finis et ne peuvent pas \^{e}tre l'objet d'une affectation.
\ct{true}, \ct{false} et \ct{nil} sont des constantes tandis que les valeurs de \ct{self}, \ct{super} et de \ct{thisContext} varient de fa\c{c}on dynamique lorsque le code est ex\'{e}cut\'{e}. 

\ct{true} et \ct{false} sont les uniques instances des classes
\clsind{Boolean}: \clsind{True} et \clsind{False} (voir \charef{basic} 
pour plus de d\'{e}tails).

\pvind{self} se r\'{e}f\`{e}re toujours au receveur de la m\'{e}thode en cours d'ex\'{e}cution.
\ct{super} se r\'{e}f\`{e}re aussi au receveur de la m\'{e}thode en
cours, mais quand vous envoyez un message \`{a} \super, la recherche
de m\'{e}thode change en d\'{e}marrant de la super-classe relative \`{a} la classe contenant la m\'{e}thode qui utilise \ct{super}
(pour plus de d\'{e}tails, voyez \charef{model}).

\ct{nil} est l'objet non d\'{e}fini.
C'est l'unique instance de la classe \clsind{UndefinedObject}. 
Les variables d'instance, les variables de classe et les variables locales  sont initialis\'{e}es \`{a} \ct{nil}.

\ct{thisContext} est une pseudo-variable qui repr\'{e}sente la structure du sommet de la pile d'ex\'{e}cution.
En d'autres termes, il repr\'{e}sente le \clsind{MethodContext} ou le \clsind{BlockClosure} en cours d'ex\'{e}cution.
En temps normal, \ct{thisContext} ne doit pas intéresser la plupart
des programmeurs, mais il est essentiel pour impl\'{e}menter des
outils de d\'{e}veloppement tels que le débogueur et il est aussi utilis\'{e} pour g\'{e}rer exceptions et continuations.

%=================================================================
\section{Les envois de messages}

Il y a trois types de messages dans \pharo.
\begin{enumerate}
  \item Les messages \emph{unaires}: messages sans argument.
  \ct{1 factorial} envoie le message  \ct{factorial} \`{a} l'objet \ct{1}.
  \item Les messages \emph{binaires}: messages avec un seul argument.
  	\ct{1 + 2} envoie le message \ct{+} avec l'argument \ct{2} \`{a} l'objet \ct{1}.
  \item Les messages \`{a} \emph{mots-cl\'{e}s}: messages qui comportent un nombre arbitraire d'arguments.
  	\ct{2 raisedTo: 6 modulo: 10} envoie le message comprenant le s\'{e}lecteur 	\ct{raisedTo: modulo:} et les arguments \ct{6} et \ct{10} vers l'objet \ct{2}.
\end{enumerate}

Les s\'{e}lecteurs des messages unaires sont constitu\'{e}s de caract\`{e}res alphanum\'{e}riques et d\'{e}butent par une lettre minuscule. 
\index{message!unaire}

Les s\'{e}lecteurs des messages binaires sont constitu\'{e}s par un ou plusieurs caract\`{e}res de l'ensemble suivant:
\index{message!binaire}
\begin{code}{}
+ - / \ * ~ < > = @ % | & ! ? ,
\end{code}
\noindent
% [\~\!\@\%\&\*\-\+\=\\\|\?\/\>\<\,]
\on{Il semble que 3 caract\`{e}res ou plus fonctionnent bien, mais il n'est pas possible d'avoir plus d'un ``-'' dans un s\'{e}lecteur binaire. Sans doute \`{a} cause d'un conflit avec l'analyseur (parser) des nombres n\'{e}gatifs?}
\ab{Bon; $-$ est \'{e}trange .}

Les s\'{e}lecteurs des messages \`{a} mots-cl\'{e}s sont form\'{e}s d'une suite de mots-cl\'{e}s alphanum\'{e}riques qui commencent par une lettre minuscule et se terminent par \lct{:}.
\index{message!à mots-clés}

Les messages unaires ont la plus haute priorit\'{e}, puis viennent les messages binaires et, pour finir, les messages \`{a} mots-cl\'{e}s; ainsi:
\begin{code}{@TEST}
2 raisedTo: 1 + 3 factorial --> 128
\end{code}
D'abord nous envoyons \ct{factorial} \`{a} \ct{3}, puis nous envoyons \ct{+ 6} \`{a} \ct{1}, et pour finir, nous envoyons \ct{raisedTo: 7} \`{a} \ct{2}.  
Rappelons que nous utilisons la notation \lct{\emph{expression}}\ct{-->}\lct{\emph{result}} pour montrer le r\'{e}sultat de l'\'{e}valuation d'une expression. 

Priorit\'{e} mise \`{a} part, l'\'{e}valuation s'effectue strictement de la gauche vers la droite, donc: 
\begin{code}{@TEST}
1 + 2 * 3 --> 9
\end{code}
et non \ct{7}.
Les parenth\`{e}ses permettent de modifier l'ordre d'une \'{e}valuation:
\begin{code}{@TEST}
1 + (2 * 3) --> 7
\end{code}
Les envois de message peuvent \^{e}tre compos\'{e}s gr\^{a}ce \`{a} des points et des points-virgules. Une suite d'expressions s\'{e}par\'{e}es par des points provoque  l'\'{e}valuation de chaque expression dans la suite comme une \emphind{instruction}, une apr\`{e}s l'autre. 
%\index{s\'{e}parateur!instruction}
\index{expression!séparateur}

\begin{code}{}
Transcript cr.
Transcript show: 'Bonjour le monde'.
Transcript cr
\end{code}

\noindent
Ce code enverra \ct{cr} \`{a} l'objet \glbind{Transcript}, puis
enverra  \ct{show: 'Bonjour le monde'}, et enfin enverra un nouveau \ct{cr}.

Quand une succession de messages doit \^{e}tre envoy\'{e}e \`{a} un \emph{m\^{e}me} receveur, 
ou pour dire les choses plus succinctement en \emphind{cascade}, le receveur est sp\'{e}cifi\'{e} une seule fois et la suite des messages est s\'{e}par\'{e}e par des points-virgules:

\begin{code}{}
Transcript cr;
    show: 'Bonjour le monde';
    cr
\end{code}
Ce code a pr\'{e}cis\'{e}ment le m\^{e}me effet que celui de l'exemple pr\'{e}c\'{e}dent.


%=================================================================
\section{Syntaxe relative aux m\'{e}thodes}
Bien que les expressions peuvent \^{e}tre \'{e}valu\'{e}es n'importe
o\`{u} dans \pharo (par exemple, dans un espace de travail (Workspace),
dans un d\'{e}bogueur (Debugger) ou dans un navigateur de classes), 
les m\'{e}thodes sont en principe d\'{e}finies dans une fen\^{e}tre du
Browser ou du d\'{e}bogueur
% (Methods can also be filed in from an external medium, but this is not the usual way to program in \pharo.)
(\arelire{les m\'{e}thodes peuvent aussi \^{e}tre rentrées} % CHANGE - retrait
                                % de  (par \menu{file in})
depuis une source externe, mais ce n'est pas une fa\c{c}on habituelle de programmer en \pharo).

Les programmes sont d\'{e}velopp\'{e}s, une m\'{e}thode \`{a} la fois,
dans l'environnement d'une classe pr\'{e}cise (une classe est d\'{e}finie en envoyant un message \`{a} une classe existante, en demandant de cr\'{e}er une sous-classe, de sorte qu'il n'y ait pas de syntaxe sp\'{e}cifique pour cr\'{e}er une classe).

Voil\`{a} la m\'{e}thode \mthind{String}{lineCount} (pour compter le
nombre de lignes) dans la classe  \clsind{String}.
La convention habituelle consiste à se ref\'{e}rer aux m\'{e}thodes
comme suit: \ct{ClassName>>>methodName}; ainsi nous nommerons cette
m\'{e}thode \ct{String>>>lineCount}~\footnote{Le commentaire de la
  m\'ethode dit: 
``Retourne le nombre de lignes représentées par le receveur, dans
    lequel chaque cr ajoute une ligne''}.

\needlines{9}
\begin{method}[lineCount]{Compteur de lignes}
String>>>lineCount
   "Answer the number of lines represented by the receiver,
   where every cr adds one line."
   | cr count |
   cr := Character cr.
   count := 1  min: self size.
   self do:
      [:c | c == cr ifTrue: [count := count + 1]].
   ^ count
\end{method}

Sur le plan de la syntaxe, une m\'{e}thode comporte:
\begin{enumerate}
  \item la structure de la m\'{e}thode avec le nom (\ie \ct{lineCount}) et tous les arguments (aucun dans cet exemple);
  \item les commentaires (qui peuvent \^{e}tre plac\'{e}s n'importe
    o\`{u}, mais conventionnellement, un commentaire doit \^{e}tre plac\'{e} au d\'{e}but afin d'expliquer le but de la m\'{e}thode);
  \item les d\'{e}clarations des variables locales (\ie \ct{cr} et
    \ct{count}); 
  \item un nombre quelconque d'expressions separ\'{e}es par des points; dans notre exemple, il y en a quatre.
\end{enumerate}

L'\'{e}valuation de n'importe quelle expression pr\'{e}c\'{e}d\'{e}e
par un \ct{^} (saisi en tapant \verb|^|) provoquera l'arr\^{e}t de la m\'{e}thode \`{a} cet endroit, donnant en retour la valeur de cette expression.
Une m\'{e}thode qui se termine sans retourner explicitement une expression retournera de fa\c{c}on implicite \pvind{self}.
\index{retour!implicite}


Les arguments et les variables locales doivent toujours d\'{e}buter par une lettre minuscule.
Les noms d\'{e}butant par une majuscule sont r\'{e}servés aux variables globales.
Les noms des classes, comme par exemple \ct{Character}, sont tout
simplement des variables globales qui se réfèrent à l'objet repr\'{e}sentant cette classe.

%=================================================================
\section{La syntaxe des blocs}

Les blocs apportent un moyen de diff\'{e}rer l'\'{e}valuation d'une expression.
Un \ind{bloc} est essentiellement une fonction anonyme. Un bloc est \'{e}valu\'{e} en lui envoyant le message \mthind{BlockClosure}{value}.
Le bloc retourne la valeur de la derni\`{e}re expression de son corps,
\`{a} moins qu'il y ait un retour explicite (avec \ct{^}) auquel cas il ne retourne aucune valeur.
\seeindex{value}{BlockClosure}
% Serge : je ne comprends pas pourquoi cela ne retourne rien ...
% Martial : j'ai traduit tel quel et je ne comprends pas non plus

\begin{code}{@TEST}
[ 1 + 2 ] value --> 3
\end{code}

Les blocs peuvent prendre des param\`etres, chacun doit \^etre
d\'eclar\'e en le précédant d'un deux-points.
Une barre verticale s\'{e}pare les d\'{e}clarations des param\`{e}tres
et le corps du bloc.
Pour \'evaluer un bloc avec un param\`{e}tre, vous devez lui envoyer le message 
 \mthind{BlockClosure}{value:} avec un argument.
Un bloc \`{a} deux param\`etres doit recevoir  \mthind{BlockClosure}{value:value:}; et ainsi de suite, jusqu'\`a 4 arguments.

\begin{code}{@TEST}
[ :x | 1 + x ] value: 2 --> 3
[ :x :y | x + y ] value: 1 value: 2 --> 3
\end{code}

Si vous avez un bloc comportant plus de quatre param\`{e}tres, vous devez utiliser
\mthind{BlockClosure}{valueWithArguments:} et passer les arguments à
l'aide d'un tableau (un bloc comportant un grand nombre de param\`{e}tres étant souvent r\'{e}v\'{e}lateur d'un probl\`{e}me au niveau de sa conception).

Des blocs peuvent aussi d\'{e}clarer des variables locales, lesquelles seront entour\'{e}es par des barres verticales, tout comme des d\'{e}clarations de variables locales dans une m\'{e}thode.
Les variables locales sont d\'{e}clar\'{e}es apr\`{e}s les éventuels
arguments:
\index{variable!déclaration}

\begin{code}{@TEST}
[ :x :y | | z | z := x+ y. z ] value: 1 value: 2 --> 3
\end{code}

% Rene : la traduction de ces deux phrases est à controler (merci).
Les blocs sont en fait des \emph{fermetures} lexicales, puisqu'ils
peuvent faire r\'ef\'erence \`a des variables de leur environnement
imm\'ediat. Le bloc suivant fait r\'ef\'erence \`a la variable \ct{x} voisine:
%Les blocs sont en fait des \emph{fermetures} lexicales, d\`{e}s lors
%qu'ils peuvent se r\'{e}f\'{e}rer \`{a} des variables de
%l'environnement qui l'entoure.
%Le bloc suivant concerne la variable \ct{x} de son environnement englobant:
\newpage
\begin{code}{@TEST}
| x |
x := 1.
[ :y | x + y ] value: 2 --> 3
\end{code}

Les blocs sont des instances de la classe \clsind{BlockClosure}; ce
sont donc des objets, de sorte qu'ils peuvent \^{e}tre affect\'{e}s
\`{a} des variables et \^{e}tre pass\'{e}s comme arguments \`{a}
l'instar de tout autre objet.

% CHANGE - retrait de la note de Mise en garde sur squeak 3.9 et les
% fermetures lexicales

%\paragraph{Really important.} \^\ acts as an escaping mechanism. 
%Return expressions inside a nested block expression will terminate the enclosing method.
%In the example 

%\begin{script}[detect]{...} when the expression \ct{^\ x@y} is executed, the method \ct{detect:}
% escapes the current iteration and returns it. 

%TwoLevelSet>>detect: aBlock

%   firstLevel keysAndValuesDo: [ :x :v |
%      v do: [ :y | (aBlock value: x@y) ifTrue: [^x@y]]
%   ].
%   ^nil
%\end{script}


%=================================================================
\section{Conditions et it\'{e}rations}

\st n'offre aucune syntaxe sp\'{e}cifique pour les structures de contr\^{o}le.
Typiquement celles-ci sont obtenues par l'envoi de messages \`{a} des bool\'{e}ens, des nombres ou des collections, avec pour arguments des blocs.

Les clauses conditionnelles sont obtenues par l'envoi des messages
\mthind{Boolean}{ifTrue:}, \mthind{Boolean}{ifFalse:} ou
\mthind{Boolean}{ifTrue:ifFalse:} au r\'{e}sultat d'une expression
bool\'{e}enne. Pour plus de détails sur les booléens, lisez \charef{basic}.

\begin{code}{}
(17 * 13 > 220)
   ifTrue: [ 'plus grand' ]
   ifFalse: [ 'plus petit' ] --> 'plus grand'
\end{code}
% ON: Not a test.
% My regex approach cannot handle multi-line expressions :-(

Les boucles (ou it\'{e}rations) sont obtenues typiquement par l'envoi de messages \`{a} des blocs, des entiers ou des collections.
Comme la condition de sortie d'une boucle peut \^{e}tre \'{e}valu\'{e}e de fa\c{c}on r\'{e}p\'{e}titive, elle se pr\'{e}sentera sous la forme d'un bloc plut\^{o}t que de celle d'une valeur bool\'{e}enne.
Voici pr\'{e}cis\'{e}ment un exemple d'une boucle proc\'{e}durale:
\index{itération}
%\index{itération|voir{Collection, itération}}
\seeindex{Collection, itération}{itération}
%seealso
\seeindex{boucle}{itération}
\seeindex{énumération}{itération}
\index{clause conditionnelle}

\begin{code}{@TEST | n |}
n := 1.
[ n < 1000 ] whileTrue: [ n := n*2 ].
n --> 1024
\end{code}
\cmindex{BlockClosure}{whileTrue:}

\noindent
\mthind{BlockClosure}{whileFalse:} inverse la condition de sortie.

\begin{code}{@TEST | n |}
n := 1.
[ n > 1000 ] whileFalse: [ n := n*2 ].
n --> 1024
\end{code}

\noindent
\mthind{Integer}{timesRepeat:} offre un moyen simple pour impl\'{e}menter un nombre donn\'{e} d'it\'{e}rations:
\begin{code}{@TEST | n |}
n := 1.
10 timesRepeat: [ n := n*2 ].
n --> 1024
\end{code}
% mark
Nous pouvons aussi envoyer le message \mthind{Number}{to:do:} \`{a} un
nombre qui deviendra alors la valeur initiale d'un compteur de boucle.
Le premier argument est la borne sup\'{e}rieure; le second est un bloc qui prend la valeur courante du compteur de boucle comme argument:

\needlines{4}
\begin{code}{@TEST | n |}
n := 0.
1 to: 10 do: [ :counter | n := n + counter ].
n --> 55
\end{code}

\paragraph{It\'erateurs d'ordre sup\'erieur.}

Les collections comprennent un grand nombre de classes diff\'{e}rentes
dont beaucoup acceptent le m\^{e}me protocole.
Les messages les plus importants pour it\'{e}rer sur des collections 
sont 
\mthind{Collection}{do:}, \mthind{Collection}{collect:}, \mthind{Collection}{select:}, \mthind{Collection}{reject:}, \mthind{Collection}{detect:} ainsi que  \mthind{Collection}{inject:into:}.
Ces messages d\'{e}finissent des it\'{e}rateurs d'ordre sup\'erieur qui nous permettent d'\'{e}crire du code tr\`{e}s compact.

Une instance \clsind{Interval} (\ie un intervalle) est une collection qui d\'{e}finit un it\'{e}rateur sur une suite de nombres depuis un d\'{e}but jusqu'\`{a} une fin.
\ct{1 to: 10} repr\'{e}sente l'intervalle de 1 \`{a} 10.
Comme il s'agit d'une collection, nous pouvons lui envoyer le message \ct{do:}.
L'argument est un bloc qui est \'{e}valu\'{e} pour chaque \'{e}l\'{e}ment de la collection.

\begin{code}{@TEST | n |}
n := 0.
(1 to: 10) do: [ :element | n := n + element ].
n --> 55
\end{code}

\ct{collect:} construit une nouvelle collection de la m\^{e}me taille, en transformant chaque \'{e}l\'{e}ment.
\begin{code}{@TEST}
(1 to: 10) collect: [ :each | each * each ] --> #(1 4 9 16 25 36 49 64 81 100)
\end{code}

\ct{select:} et \ct{reject:} construisent des collections nouvelles, contenant un sous-ensemble d'\'{e}l\'{e}ments satisfaisant (ou non) la condition du bloc bool\'{e}en.
\ct{detect:} retourne le premier \'{e}l\'{e}ment satisfaisant la condition.
Ne perdez pas de vue que les chaînes sont aussi des collections, ainsi
vous pouvez it\'{e}rer aussi sur tous les caract\`{e}res.
%Martial: ajout par rapport a l'original (a completer et surement a
%changer de place):
La m\'ethode \mthind{Character}{isVowel} renvoie \ct{true} (\ie vrai)
lorsque le receveur-caract\`ere est une \label{def:isVowel}
voyelle~\footnote{Note du traducteur: les voyelles accentu\'ees ne sont
  pas consid\'er\'ees par d\'efaut comme des voyelles; \st-80 a le
  m\^eme d\'efaut que la plupart des langages de programmation n\'es
  dans la culture anglo-saxonne.}.
%note de martial: cette remarque ne concerne que moi. Elle pourrait
%etre enlevee si elle pose probleme; par contre il est bon de dire les
%limites des manipulations de string en ST. Je l'avais mise dans le
%chapitre Collections.tex

\begin{code}{@TEST}
'bonjour Squeak' select: [ :char | char isVowel ] --> 'oouuea'
'bonjour Squeak' reject: [ :char | char isVowel ] --> 'bnjr Sqk'
'bonjour Squeak' detect: [ :char | char isVowel ] --> $o 
\end{code}
%%%$

Finalement, vous devez garder \`{a} l'esprit que les collections
acceptent aussi l'\'equivalent de l'op\'erateur \emph{fold}
issu de la programmation fonctionnelle au travers de 
la m\'{e}thode \ct{inject:into:}.
Cela vous am\`{e}ne \`{a} g\'{e}n\'{e}rer un r\'{e}sultat cumulatif
utilisant une expression qui accepte une valeur initiale puis 
injecte chaque \'{e}l\'{e}ment de la collection.
Les sommes et les produits sont des exemples typiques.
\seeindex{fold}{\ct{Collection>>>inject:into}}

\begin{code}{@TEST}
(1 to: 10) inject: 0 into: [ :sum :each | sum + each ] --> 55
\end{code}

\noindent
Ce code est \'{e}quivalent \`{a} \ct{0+1+2+3+4+5+6+7+8+9+10}.

Pour plus de d\'{e}tails sur les collections et les flux de donn\'ees,
rendez-vous dans \charefs{collections}{streams}.

%=================================================================
\section{Primitives et Pragmas}

En \st, \mantra et tout se passe par l'envoi de messages.
N\'{e}anmoins, \`{a} certains niveaux, ce mod\`ele a ses limites;
%%points nous ``touchons le fond``.
le fonctionnement de certains objets ne peut \^{e}tre achev\'e qu'en
invoquant la \ind{machine virtuelle} et les \ind{primitive}{}s.

Par exemple, les comportements suivantes sont tous impl\'{e}ment\'{e}s 
sous la forme de primitives:
l'allocation de la m\'{e}moire (\mthind{Behavior}{new} et \mthind{Behavior}{new:}),
la manipulation de bits (\mthind{Integer}{bitAnd:},
\mthind{Integer}{bitOr:} et \mthind{Integer}{bitShift:}),
l'arithm\'{e}tique des pointeurs et des entiers (\ct{+}, \ct{-},  \ct{<},  \ct{>}, \ct{*}, \ct{/ }, \ct{=}, \ct{==}\ldots)
et l'acc\`{e}s des tableaux (\mthind{Object}{at:}, \mthind{Object}{at:put:}).
\seeindex{new@{\ct{new}}}{\ct{Behavior>>>new}}

Les primitives sont invoqu\'{e}es avec la syntaxe  \ct{<primitive: aNumber>} (aNumber \'etant un nombre).
Une m\'{e}thode qui invoque une telle primitive peut aussi embarquer
du code \st qui sera \'{e}valu\'{e}  \emph{seulement} en cas d'\'echec
de la primitive.

Examinons le code pour \cmind{SmallInteger}{+}.
Si la primitive \'{e}choue, l'expression \ct{super + aNumber} sera
\'{e}valu\'{e}e et retourn\'{e}e~\footnote{Le commentaire de la
  m\'ethode dit: ``Ajoute le receveur \`a l'argument et r\'epond le
  r\'esultat s'il s'agit d'un entier de classe SmallInteger. \'Echoue
  si l'argument ou le r\'esultat n'est pas un
  SmallInteger. Essentiel Aucune recherche. Voir la documentation de
  la classe Object: \emph{whatIsPrimitive} (qu'est-ce qu'une primitive).''}.

\needlines{6}
\begin{method}[primitive]{Une m\'ethode primitive}
+ aNumber 
  "Primitive. Add the receiver to the argument and answer with the result
  if it is a SmallInteger. Fail if the argument or the result is not a
  SmallInteger  Essential  No Lookup. See Object documentation whatIsAPrimitive."

  <primitive: 1>
  ^ super + aNumber
\end{method}

%The other use of primitives is to optimize some crucial methods. The idea is that the system could work 
%without the primitive but it would be slow. The following method shows that the method \ct{@} is calling the primitive 18. Here the point creation is clearly expressible in \st therefore the code after the primitive is just the creation of a point illustrating what the primitive is actually doing. Note that such a code will be never called except if the primitive would failed which is extremely rare.  

%\begin{method}[xxx]{xxx}
%Integer>>@ y 
%   "Primitive. Answer a Point whose x value is the receiver and whose y 
%   value is the argument. Optional. No Lookup. See Object documentation 
%   whatIsAPrimitive."

%   <primitive: 18>
%   ^Point x: self y: y
%\end{method}


\arelire{Dans \pharo,} la syntaxe avec <....> est aussi utilis\'{e}e
pour les annotations de m\'{e}thode que l'on appelle des
\emph{pragmas}. % REVOIR dans PBE, il y a toujours 'Since \pharo 3.9' 2009-09-10

\sd{we should give an example}\ab{Please do!  Is don't know about these.}

%=================================================================
\section{R\'{e}sum\'{e} du chapitre}

\begin{itemize}

\item	\pharo a (seulement) six mots r\'{e}serv\'{e}s aussi appel\'{e}s
  \textit{pseudo-variables}: \ct{true}, \ct{false}, \ct{nil},
  \ct{self}, \ct{super} et  \ct{thisContext}.

\item	Il y a cinq types d'objets litt\'{e}raux: les nombres (\ct{5},
  \ct{2.5}, \mbox{\ct{1.9e15},} \ct{2r111}), les caract\`{e}res
  (\ct{$a}), %$
les chaînes (\ct{'bonjour'}), les symboles (\ct{#bonjour}) et les tableaux (\ct{#('bonjour' #bonjour)})

\item	Les chaînes sont d\'{e}limit\'{e}es par des apostrophes et les commentaires par des guillemets. Pour obtenir une apostrophe dans une chaîne, il suffit de la doubler.

\item	Contrairement aux chaînes, les symboles sont par essence globalement uniques.

\item	Employez \ct{#( ... )} pour d\'{e}finir un tableau litt\'{e}ral.
		Employez \ct|{ ... }| pour d\'{e}finir un tableau dynamique.
		Sachez que
		\ct{#( 1 + 2 ) size --> 3}, mais que 
		\ct|{ 1 + 2 } size --> 1|

\item	Il y a trois types de messages:
		%martial: c'est plus propre en sous-itemizant:
  \begin{itemize}
\item \emph{unaire}: \eg \ct{1 asString}, \ct{Array new};
\item 		\emph{binaire}: \eg \ct{3 + 4}, \ct{'salut' , ' Squeak'};
\item 		\emph{\`{a} mots-cl\'{e}s}: \eg \ct{'salue' at: 5 put: $t}%$
      \end{itemize}
\item	Un envoi de messages \emph{en cascade}  est une suite de messages envoy\'{e}s \`{a} la m\^{e}me cible, tous s\'{e}par\'{e}s par des \ct{;}:
\ct{OrderedCollection new add: #albert; add: #einstein; size --> 2}

\item	Les variables locales sont d\'{e}clar\'{e}es à l'aide de barres verticales.
		\arelire{Employez} \ct{:=} \arelire{pour les affectations.} %; \ct{_} ou
   %     \ct{UNDERSCORE} marche aussi; tous deux sont abandonnées
   %     depuis la version 3.9 de \squeak. % CHANGE - martial
		\ct{|x| x:=1}

\item	Les expressions sont les messages envoy\'{e}s, les cascades et
  les affectations; parfois regroup\'{e}es avec des parenth\`{e}ses.
		\emph{Les instructions} sont des expressions s\'{e}par\'{e}es par des points.

\item	Les blocs ou fermetures lexicales sont des expressions limit\'{e}es par des crochets.
		Les blocs peuvent prendre des arguments et peuvent contenir
        des variables locales dites aussi \emph{variables temporaires}.
		Les expressions du bloc ne sont \'{e}valu\'{e}es que lorsque
        vous envoyez un message de la forme \ct{value...} avec le bon nombre d'arguments.\\
		\ct{[:x | x + 2] value: 4 --> 6}.

\item	Il n'y a pas de syntaxe particuli\`{e}re pour les structures
  de contr\^{o}le; ce ne sont que des messages qui, sous certaines conditions, \'{e}valuent des blocs.\\
		\ct{(\st includes: Class) ifTrue: [ Transcript show: Class superclass ]}

\end{itemize}

%=================================================================
\ifx\wholebook\relax\else
\end{document}\fi
%=================================================================
%%% Local Variables:
%%% coding: utf-8
%%% mode: latex
%%% TeX-master: t
%%% TeX-PDF-mode: t
%%% ispell-local-dictionary: "english"
%%% End:



%:Messages
% $Author$ traducteur Serge
% $Date$ 
% $Revision$
% relecture par Martial Boniou: Thu Dec 13 22:40:00 CET 2007
% Une instance de Pen est appele 'crayon' (et non plus, crayon ou
% stylo...)
% relecture par Rene Mages : Thu Dec 20 12:13:40 2007
% relecture par Rene Mages : Thr Jan 10 12:11:33 2008
% adaptation pour PBE - martial - Fri Sep 11 17:52:08 CEST 2009 from
% $Author: oscar $ % $Date: 2009-08-27 12:59:05 +0200 (Thu, 27 Aug 2009) $ % $Revision: 28624 $
%=================================================================
\ifx\wholebook\relax\else
% --------------------------------------------
% Lulu:
	\documentclass[a4paper,10pt,twoside]{book}
	\usepackage[
		papersize={6.13in,9.21in},
		hmargin={.75in,.75in},
		vmargin={.75in,1in},
		ignoreheadfoot
	]{geometry}
	\input{../common.tex}
	\pagestyle{headings}
	\setboolean{lulu}{true}
% --------------------------------------------
% A4:
%	\documentclass[a4paper,11pt,twoside]{book}
%	\input{../common.tex}
%	\usepackage{a4wide}
% --------------------------------------------
    \graphicspath{{figures/} {../figures/}}
	\begin{document}
	\renewcommand{\nnbb}[2]{} % Disable editorial comments
	\sloppy
\fi
%=================================================================
\chapter{Comprendre la syntaxe des messages}
\chalabel{understanding}

Bien que la syntaxe des messages \st soit extr\^emement simple, elle n'est pas habituelle et cela peut prendre un certain temps pour s'y habituer. Ce chapitre offre quelques conseils pour vous aider \`a mieux appr\'ehender la syntaxe sp\'eciale des envois de messages.
Si vous vous sentez en confiance avec la syntaxe, vous pouvez choisir de sauter ce chapitre ou bien d'y revenir un peu plus tard.

\on{I still feel this chapter contains too much repetition.
I would like to get feedback from students.}

%=============================================================
\section{Identifier les messages}

En \st, exception faite des \'el\'ements syntaxiques rencontr\'es dans
\charef{syntax} (\ct+:= ^ . ; # () {} [ : | ]+), tout se passe par envoi de messages.
Comme en \ind{C++}, vous pouvez d\'efinir vos op\'erateurs comme \ct{+} pour vos propres classes, mais tous les op\'erateurs ont la m\^eme pr\'ec\'edence.
De plus, il n'est pas possible de changer l'arit\'e d'une m\'ethode:
\ct{-} est toujours un message binaire, et il n'est pas possible
d'avoir une forme unaire avec une surcharge diff\'erente.

Avec \st, l'ordre dans lequel les messages sont envoy\'es est
d\'etermin\'e par le type de message. Il n'y a que trois formes de
messages: les messages \emphsubind{message}{unaire},
\emphsubind{message}{binaire} et \emphsubind{message}{à mots-clés}.
 Les messages unaires sont toujours envoy\'es en premier, puis
les messages binaires et enfin ceux \`a mots-cl\'es. Comme dans la
plupart des langages,  les \ind{parenthèses} peuvent \^etre utilis\'ees pour changer l'ordre d'\'evaluation. Ces r\`egles rendent le code \st aussi facile \`a lire que possible. La plupart du temps, il n'est pas n\'ecessaire de r\'efl\'echir \`a ces r\`egles.

Comme la plupart des calculs en \st sont effectu\'es par des envois de messages, identifier correctement les messages est crucial. La terminologie suivante va nous \^etre utile:

\begin{itemize}
  \item Un message est compos\'e d'un \emphsubind{message}{sélecteur} et d'arguments optionnels.
  \item Un message est envoy\'e au \emphsubind{message}{receveur}.
  \item La combinaison d'un message et de son receveur est appel\'e un \emphsubind{message}{envoi} \emph{de message}  comme il est montr\'e dans \figref{firstScriptMessage}.
\end{itemize}

\begin{figure}[htb]
\begin{minipage}{0.53\textwidth}
	\begin{center}
	\includegraphics[width=0.95\textwidth]{message}
	\caption{Deux messages compos\'es d'un receveur, d'un s\'electeur de m\'ethode et d'un ensemble d'arguments.\figlabel{firstScriptMessage}}\end{center}
\end{minipage}
\hfill
\begin{minipage}{0.43\textwidth}
	\begin{center}
	\ifluluelse
		{\includegraphics[width=0.9\textwidth]{uKeyUnOne}}
		{\includegraphics[width=6cm]{uKeyUnOne}}
	\caption{\ct{aMorph color: Color yellow} est compos\'e de deux expressions : \ct{Color yellow} et \ct{aMorph color: Color yellow}.\figlabel{ellipse}}
	\end{center}
\end{minipage}
\end{figure}


\important{Un message est toujours envoy\'e \`a un receveur qui peut \^etre un simple litt\'eral, une variable ou le r\'esultat de l'\'evaluation d'une autre expression.}

Nous vous proposons de vous faciliter la lecture au moyen d'une
notation graphique: nous soulignerons le receveur afin de vous aider
\`a l'identifier. Nous entourerons \'egalement chaque expression dans
une ellipse et num\'eroterons les expressions \`a partir de la
premi\`ere \`a \^etre \'evalu\'ee afin de voir l'ordre d'envoi des messages.

%\begin{figure}[!ht]
%\begin{center}
%\includegraphics[width=6cm]{uKeyUnOne}
%\end{center}
%\caption{\ct{aMorph color: Color yellow} is composed of two expressions: \ct{Color yellow} and \ct{aMorph color: Color yellow}.\figlabel{ellipse}}
%\end{figure}

\Figref{ellipse} repr\'esente deux envois de messages, \ct{Color yellow} et \ct{aMorph color: Color yellow}, de telle sorte qu'il y a
deux ellipses. L'expression \ct{Color yellow} est d'abord \'evalu\'e
en premier, ainsi son ellipse est num\'erot\'ee \`a \ct{1}. Il y a
deux receveurs: \ct{aMorph} qui re\c{c}oit le message \ct{color: ...}
et \ct{Color} qui re\c{c}oit le message \ct{yellow} 
%ajout
(\emph{yellow} correspond \`a la couleur jaune en anglais). 
Chacun des receveurs est soulign\'e.

Un receveur peut \^etre le premier \'el\'ement d'un message, comme
\ct{100} dans l'expression \ct{100 + 200} ou \ct{Color} 
%ajout
(la classe des couleurs)
dans l'expression \ct{Color yellow}. Un objet receveur peut
\'egalement \^etre le r\'esultat de l'\'evaluation d'autres
messages. Par exemple, dans le message \ct{Pen new go: 100}, le
receveur de ce message \ct{go: 100} 
%ajout
(litt\'eralement, aller \`a 100)
est l'objet retourn\'e par cette expression \ct{Pen new} 
%ajout
(soit une instance de \ct{Pen}, la classe crayon). Dans tous les cas,
le message est envoy\'e \`a un objet appel\'e le \emph{receveur} qui a
pu \^etre cr\'e\'e par un autre envoi de message.

\begin{table}\centering
	\begin{tabularx}{\linewidth}{llX}
		\toprule
		Expression & Type de messages & R\'esultat \\
		\midrule
		\lct{Color yellow}
			& unaire
			& Cr\'ee une couleur.
		\\
		\lct{aPen  go: 100}
			& \`a mots-cl\'es
			& Le crayon receveur se d\'eplace en avant de 100 pixels.
		\\
		\lct{100 + 20}
			& binaire
			& Le nombre 100 re\c{c}oit le message + avec le param\`etre 20.
		\\
		\lct{Browser open}
			& unaire
			& Ouvre un nouveau navigateur de classes.
		\\
		\lct{Pen new  go: 100}
			& unaire et \`a mots-cl\'es
			& Un crayon est cr\'e\'e puis d\'eplac\'e de 100 pixels.
		\\
		\lct{aPen go: 100 + 20}
			& \`a mots-cl\'es et binaire
			& Le crayon receveur se d\'eplace vers l'avant de 120 pixels.
		\\
		\bottomrule
	\end{tabularx}
	\caption{Exemples de messages}\tablabel{messageExamples}
\end{table}

\Tabref{messageExamples} montre diff\'erents exemples de messages.
Vous devez remarquer que tous les messages n'ont pas obligatoirement
d'arguments. Un message unaire comme \ct{open} (pour ouvrir) ne n\'ecessite pas d'arguments. Les messages \`a mots-cl\'es simples ou les messages binaires comme \ct{go: 100} et \ct{+ 20} ont chacun un argument. 
Il y a aussi des messages simples et des messages
compos\'es. \ct{Color yellow} et \ct{100 + 20} sont simples: un
message est envoy\'e \`a un objet, tandis que l'expression \ct{aPen go: 100 + 20} est compos\'ee de deux messages: \ct{+ 20} est
envoy\'e \`a \ct{100} et \ct{go:} est envoy\'e \`a \ct{aPen} avec pour
argument le r\'esultat du premier message.
Un receveur peut \^etre une expression qui peut retourner un
objet. Dans \ct{Pen new go: 100}, le message \ct{go: 100} est envoy\'e
\`a l'objet qui r\'esulte de l'\'evaluation de l'expression \ct{Pen new}.

% ON: An enumerated list here is overkill!
%=============================================================
\section{Trois sortes de messages}

\st d\'efinit quelques r\`egles simples pour d\'eterminer l'ordre dans lequel les messages sont envoy\'es. Ces r\`egles sont bas\'ees sur la distinction \'etablie entre les 3 formes d'envoi de messages: 
\begin{itemize}
\item \emph{Les messages unaires} sont des messages qui sont envoy\'es
  \`a un objet sans autre information. Par exemple dans \ct{3 factorial}, \ct{factorial} (pour factorielle) est un message unaire. 
\item  \emph{Les messages binaires} sont des messages form\'es avec
  des op\'erateurs (souvent arithm\'etiques). Ils sont binaires car
  ils ne concernent que deux objets: le receveur et l'objet
  argument. Par exemple, dans \ct{10 + 20}, \ct{+} est un message
  binaire qui est envoy\'e au receveur \ct{10} avec l'argument \ct{20}. 
\item  \emph{Les messages \`a mots-cl\'es} sont des messages form\'es avec plusieurs mots-cl\'es, chacun d'entre eux se finissant par deux points (\ct{:}) et prenant un param\`etre.
Par exemple, dans \ct{anArray at: 1 put: 10}, le mot-cl\'e \ct{at:}
prend un argument \ct{1} et le mot-cl\'e \ct{put:} prend l'argument \ct{10}.
\end{itemize}

%-------------------------------------------------------------
\subsection{Messages unaires}
Les messages unaires sont des messages qui ne n\'ecessitent aucun
argument. Ils suivent le mod\`ele syntaxique suivant: \ct{receveur nomMessage}. Le s\'electeur est constitu\'e d'une s\'erie de
caract\`eres ne contenant pas de deux points (\ct{:}) (\eg
\ct{factorial}, \ct{open}, \ct{class}).
\needlines{4}
\begin{code}{@TEST}
89 sin           --> 0.860069405812453
3 sqrt           --> 1.732050807568877
Float pi         --> 3.141592653589793
'blop' size     --> 4
true not        --> false
Object class --> Object class  "La classe de Object est Object class (BANG)"
\end{code}
% ON: I changed the examples to things we can test

\important{Les messages unaires sont des messages qui ne n\'ecessitent pas d'argument.\\
Ils suivent le moule syntaxique: \lct{receveur \textbf{s\'electeur}}}

%-------------------------------------------------------------
\subsection{Messages binaires} 
Les messages binaires sont des messages qui n\'ecessitent exactement un argument \emph{et} dont le s\'electeur consiste en une s\'equence de un ou plusieurs caract\`eres de l'ensemble: \ct{+}, \ct{-}, \ct{*}, \ct{/}, \ct{&}, \ct{=}, \ct{>}, \ct{|}, \ct{<}, \ct{~}, et \ct{@}. Notez que \ct{--} n'est pas autoris\'e.

\begin{code}{@TEST}
100@100      --> 100@100  "!cr\'ee! un objet Point"
3 + 4              --> 7
10 - 1            --> 9
4 <= 3            --> false
(4/3) * 3 = 4   --> true  "!l'\'egalit\'e! est juste un message binaire et les fractions sont exactes"
(3/4) == (3/4) --> false  "!deux fractions \'egales ne sont pas le m\^eme objet!"
\end{code}

\important{Les messages binaires sont des messages qui n\'ecessitent exactement un argument \emph{et} dont le s\'electeur est compos\'e d'une s\'equence de caract\`eres parmi : \ct{+}, \ct{-}, \ct{*}, \ct{/}, \ct{\&}, \ct{=}, \ct{>}, \ct{|}, \ct{<}, \ct{\~}, et \ct{@}. \ct{--} n'est pas possible.\\
Ils suivent le moule syntaxique: \lct{receveur \textbf{s\'electeur} argument}}

%-------------------------------------------------------------
\subsection{Messages \`a mots-cl\'es}

Les messages \`a mots-cl\'es sont des messages qui n\'ecessitent un ou plusieurs arguments et dont le s\'electeur consiste en un ou plusieurs mots-cl\'es se finissant par deux points \ct{:}.  Les messages \`a mots-cl\'es suivent le moule syntaxique: 
\lct{receveur \textbf{selecteurMotUn:} argument\-Un \textbf{motDeux:} argumentDeux}

Chaque mot-cl\'e utilise un argument. Ainsi \ct{r:g:b:} est une
m\'ethode avec 3 arguments, \ct{playFileNamed:} et \ct{at:} sont des
m\'ethodes avec un argument, et \ct{at:put:} est une m\'ethode avec
deux arguments. Pour cr\'eer une instance de la classe \ct{Color} on
peut utiliser la m\'ethode \ct{r:g:b:} comme dans \ct{Color r: 1 g: 0 b: 0} cr\'eant ainsi la couleur rouge. Notez que les deux points ne font pas partie du s\'electeur.

\important{En \ind{Java} ou \ind{C++}, l'invocation de m\'ethode \st \ct{Color r: 1 g: 0 b: 0} serait \'ecrite \ct{Color.rgb(1,0,0)}.}

\begin{code}{@TEST | nums |}
1 to: 10                        --> (1 to: 10)  "!cr\'eation! d'un intervalle"
Color r: 1 g: 0 b: 0       --> Color red  "!cr\'eation! d'une nouvelle
couleur (rouge)"
12 between: 8 and: 15 --> true

nums := Array newFrom: (1 to: 5).
nums at: 1 put: 6.
nums --> #(6 2 3 4 5)
\end{code}
% ON: Changed to real examples that we can test

\important{Les messages bas\'es sur les mots-cl\'es sont des messages qui n\'ecessitent un ou plusieurs arguments. Leurs s\'electeurs consistent en un ou plusieurs mots-cl\'es chacun se terminant par deux points (\ct{\:}). Ils suivent le moule syntaxique:\\
\lct{receveur \textbf{selecteurMotUn:} argumentUn \textbf{motDeux:} argumentDeux}}

%=============================================================
\section{Composition de messages}
Les trois formes d'envoi de messages ont chacune des priorit\'es diff\'erentes, ce qui permet de les composer de mani\`ere \'el\'egante.

\begin{enumerate}
\item Les messages unaires sont envoy\'es en premier, puis les messages binaires et enfin les messages \`a mots-cl\'es.
\item Les messages entre \ind{parenthèses} sont envoy\'es avant tout autre type de messages. 
\item Les messages de m\^eme type sont envoy\'es de gauche \`a droite. 
\end{enumerate}
\index{message!ordre d'évaluation}

Ces r\`egles ont un ordre de lecture tr\`es naturel. Maintenant si
vous voulez \^etre s\^ur que vos messages sont envoy\'es dans l'ordre
que vous souhaitez, vous pouvez toujours mettre des parenth\`eses
suppl\'ementaires comme dans \figref{uKeyUn}. Dans cet exemple, le
message \ct{yellow} est un message unaire et le message \ct{color:}
est un message \`a mots-cl\'es; ainsi l'expression \ct{Color yellow}
est envoy\'e en premier. N\'eanmoins comme les expressions entre
parenth\`eses sont envoy\'ees en premier, mettre des parenth\`eses
(normalement inutiles) autour de \ct{Color yellow} permet d'accentuer
le fait qu'elle
%l'expression
doit \^etre envoy\'ee en premier. Le reste de cette section illustre
chacun de ces diff\'erents points.

\begin{figure}[ht]
\ifluluelse
	{\centerline{\includegraphics[width=0.9\textwidth]{uKeyUn}} }
	{\centerline{\includegraphics[width=10cm]{uKeyUn}} }
\caption{Les messages unaires sont envoy\'es en premier; donc ici le
  premier message est \ct{Color yellow}. Il retourne un objet de
  couleur jaune qui est pass\'e comme argument du message \ct{aPen color:}.\figlabel{uKeyUn}}
\end{figure}

%---------------------------------------------------------
\subsection*{Unaire > Binaire > Mots-cl\'es}
Les messages unaires sont d'abord envoy\'es, puis les messages
binaires et enfin les messages \`a mots-cl\'es. Nous pouvons
\'egalement dire que les messages unaires ont une priorit\'e plus
importante que les autres types de messages.

\important{\textbf{R\`egle  une.} Les messages unaires sont envoy\'es en premier, puis les messages binaires et finalement les messages \`a mots-cl\'es.\\
\centerline{\lct{Unaire > Binaire > Mots-cl\'es}}
}

Comme ces exemples suivants le montrent, les r\`egles de syntaxe de
\st permettent d'assurer une certaine lisibilit\'e des expressions:
\begin{code}{@TEST}
1000 factorial / 999 factorial --> 1000
2 raisedTo: 1 + 3 factorial     --> 128
\end{code}
\noindent

Malheureusement, les r\`egles sont un peu trop simplistes pour les
expressions arithm\'etiques. D\`es lors, des parenth\`eses doivent
\^etre introduites chaque fois que l'on veut imposer un ordre de
priorit\'e entre deux op\'erateurs binaires:
\begin{code}{@TEST}
1 + 2 * 3   --> 9
1 + (2 * 3) --> 7
\end{code}

L'exemple suivant qui est un peu plus complexe (!) est l'illustration que m\^eme des expressions \st compliqu\'ees peuvent \^etre lues de mani\`ere assez naturelle: 
\begin{code}{@TEST}
[:aClass | aClass methodDict keys select: [:aMethod | (aClass>>aMethod) isAbstract ]] value: Boolean --> an IdentitySet(#or: #| #and: #& #ifTrue: #ifTrue:ifFalse: #ifFalse: #not #ifFalse:ifTrue:)
\end{code}
% note de martial: j'ai ajoute entre parentheses le nom des messages
% pour plus de clarte
\noindent
Ici nous voulons savoir quelles m\'ethodes de la classe \ct{Boolean}
(classe des bool\'eens) sont abstraites.
Nous interrogeons la classe argument \ct{aClass} pour r\'ecup\'erer
les cl\'es (via le message unaire \ct{keys}) de son dictionnaire de
m\'ethodes (via le message unaire \ct{methodDict}), puis nous en
s\'electionnons (via le message \`a mots-cl\'es \ct{select:}) les
m\'ethodes de la classe qui sont abstraites.
Ensuite nous lions (par \ct{value:}) l'argument \ct{aClass} \`a la
valeur concr\`ete \ct{Boolean}.
Nous avons besoin des parenth\`eses uniquement pour le message binaire
\ct{>>}, qui s\'electionne une m\'ethode d'une classe, avant d'envoyer
le message unaire \mbox{\ct{isAbstract}} \`a cette m\'ethode. Le
r\'esultat (sous la forme d'un ensemble de classe \ct{IdentifySet})
nous montre quelles m\'ethodes doivent \^etre impl\'ement\'ees par les
sous-classes concr\`etes de \ct{Boolean}: \ct{True} et \ct{False}.


\paragraph{Exemple.}
Dans le message \ct{aPen color: Color yellow}, il y a un message \emph{unaire} \ct{yellow} envoy\'e \`a la classe \ct{Color} et un message \`a \emph{mots-cl\'es} \ct{color:} envoy\'e \`a \ct{aPen}. Les messages unaires sont d'abord envoy\'es, de telle sorte que l'expression \ct{Color yellow} soit d'abord ex\'ecut\'ee (1). Celle-ci retourne un objet couleur qui est pass\'e en argument du message \ct{aPen color: aColor} (2) comme indiqu\'e dans l'\egref{decColor}.
\Figref{uKeyUn} montre graphiquement comment les messages sont envoy\'es.

\needlines{5}
\begin{example}[decColor]{D\'ecomposition de l'\'evaluation de \ct{aPen color: Color yellow}}{}
        aPen color: Color yellow
(1)                       Color yellow        "message unaire !envoy\'e! en premier"
                        --> aColor
(2)   aPen color: aColor                 "puis le message !\`a mots-cl\'es!"
\end{example}

\paragraph{Exemple.} Dans le message \ct{aPen go: 100 + 20}, il y a le message \emph{binaire} \ct{+ 20} et un message \`a \emph{mots-cl\'es} \ct{go:}. Les messages binaires sont d'abord envoy\'es avant les messages \`a mots-cl\'es, ainsi \ct{100 + 20} est envoy\'e en premier (1): le message \ct{+ 20} est envoy\'e \`a l'objet \ct{100} et retourne le nombre \ct{120}. Ensuite le message \ct{aPen go: 120} est envoy\'e avec comme argument \ct{120} (2).
L'\egref{decGo} nous montre comment l'expression est \'evalu\'e. 

\begin{example}[decGo]{D\'ecomposition de \ct{aPen go: 100 + 20}}{}
      aPen go: 100 + 20   
(1)                 100 + 20           "le message binaire en premier"
                   -->   120
(2)  aPen go: 120                   "puis le message !\`a mots-cl\'es!"
\end{example}

\begin{figure}[htb]
\begin{minipage}{0.48\textwidth}
	\ifluluelse
		{\centerline{\includegraphics[width=0.9\textwidth]{uKeyBin}}}
		{\centerline{\includegraphics[width=6cm]{uKeyBin}}}
	\caption{Les messages unaires sont envoy\'es en premier, ainsi
      \ct{Color yellow} est d'abord envoy\'e. Il retourne un objet de
      couleur jaune qui est pass\'e en argument du message \ct{aPen color:}.\figlabel{uKeyBin}}
\end{minipage}
\hfill
\begin{minipage}{0.48\textwidth}
	\begin{center}
	\ifluluelse
		{\includegraphics[width=0.9\textwidth]{uunKeyBin}}
		{\includegraphics[width=6cm]{uunKeyBin}}
\caption{D\'ecomposition de \ct{Pen new go: 100 + 20}.\figlabel{unKeyBin}}
\end{center}
\end{minipage}
\end{figure}

\paragraph{Exemple.} Comme exercice, nous vous laissons d\'ecomposer
l'\'evaluation du message \ct{Pen new go: 100 + 20} qui est compos\'e
d'un message unaire, d'un message \`a mots-cl\'es et d'un message
binaire (voir \figref{unKeyBin}).

%-------------------------------------------------------------
\subsection{Les parenth\`eses en premier}

\important{\textbf{R\`egle deux.} Les messages parenth\'es\'es sont envoy\'es avant tout autre message.\\
\centerline{\lct{(Msg) > Unaire > Binaire > Mots-cl\'es}}}

\begin{code}{@TEST}
1.5 tan rounded asString = (((1.5 tan) rounded) asString) --> true    "les !parenth\`eses! sont !n\'ecessaires! ici"
3 + 4 factorial   --> 27    "(et pas 5040)"
(3 + 4) factorial --> 5040
\end{code}

Ici nous avons besoin des \ind{parenthèses} pour forcer l'envoi de \ct{lowMajorScaleOn:} avant \ct{play}.
\begin{code}{}
(FMSound lowMajorScaleOn: FMSound clarinet) play 
"(1) envoie le message clarinet !\`a! la classe FMSound pour !cr\'eer! le son de clarinette.
 (2) envoie le son !\`a! FMSound comme argument du message !\`a! !mots-cl\'es! lowMajorScaleOn:.
 (3) joue le son !r\'esultant!."
\end{code}

% ON: This has nothing to do with parentheses!
%RecordingControlsMorph new openInWorld
%"An instance of the digitizer is created then visualized. If your microphone is plugged in try a sampleBANG"

% ON: This link is broken, and the result does not understand display!
%(HTTPSocket httpShowGif:
%   'www.altavista.digital.com/av/pix/default/av-adv.gif') display

\paragraph{Exemple.}
Le message \ct{(65@325 extent: 134@100) center} retourne le centre
du rectangle dont le point sup\'erieur gauche est $(65, 325)$ et dont
la taille est $134{\times}100$. L'\egref{decExtent} montre comment le
message est d\'ecompos\'e et envoy\'e. Le message entre parenth\`eses
est d'abord envoy\'e: il contient deux messages binaires \ct{65@325}
et \ct{134@100} qui sont d'abord envoy\'es et qui retournent des
points, et un message \`a mots-cl\'es \ct{extent:} qui est ensuite
envoy\'e et qui retourne un rectangle. Finalement le message unaire
\ct{center} est envoy\'e au rectangle et le point central est retourn\'e.

\'Evaluer ce message sans parenth\`eses d\'eclencherait une erreur car
l'objet \ct{100} ne comprend pas le message \ct{center}.

\needlines{9} % CHANGE REVOIR
\begin{example}[decExtent]{Exemple avec des parenth\`eses.}{}
      (65 @ 325 extent: 134 @ 100) center
(1)   65@325                                                    "binaire"
    --> aPoint
(2)                                134@100                     "binaire"
                                 --> anotherPoint
(3)   aPoint extent: anotherPoint                       "!\`a mots-cl\'es!"
      --> aRectangle
(4)   aRectangle center                                     "unaire"
      --> 132@375
\end{example}

\subsection{De gauche \`a droite}
Maintenant nous savons comment les messages de diff\'erentes natures
ou priorit\'es sont trait\'es. Il reste une question \`a traiter:
comment les messages de m\^eme priorit\'e sont envoy\'es? Ils sont
envoy\'es de gauche \`a droite. Notez que vous avez d\'ej\`a vu ce
comportement dans l'\egref{decExtent} dans lequel les deux messages de
cr\'eation de points (\ct{@}) sont envoy\'es en premier.

\important{{\textbf{R\`egle trois.} Lorsque les messages sont de m\^eme nature, l'ordre d'\'evaluation est de gauche \`a droite.}}

%\begin{figure}
%\centerline{\includegraphics[width=8cm]{ucompoUn}} 
%\caption{The message \ct{Pen new east} is composed of two unary messages. Therefore the leftmost one, \ct{new},  is sent and it returns a new robot to which the second message \ct{east} is sent. \figlabel{compoUn}}
%\end{figure}

\paragraph{Exemple.} Dans l'expression \ct{Pen new down}, tous les
messages sont des messages unaires, donc celui qui est le plus \`a
gauche \ct{Pen new} est envoy\'e en premier. Il retourne un nouveau
crayon auquel le deuxi\`eme message \ct{down} 
%ajout
(pour poser la pointe du crayon et dessiner)
est envoy\'e comme il est montr\'e dans \figref{unaryMessages}.

\begin{figure}
	\centering
	\includegraphics[width=8cm]{ucompoUn}
	\caption{D\'ecomposition de \ct{Pen new down}.\figlabel{unaryMessages}}
\end{figure}

%-------------------------------------------------------------
%\subsection{Inconsistances arithm\'etiques}
% note de martial: j'ai fait des recherches; c'est plus correct
% qu'inconsistence; le vrai terme est irrationnalite
\subsection{Incoh\'erences arithm\'etiques}
Les r\`egles de composition des messages sont simples mais peuvent
engendrer des incoh\'erences dans l'\'evaluation des expressions
arithm\'etiques qui sont exprim\'ees sous forme de messages binaires
%ajout
(nous parlons aussi d'irrationnalit\'e arithm\'etique).
Voici des situations habituelles o\`u des parenth\`eses suppl\'ementaires sont n\'ecessaires.

\begin{code}{@TEST}
3 + 4 * 5      --> 35    "(pas 23)  les messages binaires sont !envoy\'es! de gauche !\`a! droite"
3 + (4 * 5)    --> 23
1 + 1/3         --> (2/3)    "et pas 4/3"
1 + (1/3)       --> (4/3)
1/3 + 2/3       --> (7/9)    "et pas 1"
(1/3) + (2/3)  --> 1
\end{code}

\paragraph{Exemple.} 
Dans l'expression \ct{20 + 2 * 5}, il y a seulement les messages
binaires \ct{+} et \ct{*}. En \st, il n'y a pas de priorit\'e
sp\'ecifique pour les op\'erations \ct{+} et \ct{*}. Ce ne sont que
des messages binaires, ainsi \ct{*} n'a pas priorit\'e sur \ct{+}. Ici
le message le plus \`a gauche \ct{+} est envoy\'e en premier (1) et
ensuite \ct{*} est envoy\'e au r\'esultat comme nous le voyons dans l'\egref{binaryMessages1}.  

\begin{example}[binaryMessages1]{D\'ecomposer \ct{20 + 2 * 5}}{}
"Comme il n'y a pas de !priorit\'e! entre les messages binaires, le message le plus !\`a! gauche, + est !\'evalu\'e! en premier !m\^eme! si !d'apr\`es! les !r\`egles! de !l'arithm\'etique! le * devrait d'abord !\^etre! !envoy\'e.!"

      20 + 2 * 5 
(1)  20 + 2 --> 22
(2)  22       * 5 --> 110
\end{example}

\begin{figure}
\begin{center}\includegraphics[width=8cm]{ucompoNoBracketPar}\end{center}
\end{figure}
\noindent
Comme il est montr\'e dans l'\egref{binaryMessages1} le r\'esultat de
cette expression n'est pas \ct{30} mais \ct{110}. Ce r\'esultat est
peut-\^etre inattendu mais r\'esulte directement des r\`egles
utilis\'ees pour envoyer des messages. Ceci est le prix \`a payer pour
la simplicit\'e du mod\`ele de \st. Afin d'avoir un r\'esultat
correct, nous devons utiliser des parenth\`eses. Lorsque les messages
sont entour\'es par des parenth\`eses, ils sont \'evalu\'es en
premier. Ainsi l'expression \ct{20 + (2 * 5)} retourne le r\'esultat
comme nous le voyons dans l'\egref{mathcorrect}.

\needlines{4}
\begin{example}[mathcorrect]{D\'ecomposition de \ct{20 + (2 * 5)}}{}
"Les messages !entour\'es! de !parenth\`eses! sont !\'evalu\'es! en premier ainsi * est !envoy\'e! avant + afin de produire le comportement !souhait\'e.!"

    20 + (2 * 5) 
(1)        (2 * 5) --> 10
(2) 20 + 10      --> 30
\end{example}

\begin{figure}
\begin{center}
\includegraphics[width=8cm]{ucompoNumberBracket}
\end{center}
\end{figure}

\important{En \st, les op\'erateurs arithm\'etiques comme + et * n'ont
  pas des priorit\'es diff\'erentes. \ct{+} et \ct{*} ne sont que des
  messages binaires; donc \ct{*} n'a pas priorit\'e sur
  \ct{+}. Utiliser des parenth\`eses pour obtenir le r\'esultat d\'esir\'e.}

%  At the beginning put parenthesis when you have multiple binary messages.}  HUH?  At the beginning of what?!

\begin{figure}
\begin{center}
\ifluluelse
	{\includegraphics[width=\textwidth]{uKeyUnBinPar}}
	{\includegraphics[width=0.8\textwidth]{uKeyUnBinPar}}
\ifluluelse
	{\includegraphics[width=\textwidth]{uunKeyBinPar}}
	{\includegraphics[width=10cm]{uunKeyBinPar}}
\end{center}
\caption{Messages \'equivalents en utilisant des parenth\`eses.\figlabel{uKeyUnBinPar}}
\end{figure}

Notez que la premi\`ere r\`egle, disant que les messages unaires sont
envoy\'es avant les messages binaires ou \`a mots-cl\'es, ne nous force
pas \`a mettre explicitement des parenth\`eses autour
d'eux. \Tabref{expressions} montre des expressions \'ecrites en
respectant les r\`egles et les expressions \'equivalentes si les
r\`egles n'existaient pas. Les deux versions engendrent le m\^eme
effet et retournent les m\^emes valeurs.

\begin{figure}\centering
	\begin{tabular}{l@{\qquad}l}
	\toprule
	Priorit\'e implicite & \'Equivalent explicite parenth\'es\'e\\
	\midrule
	\lct{aPen color: Color yellow}
		& \lct{aPen color: (Color yellow)}
		\\
	\lct{aPen go: 100 + 20}
		& \lct{aPen go: (100 + 20)}
		\\
	\lct{aPen penSize: aPen penSize + 2}
		& \lct{aPen penSize: ((aPen penSize) + 2)}
		\\
	\lct{2 factorial + 4}
		& \lct{(2 factorial) + 4}
		\\
	\bottomrule
	\end{tabular}
	\caption{Des expressions et leurs versions \'equivalentes compl\`etement parenth\'es\'ees.\tablabel{expressions}}
\end{figure}

%=============================================================
\section{Quelques astuces pour identifier les messages \`a mots-cl\'es}
Souvent les d\'ebutants ont des probl\`emes pour comprendre quand ils doivent ajouter des parenth\`eses. Voyons comment les messages \`a mots-cl\'es sont reconnus par le compilateur.

%-------------------------------------------------------------
\subsection{Des parenth\`eses ou pas ?}
Les caract\`eres \ct{[}, \ct{]}, and \ct{(}, \ct{)} 
d\'elimitent des zones distinctes. Dans ces zones, un message \`a mots-cl\'es est la plus longue s\'equence de mots termin\'es par (\ct{:}) qui n'est pas coup\'e par les caract\`eres (\ct{.}), ou (\ct{;}). 
Lorsque les caract\`eres \ct{[}, \ct{]}, et \ct{(}, \ct{)} entourent des mots avec des deux points, ces mots participent au message \`a mots-cl\'es \emph{local} \`a la zone d\'efinie.

Dans cet exemple, il y a deux mots-cl\'es distincts: \ct{rotatedBy:magnify:smoothing:} et \ct{at:put:}.

\begin{code}{}
aDict
   at: (rotatingForm 
          rotateBy: angle	
          magnify: 2 
          smoothing: 1)
   put: 3
\end{code}

\important{
Les caract\`eres \ct{[}, \ct{]}, et \ct{(}, \ct{)} d\'elimitent des zones distinctes. Dans ces zones, un message \`a mots-cl\'es est la plus longue s\'equence de mots qui se termine par (\ct{\:}) qui n'est pas coup\'e par les carac\`eres (\ct{.}),  ou \ct{\;}. 
Lorsque les caract\`eres \ct{[}, \ct{]}, et \ct{(}, \ct{)} entourent des mots avec des deux points, ces mots participent au message \`a mots-cl\'es local \`a cette zone.}

\on{Sounds terribly complicated.} %martial: d'accord

\hint{Si vous avez des probl\`emes avec ces r\`egles de priorit\'e, vous pouvez commencer simplement en entourant avec des parenth\`eses chaque fois que vous voulez distinguer deux messages avec la m\^eme priorit\'e.}

L'expression qui suit ne n\'ecessite pas de parenth\`eses car l'expression \ct{x isNil} est unaire donc envoy\'ee avant le message \`a mots-cl\'es \mbox{\lct{ifTrue:}.}
\begin{code}{}
(x isNil)
   ifTrue:[...]
\end{code}

L'expression qui suit n\'ecessite des parenth\`eses car les messages \ct{includes:} et \ct{ifTrue:} sont chacun des messages \`a mots-cl\'es. 
\begin{code}{}
ord := OrderedCollection new.
(ord includes: $a)
   ifTrue:[...]
\end{code}%$

\noindent
Sans les parenth\`eses le message inconnu \ct{includes:ifTrue:} serait envoy\'e \`a la collection!

%-------------------------------------------------------------
\subsection{Quand utiliser les \lct{[ ]} ou les \lct{( )} ?}

Vous pouvez avoir des difficult\'es \`a comprendre quand utiliser des crochets plut\^ot que des parenth\`eses.
Le principe de base est que vous devez utiliser des \ct{[ ]} lorsque vous ne savez pas combien de fois une expression peut \^etre \'evalu\'ee (peut-\^etre m\^eme jamais).
\lct{[\emph{expression}]} va cr\'eer une fermeture lexicale ou
\ind{bloc} (\ie un objet) \`a partir de
\mbox{\lct{\emph{expression}},} qui peut \^etre \'evalu\'ee autant de
fois qu'il le faut (voire jamais) en fonction du contexte.

Ainsi les clauses conditionnelles de \ct{ifTrue:} ou \ct{ifTrue:ifFalse:} n\'ecessitent des blocs. Suivant le m\^eme principe, \`a la fois le receveur et l'argument du message \ct{whileTrue:} n\'ecessitent l'utilisation des crochets car nous ne savons pas combien de fois le receveur ou l'argument seront ex\'ecut\'es.

Les parenth\`eses quant \`a elles n'affectent que l'ordre d'envoi des messages.
Aucun objet n'est cr\'e\'e, ainsi dans \lct{(\emph{expression})},
\lct{\emph{expression}} sera \emph{toujours} \'evalu\'e exactement une
fois 
%martial: erreur dans l'original: (en supposant que le code du son est
%\'evalu\'e une fois). En fait, il ne s'agit pas de 'sounding' mais
%'surrounding'
(en supposant que le code englobant l'expression soit \'evalu\'e une
fois).

\begin{code}{}
[ x isReady ] whileTrue: [ y doSomething ]   "!\`a! la fois le receveur et l'argument doivent !\^etre! des blocs"
4 timesRepeat: [ Beeper beep ]                   "l'argument est !\'evalu\'e! plus d'une fois, donc doit !\^etre! un bloc"
(x isReady) ifTrue: [ y doSomething ]           "le receveur est !\'evalu\'e! qu'une fois, donc n'est pas un bloc!"
\end{code}

%=============================================================
\section{S\'equences d'expression}
Les expressions (\ie envois de message, affectations\ldots) s\'epar\'ees par des points sont \'evalu\'ees en s\'equence.
Notez qu'il n'y a pas de point entre la d\'efinition d'un variable et l'expression qui suit.
La valeur d'une s\'equence est la valeur de la derni\`ere
expression. Les valeurs retourn\'ees par toutes les expressions
except\'ee la derni\`ere sont ignor\'ees. Notez que le point est un 
%\subind{statement}{s\'eparateur}
\subind{expression}{séparateur}
et non un terminateur d'expression. Le point final est donc optionnel.
\seeindex{séparateur}{expression, séparateur}

\begin{code}{@TEST}
| box |
box := 20@30 corner: 60@90.
box containsPoint: 40@50 --> true
\end{code}

%=============================================================
\section{Cascades de messages}
\st offre la possibilit\'e d'envoyer plusieurs messages au m\^eme
receveur en utilisant le point-virgule (\ct{;}). Dans le jargon \st,
nous parlons de \emphind{cascade}.
\seeindex{message!cascade}{cascade}

\important{Expression Msg1 ; Msg2}

\begin{minipage}{0.35\textwidth}
\begin{code}{}
Transcript show: 'Pharo est '.
Transcript show: 'extra '.
Transcript cr.
\end{code}
\end{minipage}
\emph{~est \'equivalent \`a :~}
\begin{minipage}{0.35\textwidth}
\begin{code}{}
Transcript        
   show: 'Pharo est';
   show: 'extra ';
   cr
\end{code}
\end{minipage}

Notez que l'objet qui re\c{c}oit la cascade de messages peut \'egalement \^etre le r\'esultat d'un envoi de message.
En fait, le receveur de la cascade est le receveur du premier message
de la cascade. Dans l'exemple qui suit, le premier message en cascade
est \ct{setX:setY} puisqu'il est suivi du point-virgule. Le receveur
du message cascad\'e \ct{setX:setY:} est le nouveau point r\'esultant
de l'\'evaluation de \ct{Point new}, et \emph{non pas} \ct{Point}. Le
message qui suit \ct{isZero} (pour tester s'il s'agit de z\'ero) est
envoy\'e au m\^eme receveur. 

\begin{code}{}
Point new setX: 25 setY: 35; isZero --> false
\end{code}

%=============================================================
\section{R\'esum\'e du chapitre}

\begin{itemize}
\item Un message est toujours envoy\'e \`a un objet nomm\'e le \emph{receveur} qui peut \^etre le r\'esultat d'autres envois de messages.

\item Les messages unaires sont des messages qui ne n\'ecessitent pas d'arguments.\\
Ils sont de la forme \lct{receveur \textbf{s\'electeur}}.

\item Les messages binaires sont des messages qui concernent deux objets, le receveur et un autre objet \emph{et} dont le s\'electeur est compos\'e de un ou deux caract\`eres de la liste suivante: \ct{+}, \ct{-}, \ct{*}, \ct{/}, \ct{|}, \texttt{\&}, \ct{=}, \ct{>}, \ct{<}, \texttt{\~}, et \ct{@}.\\
Ils sont de la forme: \lct{receveur \textbf{s\'electeur} argument}.
\item Les messages \`a mots-cl\'es sont des messages qui concernent plus d'un objet et qui contiennent au moins un caract\`ere deux points (\ct{:}).\\
Ils sont de la forme: 
\lct{receveur \textbf{s\'electeurMotUn:} argumentUn \textbf{motDeux:} argumentDeux}.

\item \textbf{R\`egle un.} Les messages unaires sont d'abord envoy\'es, puis les messages binaires et finalement les messages \`a mots-cl\'es.
\item \textbf{R\`egle deux.} Les messages entre parenth\`eses sont envoy\'es avant tous les autres.
\item \textbf{R\`egle trois.} Lorsque les messages sont de m\^eme nature, l'ordre d'\'evaluation est de gauche \`a droite.
\item En \st, les op\'erateurs arithm\'etiques traditionnels comme +
  ou * ont la m\^eme priorit\'e. \ct{+} et \ct{*} ne sont que des
  messages binaires; donc \ct{*} n'a aucune priorit\'e sur
  \ct{+}. Vous devez utiliser les parenth\`eses pour obtenir un
  r\'esultat diff\'erent.
\end{itemize}

%\end{document}
% ON: Don't ever put an \end{document} in a chapter
% It will make the book stop there!
%=================================================================
\ifx\wholebook\relax\else\end{document}\fi
%=================================================================

%---------------------------------------------------------

%=================================================================
%:PART 2 -- Developing in Pharo
\part{D\'evelopper avec Pharo}
%:Model

% $Author: oscar $
% $Translation: martial $
% $Date: Wed Oct 10 17:20:28 CEST 2007 $
% $Revision: 12715 $
%=================================================================
% translated by Martial.Boniou@ifrance.com start: (Fri, 28 Sep 2007)
% relecture par Rene Mages : Thu Dec 20 12:13:40 2007
% relecture par Martial : Sat Dec 29 16:06:42 CET 2007
% relecture par Rene Mages : Thr Jan 10 17:17:17 2008
%% j'ai enleve les 'transmissions de messages': il ne reste que 'envoi
%% de messages'; j'ai remplace 'envoi sur super/self' par 'envoi a
%% ...'; j'ai remplace '\lct{}' par '\ct{}' quand possible
%
% adaptation pour PBE - martial - Thu Sep 10 20:18:24 CEST 2009 
% relecture par Rene Mages : Tue Jan 12 17:17:17 2010
% relecture par Rene Mages : Tue Aug  8 17:17:17 2010
% relecture par Rene Mages : Mon Apr 11 17:17:17 2011
% relecture par Rene Mages : Sun May 29 17:17:17 2011
% $Author: oscar $ $Date: 2009-08-28 11:02:05 +0200 (Fri, 28 Aug 2009) $ $Revision: 28659 $
% sync avec la version: 29660
%=================================================================
\ifx\wholebook\relax\else
% --------------------------------------------
% Lulu:
	\documentclass[a4paper,10pt,twoside]{book}
	\usepackage[
		papersize={6.13in,9.21in},
		hmargin={.75in,.75in},
		vmargin={.75in,1in},
		ignoreheadfoot
	]{geometry}
	\input{../common.tex}
	\pagestyle{headings}
	\setboolean{lulu}{true}
% --------------------------------------------
% A4:
%	\documentclass[a4paper,11pt,twoside]{book}
%	\input{../common.tex}
%	\usepackage{a4wide}
% --------------------------------------------
    \graphicspath{{figures/} {../figures/}}
	\begin{document}
	\renewcommand{\nnbb}[2]{} % Disable editorial comments
	\sloppy
\fi
%=================================================================
\chapter{Le modèle objet de \st}
\chalabel{model}

Le modèle de programmation de \st est simple et homogène: \mantra et les objets communiquent les uns avec les autres uniquement par envoi de messages.
Cependant, ces caractéristiques de simplicité et d'homogénéité peuvent être source de quelques difficultés pour le développeur habitué à d'autres langages de programmation. Dans ce chapitre nous présenterons les concepts de base du modèle objet de \st; en particulier nous discuterons des conséquences de la représentation des classes comme des objets.

%=========================================================
\section{Les règles du modèle}
\seclabel{rules}

Le modèle objet de \st repose sur un ensemble de règles simples 
qui sont appliquées de manière \emph{uniforme}. Les règles s'énoncent comme suit:

\begin{enumerate}[label={\textbf{Règle \arabic{*}}.}, ref={Règle \arabic{*}}, leftmargin=*]
\item{} \rulelabel{everything}
	\Mantra.

\item{} \rulelabel{instance}
	Tout objet est instance de classe.

\item{} \rulelabel{inheritance}
	Toute classe a une super-classe.

\item{}  \rulelabel{message}
	Tout se passe par envoi de messages. % REVOIR dans PBE 'by sending messages'

\item{}  \rulelabel{lookup}
	La recherche des méthodes suit la chaîne de l'héritage.

\end{enumerate}

\noindent
Prenons le temps d'étudier ces règles en détail.


%=========================================================
\section{\Mantra}

%\ruleref{everything}

Attention ! le mantra ``\mantra'' est très contagieux.
Après seulement quelques heures passées avec \st, vous serez progressivement surpris par la façon dont cette règle simplifie tout ce que vous faites.
Par exemple, les entiers sont véritablement des objets (de la classe Integer). Dès lors vous pouvez leur envoyer des messages, comme vous le feriez avec n'importe quel autre objet.

\begin{code}{@TEST}
3 + 4            --> 7    "!envoie '+ 4' à 3, donnant 7!"
20 factorial  --> 2432902008176640000   "envoie factorial, donnant un grand nombre"
\end{code}

La représentation de \ct{20 factorial} est certainement différente de la représentation de \ct{7}, mais aucune partie du code\,---\,pas même l'implémentation de 
\ct{factorial}~\footnote{En français, factorielle.}
\,---\,n'a besoin de le savoir puisque ce sont des objets tous deux.

\needlines{3}
La conséquence fondamentale de cette règle pourrait s'énoncer ainsi:

\important{Les classes sont aussi des objets.}
Plus encore, les classes ne sont pas des objets de seconde zone: elles sont véritablement des objets de premier plan auquels vous pouvez envoyer des messages, que vous pouvez inspecter, \etc.
Ainsi \pharo est vraiment un système réflexif offrant une grande expressivité
aux développeurs.

Si on regarde plus précisemment l'implémentation de \st, nous trouvons
trois sortes différentes d'objets. Il y a (1) les objets ordinaires
avec des variables d'instance passées par référence; il y a (2)
\emph{les petits entiers}~\footnote{En anglais, \emph{small
    integers}.} qui sont passés par valeur, et enfin, il y a (3) les
objets 
%martial: remplacement de 'indexables' (Tue Dec 25 11:37:17 CET 2007)
indexés comme les Array (tableaux) qui occupent une portion contig\"ue de mémoire. La beauté de \st réside 
dans le fait que vous n'avez aucunement à vous soucier des différences entre ces trois types
d'objet.


%=========================================================
\section{Tout objet est instance de classe}

%\ruleref{instance}

Tout objet a une classe; pour vous en assurer, vous pouvez envoyer à un objet le message \ct{class} (classe en anglais).

\begin{code}{@TEST}
1 class                 --> SmallInteger
20 factorial class --> LargePositiveInteger
'hello' class          --> ByteString
#(1 2 3) class       --> Array
(4@5) class         --> Point
Object new class --> Object
\end{code}

Une classe définit la \emph{structure} pour ses instances via les variables d'instance (instance variables en anglais)
et leur \emph{comportement} (\emph{behavior} en anglais) via les méthodes.
Chaque méthode a un nom. C'est le \emphsubind{méthode}{sélecteur}. Il est unique pour chaque classe.

Puisque \emph{les classes sont des objets} et que \emph{tout objet est une instance d'une classe}, nous en concluons que les classes doivent aussi être des instances de classes.
Les classes dont les instances sont des classes sont nommées des \emph{méta-classes}.
À chaque fois que vous créez une classe, le système crée pour vous une méta-classe
automatiquement.
La méta-classe définit la structure et le comportement de la classe qui est son instance.
La plupart du temps vous n'aurez pas à penser aux méta-classes et vous pourrez joyeusement les ignorer.
(Nous porterons notre attention aux méta-classes dans \charef{metaclasses}.)

%---------------------------------------------------------
\subsection{Les variables d'instance}

Les variables d'instance en \st sont privées vis-à-vis de l'\emph{instance} elle-même.
Ceci diffère de langages comme \ind{Java} et \ind{C++} qui permettent l'accès aux variables d'instance (aussi connues sous le nom d'``attributs'' ou ``variables membre'') depuis n'importe qu'elle autre instance de la même classe.
Nous disons que la \emphind{frontière d'encapsulation}~\footnote{En anglais, encapsulation boundary.} des objets en Java et en C++ est la classe, là où, en \st, c'est l'instance.
\seeindex{variable!instance}{variable d'instance}
\seeindex{attribut}{variable d'instance}
%\seeindex{attribute}{instance variable}
\seeindex{slot}{variable d'instance}
%\index{variable instance}
\index{variable d'instance}

En \st, deux instances d'une même classe ne peuvent pas accèder aux variables d'instance l'une de l'autre à moins que la classe ne définisse des ``méthodes d'\subind{méthode}{accès}'' (en anglais, \emph{accessor methods}).
Aucun élément de la syntaxe ne permet l'accès direct à la variable d'instances de n'importe quel autre objet.
(En fait, un mécanisme appelé \ind{réflexivité}
%, discussed in \charef{metaprog},
offre une  possibilité d'interroger un autre objet sur la valeur de ses variables d'instance; ces facilités de méta-programmation permettent d'écrire des outils tel que l'\ind{inspecteur} d'objets (nous utiliserons aussi le terme \ind{Inspector}). La seule vocation de ce dernier est de regarder le contenu des autres objets.)

Les variables d'instance peuvent être accédées par 
nom dans toutes les méthodes d'instance de la classe qui les définit
ainsi que dans les méthodes définies dans les sous-classes de cette classe.
Cela signifie que les variables d'instance en \st sont semblables aux
variables \emph{protégées} \mbox{(\texttt{protected})} en C++ et en Java. Cependant,
nous préférons dire qu'elles sont privées parce qu'il n'est pas d'usage en \st d'accéder à une variable d'instance directement depuis une sous-classe. 
\subsubsection{Exemple}
La méthode \cmind{Point}{dist:} (\mthref{dist:}) calcule la distance entre le receveur et un autre point. Les variables d'instance \ct{x} et \ct{y} du receveur sont accédées directement par le corps de la méthode. Cependant, les variables d'instance de l'autre point doivent être accédées en lui envoyant les messages \ct{x} et \ct{y}.
% rene : s/messasges/messages/ ligne 172

\needlines{7}
\begin{method}[dist:]{La distance entre deux points. le nom arbitraire \ct{aPoint} est utilisé dans le sens de \emph{a point} qui, en anglais, signifie ``un point''}
Point>>>dist: aPoint 
	"Retourne la distance entre aPoint et le receveur."
	| dx dy |
	dx := aPoint x - x.
	dy :=  aPoint y - y.
	^ ((dx * dx) + (dy * dy)) sqrt
\end{method}

\begin{code}{@TEST}
1@1 dist: 4@5 --> 5.0
\end{code}

La raison-clé de préférer l'encapsulation basée sur l'instance
à l'encapsulation basée sur la classe tient au fait qu'elle permet
à différentes implémentations d'une même abstraction de c\oe{}xister.
Par exemple, la méthode \ct{point>>>dist:} n'a besoin ni de surveiller, ni 
même de savoir si l'argument \ct{aPoint} est une instance de la même classe
que le receveur.  L'argument objet pourrait être représenté par des
coordonnées polaires, voire comme un enregistrement dans une base de données ou sur une autre machine d'un réseau distribué; tant qu'il peut répondre
aux messages \ct{x} et \ct{y}, le code de la \mthref{dist:} fonctionnera toujours.

%---------------------------------------------------------
\subsection{Les méthodes}

Toutes les méthodes sont \subind{méthode}{publique}{}s~\footnote{En fait, presque toutes.  En \pharo, des méthodes dont les sélecteurs commencent par
la chaîne de caractères \ct{pvt} sont privées: un message \ct{pvt}
ne peut être envoyé qu'à \self \emph{uniquement}.  N'importe comment, les méthodes \ct{pvt} sont très peu utilisées.}.
Les méthodes sont regroupées en protocoles qui indiquent leur objectif.
Certains noms de protocoles courants ont été attribués par convention,
par exemple, \protind{accessing} pour les méthodes d'\subind{méthode}{accès}, 
et \protind{initialization} pour construire un état initial stable pour l'objet.
\index{initialisation}
Le protocole \protind{private} est parfois utilisé pour réunir les
méthodes qui ne devraient pas être visibles depuis l'extérieur.
Rien ne vous empêche cependant d'envoyer un message qui est implémenté
par une telle méthode ``privée''.

Les méthodes peuvent accéder à toutes les variables d'instance de l'objet.
Certains programmeurs en \st préfèrent accéder aux variables d'instance
uniquement au travers des méthodes d'accès.
Cette pratique a un certain avantage, mais elle tend à rendre l'interface de vos classes chaotique, ou pire, à exposer des états privés à tous les regards.

%---------------------------------------------------------
\subsection{Le côté instance et le côté classe}

Puisque les classes sont des objets, elles peuvent avoir leurs propres variables d'instance ainsi que leurs propres méthodes.
Nous les appelons \emph{variables d'instance de classe} (en anglais \emph{class instance variables}) et \emph{méthodes de classe}, mais elles ne sont véritablement pas différentes des variables et méthodes d'instances ordinaires:
les variables d'instance de classe ne sont seulement que des variables d'instance définies par une méta-classe. Quant aux méthodes de classe,
elles correspondent juste aux méthodes définies par une \ind{méta-classe}. 
\index{classe!variable d'instance}
\seeindex{variable!instance de classe}{classe, variable d'instance}
\index{classe!méthode}

Une classe et sa \ind{méta-classe} sont deux classes distinctes, et ce, même si cette première est une instance de l'autre. Pour vous, tout ceci sera somme toute largement trivial: vous n'aurez qu'à vous concentrer sur la définition du comportement de vos objets et des classes qui les créent.

\begin{figure}[htb]
\begin{center}
\includegraphics[width=\textwidth]{Color-Buttons}
\caption{Naviguer dans une classe et sa méta-classe.
% \sd{Do we use Key everywhere in the picture as a legend indicator?}
% \on{sure, wherever appropriate}
\figlabel{Buttons}}
\end{center}
\end{figure}

De ce fait, le navigateur de classes nommé \ind{Browser} vous
aide à parcourir à la fois classes et méta-classes comme si elles
n'étaient qu'une seule entité avec deux ``côtés'': le ``\subind{Browser}{côté instance}'' et le ``\subind{Browser}{côté classe}'', comme le montre \figref{Buttons}. 
\damien{something wrong here. The text between quotes is not generated}
% true: \ind should replace \index there! (martial)
En \clickant{} sur le bouton \button{instance}, vous voyez la présentation de la classe \ct{Color} et vous donc pouvez naviguer dans les méthodes qui sont exécutées quand les messages de même nom sont envoyés à une instance de \ct{Color}, comme par exemple \lct{blue} (correspondant à la couleur bleu). En appuyant
sur le bouton \button{class} (pour classe), vous naviguez dans la classe \ct{Color class}, autrement dit vous voyez les méthodes qui seront exécutées en envoyant les messages directement à la classe \ct{Color} elle-même.
Par exemple, \ct{Color blue} envoie le message \ct{blue} (pour \emph{bleu}) à la classe \clsind{Color}.
Vous trouverez donc la méthode \ct{blue} définie côté classe de la classe \ct{Color} et non du côté instance.
\seeindex{côté instance}{Browser!côté instance}
\seeindex{côté classe}{Browser!côté classe}
\seeindex{navigateur de classes!côté instance}{Browser!côté instance}
\seeindex{navigateur de classes!côté classe}{Browser!côté classe}


\needlines{5}
\begin{code}{@TEST | aColor |}
aColor := Color blue.               "!Méthode! de classe blue"
aColor        --> Color blue
aColor red  --> 0.0         "!Méthode d'accès red (rouge) côté instance!"
aColor blue --> 1.0        "!Méthode d'accès blue (bleu) côté instance!"
\end{code}

Vous définissez une classe en remplissant le patron (ou \emph{template} en anglais) proposé
dans le \subind{Browser}{côté instance}.
Quand vous acceptez ce patron, le système crée non seulement la classe
que vous définissez mais aussi la méta-classe correspondante.
Vous pouvez naviguer dans la méta-classe en \clickant{} sur le bouton \button{class}.
Du patron employé pour la création de la méta-classe, seule la
liste des noms des variables d'instance vous est proposée pour une édition directe.  

Une fois que vous avez créé une classe, \click{} sur
le bouton \button{instance} vous permet d'éditer et de parcourir les
méthodes qui seront possédées par les instances de cette classe (et de ses sous-classes). Par exemple, nous pouvons voir dans \figref{Buttons} que 
la méthode \ct{hue} est définie pour les instances de la classe \ct{Color}.
A contrario, le bouton \button{class} vous laisse parcourir et éditer
la méta-classe (dans ce cas \ct{Color class}).

%---------------------------------------------------------
\subsection{Les méthodes de classe} 

Les méthodes de classe peuvent être relativement utiles; naviguez dans \ct{Color class} pour voir quelques bons exemples.
Vous verrez qu'il y a deux sortes de \subind{classe}{méthode}{s} définies dans une classe: celles qui créent les instances de la classe, comme \cmind{Color class}{blue} et celles qui ont une action \emph{utilitaire}, 
% martial: j'ai mis volontairement \emph pour l'adjectif utilitaire
comme \cmind{Color class}{showColorCube}.
Ceci est courant, bien que vous trouverez occasionnellement des méthodes de classe utilisées d'une autre manière.

Il est commun de placer des méthodes utilitaires dans le \subind{Browser}{côté classe} parce qu'elles peuvent être exécutées
sans avoir à créer un objet additionnel dans un premier temps. 
% coutumier -> commun
En fait, beaucoup d'entre elles contiennent un commentaire pour les rendre plus compréhensibles pour l'utilisateur qui les exécute.

\dothis{Naviguez dans la méthode \ct{Color class>>>showColorCube}, double-cliquez à l'intérieur des guillemets englobant le commentaire \ct{"Color showColorCube"} et tape au clavier \short{d}.}
Vous verrez l'effet de l'exécution de cette méthode.  (Sélectionnez \menu{World \go \ind{restore display}~(r)} pour annuler les effets.)

Pour les familiers de \ind{Java} et \ind{C++},  les méthodes de classe peuvent être assimilées aux méthodes statiques. 
Néanmoins, l'homogénéité de \st induit une différence:  les méthodes statiques de Java sont des fonctions résolues de manière statique alors que les méthodes de classe de \st sont des méthodes à transfert dynamique~\footnote{En anglais, dynamically-dispatched methods.} Ainsi, l'héritage, la surcharge et l'utilisation de \emph{super} fonctionnent avec les méthodes de classe dans \st, ce qui n'est pas le cas avec les méthodes statiques en Java.  

%---------------------------------------------------------
\subsection{Les variables d'instance de classe}
Dans le cadre des variables d'instance ordinaires, toutes les instances d'une classe partagent le même ensemble
de noms de variable et les instances de ses sous-classes héritent de ces noms; cependant, chaque instance possède son propre jeu de valeurs.
C'est exactement la même histoire avec les \subind{classe!variables d'instance}{variables d'instance de classe}: chaque classe a ses propres variables d'instance de classe privées.
Une sous-classe héritera de ces  variables d'instance de classe, \emph{mais elle aura ses propres copies privées de ces variables}.
Aussi vrai que les objets ne partagent pas les variables d'instance, les classes et leurs sous-classes ne partagent pas les variables d'instance de classe.

Vous pouvez utiliser une variable d'instance de classe \ct{count}~\footnote{En français, compteur.} afin de suivre le nombre d'instances que vous créez pour une classe donnée.
Cependant, les sous-classes ont leur propre variable \ct{count}, 
les instances des sous-classes seront comptées séparément.

\paragraph{Exemple: les variables d'instance de classe ne sont pas partagées avec les sous-classes.}
Soient les classes \ct{Dog} et \ct{Hyena}~\footnote{En français, chien et hyène.} telles que \ct{Hyena} hérite de la variable d'instance de classe \ct{count} de la classe \ct{Dog}.

%martial: si mauvaise coupe:
%\newpage
\begin{classdef}[dog]{Créer Dog et Hyena}
Object subclass: #Dog
	instanceVariableNames: ''
	classVariableNames: ''
	poolDictionaries: ''
	category: 'PBE-CIV'

Dog class
	instanceVariableNames: 'count'

Dog subclass: #Hyena
	instanceVariableNames: ''
	classVariableNames: ''
	poolDictionaries: ''
	category: 'PBE-CIV'
\end{classdef}

Supposons que nous ayons des méthodes de classe de \ct{Dog} pour initialiser sa variable \ct{count} à \ct{0} et pour incrémenter cette dernière quand de nouvelles instances sont créées:
\begin{method}[dogcount]{Comptabiliser les nouvelles instances de \ct{Dog} via \ct{Dog class>>>count}}
Dog class>>>initialize
	super initialize.
	count := 0.

Dog class>>>new
	count := count +1.
	^ super new

Dog class>>>count
	^ count
\end{method}

Maintenant, à chaque fois que nous créons un nouveau \ct{Dog}, son compteur
count est incrémenté. Il en est de même pour toute nouvelle instance de \ct{Hyena}, mais elles sont comptées séparément:
\begin{code}{}
Dog initialize.
Hyena initialize.
Dog count     --> 0
Hyena count --> 0
Dog new.
Dog count     --> 1
Dog new.
Dog count     --> 2
Hyena new.
Hyena count --> 1
\end{code}
% ON: In order to make this a test, I need the previous code to be part of the setup. Bleh.

Remarquons aussi que les variables d'instance de classe sont privées à la classe tout comme les variables d'instance sont privées à l'instance. 
Comme les classes et leurs instances sont des objets différents,
il en résulte que:
\important{Une classe n'a pas accès aux variables d'instance de ses propres instances.}
\important{Une instance d'une classe n'a pas accès aux variables d'instance de classe de sa classe.}
C'est pour cette raison que les méthodes d'initialisation d'instance doivent 
toujours être définies dans le \subind{Browser}{côté instance}\,---\,le \subind{Browser}{côté classe} n'ayant pas accès aux variables d'instance, il ne pourrait y avoir initiali\-sation!  
% 2007-11-05 correction errata SBE
Tout ce que peut faire la classe, c'est d'envoyer des messages d'\ind{initialisation} à des instances nouvellement créées; ces messages pouvant bien sûr utiliser les méthodes d'\subind{méthode}{accès}.

De même, les instances ne peuvent accéder aux variables d'instance de classe
que de manière indirecte en envoyant les messages d'accès à leur classe.

\ind{Java} n'a rien d'équivalent aux variables d'instance de classe.  
Les variables statiques en Java et en \ind{C++} ont plutôt des similitudes 
avec les variables de classe de \st dont nous parlerons dans \secref{classVars}: toutes les sous-classes et leurs instances partagent la même variable statique.

\paragraph{Exemple: Définir un Singleton.}
Le patron de conception~\footnote{En anglais, nous parlons de \emph{Design Patterns}.} nommé \ind{Singleton}~\cite{Alpe98a} offre un exemple-type de l'usage de variables d'instance de classe et de méthodes de classe.
Imaginez que nous souhaitions d'une part, créer une classe \ct{WebServer} et d'autre part, s'assurer qu'il n'a qu'une et une seule instance en faisant appel au patron Singleton.

En \clickant{} sur le bouton \button{instance} dans le navigateur de classe, nous définissons la classe \clsind{WebServer} comme suit (\clsref{singleton}).

\begin{classdef}[singleton]{Une classe Singleton}
Object subclass: #WebServer
	instanceVariableNames: 'sessions' 	
	classVariableNames: '' 	
	poolDictionaries: ''
	category: 'Web'
\end{classdef}

Ensuite, en \clickant{} sur le bouton \button{class}, nous pouvons ajouter une variable d'instance \ct{uniqueInstance} au \subind{Browser}{côté classe}.

\begin{classdef}[webserver]{Le côté classe de la classe Singleton}
WebServer class 	
	instanceVariableNames: 'uniqueInstance'
\end{classdef}

Par conséquence, la classe \ct{WebServer} a désormais une autre variable d'instance, en plus des variables héritées telles que \ct{superclass} et \ct{methodDict}.

Nous pouvons maintenant définir une \subind{classe}{méthode} de classe que nous appellerons \ct{uniqueInstance} comme dans \tmthref{uniqueInstance}. 
Pour commencer, cette méthode vérifie si \ct{uniqueInstance} a été initialisée ou non: dans ce dernier cas,
la méthode crée une instance et l'assigne à la variable d'instance de classe \ct{uniqueInstance}.  
\emph{In fine}, la valeur de \ct{uniqueInstance} est retournée.
Puisque \ct{uniqueInstance} est une variable d'instance de classe, cette méthode peut directement y accéder.

\needlines{4}
\begin{method}[uniqueInstance]{\ct{WebServer class>>>uniqueInstance} (côté classe)}
WebServer class>>>uniqueInstance
     uniqueInstance ifNil: [uniqueInstance := self new].
     ^ uniqueInstance
\end{method}

La première fois que l'expression \ct{WebServer uniqueInstance} est exécutée, une instance de la classe \ct{WebServer} sera créée et affectée à la variable \ct{uniqueInstance}. 
La seconde fois, l'instance précédemment créée sera retournée au lieu d'y avoir une nouvelle création. 

Remarquons que la clause conditionnelle à l'intérieur du code de création
de \tmthref{uniqueInstance} est écrite \ct{self new} et non 
\mbox{\ct{WebServer new}.}
Quelle en est la différence?   Comme la méthode \ct{uniqueInstance} est définie dans \mbox{\ct{WebServer class},} vous pouvez penser qu'elles sont identiques.    En fait, tant que personne ne crée une sous-classe de \ct{WebServer}, elles sont pareilles. Mais en supposant que \ct{ReliableWebServer} est une sous-classe de \ct{WebServer} et qu'elle hérite de la méthode \ct{uniqueInstance},
nous devrions nous attendre à ce que \ct{ReliableWebServer uniqueInstance} réponde un \ct{ReliableWebServer}. L'utilisation de \self assure que cela arrivera car il sera lié à la classe correspondante.
Du reste, notez que \ct{WebServer} et \ct{ReliableWebServer} ont chacune
leur propre variable d'instance de classe nommée \ct{uniqueInstance}.  

% proposition de Rene :
% Ces deux variables ont, bien entendu, différentes valeurs.
Ces deux variables ont, bien entendu, différentes valeurs.

%=========================================================
\section{Toute classe a une super-classe}

%\ruleref{inheritance}

Chaque classe en \st hérite de son comportement et de la description
de sa structure d'une unique \emphind{super-classe}.
Ceci est équivalent à dire que \st a un \ind{héritage} simple.

\needlines{2}
\begin{code}{@TEST}
SmallInteger superclass --> Integer
Integer superclass          --> Number
Number superclass        --> Magnitude
Magnitude superclass    --> Object
Object superclass           --> ProtoObject
ProtoObject superclass  --> nil
\end{code}

Traditionnellement, la racine de la hiérarchie d'héritage en \st est la classe \clsind{Object} (``Objet'' en anglais; puisque \mantra).
En \pharo, la racine est en fait une classe nommée \clsind{ProtoObject}, mais
normalement, vous n'aurez aucune attention à accorder à cette classe.
\ct{ProtoObject} encapsule le jeu de méthodes restreint que tout objet \emph{doit} avoir.  
N'importe comment, la plupart des classes héritent de \ct{Object} qui, pour
sa part, définit beaucoup de méthodes supplémentaires que presque tous les
objets devraient comprendre et auquels ils devraient pouvoir répondre.
à moins que vous ayez une autre raison de faire autrement, vous devriez
normalement générer des classes d'application par l'héritage
de la classe \ct{Object} ou d'une de ses sous-classes lors de la création de classe.

\dothis{Une nouvelle classe est normalement créée par l'envoi du message
\ct{subclass: instanceVariableNames: ...}
à une classe existante.
Il y a d'autres méthodes pour créer des classes.
Veuillez jeter un coup d'\oe il au protocole \prot{Kernel-Classes \go Class \go subclass creation} pour voir desquelles il s'agit.}
\scatindex{Kernel-Classes}
\protindex{création}

%There is no special syntax for creating abstract classes in \st.
%An abstract class is an ordinary class in which the implementation of some methods is deferred to a subclass.
%This is repeated in the next section

Bien que \pharo ne dispose pas d'héritage multiple, il dispose d'un mécanisme appelé % CHANGE 
% martial: footnote ajoute
\emphind{trait}{}s~\footnote{Dans le sens de trait de caractères, nous faisons allusion ainsi à la génétique du comportement d'une méthode.} 
pour partager le comportement entre des classes distincts.
Les \emph{traits} sont des collections de méthodes qui peuvent être réutilisées par plusieurs classes sans lien d'héritage. Employer les \emph{traits} vous permet de partager du code entre les différentes classes sans reproduire ce code.

%---------------------------------------------------------
\subsection{Les méthodes abstraites et les classes abstraites}

Une classe \subind{classe}{abstraite} est une classe qui n'existe que pour être héritée, au lieu d'être instanciée.
Une classe abstraite est habituellement incomplète, dans le sens qu'elle ne définit pas toutes les méthodes qu'elle utilise.
Les méthodes ``manquantes''\,---\,celle que les autres méthodes envoyent, mais qui ne sont pas définies elles-mêmes\,---\,sont dites mé\-tho\-des \subind{méthode}{abstraite}{s}.
\seeindex{classe abstraite}{classe, abstraite}
\seeindex{méthode abstraite}{méthode, abstraite}

\st n'a pas de syntaxe dédiée pour dire qu'une méthode ou qu'une classe est abstraite. 
Par convention, le corps d'une méthode abstraite contient l'expression \ct{self subclassResponsibility}~\footnote{Dans le sens, laissée à la responsabilité de la sous-classe.}. 
Ceci est connu sous le nom de ``marker method'' ou marqueur de méthode; il indique que les sous-classes ont la responsabilité de définir une version concrète de la méthode. 
Les méthodes \ct{self subclassResponsibility} devraient toujours être surchargées, et ainsi, ne devraient jamais être exécutées.
Si vous oubliez d'en surcharger une et que celle-ci est exécutée, une exception sera levée. 
\cmindex{Object}{subclassResponsibility}

Une classe est considérée comme abstraite si une de ses méthodes est abstraite.
Rien ne vous empêche de créer une instance d'une classe abstraite; tout fonctionnera jusqu'à ce qu'une méthode abstraite soit invoquée. 

\subsubsection{Exemple: la classe \ct{Magnitude}.}
\clsind{Magnitude} est une classe abstraite qui nous aide à définir
des objets pouvant être comparables les uns avec les autres. Les
sous-classes de \ct{Magnitude} devraient implémenter les méthodes
\ct{<}, \ct{=} et \ct{hash}~\footnote{Relatif au code de hachage.}.
Grâce à ces messages, \ct{Magnitude} définit d'autres méthodes telles que
\mbox{\ct{>},} \mbox{\ct{>=},} \mbox{\ct{<=},} \mbox{\ct{max:},} \mbox{\ct{min:},} \ct{between:and:} et
d'autres encore pour comparer des objets.
Ces méthodes sont héritées par les sous-classes.
La méthode \mthind{Magnitude}{<} est abstraite et est définie comme
dans \tmthref{MagnitudeLessThan}.

\begin{method}[MagnitudeLessThan]{\ct{Magnitude>>><}. Le commentaire dit: ``répond si le receveur est inférieur à l'argument''}
Magnitude>>>< aMagnitude 
	"Answer whether the receiver is less than the argument."
	^ self subclassResponsibility
\end{method}

\noindent
A contrario, la méthode \mthind{Magnitude}{>=} est concrète; elle est définie en fonction de \ct{<}:

\begin{method}[Magnitude>=]{\ct{Magnitude>>>>=}. Le commentaire dit: ``répond si le receveur est plus grand ou égal à l'argument''}
>= aMagnitude 
	"Answer whether the receiver is greater than or equal to the argument."
	^ (self < aMagnitude) not
\end{method}
Il en va de même des autres méthodes de comparaison.

\clsind{Character} est une sous-classe de \ct{Magnitude}; elle surcharge la méthode \mthind{Object}{subclassResponsibility} de \ct{<} avec sa propre version de \ct{<} (voir \mthref{CharacterLessThan}).  \ct{Character} définit aussi les méthodes \ct{=} et \ct{hash}; elles héritent entre autres des méthodes \mbox{\ct{>=},} \ct{<=} et \ct{~=} de la classe \ct{Magnitude}.

\begin{method}[CharacterLessThan]{\ct{Character>>><}. Le commentaire dit: ``répond vrai si la valeur du receveur est inférieure à la valeur du l'argument''}
Character>>>< aCharacter 
	"Answer true if the receiver's value < aCharacter's value."
	^ self asciiValue < aCharacter asciiValue
\end{method}

%---------------------------------------------------------
\subsection{Traits}
Un \emphind{trait} est une collection de méthodes qui peut être incluse dans le comportement d'une classe sans le besoin d'un héritage.
Les classes disposent non seulement d'une seule super-classe mais aussi de la facilité offerte par le partage de méthodes utiles avec d'autres méthodes sans lien de parenté vis-à-vis de l'héritage.

Définir un nouveau \emph{trait} se fait en remplaçant simplement le patron
pour la création de la sous-classe par un message à la classe \clsind{Trait}.

\needspace{5\baselineskip}
\begin{classdef}[tauthor]{Définir un nouveau \emph{trait}}
Trait named: #TAuthor
	uses: { }
	category: 'PBE-LightsOut'
\end{classdef}

\noindent
Nous définissons ici le \emph{trait} \ct{TAuthor} dans la catégorie \scat{PBE-LightsOut}.
Ce \emph{trait} n'\emph{utilise}~\footnote{Terme anglais: \emph{uses}: il signifie ``utilise''.} aucun autre \emph{trait} existant.
En général, nous pouvons spécifier l'\emph{expression de composition d'un trait} par d'autres \emph{traits} en u\-ti\-li\-sant le mot-clé \ct{uses:}.
Dans notre cas, nous écrivons un tableau vide \mbox{(\{ \}).}

Les \emph{traits} peuvent contenir des méthodes, mais aucune variable d'instance.
Supposons que nous voulons ajouter une méthode \ct{author} (auteur en anglais) à différentes classes sans lien hiérarchique;
nous le ferions ainsi:

\begin{method}[author]{Définir la méthode \ct{TAuthor>>>author}}
TAuthor>>>author
    "Returns author initials"
	^ 'on'    "oscar nierstrasz"
\end{method}

\noindent
Maintenant nous pouvons employer ce \emph{trait} dans une classe ayant déjà sa propre super-classe, disons, la classe \ct{LOGame} que nous avons définie dans \charef{firstApp}.
Nous n'avons qu'à modifier le patron de création de la classe \ct{LOGame} pour y inclure cette fois l'argument-clé \ct{uses:} suivi du \emph{trait} à utiliser: \ct{TAuthor}.

\begin{classdef}[sbegamewithtrait]{Utiliser un trait}
BorderedMorph subclass: #LOGame
	uses: TAuthor
	instanceVariableNames: 'cells'
	classVariableNames: ''
	poolDictionaries: ''
	category: 'PBE-LightsOut'
\end{classdef}

Si nous instançions maintenant \ct{LOGame}, l'instance répondra comme prévu au message \ct{author}.

\begin{code}{}
LOGame new author --> 'on'
\end{code}

Les expressions de composition de \emph{trait} peuvent combiner plusieurs \emph{traits} via l'opérateur \ct{+}.
En cas de conflit (\ie quand plusieurs \emph{traits} définissent des méthodes avec le même nom), ces conflits peuvent être résolus en retirant explicitement ces méthodes (avec \ct{-}) ou en redéfinissant ces méthodes dans la classe ou le \emph{trait} que vous êtes en train de définir.
Il est possible aussi de créer un \emph{alias} des méthodes (avec \ct{@}) 
leur fournissant ainsi un nouveau nom.

Les \emph{traits} sont employés dans le noyau du système~\footnote{En anglais, System kernel.}.
Un bon exemple est la classe \mbox{\clsind{Behavior}.}

\needlines{8}
\begin{classdef}[behaviorwithtraits]{Behavior définit par les \emph{traits}}
Object subclass: #Behavior
	uses: TPureBehavior @ {#basicAddTraitSelector:withMethod:->#addTraitSelector:withMethod:}
	instanceVariableNames: 'superclass methodDict format'
	classVariableNames: 'ObsoleteSubclasses'
	poolDictionaries: ''
	category: 'Kernel-Classes'
\end{classdef}
\noindent
Ici, nous voyons que la méthode \ct{basicAddTraitSelector:withMethod:} définie dans le \emph{trait} \ct{TPureBehavior} a été renommée en \mbox{\ct{addTraitSelector:withMethod:}.}
%-Here we see that the method \ct{addTraitSelector:withMethod:} defined in the trait \ct{TPureBehavior} has been aliased to \ct{basicAddTraitSelector:withMethod:}.
Les \emph{traits} sont à présent supportés par les navigateurs de classe (ou \emph{browsers}).

%=========================================================
\section{Tout se passe par envoi de messages} % REVOIR 'by sending messages'

%\ruleref{message}

Cette règle résume l'essence même de la programmation en \st.

Dans la programmation procédurale, lorsqu'une procédure est appelée, l'appelant (\emph{caller}, en anglais) fait le choix du morceau de code à exécuter; il choisit la procédure ou la fonction à exécuter \emph{statiquement}, par nom.  

En programmation orientée objet, nous ne faisons \emph{pas}
d'``appel de méthodes''. Nous faisons un ``\subind{message}{envoi}
de messages.'' % REVOIR 'by sending messages'
Le choix de terminologie est important.
Chaque objet a ses propres responsabilités.
Nous ne pouvons \emph{dire} à un objet ce qu'il faut faire en lui imposant 
une procédure.
Au lieu de cela, nous devons lui \emph{demander} poliment de faire quelque chose en lui envoyant un message.
Le message n'est \emph{pas} un morceau de code: ce n'est rien d'autre qu'un nom (sélecteur) et une liste d'arguments.
Le receveur décide alors de comment y répondre en sélectionnant en retour
sa propre méthode correspondant à ce qui a été demandé.
Puisque des objets distincts peuvent avoir différentes méthodes pour répondre à un même message, le choix de la méthode doit se faire \emph{dynamiquement} à la réception du message.
\begin{code}{@TEST}
3 + 4         --> 7          "!envoie le message + d'argument 4 à l'entier  3!"
(1@2) + 4 --> 5@6    "envoie le message + d'argument 4 au point (1@2)"
\end{code}
\noindent
En conséquence, nous pouvons envoyer le \emph{même message} à différents objets, chacun pouvant avoir \emph{sa propre méthode} en réponse au message.
Nous ne disons pas à \ct{SmallInteger} \ct{3} ou au \ct{Point} \ct{1@2} comment répondre au message \ct{+ 4}.
Chacun a sa propre méthode pour répondre à cet envoi de message, et répond ainsi selon le cas.

L'une des conséquences du modèle d'envoi de messages de \st est
qu'il encourage un style de programmation dans lequel les objets
tendent à avoir des méthodes très compactes en déléguant des
tâches aux autres objets, plutôt que d'implémenter de
gigantesques méthodes procédurales engendrant trop de
responsabilité. % REVOIR 'by sending messages'
Joseph Pelrine
\ab{Citation?}
\on{désolé, simple communiqué personel de l'auteur et notes de lecture!}
dit succintement le principe suivant:
\important{Ne fais rien que tu ne peux déléguer à quelqu'un d'autre~\traduit{Don't do anything that you can push off onto someone else.}.}
\index{Pelrine, Joseph}

Beaucoup de langages orientés objets disposent à la fois
d'opérations statiques et dynamiques pour les objets; en \st il n'y
a qu'envois de messages dynamiques. Au lieu de fournir des
opérations statiques sur les classes, nous leur envoyons simplement
des messages, puisque les classes sont aussi des objets. % REVOIR 'by sending messages'

\emph{Pratiquement} tout en \st se passe par envoi de messages. % REVOIR 'by sending messages'
À certains stades, le pragmatisme doit prendre le relais:
\begin{itemize}
  \item Les \emph{déclarations de variable} ne reposent pas sur l'envoi de messages.
  		En fait, les \subind{variable}{déclaration}{}s de variable ne sont même pas exécutables.
  		Déclarer une variable produit simplement l'allocation d'un espace pour la référence de l'objet.
  \item Les \emph{affectations} (ou assignations) ne reposent pas sur l'envoi de messages.
  		L'\ind{affectation} d'une variable produit une liaison de nom de variable dans le cadre de sa définition.
  \item Les \emph{retours} (ou renvois) ne reposent pas sur l'envoi de messages.
  		Un \ind{retour} ne produit que le retour à l'envoyeur du résultat calculé.
  \item Les \emph{primitives} ne reposent pas sur l'envoi de messages.
  		Elles sont codées au niveau de la \ind{machine virtuelle}.
		\index{primitive}
\end{itemize}

à quelques autres exceptions près, presque tout le reste se déroule véritablement par l'envoi de messages. 
En particulier, la seule façon de mettre à jour une \ind{variable d'instance} d'un autre objet est de lui envoyer un message réclamant le changement de son propre attribut (ou champ) car ces derniers ne sont pas des ``attributs publics'' en \st.
Bien entendu, offrir des méthodes d'\subind{méthode}{accès en lecture} dites accesseurs (\emph{getter}, en anglais, retournant l'état de la variable) et mutateurs ou méthodes d'\subind{méthode}{accès en écriture} (\emph{setter} en anglais, changeant la variable) pour chaque variable d'instance d'un objet n'est pas une bonne méthodologie orientée objet.
Joseph Pelrine annonce aussi à juste titre:
\important{Ne laissez jamais personne d'autre jouer avec vos données~\traduit{Don't let anyone else play with your data.}.}

%=========================================================
\section{La recherche de méthode suit la chaîne d'héritage} 

%\ruleref{lookup}

Qu'arrive-t-il exactement quand un objet reçoit un message?

Le processus est relativement simple:
la classe du receveur cherche la méthode à utiliser pour opérer le message.
Si cette classe n'a pas de méthode, elle demande à sa \ind{super-classe} et remonte ainsi de suite la chaîne d'\ind{héritage}.
Quand la méthode est enfin trouvée, les arguments sont affectés aux paramètres de la méthode et la \ind{machine virtuelle} l'exécute.
\index{méthode!lookup}
\index{méthode!réferencement}

C'est, en essence, aussi simple que cela.
Mais il reste quelques questions auxquelles nous devons prendre soin de répondre:

\begin{itemize}
  \item \emph{Que se passe-t-il lorsque une méthode ne renvoie pas explicitement une valeur?}
  \item \emph{Que se passe-t-il quand une classe réimplémente une méthode d'une super-classe?}
  \item \emph{Quelle différence y a-t-il entre les envois faits à \pvind{self} et ceux faits à \pvind{super}?}
  \item \emph{Que se passe-t-il lorsqu'aucune méthode est trouvée?}
\end{itemize}

Les règles pour la recherche par référencement (en anglais \emph{lookup}) présentées ici sont conceptuelles: des réalisations au sein de la machine virtuelle rusent pour optimiser la vitesse de recherche des méthodes. 
C'est leur travail mais tout est fait pour que vous ne remarquiez jamais qu'elles font quelque chose de différent des règles énoncées.
% Whatever the implementation does, these rules will give you a clear understanding of the semantics of sending messages to \self and \super.

Tout d'abord, penchons-nous sur la stratégie de base de la recherche. Ensuite nous répondrons aux questions.

%---------------------------------------------------------
\subsection{La recherche de méthode}
Supposons la création d'une instance de \ct{EllipseMorph}.
\begin{code}{@TEST | anEllipse |}
anEllipse := EllipseMorph new.
\end{code}
\noindent
Si nous envoyons à cet objet le message \ct{defaultColor}, nous obtenons le résultat \ct{Color yellow}~\footnote{Yellow est la couleur jaune.}:
\begin{code}{@TEST | anEllipse | anEllipse := EllipseMorph new.}
anEllipse defaultColor --> Color yellow
\end{code}
\noindent
La classe \ct{EllipseMorph} implémente \ct{defaultColor}, donc la méthode adéquate est trouvée immédiatement.

\needlines{5}
\begin{method}[defaultColor]{Une méthode implémentée localement. Le commentaire dit: ``retourne la couleur par défaut; le style de remplissage pour le receveur''}
EllipseMorph>>>defaultColor
	"answer the default color/fill style for the receiver"
	^ Color yellow
\end{method}
\cmindex{EllipseMorph}{defaultColor}

A contrario, si nous envoyons le message \ct{openInWorld} à \mbox{\ct{anEllipse},} la méthode n'est pas trouvée immédiatement parce que la classe \ct{EllipseMorph} n'implémente pas \ct{openInWorld}.
La recherche continue plus avant dans la super-classe \mbox{\ct{BorderedMorph},} puis ainsi de suite, jusqu'à ce qu'une méthode \ct{openInWorld} soit trouvée dans la classe \ct{Morph} (voir \figref{openInWorldLookup}).

\begin{method}[openInWorld]{Une méthode héritée. Le commentaire dit: ``Ajoute ce morph dans le monde (world).''}
Morph>>>openInWorld
	"Add this morph to the world."

  self openInWorld: self currentWorld
\end{method}
\cmindex{Morph}{openInWorld}

\begin{figure}[htb]
\begin{center}
	{\includegraphics[width=0.8\textwidth]{openInWorldLookup}}
\caption{Recherche par référencement d'une méthode suivant la hiérarchie d'héritage.\figlabel{openInWorldLookup}}
\end{center}
\end{figure}

%---------------------------------------------------------
\subsection{Renvoyer self}

Remarquez que \ct{EllipseMorph>>>defaultColor} (\mthref{defaultColor}) renvoie explicitement \ct{Color yellow} alors que \ct{Morph>>>openInWorld} (\mthref{openInWorld}) semble ne rien retourner.

En réalité une méthode renvoie \emph{toujours} une valeur\,---\,qui est, bien entendu, un objet.
La réponse peut être explicitement définie par l'utilisation du symbole \ct{^} dans la méthode. Si lors de l'exécution, on atteint la fin de la méthode sans avoir rencontré de \ct{^}, la méthode retournera toujours une valeur: l'objet receveur lui-même.
On dit habituellement que la méthode ``renvoie \self'', parce qu'en
\st la pseudo-variable \self représente le receveur du message. En \ind{Java}, on utilise le mot-clé \ct{this}.
\index{variable!pseudo}
%\index{return}
\index{renvoi}
\index{retour}

Ceci induit le constat suivant: \tmthref{openInWorld} est équivalent à \mbox{\tmthref{openInWorldReturnSelf}:}

\needlines{5}
\begin{method}[openInWorldReturnSelf]{Renvoi explicite de \mbox{\self.} Le dernier commentaire dit: ``Ne faites pas cela à moins d'en être sûr''}
Morph>>>openInWorld
	"Add this morph to the world."

  self openInWorld: self currentWorld.
	^ self		"Don't do this unless you mean it!"
\end{method} % REVOIR CHANGE (allusion à MVC retirée)

Pourquoi écrire \ct{^ self} explicitement n'est pas une bonne chose à faire?
Parce que, quand vous renvoyez explicitement quelque chose, vous communiquez
que vous retournez quelque chose d'importance à l'expéditeur du message.
Dès lors vous spécifiez que vous attendez que l'expéditeur fasse quelque chose de la valeur retournée.
Puisque ce n'est pas le cas ici, il est préférable de ne pas renvoyer explicitement \self.

C'est une convention en \st, ainsi résumé par Kent Beck se référant à la \emph{valeur de retour importante} ``Interesting return value'' \cite{Beck97a}:
\index{Beck, Kent}

\important{Renvoyez une valeur seulement quand votre objet expéditeur en a l'usage~\traduit{Return a value only when you intend for the sender to use the value.}.}

%---------------------------------------------------------
\subsection{Surcharge et extension.}

Si nous revenons à la hiérarchie de classe \ct{EllipseMorph} dans \figref{openInWorldLookup}, nous voyons que les classes \ct{Morph} et \mbox{\ct{EllipseMorph}} implémentent toutes deux \ct{defaultColor}.
En fait, si nous ouvrons un nouvel élément graphique \emph{Morph} (\ct{Morph new openInWorld}), nous constatons que nous obtenons un morph bleu, là où l'ellipse (\ct{EllipseMorph}) est jaune \mbox{(\lct{yellow})} par défaut.
\index{méthode!surcharge}
\index{méthode!extension}
\seeindex{surcharge}{méthode, surcharge}
\seeindex{extension}{méthode, extension}

Nous disons que \ct{EllipseMorph} \emph{surcharge} la méthode \ct{defaultColor} qui hérite de \ct{Morph}.
La méthode héritée n'existe plus du point de vue \ct{anEllipse}.

Parfois nous ne voulons pas surcharger les méthodes héritées, mais plutôt les \emph{étendre} avec de nouvelles fonctionnalités; autrement dit, nous souhaiterions pouvoir invoquer la méthode surchargée \emph{complétée} par la nouvelle fonctionnalité que nous aurons définie dans la sous-classe.
En \st, comme dans beaucoup de langages orientés objet reposant sur l'héritage simple, nous pouvons le faire à l'aide d'un envoi de message à \super.

La méthode \ct{initialize} est l'exemple le plus important de l'application de ce mécanisme.
Quand une nouvelle instance d'une classe est initialisée, il est vital
d'initialiser toutes les variables d'instance héritées.
Cependant, les méthodes \ct{initialize} de chacune des super-classes
de la chaîne d'héritage fournissent déjà la connaissance nécessaire. 
La sous-classe n'a pas à s'occuper d'initialiser les variables d'instance héritées!

Envoyer \ct{super initialize} avant tout autre considération lors de la création d'une méthode d'\ind{initialisation} est une bonne pratique:
\arevoir{}% subpvindex
\subpvindex{super}{\ct{initialize}}

\needlines{6}
\begin{method}[morphinit]{Initialisation de la super-classe. Le commentaire dit: ``initialise l'état du receveur''}
BorderedMorph>>>initialize
	"initialize the state of the receiver"
	super initialize.
	self borderInitialize
\end{method}

\important{Une méthode \ct{initialize} devrait toujours commencer par la ligne \ct{super initialize}.}

%---------------------------------------------------------
\subsection{Envois à self et à super}

Nous avons besoin des \subpvind{super}{envoi}{}s sur \super pour 
réutiliser le comportement hérité qui pourrait sinon être
surchargé.
Cependant, la technique habituelle de composition de méthodes,
héritées ou non, est basée sur l'\subpvind{self}{envoi} à
\self.

Comment l'envoi à \self diffère de celui à \super?
Comme \self, \super représente le receveur du message.
La seule différence est dans la méthode de \subind{méthode}{recherche}.
Au lieu de faire partir la recherche depuis la classe du receveur,
celle-ci démarre dans la super-classe de la méthode dans laquelle
l'envoi à \super se produit. 
\seeindex{recherche!méthode}{méthode, recherche}

Remarquez que \super n'est \emph{pas} la super-classe!
C'est une erreur courante et normale que de le penser.
C'est aussi une erreur de penser que la recherche commence dans la super-classe du receveur.
Nous allons voir précisemment comment cela marche avec l'exemple suivant.

Considérons le message \ct{constructorString}, que nous pouvons envoyer à 
n'importe quel morph:
\begin{code}{@TEST | anEllipse | anEllipse := EllipseMorph new.}
anEllipse constructorString --> '(EllipseMorph newBounds: (0@0 corner: 50@40) color: Color yellow)'
\end{code}
La valeur de retour est une chaîne de caractères qui peut être évaluée pour recréer un morph.

Comment ce résultat est-il exactement obtenu grâce à l'association de \self et de \mbox
{\super{}?}
Pour commencer, 
\ct{anEllipse constructorString} trouvera la méthode \ct{constructorString} dans la 
classe \ct{Morph},
comme vu dans \figref{constructorStringLookup}.

\begin{figure}[htb]
\begin{center}
\ifluluelse
	{\includegraphics[width=\textwidth]{constructorStringLookup}}
	{\includegraphics[width=0.8\textwidth]{constructorStringLookup}}
\caption{Les envois à \self et \super.\figlabel{constructorStringLookup}}
\end{center}
\end{figure}

\needlines{2} %CHANGE
\begin{method}[constructorString]{Un envoi à \self}
Morph>>>constructorString
	^ String streamContents: [:s | self printConstructorOn: s indent: 0]
\end{method}

% \arelire{La méthode \cmind{Morph}{constructorString} envoie
%  \ct{printConstructorOn:indent:} à \self.
% Ce message est \arevoir{recherché} dans la hiérarchie en commençant dans la classe
% \lct{EllipseMorph} et finalement trouvé dans \ct{Morph}.}
% This message is also looked up, starting in the class \lct{EllipseMorph}, and found in \ct{Morph}.
% \arelire{Cette méthode en retour envoie à \self le message
% \lct{printConstructorOn:indent:nodeDict:}, qui, à son tour, envoie
% \ct{fullPrintOn:} à \self.
% Encore une fois, \ct{fullPrintOn:} est \arevoir{recherché} depuis la classe
% \ct{EllipseMorph} et \mthind{BorderedMorph}{fullPrintOn:} 
% est trouvé dans \ct{BorderedMorph} 
% (revoir\figref{constructorStringLookup})} % CHANGE

La méthode \cmind{Morph}{constructorString} envoie le message
\ct{printConstructorOn:indent:} à \self. La méthode correspondante à ce
message est alors recherchée dans la hiérarchie de classes d'abord en
en commençant dans la classe \lct{EllipseMorph} et finalement trouvé dans \ct{Morph}.
Cette méthode en retour envoie à \self le message
\lct{printConstructorOn:indent:nodeDict:}, qui, à son tour, envoie
\ct{fullPrintOn:} à \self. Encore une fois, \ct{fullPrintOn:} est recherché
depuis la classe \ct{EllipseMorph} et \mthind{BorderedMorph}{fullPrintOn:} 
est trouvé dans \ct{BorderedMorph} 
(revoir\figref{constructorStringLookup}) % CHANGE
Ce qui est crucial à observer est le fait qu'un envoi à self provoque une recherche de méthode
qui débute dans la classe du receveur, à savoir la classe de anEllipse.

\important{Un envoi à \self déclenche le départ de la recherche \emph{dynamique} de méthode dans la classe du receveur.}

\needlines{4}
\begin{method}[fullPrintOn]{Combiner l'usage de \super et \self}
BorderedMorph>>>fullPrintOn: aStream
	aStream nextPutAll: '('.
	!\emcode{super fullPrintOn: aStream.}!
	aStream nextPutAll: ') setBorderWidth: '; print: borderWidth;
		nextPutAll: ' borderColor: ' , (self colorString: borderColor)
\end{method}
Maintenant, \ct{BorderedMorph>>>fullPrintOn:} utilise l'envoi à \super pour étendre le comportement \ct{fullPrintOn:} hérité de sa super-classe.
Parce qu'il s'agit d'un envoi à \super, la recherche démarre alors
depuis la super-classe de la classe dans laquelle se produit l'envoi à
\super, autrement dit, dans 
\ct{Morph}.
Nous trouvons ainsi immédiatemment \ct{Morph>>>fullPrintOn:} que nous évaluons.

Notez que la recherche sur \super n'a pas commencé dans la 
super-classe du receveur. Ainsi il en aurait résulté
un départ de la recherche depuis \ct{BorderedMorph}, 
créant alors une boucle infinie!

\important{Un envoi à \super déclenche un départ de recherche \emph{statique} de méthode dans la super-classe de la classe dont la méthode envoie le message à \super.}

Si vous regardez attentivement l'envoi à \super et \figref{constructorStringLookup}, vous réaliserez que les liens à \super sont statiques: tout ce qui importe
est la classe dans laquelle le texte de l'envoi à \super est trouvé.
A contrario, le sens de \self est dynamique: \self représente toujours le
receveur du message courant exécuté. Ce qui signifie que  \emph{tout} message envoyé à \self est recherché en partant de la classe du receveur.

%---------------------------------------------------------
\subsection{MessageNotUnderstood}
%\subsection{Message incompris}

Que se passe-t-il si la méthode que nous cherchons n'est pas trouvée?
\index{message!not understood}
%\index{message!incompris}

Supposons que nous envoyons le message \ct{foo} à une ellipse \ct{anEllipse}.
Tout d'abord, la recherche normale de cette méthode aurait à parcourir
toute la chaîne d'héritage jusqu'à la classe \clsind{Object} (ou
plutôt \clsind{ProtoObject}).
Comme cette méthode n'est pas trouvée, la \ind{machine virtuelle} veillera
à ce que l'object envoie \ct{self doesNotUnderstand: #foo}.
(voir \figref{fooNotFound}.)

\begin{figure}[htb]
\begin{center}
\ifluluelse
	{\includegraphics[width=\textwidth]{fooNotFound}}
	{\includegraphics[width=0.8\textwidth]{fooNotFound}}
\caption{Le message \ct{foo} n'est pas compris (not understood).\figlabel{fooNotFound}}
\end{center}
\end{figure}

Ceci est un envoi dynamique de message tout à fait normal. Ainsi
la recherche recommence depuis la classe \ct{EllipseMorph}, mais
cette fois-ci en cherchant la méthode \ct{doesNotUnderstand:}~\footnote{Le nom du message peut se traduire par: \emph{``ne comprend pas''}.}.
Il apparaît que \lct{Object} implémente \ct{doesNotUnderstand:}.
Cette méthode créera un nouvel objet \ct{MessageNotUnderstood} (en français: Message incompréhensible) capable de démarrer Debugger, le débogueur, dans le contexte actuel de l'exécution.

Pourquoi prenons-nous ce chemin sinueux pour gérer une erreur si évidente?
Parce qu'en faisant ainsi, le développeur dispose de tous les outils pour
agir alternativement grâce à l'interception de ces erreurs.
N'importe qui peut surcharger la méthode \mthind{Object}{doesNotUnderstand:} 
dans une sous-classe de \ct{Object} en étendant ses possibilités en
offrant une façon différente de capturer l'erreur.

En fait, nous nous simplifions la vie en implémentant une 
délégation automatique de messages d'un objet à un autre.
Un objet \ct{Delegator} peut simplement déléguer tous les messages
qu'il ne comprend pas à un autre objet dont la responsabilité est de les gérer ou
de lever une erreur lui-même!

%=========================================================
\section{Les variables partagées}

Maintenant, intéressons-nous à un aspect de \st que nous n'avons pas couvert
par nos cinq règles: les variables \subind{variable}{partagée}{}s.

\st offre trois sortes de variables partagées: (1) les variables \emph{globales}; (2) les \emph{variables de classe} partagées entre les instances et les classes, et (3) les variables partagées parmi un groupe de classes ou \emph{variables de pool}.  Les noms de toutes ces variables partagées commencent par une lettre capitale (majuscule), pour nous informer qu'elles sont partagées entre plusieurs objets.
\index{variable!globale}
\index{classe!variable}
\index{variable!pool}

%---------------------------------------------------------
\subsection{Les variables globales}
En \pharo, toutes les variables globales sont stockées dans un espace de nommage appelé \glbind{Smalltalk} qui est une instance de la classe \clsind{SystemDictionary}.
Les variables globales sont accessibles de partout.
Toute classe est nommée par une variable globale; en plus, quelques variables globales sont utilisées pour nommer des objets spéciaux ou couramment utilisés.  

La variable \glbind{Transcript} nomme une instance de \clsind{TranscriptStream}, un flux de données ou \emph{stream} qui écrit dans une fenêtre à ascenseur (dite aussi \emph{scrollable}).
Le code suivant affiche des informations dans le \ct{Transcript} en passant une ligne.

\begin{code}{}
Transcript show: 'Pharo est extra' ; cr
\end{code} % A REVOIR 'Pharo is fun and powerful'

\noindent
Avant de lancer la commande \menu{do it}, ouvrez un Transcript en 
sélectionnant \menu{World \go Tools \ldots \go Transcript}. % CHANGE

\hint{Écrire dans le Transcript est lent, surtout quand la fenêtre Transcript est ouverte. 
Ainsi, si vous constatez un manque de réactivité de votre système alors que vous êtes en train d'écrire dans le Transcript, pensez à le minimiser (bouton \emph{collapse this window}).}

\subsubsection{D'autres variables globales utiles}

\begin{itemize}
\item
\ct{Smalltalk} est une instance de \ct{SystemDictionary} (Dictionnaire Système) définissant toutes les variables globales\,---\,dont l'objet \ct{Smalltalk} lui-même.   
Les clés de ce dictionnaire sont des symboles nommant les objets globaux dans le code \st.
Ainsi par exemple,
\begin{code}{@TEST}
Smalltalk at: #Boolean --> Boolean
\end{code}
Puisque \ct{Smalltalk} est aussi une variable globale lui-même,
\begin{code}{}
Smalltalk at: #Smalltalk-->a SystemDictionary(lots of globals)}
\end{code} 
et
\begin{code}{@TEST}
(Smalltalk at: #Smalltalk) == Smalltalk --> true
\end{code}

\item \clsind{Sensor} est une instance of \clsind{EventSensor}. Il représente les entrées interactives ou interfaces de saisie (en anglais, \emph{input}) dans \pharo. Par exemple, \ct{Sensor keyboard} retourne le caractère suivant saisi au clavier, et \ct{Sensor leftShiftDown} répond \ct{true} (vrai en booléen) si la touche \emph{shift} gauche est maintenue enfoncée, alors que \ct{Sensor mousePoint} renvoie un \ct{Point} indiquant la position actuelle de la souris.

\item \glbind{World} (Monde en anglais) est une instance de \clsind{PasteUpMorph} représentant l'écran.
\ct{World bounds} retourne un rectangle définissant l'espace tout entier de l'écran; tous les morphs (objet Morph) sur l'écran sont des sous-morphs ou \emph{submorphs} de \ct{World}.
\index{Morphic}

\item \glbind{ActiveHand} est l'instance courante de \clsind{HandMorph}, la représentation graphique du curseur. Les sous-morphs de \ct{ActiveHand} tiennent tout ce qui est glissé par la souris.
\ab{I have never used this, and had to browse the image to see what it is!  What do you use it for?}

\item
\glbind{Undeclared}~\footnote{Non déclaré, en français.} est un autre dictionnaire\,---\,il contient toutes les variables non déclarées.
Si vous écrivez une méthode qui référence une variable non déclarée,
le navigateur de classe (Browser) vous l'annoncera normalement
pour que vous la déclariez, par exemple, en tant que variable globale ou variable d'instance de la classe.
Cependant, si par la suite, vous effacez la déclaration, le code référencera une variable non déclarée. 
Inspecter \ct{Undeclared} peut parfois expliquer des comportements bizarres!

\item
\glbind{SystemOrganization} est une instance de
\clsind{SystemOrganizer}: il enregistre l'organisation des classes en
paquets.  Plus précisement, il catégorise les \emph{noms} des
classes, ainsi
\end{itemize} %CHANGE
\begin{code}{@TEST}
SystemOrganization categoryOfElement: #Magnitude --> #'Kernel-Numbers'
\end{code}
%\end{itemize} % CHANGE dans PBE, c'est \end{itemize} du dessus qui
               % prend le relais

Une pratique courante est de limiter fortement l'usage des variables globales;
il est toujours préférable d'utiliser des variables d'instance de classe ou des variables de classes et de fournir des méthodes de classe pour y accéder.
En effet, si aujourd'hui \pharo devait être reprogrammé à partir de
zéro~\footnote{Le terme anglais est: \emph{from scratch}, signifiant depuis le début.}, la plupart des variables globales qui ne sont pas des classes seraient remplacées par des Singletons.

La technique habituellement employée pour définir une variable globale
est simplement de faire un \menu{do it} sur une affectation d'un identifiant
non déclaré commençant par une majuscule. Dès lors,
l'analyseur syntaxique ou \emph{parser} vous 
la déclarera en tant que variable globale.  
Si vous voulez en définir une de manière programmatique, exécutez
\ct{Smalltalk at: #AGlobalName put: nil}.
Pour l'effacer, exécutez \ct{Smalltalk removeKey: #AGlobalName}.
\glbindex{Smalltalk}

%---------------------------------------------------------
\subsection{Les variables de classe}
\seclabel{classVars}

Nous avons besoin parfois de partager des données entre les instances d'une classe et la classe elle-même.
C'est possible grâce aux \emph{variables de classe}. 
Le terme \subind{classe}{variable} de classe indique que le cycle de vie
de la variable est le même que celui de la classe. Cependant, 
le terme ne véhicule pas l'idée que ces variables sont partagées aussi bien parmi toutes
les instances d'une classe que dans la classe elle-même comme nous pouvons le voir sur \figref{privateSharedVar}.
En fait, \emph{variables partagées} (ou \emph{shared variables}, en anglais) aurait
été un meilleur nom car ce dernier exprime plus clairement leur rôle
tout en pointant le danger de les utiliser, en particulier si elles sont 
sujettes aux modifications.

\begin{figure}[htb]
\begin{center}
\ifluluelse
	{\includegraphics[width=\textwidth]{privateSharedVarColor}}
	{\includegraphics[width=0.7\textwidth]{privateSharedVarColor}}
\caption{Des méthodes d'instance et de classe accédant à différentes
variables.\figlabel{privateSharedVar}}
\end{center}
\end{figure}

Sur \figref{privateSharedVar} nous voyons que \ct{rgb} et \ct{cachedDepth} sont
des variables d'instance de \ct{Color} uniquement accessibles par les 
instances de \clsind{Color}.
Nous remarquons aussi que \ct{superclass}, \ct{subclass}, \ct{methodDict}\ldots \etc, sont des variables d'instance de classe, \ie des variables d'instance  accessibles seulement par \ct{Color} class.

Mais nous pouvons noter quelque chose de nouveau: \ct{ColorNames} et \ct{CachedColormaps} sont des \emph{variables de classe} définies pour \ct{Color}.
La capitalisation du nom de ces variables nous donne un indice sur le fait qu'elles sont partagées.
En fait, non seulement toutes les instances de \ct{Color} peuvent accéder
à ces variables partagées, mais aussi la classe \ct{Color} elle-même, ainsi que \emph{toutes ses sous-classes}.
Les méthodes d'instance et de classe peuvent accéder toutes les deux
à ces variables partagées.

%\begin{figure}
%\begin{center}\includegraphics[width=6cm]{dateToday}\caption{A date is an object that  represents only anumber of days; all the information about month names, day names, etc.\ is shared among all the instances \figlabel{dateToday}}\end{center}.
%\end{figure}

Une \subind{classe}{variable} de classe est déclarée dans le patron de définition de la classe.
Par exemple, la classe \ct{Color} définit un grand nombre de variables de classe pour accélérer la création des couleurs;
sa définition est visible ci-dessous (\clsref{Color}).

\needlines{5} % CHANGE
\begin{classdef}[Color]{Color et ses variables de classe}
Object subclass: #Color 	
        instanceVariableNames: 'rgb cachedDepth cachedBitPattern'
        classVariableNames: 'Black Blue BlueShift Brown CachedColormaps ColorChart ColorNames ComponentMask ComponentMax Cyan DarkGray Gray GrayToIndexMap Green GreenShift HalfComponentMask HighLightBitmaps IndexedColors LightBlue LightBrown LightCyan LightGray LightGreen LightMagenta LightOrange LightRed LightYellow Magenta MaskingMap Orange PaleBlue PaleBuff PaleGreen PaleMagenta PaleOrange PalePeach PaleRed PaleTan PaleYellow PureBlue PureCyan PureGreen PureMagenta PureRed PureYellow RandomStream Red RedShift TranslucentPatterns Transparent VeryDarkGray VeryLightGray VeryPaleRed VeryVeryDarkGray VeryVeryLightGray White Yellow'
        poolDictionaries: '' 	
        category: 'Graphics-Primitives'
\end{classdef}

La variable de classe \cvind{ColorNames} est un tableau contenant le nom des couleurs fréquemment utilisées. Ce tableau est partagé par toutes les instances de \ct{Color} et de sa sous-classe \clsind{TranslucentColor}. 
Elles sont accessibles via les méthodes d'instance et de classe.

% (see \figref{ClassVarAccess2}).

\ct{ColorNames} est initialisée une fois dans \cmind{Color class}{initializeNames}, mais elle est en libre accès depuis les instances de \ct{Color}.
La méthode \cmind{Color}{name} utilise la variable pour trouver
le nom de la couleur.
Il semble en effet inopportun d'ajouter une variable d'instance \ct{name}
à chaque couleur car la plupart des couleurs n'ont pas de noms.
 
\subsubsection{L'initialisation de classe}

La présence de variables de classe soulève une question: comment les initialiser?

Une solution est l'\subind{classe}{initialisation} dite paresseuse (ou \emph{lazy initialization} en anglais).
Cela est possible avec l'introduction d'une méthode d'\subind{méthode}{accès} qui
initialise la variable, durant l'exécution, si celle-ci n'a pas été
encore initialisée.
Ceci nous oblige à utiliser la méthode d'accès tout le temps et à ne jamais
faire appel à la variable de classe directement.
De plus, notons que le coût de l'envoi d'un accesseur et le test d'initialisation sont à prendre en compte.
Ceci va à l'encontre de notre motivation à utiliser une variable de classe, parce qu'en réalité elle n'est plus partagée.

\begin{method}[colorclasscolornames]{\ct{Color class>>>colorNames}}
Color class>>>colorNames	
	ColorNames ifNil: [self initializeNames].
	^ ColorNames
\end{method}	
\cmindex{Color class}{colorNames}

Une autre solution consiste à surcharger la méthode \ct{initialize}.

\needlines{3}
\begin{method}[colorclassinit]{\ct{Color class>>>initialize}}
Color class>>>initialize	
	!\ldots!
	self initializeNames
\end{method}
\cmindex{Color class}{initialize}

\noindent
Si vous adoptez cette solution, vous devez vous rappeler qu'il faut
invoquer la méthode \ct{initialize} après que vous l'ayez définie,
\eg en utilisant \ct{Color initialize}.
Bien que les méthodes \subind{Browser}{côté classe} \ct{initialize} soient exécutées automatiquement lorsque le code est chargé en mémoire,
elles ne sont \emph{pas} exécutées durant leur saisie et leur compilation dans le navigateur Browser ou en phase d'édition et de recompilation.

%---------------------------------------------------------
\subsection{Les variables de pool}
Les variables de \emph{pool}~\footnote{Pool signifie piscine en anglais, ces variables sont dans un même bain!} sont des variables qui sont partagées entre plusieurs classes qui ne sont pas liées par une arborescence d'héritage.
à la base, les variables de pool sont stockées dans des dictionnaires
de pool;  maintenant elles devraient être définies comme variables
de classe dans des classes dédiées (sous-classes de SharedPool).
Notre conseil: évitez-les. Vous n'en aurez besoin qu'en des circonstances
exceptionnelles et spécifiques.
Ici, notre but est de vous expliquer suffisamment les variables de
\subind{variable}{pool} pour comprendre leur fonction quand vous les
rencontrez durant la lecture de code.

Une classe qui accède à une variable de pool doit mentionner le \emph{pool}
dans sa définition de classe.
Par exemple, la classe \clsind{Text} indique qu'elle emploie le dictionnaire
de pool \ct{TextConstants} qui contient toutes les constantes textuelles
telles que 
\glbind{CR} and \glbind{LF}. 
Ce dictionnaire a une clé \ct{#CR} à laquelle est affectée la valeur
\ct{Character cr}, \ie le caractère retour-chariot ou \emph{carriage return}.
\cmindex{Character class}{cr}

\begin{classdef}[textpooldict]{Dictionnaire de pool dans la classe \ct{Text}}
ArrayedCollection subclass: #Text
        instanceVariableNames: 'string runs' 	
        classVariableNames: '' 	
        !\emcode{poolDictionaries: 'TextConstants'}!
        category: 'Collections-Text'
\end{classdef}
   
Ceci permet aux méthodes de la classe \ct{Text} d'accéder aux clés
du dictionnaire \emph{directement} dans le corps de la méthode, \ie 
en utilisant la syntaxe de variable plutôt qu'une recherche explicite
dans le dictionnaire.
Par exemple, nous pouvons écrire la méthode suivante.

\needlines{3}
\begin{method}[texttestcr]{\ct{Text>>>testCR}}
Text>>>testCR 	
      ^ CR == Character cr
\end{method}

Encore une fois, nous recommandons d'éviter d'utiliser
les variables et les dictionnaires de pool.

%=========================================================
\section{Résumé du chapitre}

Le modèle objet de \pharo est à la fois simple et uniforme.
\Mantra et quasiment tout se passe via l'envoi de messages. % REVOIR
                                % 'by sending messages'

\begin{itemize}

\item \Mantra.
  Les entités primitives telles que les entiers sont des objets, ainsi que les classes qui sont des objets comme les autres.

\item Tout objet est instance d'une classe.
Les classes définissent la structure de leurs instances via des
variables d'instance \emph{privées} et leur comportement 
via des méthodes \emph{publiques}. Chaque classe est l'unique
instance de sa méta-classe.
Les variables de classe sont des variables privées partagées par la classe
et toutes les instances de la classe.
Les classes ne peuvent pas accéder directement aux variables d'instance de
leurs instances et les instances ne peuvent pas accéder aux variables de
classe de leur classe.
Des méthodes d'\subind{méthode}{accès} (accesseurs et mutateurs) doivent être
définies si besoin.  

  \item Toute classe a une super-classe.
  La racine de la hiérarchie basée sur l'héritage simple est \ct{ProtoObject}.
	Cependant les classes que vous définissez devrait normalement hériter de la classe \ct{Object} ou de ses sous-classes.
Il n'y a pas d'élément sémantique pour la définition de classes abstraites.
Une classe abstraite est simplement une classe avec au moins une méthode abstraite
\,---\,une dont l'implémentation contient l'expression 
\ct{self subclassResponsibility}.
  Bien que \pharo ne dispose que du principe d'héritage simple, 
il est facile de partager les implémentations de méthodes
en regroupant ces dernières en \emph{traits}.

  \item Tout se passe par envoi de messages. % REVOIR 'by sending
                                             % messages'

	Nous ne faisons pas des ``appels de méthodes'', nous faisons des ``envois de messages''.
        Le receveur choisit alors sa propre méthode pour répondre au message.

  \item La recherche de méthodes suit la chaîne d'héritage;
        Les envois à \self sont dynamiques et la recherche de méthode
        démarre dans le receveur de la classe, alors que
        celles à \super sont statiques et la recherche commence dans la super-classe de la 
        classe dans laquelle l'envoi à \super est écrit.

  \item Il y a trois sortes de variables partagées.
        Les variables globales sont accessibles partout dans le système.
        Les variables de classe sont partagées entre une classe, ses sous-classes et ses instances.
        Les variables de pool sont partagées dans un ensemble de classes particulier.
        Vous devez éviter l'emploi de variables partagées autant que possible.

\end{itemize}

%=========================================================
\ifx\wholebook\relax\else
   \bibliographystyle{jurabib}
   \nobibliography{scg}
   \end{document}
\fi
%=========================================================

%=================================================================
%%% Local Variables:
%%% coding: utf-8
%%% mode: latex
%%% TeX-master: t
%%% TeX-PDF-mode: t
%%% End:

%---------------------------------------------------------

%:Environment
% $Author: oscar $
% $Date: 2007-09-23 11:56:47 +0200 (Sun, 23 Sep 2007) $
% $Translation: martial $
% $Date: Wed Oct 10 17:22:05 CEST 2007 $
% $Revision: 12715 $ 
% translated by Martial.Boniou@ifrance.com start: (Fri, 5 Oct 2007)
% relecture : par Rene Mages  (Wed 9 Jan 2008) de la version 14879
% adaptation pour PBE - martial - Tue Sep  8 21:34:54 CEST 2009 
% relecture : par Rene Mages  (Sun 10 Jan 2010) de la version 30225
% relecture : par Rene Mages  (Sun 8 Aug 2010) de la version 34361
% relecture : par Rene Mages  (Wed 13 Apr 2011) 
% $Author: black$ $Date: 2009-09-06 15:31:38 +0200 (Sun, 06 Sep 2009) $
% $Revision: 28940 $
% $Revision: 29158 $
% 2010-03-05 - Alexandre minor correction (thanks Ralph Boland)
% sync avec la revision: 33691
%==================================================================
\ifx\wholebook\relax\else
% --------------------------------------------
% Lulu:
	\documentclass[a4paper,10pt,twoside]{book}
	\usepackage[
		papersize={6.13in,9.21in},
		hmargin={.75in,.75in},
		vmargin={.75in,1in},
		ignoreheadfoot
	]{geometry}
	\input{../common.tex}
	\pagestyle{headings}
	\setboolean{lulu}{true}
% --------------------------------------------
% A4:
%	\documentclass[a4paper,11pt,twoside]{book}
%	\input{../common.tex}
%	\usepackage{a4wide}
% --------------------------------------------
    \graphicspath{{figures/} {../figures/}}
	\begin{document}
	%\renewcommand{\nnbb}[2]{} % Disable editorial comments
	\sloppy
\fi
%=================================================================
%\newcommand{\debugHandle}{\raisebox{-0.4ex}{\includegraphics[width=1em]{debugHandle}}}

%=================================================================
\chapter{\titreEnvironment}
\chalabel{env}

% cf. Email 2009-10-21 \hw dans PharoBook/Environment/Environment.tex

% martial: global index
%\seeindex{souris!bouton jaune}{bouton jaune} % REVOIR
%\seeindex{souris!bouton rouge}{bouton rouge} % REVOIR
%\seeindex{souris!bouton bleu}{bouton bleu}   % REVOIR

L'objectif de ce chapitre est de vous montrer comment développer des programmes 
dans l'environnement de programmation de \pharo.
Vous avez déjà vu comment définir des méthodes et des classes
en utilisant le navigateur de classes; ce chapitre
vous présentera plus de caractéristiques du Browser ainsi que
d'autres navigateurs. % CHANGE

Bien entendu, vous pouvez occasionnellement rencontrer
des situations dans lesquelles votre programme ne marche pas comme voulu.
\pharo a un excellent débogueur, mais comme la plupart des outils puissants, il peut s'avérer déroutant au début.
Nous vous en parlerons au travers de sessions de débogages et vous
montrerons certaines de ses possibilités.

Lorsque vous programmez, vous le faites dans un monde d'objets vivants et
non dans un monde de programmes textuels statiques; c'est 
une des particularités uniques de Smalltalk.
Elle permet d'obtenir une réponse très rapide de vos programmes et vous rend plus productif. 
Il y a deux outils vous permettant l'observation et aussi la modification de ces objets 
vivants: l'\emph{Inspector} (ou inspecteur) et l'\emph{Explorer} (ou explorateur).

La programmation dans un monde d'objets vivants plutôt qu'avec des fichiers et un éditeur 
de texte vous oblige à agir explicitement pour exporter votre programme depuis l'image \st.

La technique traditionnelle, aussi supportée par tous les dialectes \st consiste à créer un 
fichier d'exportation \emph{fileout} ou une archive d'échange dit \changeset. Il s'agit 
principalement de fichiers textes encodés pouvant être importés dans un autre système.
Une technique plus récente de \pharo est le chargement de code dans un dépôt de versions sur un serveur.
Elle est plus efficace surtout en travail coopératif et est rendue possible via un outil nommé \ind{Monticello}.
%\seeindex{change set}{file, filing out}
\seeindex{change set}{fichier, exportation}
%\index{file!filing out}
\index{fichier!exportation}

Finalement, en travaillant, vous pouvez trouver un \emph{bug} (dit aussi bogue) dans \pharo;
nous vous expliquerons aussi comment reporter les bugs
et comment soumettre les corrections de bugs ou \emph{bug fixes}.
\ab{Or I would, if I knew how.   We should do this, or remove the paragraph.}

%=========================================================
\section{Une vue générale}
%Overview
\seclabel{overview}

Smalltalk et les interfaces graphiques modernes ont été développées ensemble.
Bien avant la première sortie publique de Smalltalk en 1983, Smalltalk
avait un environnement de développement graphique écrit lui-même en Smalltalk et
tout le développement est intégré à cet environnement.
Commençons par jeter un coup d'\oe il sur les principaux outils de
\pharo. % CHANGE

\begin{itemize}
	\item Le {\menu{Browser}} ou \emph{navigateur de classes} est l'outil de développement central.
Vous l'utiliserez pour créer, définir et organiser vos classes et vos méthodes. Avec lui, vous pourrez aussi naviguer dans toutes les classes de la bibliothèque de classes: contrairement aux autres environnements où le code source est réparti dans des fichiers séparés, en \st toutes les classes et méthodes sont contenues dans l'image.
	\index{Browser}
	\index{navigateur de classes}
    \seeindex{Browser}{navigateur de classes}

	\item L'outil {\menu{Message Names}} sert à voir toutes les méthodes ayant un sélecteur (noms de messages sans argument) spécifique ou dont le sélecteur contient une certaine sous-chaîne de caractères.
	\index{Message Name Finder}
	
	\item Le {\menu{Method Finder}} vous permet aussi de trouver des méthodes, soit selon leur \emph{comportement}, soit en fonction de leur nom.
	\index{Method Finder}
	
	\item Le {\menu{Monticello Browser}} est le point de départ pour le chargement ou la sauvegarde de code via des paquetages \ind{Monticello} dit aussi \emph{packages}.
	
	\item Le {\menu{Process Browser} offre une vue sur l'ensemble des processus (threads) exécutés dans \st.}
	\index{Process Browser}
	
	\item Le {\menu{Test Runner}} permet de lancer et de déboguer les tests unitaires \SUnit. Il est décrit dans \charef{SUnit}.
	\index{Test Runner}
	\index{SUnit}
	
	\item Le {\menu{Transcript}} est une fenêtre sur le flux de données sortant de \glbind{Transcript}. Il est utile pour écrire des fichiers-journaux ou \emph{log} et a déjà été étudié dans \secref{transcript}.
	
	\item Le {\menu{Workspace}} ou \emph{espace de travail} est une fenêtre dans laquelle vous pouvez entrer des commandes.  
	Il peut être utilisé dans plusieurs buts mais il l'est plus généralement pour taper des expressions Smalltalk et les exécuter avec 
\menu{do it}~\footnote{En anglais, \emph{do it} correspond à l'exclamation ``fais-le!''.}. L'utilisation de \ind{Workspace} a déjà été vu dans \secref{transcript}.
\end{itemize}

L'outil \menu{Debugger} a un rôle évident, mais vous découvrirez qu'il a une place plus centrale comparé aux débogueurs d'autres langages de programmation
% en comparaison des débogueurs dans d'autres langages de programmation 
car en \st vous pouvez \emph{programmer} dans l'outil \ind{Debugger}.  Il n'est pas lancé depuis un menu; il apparaît normalement en situation d'erreur,
en tapant \short{\textbf{.}} pour interrompre un processus lancé ou
encore en insérant une expression \ct{self halt} dans le code.
\index{processus!interruption}


%=========================================================
\section{Le Browser} % CHANGE
\seclabel{browser} % (fold)
%apb: what does the fold comment mean?

De nombreux navigateurs de classes ou \emph{browsers} ont été
développés pour \st durant des années. \pharo simplifie l'histoire
en proposant un unique navigateur disposant de multiples vues,
le \ind{Browser}. 
\Figref{SystemBrowser0} présente le Browser tel qu'il apparaît
lorsque vous l'ouvrez pour la première fois\footnote{Rappelez-vous que si votre navigateur ne ressemble pas à 
celui présenté sur \figref{SystemBrowser0}, vous pourriez avoir besoin 
de changer le navigateur par défaut (voir \faqref{packagebrowser}).}.
% REVOIR figref SystemBrowser0  au lieu  de classbrowser

\begin{figure}[htbp]
   \centering
   \ifluluelse
	 {\includegraphics[width=\textwidth]{SystemBrowser0} }
	 {\includegraphics[width=0.7\textwidth]{SystemBrowser0} }
   \caption{Le navigateur de classes.}
   \figlabel{SystemBrowser0}
\end{figure}

Les quatre petits panneaux en haut du Browser représentent la vue
hiérarchique des méthodes dans le système de la même
manière que le \textit{File Viewer} de \ind{NeXTstep} et le
\textit{Finder} de Mac OS X fournissent une vue en colonnes 
des fichiers du disque.

Le premier panneau de gauche liste les \emph{paquetages} de
classes ou, en anglais, \emph{packages};
sélectionnez-en une (disons \scat{Kernel}) et alors le
panneau immédiatemment à droite affichera toutes les classes incluses
dans ce paquetage. % CHANGE

\begin{figure}[htbp]
   \centering
   \ifluluelse
	   {\includegraphics[width=\textwidth]{SystemBrowser1} }
	   {\includegraphics[width=0.7\textwidth]{SystemBrowser1} }
   \caption{Le Browser avec la classe \ct{Model} sélectionnée.}
   \figlabel{SystemBrowserModel}
\end{figure}

De même, si vous sélectionnez une des classes de ce second panneau,
disons \menu{Model} (voir  \figref{SystemBrowserModel}), le troisième
panneau vous affichera tous les \emph{protocoles} définis pour cette
classe ainsi qu'un protocole virtuel \mbox{\prot{-{}-all-{}-}} (désignant l'ensemble 
des méthodes). Ce dernier est sélectionné par défaut. 
Les protocoles sont une façon de catégoriser les méthodes;
ils rendent la recherche des méthodes plus facile et détaillent
le comportement d'une classe en le découpant en petites divisions 
cohérentes.
% they make it easier to find and think about the behaviour of a class by breaking it up into smaller, conceptually coherent pieces.  
Le quatrième panneau montre les noms de toutes les méthodes définies dans le protocole sélectionné.
Si vous sélectionnez enfin un nom de méthode, le code source de la 
méthode correspondante apparaît dans le grand panneau inférieur
du navigateur. Là, vous pouvez voir, éditer et sauvegarder la version éditée.
Si vous sélectionnez la classe
\mbox{\menu{Model},} le protocole \mbox{\protind{dependents}} et la méthode 
\mbox{\menu{myDependents},} 
le navigateur devrait ressembler à \figref{SystemBrowserMyDependents}.
\protindex{all}
\cmindex{Model}{myDependents}

\begin{figure}[htbp]
   \centering
   \ifluluelse
	   {\includegraphics[width=\textwidth]{SystemBrowserMyDependents}}
	   {\includegraphics[width=0.7\textwidth]{SystemBrowserMyDependents}}
   \caption{Le Browser affichant la méthode \ct{myDependents} de la classe \ct{Model}.
   \figlabel{SystemBrowserMyDependents}}
\end{figure}

Contrairement aux répertoires du \emph{Finder} de Mac OS X, les quatre panneaux 
supérieurs ne sont aucunement égaux.
Là où les classes et les méthodes font partie du langage \st,
les paquetages et les protocoles ne sont que des commodités
introduites par le navigateur pour limiter la quantité d'information que chaque panneau pourrait présenter.
Par exemple, s'il n'y avait pas de protocoles, le navigateur devrait afficher
la liste de toutes les méthodes dans la classe choisie; pour la plupart des classes, 
cette liste sera trop importante pour être parcourue aisément.
\index{Mac OS X Finder}

De ce fait, la façon dont vous créez un nouveau paquetage ou
un nouveau protocole est différent de la manière avec laquelle
vous créez une nouvelle classe ou une nouvelle méthode. 
Pour créer un nouveau paquetage, sélectionnez 
 \menu{new package} dans le menu contextuel accessible 
en \actclickant dans le  panneau des paquetages; 
pour créer un nouveau protocole, sélectionnez \menu{new protocol} via 
le menu accessible en \actclickant dans le panneau des protocoles.
Entrez le nom de la nouvelle entité (paquetage ou protocole) dans
la zone de saisie, et voilà! 
% dialog
Un paquetage ou un protocole, ça n'est qu'un nom et son contenu.
\index{paquetage!création}

\begin{figure}[htbp]
   \centering
   \ifluluelse
	   {\includegraphics[width=\textwidth]{SystemBrowserClassCreation}}
	   {\includegraphics[width=0.7\textwidth]{SystemBrowserClassCreation}}
   \caption{Le Browser montrant le patron de création de classe.
   \figlabel{SystemBrowserClassCreation}}
\end{figure}

À l'opposé, créer une classe ou une méthode nouvelle nécessite 
l'écriture de code \st.
Si vous \clickz sur le paquetage actuellement sélectionné dans le
panneau de gauche, le panneau inférieur affichera un patron de création de classe
(voir \figref{SystemBrowserClassCreation}).
Vous créez une nouvelle classe en éditant ce patron ou \emph{template}. 
Pour ce faire, remplacez \ct{Object} par le nom de la classe existante
que vous voulez dériver, puis remplacez \ct{NameOfSubclass} par le nom
que vous avez choisi pour votre nouvelle classe (sous-classe de la première) 
et enfin, remplissez la liste des noms de variables d'instance si vous en connaissez.  
La catégorie pour la nouvelle classe est par défaut la
catégorie du paquetage actuellement sélectionné\footnote{Rappelez-vous que paquetages et catégories ne sont pas exactement la même chose. 
Nous verrons la relation qui existe entre eux dans \secref{packages}.}, 
mais vous pouvez à loisir la changer si vous voulez.
Si vous avez déjà la classe à dériver sélectionnée dans 
le Browser, vous pouvez obtenir le même patron avec une initialisation
quelque peu différente en \actclickant dans le panneau des classes et en sélectionnant 
\menu{class templates \ldots \go subclass template}.
Vous pouvez aussi éditer simplement la définition de la classe existante en changeant 
le nom de la classe en quelque chose d'autre.
Dans tous les cas, à chaque fois que vous acceptez la nouvelle définition, la nouvelle classe 
(celle dont le nom est précédé par \ct{#}) est créée (ainsi que sa méta-classe associée).  
Créer une classe crée aussi une variable globale référençant
la classe. En fait, l'existence de celle-ci vous permet de vous 
référer à toutes les classes existantes en utilisant leur nom.
\index{classe!création}
\seeindex{classe!création}{Browser, définir une classe}
\index{Browser!définir une classe}
\seeindex{navigateur de classes!définition d'une classe}{Browser, définir une classe}

Voyez-vous pourquoi le nom d'une nouvelle classe doit apparaître
comme un \clsind{Symbol} (\ie préfixé avec \ct{#}) dans le
patron de création de classe, mais qu'après la création
de classe, le code peut s'y référer en utilisant
son nom comme identifiant (\ie sans le \ct{#}).

Le processus de création d'une nouvelle méthode
est similaire. Premièrement sélectionnez la classe dans laquelle vous
voulez que la méthode apparaisse, puis sélectionnez un protocole.
Le navigateur affichera un patron de création de méthode que
vous pouvez remplir et éditer, comme indiqué par
\figref{SystemBrowserMethodTemplate}.
\index{méthode!création}
\seeindex{méthode!création}{Browser, définir une méthode}
\index{Browser!définir une méthode}
\seeindex{navigateur de classes!définition d'une méthode}{Browser, définir une méthode}

\begin{figure}[htbp]
   \centering
   \ifluluelse
	   {\includegraphics [width=\textwidth]{SystemBrowserMethodTemplate}}
	   {\includegraphics[width=.7\textwidth]{SystemBrowserMethodTemplate}}
   \caption{Le Browser montrant le patron de création de méthode.
   \figlabel{SystemBrowserMethodTemplate}}
\end{figure}

%---------------------------------------------------------
\subsection{Naviguer dans l'espace de code} % CHANGE
\seclabel{ButtonBar}

Le navigateur de classes fournit plusieurs outils pour l'exploration et 
l'analyse de code. 
Ces outils sont accessibles en \actclickant dans divers menus
contextuels ou (pour les outils les plus communs) via des
raccourcis-clavier.

\subsubsection{Ouvrir une nouvelle fenêtre de Browser}
\seclabel{browsing}

Vous aurez besoin parfois d'ouvrir de multiples navigateurs de classes.
Lorsque vous écrivez du code, vous aurez presque toujours besoin d'au moins deux
fenêtres: une pour la méthode que vous éditez et une pour naviguer
dans le reste du système pour y voir ce dont vous aurez besoin pour la
méthode éditée dans la première. % CHANGE
Vous pouvez ouvrir un Browser sur une classe en sélectionnant son nom et en
utilisant le raccourci-clavier \short{b} \ind{raccourci-clavier}.
\index{Browser!browse} % REVOIR !bouton!
\seeindex{navigateur de classes!browse}{Browser, browse} % CHANGE
\index{raccourci-clavier!browse it}

\dothis{Essayez ceci: dans un espace de travail ou Workspace, saisissez le nom d'une 
classe (par exemple, \ct{Morph}), sélectionnez-le et pressez \short{b}. Cette astuce 
est souvent utile; elle marche depuis n'importe quelle fenêtre de texte.}

\subsubsection{\Senders et \implementors d'un message}
\seclabel{sendersImplementors}

\index{Browser!senders}%\index{Browser!bouton!senders}

\Actclick sur \menu{browse \ldots \go senders (n)} dans le
panneau des méthodes vous renvoie une liste de toutes les méthodes
pouvant utiliser la méthode sélectionnée. En prenant le Browser
ouvert sur \ct{Morph}, \clickz sur la méthode  \mthind{Morph}{drawOn:}
dans le panneau des méthodes; le corps de \ct{drawOn:} s'affiche dans
la partie inférieure du navigateur.
Si vous sélectionnez \menu{senders (n)} (voir \figref{SendersOfDrawOn}), un 
menu apparaîtra avec \ct{drawOn:} comme premier élement de
la pile,  suivi de tous les messages que 
\ct{drawOn:} envoie (voir \figref{SendersOfDrawOn2}). %CHANGE
Sélectionner un message dans ce menu ouvrira un navigateur avec la
liste de toutes les méthodes dans l'image qui envoie le message
choisi (voir \figref{CanvasDraw}).

% \caption{Un navigateur de classes ouvert sur la classe \ct{ScaleMorph}. Notez la barre horizontale de boutons en son centre; nous appuyons ici sur le bouton \button{senders}.
%\label{fig:SendersOfCheckEvent}}

%\begin{figure}[htb]
%\begin{minipage}[b]{0.74\textwidth}
%\centerline {\includegraphics[width=\textwidth]{SendersOfDrawOn}}
%\caption{The \menu{senders (n)} menu item.\figlabel{SendersOfDrawOn}}
%\end{minipage}
%\hfill
%\begin{minipage}[b]{0.24\textwidth}
%\centerline {\includegraphics[width=\textwidth]{SendersOfDrawOn2}}
%\caption{Choose senders of which message.\figlabel{SendersOfDrawOn2}}
%\end{minipage}
%\end{figure}

\begin{figure}[htb]
\centerline {\includegraphics[width=\textwidth]{SendersOfDrawOn}}
\caption{L'élement de menu \menu{senders (n)}.\figlabel{SendersOfDrawOn}}
 \end{figure}

\begin{figure}[htb]
\centerline {\includegraphics[width=0.4\textwidth]{SendersOfDrawOn2}}
\caption{Choisir un message dans la liste pour avoir ses \emph{senders}.\figlabel{SendersOfDrawOn2}}
\end{figure}

%\index{Browser!bouton!implementors}
%Le bouton \button{implementors} fonctionne de la même manière mais,
% au lieu de renvoyer une liste de \senders d'un message (ou méthodes-envoyeuses), il sort toutes les
% classes qui implémentent une méthode avec le même sélecteur.
% Pour le voir, sélectionnez \lct{drawOn:} dans le panneau des méthodes
% puis affichez le navigateur ``implementors of drawOn:'', 
% soit en utilisant le bouton \button{implementors}, soit via le 
% \ind{bouton jaune}, soit encore en tapant simplement \short{m} (pour {i\textbf{m}ple\textbf{m}entors}) avec la méthode \menu{drawOn:} sélectionnée dans le panneau des méthodes. 
% Vous devriez avoir une fenêtre à ascenseur montrant une liste des 96
% classes implémentant une méthode \ct{drawOn:}.
% Il n'y a rien de surprenant à ce qu'autant de classes implémentent cette
% méthode: \ct{drawOn:} est le message compris par tout objet apte à se
% dessiner lui-même sur l'écran.
%% REVOIR ? il y a du 'senders' implementators' 'emettrices'...
%\arevoir{Essayez de naviguer dans la liste des \senders du message \ct{drawOn:}; nous nous trouvons face à 63 méthodes émettrices. Vous pouvez aussi ouvrir
%un navigateur d'implementors chaque fois que vous sélectionnez un message
% (en incluant les arguments s'il s'agit d'un message à mots-clés) 
%\arevoir{et que vous appuyez sur \short{m}}.}

Le ``n'' dans \menu{senders (n)} vous informe que le
raccourci-clavier pour trouver les \senders (\ie les méthodes émettrices) d'un message
est \short{n}. Cette commande fonctionne depuis \emph{n'importe
quelle} fenêtre de texte. % CHANGE text window

\dothis{Sélectionnez le texte ``drawOn:'' dans le panneau de
    code et pressez \short{n} pour faire apparaître immédiatement les
     \senders de \ct{drawOn:}.}

\begin{figure}[htbp]
	\begin{center}
   \ifluluelse
		{\includegraphics[width=\textwidth]{CanvasDraw}}
		{\includegraphics[width=0.7\textwidth]{CanvasDraw}}
	\end{center}
	\caption{Le navigateur Senders Browser montrant que la méthode \ct{Canvas>>>>draw} envoie le message \ct{drawOn:} à son argument.	\figlabel{CanvasDraw}}
\end{figure}

%% REVOIR martial: c'est totalement faux dans l'image Core actuelle
%% rene : c'est vrai avec l'image https://gforge.inria.fr/frs/download.php/27023/PBE-1.0.zip
% \arevoir{Si vous regardez bien les \senders de
%  \ct{drawOn:} dans \ct{AtomMorph>>>drawOn:}, vous verrez qu'il s'agit
%  d'un \subpvind{super}{envoi}{} à \super. Nous savons ainsi que la
%  méthode qui sera exécutée est dans la superclasse de \ct{AtomMorph}.
% Quelle est cette classe? \Actclickz sur \menu{browse~\go~hierarchy~implementors} et vous 
% verrez qu'il s'agit de \ct{EllipseMorph}.}
% \index{Browser!bouton!hierarchy} % CHANGE Browser!bouton!hierarchy
% \seeindex{navigateur de classes!hiérarchie}{Browser, hierarchy} % CHANGE

Si vous regardez bien les \senders de \ct{drawOn:} dans \mbox{\ct{AtomMorph>>>drawOn:},} vous verrez que la méthode implémente un \subpvind{super}{envoi}{} à \super. Nous savons ainsi que la méthode qui sera exécutée est dans la superclasse de \ct{AtomMorph}. Quelle est cette classe? \Actclickz sur \menu{browse~\go~hierarchy~implementors} et vous verrez qu'il s'agit de \ct{EllipseMorph}.
\index{Browser!bouton!hierarchy} % CHANGE Browser!bouton!hierarchy
\seeindex{navigateur de classes!hiérarchie}{Browser, hierarchy} % CHANGE

Maintenant observons le sixième \emph{sender} de la liste,
\ct{Canvas>>>draw}, comme le montre \figref{CanvasDraw}.
Vous pouvez voir que cette méthode envoie \ct{drawOn:} à n'importe quel objet 
passé en argument; ce peut être une instance de n'importe quelle classe.
L'analyse du flux de données peut nous aider à mettre la main sur la classe du 
receveur de certains messages, mais de manière générale, il n'a pas de moyen simple 
pour que le navigateur sache quelle méthode sera exécutée à l'envoi d'un message.
%Dataflow analysis can help figure out the class of the receiver of some messages, but in general, there is no simple way for the browser to know which message-sends might cause which methods to be executed.

C'est pourquoi, le navigateur de ``\senders'' (\ie le Browser des
méthodes émettrices) nous montre exactement ce que son nom
suggère: tous  les \senders d'un message relatifs à un sélecteur donné. 
%CHANGE martial: la phrase trop lourde 
% rene: tous les "senders" d'un message relatifs à un selecteur donné
Ce navigateur devient grandement indispensable quand vous avez besoin
de comprendre le \emph{rôle} d'une méthode: il vous permet de
naviguer rapidement à travers les exemples d'usage. % CHANGE
%The \button{senders} button is nevertheless extremely useful when you need to understand how you can \emph{use} a method: it lets you navigate quickly through example uses.  
Puisque toutes les méthodes avec un même sélecteur devraient être utilisées de la même manière, toutes les utilisations d'un message donné devraient être semblables.
%\index{Browser!bouton!senders}
\index{Browser!senders} % REVOIR navigateur de classes!senders >
                        % Browser!senders
\seeindex{Senders Browser}{Browser!senders}
\seeindex{navigateur de classes!méthodes émettrices}{Browser, senders}
% CHANGE TEMP
\seeindex{navigateur de classes!senders}{Browser, senders} % CHANGE

\index{Browser!implementors}
\seeindex{Implementors Browser}{Browser!implementors}
\seeindex{navigateur de classes!classes contenantes}{Browser, implementors} % CHANGE TEMP
\seeindex{navigateur de classes!implementors}{Browser, implementors} % CHANGE

L'Implementors Browser fonctionne de même mais, au lieu de
lister les \senders d'un message, il affiche toutes les classes
contenantes ou \implementors, \ie les classes qui implémentent une
méthode avec le même selecteur
% martial: ajout de 'que celui sélectionné' plus clair!?
que celui sélectionné.
Sélectionnez, par exemple, \lct{drawOn:} dans le panneau des méthodes
et sélectionne \menu{browse \go implementors (m)} (ou selectionnez le
texte ``drawOn:'' dans la zone inférieure du code et pressez 
\short{m}).
Vous devriez voir une fenêtre listant des méthodes montrant ainsi la
liste déroulante des \arevoir{\drawOnImplNumber} % 90 dans l'original
% rene : ni 90 ni 63 mais 77
% rene : il me semble préférable de s'éloigner un peu de l'original et
%        ne plus préciser le nombre exact
% martial: j'ai mis une variable globale (ligne 761 de common.tex)
classes qui implémentent une méthode \ct{drawOn:}.
Ceci ne devrait pas être si surprenant qu'il y ait tant de classes
implémentant cette méthode: \ct{drawOn:} est le message qui est
compris par chaque objet capable de se représenter graphiquement à
l'écran.

\subsubsection{Les versions d'une méthode}
\seclabel{versions}

Quand vous sauvegardez une nouvelle \subind{méthode}{version} d'une méthode,
l'ancienne version n'est pas perdue. \pharo garde toutes les versions passées et vous permet 
de comparer les différentes versions entre elles et de revenir (en anglais, ``revert'') à une ancienne version.
\begin{figure}[btp]
   \centering
	   {\includegraphics[width=\textwidth]{Versions} }
   \caption{\arelire{Le \ind{Versions Browser} montre deux versions de la
     méthode \ct{TheWorldMenul>>>buildWorldMenu:}.}}
   \figlabel{buildWorldMenuVersions}
\end{figure} % CHANGE

%\index{Browser!bouton!versions}
% Rene n'arrive pas à voir les deux versions de la methode buildWorldMenu
\index{Browser!versions}
\seeindex{Versions Browser}{Browser, versions} % CHANGE
\arelire{L'élement de menu \menu{browse \go versions (v)}  donne accès aux modifications successives
effectuées sur la méthode sélectionnée.
Dans \figref{buildWorldMenuVersions} nous pouvons voir deux versions de la
méthode \ct{buildWorldMenu:}.}


\index{System Browser!bouton!versions}

Le panneau supérieur affiche une ligne pour chaque version de la méthode
incluant les initiales du programmeur qui l'a écrite, la date et l'heure de
sauvegarde, les noms de la classe et de la méthode et le protocole dans 
lequel elle est définie.
La version courante (active) est au sommet de la liste; quelle que soit la version
sélectionnée affichée dans le panneau inférieur.
%Si le bouton ou \emph{checkbox} \menu{diffs} est sélectionné, 
%comme c'est le cas dans \figref{mouseUpVersions}, 
%les différences entre la version sélectionnée et celle qui la précède
%immédiatemment sont affichées.
Les boutons offrent aussi l'affichage des différences entre la méthode sélectionnée et la version courante et la possibilité de revenir à la version choisie.  
%Le bouton \menu{prettyDiffs} est utile s'il y a eu changement dans la mise en
%pages: il affiche en mode \emph{pretty-print} (affichage élégant) 
%à la fois les versions antérieures et choisies de façon à
%ce que les différences liées au formatage ne soient pas prises en compte.

Le \ind{Versions Browser} existe pour que vous ne vous inquiétez jamais
de la préservation de code que vous pensiez ne plus avoir besoin: effacez-le simplement. 
Si vous vous rendez compte que vous en avez \emph{vraiment} besoin, 
vous pouvez toujours revenir à l'ancienne version ou copier le morceau de 
code utile de la version antérieure pour le coller dans une autre méthode.

Ayez pour habitude d'utiliser les versions; ``commenter'' le code qui n'est 
plus utile n'est pas une bonne pratique car ça rend le code courant plus difficile à lire. 
Les Smalltalkiens~\footnote{En anglais, nous les appelons
 \emph{Smalltalkers}.} accordent une extrême importance à la lisibilité du code.

\hint{Qu'en est-il du cas où vous décidez de revenir à une méthode que vous
avez entièrement effacée?  
Vous pouvez trouver l'effacement dans un \changeset
dans lequel vous pouvez demander à visiter les versions 
en \actclickant{}. % CHANGE
Le navigateur de \changeset est décrit dans
\secref{env:changeSet}}

\subsubsection{Les surcharges de méthodes}
\seclabel{overriding}

Le Inheritance Browser est un navigateur spécialisé
affichant toutes les méthodes surchargées par la méthode affichée.
Pour le voir à l'action, sélectionnez la méthode
\cmind{ImageMorph}{drawOn:} dans le Browser.
Remarquez les icônes triangulaires juxtant le nom des méthodes (voir \figref{OBinheritanceBrowser}).
Le triangle pointant vers le haut vous indique que 
\ct{ImageMorph>>>drawOn:} surcharge une méthode héritée
 (\ie \ct{Morph>>>drawOn:}) et triangle pointant vers le bas vous
 indique que cette méthode est surchargée dans ses sous-classes (vous
 pouvez aussi \click sur les icônes pour naviguer vers ces méthodes).
Sélectionnez maintenant \menu{browse \go inheritance}.
L'Inheritance Browser vous montre la hiérarchie de méthodes
surchargées (voir \figref{OBinheritanceBrowser}).

\begin{figure}[tbp]
	\begin{center}
   \ifluluelse
		{\includegraphics[width=\textwidth]{OBInheritanceOverriding}}
		{\includegraphics[width=0.7\textwidth]{OBInheritanceOverriding}}
	\end{center}
	\caption{\ct{ImageMorph>>>drawOn:} et les méthodes qu'il
      surcharge. \arelire{Les méthodes apparentées ou \emph{siblings} des
        méthodes sélectionnées sont visibles dans les listes déroulantes .}} % CHANGE
	\figlabel{OBinheritanceBrowser}
\end{figure}

\subsubsection{La vue hiérarchique}
\seclabel{hierarchy}

Par défaut, le navigateur présente une liste de paquetages dans son
panneau supérieur gauche.
Cependant, il est possible de changer le contenu de ce panneau pour
avoir une vue hiérarchique des classes. Pour cela, sélectionner tout
simplement une classe de votre choix, disons \ct{ImageMorph} et \click
sur le bouton \button{hier.}.
Vous verrez alors dans le panneau le plus à gauche une hiérarchie de
classes affichant toutes les super-classes et sous-classes de la
classe sélectionnée.
Le second panneau liste les paquetages implémentant les
méthodes de la classe sélectionnée. % CHANGE à vérifier
%The \button{hierarchy} button opens a \ind{hierarchy browser} on the current class; this
%browser can also be opened by using the \menu{browse hierarchy} menu item in the class pane.
%The hierarchy browser is similar to the browser, but instead of displaying the categories and the classes in each category, it shows a single list of classes, indented to represent inheritance.
%The category of the selected class is displayed in the small annotation pane at the top of the browser.
Sur \figref{OBinheritanceBrowser}, la vue hiérarchique dans le navigateur 
% rene : s /hierarchyBrowser/OBinheritanceBrowser/
montre que \clsind{ImageMorph} est la super-classe directe de
\mbox{\clsind{Morph}.}
\index{Browser!bouton!hierarchy} % REVOIR

\begin{figure}[btp]
	\begin{center}
   \ifluluelse
		{\includegraphics[width=\textwidth]{hierarchyBrowser}}
		{\includegraphics[width=0.7\textwidth]{hierarchyBrowser}}
	\end{center}
	\caption{Une vue hiérarchique de \ct{ImageMorph}.}
	\figlabel{OBinheritanceBrowser}
\end{figure}

\subsubsection{Trouver les références aux variables}
\seclabel{variables}

\index{Browser!variables}
En \actclickant sur une classe dans le panneau de classes du
navigateur, puis en sélectionnant
\menu{browse \go chase variables}~\footnote{\emph{Chase} signifie
  ``poursuivre'' en anglais.}, vous pouvez trouver 
où une certaine variable\,---\,d'instance ou de classe\,---\,est
utilisée.
Vous naviguez au travers des méthodes d'accès de toutes les variables
d'instance ou de classe via ce \emph{navigateur de poursuite} et, en
retour, visitez les méthodes qui envoyent ces accesseurs, et
ainsi de suite (voir \figref{chasingBrowser}).

\begin{figure}[btp]
	\begin{center}
	\ifluluelse
		{\includegraphics[width=\textwidth]{chasingBrowser}}
		{\includegraphics[width=0.7\textwidth]{chasingBrowser}}
	\end{center}
	\caption{Un navigateur de poursuite ouvert sur \ct{Morph}.}
	\figlabel{chasingBrowser}
\end{figure}

% Le menu permet aussi d'afficher le jeu 
% %subset 
% des références aux variables d'instance qui affecte la variable choisie
% par \menu{inst var defs}.
% Une fois que vous avez cliqué sur le bouton ou que vous avez choisi une
% des propositions de menu, un menu flottant s'affichera, vous invitant ainsi
% à sélectionner une variable parmi toutes les variables définies et
% héritées dans la classe courante.
% La liste suit l'ordre d'héritage; il peut d'ailleurs être utile d'afficher
% cette liste à chaque fois vous avez besoin de vous remémorer le nom d'une
% variable d'instance. Si vous cliquer en dehors de la liste, cette dernière
% disparaîtra sans avoir affiché le navigateur de variable.


\subsubsection{Le panneau de code}
% rene propose Le code source
\seclabel{sources}

\index{Browser!panneau de code}
\seeindex{Browser!panneau source}{Browser, panneau de code} % CHANGE
\seeindex{Browser!view}{Browser, panneau de code} % REVOIR
L'option de menu \menu{various \go view \ldots} 
% ajout - martial
disponible en \actclickant{} dans le panneau des méthodes
affiche le menu ``comment faut-il l'afficher''
qui vous permet de choisir comment le navigateur va afficher
% ajout - martial
la méthode sélectionnée dans le panneau inférieur \ie le panneau de code
(ou \emph{panneau source}).
%Le bouton \button{source} affiche un menu que nous pourrions appeler
%%``what to show'' menu, 
%``ce qui est à voir''; il nous permet de choisir ce que le navigateur
%affiche dans le panneau inférieur ou panneau source.
Parmi les propositions, nous avons l'affichage du code \menu{source}, 
du code source en mode \menu{prettyPrint} (affichage élégant), 
du code compilé ou \menu{byteCode}, ou encore du code source
décompilé depuis le \emph{bytecode} via \menu{decompile}.
%Le label du bouton change pour afficher le mode choisi. Il y a d'autres
%options; si vous promenez la souris sur ces options, vous verrez
%apparaître un ballon d'aide (ou \emph{help balloon}). Essayez-en
%quelques-uns. 
\index{méthode!pretty-print}
\index{méthode!decompile}
\index{méthode!byte code}

Remarquez que le choix de \menu{prettyPrint} dans ce menu n'est
\emph{absolument pas} le même que le travail en mode \emph{pretty-print} d'une méthode
avant sa sauvegarde\footnote{\menu{pretty print (r)} est la première option de menu 
dans le panneau des méthodes ou celui à mi-hauteur dans le menu du panneau de code.}.
Le menu contrôle seulement l'affichage du navigateur et n'a aucun effet sur
le code enregistré dans le système.
Vous pouvez le vérifier en ouvrant deux navigateurs et en sélectionnant
\menu{prettyPrint} pour l'un et \menu{source} pour l'autre.
Pointer les deux navigateurs sur la même méthode et en choisissant
\menu{byteCode} dans l'un et \menu{decompile} dans l'autre est vraiment
une bonne manière d'en apprendre plus sur le jeu d'instructions codées
(dit aussi \emph{byte-codées}) de la machine virtuelle \pharo. % CHANGE

\subsubsection{La refactorisation}


Les menus contextuels proposent un grand nombre de refactorisations (ou
\emph{refactoring}) classiques. \Actclickz sur l'un des quatre panneaux
supérieurs pour voir les opérations de refactorisation actuellement
disponibles (voir \figref{refactoring}).
À l'origine, cette fonction était disponible uniquement
par un navigateur spécifique nommé Refactoring Browser, mais
elle peut désormais être accessible depuis n'importe quel
navigateur. % CHANGE

\begin{figure}[btp]
	\begin{center}
	\ifluluelse
		{\includegraphics[width=\textwidth]{refactoring}}
		{\includegraphics[width=0.7\textwidth]{refactoring}}
	\end{center}
	\caption{Le refactorisation à la souris.}
	\figlabel{refactoring}
\end{figure} % REVOIR
%---------------------------------------------------------
\subsection{Les menus du navigateur}

De nombreuses fonctions complémentaires sont disponibles 
en \actclickant{} dans les panneaux du Browser.
Même si les options de menus portent le même nom,
leur \emph{signification} dépend du contexte.
Par exemple, le panneau des paquetages, le
panneau des classes, le panneau des protocoles et enfin, celui des méthodes
ont tous \menu{file out} dans leurs menus respectifs. Cependant, chaque 
\menu{file out} fait une chose différente: dans le panneau des paquetages,
il enregistre entièrement dans un fichier le paquetage
sélectionné; dans le celui des classes, des protocoles ou des
méthodes, il exporte respectivement la classe entière, le protocole
entier ou la méthode affichée. % CHANGE REVOIR
Bien qu'apparemment évident, ce peut être une source de confusion pour
les débutants.
\index{fichier!filing in}
\index{fichier!importation}
\index{fichier!filing out}
\index{fichier!exportation}

L'option de menu probablement la plus utile est \menu{find class\ldots (f)} 
dans le panneau des paquetages. 
Elle permet de trouver une classe.
% ajout
Bien que les catégories soient utiles pour arranger le code que nous 
sommes en train de développer, la plupart d'entre nous ne connaissent pas
la catégorisation de tout le système, et c'est beaucoup plus rapide
en tapant \short{f} suivi par les premiers caractères du nom d'une 
classe que de deviner dans quel paquetage elle peut bien être.
\menu{recent classes\ldots } vous aide aussi à retrouver rapidement
une classe parmi celles que vous avez visitées récemment, même si vous
avez oublié son nom.
% FAUX dans la VO
% rene : tout semble fonctionne (mais il faut supprimer (r) dans \menu{recent classes\ldots (r)}  
\index{classe!recherche}
\index{classe!récente}

Vous pouvez aussi rechercher une classe ou une méthode en particulier en
saisissant son nom dans la boîte de requête située dans la partie 
supérieure gauche de votre navigateur. Quand vous tapez
sur la touche ``entrée'', une requête sera envoyée au système et son
résultat sera affiché. % REVOIR CHANGE
Notez qu'en préfixant votre requête avec \ct{#}, vous pouvez chercher
toutes les références à une classe ou tous les \senders d'un
message.
%Dans le panneau de classes, le menu propose \menu{find method} (pour
%``trouver une méthode'') et \menu{find method wildcard\ldots} qui
%s'avèrent utiles si vous souhaitez naviguer dans une méthode
%particulière.
Cependant, si vous recherchez une méthode dans une classe sélectionnée, il est
souvent plus efficace de naviguer dans le protocole \prot{-{}-all-{}-} 
(qui d'ailleurs est le choix par défaut), placer la souris dans le
panneau des méthodes et taper la première lettre du nom de la méthode
que vous recherchez. % CHANGE
Ceci va faire glisser l'ascenseur du panneau jusqu'à ce que la méthode souhaitée soit visible.
\seeindex{méthode!trouver}{méthode, recherche}
\index{méthode!recherche}
\protindex{all}

\dothis{Essayez les deux techniques de navigation pour \cmind{OrderedCollection}{removeAt:}}

Il y a beaucoup d'autres options dans les menus.  Passer quelques
minutes à tester les possibilités du navigateur est véritablement payant.  

\dothis{Comparez le résultat de \menu{Browse Protocol}, \menu{Browse Hierarchy},  et \menu{Show Hierarchy} dans le menu contextuel du panneau de classes.}

%---------------------------------------------------------
\subsection{Naviguer par programme} %{Browsing programmatically}

La classe \glbind{SystemNavigation} offre de nombreuses méthodes utiles
pour naviguer dans le système.
Beaucoup de fonctionnalités offertes par le navigateur classique
sont programmées par
\ct{SystemNavigation}.
\index{navigation par programme}

\dothis{
Ouvrez un espace de travail Workspace et exécutez le code suivant pour naviguer dans la liste des \senders du message 
\ct{drawOn:} en utilisant \menu{do it}:}

\begin{code}{}
SystemNavigation default browseAllCallsOn: #drawOn: .
\end{code}
Pour restreindre le champ de la recherche à une classe spécifique:
\begin{code}{}
SystemNavigation default browseAllCallsOn: #drawOn: from: ImageMorph .
\end{code}
Les outils de développement sont complètement accessibles depuis un
programme car \emph{ceux-ci sont aussi des objets}. Vous pouvez dès lors
développer vos propres outils ou adapter ceux qui existent déjà
selon vos besoins.

L'équivalent programmatique de l'option de menu
\menu{implementors} est: % CHANGE
\begin{code}{}
SystemNavigation default browseAllImplementorsOf: #drawOn: .
\end{code}

Pour en apprendre plus sur ce qui est disponible, explorez la classe
\ct{SystemNavigation} avec le navigateur.

Des exemples supplémentaires peuvent être trouvés dans
 \charef{faq}.

%=========================================================
\section{Monticello}

Nous vous avons donné un aperçu de \ind{Monticello}, l'outil de gestion
de paquetages de \pharo dans \secref{Monticello}.  
Cependant Monticello a beaucoup plus de fonctions que celles dont nous allons
discuter ici.
Comme Monticello gère des \emph{paquetages} dits \emph{packages}, nous allons expliquer ce qu'est
un \ind{paquetage} avant d'aborder Monticello proprement dit.

%---------------------------------------------------------
\subsection{Les paquetages: une catégorisation déclarative du code de Pharo}
\seclabel{packages}

Dans \secref{categoriesPackages}, nous avons pointé le fait
que les paquetages sont plus ou moins équivalents aux catégories. 
Nous allons désormais voir la relation qui existe entre eux.
Le système du paquetage est une façon simple et légère
d'organiser le code source de Smalltalk; il exploite une simple
convention de nommage pour les catégories et les protocoles.

Prenons l'exemple suivant en guise d'explication.
Supposons que nous soyons en train de développer une librairie pour
nous faciliter l'utilisation d'une base de données relationnelles depuis
\pharo. Vous avez décidé d'appeler votre librairie (ou \emph{framework})
\ct{PharoLink} et vous avez créé une série de catégories
contenant toutes les classes que vous avez écrites, par exemple la catégorie
\ct{'PharoLink-Connections'} contient les classes
\ct{OracleConnection MySQLConnection PostgresConnection} et la catégorie
\ct{'PharoLink-Model'} contient les classes
\ct{DBTable DBRow DBQuery} % CHANGE
et ainsi de suite. Cependant, tout le code ne résidera pas dans ces classes.
Par exemple, vous pouvez aussi avoir une série de méthodes pour 
convertir des objets dans un format sympathique pour notre format
SQL~\footnote{Nous dirions que ce format est SQL-friendly.}:

\begin{code}{}
Object>>>asSQL
String>>>asSQL
Date>>>asSQL
\end{code}

\noindent
Ces méthodes appartiennent au même paquetage que les classes
dans les catégories \ct{PharoLink-Connections} et \ct{PharoLink-Model}.
Mais la classe \ct{Object} n'appartient clairement pas à notre paquetage!
Vous avez donc besoin de trouver un moyen pour associer certaines
\emph{méthodes} à un paquetage même si le reste de la classe est dans
un autre. 
\index{paquetage!extension}
\seeindex{extension de paquetage}{paquetage, extension}
\seeindex{extension de package}{paquetage, extension}

Pour ce faire, nous plaçons ces méthodes (de \ct{Object}, \ct{String}, \ct{Date} \etc) dans un protocole nommé \prot{*PharoLink} (remarquez l'astérisque en début de nom). L'association des
catégories en \scat{PharoLink-\ldots} et des protocoles \prot{*PharoLink} 
forme un paquetage nommé \ct{PharoLink}.
Précisement, les règles de formation d'un paquetage s'énoncent comme suit.

Un paquetage appelé \ct{Foo} contient:

\begin{enumerate}		\seclabel{packageRules}
	\item{} toutes les \emph{définitions de classe} des classes présentes dans
la catégorie \scat{Foo} ou toutes catégories avec un nom commençant par
\scat{Foo-};
	\item{} \label{env:extensions} toutes les \emph{méthodes}
dans \emph{n'importe quelle classe} dont le protocole se nomme
\prot{*Foo} ou \prot{*foo}~\footnote{Durant la comparaison
de ces noms, la casse des lettres est parfaitement ignorée.}, et;
%%ou n'importe quel nom commençant par \prot{*foo-} et;
\item{} toutes les \emph{méthodes} dans les classes présentes dans
\scat{Foo} ou toutes catégories avec un nom commençant par \scat{Foo-}, 
\emph{exception} faite des méthodes dont le nom des protocoles débute par 
\prot{*}.
\end{enumerate}

\noindent
Une conséquence de ces règles est que chaque définition de classe et chaque 
méthode appartiennent exactement à un paquetage. 
L'\emph{exception} de la dernière règle est justifiée parce que
ces méthodes doivent appartenir à d'autres paquetages.
La raison pour laquelle la casse
%%~\footnote{La hauteur minuscule ou majuscule d'une lettre.} 
est ignorée dans la règle \ref{env:extensions} 
% Rene rewording
% est que, par convention, les noms de protocole sont tous en
% minuscules (et peuvent inclure des  
% espaces), alors que les noms de catégorie utilise une écriture
% en chameau c'est-à-dire les 
% mots composants ces noms sont en capitales et forment les noms sans espaces comme
% dans CamelCase (nom anglais de cette technique de formatage de nom).
est que, conventionnellement les noms de protocoles sont typiquement
(mais pas nécessairement) en minuscules (et peuvent inclure des espaces); alors que 
les noms de catégories utilisent un format d'écriture dit \emph{casse de chameau} comme par 
exemple AlanKay, LargePositiveInteger ou CamelCase (d'ailleurs CamelCase est 
le nom anglais de ce type de format de noms).
\index{écriture en chameau}
\seeindex{CamelCase}{écriture en chameau}
% martial: au-dessus, dans la correction de rene, je n'ai mis
% LargePositiveInteger a la place de SmallTalk parce que ce n'est pas
% (plus) la convention d'ecriture

La classe \ct{PackageInfo} implémente ces règles et vous pouvez mieux les 
appréhender en expérimentant cette classe.

\dothis{Évaluez l'expression suivante dans un espace de travail:}
%\seeindex{refactoring}{refactorisation} REVOIR


\begin{code}{}
mc := PackageInfo named: 'Monticello'
\end{code}

Il est possible maintenant de faire une introspection de ce paquetage.
Par exemple, imprimer via \menu{print it} le code \ct{mc classes} dans l'espace de travail nous retourne la longue liste des classes qui font le paquetage Monticello. L'expression \ct{mc coreMethods}
nous renvoie une liste de \mbox{\ct{MethodReference}{s}} ou références de méthodes
pour toutes les méthodes de ces classes.
La requête \ct{mc extensionMethods} est peut-être une des plus
intéressantes: elle retourne la liste de toutes les méthodes contenues
dans le paquetage \ct{Monticello} qui ne sont pas dans une classe de
\lct{Monticello}. % CHANGE

\arevoir{Les paquetages sont des ajouts à \pharo relativement récents
mais, puisque les conventions de nommage de paquetage sont basées
sur celles déjà existantes, il est possible d'utiliser
\ct{PackageInfo} pour analyser du code plus ancien qui n'a pas 
été explicitement adapté pour pouvoir y répondre.} % REVOIR -
% martial : FAUX? DANS PBE
% rene : nous pourrions remplacer \pharo par \st

\dothis{Imprimer le code \ct{(PackageInfo named: 'Collections') externalSubclasses}; 
cette expression répond une liste de toutes les sous-classes de \ct{Collection}
qui ne sont \emph{pas} dans le paquetage \ct{Collections}.}

%---------------------------------------------------------

\subsection{Les fondamentaux de Monticello}
% les fondamentaux, les bases

\ind{Monticello} est nommé ainsi d'après la villégiature 
de Thomas Jefferson, troisième président des États-Unis d'Amérique
et auteur de la statue pour les libertés religieuses (Religious Freedom) en
Virginie. Le nom signifie ``petite montagne'' en italien, en ainsi, il est
toujours prononcé avec un ``c'' italien, \ie avec le son \emph{tch} comme
dans ``quetsche'':
Monn-ti-tchel-lo%
%Mont-y'-che-llo.
~\footnote{Note du traducteur: c'est aussi une commune de Haute-Corse.}.

\begin{figure}[btp]
	\begin{center}
	\ifluluelse
		{\includegraphics[width=\textwidth]{freshMonticello}}
		{\includegraphics[width=0.7\textwidth]{freshMonticello}}
	\end{center}
	\caption{Le navigateur Monticello.}
	\figlabel{freshMonticello}
\end{figure}

Quand vous ouvrez le navigateur Monticello, vous voyez deux panneaux
de listes et une ligne de boutons, comme sur \figref{freshMonticello}.

Le panneau de gauche liste tous les paquetages qui ont été chargés
dans l'image actuelle; la version courante du paquetage est
présentée entre parenthèses à la suite de son nom.

Celle de droite liste tous les dépôts (ou \emph{repository}) de code
source que Monticello connaît généralement pour les avoir utilisés
pour charger le code. Si vous sélectionnez un paquetage dans le panneau de 
gauche, celui de droite est filtré pour ne montrer que les dépôts
qui contiennent des versions du paquetage choisi.

Un des dépôts est un répertoire nommé \emph{package-cache} qui
est un sous-répertoire du répertoire courant où vous avez
votre image.
Quand vous chargez du code depuis un dépôt distant (ou remote repository)
ou quand vous écrivez du code, une copie est effectuée aussi dans ce
répertoire de cache. Il peut être utile si le réseau n'est pas 
disponible et que vous ayez besoin d'accéder à un paquetage. De plus,
si vous avez directement reçu un fichier Monticello (.mcz), par exemple, 
en pièce jointe dans un courriel, la façon la plus convenable d'y accéder
depuis \pharo est de le placer dans le répertoire package-cache.
\index{package!cache}

Pour ajouter un nouveau dépôt à la liste, cliquez sur le bouton 
\button{+Repository} et choisissez le type de dépôt dans le menu
flottant. Disons que nous voulons ajouter un dépôt HTTP.

\dothis{Ouvrez Monticello, cliquez sur \button{+Repository} et choisissez \menu{HTTP}.
Éditez la zone de texte à lire:}
%\ab{How does one continue the $\backslash$dothis to include the code?}
%\on{Don't.  Just close the \dothis{} and follow with the code.}
\needlines{4}
\begin{code}{}
MCHttpRepository
	location: 'http://squeaksource.com/PharoByExample'
	user: ''
	password: ''
\end{code}

\begin{figure}[btp]
	\begin{center}
	\ifluluelse
		{\includegraphics[width=0.7\textwidth]{SqueakSource-PBE}}
		{\includegraphics[width=0.7\textwidth]{SqueakSource-PBE}}
	\end{center}
	\caption{Un navigateur de dépôts ou Repository Browser.}
	\figlabel{SqueakSource:PBE}
\end{figure}
\noindent

Ensuite cliquez sur \button{Open} pour ouvrir un navigateur de dépôts ou
Repository Browser. Vous devriez voir quelque chose comme
 \figref{SqueakSource:PBE}.  
Sur la gauche, nous voyons une liste de tous les paquetages présents dans le
dépôt; si vous en sélectionnez un, le panneau de droite affichera
toutes les versions du paquetage choisi dans ce dépôt.

Si vous choisissez une des versions, vous pourrez naviguer dans son contenu (sans le charger dans votre image) via le bouton \button{Browse}, le charger
par le bouton \button{Load} ou encore inspecter les modifications
via \button{Changes} qui seront faites à votre image en chargeant la version
sélectionnée. Vous pouvez aussi créer une copie grâce au bouton \button{Copy}
d'une version d'un paquetage que vous pourriez ensuite écrire dans un
autre dépôt.

Comme vous pouvez le voir, les noms des versions contiennent le nom du paquetage, les initiales de l'auteur de la version et un numéro de version.
Le nom d'une version est aussi le nom du fichier dans le dépôt. 
Ne changez jamais ces noms; le déroulement correct des opérations
effectuées dans Monticello dépend d'eux!
Les fichiers de version de Monticello sont simplement des archives compressées
 et, si vous êtes curieux vous pouvez les décompresser avec un outil 
de décompression ou \emph{dézippeur}, mais la meilleure façon 
d'explorer leur contenu consiste à faire appel à Monticello lui-même.

Pour créer un paquetage avec Monticello, vous n'avez que deux choses à faire:
écrire du code et le mentionner à Monticello.

\dothis{Créez une paquetage appelé \scat{PBE-Monticello}, et mettez-y
une paire de classes, comme vu sur \figref{MCnewcategory}. 
Créez une méthode dans une classe existante, par exemple
\ct{Object}, et mettez-la dans le 
même paquetage que vos classes en utilisant les règles de la page~\pageref{sec:packageRules}\,---\,voir \figref{MCnewmethod}.}

\begin{figure}[btp]
	\begin{center}
	\ifluluelse
		{\includegraphics[width=\textwidth]{MCnewcategory}}
		{\includegraphics[width=0.7\textwidth]{MCnewcategory}}
	\end{center}
	\caption{Deux classes dans le paquetage ``PBE''.}
	\figlabel{MCnewcategory}
\end{figure}

\begin{figure}[btp]
	\begin{center}
	\ifluluelse
		{\includegraphics[width=\textwidth]{MCnewmethod}}
		{\includegraphics[width=0.7\textwidth]{MCnewmethod}}
	\end{center}
	\caption{Une extension de méthode qui sera aussi incluse dans le paquetage ``PBE''.}
	\figlabel{MCnewmethod}
\end{figure}

Pour mentionner à Monticello l'existence de votre paquetage, 
cliquez sur le bouton \button{+Package} et tapez le nom du paquetage,
dans notre cas ``PBE''.
Monticello ajoutera \ct{PBE} à sa liste de paquetages;
l'entrée du paquetage sera marquée avec une astérisque pour
montrer que la version présente dans votre image n'a pas
été encore écrite dans le dépôt.
Remarquez que vous devriez avoir maintenant deux paquetages;
un nommé \ct{PBE} et un autre nommé \ct{PBE-Monticello}. C'est
normal puisque \ct{PBE} contiendra \ct{PBE-Monticello} ainsi que
tout autre paquetage dont le nom commence par \ct{PBE-}. % CHANGE

Initialement, le seul dépôt associé à ce paquetage sera votre
\emph{package cache} comme sur \figref{MC+PBE}.
C'est parfait: vous pouvez toujours sauvegarder le code en l'écrivant
dans ce répertoire local de cache.
Maintenant, cliquez sur \button{Save} et vous serez invité à
fournir des informations ou \emph{log message} pour la version de ce 
paquetage, comme le montre \figref{PBE-on}; 
quand vous acceptez le message entré, Monticello sauvegardera votre paquetage
et l'astérisque décorant le nom du paquetage du panneau de gauche
de Monticello disparaîtra avec le changement du numéro de version.

Si vous faites ensuite une modification dans votre paquetage,---\,disons
en ajoutant une méthode à une des classes\,---\,l'astérisque réapparaîtra pour 
signaler que vous avez des changements non-sauvegardés.
Si vous ouvrez un Repository Browser sur le package cache, vous
pouvez choisir une version sauvée et utiliser le bouton \button{Changes}
ou d'autres boutons.
Vous pouvez aussi bien sûr sauvegarder la nouvelle version dans
ce dépôt; une fois que vous rafraîchissez la vue
du dépôt via le bouton \button{Refresh}, vous devriez voir
la même chose que sur
\figref{package-cache-browser}.
\index{paquetage!package cache}
\seeindex{package!cache}{paquetage, package cache}
\seeindex{Monticello!package cache}{paquetage, package cache}

\begin{figure}[tbp]
	\begin{center}
		\includegraphics[width=\textwidth]{MC+PBE}
	\end{center}
	\caption{Le paquetage PBE pas encore sauvegardé dans Monticello.}
	\figlabel{MC+PBE}
\end{figure}

\begin{figure}[tbp]
	\begin{center}
		{\includegraphics[width=0.8\textwidth]{PBE-on}}
	\end{center}
	\caption{Fournir un \emph{log message} pour une version d'un paquetage.}
	\figlabel{PBE-on}
\end{figure}

\begin{figure}[tbp]
	\begin{center}
		{\includegraphics[width=\textwidth]{package-cache-browser}}
	\end{center}
	\caption{Deux versions de notre paquetage sont maintenant le dépôt \emph{package cache}.}
	\figlabel{package-cache-browser}
\end{figure}

Pour sauvegarder notre nouveau paquetage dans un autre
dépôt (autre que package-cache), vous avez besoin de vous 
assurer tout d'abord que Monticello connaît
ce dépôt en l'ajoutant si nécessaire.
Alors vous pouvez utiliser le bouton \button{Copy} dans le 
Repository Browser de package-cache et choisir le dépôt vers lequel
le paquetage doit être copié.
Vous pouvez aussi associer le dépôt désiré avec le paquetage
en sélectionnant \menu{add to package \ldots} 
dans le menu contextuel du répertoire accessible en \actclickant,
comme nous pouvons le voir dans \figref{associateRepository}.
Une fois que le paquetage est lié à un dépôt, vous pouvez sauvegarder
toute nouvelle version en sélectionnant le dépôt et le paquetage
dans le Monticello Browser puis en cliquant sur le bouton 
 \button{Save}.  
Bien entendu, vous devez avoir une permission d'écrire dans un dépôt.
Le dépôt \ct{PharoByExample} sur \emphind{\sqsrc} est
lisible pour tout le monde mais n'est pas ouvert en écriture à tout le monde; 
ainsi, si vous essayez d'y sauvegarder quelque chose, vous aurez un message d'erreur.
Cependant, vous pouvez créer votre propre dépôt sur 
\sqsrc en utilisant l'interface web de \url{http://www.squeaksource.com} et 
en l'utilisant pour sauvegarder votre travail.
Ceci est particulièrement utile pour partager votre code avec vos amis ou
si vous utilisez plusieurs ordinateurs.

\begin{figure}[tbp]
	\begin{center}
		\includegraphics[width=\textwidth]{MCaddToPackage}
	\end{center}
	\caption{Ajouter un dépôt à l'ensemble des dépôts liés
au paquetage.}
	\figlabel{associateRepository}
\end{figure}

Si vous essayez de sauvegarder dans un répertoire dans lequel vous n'avez
pas les droits en écriture, une version sera de toute façon écrite
dans le package-cache.
Donc vous pourrez corriger en éditant les informations du dépôt
(en \actclickant{} dans Monticello Browser) ou
en choisissant un dépôt différent puis, en le copiant
depuis le navigateur ouvert sur package-cache avec le bouton \button{Copy}.

%=========================================================
\section{L'inspecteur  et l'explorateur} % CHANGE
%\section{L'inspecteur Inspector et l'explorateur Explorer}
\seclabel{inspector} % (fold)

Une des caractéristiques de \st qui le rend différent de nombreux 
environnements de programmation est qu'il vous offre une fenêtre 
sur une monde d'objets vivants et non pas sur un monde de codes statiques.
Chacun de ces objets peut être examiné par le programmeur et même
changé\,---\,bien qu'un certain soin doit être apporté lorsqu'il s'agit
de modifier des objets bas niveau qui soutiennent le système.
De toute façon, expérimentez à votre guise, mais sauvegardez votre
image avant!

%---------------------------------------------------------
\subsection{Inspector}

\dothis{Pour illustrer ce que vous pouvez faire avec
  l'\ind{inspecteur} ou \ind{Inspector}, tapez \ct{TimeStamp now} dans
  un espace de travail puis \actclickz et choisissez \menu{inspect it}.} 
(Il n'est pas nécessaire de sélectionner le texte avant d'utiliser le menu;
si aucun texte n'est sélectionné, les opérations du menu fonctionnent
sur la ligne entière.
Vous pouvez aussi entrer \short{i} pour \menu{\textbf{i}nspect it}.)
\clsindex{TimeStamp}
\index{raccourci clavier!inspect it}

\begin{figure}[btp]
	\begin{center}
		\includegraphics[width=\textwidth]{inspectTimeNow1}
	\end{center}
	\caption{Inspecter \ct{TimeStamp now}.}
	\figlabel{inspectTimeNow1}
\end{figure}

Une fenêtre comme celle de \figref{inspectTimeNow1} apparaîtra.
Cet inspecteur peut être vu comme une fenêtre sur les états internes
d'un objet particulier\,---\,dans ce cas, l'instance particulière
de
 \mbox{\ct{TimeStamp}} 
% the \mbox is here because without it, the listings macros puts a space between TimeStamp 
% and the following word, and that space happens to come out at the start of a line.
qui a été créée en évaluant l'expression 
\ct{TimeStamp now}.
La barre de titre de la fenêtre affiche  \arelire{la représentation textuelle} de l'objet
en cours d'inspection.
Si vous sélectionnez la ligne la plus haute dans le panneau supérieur de gauche,
le panneau de droite affichera \arelire{la description textuelle de l'objet,
dite aussi \emph{printString} de l'objet.}
% Si vous sélectionnez \menu{all inst vars} dans le panneau de gauche,
% celui de droite vous présentera une liste de 
% toutes les variables d'instance de l'objet accompagnées de leur
% description printstring.
% Les éléments à suivre dans la liste de ce panneau de gauche
% représentent les variables d'instance; une par une, elles peuvent ainsi
% être facilement examinées et même modifiées dans le panneau
% de droite. % CHANGE

Le panneau de gauche montre une vue arborescente de l'objet
avec \self{} pour racine. Les variables d'instance peuvent être
explorées en \clickant{} sur les triangles à côté de leurs noms.% REVOIR

% The left pane shows a tree view of the object, with \self at the root.
% Instance variables can be explored by expanding the triangles next to their names.

Le panneau horizontal inférieur de l'Inspector est un petit espace de
travail ou Workspace.
C'est utile car dans cette fenêtre, la pseudo-variable \ct{self}
correspond à l'objet que vous avez sélectionné dans le panneau de gauche. 
% CHANGE
% is bound to the object that you have selected in the left pane.
Ainsi, si vous inspectez via \menu{inspect it} l'expression:
\begin{code}{}
self - TimeStamp today
\end{code}
dans ce panneau-espace de travail, le résultat sera un objet 
\clsind{Duration} qui représente l'intervalle temporel entre 
%midnight today and the instant at which you evaluated  \ct{TimeStamp now} and created the \ct{TimeStamp} object that you are inspecting.
la date d'aujourd'hui (en anglais, \ct{today}, le nom du message envoyé) à
minuit et le moment où vous avez évalué \ct{TimeStamp now}
et ainsi créé l'objet \ct{TimeStamp} que vous inspectez.
Vous pouvez aussi essayer d'evaluer \ct{TimeStamp now - self}; 
ce qui vous donnera le temps que vous avez mis à lire la section de ce livre!

En plus de \ct{self}, toutes les variables d'instance de l'objet sont
visibles dans le panneau-espace de travail; dès lors vous pouvez
les utiliser dans des expressions ou même les affecter.
Par exemple, si vous sélectionnez l'objet racine et que vous
évaluez \ct{jdn  := jdn - 1} dans ce panneau,
vous verrez que la valeur de la variable d'instance \ct{jdn} 
changera réellement et que la valeur de \ct{TimeStamp now - self} 
sera augmentée d'un jour.

% ON: Does not work anymore
%Vous pouvez changer les variables d'instance directement en les sélectionnant,
%puis en remplaçant l'ancienne valeur dans le panneau de droite
%par une expression \pharo et en acceptant cette dernière.
%\pharo évaluera l'expression et assignera le résultat à la variable
%d'instance.

Il y a des variantes spécifiques de l'inspecteur pour les dictionnaires (sous-classes 
de Dictionaries), pour les collections ordonnées (sous-classes de OrderedCollections), 
pour les CompiledMethods (objets des méthodes compilées) et pour quelques autres classes 
facilitant ainsi l'examen du contenu de ces objets spéciaux.

%---------------------------------------------------------
\subsection{Object Explorer}

\arevoir{L'\emph{Object Explorer} ou \ind{explorateur} d'objets est sur le plan
conceptuel semblable à l'inspecteur mais présente ses informations
de manière différente.
Pour voir la différence, nous allons \emph{explorer} le même objet
que nous venons juste d'inspecter.} % martial: OBSOLETE ou pas

\begin{figure}[tbp]
\begin{minipage}{0.48\textwidth}
	\begin{center}
	\ifluluelse
		{\includegraphics[width=\textwidth]{exploreTimeStampNow}}
		{\includegraphics[width=0.7\textwidth]{exploreTimeStampNow}}
	\end{center}
	\caption{Explorer \ct{TimeStamp now}.}
	\figlabel{exploreTimeStampNow}
\end{minipage}
\hfill
\begin{minipage}{0.48\textwidth}
	\begin{center}
	\ifluluelse
		{\includegraphics[width=\textwidth]{exploreTimeStampNow2}}
		{\includegraphics[width=0.7\textwidth]{exploreTimeStampNow2}}
	\end{center}
	\caption{Explorer les variables d'instance.}
	\figlabel{exploreTimeStampNow2}
\end{minipage}
\end{figure}

% rene : pourquoi ne pas remplacer self par "la plus haute ligne" ?
% martial : bonne idée mais ce n'est plus (explore (I) mais Explore Pointers (e)) 
\dothis{Sélectionnez %\arevoir{\menu{self}} dans le panneau gauche de notre
la plus haut ligne dans le panneau gauche de notre
inspecteur et choisissez % \menu{explore (I)} dans le menu contextuel
\menu{Explore Pointers (e)} dans le menu contextuel
obtenu en \actclickant.}
La fenêtre \ind{Explorer} apparaît alors comme sur
 \figref{exploreTimeStampNow}.
Si vous cliquez sur le petit triangle à gauche de \ct{root} (racine, en anglais), 
la vue changera comme dans \figref{exploreTimeStampNow2} qui
nous montre les variables d'instance de l'objet que nous explorons.
Cliquez sur le triangle proche d'\ct{offset} et vous verrez
\emph{ses} variables d'instance.
L'explorateur est véritablement un outil puissant lorsque vous avez besoin
d'explorer une structure hiérarchique complexe\,---\,d'où son nom.
\index{raccourci clavier!explore it}

Le panneau Workspace de l'Object Explorer fonctionne de façon
 légèrement différente de celui de l'Inspector.
\ct{self} n'est pas lié à l'objet racine root mais plutôt
à l'objet actuellement sélectionné; les variables d'instance de
l'objet sélectionné sont aussi à portée~\footnote{En anglais, vous
entendrez souvent le terme ``scope'' pour désigner la portée des
variables d'instance.}.

Pour comprendre l'importance de l'explorateur, employons-le pour
explorer une structure profonde imbriquant beaucoup d'objets.

\dothis{Évaluez \ct{Object explore} dans un espace de travail.}
C'est l'objet qui représente la classe  \ct{Object} dans \pharo.
Notez que vous pouvez naviguer directement dans les objets
représentants le dictionnaire de méthodes et même explorer les
méthodes compilées de cette classe (voir \figref{ExploreObject}). % CHANGE

\begin{figure}[tbp]
	\begin{center}
		\includegraphics[width=0.5\textwidth]{ExploreObject}
	\end{center}
	\caption{Explorer un \ct{Object}.}
	\figlabel{ExploreObject}
\end{figure}

% \dothis{Ouvrez un navigateur et \actclickz{} cinq fois 
%  sur le panneau des méthodes
% de manière à afficher le \emph{halo} Morphic sur
% le morph \ct{PluggableListMorph} qui est utilisé pour représenter
% la liste des messages.
% \Clickz{}sur l'icône
% \emph{debug} \debugHandle{} et sélectionnez dans le menu flottant
% \menu{explore morph}.  
% Ceci ouvrira un Explorer sur l'objet \clsind{OBPluggableListMorph} qui
% représente la liste de méthodes du navigateur à l'écran.
% Ouvrez l'objet root (en cliquant sur son triangle), ouvrez ses sous-morphs
% \ct{submorphs} et continuez d'explorer la structure des objets sur lesquels
% reposent ce morph comme nous pouvons le voir sur
% \figref{explorePluggableListMorph}.}

% \begin{figure}[tbp]
% 	\begin{center}
% 		\includegraphics[width=0.7\textwidth]{explorePluggableListMorph}
% 	\end{center}
% 	\caption{Explorer une \ct{OBPluggableListMorph}.}
% 	\figlabel{explorePluggableListMorph}
% \end{figure}

%=========================================================
\section{Le débogueur}
%\section{Debugger, le débogueur} % CHANGE
\seclabel{debugger} % (fold)

Le \ind{débogueur} \ind{Debugger} est sans conteste l'outil le plus
puissant dans la suite d'outils de \pharo. 
%is arguably the most powerful tool in the Squeak tool suite.
Il est non seulement employé pour déboguer c'est-à-dire pour corriger les erreurs
%ajout
mais aussi pour écrire du code nouveau.
Pour démontrer la richesse du Debugger, commençons par
créer un \emph{bug}!

\dothis{Via le navigateur, ajouter la méthode suivante dans la classe \ct{String}:}

\needlines{7} % REVOIR
\begin{method}[buggy]{Une méthode boguée}
suffix
	"disons que je suis un nom de fichier et que je fournis mon suffixe, la partie suivant le dernier point"
	| dot dotPosition |
	dot := FileDirectory dot.
	dotPosition := (self size to: 1 by: -1) detect: [ :i | (self at: i) = dot ].
	^ self copyFrom: dotPosition to: self size 
\end{method}

Bien sûr, nous sommes certain qu'une méthode si triviale fonctionnera.
Ainsi plutôt que d'écrire un test \emph{SUnit} 
%ajout
(que nous verrons dans \charef{SUnit}),
nous entrons simplement \ct{'readme.txt' suffix} dans un Workspace
et nous en imprimons l'exécution via \menu{print it (p)}.
Quelle surprise! Au lieu d'obtenir la réponse attendu \ct{'txt'}, 
une notification \clsind{PreDebugWindow} s'ouvre comme sur
\figref{PreDebugWindow}.

\begin{figure}[btp]
	\begin{center}
		{\includegraphics[width=0.8\textwidth]{PreDebugWindow}}
	\end{center}
	\caption{Un \ct{PreDebugWindow} nous alarme de la présence d'un bug.}
	\figlabel{PreDebugWindow}
\end{figure}

\tradalert{martial}{modif récente 2011-04-19}\\
Le \ct{PreDebugWindow} nous indique dans sa barre de titre
qu'une erreur s'est produite et nous affiche une trace de la pile d'exécution
ou \emphind{stack trace} des messages qui ont conduit à l'erreur:
% ajout vf martial
\arelire{la dernière exécution est représentée par la plus haute ligne sur la pile.
En descendant la trace (donc en remontan le temps), nous tombons sur la ligne
 \ct{UndefinedObject>>>DoIt} qui représente le code qui vient d'être compilé
et lancé quand nous avons demandé à \pharo d'imprimer 
le code \ct{'readme.txt' suffix} dans notre espace de travail
par \menu{print it}.}
Ce code a envoyé le message \ct{suffix} à
l'objet \clsind{ByteString} (\ct{'readme.txt'}).
S'en est suivi l'exécution de la méthode \ct{suffix} héritée de la
classe \ct{String};
toutes ces informations sont disponibles dans la ligne précédente de la trace,
\ct{ByteString(String)>>>suffix}.
En remontant la pile, nous pouvons voir que \ct{suffix} envoie
à son tour \ct{detect:}; cette dernière méthode envoie à son tour \ct{detect:ifNone} qui émet 
\ct{errorNotFound}.
\clsindex{UndefinedObject}

\begin{figure}[btp]
	\begin{center}
	\ifluluelse
		{\includegraphics[width=\textwidth]{debuggerDetectIfNone}}
		{\includegraphics[width=0.7\textwidth]{debuggerDetectIfNone}}
	\end{center}
	\caption{Le débogueur.}
	\figlabel{debuggerDetectIfNone}
\end{figure}

Pour trouver \emph{pourquoi} le point (\ct{dot}) n'a pas été trouvé,
nous avons besoin du débogueur lui-même que nous pouvons appeler en
\clickant{} sur le bouton \button{Debug}
%ajout vf martial
\arelire{ou en \clickant{} sur une ligne de la pile}.

% \dothis{Vous pouvez aussi ouvrir Debugger en \clickant
% sur n'importe quelle ligne du \emph{stack trace}. 
% Si vous faites ainsi, le débogueur s'ouvrira sur la méthode correspondante.}

Le débogueur est visible sur \figref{debuggerDetectIfNone}; 
il semble intimidant au début, mais il est assez facile à utiliser.
La barre de titre et le panneau supérieur sont très similaires
à ceux que nous avons vu dans le notificateur \lct{PreDebugWindow}.
Cependant, le Debugger combine la trace de la pile avec un navigateur de
méthode, ainsi quand vous sélectionnez une ligne dans le \emph{stack
trace}, la méthode correspondante s'affiche dans le panneau inférieur.
Vous devez absolument comprendre que l'exécution qui a causée l'erreur
est toujours dans l'image mais dans un état suspendu.
Chaque ligne de la trace représente une tranche de la pile
d'exécution qui contient toutes les informations nécessaires
pour poursuivre l'exécution. Ceci comprend tous les objets impliqués
dans le calcul, avec leurs variables d'instance et toutes les variables
temporaires des méthodes exécutées.

Dans \figref{debuggerDetectIfNone} nous avons sélectionné
la méthode \ct{detect:ifNone:} dans le panneau supérieur.
Le corps de la méthode est affiché dans le panneau central;
%le surlignage?
la sélection bleue entourant le message \ct{value} nous montre
que la méthode actuelle a envoyé le message \ct{value} et
attend une réponse.

Les quatre panneaux inférieurs du débogueur sont véritablement deux
mini-inspecteurs (sans panneaux-espace de travail).
L'inspecteur de gauche affiche l'objet actuel,
c'est-à-dire l'objet nommé \self dans le panneau central.
En sélectionnant différentes lignes de la pile, l'identité de \self
peut changer ainsi que le contenu de
l'inspecteur du \self{}.
Si vous cliquez sur \self dans le panneau inférieur gauche, vous verrez
que \self est un intervalle \ct{(10 to: 1 by -1)}, ce à quoi nous devions
nous attendre.
Les panneaux Workspace ne sont pas nécessaires dans les mini-inspecteurs
de Debugger car toutes les variables sont aussi à portée dans
le panneau de méthode; vous pouvez entrer et évaluer à loisir 
n'importe quelle expression.
Vous pouvez toujours annuler vos changements en utilisant 
\menu{cancel (l)} dans le menu ou en tapant \short{\textit{l}}. 
% apb: that lower-case-L is in italics so that it doesn't look like a 1 or a |
\index{raccourci clavier!cancel}

L'inspecteur de droite affiche les variables temporaires du contexte courant.
\arelire{%
% ajout vf martial
Sur \figref{debuggerDetectIfNone},
nous voyons 
dans le panneau de méthode au centre du débogueur que le message
\ct{value} a été envoyé au paramètre-bloc \ct{exceptionBlock} passé en argument de la
méthode \ct{detect:ifNone:};
ce paramètre se retrouve bien dans notre troisième colonne de variables temporaires.}
% rene : il nous faudrait clarifier l'explication
% martial: et voilà! (j'ai préféré rattaché au paragraphe du haut)
\arelire{Comme nous pouvons le voir sur la méthode \ct{detect:} plus bas sur la
  pile, notre paramètre \ct{exceptionBlock} est \ct{[self errorNotFound: aBlock]}. Il n'y
a donc rien de surprenant à voir le message d'erreur correspondant.}

Si vous voulez ouvrir un inspecteur complet sur une des variables
affichées dans les mini-inspecteurs, vous n'avez qu'à double-cliquer
sur le nom de la variable ou alors sélectionner le nom de la variable et 
\actclickz pour choisir
 \menu{inspect (i)} ou \menu{explore (I)}: % ce raccourci-clavier ``I'' est toujours valable ici
utile si vous voulez suivre le changement d'une variable lorsque vous exécutez un autre code.
\index{raccourci clavier!inspect it}
\index{raccourci clavier!explore it}

\arelire{% modif
En revenant sur le panneau de méthode, nous voyons que nous
nous attendions à trouver \ct{dot} dans la chaîne de caractère
\ct{'readme.txt'} à l'avant-dernière 
ligne de la méthode et que l'exécution n'aurait jamais du atteindre 
la dernière ligne.}
\pharo ne nous permet pas de lancer une exécution en arrière mais
il permet de \emph{relancer une méthode}, ce qui marche parfaitement dans
notre code qui ne change pas les objets mais qui en crée de nouveaux.

\dothis{\arelire{En restant sur la ligne \ct{detect:ifNone:}, cliquez sur 
le bouton \button{Restart} et vous verrez que 
l'exécution retournera dans l'état premier de la méthode courante.}
La sélection bleue englobe maintenant le message  
\mbox{\ct{do:}} (voir \figref{RestartDetectIfNone}).}

\begin{figure}[btp]
	\begin{center}
	\ifluluelse
		{\includegraphics[width=\textwidth]{RestartDetectIfNone}}
		{\includegraphics[width=0.7\textwidth]{RestartDetectIfNone}}
	\end{center}
	\caption{Debugger après avoir relancé la méthode \ct{detect: ifNone:}.}
	\figlabel{RestartDetectIfNone}
\end{figure}

Les boutons \button{Into} et \button{Over} offrent deux façons différentes de 
parcourir l'exécution pas-à-pas.
Si vous cliquez sur le bouton \button{Over} \arelire{(en français, ``par dessus'')},
\pharo exécutera sauf erreur l'envoi du message actuel (dans notre cas \ct{do:}) d'un
seul pas (en anglais, \emph{step}).
Ainsi \button{Over} nous amènera sur le prochain message à envoyer dans la méthode courante.
Ici nous passons à \ct{value}\,---\,c'est exactement l'endroit d'où nous avons démarré
et ça ne nous aide pas beaucoup.
En fait, nous avons besoin de trouver pourquoi \ct{do:} ne trouve pas
le caractère que nous cherchons.

\dothis{Après avoir \clicke sur le bouton \button{Over}, \clickz sur le 
bouton \button{Restart} pour revenir encore une fois au début de la méthode
dans le même état que sur \figref{RestartDetectIfNone}.}

%ajout vf
\arelire{Après ce coup pour rien, essayons de parcourir l'exécution autrement. Pour ce faire, nous utiliserons le bouton \button{Into} (en français, ``dedans'') permettant de rentrer dans la méthode pour une exécution pas-à-pas.}

\dothis{Cliquez sur le bouton \button{Into}; \pharo ira dans la méthode correspondante 
au message surligné par la sélection bleue; dans ce cas, \ct{Collection>>>do:}.}

\arelire{Malheureusement,} ceci ne nous aide pas plus: nous pouvons être confiant
dans le fait que la méthode \ct{Collection>>>do:} n'est pas erronée. 
Le bug se situe plutôt dans \emph{ce que} nous demandons à \pharo de faire.
%\emph{what}
\button{Through} (en français, ``à travers'') est le bouton approprié à ce cas: nous
voulons ignorer les détails de \ct{do:} lui-même et se focaliser sur
l'\emph{exécution du bloc}, argument de \ct{do:}.

\dothis{Sélectionnez encore la méthode \ct{detect:ifNone:} et
\clickz{} sur le bouton \button{Restart} pour revenir à l'état de
\figref{RestartDetectIfNone}.
\Clickz maintenant sur le bouton \button{Through} plusieurs fois. 
Sélectionnez \ct{each} dans le mini-inspecteur de contexte (en bas à droite).
%in the context window as you do so.
Vous remarquez que \ct{each} décompte depuis \ct{10} au fur et à mesure de
l'exécution de la méthode \ct{do:}.}

Quand \ct{each} est \ct{7} 
% ajout vf 
\arelire{(normalement après sept \clickbtn{}s sur \button{Through}\,--\,si vous êtes perdus vous pouvez toujours redémarrer en \clickant{} sur le bouton \button{Restart}),}
nous nous attendons à ce que le bloc \ct{ifTrue:} soit exécuté, mais ce n'est pas le cas:
% ajout vf
\arelire{si vous \clickz{} encore sur le bouton \button{Through}, vous passerez à \ct{6} comme
si notre point (\ct{dot}) n'était pas vu.}
Pour voir ce qui ne marche pas, \arelire{rendez-vous
après septième \clickbtn sur le bouton \button{Through} dans l'état illustré par
\figref{steppingIntoValue}. De là, allez \emph{dans} l'exécution de 
\ct{value:}:
% ajout vf
\clickz{} pour ce faire sur le bouton \button{Into}}.

\begin{figure}[btp]
	\begin{center}
	\ifluluelse
		{\includegraphics[width=\textwidth]{steppingIntoValue}}
		{\includegraphics[width=0.7\textwidth]{steppingIntoValue}}
	\end{center}
	\caption{Debugger après un \emph{pas} \arelire{\button{Into}} dans la méthode
      \ct{do:} plusieurs fois grâce au bouton \button{Through}.}
	\figlabel{steppingIntoValue}
\end{figure}

\arelire{%
Après avoir \clicke{} sur le bouton \button{Into}, 
vous obtiendrez une fenêtre de Debugger dans la même position que sur
\figref{dotIsAString}.}
Tout d'abord, il semble que nous soyons \emph{revenus} à la méthode 
\ct{suffix} mais c'est parce que nous exécutons désormais le bloc
que \ct{suffix} fourni en argument au message \ct{detect:}.
%\on{does not work any more! the debugger does not know about block variables!}  
% Si vous sélectionnez \ct{i} dans le mini-inspecteur contextuel, 
% vous pouvez voir sa valeur actuelle, qui devrait être \ct{7} 
% si vous avez suivi jusqu'ici la procédure.
% Vous pouvez alors sélectionner l'élément correspondant de \self
% dans l'inspecteur de \self.
% %\self{}-inspector.
% Dans \figref{dotIsAString}, vous pouvez voir que l'élément
% \ct{7} de la chaîne de caractères est le caractère 46: ce
% n'est pas un caractère-point.
Si vous sélectionnez \ct{dot} dans l'inspecteur contextuel, 
vous verrez que sa valeur est \ct{'.'}.
Vous constatez maintenant qu'ils ne sont pas égaux: le septième caractère
de \ct{'readme.txt'} est pourtant un objet \ct{Character} (donc un caractère), 
alors que \ct{dot} est un \ct{String} (\ie une chaîne de caractères entre \emph{quotes}).

\begin{figure}[btp]
	\begin{center}
	\ifluluelse
		{\includegraphics[width=\textwidth]{dotIsAString}}
		{\includegraphics[width=0.7\textwidth]{dotIsAString}}
	\end{center}
	\caption{Debugger montrant pourquoi \ct{'readme.txt' at: 7} n'est pas égal à \ct{dot}.}
	\figlabel{dotIsAString}
\end{figure}

Maintenant nous pouvons mettre le doigt sur le bug, 
la correction~\footnote{En anglais, nous parlons de \emph{bug fix}.} 
est évidente: nous devons convertir \ct{dot} en un caractère avant de 
recommencer la recherche.
%before starting to search for it.  

\begin{figure}[btp]
 	\begin{center}
 	\ifluluelse
 		{\includegraphics[width=\textwidth]{revertDialog}}
 		{\includegraphics[width=0.7\textwidth]{revertDialog}}
 	\end{center}
 	\caption{Changer la méthode \ct{suffix} dans Debugger: demander
      la confirmation de la sortie du bloc interne. La boîte
      d'alerte nous dit: ``Je devrais revenir à la méthode d'où ce bloc est originaire. Est-ce bon?''. }
 	\figlabel{revertDialog}
 \end{figure}

\dothis{Changez le code directement dans le débogueur de 
façon à ce que l'affectation soit de la forme
\ct{dot := FileDirectory dot first}:
%ajout vf
\arelire{%
\ct{SequenceableCollection>>>first} renvoie le élément de la collection donc ici le premier caractère de la chaîne de caractères et ainsi, \ct{dot} correspond bien au caractère \ct{.} désormais}.
Acceptez la modifications via l'option \menu{accept} du menu contextuel.}
% rene propose : \menu{accept} pour accepter la modification  ?
% martial : pas utile; à ce stade le lecteur connait bien accept/print it/do it; bon! c'est fait

Puisque nous sommes en train d'exécuter le code dans un bloc à
l'intérieur d'un \lct{detect:}, plusieurs trames de la pile 
devront être abandonnées de manière à valider le changement.
\pharo nous demande si c'est ce que nous voulons (voir \figref{revertDialog})
et, à condition de \click{} sur \menu{yes}, \pharo sauvegardera
(et compilera) la nouvelle méthode.

%\dothis{Cliquez sur le bouton \button{Restart} et ensuite \button{Proceed}; Debugger disparaîtra et l'évaluation de l'expression \ct{'readme.txt' suffix} sera complète et affichera la réponse \ct{'.txt'}}

\arelire{L'évaluation de l'expression \ct{'readme.txt' suffix} sera complète et affichera la réponse \ct{'.txt'}.} 

Est-ce pour autant une réponse correcte?  Malheureusement nous ne pouvons
répondre avec certitude.
%Unfortunately, we can't say for sure.  
Le suffixe devrait-il être \ct{.txt} ou \ct{txt}?
Le commentaire dans la méthode \ct{suffix} n'est pas très précis.
% note de martial: commentaire traduit car en reference
La façon d'éviter ce type de problème est d'écrire
un test \ind{SUnit} pour définir la réponse.

\begin{method}[testSuffix]{Un simple test pour la méthode \ct{suffix}}
testSuffixFound
	self assert: 'readme.txt' suffix = 'txt'
\end{method}

L'effort requis pour ce faire est à peine plus important que celui
qui consiste à lancer le même test dans un espace de travail;
l'avantage de \sunit est de sauvegarder ce test sous la forme d'une
documentation exécutable et de faciliter l'accessibilité des usagers
de la méthode.
% note de martial: j'ai retourne un peu le sens de la phrase: The effort required to do that was little more than to run the same test in the workspace, but using \sunit saves the test as executable documentation, and makes it easy for others to run.
% rene approuve
En plus, si vous ajoutez \tmthref{testSuffix} à la classe
\ct{StringTest} et que vous lancez ce test avec \sunit, vous
pouvez très facilement revenir pour déboguer l'actuelle erreur.
% very quickly get back to debugging the error.
\sunit ouvre Debugger sur l'assertion fautive mais là vous
avez simplement besoin de descendre d'une ligne dans la pile,
% you need only go back down the stack one frame,
redémarrez le test avec le bouton \button{Restart} et allez
dans la méthode \ct{suffix} par le bouton \button{Into}. Vous
pouvez alors corriger l'erreur, comme nous l'avons fait dans
\figref{fixOffByOne}.
Il s'agit maintenant de cliquer sur le bouton \button{Run Failures} dans
le \sunit Test Runner et de se voir confirmer que le test passe (en anglais, \emph{pass}) 
normalement. Rapide, non?

\begin{figure}[btp]
	\begin{center}
		\includegraphics[width=\textwidth]{fixOffByOne}
	\end{center}
	\caption{Changer la méthode \ct{suffix} dans Debugger: corriger l'erreur du plus-d'un-point après l'assertion fautive \sunit.}
	\figlabel{fixOffByOne}
\end{figure} % CHANGE

Voici un meilleur test:

\begin{method}[testSuffix2]{Un meilleur test pour la méthode \ct{suffix}}
testSuffixFound
	self assert: 'readme.txt' suffix = 'txt'.
	self assert: 'read.me.txt' suffix = 'txt'
\end{method}
\noindent
Pourquoi ce test est-il meilleur? Simplement parce que
nous informons le lecteur de ce que la méthode devrait faire 
s'il y a plus d'un point dans la chaîne de caractères, instance de String.

Il y a d'autres moyens d'obtenir une fenêtre de débogueur en plus de ceux
qui consistent à capturer une erreur effective ou à faire une assertion
fautive (ou \emph{assertion failures}).
Si vous exécutez le code qui conduit à une boucle infinie, vous pouvez
l'interrompre et ouvrir un débogueur durant le calcul en tapant \short{.}%
% (that's a full stop or a period, depending  on where you learned English).
~\footnote{Sachez que vous pouvez ouvrir un débogueur d'urgence n'importe quand en tapant
\short{{\sc shift}.}}
Vous pouvez aussi éditer simplement le code suspect en insérant l'expression \ct{self halt}.
Ainsi, par exemple, nous pourrions éditer la méthode \ct{suffix} comme suit:
\index{processus!interruption}

\needspace{11ex}
\begin{method}[suffix]{Insérer une pause par \ct{halt} dans la méthode \ct{suffix}}
suffix
	"disons que je suis un nom de fichier et que je fournis mon suffixe, la partie suivant le dernier point"
	| dot dotPosition |
	dot := FileDirectory dot first.
	dotPosition := (self size to: 1 by: -1) detect: [ :i | (self at: i) = dot ].
	self halt.
	^ self copyFrom: dotPosition to: self size 
\end{method}

Quand nous lançons cette méthode, l'exécution de \ct{self halt} ouvre 
un \ind{notificateur} ou \emph{\ind{pre-debugger}} d'où nous pouvons continuer 
%ajout
en cliquant sur \menu{proceed}
ou déboguer et explorer l'état des variables, parcourir pas-à-pas la pile d'exécution et éditer le code.

C'est tout pour le débogueur mais nous n'en avons pas fini avec la méthode \ct{suffix}.
Le bug initial aurait dû vous faire réaliser que s'il n'y a pas de point dans la chaîne 
cible la méthode \ct{suffix} lèvera une erreur.
Ce n'est pas le comportement que nous voulons. Ajoutons ainsi un second test
pour signaler ce qu'il pourrait arriver dans ce cas.  

\begin{method}[testNoSuffix]{Un second test pour la méthode \ct{suffix}: la cible n'a pas de suffixe}
testSuffixNotFound
	self assert: 'readme' suffix = ''
\end{method}

\dothis{Ajoutez \tmthref{testNoSuffix} à la suite de tests dans la classe \clsind{StringTest} 
et observez l'erreur levée par le test.
Entrez dans Debugger en sélectionnant le test erroné dans \sunit puis éditez
le code de façon à passer normalement le test (donc sans erreur).
La méthode la plus facile et la plus claire consiste à remplacer le message \ct{detect:} 
par \ct{detect: ifNone:}~\footnote{En anglais, \emph{if none} signifie ``s'il n'y a rien''.}  
où le second argument un bloc qui retourne tout simplement une chaîne.}

Nous en apprendrons plus sur SUnit dans \charef{SUnit}.

% section debugger (end)

%=========================================================
\section{Le navigateur de processus}
% Process Browser

\st est un systême multitâche: plusieurs processus légers (aussi
connu sous le nom de \emph{threads}) tournent simultanément dans
votre image.
%fonctionnent de façon concourrante dans votre image.
Dans l'avenir la machine virtuelle de \pharo bénéficiera davantage
des multi-processeurs lorsqu'ils seront disponibles, mais le partage
d'accès est actuellement programmé sur le principe de 
tranches temporelles (ou \emph{time-slice}).
%concurrency is implemented by time-slicing.

\begin{figure}[btp]
	\begin{center}
	\ifluluelse
		{\includegraphics[width=\textwidth]{processBrowser}}
		{\includegraphics[width=0.7\textwidth]{processBrowser}}
	\end{center}
	\caption{Le Process Browser.}
	\figlabel{processBrowser}
\end{figure}

Le Process \subind{processus}{Browser} ou navigateur de \subind{navigateur}{processus} 
est un cousin de Debugger qui vous permet d'observer les divers processus tournant
dans le système \pharo.
\Figref{processBrowser} nous en présente une capture d'écran.
Le panneau supérieur gauche liste tous les processus présents dans \pharo, 
dans l'ordre de leur priorité depuis le \emph{timer interrupt watcher} 
%?
(système de surveillance d'interruption d'horloge) de priorité
80 au \emph{idle process} ou processus inactif du système de priorité 10.
Bien sûr, sur un système mono-processeur, le seul processus pouvant être 
lancé en phase de visualisation est le \emph{UI~\footnote{UI désigne 
\emph{User Interface}; en français, interface utilisateur.} process} 
ou processus graphique;
%when you look is the UI process; 
tous les autres processus seront en attente d'un quelconque événement.
%:===> Process browser context menu is broken!
\on{broken -- to be fixed!}
Par défaut, l'affichage des processus est statique; il peut être
mis à jour en \actclickant{} et en sélectionnant \menu{turn on auto-update (a)}. % CHANGE

Si vous sélectionnez un processus dans le panneau supérieur gauche, le panneau de droite
affichera son \emph{stack trace} tout comme le fait le débogueur.
Si vous en sélectionnez un, la méthode correspondante est affichée dans le panneau
inférieur.
Le Process Browser n'est pas équipé de mini-inspecteurs pour 
\self et \lct{thisContext} mais \actclick{} sur les
tranches de la pile offre une fonctionnalité  équivalente. % CHANGE

%=========================================================
\section{Trouver les méthodes}
\seclabel{methodFinder} 

Il y a deux outils dans \pharo pour vous aider à trouver des
messages. %CHANGE
Ils diffèrent en termes d'interface et de fonctionnalité.

Le \emph{Method Finder} (ou chercheur de méthodes) a été longuement décrit
dans \secref{quick:methodFinder}; vous pouvez l'utiliser pour trouver des méthodes
par leur nom ou leur fonction.
Cependant, pour observer le corps d'une méthode, le Method Finder ouvre 
un nouveau navigateur. Cela peut vite devenir pénible.
% overwhelming.

\begin{figure}[btp]
	\begin{center}
	\ifluluelse
		{\includegraphics[width=\textwidth]{methodNamesRandom}}
		{\includegraphics[width=0.7\textwidth]{methodNamesRandom}}
	\end{center}
	\caption{Le Message Names Browser montrant toutes les méthodes
      contenant le sous-élement de chaîne \ct{random} dans leur
      sélecteur.} % CHANGE : Method Names Browser -> Message Names Browser
	\figlabel{methodNamesRandom}
\end{figure}

\index{Message Names Browser}
\seeindex{navigateur de noms de messages}{Message Names Browser}

La fonctionnalité de recherche du Message Names Browser ou navigateur de 
\emph{noms de messages} est plus limitante: vous entrez un morceau d'un sélecteur de 
message dans la boîte de recherche et le navigateur liste toutes les méthodes
contenant ce fragment dans leurs noms, comme nous pouvons le voir dans
\figref{methodNamesRandom}.
Cependant, c'est un navigateur complet:
%full-fledged browser:
si vous sélectionnez un des noms dans le panneau de gauche, toutes les méthodes
ayant ce nom seront listées dans celui de droite et vous pourrez alors naviguer dans
le panneau inférieur.
Le Message Names Browser a une barre de bouton, comme le Browser, % CHANGE
pouvant être utilisée pour ouvrir d'autres navigateurs sur la méthode choisie
ou sur sa classe.
% section methodFinder (end)

%=========================================================
\section{Change set et son gestionnaire Change Sorter}
% Change sets and the Change Sorter
\seclabel{env:changeSet} % (fold)

Chaque fois que vous travaillez dans \pharo, tous les changements que vous effectuez
sur les méthodes et les classes sont enregistrés dans un
\ct{change set} (traduisible par ``ensemble des modifications'').
%note temporaire de martial: je dois changer sauf contre-ordre change set en \ct{change set} sans traduction
Ceci inclus la création de nouvelles classes, le renommage de classes, le changement de
catégories, l'ajout de méthodes dans une classe existante\,---\,en bref, tout ce qui a un impact sur le système.
Cependant, les exécutions arbitraires avec \emph{do it} ne sont pas
incluses; par exemple, si vous créez une nouvelle variable globale par affectation dans 
un espace de travail, la création de variable ne sera pas dans un 
\subind{fichier}{change set}.
%%\index{Change Set Browser}
%%\seeindex{navigateur de change set}{Change Set Browser}
\index{Change Sorter}

A tout moment, beaucoup de \changesets existent, mais un seul d'entre eux\,---\,\ct{ChangeSet current}\,---\,collecte les changements qui sont en cours dans l'image actuelle.
Vous pouvez voir quel \changeset est le \changeset actuel et vous pouvez examiner
tous les \changesets en utilisant le Change Sorter
(ou trieur de \changeset) disponible dans le menu
principal dans \menu{World\go{} Tools \ldots \go{} Change Sorter}.

\begin{figure}[btp]
	\begin{center}
		\includegraphics[width=\linewidth]{changeSorter}
	\end{center}
	\caption{Le Change Sorter.}
	\figlabel{changeSorter}
\end{figure}

\Figref{changeSorter} nous montre ce navigateur. La barre de titre affiche le \changeset actuel et ce \changeset est sélectionné quand le navigateur s'ouvre.

Les autres \changesets peuvent être choisis dans le panneau supérieur de gauche;
le menu contextuel accessible via le \ind{bouton jaune} vous permet de faire de
n'importe quel \changeset votre \changeset actuel ou de créer un nouveau \changeset.
Le panneau supérieur de droite liste toutes les classes 
(accompagnées de leurs catégories) affectées par le \changeset sélectionné.
Sélectionner une des classes affiche les noms de ses méthodes qui sont aussi dans
le \changeset (\emph{pas} toutes les méthodes de la classe) dans le panneau central
et sélectionner un de ces noms de méthodes affiche sa définition dans le panneau
inférieur.
Remarquez que le navigateur ne montre \emph{pas} si la création de la classe elle-même
fait partie du \changeset bien que cette information soit stockée dans la structure
de l'objet qui est utilisé pour représenter le \changeset.

Le Change Sorter vous permet d'effacer des classes et des méthodes du \changeset
en  \actclickant{} sur les élements correspondants.
% Cependant, pour une édition plus élaborée, vous devez utiliser un deuxième
% programme, le \textit{Change Sorter} (ou trieuse de \changeset), disponible sous ce nom dans 
% \toolsflap ou en passant par \menu{World\go{}open...\go{}dual change sorter}. Nous
% pouvons le voir dans \figref{changeSorter}.

Le Change Sorter vous permet de voir simultanément deux
\changesets: un \changeset à gauche et un autre à droite.
Cette fonctionalité offre les principales fonctions du Change Sorter 
% rene propose :  \arevoir{disposition offre} 
telles que la possibilité de déplacer ou copier les changements d'un \changeset à un autre,
comme nous pouvons le voir sur \figref{changeSorter},
dans le menu contextuel accessible en \actclickant.
Nous pouvons aussi copier des méthodes d'un pan à un autre.

Vous pouvez vous demander pourquoi vous devez accorder de l'importance à la composition
d'un \changeset: la réponse est que les \changesets fournissent un mécanisme simple
pour exporter du code depuis \pharo vers le système de fichiers d'où il peut
être importé dans une autre image \pharo ou vers un autre \st que \pharo.
L'exportation de \changeset est connu sous le nom ``filing-out'' et peut être réalisé
en utilisant le menu contextuel obtenu en \actclickant{} sur n'importe quel \changeset, classe ou
méthode dans n'importe quel navigateur.
Des exportations (ou fileouts) répétées créent une nouvelle version du fichier
mais les \changesets ne sont pas un outil de versionnage (gestion de
versions) comme peut l'être Monticello:
ils ne conservent pas les dépendances.
\index{fichier!filing-out}
\index{fichier!exportation}

Avant l'avènement de Monticello, les \changesets étaient la
technique majeure d'échange de code entre les Smalltalkiens. % AREVOIR
%%utilisateurs %de \pharo (en anglais \emph{\pharo{}ers}). % avant 33691

% [squeak-fr] \sq{}ers: squeakiens, squeakois, squeakais, squeakeux?
Ils ont l'avantage d'être simples et relativement portables (le fichier d'exportation 
n'est qu'un fichier texte; \emph{nous ne vous recommandons pas} d'éditer ce fichier 
avec un éditeur de texte).
Il est assez facile aussi de créer un \changeset qui modifie
beaucoup de parties différentes du système sans aucun rapport
entre elles\,---\,ce pour quoi Monticello n'est pas encore équipé.
% martial: ou 'ce pourquoi'; je crois que les deux marches
% Rene : ce pour quoi est correct
%\ab{Or is it?}
%\on{you mean something different than extensions to foreign packages using the *package protocol notation?}

Le principal inconvénient des \changesets par rapport aux paquetages \ind{Monticello}
est leur absence de notion de dépendances.
Une exportation de \changeset est un ensemble d'\emph{actions} transformant n'importe quelle
image dans laquelle elle est chargée. Pour en charger avec succès, l'image doit être
dans un état approprié.
Par exemple, le \changeset pourrait contenir une action pour ajouter une méthode à une
classe; ceci ne peut être fait que si la classe est déjà définie dans l'image.
De même, le \changeset pourrait renommer ou re-catégoriser une classe, ce qui ne 
fonctionnerait évidemment que si la classe est présente dans l'image; les méthodes
pourraient utiliser des variables d'instance déclarées lors de l'exportation mais
inexistantes dans l'image dans laquelle elles sont importées.
Le problème est que les \changesets ne contiennent pas explicitemment les conditions 
sous lesquelles ils peuvent être chargés:
le fichier en cours de chargement marche \emph{au petit bonheur la chance} jusqu'à
ce qu'un message d'erreur énigmatique et un \emph{stack trace} surviennent
quand les choses tournent mal.
%the file in process just hopes for the best, usually resulting in a cryptic error message and a stack trace when things go wrong.
Même si le fichier fonctionne, un \changeset peut annuler silencieusement 
un changement fait par un autre.

À l'inverse, les paquetages (dits aussi packages) de Monticello représentent le code d'une manière 
déclarative: ils décrivent l'état que l'image devrait avoir une fois le chargement
effectué.
Ceci permet à Monticello de vous avertir des conflits (quand deux paquetages ont des
objectifs incompatibles)
%require contradictory final states
et vous permet de charger une série de paquetages dans un ordre de dépendances.

Malgré ces imperfections, les \changesets reste utiles; vous pouvez, en particulier, en trouver sur Internet pour en observer le contenu, voire les utiliser.
Maintenant que nous avons vu comment exporter des \changesets avec le Change Sorter,
nous allons voir comment les importer.
Cette étape requiert l'usage d'un autre outil, le File List Browser.
% section changeSet (end)

%=========================================================
\section{Le navigateur de fichiers File List Browser}

\begin{figure}[btp]
	\begin{center}
	\ifluluelse
		{\includegraphics[width=\textwidth]{fileList}}
		{\includegraphics[width=0.7\textwidth]{fileList}}
	\end{center}
	\caption{Le File List Browser.}
	\figlabel{fileList}
\end{figure}

Le navigateur de \subind{navigateur}{fichiers} ou \ind{File List Browser} est
en réalité un outil générique pour naviguer au travers d'un système de fichiers
(et aussi sur des serveurs FTP) depuis \pharo.
Vous pouvez l'ouvrir depuis le menu \menu{World \go{}Tools \ldots{}
  \go{}File Browser}. % CHANGE
Ce que vous y voyez dépend bien sûr du contenu de votre système de fichiers local
mais une vue typique du navigateur est illustrée sur 
\figref{fileList}.
\seeindex{fichier!navigation}{File List Browser}

Quand vous ouvrez un navigateur de fichiers, il pointera tout d'abord le répertoire
actuel, \ie celui depuis lequel vous avez démarré \pharo. La barre de titre
montre le chemin de ce répertoire.
Le panneau de gauche est utilisé pour naviguer dans le système de fichiers de manière
conventionnelle.
% note de martial; j'ai enleve le terme: larger pane parce que ce n'est pas forcement le plus grand 

Quand un répertoire est sélectionné, les fichiers qu'ils contiennent (mais pas les répertoires) 
sont affichés sur la droite. Cette liste de fichiers peut être filtrée en entrant dans la petite boîte
dans la zone supérieure gauche de la fenêtre un modèle de filtrage ou 
\emph{pattern} dans le style Unix.
Initialement, ce \emph{pattern} est \ct{*}, ce qui est égal à l'ensemble des fichiers, mais vous 
pouvez entrer une chaîne de caractères différente et l'accepter pour changer ce filtre. 
%ajout/sans les parentheses
Notez qu'un \ct{*} est implicitement joint ou pré-joint au \emph{pattern}
que vous entrez.
L'ordre de tri des fichiers peut être modifié via les boutons \button{name} (par nom), 
\button{date} (par date) et \button{size} (par taille).
Le reste des boutons dépend du nom du fichier sélectionné dans le navigateur.
Dans \figref{fileList}, le nom des fichiers a le suffixe \ct{.cs}, donc le navigateur
suppose qu'il s'agit de \changeset et ajoute les boutons \button{install} (pour
\textit{l'importer} dans un nouveau \changeset dont le nom est dérivé
de celui du fichier), \button{changes} (pour naviguer dans le changement du fichier),
\button{code} (pour l'examiner) et \button{filein} (pour charger
le code dans le \changeset \emph{actuel}).
Vous pourriez penser que le bouton \button{conflicts} vous informerait 
des modifications du \changeset pouvant être source de conflits dans le code existant
dans l'image mais ça n'est pas le cas.
\ab{Does anyone know what it does do?  I've never found it useful.}
\on{I tried it and found that it complained about linefeeds.}
En réalité, il vérifie juste d'éventuels problèmes dans le fichier (tel que
la présence de sauts de lignes ou \textit{linefeeds})
pouvant indiquer qu'il ne pourrait pas être proprement chargé.

\begin{figure}[btp]
	\begin{center}
	\ifluluelse
		{\includegraphics[width=\textwidth]{fileContentsBrowser}}
		{\includegraphics[width=0.7\textwidth]{fileContentsBrowser}}
	\end{center}
	\caption{Le File Contents Browser.}
	\figlabel{fileContentsBrowser}
\end{figure}

Puisque le choix des boutons affichés dépend du \emph{nom} du fichier et
non de son contenu, parfois le bouton dont vous avez besoin pourrait ne pas être
affiché.
De toutes façons, le jeu complet des options est toujours disponible
grâce
% martial: "de toute façon" est plus correct selon l'Académie française
à l'option \menu{more \ldots} du menu contextuel accessible en \actclickant,
ainsi vous pouvez facilement contourner ce problème.

Le bouton \button{code} est certainement le plus utile pour travailler avec les \changesets;
il ouvre un navigateur sur le contenu du fichier. Un exemple est présenté dans
\figref{fileContentsBrowser}.
Le File Contents Browser est proche d'un Browser à l'exception
des catégories; seuls les classes, les protocoles et les méthodes sont présentés.
Pour chaque classe, ce navigateur précise si la classe existe déjà dans
le système ou non et si elle est définie dans le fichier (mais \emph{pas} si
les définitions sont identiques).
Il affichera les méthodes de chaque classe
% note: doublon de 'the' a signaler dans l'original
ainsi que les différences entre la version actuelle et celle dans le fichier; ce que
nous montre \figref{fileContentsBrowser}.
Les options du menu contextuel de chacun des quatre panneaux supérieurs vous
permettront de charger (en anglais, \emph{file in}) le \changeset complet, la classe, 
le protocole ou la méthode correspondante.

%=========================================================
\section{En Smalltalk, pas de perte de codes}
%you can't lose code
\seclabel{cantLoseCode} % (fold)

\pharo peut parfois planter: en tant que système expérimental, \pharo vous permet
de changer n'importe quoi dont les élements vitaux qui font que \pharo fonctionne!
%It is quite possible to crash \pharo

\dothis{Pour \emph{crasher} malicieusement \pharo, évaluez \ct{Object become: nil}.}

La bonne nouvelle est que vous ne perdez jamais votre travail, même si votre 
image plante et revient dans l'état de la dernière version sauvegardée il y 
a de cela peut être des heures.
La raison en est que tout code exécuté est sauvegardé dans le fichier
\emph{.changes}.
Tout ceci inclut les expressions que vous évaluez dans un espace de travail Workspace,
tout comme le code que vous ajoutez à une classe en la programmant.
\index{fichier!changes} %voir QuickTour.tex

Ainsi, voici les instructions permetant de retrouver ce code.
Il n'est pas utile de lire ce qui suit tant que vous n'en avez pas besoin.
Cependant, quand vous en aurez besoin, vous saurez où le trouver.

Dans le pire des cas, vous pouvez toujours utiliser un éditeur de texte
sur le fichier \emph{.changes}, mais quand celui-ci pèse plusieurs méga-octets,
cette technique pourrait s'avérer lente et peu recommandable.
\pharo vous offre de meilleures façons de vous en sortir.

%---------------------------------------------------------
\subsection{La pêche au code}
%\subsection{Comment ramener du code}

Redémarrez \pharo depuis la sauvegarde (ou \emph{snapshot}) la plus récente et
sélectionnez \menu{World\go{}Tools \ldots \go{}Recover lost changes}. 
% Ceci vous ouvrira un Workspace plein d'expressions utiles. Les trois premières,

% \begin{code}{}
% Smalltalk recover: 10000.
% ChangeList browseRecentLog.
% ChangeList browseRecent: 2000.
% \end{code}

% \noindent
% sont les plus utiles pour le recouvrement de données.

% Si vous exécutez \ct{ChangeList browseRecentLog},
Vous aurez ainsi l'opportunité
de décider jusqu'où vous souhaitez revenir dans l'historique.
Normalement, naviguer dans les changements depuis la dernière sauvegarde
est suffisant (et vous pouvez obtenir le même effet en éditant 
\ct{ChangeList browseRecent: 2000} en tâtonnant sur le chiffre empirique
\ct{2000}).

Une fois que vous avez le navigateur des modifications récentes nommé
\emph{Recent Changes Browser} vous affichant les changements, disons, depuis votre
dernière sauvegarde, vous aurez une liste de tout ce que vous avez effectué 
dans \pharo durant tout ce temps.
Vous pouvez effacer des articles de cette liste en utilisant le menu
accessible en \actclickant.
Quand vous êtes satisfait, vous
pouvez charger (c'est-à-dire faire un \emph{file-in}) ce qui a été laissé
et ainsi incorporer les modifications dans un nouveau \changeset.

Une chose utile à faire dans le \emph{Recent Changes} Browser est
d'effacer les évaluations \emph{do it} via \menu{remove doIts}. 
Habituellement vous ne voudriez pas charger (c'est-à-dire re-exécuter) ses expressions.
Cependant, il existe une exception.
Créer une classe apparaît comme un \menu{doIt}.
\emph{Avant de charger les méthodes d'une classe, la classe doit exister.}
Donc, si vous avez créer des nouvelles classes, chargez \emph{en premier lieu} 
les \emph{doIts} créateur de classes, ensuite utilisez \menu{remove doIts} 
%ajout
(pour ne pas charger les expressions d'un espace de travail)
et enfin charger les méthodes.
%\lr{Maybe mention that class renames are not logged and completely screw up the change-set mechanism. (p. 174)}

Quand j'en ai fini avec le recouvrement (en anglais, \emph{recover}), 
j'aime exporter (par \emph{file-out}) mon nouveau \changeset, quitter \pharo
sans sauvegarder l'image, redémarrer et m'assurer que mon nouveau fichier
se charge parfaitement.
% section cantLoseCode (end)

%=========================================================
\section{Résumé du chapitre}

Pour développer efficacement avec \pharo, il est
important d'investir quelques efforts dans l'apprentissage des outils
disponibles dans l'environnement.

\begin{itemize}
  \item Le navigateur de classes standard ou \emph{Browser} est votre principale interface pour naviguer dans les catégories, les classes, les protocoles et les méthodes existants et pour en définir de nouveaux.
Ce navigateur offre plusieurs onglets pour accéder directement aux \senders (\ie{} les méthodes émettrices) ou aux \implementors (\ie{} les méthodes contenantes) d'un message, aux versions d'une méthode, 
\etc.
  \item Plusieurs navigateurs différents existent
(comme l'OmniBrowser )  
et plusieurs sont spécialisés (comme le Hierarchy Browser) pour fournir différentes vues sur les classes et les méthodes.
  \item Depuis n'importe quel outil, vous pouvez sélectionner en surlignant le nom d'une classe ou celui d'une méthode pour obtenir immédiatemment
un navigateur en utilisant le raccourci-clavier \short{b}.
  \item Vous pouvez aussi naviguer dans le système \st de manière
programmatique en envoyant des messages à \ct{SystemNavigation default}.
  \item \emph{Monticello} est un outil d'import-export, de versionnage
    (organisation et maintien de versions, en anglais, \emph{versioning}) et de partage de \emph{paquetages} de classes et de méthodes nommés aussi \emph{packages}.
  Un paquetage Monticello comprend une catégorie, des sous-catégories et des protocoles de méthodes associés dans d'autres catégories. 
  \item L'\emph{Inspector} et l'\emph{Explorer} sont deux outils utiles
pour explorer et interagir avec les objets vivants dans votre image.
Vous pouvez même inspecter des outils en \actclickant{} pour
afficher leur \emph{halo} et en sélectionnant l'icône
\emph{debug} \debugHandle .
  \item Le \emph{Debugger} ou débogueur est un outil qui non seulement vous permet
d'inspecter la pile d'exécution (\emph{runtime stack}) de votre
programme lorsqu'une erreur est signalée, mais aussi, 
vous assure une interaction avec tous les objets de votre application,
incluant le code source. Souvent, vous pouvez modifier votre 
code source depuis le Debugger et continuer l'exécution.
Ce débogueur est particulièrement efficace comme outil pour
le développement orienté test (ou, en anglais, \emph{test-first development}) en tandem
avec 
SUnit (\charef{SUnit}).
  \item Le \emph{Process Browser} ou navigateur de processus vous permet de piloter (monitoring), chercher (querying) et interagir avec les processus courants lancés dans votre image.
  \item Le \emph{Method Finder} et le \emph{Message Names Browser} sont 
deux outils destinés à la localisation de méthodes. Le
premier excelle lorsque vous n'êtes pas sûr du nom mais que vous
connaissez le comportement.
% note perso de martial: (attendu) 
Le second dispose d'une interface de navigation plus avancée pour le cas où
vous connaissez au moins une partie du nom.
  \item Les \changesets sont des journaux de bord (ou log) automatiquement générés pour 
tous les changements du code source dans l'image.
Bien que rendus obsolètes par la présence de Monticello comme
moyen de stockage et d'échange des versions de votre code source,
ils sont toujours utiles, en particulier pour réparer des erreurs
catastrophiques aussi rares soient-elles.
  \item Le \emph{File List Browser} est un programme pour parcourir le
système de fichiers. Il vous permet aussi d'insérer du code
source depuis le système de fichiers via \menu{fileIn}.
  \item Dans le cas où votre image plante~\footnote{Nous parlons de \emph{crash}, en anglais.} avant que
vous l'ayez sauvegardée ou que vous ayez enregistré le code source
avec Monticello, vous pouvez toujours retrouver vos modifications les
plus récentes en utilisant un \emph{Change List Browser}.
Vous pouvez alors sélectionner les changements ou \emph{changes} (en anglais) que vous voulez 
reprendre et les charger dans la copie la plus récente de votre image.
\end{itemize}

%=================================================================
\ifx\wholebook\relax\else\end{document}\fi
%=================================================================

%=========================================================
%---------------------------------------------------------

%:SUnit
% $Author: oscar $
% $Date: 2007-09-23 09:56:47 +0000 (Sun, 23 Sep 2007) $
% traduit par Alain Plantec
% relu par Martial Boniou (Sat Nov 24 14:53:40 CET 2007)  
% $Revision: 12130 $
% relu par Rene Mages (Fri Dec 21 8:57;55 CET 2007)
% $Revision: 14579 $ (fusion le: Wed Dec 26 18:15:25 CET 2007)
% relu par Rene Mages (Sun Jan 13 11:22:33 CET 2008)
% relu par Martial Boniou (Mon Feb  4 17:35:46 CET 2008)
% adaptation pour Pharo : martial - Fri Sep 11 13:23:52 CEST 2009 from
% $Author: oscar $ % $Date: 2009-08-28 10:41:41 +0200 (Fri, 28 Aug
% 2009) $ % $Revision: 28654 $
% sync avec la revision: 29170
%=================================================================
\ifx\wholebook\relax\else
% --------------------------------------------
% Lulu:
	\documentclass[a4paper,10pt,twoside]{book}
	\usepackage[
		papersize={6.13in,9.21in},
		hmargin={.75in,.75in},
		vmargin={.75in,1in},
		ignoreheadfoot
	]{geometry}
	\input{../common.tex}
	\pagestyle{headings}
	\setboolean{lulu}{true}
% --------------------------------------------
% A4:
%	\documentclass[a4paper,11pt,twoside]{book}
%	\input{../common.tex}
%	\usepackage{a4wide}
% --------------------------------------------
    \graphicspath{{figures/} {../figures/}}
	\begin{document}
%	\renewcommand{\nnbb}[2]{} % Disable editorial comments
	\sloppy
\fi
%=================================================================
%%martial (remarque): bonne idee d'Alain; a reutiliser ailleurs avec
%%une balise a la 'lulu'
\newcommand{\aconfirmer}[1]{#1}
\chapter{SUnit}
\chalabel{SUnit}

%=================================================================
\section{Introduction}

\on{Would be nice to have an example of test-driven development with SUnit from beginning to end. Perhaps this is for another chapter?}

\indmain{SUnit} est un environnement simple mais pourtant puissant 
pour la création et le déploiement de tests.
Comme son nom l'indique, \sunit est conçu plus particulièrement pour les \emph{tests unitaires},
mais en fait, il peut être aussi utilisé pour des tests d'intégration ou des tests fonctionnels. 
\sunit a été développé par Ken Beck et ensuite grandement étendu par d'autres développeurs, dont notamment Joseph Pelrine, avec la prise en compte de la notion de ressource décrite dans \secref{resource}. 
%\indmain{SUnit} is a minimal yet powerful framework that supports the
%creation and deployment of tests.
%but in fact it can be used for integration tests and functional tests as well.
%As might be guessed from its name, the design of \sunit focussed on \emph{Unit Tests}, 
%\sunit was originally developed by Kent Beck and subsequently extended by Joseph
%Pelrine and others to incorporate the
%notion of a resource, which we will describe in \secref{resource}.
\index{Beck, Kent}
\index{Pelrine, Joseph}
\seeindex{ressource}{test, ressource}
L'intérêt pour le test et le \ind{développement dirigé par les tests} ne se limite pas à \pharo ou \st.
L'automatisation des tests est devenue une pratique fondamentale des
méthodes de \ind{développement agile}{}s et 
tout développeur concerné par l'amélioration de la qualité du logiciel ferait bien de l'adopter.  
En effet, de nombreux développeurs apprécient la puissance du test unitaire et des versions de \xUnit{} sont maintenant disponibles pour de nombreux langages dont \ind{Java}, \ind{Python}, \ind{Perl}, .Net et \ind{Oracle}.
% The interest in testing and \ind{Test Driven Development}
% is not limited to \pharo or \st.  
% Automated testing has become a hallmark of the \ind{Agile software development} movement, and any
% software developer concerned with improving software quality would do well to adopt it.
% Indeed, developers in many languages have come to appreciate
% the power of unit testing, and versions of
% \xUnit{}  now exist for many languages, including \ind{Java}, \ind{Python}, \ind{Perl}, .Net and \ind{Oracle}.

\arevoir{} % martial: je ne comprends pas pourquoi il y avait ce lien
           % dans l'index
%\seeindex{Matrix!free will}{Oracle} % sorry, couldn't resist

%  OSCAR: There was a broken citation here for the xprogramming web site
% I could not figure out what it was supposed to refer to.

Ce chapitre décrit \SUnit~3.3 (la version courante lors de l'écriture de ce document); le site officiel de \sunit est \url{sunit.sourceforge.net}, dans lequel les mises à jour sont disponibles.
% This chapter describes \SUnit~3.3 (the current version as of this writing); the official web site of \sunit is
% \url{sunit.sourceforge.net}, where updates can be found.
\index{xUnit}
\seeindex{xUnit!SUnit}{SUnit}
\index{Net@.Net}

Le test et la construction de lignes de tests ne sont pas des pratiques nouvelles~: il est largement reconnu que les tests sont utiles pour débusquer les erreurs. En considérant le test comme une pratique fondamentale et en promouvant les tests \emph{automatisés}, l'\mbox{\ind{eXtreme Programming}} a contribué à rendre le test productif et excitant plutôt qu'une corvée routinière dédaignée des développeurs. La communauté liée à \st bénéficie d'une longue tradition du test grâce au style de programmation incrémental supporté par l'environnement de développement.
% Neither testing, nor the building of test suites, is new:  everybody knows that
% tests are a good way to catch errors.
% \mbox{\ind{eXtreme Programming},} by making testing a
% core practice and by emphasizing \emph{automated} tests, 
% has helped to make testing productive and fun, rather than a 
% chore that programmers dislike.
% The \st community has a long tradition of
% testing because of the incremental style of development supported by its
% programming environment.
Traditionnellement, un programmeur \st écrirait des tests dans un \ct{Workspace} dès qu'une méthode est achevée. Quelquefois, un test serait intégré comme commentaire en tête de méthode en cours de mise au point, ou bien les tests plus élaborés seraient inclus dans la classe sous la forme de méthodes exemples. L'inconvénient de ces pratiques est que les tests édités dans un \ct{Workspace} ne sont pas disponibles pour les autres développeurs qui modifient le code; les commentaires et les méthodes exemples sont de ce point de vue préférables mais ne permettent toujours pas ni leur suivi ni leur automatisation. 
% In traditional \st development, the programmer would write tests in a workspace 
% as soon as a method was finished.
% Sometimes a test would be incorporated as a comment at the head of the method that it exercised, or tests that needed some set up would be included as example methods in the class.
% The problem with these practices is that tests in a workspace are not available to other programmers who modify the code; comments and example methods are better in this respect, 
% but there is still no easy way to keep track of them and
% to run them automatically.
Les tests qui ne sont pas exécutés ne vous aident pas à trouver les bugs~! De plus, une méthode exemple ne donne au lecteur aucune information concernant le résultat attendu~: vous pouvez exécuter l'exemple et voir le \,---\,peut-être surprenant\,---\,résultat, mais vous ne saurez pas si le comportement observé est correct.
% Tests that are not run do not help you to find bugs!
% Moreover, an example method does not
% inform the reader of the expected result:
% you can run the example and see the\,---\,perhaps surprising\,---\,result, 
% but you will not know if the observed behavior is correct.

\sunit est productif car il nous permet d'écrire des tests capables de s'auto-vérifier~: le test définit lui-même quel est le résultat attendu. \sunit nous aide aussi à organiser les tests en groupes, à décrire le contexte dans lequel les tests doivent être exécutés, et à exécuter automatiquement un groupe de tests. En utilisant \sunit, vous pouvez écrire des tests en moins de deux minutes; alors, au lieu d'écrire des portions de code dans un \ct{Workspace}, nous vous encourageons à utiliser \sunit et à bénéficier de tous les avantages de tests sauvegardés et exécutables automatiquement.

% \sunit is valuable because it allows us to write tests that are self-checking:
% the test itself defines what the correct result should be.
% It also helps us to
% organize tests into groups, to describe the context in which the tests must run, and to
% run a group of tests
% automatically.  In less than two minutes you can write tests using
% \sunit, so instead of writing small code snippets in a workspace, we encourage you
% to use \sunit and get all the
% advantages of stored and automatically executable tests.
Dans ce chapitre, nous commencerons par discuter de la raison des
tests et de ce qu'est un bon test. Nous présenterons alors une séries de petits exemples montrant comment utiliser \sunit. Finalement, nous étudierons l'implémentation de \sunit, de façon à ce que vous compreniez comment \st utilise la puissance de la \ind{réflexivité} pour la mise en {\oe}uvre de ses outils.

% In this chapter we start by discussing why we test, and what makes a good test. We then present a series of small 
% examples showing how to use \sunit.
% Finally, we look at the implementation of \sunit, so that you can understand how
% \st uses the power of \ind{reflection} in supporting its tools. 

%=================================================================
%\section{Why testing is important}
\section{Pourquoi tester est important}
\seclabel{why}

Malheureusement, beaucoup de développeurs croient perdre leur temps avec les tests. 
Après tout, \emph{ils} n'écrivent pas de bug\,---\,seulement les \emph{autres} programmeurs le font. La plupart d'entre nous avons dit, à un moment ou à un autre~: ``j'écrirais des tests si j'avais plus de temps''.
Si vous n'écrivez jamais de bugs, et si votre code n'est pas destiné à être modifié dans le futur, alors, en effet, les tests sont une perte de temps. Pourtant, cela signifie très probablement que votre application est triviale, ou qu'elle n'est pas utilisée, ni par vous, ni par quelqu'un d'autre. Pensez aux tests comme un investissement sur le futur~: disposer d'une suite de tests est dès à présent tout à fait utile, mais sera \emph{extrêmement} utile dans le futur, lorsque votre application ou votre environnement dans lequel elle s'exécute évoluera. 

% Unfortunately, many developers believe that tests are a waste of their time.  
% After all, \emph{they} do not write bugs\,---\,only \emph{other} programmers do that.
% Most of us have said, at some time or other:
% ``I would write tests if I had more time.''
% If you never write a bug, and if your code will never be changed in the future,
% then indeed tests are a waste of your time.
% However, this most likely
% also means that your application is trivial, or that it is not used by you or anyone else.  
% Think of tests as an investment for the future: having a
% suite of tests will be quite useful now, but it will be \emph{extremely} useful when
% your application, or the environment in which it executes, changes in the future.

Les tests jouent plusieurs rôles. Premièrement, ils fournissent une
documentation pour la fonctionnalité qu'ils couvrent. De plus, la
documentation est active~: l'observation des passes de tests vous
indique que votre documentation est à jour. Deuxièmement, les tests
aident les développeurs à garantir que certaines modifications qu'ils
viennent juste d'apporter à un package n'ont rien cassé dans le
système\,---\, et à trouver quelles parties sont cassées si leur
confiance s'avère contredite. Finalement, écrire des tests en même
temps que\,---\, ou même avant de \,---\,  programmer vous force à
penser à la fonctionnalité que vous désirez concevoir et à
\emph{comment elle devrait apparaître au client}, plutôt qu'à comment
la mettre en {\oe}uvre. En écrivant les tests en premier\,---\, avant
le code\,---\, vous êtes contraint d'établir le contexte dans lequel
votre fonctionnalité s'exécutera, la façon dont elle interagira avec
le code client et les résultats attendus. Votre code s'améliorera~:
essayez donc!
 
% Tests play several roles. First, they provide documentation of the functionality that they cover.  
% Moreover, the documentation is active: watching the tests pass tells you that the documentation is up-to-date.
% Second,
% tests help developers to confirm that some changes that they have just made to a package
% have not broken anything else in the system\,---\,and to find the parts that break when that confidence turns out to be misplaced.
% Finally, writing tests at the same time as\,---\,or even
% before\,---\,programming forces you to think about the functionality
% that you want to design, \emph{and how it should appear to the client}, 
% rather than about how to implement it.
% By writing the tests first\,---\,before the code\,---\,you are compelled to state
% the context in which your functionality will run, the way it will
% interact with the client code, and the expected results.  
% Your code will improve: try it.

%The culture of tests has always been present in the \st
%community because after writing a method, we would write a small
%expression to test it.  This practice supports the extremely tight
%incremental development cycle promoted by \st.  However, doing
%so does not bring the maximum benefit from testing because the tests
%are not saved and run automatically.  Moreover it often happens that
%the context of the tests is left unspecified so the reader has to
%interpret the results and assess if they are right or wrong.

Nous ne pouvons pas tester tous les aspects d'une application
réaliste. Couvrir une application complète est tout simplement
impossible et ne devrait pas être l'objectif du test. Même avec une
bonne suite de tests, certains bugs seront quand même présents dans
votre application, sommeillant en attendant l'occasion d'endommager votre système. Si vous constatez que c'est arrivé, tirez-en parti~! Dès que vous découvrez le bug, écrivez un test qui le met en évidence, exécutez le test et observez qu'il échoue. Alors vous pourrez commencer à corriger le bug~: le test vous indiquera quand vous en aurez fini.

% We cannot test all aspects of any realistic application.
% Covering a complete application is simply impossible and should not be the
% goal of testing. 
% Even with a good test suite
% some bugs will still creep into the application, where they can lay dormant
% waiting for an opportunity to damage your system.  
% If you find that this has happened, take advantage of it!
% As soon as you uncover the bug, write a test that exposes it, run the test, and watch it fail.
% Now you can start to fix the bug: the test will tell you when you are done.
%=================================================================
%\section{What makes a good test?}
\section{De quoi est fait un bon test ?}

\'Ecrire de bons tests constitue un savoir-faire qui peut s'apprendre facilement par la pratique.
\aconfirmer{Regardons comment concevoir les tests de façon à en tirer le maximum de bénéfices.}
 
% Writing good tests is a skill that can be learned most easily by
% practicing.  Let us look at the properties that tests should have to
% get a maximum benefit.

\begin{enumerate}
\item \aconfirmer{Les tests doivent pouvoir être réitérés}. Vous devez pouvoir exécuter un test aussi souvent que vous le voulez et vous devez toujours obtenir la même réponse.

% \item Tests should be repeatable.  You should be able to run a test
%   as often as you want, and always get the same answer.

\item Les tests doivent pouvoir s'exécuter sans intervention humaine. Vous devez même être capable de les exécuter pendant la nuit.

% \item Tests should run without human intervention.  You should even be
%   able to run them during the night.

\item Les tests doivent vous raconter une histoire.  Chaque test doit couvrir un aspect d'une partie de code. Un test doit agir comme un scénario que vous ou quelqu'un d'autre peut lire de façon à comprendre une partie de fonctionnalité.\label{prop:oneAspect} 

% \item Tests should tell a story.  Each test should cover one aspect of a 
%   piece of code.  A test should act as a scenario that you or some else can
%   read to understand a piece of functionality. \label{prop:oneAspect}

\item Les tests doivent changer moins fréquemment que la fonctionnalité qu'ils couvrent~: vous ne voulez pas changer tous vos tests à chaque fois que vous modifiez votre application. Une façon d'y parvenir est d'écrire des tests basés sur l'interface publique de la classe que vous êtes en train de tester. 
Il est possible d'écrire un test pour une méthode utilitaire privée si vous sentez que la méthode est suffisamment compliquée pour nécessiter le test, mais vous devez être conscient qu'un tel test est susceptible d'être modifié ou intégralement supprimé quand vous pensez à une meilleure mise en {\oe}uvre.
% \item Tests should have a change frequency lower than that of the
%   functionality they cover:  you do not want to have to change all your
%   tests every time you modify your application.  One way to achieve
%   this is to write tests based on the public interfaces of the
%   class that you are testing.  
%   It is OK to write a test for a private ``helper'' method if you feel that the method
%   is complicated enough to need the test, but you should be aware that such a test 
%   may have to be changed, or thrown away entirely, when you think of a better
%   implementation.
\end{enumerate}

Une conséquence du point (\ref{prop:oneAspect}) est que le nombre de tests doit être proportionnel au nombre de fonctions à tester~: changer un aspect du système ne doit pas altérer tous les tests mais seulement un nombre limité. C'est important car avoir 100 échecs de tests doit constituer un signal beaucoup plus fort que d'en avoir 10. Cependant, cet idéal n'est pas toujours possible à atteindre~: en particulier, si une modification casse l'initialisation d'un objet ou la mise en place du test, une conséquence probable peut être l'échec de tous les tests.
% A consequence of property (\ref{prop:oneAspect}) is that 
% the number of tests should be somewhat proportional to the number of
% functions to be tested: changing one aspect of the
% system should not break all the tests but only a limited
% number.  This is important because having 100 tests fail should send a
% much stronger message than having 10 tests fail.
% However, it is not always possible to achieve this ideal: 
% in particular, if a change breaks the initialization of an object, or the
% set-up of a test, it is likely to cause all of the tests to fail. 

L'\ind{eXtreme Programming} recommande d'écrire des tests avant de
coder. Cela semble contredire nos instincts profonds de
développeur. Tout ce que nous pouvons dire est~: allez de l'avant et
essayez donc! Nous trouvons qu'écrire les tests avant le code nous aide à déterminer ce que nous voulons coder, nous aide à savoir quand nous avons terminé et nous aide à conceptualiser la fonctionnalité d'une classe et à concevoir son interface.
De plus, le développement \flqq{}\emph{test d'abord}\frqq{} (\ct{test-first}) nous donne le courage d'avancer rapidement parce que nous n'avons pas peur d'oublier quelque chose d'important. 
% \ind{eXtreme Programming} advocates writing tests before writing code.  This may seem to go
% against our deep instincts as software developers.  
% All we can say is: go ahead and try it.
% We have found that writing the tests before the code helps us
% to know what we want to code, helps us know when we are done,
% and helps us conceptualize the functionality of a class and to
% design its interface.
% Moreover, test-first development gives us the courage to go fast, because we are not afraid that we will forget something important.  

% \on{I cannot understand this without some explanation!}

%Writing tests is not difficult in itself. What is more difficult is choosing what to test.
%The pragmatic programmers\footnote{\url{www.pragmaticprogrammer.com}} offer the right-BICEP principle. It stands for: 
%\begin{itemize}
%\item Right -- Are the results right?
%\item B -- Are all the boundary conditions correct?
%\item I -- Can you check inverse relationships?
%\item C -- Can you cross-check results using other means?
%\item E -- Can you force error conditions to happen?
%\item P -- Are performance characteristics within bounds?
%\end{itemize}


% Now let's write our first test, and show you the benefits of using \SUnit.
%=================================================================
%\section{\sunit by example}
\section{\sunit par l'exemple}

Avant de considérer \SUnit en détails, nous allons montrer un exemple, étape par étape. Nous utilisons un exemple qui teste la classe \ct{Set}. Essayez de saisir le code au fur et à mesure que nous avançons.
% Before going into the details of \SUnit, we will show a step by step
% example.  We use an example that tests the class \ct{Set}.  Try
% entering the code as we go along.
%---------------------------------------------------------
%\subsection{Step 1: create the test class}
\subsection{\'Etape 1: créer la classe de test}

\dothis{Créez tout d'abord une nouvelle sous-classe de \clsind{TestCase} nommée
\ct{ExampleSetTest}. Ajoutez-lui deux variables d'instance de façon à ce que votre classe ressemble à ceci~:}

%ATTENTION
\newpage{}
\begin{classdef}[exampleSetTest]{Un exemple de classe de test pour Set}
TestCase subclass: #ExampleSetTest
	instanceVariableNames: 'full empty'
	classVariableNames: ''
  poolDictionaries: ''
	category: 'MySetTest'
\end{classdef}
% \dothis{First you should create a new subclass of \clsind{TestCase} called
% \ct{ExampleSetTest}.   Add two instance variables so that your new
% class looks like this:}
% 
% \begin{classdef}[exampleSetTest]{An Example Set Test class}
% TestCase subclass: #ExampleSetTest
% 	instanceVariableNames: 'full empty'
% 	classInstanceVariableNames: ''
% 	category: 'MyTest'
% \end{classdef}

Nous utiliserons la classe \ct{ExampleSetTest} pour regrouper tous les tests relatifs
à la classe \ct{Set}. Elle définit le contexte dans lequel les tests s'exécuteront. Ici, le contexte est décrit par les deux variables d'instance \ct{full} et \ct{empty} qui seront utilisées pour représenter respectivement, un \ct{Set} plein et un \ct{Set} vide.
% We will use the class \ct{ExampleSetTest} to group all the tests related to
% the class \ct{Set}.  It defines the context in which the tests
% will run.  Here the context is described by
% the two instance variables \ct{full} and \ct{empty}
% that we will use to represent a full and an empty set.

Le nom de la classe n'est pas fondamental, mais par convention il devrait se terminer par \ct{Test}.
Si vous définissez une classe nommée \ct{Pattern} et que vous nommez la classe de test correspondante \ct{PatternTest}, les deux classes seront présentées ensemble, par ordre alphabétique, dans le \ct{System Browser} (en considérant qu'elles sont dans la même catégorie). Il est indispensable que votre classe soit une sous-classe de \ct{TestCase}.

% The name of the class is not critical, but by convention it should end in \ct{Test}.
% If you define a class called \ct{Pattern} and call the corresponding test class \ct{PatternTest}, the two classes will be alphabetized together in the browser (assuming that they are in the same category).  It \emph{is} critical that your class be a subclass of \ct{TestCase}.
%---------------------------------------------------------
%\subsection{Step 2: initialize the test context}
\subsection{\'Etape 2: initialiser le contexte du test}

La méthode \mthind{TestCase}{setUp} (en anglais, \emph{configurer}) définit le contexte dans lequel les tests vont s'exécuter, un peu comme la méthode \ct{initialize}. \ct{setUp} est invoquée avant l'exécution de chaque méthode de test définie dans la classe de test.
\index{SUnit!set up method}
\seeindex{test!SUnit}{SUnit}

%martial: phrase a revoir
\dothis{Définissez la méthode \ct{setUp} de la façon suivante pour initialiser la variable \ct{empty}, de sorte qu'elle référence un \ct{Set} vide; et la variable \ct{full}, de sorte qu'elle référence un \ct{Set} contenant deux éléments.}
% The method \mthind{TestCase}{setUp} defines the context in which the tests will run, a bit like an initialize method.
% \ct{setUp} is invoked before the execution of each test
% method defined in the test class.
% \index{SUnit!set up method}
% \seeindex{testing}{SUnit}
% 
% \dothis{Define the \ct{setUp} method as follows, to initialize the \ct{empty} variable to refer to an empty set and the \ct{full} variable to
% refer to a set containing two elements. }

%\begin{method}[setupExampleSetTest]{Setting up a fixture}
\needlines{3} % CHANGE
\begin{method}[setupExampleSetTest]{Mettre au point une installation}
ExampleSetTest>>>setUp
	empty := Set new.
	full := Set with: 5 with: 6
\end{method}

\noindent
Dans le jargon du test, le contexte est appelé \emph{l'installation}
du test (en anglais, \emph{fixture}).
\index{SUnit!installation}
\seeindex{fixture}{SUnit, installation}
% \noindent
% In testing jargon the context is called the \emph{fixture} for the
% test.
% \index{SUnit!fixture}
% \seeindex{fixture}{SUnit, fixture}

%---------------------------------------------------------
\subsection{\'Etape 3: écrire quelques méthodes de test}
%\subsection{Step 3: write some test methods}

Créons quelques tests en définissant quelques méthodes dans la classe \ct{ExampleSetTest}.
Chaque méthode représente un test;
le nom de la méthode devrait commencer par la chaîne ``\ct{test}'' pour que \sunit les regroupe en suites de tests.
Les méthodes de test ne prennent pas d'arguments.

% Let's create some tests by defining some methods in the class
% \ct{ExampleSetTest}.  
% Each method represents one test; 
% the name of the method should start with the string `\ct{test}' so that \sunit
% will collect them into test suites.
% Test methods take no arguments.

\dothis{Définissez les méthodes de test suivantes.} 
Le premier test, nommé \ct{testIncludes}, teste la méthode \ct{includes:} de \ct{Set}. Le test dit que, envoyer le message \ct{includes: 5} à un \ct{Set} contenant 5 devrait retourner \ct{true}. Clairement, ce test repose sur le fait que la méthode \ct{setUp} s'est déjà exécutée.

% \dothis{Define the following test methods.}
% The first test, named \ct{testIncludes}, tests the
% \ct{includes:} method of \ct{Set}.  The test says that sending the
% message \ct{includes: 5} to a set containing 5 should return
% \ct{true}.  Clearly, this test relies on the fact
% that the \ct{setUp} method has already run.

%\begin{method}[testIncludes]{Testing set membership}
\begin{method}[testIncludes]{Tester l'appartenance à un Set}
ExampleSetTest>>>testIncludes
	self assert: (full includes: 5).
	self assert: (full includes: 6)
\end{method}

Le second test nommé \ct{testOccurrences} vérifie que le nombre d'occurences de~5 dans le \ct{Set} \ct{full} est égal à un, même si nous ajoutons un autre élément~5 au \ct{Set}.
% The second test, named \ct{testOccurrences}, verifies that the
% number of occurrences of~5 in \ct{full} set is equal to one, even if we
% add another element~5 to the set.

\needlines{6}
%\begin{method}[testOccurrences]{Testing occurrences}
\begin{method}[testOccurrences]{Tester des occurrences}
ExampleSetTest>>>testOccurrences
	self assert: (empty occurrencesOf: 0) = 0.
	self assert: (full occurrencesOf: 5) = 1.
	full add: 5.
	self assert: (full occurrencesOf: 5) = 1
\end{method}

Finalement, nous testons que le \ct{Set} n'a plus d'élément~5 après que nous l'ayons supprimé.
% Finally, we test that the set no
% longer contains the element 5 after we have removed it.

%\begin{method}[testRemove]{Testing removal}
\begin{method}[testRemove]{Tester la suppression}
ExampleSetTest>>>testRemove
	full remove: 5.
	self assert: (full includes: 6).
	self deny: (full includes: 5)
\end{method}

\noindent
Notez l'utilisation de la méthode \mthind{TestCase}{deny:} pour garantir que quelque chose ne doit pas être vrai.
\ct{aTest deny: anExpression} est équivalent à \ct{aTest assert: anExpression not}, mais en beaucoup plus lisible.
% Note the use of the method \mthind{TestCase}{deny:} to assert something that should not be true.
% \ct{aTest deny: anExpression} is equivalent to \ct{aTest assert: anExpression not}, but is much more readable.
%---------------------------------------------------------
%\subsection{Step 4: run the tests}
\subsection{\'Etape 4: exécuter les tests}
\arelire{%
L'exécution des tests se fait le plus simplement en utilisant
directement le Browser. \Actclickz{} sur le paquetage, le nom de la
classe ou une méthode de tests et, de là, sélectionnez
\menu{run the tests (t)}.
Les méthodes de tests seront alors, entièrement ou partiellement,
signalées par une puce rouge ou verte selon le succès complet, partiel
ou bien l'échec des tests.} % REVOIR

\begin{figure}[tbh]
  \begin{center}
	\includegraphics[width=\linewidth]{browser-tests}
	\caption{Lancer les tests \sunit depuis le Browser.}
	\figlabel{browser-tests}
  \end{center}
\end{figure}

\arelire{Vous pouvez aussi sélectionner \arevoir{certaines suites de
    tests}
à lancer et obtenir un listing plus détaillé des
résultats en lançant  l'\emphind{exécuteur de tests}  \emphind{Test
  Runner} de \sunit depuis le menu \menu{World \go Test Runner}.}
 % REVOIR listing = 'log' dans PBE
L'exécuteur de tests, montré dans la \figref{test-runner}, est conçu pour faciliter l'exécution de groupes de tests.
Le panneau le plus à gauche présente toutes les catégories qui
contiennent des classes de test (\ie  sous-classes de
\ct{TestCase}). Lorsque certaines de ces catégories sont
sélectionnées, les classes de test qu'elles contiennent apparaissent
dans le panneau de droite.
Les classes abstraites sont en italique et la hiérarchie des classes de test est visible par l'indentation, ainsi les sous-classes de  \ct{ClassTestCase} sont plus indentées que les sous-classes de \ct{TestCase}.

\begin{figure}[tbh]
  \begin{center}
	\includegraphics[width=\linewidth]{test-runner}
	\caption{\sunit, l'exécuteur de test de \pharo.}
	\figlabel{test-runner}
  \end{center}
\end{figure}


\dothis{Ouvrez le Test Runner, sélectionnez la catégorie \menu{MyTest} et cliquez le bouton \button{Run Selected}.}

% ON: With OB, you don't need this.
%Vous pouvez aussi exécuter votre test en évaluant un \menu{print it} sur le code suivant~: \ct{(ExampleSetTest selector: #testRemove) run}. L'expression suivante est équivalente mais plus concise~: \ct{ExampleSetTest run: #testRemove}. Il nous est habituel d'inclure un commentaire exécutable à notre méthode de test ce qui nous permet de les exécuter avec un \menu{do it} depuis le \ct{System Browser}, comme il est montré dans \mthref{ExampleSetTestTestRemoveii}.

%\needlines{6}
%\begin{method}[ExampleSetTestTestRemoveii]{Les commentaires exécutables dans les méthodes de test}
%ExampleSetTest>>>testRemove
%	"self run: #testRemove"
%	full remove: 5.
%	self assert: (full includes: 6).
%	self deny: (full includes: 5)
%\end{method} - RETRAIT

\dothis{Introduisez un bug dans \ct{ExampleSetTest>>testRemove} et évaluez le test à nouveau. Par exemple, remplacez \ct{5} par \ct{4}.}
%\dothis{Introduce a bug in \ct{ExampleSetTest>>>testRemove} and run the tests again. For example, change \ct{5} to \ct{4}.}

Les tests qui ne sont pas passés (s'il y en a) sont listés dans les panneaux de droite du \emph{Test Runner}.
Si vous voulez en déboguer un et voir pourquoi il échoue, il suffit juste de cliquer sur le nom. 
% Une alternative est d'évaluer l'expression suivante~:
% \begin{code}{}
% (ExampleSetTest selector: #testRemove) debug
% \end{code}
% ou bien
% \begin{code}{}
% ExampleSetTest debug: #testRemove
% \end{code
% } - RETRAIT

%---------------------------------------------------------
%\subsection{Step 5: interpret the results}
\subsection{\'Etape 5: interpréter les résultats}

La méthode \mthind{TestCase}{assert:}\,, définie dans la classe
\ct{TestCase}, prend un booléen en argument; habituellement la valeur
d'une expression testée. Quand cet argument est à vrai (\ct{true}), le
test est réussi; quand cet argument est à faux (\ct{false}), le test échoue.  

Il y a actuellement trois résultats possibles pour un test. Le
résultat espéré est que toutes les assertions du test soient vraies,
dans ce cas le test réussit. Dans l'exécuteur de tests
(\ct{TestRunner}), quand tous les tests réussissent, la barre du haut
devient verte. Pourtant, il reste deux possibilités pour que quelque
chose se passe mal quand vous évaluez le test. Le plus évident est
qu'une des assertions peut être fausse, entraînant \emph{l'échec} du
test. Pourtant, il est aussi possible qu'une erreur intervienne
pendant l'exécution du test, telle qu'une erreur \emph{message non
  compris} ou une erreur 
d'\emph{indice hors limites}.
%d'\emph{index hors limites}. 

Si une erreur survient, les assertions de la méthode de test peuvent ne pas avoir été exécutées du tout, ainsi nous ne pouvons pas dire que le test a échoué. 
Toutefois, quelque chose est clairement faux~!
Dans l'exécuteur de tests (\ct{TestRunner}), la barre du haut devient jaune pour les tests en échec et ces tests sont listés dans le panneau du milieu à droite, alors que pour les tests erronés, la barre devient rouge et ces tests sont listés dans le panneau en bas à droite.


%\dothis{Modify your tests to provoke both errors and failures.}
\dothis{Modifiez vos tests de façon à provoquer des erreurs et des échecs.}
%=================================================================
%\section{The \SUnit cook book}
\section{Les recettes pour \SUnit}
Cette section vous donne plus d'informations sur la façon d'utiliser \SUnit. Si vous avez utilisé un autre environnement de test comme \JUnit~\footnote{\url{http://junit.org}}, ceci vous sera familier puisque tous ces environnements sont issus de  \SUnit. \aconfirmer{Normalement, vous utiliserez l'IHM~\footnote{Interface Homme Machine.} de \SUnit pour exécuter les tests à l'exception de certains cas.}


%---------------------------------------------------------
\subsection{Autres assertions}
En supplément de \ct{assert:} et \ct{deny:}, il y a plusieurs autres méthodes pouvant être utilisées pour spécifier des assertions. 



Premièrement, \mthind{TestCase}{assert:description:} et \mthind{TestCase}{deny:description:} prennent un second argument qui est un message sous la forme d'une chaîne de caractères pouvant être utilisé pour décrire la raison de l'échec au cas où elle n'apparaît pas évidente à la lecture du test lui-même. Ces méthodes sont décrites dans \secref{descriptionStrings}.



Ensuite, \sunit dispose de deux méthodes supplémentaires, \mthind{TestCase}{should:raise:} et \mthind{TestCase}{shouldnt:raise:}  pour la propagation des exceptions de test. Par exemple, \ct{self should: aBlock raise: anException} vous permet de tester si une exception particulière est levée pendant l'exécution de \ct{aBlock}. \Tmthref{ESTtestIllegal} illustre l'utilisation de \mbox{\ct{should:raise:}}.


%\dothis{Try running this test.}
\dothis{Essayez d'évaluer ce test.}

Notez que le premier argument des méthodes \ct{should:} et \ct{shouldnt:} est un \emphind{bloc} qui \emph{contient}
%\emphind{Block}
l'expression à évaluer.
 

%\begin{method}[ESTtestIllegal]{Testing error raising}
\begin{method}[ESTtestIllegal]{Tester la levée d'une erreur}
ExampleSetTest>>>testIllegal
	self should: [empty at: 5] raise: Error.
	self should: [empty at: 5 put: #zork] raise: Error
\end{method}

\sunit est portable~: il peut être utilisé avec tous les dialectes de \st. Afin de rendre \sunit portable, ses développeurs ont retiré les parties dépendantes des dialectes. La méthode de classe \cmind{TestResult class}{error} retourne la classe erreur du système de façon indépendante du dialecte. Vous pouvez en profiter aussi~: si vous voulez écrire des tests qui fonctionnent quelque soit le dialecte de \st, vous pouvez écrire \tmthref{ESTtestIllegal} ainsi:


\needlines{4}
%\begin{method}[portabletestillegal]{Portable error handling}
\begin{method}[portabletestillegal]{Gestion portable des erreurs}
ExampleSetTest>>>testIllegal
	self should: [empty at: 5] raise: TestResult error.
	self should: [empty at: 5 put: #zork] raise: TestResult error
\end{method}

%\dothis{Give it a try.}
\dothis{Essayez-le!}

%---------------------------------------------------------
%\subsection{Running a single test}
\subsection{Exécuter un test simple}
Normalement, vous exécuterez vos tests avec l'exécuteur de tests
(\ct{TestRunner}).
\arelire{Vous pouvez aussi le lancer en faisant un \menu{do it} du code}
  \ct{TestRunner open}. % CHANGE REVOIR - martial - simplifié car PBE se trompe:
%Si vous ne voulez pas lancer l'exécuteur de tests depuis le menu \menu{open\,\ldots}, vous pouvez évaluer  \ct{TestRunner open} à l'aide d'un \menu{print it}.
%\index{Tools flap}

Vous pouvez exécuter un simple test de la façon suivante:


\begin{code}{}
ExampleSetTest run: #testRemove --> 1 run, 1 passed, 0 failed, 0 errors
\end{code}

%---------------------------------------------------------
\subsection{Exécuter tous les tests d'une classe de test}
%\subsection{Running all the tests in a test class}

Toute sous-classe de \ct{TestCase} répond au message \ct{suite} qui construira une suite de tests contenant toutes les méthodes de la classe dont le nom commence par la chaîne ``\ct{test}''.
Pour exécuter les tests de la suite, envoyez-lui le message \ct{run}.
Par exemple~:
% Any subclass of \ct{TestCase} responds to the message \ct{suite}, which will build a test suite that contains all the
% methods in the class whose names start with the string ``\ct{test}''.
% To run the tests in the suite, send it the message \ct{run}.
% For example:

\begin{code}{}
ExampleSetTest suite run --> 5 run, 5 passed, 0 failed, 0 errors
\end{code}

%---------------------------------------------------------
%\subsection{Must I subclass TestCase?}
\subsection{Dois-je sous-classer TestCase ?}
Avec \JUnit{}, vous pouvez construire un \clsind{TestSuite} dans n'importe quelle classe contenant des méthodes \ct{test*}. En \st, vous pouvez faire la même chose mais vous aurez à créer une suite manuellement et votre classe devra mettre en {\oe}uvre toutes les méthodes esentielles de \ct{TestCase} comme \ct{assert:}.
Nous ne vous le recommandons pas. L'environnement est déjà là~: utilisez-le.
% In \JUnit{} you can build a \clsind{TestSuite} from an arbitrary class
% containing \ct{test*} methods.  In \st you can do the same
% but you will then then have to create a suite by hand and your class will
% have to implement all the essential \ct{TestCase} methods like \ct{assert:}.
% We recommend that you not try to do this.  The framework is there: use it.
%=================================================================
%\section{The SUnit framework}
\section{L'environnement SUnit}

Comme montré dans \figref{sunit-classes}, \sunit consiste en quatre classes principales~: \clsind{TestCase},\clsind{TestSuite}, \clsind{TestResult} et \clsind{TestResource}.
La notion de \emph{ressource de test} a été introduite dans \sunit~3.1 pour représenter une ressource coûteuse à installer mais qui peut être utilisée par toute une série de tests. Un \ct{TestResource} spécifie une méthode \ct{setUp} qui est exécutée une seule fois avant la suite de tests; à la différence de la méthode \ct{TestCase>>>setUp} qui est exécutée avant chaque test.
% \sunit consists of four main classes: \clsind{TestCase},
% \clsind{TestSuite}, \clsind{TestResult}, and \clsind{TestResource}, as shown in \figref{sunit-classes}.
% The notion of a \emph{test resource} was introduced in \sunit 3.1 to represent a resource that is expensive to set-up but which can be used by
% a whole series of tests.  A \ct{TestResource}
% specifies a \ct{setUp} method that is executed just once before a suite of tests;
% this is in distinction to the \ct{TestCase>>>setUp} method, which is executed before
% each test.

\begin{figure}[htb]
  \begin{center}
		{\includegraphics[width=0.8\textwidth]{sunit-classes}}
	\caption{Les quatres classes constituant le noyau de \SUnit.}
	\label{fig:sunit-classes}
  \end{center}
\end{figure}


%---------------------------------------------------------
\subsection{TestCase}

\clsindmain{TestCase} est une classe abstraite conçue pour avoir des sous-classes; chacune de ses sous-classes représente un groupe de tests qui partagent un contexte commun (ce qui constitue une suite de tests). 
Chaque test est évalué par la création d'une nouvelle instance d'une sous-classe de \ct{TestCase} par l'exécution de \mthind{TestCase}{setUp}, par l'exécution de la méthode de test elle-même puis par l'exécution de \mthind{TestCase}{tearDown}~\footnote{En français, démolir.}.
% \clsindmain{TestCase} is an abstract class that is designed to be subclassed; each of its subclasses represents a group of tests that share a common context (that is, a test suite).
% Each test is run by creating a new instance of a subclass of \ct{TestCase},
% running \mthind{TestCase}{setUp}, running the test method itself, and then running \mthind{TestCase}{tearDown}.

Le contexte est porté par des variables d'instance de la sous-classe et par la spécialisation de la méthode \ct{setUp} qui initialise ces variables d'instance.  Les sous-classes de \ct{TestCase} peuvent aussi surcharger la méthode \ct{tearDown} qui est invoquée après l'exécution de chaque test et qui peut être utilisée pour libérer tous les objets alloués pendant \ct{setUp}.
% The context is specified
% by instance variables of the subclass
% and by the specialization of the method
% \ct{setUp}, which initializes those instance variables.
% Subclasses of \ct{TestCase} can also override method
% \ct{tearDown}, which is invoked after the execution of each test,
% and can be used to release any objects
% allocated during \ct{setUp}.
%---------------------------------------------------------
\subsection{TestSuite}

Les instances de la classe \clsindmain{TestSuite} contiennent une collection de cas de tests. Une instance de \ct{TestSuite} contient des tests et d'autres suites de tests. En fait, une suite de tests contient des instances de sous-classes de \ct{TestCase} et de \ct{TestSuite}.
Individuellement, les \lct{TestCase}s et les \lct{TestSuite}s comprennent le même protocole, ainsi elles peuvent être traitées de la même façon~; par exemple, elles comprennent toutes \ct{run}.
Il s'agit en fait de l'application du patron de conception 
%pas \ct{composite} mais plutôt
\emph{Composite}
pour lequel \ct{TestSuite} est le composite et les \ct{TestCase}s sont les feuilles\,---\,voir les \textit{Design Patterns} pour plus d'informations sur ce patron\cite{Gamm95a}.
% Instances of the class \clsindmain{TestSuite} contain a collection of test cases.  An
% instance of \ct{TestSuite} contains tests, and other test suites.
% That is, a test suite contains sub-instances of
% \ct{TestCase} and \ct{TestSuite}.
% Both individual \ct{TestCase}s and \ct{TestSuite}s understand the same protocol, so they can be treated in the same way; for example, both can be \ct{run}.
% This is in fact an application of the composite
% pattern in which \ct{TestSuite} is the composite and the
% \ct{TestCase}s are the leaves\,---\,see \textit{Design Patterns} for more information on this pattern\cite{Gamm95a}.
%---------------------------------------------------------
\subsection{TestResult}

La classe \clsindmain{TestResult} représente les résultats de l'exécution d'un \ct{TestSuite}. Elle mémorise le nombre de tests passés, le nombre de tests en échec et le nombre d'erreurs levées.
% The class \clsindmain{TestResult} represents the results of a
% \ct{TestSuite} execution.  It records the number of tests passed,
% the number of tests failed, and the number of errors raised.

%---------------------------------------------------------
\subsection{TestResource}
\seclabel{resource}

Une des caractéristiques importantes d'une suite de tests est que les tests doivent être indépendants les uns des autres~: l'échec d'un test ne doit pas entraîner l'échec des autres tests qui en dépendent; l'ordre dans lequel les tests sont exécutés ne doit pas non plus importer.
\'Evaluer \ct{setUp}  avant chaque test et \ct{tearDown} après permet de renforcer cette indépendance.
% One of the important  features of a suite of tests is that they should be independent of each other: the failure of one test should not cause an avalanche of failures of other tests that depend upon it, nor should the order in which the tests are run matter.
% Performing \ct{setUp} before each test and \ct{tearDown} afterwards helps to reinforce this independence. 

Malgré tout, il y a certains cas pour lesquels la préparation du contexte nécessaire est simplement trop lent pour qu'il soit 
réalisable
% plutôt que praticable 
de le faire avant l'exécution de chaque test. De plus, si nous savons que les tests n'altèrent pas les ressources qu'ils utilisent, alors il est prohibitif de les initialiser pour chaque test; il est suffisant de les initialiser une seule fois pour chaque suite de tests. Supposez, par exemple, qu'une suite de tests ait besoin d'interroger une base de données ou d'effectuer certaines analyses sur du code compilé.
% rene : elle est censée (car c'est la suite de tests)
Pour ces situations, elle est censée initialiser et ouvrir une
connexion vers la base de données ou compiler du code source avant
l'exécution des tests.
% However, there are occasions where setting up the necessary context is just too time-consuming for it to be practical to do it once before the execution of each test.
% Moreover, if it is known that the test cases do not disrupt the resources used by the tests, then it is wasteful to set them up afresh for each test; it is sufficient to set them up once for each suite of tests.
% Suppose, for example, that a suite of tests need to query a database, or do some analysis on some compiled code.
% In such cases, it may make sense to set up the database and open a connection to it, or to compile some source code, before any of the tests start to run.

Où pourrions nous conserver ces ressources de façon à ce qu'elles puissent être partagées par les tests d'une suite~?
Les variables d'instance d'une sous-classe de \ct{TestCase} particulière ne le pourraient pas parce que ses instances ne subsistent que pendant la durée d'un seul test.
Une variable globable ferait l'affaire, mais utiliser trop de variables globales pollue l'espace de nommage et la relation entre la variable globale et les tests qui en dépendent ne serait pas explicite.
Une meilleure solution est de placer les ressources nécessaires dans l'objet singleton d'une certaine classe. La classe \clsindmain{TestResource} est définie pour avoir des sous-classes utilisées comme classes de ressource. Chaque sous-classe de \ct{TestResource} comprend le message \ct{current} qui retournera son instance singleton.
Les méthodes  \ct{setUp} et \ct{tearDown} doivent être surchargées dans la sous-classe pour permettre à la ressource d'être initialisée et libérée.
% Where should we cache these resources, so that they can be shared by a suite of tests?
% The instance variables of a particular \ct{TestCase} sub-instance won't do, because such an instance persists only for the duration of a single test.
% A global variable would work, but using too many global variables pollutes the name space, and the binding between the global and the tests that depend on it will not be explicit.
% A better solution is to put the necessary resources in a singleton object of some class.
% The class \clsindmain{TestResource} exists to be subclassed by such resource classes.
% Each subclass of \ct{TestResource} understands the message  \ct{current}, which will answer a singleton instance of that subclass.
% Methods \ct{setUp} and \ct{tearDown} should be overridden in the subclass to ensure that the resource is initialized and finalized.

Une chose demeure~: d'une certaine façon, \sunit doit être informé de quelles ressources sont associées avec quelle suite de tests. Une ressource est associée à une sous-classe particulière de \ct{TestCase} par la surcharge de la méthode de \emph{classe} \ct{resources}.
Par défaut, les ressources d'un \ct{TestSuite} sont constituées par l'union des ressources des \ct{TestCase}s qu'il contient.

% One thing remains: somehow, \sunit has to be told which resources are associated with which test suite.
% A resource is associated
% with a particular subclass of \ct{TestCase} 
% by overriding the \emph{class} method \ct{resources}.
% \ab{The set of resources attributed to each test is actually the closure of these resources under the resources message, but I think that we don't want to say that!}
% By default, the resources of 
% a \ct{TestSuite} are
% the union of the resources of
% the \ct{TestCase}s that it contains.

Voici un exemple. Nous définissons une sous-classe de \ct{TestResource} nommée \ct{MyTestResource} et nous l'associons à  \ct{MyTestCase} en spécialisant la méthode de classe \ct{resources} de sorte qu'elle retourne un tableau contenant \aconfirmer{les classes de test qu'il utilisera}.

% Here is an example. 
% We define a subclass of \ct{TestResource} called
% \ct{MyTestResource} and we associate it with \ct{MyTestCase}
% by specializing the class method \ct{resources} to return an array
% of the test classes that it will use.

\needlines{8}
\begin{classdef}[mytestresource]{Un exemple de sous-classe de TestResource}
TestResource subclass: #MyTestResource
	instanceVariableNames: ''

MyTestCase class>>>resources
	"associe la ressource avec cette classe de test"
	^{ MyTestResource }
\end{classdef}
% \begin{classdef}[mytestresource]{An example of a TestResource subclass}
% TestResource subclass: #MyTestResource
% 	instanceVariableNames: ''
% 
% MyTestCase class>>>resources
% 	"associate the resource with this class of test cases"
% 	^{ MyTestResource }
% \end{classdef}

%\needlines{10}
%\begin{classdef}[mytestresource]{An example of a TestResource subclass}
%TestResource subclass: #MyTestResource
%	instanceVariableNames: ''

%MyTestResource>>>setUp
%	"Set up resources here."

%MyTestResource>>>tearDown
%	"Tear down resources here."

%MyTestCase class>>>resources
%	"associate the resource with this class of test cases"
%	^{ MyTestResource }
%\end{classdef}

% \on{Do we really need the empty setUp and tearDown methods here?}

%=================================================================
%\section{Advanced features of SUnit}
\section{Caractéristiques avancées de SUnit}

En plus de \ct{TestResource}, la version courante de \sunit dispose de la description des assertions avec des chaînes, d'une gestion des traces et de la reprise sur un test en échec 
%Martial: ajout pour plus de précision pour les non-anglophones
(cette dernière faisant appel aux méthodes avec terme anglophone \ct{resumable}).  
% In addition to \ct{TestResource}, the current version of \sunit contains assertion
% description strings, logging support, and resumable test failures.

%---------------------------------------------------------
\subsection{Description des assertions avec des chaînes \arevoir{de
    caractères}} % CHANGE
%\subsection{Assertion description strings}
\seclabel{descriptionStrings}

Le protocole des assertions de \ct{TestCase} comprend un certain nombre de méthodes permettant au programmeur de fournir une description de l'assertion. La description est une chaîne de caractères; si le test échoue, cette chaîne est affichée par l'exécuteur de tests. Bien sûr, cette chaîne peut être construite dynamiquement.
% The \ct{TestCase} assertion protocol includes a
% number of methods that allow the programmer to supply a description of the assertion.  The description is a \ct{String}; if the test case
% fails, this string will be displayed by the test runner.  Of
% course, this string can be constructed dynamically.
% \begin{code}{}
% | e |
% e := 42.
% self assert: e = 23
% 	description: 'expected 23, got ', e printString
% \end{code}
\begin{code}{}
| e |
e := 42.
self assert: e = 23
	description: 'attendu 23, obtenu ', e printString
\end{code}

%The relevant methods in \ct{TestCase} are:
Les méthodes correspondantes  de \ct{TestCase} sont~:
\begin{code}{}
#assert:description:
#deny:description:
#should:description:
#shouldnt:description:
\end{code}
\cmindex{TestCase}{assert:description:}
\cmindex{TestCase}{deny:description:}
\cmindex{TestCase}{should:description:}
\cmindex{TestCase}{shouldnt:description:}

%---------------------------------------------------------
\subsection{Gestion des traces}
Les chaînes descriptives présentées précédemment peuvent aussi être tracées dans un flux de données \ct{Stream} tel que le \ct{Transcript} ou un flux associé à un fichier. Vous pouvez choisir de tracer ou non en surchargeant \cmind{TestCase}{isLogging} dans votre classe de test; vous devez aussi choisir dans quoi tracer en surchargeant \cmind{TestCase}{failureLog} de façon à fournir un \emph{stream} approprié.
% The description strings described above may also be logged to a
% \ct{Stream} such as the \ct{Transcript}, or a file stream.
% You can choose whether to log by overriding
% \cmind{TestCase}{isLogging} in your test class; you must also choose where
% to log by overriding \cmind{TestCase}{failureLog} to answer an appropriate stream.

%---------------------------------------------------------
%\subsection{Continuing after a failure}
\subsection{Continuer après un échec}
\sunit nous permet aussi d'indiquer si un test doit ou non continuer après un échec. Il s'agit d'une possibilité vraiment puissante qui utilise les mécanismes d'exception offerts par \st. Pour comprendre dans quel cas l'utiliser, voyons un exemple. Observez l'expression de test suivante~:
% \sunit also allows us to specify whether or not a test should continue after a failure.  This is a really
% powerful feature that uses the exception mechanisms offered
% by \st.  To see what this can be used for, let's look at an
% example. Consider the following test expression:
\begin{code}{}
aCollection do: [ :each | self assert: each even]
\end{code}
Dans ce cas, dès que le test trouve le premier élément de la collection qui n'est pas pair (en anglais, \ct{even}), le test s'arrête. Pourtant, habituellement, nous voudrions bien continuer et voir aussi quels éléments (et donc combien) ne sont pas pairs (\ie ne répondent pas à \ct{even}) et peut-être aussi tracer cette information. Vous pouvez le faire de la façon suivante~:
% In this case, as soon as the test finds the first element of the collection that isn't
% \ct{even}, the test stops. 
% However, we would usually like to
% continue, and see both how many elements, and which elements, aren't
% \ct{even}, and maybe also log this information.  You can do this
% as follows:
% \begin{code}{}
% aCollection do:
% 	[:each |
% 	self
% 		assert: each even
% 		description: each printString , ' is not even'
% 		resumable: true]
% \end{code}
\begin{code}{}
aCollection do:
	[:each |
	self
		assert: each even
		description: each printString , ' n''est pas pair'
		resumable: true]
\end{code}
Pour chaque élément en échec, un message sera affiché dans le flux des traces. Les échecs ne sont pas cumulés, \ie si l'assertion échoue 10~fois dans la méthode de test, vous ne verrez qu'un seul échec. Toutes les autres méthodes d'assertion que nous avons vues ne permettent pas la reprise;
\ct{assert: p description: s} est équivalente à \ct{assert: p description: s resumable: false}.

% This will print out a message on your logging stream for each element
% that fails.  It doesn't accumulate failures, \ie if the assertion
% fails 10~times in your test method, you'll still only see one failure.
% All the other assertion methods that we have seen are not resumable;
% \ct{assert: p description: s} is equivalent to \ct{assert: p description: s resumable: false}.
\cmindex{Collection}{do:}
%=================================================================
%\section{The implementation of SUnit}
\section{La mise en {\oe}uvre de SUnit}

La mise en {\oe}uvre de \sunit constitue un cas d'étude intéressant de \ct{framework} \st.
\'Etudions quelques aspects clés de la mise en {\oe}uvre en suivant l'exécution d'un test.
% The implementation of \sunit makes an interesting case study of a \st framework.
% Let's look at some key aspects of the implementation by following the
% execution of a test.
%---------------------------------------------------------
% \subsection{Running one test}
\subsection{Exécuter un test}

Pour exécuter un test, nous évaluons l'expression \ct{(aTestClass selector: aSymbol) run.}

\begin{figure}[tbh]
  \begin{center} 
		{\includegraphics[width=0.7\textwidth]{sunit-scenario}}
	\caption{Exécuter un test.}
	\figlabel{sunit-scenario}
  \end{center}
\end{figure}

La méthode \cmind{TestCase}{run} crée une instance de \clsind{TestResult} qui collectera les résultats des tests; ensuite, elle s'envoie le message \mthind{TestCase}{run:}
(voir \figref{sunit-scenario}).
% The method \cmind{TestCase}{run} creates an instance of
% \clsind{TestResult} that will accumulate the results of the
% tests, then it sends itself the message \mthind{TestCase}{run:}.
% (See \figref{sunit-scenario}.)

\needlines{6}
\begin{method}[tastecaserun]{Exécuter un cas de test}
TestCase>>>run
	| result |
	result := TestResult new.
	self run: result.
	^result
\end{method}

% Note that in a future release, the class of the \ct{TestResult} to
% be created will be returned by a method so that new
%\ct{TestResult} can be introduced. }

La méthode \cmind{TestCase}{run:} envoie le message \mthind{TestResult}{runCase:} au résultat de test de classe \clsind{TestResult}~:

\begin{method}[testcaserun:]{Passage du case de test au TestResult}
TestCase>>>run: aResult
	aResult runCase: self
\end{method}
La méthode \ct{TestResult>>>runCase:} envoie le message \mthind{TestCase}{runCase} à un seul test pour l'exécuter.
\ct{TestResult>>>runCase} s'arrange avec toute exception qui pourrait être levée pendant l'exécution d'un test, évalue un \ct{TestCase} en lui envoyant le message \ct{runCase} et compte les erreurs, les échecs et les passes.

\begin{method}[testresultruncase]{Capture des erreurs et des échecs de test}
TestResult>>>runCase: aTestCase
	| testCasePassed |
	testCasePassed := true.
	[[aTestCase runCase] 
			on: self class failure
			do: 
				[:signal | 
				failures add: aTestCase.
				testCasePassed := false.
				signal return: false]]
					on: self class error
					do:
						[:signal |
						errors add: aTestCase.
						testCasePassed := false.
						signal return: false].
	testCasePassed ifTrue: [passed add: aTestCase]
\end{method}

La méthode \ct{TestCase>>>runCase} envoie les messages
\mthind{TestCase}{setUp} et \mthind{TestCase}{tearDown} comme montré
ci-dessous.
\needlines{3} % CHANGE
\begin{method}[testcaseruncase]{Modèle de méthode de test}
TestCase>>>runCase
	[self setUp.
	self performTest] ensure: [self tearDown]
\end{method} %CHANGE

\lr{Methods 7.15 and 7.17: use \#ensure: instead of \#sunitEnsure:, this has been changed in 3.9}

%---------------------------------------------------------
%\subsection{Running a \lct{TestSuite}}
\subsection{Exécuter un \ct{TestSuite}}
Pour exécuter plus d'un test, nous envoyons le message \ct{run} à un \ct{TestSuite} qui contient les tests adéquats. \ct{TestCase class} procure des fonctionnalités lui permettant de construire  une suite de tests. L'expression \ct{MyTestCase buildSuiteFromSelectors} retourne une suite contenant tous les tests définis dans la classe {\ct{MyTestCase}. Le c{\oe}ur de ce processus est:

\begin{method}[testcasetestselectors]{Auto-construction de la suite de test}
TestCase>>>testSelectors 
	^self selectors asSortedCollection asOrderedCollection select: [:each | 
		('test*' match: each) and: [each numArgs isZero]]
\end{method}
\cmindex{MyTestCase class}{buildSuiteFromSelectors}

La méthode \cmind{TestSuite}{run} crée une instance de \ct{TestResult}, vérifie que toutes les ressources sont disponibles avec \ct{areAllResourcesAvailable} puis envoie elle-même le message \mthind{TestSuite}{run:} qui exécute tous les tests de la suite. Toutes les ressources sont alors libérées.

\begin{method}[testsuiterun]{Exécuter une suite de tests}
TestSuite>>>run
	| result |
  self resources do: [ :res |
		res isAvailable ifFalse: [^res signalInitializationError]].
	[self run: result] ensure: [self resources do: [:each | each reset]].
	^result
\end{method}

\lr{Methods 7.15 and 7.17: use \#ensure: instead of \#sunitEnsure:, this has been changed in 3.9}

%\begin{method}[testsuiterun:]{Passing the test suite to the test result}
\begin{method}[testsuiterun:]{Passage de la suite de tests au TestResult}
TestSuite>>>run: aResult
	self tests do: [:each |
	  self changed: each.
	  each run: aResult]
\end{method}
La classe \clsind{TestResource} et ses sous-classes conservent la trace de leurs instances en cours (une par classe) pouvant être accédées et créées en utilisant la méthode de classe \mthind{TestResource class}{current}. Cette instance est nettoyée quand les tests ont fini de s'exécuter et que les ressources sont libérées.


Comme le montre la méthode de classe \cmind{TestResource class}{isAvailable} (en anglais, \emph{est-disponible}), le contrôle de la disponibilité de la ressource permet de la recréer en cas de besoin. 
Pendant sa création, l'instance de \ct{TestResource} est initialisée et la méthode \mthind{TestResource}{setUp} est invoquée.


\needlines{4}
\begin{method}[testresourceisavailable]{Disponibilité de la ressource de test}
TestResource class>>>isAvailable
	^self current notNil and: [self current isAvailable]
\end{method}
\begin{method}[testresourcecurrent]{Création de la ressource de test}
TestResource class>>>current
	current isNil ifTrue: [current := self new].
	^current
\end{method}
\begin{method}[restresourceinitialize]{Initialisation de la ressource de test}
TestResource>>>initialize
  super initialize.
	self setUp
\end{method}

%=================================================================
%\section{Some advice on testing}
\section{Quelques conseils sur les tests}

Bien que les mécanismes de tests soient simples, il n'est pas toujours
facile d'en écrire de bons.
Voici quelques conseils pour leur conception.
% While the mechanics of testing are easy, writing good tests is not.
% Here is some advice on how to design tests.

\begin{description}
%\item[Self-contained tests.] You do not
%  want to have to change your tests  each time you change your code, so try to write the tests
%  so that they are self-contained.  This can be difficult, but pays off in the
%  long term.  Writing tests in terms of stable interfaces supports
%  self-contained tests.
%  \on{I have no idea what you are trying to tell me.
%  What specifically should I do or not do?
%  Give an example!}

%\item[Do not over-test.] Try to build your tests so that they do not
%  overlap.  It is annoying to have many tests covering the same
%  functionality, because one bug in the code will then break many tests at the same time.
%  This is covered by Black's rule, below.

\index{Feathers, Michael}
\item[Les règles de Feathers.]
Michael Feathers, un auteur et consultant en processus agile
écrit~\footnote{Voir
  \url{http://www.artima.com/weblogs/viewpost.jsp?thread=126923} -- 9 Septembre 2005.}~:
\begin{quotation}
\noindent
 {\it
Un test n'est pas un test unitaire si~:
\begin{itemize}
\item il communique avec une base de données,
\item il communique au travers du réseau,
\item il modifie le système de fichiers,
\item il ne peut pas s'exécuter en même temps qu'un autre de vos tests unitaires ou
\item vous devez préparer votre environnement de façon particulière pour l'exécuter (comme éditer un fichier de configuration).
\end{itemize}
Des tests qui s'exécutent ainsi ne sont pas mauvais.   
Souvent ils valent la peine d'être écrits et ils peuvent être développés au sein d'un environnement de tests. 
Cependant, il est important de pouvoir les séparer des vrais tests unitaires de façon à ce qu'il soit possible de maintenir un ensemble de tests que nous pouvons exécuter rapidement à chaque fois que nous apportons nos modifications.
}
\end{quotation}
Ne vous placez jamais dans une situation où vous ne voulez pas lancer votre suite de tests unitaires parce que cela prend trop de temps.
% \item[Feathers' Rules for Unit tests.]
%   Michael Feathers, an  agile process consultant and author, writes:\footnote{See \url{http://www.artima.com/weblogs/viewpost.jsp?thread=126923}. 9 September 2005} 
%   \begin{quotation}
%   \noindent
%   {\it
%   A test is not a unit test if:
%   \begin{itemize}
% 	\item it talks to the database,
% 	\item it communicates across the network,
% 	\item it touches the file system,
% 	\item it can't run at the same time as any of your other unit tests, or
% 	\item you have to do special things to your environment (such as editing config files) to run it.
%  \end{itemize}
% Tests that do these things aren't bad. Often they are worth writing, and they can be written in a unit test harness. However, it is important to be able to separate them from true unit tests so that we can keep a set of tests that we can run fast whenever we make our changes.
%  }
%   \end{quotation}
% Never get yourself into a situation where you don't want to run your unit test suite because it takes too long.   
 
\item[Tests unitaires \textit{contre} tests d'acceptation.]
% Martial: j'ai préféré contre à versus
%\item[Tests unitaire \textit{vs.}\ tests d'acceptation.] 
Des tests unitaires capturent une partie de la fonctionnalité et, comme tels, permettent de faciliter l'identification des bugs de cette fonctionnalité. Essayez d'avoir, autant que possible, des tests unitaires pour chaque méthode pouvant potentiellement poser problème et regroupez-les par classe. Cependant, pour des situations profondément récursives ou complexes à installer, il est plus facile d'écrire des tests qui représentent un scénario cohérent pour l'application visée; ce sont des tests d'acceptation ou tests fonctionnels.
Des tests qui violent les principes de  Feathers peuvent faire de bons tests d'acceptation.
Groupez les tests d'acceptation en cohérence avec la fonctionnalité qu'ils testent.
Par exemple, si vous écrivez un compilateur, vous pourriez écrire des tests d'acceptation avec des assertions qui concernent le code généré pour chaque instruction utilisable du langage source.
De tels tests pourraient concerner beaucoup de classes et pourraient prendre beaucoup de temps pour s'exécuter parce qu'ils modifient le système de fichiers.
Vous pouvez les écrire avec \sunit, mais vous ne voudriez pas les exécuter à chaque modification mineure, ainsi ils doivent être séparés des vrais tests unitaires.
% \item[Unit Tests \textit{vs.}\ Acceptance Tests.] Unit tests capture one piece of
%   functionality, and as such make it easier to identify bugs in that functionality.
%   As far as
%   possible try to have unit tests for each method that could possibly fail, and group them per class.
%   However,
%   for certain deeply recursive or complex setup situations, it is
%   easier to write tests that represent a scenario in the larger application; these are called acceptance 
%   tests or functional tests.
%   Tests that break Feathers' rules may make good acceptance tests.
%   Group acceptance tests according to the functionality that they test.
%   For example, if you are writing a compiler, you might write acceptance tests that make 
%   assertions about the code generated for each possible source language statement.
%   Such tests might exercise many classes, and might take a long time to run because they touch the 
%   file system.
%   You can write them using \sunit, but you won't want to run them each time you make a small change,
%   so they should be separated form the true unit tests.

\item[Les règles de Black.]
\aconfirmer{Pour tout les tests du système, vous devriez être en mesure d'identifier une propriété pour laquelle le test renforce votre confiance. Il est évident qu'il ne devrait pas y avoir de propriété importante que vous ne testez pas. Cette règle établit le fait moins évident qu'il ne devrait pas y avoir de tests sans valeur ajoutée de nature à accroître votre confiance envers une propriété utile.}
Par exemple, il n'est pas bon d'avoir plusieurs tests pour la même propriété. En fait, c'est nuisible~: ils rendent la compréhension de la classe plus difficile à déduire à la lecture des tests et un bug dans le code est susceptible de casser beaucoup de tests en même temps. Ne pensez qu'à une seule propriété quand vous écrivez un test.  
% \item[Black's Rule of Testing.]
%   For every test in the system, you should be able to identify some property for which
%   the test increases your confidence.
%   It's obvious that there should be no important property that you are not testing.
%   This rule states the less obvious fact that there should be
%   no test that does not add value to the system by increasing your confidence that a useful property
%   holds.
%   For example, several tests of the same property do no good. 
%   In fact they do harm: they make it harder to infer the behaviour of the class by reading the tests, and
%   because one bug in the code might then break many tests at the same time.
%   Have a property in mind when you write a test.
\end{description}

%\section{Extending \SUnit}
%\seclabel{extending}

%In this section we will explain how to extend \sunit so that it uses
%a \ct{setUp} and \ct{tearDown} that are shared by all of the
%tests in a \ct{TestCase} subclass.  We will define a new sublass
%of \ct{TestCase} called \ct{SharingSetUpTestCase}, and a
%subclass of \ct{SharingSetUpTestCase} called \ct{SharedOne}.
%We will also need to define a new subclass of \ct{TestSuite}
%called \ct{SharedSetUpTestSuite}, and we will make some minor
%adjustments to \ct{TestCase}.

%Our tests will be in \ct{SharedOne}.  When we execute
%\begin{script}
%Transcript clear.
%SharedOne suite run
%\end{script}
%we will obtain the following trace.
%\begin{code}{}
%SharedOne>>>setUp
%SharedOne class>>>sharedSetUp
%SharedOne>>>testOne
%SharedOne>>>tearDown
%SharedOne>>>setUp
%SharedOne>>>testTwo
%SharedOne>>>tearDown
%SharedOne class>>>sharedTearDown
%2 run, 2 passed, 0 failed, 0 errors
%\end{code}
%You can see that the shared code is executed just once for both
%tests.

%\subsection{\ct{SharedSetUpTestCase}}

%The extension of the \sunit framework is based on the introduction
%of two classes: \ct{SharedSetUpTestCase} and
%\ct{SharedSetUpTestSuite}.  The basic idea is to use a flag that
%is flushed (cleared) after a certain number of tests have been run.
%The class \ct{SharedSetUpTestCase} defines one instance variable
%that indicates whether each test is run individually or in the context
%of a shared \ct{setUp} and \ct{tearDown}.  There are also two
%class instance variables.  One indicates the number of tests for which
%the shared \ct{setUp} should be in effect, and the other indicates
%whether the shared \ct{setUp} is in effect.
%\begin{method}[xxx]{xxx}
%SharedSetUpTestCase
%	superclass: TestCase
%	instanceVariableNames: 'runIndividually '
%	classInstanceVariableNames: 'numberOfTestsToTearDown
%								 sharedSetUp '
%\end{method}
%\ct{suiteClass} is used by \ct{TestCase} to determine the
%suite that is running.
%\begin{method}[xxx]{xxx}
%SharedSetUpTestCase class>>>suiteClass
%	^SharedSetUpTestSuite
%\end{method}
%\begin{method}[xxx]{xxx}
%SharedSetUpTestCase class>>>sharedSetUp
%	"A subclass should only override this hook to define
%	 a sharedSetUp"
%\end{method}
%\begin{method}[xxx]{xxx}
%SharedSetUpTestCase class>>>sharedTearDown
%	"Here we specify the teardown of the shared setup"
%\end{method}
%\begin{method}[xxx]{xxx}
%SharedSetUpTestCase class>>>flushSharedSetUp
%	sharedSetUp := nil
%\end{method}
%The \ct{SharedSetUpTestCase} class is initialized with the number
%of tests for which the shared \ct{setUp} should be in effect.
%\begin{method}[xxx]{xxx}
%SharedSetUpTestCase class>>>armTestsToTearDown: aNumber
%	self flushSharedSetUp.
%	numberOfTestsToTearDown := aNumber.
%\end{method}
%Every time a test is run, the method \ct{anothertestHasBeenRun} is
%invoked.  Once the specified number of tests is reached the
%\ct{sharedSetUp} is flushed and the \ct{sharedTearDown} is
%executed.
%\begin{method}[xxx]{xxx}
%SharedSetUpTestCase class>>>anotherTestHasBeenRun
%	"Everytimes a test is run this method is called,
%	 once all the tests of the suite
%	 are run the shared setup is reset"
%	numberOfTestsToTearDown := numberOfTestsToTearDown - 1.
%	numberOfTestsToTearDown isZero
%		ifTrue:
%			[self flushSharedSetUp.
%			self sharedTearDown]
%\end{method}
%When a test is run its \ct{setUp} is executed and it then it calls
%the class method \ct{privateSharedSetUp}.  This method will only
%invoke the \ct{sharedSetUp} if the \ct{sharedSetUp} test
%indicates that it hasn't been done yet.
%\begin{method}[xxx]{xxx}
%SharedSetUpTestCase class>>>privateSharedSetUp
%	sharedSetUp isNil
%		ifTrue:
%			[sharedSetUp := 1.
%			self sharedSetUp]
%\end{method}
%\begin{method}[xxx]{xxx}
%SharedSetUpTestCase>>>setUp
%	self class privateSharedSetUp
%\end{method}
%\begin{method}[xxx]{xxx}
%SharedSetUpTestCase>>>tearDown
%	self class anotherTestHasBeenRun
%\end{method}
%When a test case is created we assume that it will be run once.  We
%can change this later by invoking the method
%\ct{executedFromASuite}.
%\begin{method}[xxx]{xxx}
%SharedSetUpTestCase>>>setTestSelector: aSymbol
%	"Must do it this way because there is no initialize"

%	runIndividually := true.
%	super setTestSelector: aSymbol
%\end{method}
%\begin{method}[xxx]{xxx}
%SharedSetUpTestCase>>>executedFromASuite
%	runIndividually := false
%\end{method}
%The methods responsible for test execution are then specialized as
%follows.
%\begin{method}[xxx]{xxx}
%runIndividually
%	^runIndividually
%\end{method}
%\begin{method}[xxx]{xxx}
%SharedSetUpTestCase>>>armTearDownCounter
%	self runIndividually
%		ifTrue: [self class armTestsToTearDown: 1]
%\end{method}
%\begin{method}[xxx]{xxx}
%SharedSetUpTestCase>>>runCaseAsFailure
%	self armTearDownCounter.
%	super runCaseAsFailure
%\end{method}
%\begin{method}[xxx]{xxx}
%SharedSetUpTestCase>>>runCase
%	self armTearDownCounter.
%	super runCase
%\end{method}

%\subsection{\ct{SharedOne}}

%\ct{SharedOne} is a new class which inherits from
%\ct{SharingSetUpTestCase} as follows.  We define two simple tests
%\ct{testOne} and \ct{testTwo}.
%\begin{method}[xxx]{xxx}
%SharedOne
%	superclass: SharingSetUpTestCase
%\end{method}
%\begin{method}[xxx]{xxx}
%SharedOne>>>testOne
%	Transcript
%		show: 'SharedOne>>>testOne';
%		cr
%\end{method}
%\begin{method}[xxx]{xxx}
%SharedOne>>>testTwo
%	Transcript
%		show: 'SharedOne>>>testTwo';
%		cr
%\end{method}
%Then we define the methods \ct{setUp} and \ct{tearDown} that
%will be executed before and after the execution of the tests exactly
%in the same way as with non sharing tests.  Note however, the fact
%that with the solution we will present we have to explicitly invoke
%the \ct{setUp} method and \ct{tearDown} of the superclass.
%\begin{method}[xxx]{xxx}
%SharedOne>>>setUp
%	Transcript
%		show: 'SharedOne>>>setUp';
%		cr.
%	super setUp
%\end{method}
%\begin{method}[xxx]{xxx}
%SharedOne>>>tearDown
%	Transcript
%		show: 'SharedOne>>>tearDown';
%		cr.
%	super tearDown
%\end{method}
%Finally, we define the methods \ct{sharedSetUp} and
%\ct{sharedTearDown} that will be only executed once for the two
%tests.  Note that this solution assumes that the tests are not
%destructive to the shared fixture, but just query it.
%\begin{method}[xxx]{xxx}
%SharedOne class>>>sharedSetUp
%	Transcript
%		show: 'SharedOne class>>>sharedSetUp';
%		cr
%	"My set up here."
%\end{method}
%\begin{method}[xxx]{xxx}
%SharedOne class>>>sharedTearDown
%	Transcript
%		show: 'SharedOne class>>>sharedTearDown';
%		cr
%	"My tear down here."
%\end{method}

%\subsection{\ct{SharedSetUpTestSuite}}

%The \ct{SharedSetUpTestSuite} defines just one instance variable
%\ct{testCaseClass} and redefines the two methods necessary to run
%the test suite \ct{run:} and \ct{run}.
%\ct{checkAndArmSharedSetUp} initializes the number of tests to run
%before the shared \ct{tearDown} is executed.
%\begin{method}[xxx]{xxx}
%SharedSetUpTestSuite
%	superclass: TestSuite
%	instanceVariableNames: 'testCaseClass'
%\end{method}
%\begin{method}[xxx]{xxx}
%SharedSetUpTestSuite>>>checkAndArmSharedSetUp
%	self tests isEmpty
%		ifFalse: [self tests first class
%				 armTestsToTearDown: self tests size]
%\end{method}
%\begin{method}[xxx]{xxx}
%SharedSetUpTestSuite>>>run: aResult
%	self checkAndArmSharedSetUp.
%	^super run: aResult
%\end{method}
%\begin{method}[xxx]{xxx}
%SharedSetUpTestSuite>>>run
%	self checkAndArmSharedSetUp.
%	^super run
%\end{method}
%Finally the method \ct{addTest:} is specialized so that it marks
%all its tests with the fact that they are executed in a
%\ct{TestSuite} and checks whether all its tests are from the same
%class to avoid inconsistency.
%\begin{method}[xxx]{xxx}
%SharedSetUpTestSuite>>>addTest: aTest
%	"Sharing a setup only works if the test case
%	composing the test suite are from
%	the same class so we test it"

%	aTest executedFromASuite.
%	testCaseClass isNil
%		ifTrue: [testCaseClass := aTest class.
%				super addTest: aTest ]
%		ifFalse: [aTest class == testCaseClass
%				  ifFalse: [self error:
%						   'you cannot have test case of
%							different classes in
%							a SharingSetUpTestSuite'.]
%				  ifTrue: [super addTest: aTest]]
%\end{method}

%\subsection{Changes to \ct{TestCase}}

%In order for the above changes to work, you must make
%\ct{TestCase} aware of your new test suite.
%\begin{method}[xxx]{xxx}
%TestCase class>>>buildSuite
%	| suite |
%	^self isAbstract
%		ifTrue:
%			[suite := self suiteClass new.
%			suite name: self name asString.
%			self allSubclasses
%				do: [:each |
%					each isAbstract
%						ifFalse: [suite addTest:
%						  each buildSuiteFromSelectors]].
%			suite]
%		ifFalse: [self buildSuiteFromSelectors]
%\end{method}
%\begin{method}[xxx]{xxx}
%TestCase class>>>buildSuiteFromMethods: testMethods
%	^testMethods
%		inject: ((self suiteClass new)
%				name: self name asString;
%				yourself)
%		into:
%			[:suite :selector |
%			suite
%				addTest: (self selector: selector);
%				yourself]
%\end{method}
%If you have made all the changes correctly, you should be able to run
%your tests and see the results shown in section~\ref{sec:extending}.
%
%\section{Exercise}

%The previous section was designed to give you some insight into the
%workings of \SUnit.  You can obtain the same effect by using \SUnit's
%resources.

%Create new classes \ct{MyTestResource} and \ct{MyTestCase}
%which are subclasses of \ct{TestResource} and \ct{TestCase}
%respectively.  Add the appropriate methods so that the following
%messages are written to the \ct{Transcript} when you run your
%tests.

%\begin{method}[xxx]{xxx}
%MyTestResource>>>setUp has run.
%MyTestCase>>>setUp has run.
%MyTestCase>>>testOne has run.
%MyTestCase>>>tearDown has run.
%MyTestCase>>>setUp has run.
%MyTestCase>>>testTwo has run.
%MyTestCase>>>tearDown has run.
%MyTestResource>>>tearDown has run.
%\end{method}

%% You need to write the following six methods.

%% MyTestCase>>>setUp
%%	 Transcript
%%		 show: 'MyTestCase>>>setUp has run.';
%%		 cr

%% MyTestCase>>>tearDown
%%	 Transcript
%%		 show: 'MyTestCase>>>tearDown has run.';
%%		 cr

%% MyTestCase>>>testOne
%%	 Transcript
%%		 show: 'MyTestCase>>>testOne has run.';
%%		 cr

%% MyTestCase>>>testTwo
%%	 Transcript
%%		 show: 'MyTestCase>>>testTwo has run.';
%%		 cr

%% MyTestCase class>>>resources
%%	 ^Array with: MyTestResource

%% MyTestResource>>>setUp
%%	 Transcript
%%		 show: 'MyTestResource>>>setUp has run';
%%		 cr

%% MyTestResource>>>tearDown
%%	 Transcript
%%		 show: 'MyTestResource>>>tearDown has run.';
%%		 cr
%=================================================================
\section{Résumé du chapitre}

\aconfirmer{Ce chapitre a expliqué en quoi les tests constituent un investissement important pour le futur de votre code.
Nous avons expliqué, étape par étape, comment spécifier quelques tests pour la classe \ct{Set}.
Ensuite, nous avons décrit simplement le c{\oe}ur de l'environnement \sunit en présentant les classes \ct{TestCase}, \ct{TestResult}, \ct{TestSuite} et \ct{TestResources}. Finalement, nous avons détaillé \sunit en suivant l'exécution d'un test et d'une suite de tests.}
% This chapter explained why tests are an important investment in 
% the future of your code.  
% We explained in a step-by-step fashion how
% to define a few tests for the class \ct{Set}.
% Then we gave an overview of the core of the \sunit framework by presenting
% the classes \ct{TestCase}, \ct{TestResult}, \ct{TestSuite}
% and \lct{TestResources}.  Finally we looked deep inside \sunit by
% following the execution of a test and a test suite.

\begin{itemize}
\item Pour maximiser leur potentiel, des tests unitaires devraient être rapides, réitérables, indépendants d'une intervention humaine et couvrir une seule partie de fonctionnalité.
\item Les tests pour la classe nommée \ct{MyClass} sont dans la classe nommée \ct{MyClassTest} qui devrait être implantée comme une sous-classe de \ct{TestCase}.
\item Initialisez vos données de test dans une méthode \ct{setUp}.
\item Chaque méthode de test devrait commencer par le mot ``\ct{test}''.
\item Utilisez les méthodes de \ct{TestCase} comme \ct{assert:}, \ct{deny:} et autres, pour établir vos assertions.
\item Exécutez les tests en utilisant l'exécuteur de tests \sunit (dans \toolsflap).
%   \item To maximize their potential, unit tests should be fast, repeatable, independent of any direct human interaction and cover a single unit of functionality.
%   \item Tests for a class called \ct{MyClass} belong in a class classed \ct{MyClassTest}, which should be introduced as a subclass of \lct{TestCase}.
%   \item Initialize your test data in a \ct{setUp} method.
%   \item Each test method should start with the word ``test''.
%   \item Use the \ct{TestCase} methods \ct{assert:}, \ct{deny:} and others to make assertions.
%   \item Run tests using the SUnit test runner tool (in the tool bar).
\end{itemize}

%=============================================================
\ifx\wholebook\relax\else
   \bibliographystyle{jurabib}
   \nobibliography{scg}
   \end{document}
\fi
%=============================================================
%%% Local Variables:
%%% coding: utf-8
%%% mode: latex
%%% TeX-master: t
%%% TeX-PDF-mode: t
%%% ispell-local-dictionary: "english"
%%% End:

%:Basic Classes
% $Author: oscar $
% $Date: 2007-09-23 11:56:47 +0200 (dim, 23 sep 2007) $
% $Revision: 12130 $
% traduction : Nicolas Petton 
% relecture : Alain Plantec 
% relecture: Martial Boniou (sun, 23 dec 2007) de la version #14626 
% relecture : Rene Mages (sat, 9 jan 2008) de la version #14856
% export Pharo : Martial Boniou à partir de la revision 28661 - oscar
% 2009-08-28 11:31:14 +0200 (Fri, 28 Aug 2009) $
%=================================================================
\ifx\wholebook\relax\else
% --------------------------------------------
% Lulu:
	\documentclass[a4paper,10pt,twoside]{book}
	\usepackage[
		papersize={6.13in,9.21in},
		hmargin={.75in,.75in},
		vmargin={.75in,1in},
		ignoreheadfoot
	]{geometry}
	\input{../common.tex}
	\pagestyle{headings}
	\setboolean{lulu}{true}
% --------------------------------------------
% A4:
%	\documentclass[a4paper,11pt,twoside]{book}
%	\input{../common.tex}
%	\usepackage{a4wide}
% --------------------------------------------
    \graphicspath{{figures/} {../figures/}}
	\begin{document}
	\renewcommand{\nnbb}[2]{} % Disable editorial comments
	\sloppy
\fi
%=================================================================
\chapter{Les classes de base}
\chalabel{basic}
Une grande partie de la magie de \st ne r\'eside pas dans son langage mais dans ses biblioth\`eques de classes. Pour programmer efficacement en \st, vous devez apprendre comment les biblioth\`eques de classes servent le langage et l'environnement. Les biblioth\`eques de classes sont enti\`erement \'ecrites en \st et peuvent facilement \^etre \'etendues, puisqu'un paquetage peut ajouter une nouvelle fonctionnalit\'e \`a une classe m\^eme s'il ne d\'efinit pas cette classe. 

%martial: typo: 'methodes-cle' et 'classes-cle' mais bien 'classes et
%methodes cles' (cles est ici adjectifs)
Notre but ici n'est pas de pr\'esenter en d\'etail l'int\'egralit\'e
des biblioth\`eques de classes de \pharo, mais plut\^ot d'indiquer
quelles classes et m\'ethodes cl\'es vous devrez utiliser ou surcharger pour programmer efficacement. Ce chapitre couvre les classes de base qui vous seront utiles dans la plupart de vos applications: \ct{Object}, \ct{Number} et ses sous-classes, \ct{Character}, \ct{String}, \ct{Symbol} et \ct{Boolean}.

\md{Here are some comments:\\
- copying: Good question... the copying in Pharo is much too complicated... there is for one the "old" smalltalk way of
  overrifing postCopy, and then the "automatic" deepCopy... which is quite complex and (I think) was no good idea...
 (see class comment in  DeepCopier)\\
- Debugging: Yes, needs its own chapter. We should talk about haltIf, haltOnce...\\
- assert: Object>>>assert: can take both a block and a boolean, because boleen implements \#value.
  (I will fix SUnit to allow both, too).\\
- Characters and Strings: we should talk about Unicode stuff... but I don't know too much myself.}

%=================================================================
\section{Object}

Dans tous les cas, \clsindmain{Object} est la racine de la hi\'erarchie d'h\'eritage. En r\'ealit\'e, dans \pharo , la vraie racine de la hi\'erarchie est \clsind{ProtoObject} qui est utilis\'ee pour d\'efinir les entit\'es minimales qui se font passer pour des objets, mais nous pouvons ignorer ce point pour l'instant.
% (more on this later in the chapter on reflection).

La classe \ct{Object} peut \^etre trouv\'ee dans la cat\'egorie
\scatind{Kernel-Objects}. \'Etonnamment, nous y trouvons plus de 400
m\'ethodes (avec les extensions). En d'autres termes, toutes les
classes que vous d\'efinirez seront automatiquement munies de ces 400
m\'ethodes, que vous sachiez ou non ce qu'elles font.
%martial: changement a la relecture 
Notez que certaines de ces m\'ethodes devraient \^etre supprim\'ees et
que dans les nouvelles versions de \pharo certaines m\'ethodes superflues
pourraient l'\^etre.

\sd{I do not like to quote something that can change and that people can find simply in the image but let us keep it for now.}
Le commentaire de la classe \ct{Object} indique:
\needlines{4}
\begin{quote}
\textit{\ct{Object} est la classe racine de la plupart des autres classes dans la hi\'erarchie des classes. Les exceptions sont \ct{ProtoObject} (super-classe de \ct{Object}) et ses sous-classes.
La classe \ct{Object} fournit le comportement par d\'efaut, commun \`a tous les objets classiques, comme l'acc\`es, la copie, la comparaison, le traitement des erreurs, l'envoi de messages et la \ind{r\'eflexion}. Les messages utiles auxquels tous les objets devraient r\'epondre sont \'egalement d\'efinis ici.
\ct{Object} n'a pas de variable d'instance, aucune ne devrait \^etre
cr\'e\'ee. Ceci est d\^u aux nombreuses classes d'objets qui
h\'eritent de \ct{Object} et qui ont des impl\'ementations
particuli\`eres (\ct{SmallInteger} et \ct{UndefinedObject} par
exemple) ou \`a certaines classes standards que la VM conna\^it et
pour lesquelles leur structure et leur organisation sont importantes.}
\end{quote}

Si nous naviguons dans les cat\'egories des m\'ethodes d'instance de \ct{Object}, nous commen\c{c}ons \`a voir quelques-uns des comportements-cl\'e qu'elle offre.

%-----------------------------------------------------------------
%titre de sous-section = mot-cle sans article
\subsection{Impression}
Tout objet en \st peut renvoyer une forme imprim\'ee de lui-m\^eme. Vous pouvez s\'electionner n'importe quelle expression dans un Workspace et s\'electionner le menu \menu{print it}: ceci ex\'ecute l'expression et demande \`a l'objet renvoy\'e de s'imprimer. En r\'ealit\'e le message \ct{printString} est envoy\'e \`a l'objet retourn\'e. La m\'ethode \mthind{Object}{printString}, qui est une \ind{méthode générique}, envoie le message \mthind{Object}{printOn:} \`a son receveur. Le message \ct{printOn:} est un point d'entr\'ee qui peut \^etre sp\'ecialis\'e. 

%martial: il faudra peut-etre ajouter une remarque et un lien pour
%expliquer pourquoi 'a' ou 'an' (article indefini). Dans le chapitre
%Syntax, nous devrions mettre une note de traducteurs. 
\ct{Object>>>printOn:} est une des m\'ethodes que vous surchargerez le
plus souvent. Cette m\'ethode prend comme argument un flux
(\clsind{Stream}) dans lequel une repr\'esentation en cha\^{\i}ne de
caract\`eres (\clsind{String}) de l'objet sera
\'ecrite. L'impl\'ementation par d\'efaut \'ecrit simplement le nom de
la classe pr\'ec\'ed\'ee par ``\ct{a}'' ou
``\ct{an}''. \ct{Object>>>printString} retourne la cha\^{\i}ne de
caract\`eres (\ct{String}) qui est \'ecrite.

Par exemple, la classe \clsind{Browser} ne red\'efinit pas la
m\'ethode \ct{printOn:} et, envoyer le message \ct{printString} \`a
une %
%ajout
de ces instances %
ex\'ecute les m\'ethodes d\'efinies dans \ct{Object}. 
\begin{code}{@TEST}
Browser new printString --> 'a Browser'
\end{code}

\arelire{La classe \ct{Color} montre un exemple de spécialisation de 
\mthind{Color}{printOn:}. Elle imprime le nom de la classe suivi par
le nom de la méthode de classe utilisée pour générer cette couleur,
comme le montre le code ci-dessous qui imprime une instance de cette
classe.} % CHANGE

% \needlines{7}
%martial: caption plus coherent que 'printOn: redefinition.'
\begin{method}[zork]{Red\'efinir printOn:}
Color>>>printOn: aStream
	| name |
	(name := self name) ifNotNil: 
		[ ^ aStream
			nextPutAll: 'Color ';
			nextPutAll: name ].
	self storeOn: aStream
\end{method}\ignoredollar$ % CHANGE $

\begin{code}{@TEST}
Color red printString --> 'Color red'
\end{code} % CHANGE

Notez que le message \ct{printOn:} n'est pas le m\^eme que \mthind{Object}{storeOn:}. Le message \ct{storeOn:} ajoute au flux pass\'e en argument une expression pouvant \^etre utilis\'ee pour recr\'eer le receveur. Cette expression est \'evalu\'ee quand le flux est lu avec le message \ct{readFrom:}. \ct{printOn:} retourne simplement une version textuelle du receveur. Bien s\^ur, il peut arriver que cette repr\'esentation textuelle puisse repr\'esenter le receveur sous la forme d'une expression auto-\'evalu\'ee.

\paragraph{Un mot \`a propos de la repr\'esentation et de la repr\'esentation auto-\'evalu\'ee.}
En programmation fonctionnelle, les expressions retournent des valeurs
quand elles sont \'evalu\'ees. En \st, les messages (expressions)
retournent des objets (valeurs). Certains objets ont la propri\'et\'e
sympathique d'\^etre eux-m\^emes leur propre valeur. Par exemple, la
valeur de l'objet \ct{true} est lui-m\^eme, \ie l'objet
\ct{true}. Nous appelons de tels objets des \emph{objets}
\emphsubind{objet}{auto-évalué}{}\emph{s}. 
Vous pouvez voir une version  \emph{imprimée} de la valeur d'un objet quand vous imprimez l'objet dans un \ct{Workspace}. Voici quelques exemples de telles expressions auto-\'evalu\'ees. 

\begin{code}{@TEST}
true      --> true
3@4       --> 3@4
$a        --> $a
#(1 2 3)  --> #(1 2 3)
Color red --> Color red
\end{code} % CHANGE

Notez que certains objets comme les tableaux sont auto-\'evalu\'es ou
non suivant les objets qu'ils contiennent. Par exemple, un tableau de
bool\'eens est auto-\'evalu\'e alors qu'un tableau de personnes ne
l'est pas.
%Dans \squeak 3.9, un m\'ecanisme a \'et\'e introduit (via le % CHANGE
%message \mthind{Object}{isSelfEvaluating}) pour imprimer autant que
%possible des collections dans leur forme auto-\'evalu\'ee. Ceci est
%particuli\`erement vrai pour les tableaux dynamiques.
L'exemple suivant montre qu'un tableau \subind{tableau}{dynamique} est
auto-\'evalu\'e seulement si ses \'el\'ements le sont:
\begin{code}{@TEST}
{10@10. 100@100}          --> {10@10. 100@100}
{Browser new . 100@100}    --> an Array(a Browser 100@100)
\end{code}

Rappelez-vous que les tableaux littéraux ne peuvent contenir que des litt\'eraux. Ainsi le tableau suivant ne contient pas deux \'el\'ements mais six \'el\'ements litt\'eraux.
\index{littéral!tableau}
\seeindex{tableau!littéral}{littéral, tableau}
\begin{code}{@TEST}
#(10@10 100@100) --> #(10 #@ 10 100 #@ 100)
\end{code}

Beaucoup de sp\'ecialisations de la m\'ethode \ct{printOn:} impl\'ementent le comportement d'auto-\'evaluation. Les impl\'ementations de \cmind{Point}{printOn:} et \cmind{Interval}{printOn:} sont auto-\'evalu\'ees.

\begin{method}[Self-evaluating points]{Auto-\'evaluation de \ct{Point}}
Point>>>printOn: aStream 
    "The receiver prints on aStream in terms of infix notation."
    x printOn: aStream.
    aStream nextPut: $@.
    y printOn: aStream
\end{method}\ignoredollar$

%martial: ajout de la traduction du commentaire
Le commentaire de cette m\'ethode dit que le receveur imprime sur le
flux \ct{aStream} avec une insertion dans la notation. %infixe

\begin{method}[Self-evaluating intervals]{Auto-\'evaluation de \ct{Interval}}
Interval>>>printOn: aStream
    aStream nextPut: $(;
        print: start;
        nextPutAll: ' to: ';
        print: stop.
    step ~= 1 ifTrue: [aStream nextPutAll: ' by: '; print: step].
    aStream nextPut: $)
\end{method}

\begin{code}{@TEST}
1 to: 10 --> (1 to: 10)    "!les intervalles sont auto-\'evalu\'es!"
\end{code}

%-----------------------------------------------------------------
\subsection{Identit\'e et \'egalit\'e}

En \st, le message \ct{=} teste l'\emphsubindmain{Object}{\'egalit\'e} d'objets (\ie si deux objets repr\'esentent la m\^eme valeur) alors que \ct{==} teste l'\emphsubindmain{Object}{identit\'e} (\ie si deux expressions repr\'esentent le m\^eme objet).
\seeindex{\ct{=}}{Object, \'egalit\'e}
\seeindex{\ct{==}}{Object, identit\'e}
\seeindex{\'egalit\'e}{Object, \'egalit\'e}
\seeindex{identit\'e}{Object, identit\'e}

L'impl\'ementation par d\'efaut de l'\'egalit\'e entre objets teste l'identit\'e d'objets:
%martial: j'ai prefere mettre un titre a la methode
\begin{method}{\'Egalit\'e par d\'efaut}
Object>>>= anObject
    "Answer whether the receiver and the argument represent the same object.
    If = is redefined in any subclass, consider also redefining the message hash."
    ^ self == anObject
\end{method}
\cmindex{Object}{=}

C'est une m\'ethode que vous voudrez souvent surcharger. Consid\'erez le cas de la classe des nombres complexes \ct{Complex}:

\begin{code}{@TEST}
(1 + 2 i) = (1 + 2 i)   --> true     !"m\^eme valeur"!
(1 + 2 i) == (1 + 2 i)  --> false    !"mais objets diff\'erents"!
\end{code}

Ceci fonctionne parce que \ct{Complex} surcharge \ct{=} comme suit:

\needlines{5}
\begin{method}{\'Egalit\'e des nombres complexes}
Complex>>>= anObject
    anObject isComplex
        ifTrue: [^ (real = anObject real) & (imaginary = anObject imaginary)]
        ifFalse: [^ anObject adaptToComplex: self andSend: #=]
\end{method}
\cmindex{Complex}{=}

L'impl\'ementation par d\'efaut de \ct{Object>>>~=} renvoie simplement l'inverse de \ct{Object>>>=} et ne devrait normalement pas \^etre modifi\'ee.
%\cmindex{Object}{\~=}
\index{Object!~=@\ct{~=}} % needs special treatment due to ~

\begin{code}{@TEST}
(1 + 2 i) ~= (1 + 4 i) --> true
\end{code}

Si vous surchargez \ct{=}, vous devriez envisager de surcharger
\mthind{Object}{hash}. Si des instances de votre classe sont
utilis\'ees comme cl\'es dans un dictionnaire (\clsind{Dictionary}),
vous devrez alors vous assurer que les instances qui sont
consid\'er\'ees \'egales ont la m\^eme valeur de hachage (\ct{hash}):

\begin{method}{\ct{hash} doit \^etre r\'e-impl\'ement\'ee pour les nombres complexes}
Complex>>>hash
    "Hash is reimplemented because = is implemented."
    ^ real hash bitXor: imaginary hash.
\end{method}
\cmindex{Complex}{hash}

%martial: j'ai retire la phrase sur l'identite des objets des
%parentheses parce que c'est important.
Alors que vous devez surcharger \`a la fois \ct{=} et \ct{hash}, vous
ne devriez \emph{jamais} surcharger \ct{==} puisque la s\'emantique de l'identit\'e d'objets est la m\^eme pour toutes les classes.  \ct{==} est une m\'ethode primitive de \clsind{ProtoObject}.

Notez que \pharo a certains comportements \'etranges compar\'e \`a d'autres \st{}s: par exemple, un symbole et une cha\^{\i}ne de caract\`eres peuvent \^etre \'egaux si la cha\^{\i}ne de caract\`eres associ\'ee au symbole est \'egale \`a la cha\^{\i}ne de caract\`eres (nous consid\'erons ce comportement comme un bug, pas comme une fonctionnalit\'e).

\begin{code}{@TEST}
#'lulu' = 'lulu' --> true
'lulu' = #'lulu' --> true
\end{code}


%-----------------------------------------------------------------
\subsection{Appartenance \`a une classe}
Plusieurs m\'ethodes vous permettent de conna\^itre la classe d'un objet.
%martial: exceptionnellement, il ne faut pas mettre de point dans
%paragraph parce que ce sont des noms de methodes (de plus, le
%changement de fonte est assez explicite (mthind))

\paragraph{\mthind{Object}{class}} Vous pouvez demander \`a tout objet sa classe en utilisant le message \ct{class}.
\begin{code}{@TEST}
1 class --> SmallInteger
\end{code}

Inversement, vous pouvez demander si un objet est une instance 
%ajout
(\ct{isMemberOf:})
d'une classe sp\'ecifique:
\cmindex{Object}{isMemberOf:}
\begin{code}{@TEST}
1 isMemberOf: SmallInteger --> true    "doit !\^etre pr\'ecis\'ement cette classe!"
1 isMemberOf: Integer      --> false
1 isMemberOf: Number       --> false
1 isMemberOf: Object       --> false
\end{code}

Puisque \st est \'ecrit en lui-m\^eme, vous pouvez vraiment naviguer au travers de sa structure en utilisant la bonne combinaison de messages \ct{superclass} et \ct{class} (voir \charef{metaclasses}). 

\paragraph{\ct{isKindOf:}}
\cmind{Object}{isKindOf:} r\'epond \ct{true} si la classe du receveur est la m\^eme ou une des sous-classes de la classe de l'argument.

\begin{code}{@TEST}
1 isKindOf: SmallInteger --> true
1 isKindOf: Integer          --> true
1 isKindOf: Number         --> true
1 isKindOf: Object           --> true
1 isKindOf: String            --> false

1/3 isKindOf: Number      --> true
1/3 isKindOf: Integer        --> false
\end{code}

\ct{1/3}, qui est une \clsind{Fraction}, est aussi une sorte de nombre (\clsind{Number}), puisque la classe \ct{Number} est une super-classe de la classe \ct{Fraction}, mais \ct{1/3} n'est pas un entier (\ct{Integer}).

\paragraph{\ct{respondsTo:}}
\cmind{Object}{respondsTo:} r\'epond \ct{true} si le receveur comprend le message dont le s\'electeur est pass\'e en argument.

\begin{code}{@TEST}
1 respondsTo: #, --> false
\end{code}

C'est normalement une mauvaise id\'ee de demander sa classe \`a un
objet ou de lui demander quels messages il comprend.
Au lieu de prendre des d\'ecisions bas\'ees sur la classe d'un objet, vous devriez simplement envoyer un message \`a cet objet et le laisser d\'ecider (\ie sur la base de sa classe) comment il doit se comporter.

%-----------------------------------------------------------------
\subsection{Copie}

Copier des objets introduit quelques probl\`emes subtils. Puisque les variables d'instance sont accessibles par r\'ef\'erence, une \emphsubind{Object}{copie superficielle}, les r\'ef\'erences port\'ees par les variables d'instance devraient \^etre partag\'ees entre l'objet produit par la copie et l'objet original:
\seeindex{copie}{Object, \ct{copy}}
\seeindex{copie superficielle}{Object, \ct{shallowCopy}}
\seeindex{copie profonde}{Object, \ct{deepCopy}}

\begin{code}{@TEST | a1 a2 |}
a1 := { { 'harry' } }.
a1 --> #(#('harry'))
a2 := a1 shallowCopy.
a2 --> #(#('harry'))
(a1 at: 1) at: 1 put: 'sally'.
a1 --> #(#('sally'))
a2 --> #(#('sally'))    "!le tableau contenu est partag\'e!"
\end{code}

\cmind{Object}{shallowCopy} est une m\'ethode primitive qui cr\'ee une copie superficielle d'un objet. Puisque \ct{a2} est seulement une copie superficielle de \ct{a1}, les deux tableaux partagent une r\'ef\'erence au tableau (\ct{Array}) qu'ils contiennent.

\ct{Object>>>shallowCopy} est une ``interface publique'' pour \cmind{Object}{copy} et devrait \^etre surcharg\'ee si les instances sont uniques. C'est le cas, par exemple, avec les classes \clsind{Boolean}, \clsind{Character}, \clsind{SmallInteger}, \clsind{Symbol} et \clsind{UndefinedObject}.

\cmind{Object}{copyTwoLevel} est utilis\'ee quand une simple copie superficielle ne suffit pas:
%does the obvious thing when a simple shallow copy does not suffice:

\begin{code}{@TEST | a1 a2 |}
a1 := { { 'harry' } } .
a2 := a1 copyTwoLevel.
(a1 at: 1) at: 1 put: 'sally'.
a1 --> #(#('sally'))
a2 --> #(#('harry'))    "!\'etat compl\`etement ind\'ependant!"
\end{code}

\cmind{Object}{deepCopy} effectue une copie profonde et arbitraire d'un objet.

\begin{code}{@TEST | a1 a2 |}
a1 := { { { 'harry' } } } .
a2 := a1 deepCopy.
(a1 at: 1) at: 1 put: 'sally'.
a1 --> #(#('sally'))
a2 --> #(#(#('harry')))
\end{code}

Le probl\`eme avec \ct{deepCopy} est qu'elle ne se termine pas si elle est appliqu\'ee \`a une structure mutuellement r\'ecursive:

\begin{code}{NB: CANNOT TEST}
a1 := { 'harry' }.
a2 := { a1 }.
a1 at: 1 put: a2.
a1 deepCopy --> !\emph{... ne se termine jamais}!
\end{code}
% NB: Not a test!

M\^eme s'il est possible de surcharger \ct{deepCopy} pour qu'elle fonctionne mieux, \cmind{Object}{copy} offre une meilleure solution:

\begin{method}{Mod\`ele de m\'ethode pour la copie d'objets}
Object>>>copy
    "Answer another instance just like the receiver.
    Subclasses typically override postCopy;
    they typically do not override shallowCopy."
    ^self shallowCopy postCopy
\end{method}

%ajout
Comme le dit le commentaire de la m\'ethode,
vous pouvez surcharger \mthind{Object}{postCopy} pour copier une variable d'instance qui ne devrait pas \^etre partag\'ee. \ct{postCopy} doit toujours ex\'ecuter \ct{super postCopy}.

\on{I looked, but did not find a good example in the system.}

%-----------------------------------------------------------------
\subsection{D\'ebogage}

La m\'ethode la plus importante ici est \mthind{Object}{halt}. Pour
placer un point d'arr\^et dans une m\'ethode, il suffit d'ins\'erer
l'envoi de message \ct{self halt} \`a une certaine position dans le
corps de la m\'ethode.  Quand ce message est envoy\'e, l'ex\'ecution
est interrompue et un \ind{débogueur} s'ouvre \`a cet endroit de votre programme
(voir \charef{env} pour plus de d\'etails sur le d\'ebogueur).

\sd{in another chapter haltIf:, haltOnce, inspectOnce, flagging: isThisEverCalled, }

Un autre message important est \mthind{Object}{assert:}, qui prend un
\ind{bloc} comme argument. Si le bloc renvoie \ct{true}, l'ex\'ecution
se poursuit. Autrement une exception sera lev\'ee. Si  cette exception
n'est pas intercept\'ee, le d\'ebogueur s'ouvrira \`a ce point pendant
l'ex\'ecution. \ct{assert:} est particuli\`erement utile pour la
\emphind{programmation par contrat}. L'utilisation la plus typique
consiste \`a v\'erifier des pr\'e-conditions non triviales pour des
m\'ethodes publiques. \cmind{Stack}{pop} 
%ajout
(\ct{Stack} est la classe des piles)
aurait pu ais\'ement \^etre implement\'ee de la fa\c{c}on suivante
%martial: ajout
(en commentaire de la m\'ethode: ``renvoie le premier \'el\'ement et
l'enl\`eve de la pile''):

\begin{method}{V\'erifier une pr\'e-condition}
Stack>>>pop
    "Return the first element and remove it from the stack."
    self assert: [ self isEmpty not ].
    ^self linkedList removeFirst element
\end{method}

Il ne faut pas confondre \ct{Object>>>assert:} avec \cmind{TestCase}{assert:}, m\'ethode de l'environnement de test SUnit (voir \charef{SUnit}). Alors que la premi\`ere attend un bloc en argument~\footnote{En fait, elle peut prendre n'importe quel argument qui comprend \ct{value}, dont un \ct{Boolean}.}, la deuxi\`eme attend un \clsind{Boolean}. M\^eme si les deux sont utiles pour d\'eboguer, elles ont chacune un but tr\`es diff\'erent.

%-----------------------------------------------------------------
\subsection{Gestion des erreurs}

Ce protocole contient plusieurs m\'ethodes utiles pour signaler les erreurs d'ex\'ecution.

Envoyer \lct{self deprecated: \emph{unCha\^{\i}neExplicative}} indique que la m\'ethode courante ne devrait plus \^etre utilis\'ee si le param\`etre \ct{deprecation} a \'et\'e activ\'e dans le protocole \protind{debug} du navigateur des pr\'ef\'erences (\ind{Preference Browser}).
L'argument \ct{String} devrait proposer une alternative.
\cmindex{Object}{deprecated:}
\index{deprecation} %martial: pas d\'epr\'ecation

% pas de \emph{} entre les ! depuis le changement de ! dans la version francaise
\begin{code}{NB: CANNOT TEST}
1 doIfNotNil: [ :arg | arg printString, ' n''est pas nil' ]
	--> !SmallInteger(Object)>>doIfNotNil: has been deprecated. use ifNotNilDo:!
\end{code}

%martial: IMPORTANT: ajout pour definir le terme 'deprecation' en francais
%ajout deprecated
L'impression via \menu{print it} de la m\'ethode pr\'ec\'edente
r\'epond que l'usage de la m\'ethode \mthind{Object}{doIfNotNil:} a
\'et\'e consid\'er\'e comme d\'esapprouv\'e (en anglais,
\emph{deprecated}; \emph{deprecation} signifiant d\'esapprobation). Il
est dit que nous devons plut\^ot utiliser \mthind{Object}{ifNotNilDo:}.
\seeindex{d\'esapprobation}{deprecation}
%fin de l'ajout deprecated

\ct{doesNotUnderstand:} est envoy\'e \`a chaque fois que la recherche d'un message \'echoue. L'impl\'ementation par d\'efaut, \ie \cmind{Object}{doesNotUnderstand:} d\'eclenchera l'ouverture d'un d\'ebogueur \`a cet endroit. Il peut \^etre utile de surcharger \lct{does\-Not\-Un\-der\-stand:} pour introduire un autre comportement.

\on{Add a chapter ref when we write the chapter on exceptions.}

\cmind{Object}{error} et \cmind{Object}{error:} sont des m\'ethodes g\'en\'eriques qui peuvent \^etre utilis\'ees pour lever des exceptions
(il est g\'en\'eralement pr\'ef\'erable de lever vos propres exceptions, pour que vous puissiez distinguer les erreurs lev\'ees par votre code de celles lev\'ees par les classes du syst\`eme).
\lr{Maybe mention that it is preferred to create your own custom exception class. (p. 208)}

Les m\'ethodes abstraites en \st sont impl\'ement\'ees par convention
avec le corps \lct{self sub\-class\-Res\-pon\-si\-bi\-li\-ty}. Si une
classe abstraite est instanci\'ee par accident, alors l'appel \`a une
m\'ethode abstraite provoquera l'\'evaluation de
\cmind{Object}{subclassResponsibility}.

\begin{method}{Indiquer qu'une m\'ethode est abstraite}
Object>>>subclassResponsibility
    "This message sets up a framework for the behavior of the class' subclasses.
    Announce that the subclass should have implemented this message."
    self error: 'My subclass should have overridden ', thisContext sender selector printString
\end{method}

%martial: traduction de commentaire
%ajout d'une explication pour la methode suivante
Son commentaire dit que ``ce message installe un cadre pour le
comportement des sous-classes de la classe. Il affirme que la sous-classe
devrait avoir impl\'ement\'e ce message''. La phrase-argument de
l'envoi du message d'erreur \ct{error:} vous pr\'evient que la
m\'ethode devra \^etre surcharg\'ee dans une sous-classe concr\`ete.
%fin de l'ajout d'une explication pour la methode suivante

\clsind{Magnitude}, \clsind{Number} et \clsind{Boolean} sont des exemples classiques de  classes \subind{classe}{abstraite}{}s que nous verrons rapidement dans ce chapitre.

\begin{code}{NB: CANNOT TEST}
Number new + 1 --> !\emph{Error: My subclass should have overridden \#+}!
\end{code}

\ct{self shouldNotImplement} est envoy\'ee par convention pour signaler qu'une m\'ethode h\'erit\'ee est inappropri\'ee pour cette sous-classe. C'est g\'en\'eralement le signe que quelque chose ne va pas dans la conception de la hi\'erarchie de classes. \`A cause des limitations de l'h\'eritage simple, malgr\'e tout, il est des fois tr\`es difficile d'\'eviter de telles solutions.
\cmindex{Object}{shouldNotImplement}
%\apl:jenesaispastraduire\index{inheritance!canceling}

Un exemple classique est la m\'ethode \cmind{Collection}{remove:} qui est h\'erit\'ee de \clsind{Dictionary} mais marqu\'ee comme non impl\'ement\'ee (\ct{Dictionary} fournit la m\'ethode \mthind{Dictionary}{removeKey:} \`a la place).

%-----------------------------------------------------------------
\sd{ subsection{Deprecation} }
\sd{to be done}

\on{There already is some text above!  See second paragraph on Error handling.}

%-----------------------------------------------------------------
\subsection{Test}

Les m\'ethodes de \protind{test} n'ont aucun rapport avec SUnit! Une m\'ethode de test vous permet de poser une question sur l'\'etat du receveur et retourne un bool\'een (\clsind{Boolean}).

De nombreuses m\'ethodes de test sont fournies par \ct{Object}. Nous
avons d\'ej\`a vu \mthind{Object}{isComplex}. Il existe \'egalement
\mthind{Object}{isArray}, \mthind{Object}{isBoolean},
\mthind{Object}{isBlock}, \mthind{Object}{isCollection}, parmi
d'autres. G\'en\'eralement ces m\'ethodes sont \`a \'eviter car
demander sa classe \`a un objet est une forme de violation de
l'encapsulation. Au lieu de tester la classe d'un objet, nous devrions
simplement envoyer un message et laisser l'objet d\'ecider de sa propre r\'eaction.

Cependant certaines de ces m\'ethodes de test sont ind\'eniablement utiles. Les plus utiles sont probablement \cmind{ProtoObject}{isNil} et \cmind{Object}{notNil} (bien que le patron de conception \patind{Null Object}\cite{Wool98a} permet d'\'eviter le besoin de ces m\'ethodes \'egalement).

% \footnote{However the \emph{Null Object} design pattern can obviate the need for even these methods. See, Bobby Woolf, ``Null Object,'' Pattern Languages of Program Design 3, Robert Martin, Dirk Riehle and Frank Buschmann (Eds.), pp. 5-18, Addison Wesley, 1998.}.

%-----------------------------------------------------------------
\subsection{Initialisation}
%Initialize release
\mthind{ProtoObject}{initialize} est une m\'ethode-cl\'e qui ne se
trouve pas dans \ct{Object} mais dans \ct{ProtoObject}.
%martial: ajout d'une traduction de commentaire
Comme le texte de commentaire de la m\'ethode l'indique, vos sous-classes
devront red\'efinir cette m\'ethode pour faire des initialisations
dans la phase de cr\'eation d'instance.

%FIXME ``as an empty hook method``
\begin{method}{La m\'ethode g\'en\'erique \lct{initialize}}
ProtoObject>>>initialize
   "Subclasses should redefine this method to perform initializations on instance creation"
\end{method}

\arelire{Ceci est important parce que, dans \pharo, 
 la m\'ethode \mthind{Behavior}{new}, d\'efinie pour chaque classe du
syst\`eme, envoie \ct{initialize} aux instances nouvellement
cr\'e\'ees.}

\begin{method}{Mod\`ele pour la m\'ethode de classe \lct{new}. Le commentaire dit: ``R\'epond une nouvelle instance initialis\'ee du receveur (qui est une classe) sans aucune variables index\'ees. \'Echoue si la classe est index\'ee''}
Behavior>>>new
    "Answer a new initialized instance of the receiver (which is a class) with no indexable variables. Fail if the class is indexable."
    ^ self basicNew initialize
\end{method}
\cmindex{Behavior}{new}

Ceci signifie qu'en surchargeant simplement la méthode \subind{méthode}{générique} \ct{initialize}, les nouvelles instances de votre classe seront automatiquement initialis\'ees. La m\'ethode \ct{initialize} devrait normalement ex\'ecuter \ct{super initialize} pour \'etablir les \subind{classe}{invariant}{}s de la classe pour toutes les variables d'instance h\'erit\'ees.
Notons que ceci n'est \emph{pas} le comportement standard dans les autres \st{}s.

%=================================================================
\section{Les nombres}
\seclabel{Number}
Il faut remarquer que les nombres en \st ne sont pas des donn\'ees primitives mais de vrais objets. Bien s\^ur les nombres sont impl\'ement\'es efficacement dans la machine virtuelle, mais la hi\'erarchie de la classe \clsindmain{Number} est aussi accessible et extensible que n'importe quelle autre portion de la hi\'erarchie de classe de \st.

\begin{figure}[ht]
\centerline {\includegraphics[width=8cm]{NumberHierarchy}}
\caption{La hi\'erarchie de la classe Number.\figlabel{numbers}}
\end{figure}

On trouve les nombres dans la cat\'egorie \scatind{Kernel-Numbers}. La racine abstraite de cette cat\'egorie est \clsind{Magnitude}, qui repr\'esente toutes les sortes de classes qui supportent les op\'erateurs de comparaison. La classe \ct{Number} ajoute plusieurs op\'erateurs arithm\'etiques et autres, principalement des m\'ethodes abstraites. \clsind{Float} et \clsind{Fraction} repr\'esentent, respectivement, les nombres \`a virgule flottante et les valeurs fractionnaires.  \clsind{Integer} est \'egalement une classe abstraite et contient trois sous-classes \clsind{SmallInteger}, \clsind{LargePositiveInteger} et \clsind{LargeNegativeInteger}. Le plus souvent les utilisateurs n'ont pas \`a conna\^itre la diff\'erence entre les trois classes d'entiers, car les valeurs sont automatiquement converties si besoin est.

%-----------------------------------------------------------------
\subsection{Magnitude}

\clsindmain{Magnitude} n'est pas seulement la classe parente des classes de nombres, mais \'egalement des autres classes supportant les op\'erateurs de comparaison, comme \clsind{Character}, \clsind{Duration} et \clsind{Timespan} (les nombres complexes (classe \clsind{Complex}) ne sont pas comparables et n'h\'eritent pas de la classe \clsind{Number}).

Les m\'ethodes \mthind{Magnitude}{<} et \mthind{Magnitude}{=} sont abstraites. Les autres op\'erateurs sont d\'efinis de mani\`ere g\'en\'erique. Par exemple:

%martial: je n'ai pas traduit les commentaires ici (ils n'apprennent rien)
\begin{method}{M\'ethodes de comparaison abstraites}
Magnitude>>> < aMagnitude 
    "Answer whether the receiver is less than the argument."
    ^self subclassResponsibility

Magnitude>>> > aMagnitude 
    "Answer whether the receiver is greater than the argument."
    ^aMagnitude < self
\end{method}
\cmindex{Magnitude}{>}

%-----------------------------------------------------------------
\subsection{Number}

De la m\^eme mani\`ere, la classe \clsindmain{Number} d\'efinit \mthind{Number}{+}, \mthind{Number}{-}, \mthind{Number}{*} et \mthind{Number}{/} comme des m\'ethodes abstraites, mais tous les autres op\'erateurs arithm\'etiques sont d\'efinis de mani\`ere g\'en\'erique.

Tous les nombres supportent plusieurs op\'erateurs de  \emph{conversion}, comme \mthind{Number}{asFloat} et \mthind{Number}{asInteger}. Il existe \'egalement des \emphind{constructeurs} num\'eriques,
%\emphind{shortcut constructor methods}
comme \mthind{Number}{i}, qui convertit une instance de \ct{Number} en
une instance de \clsind{Complex} avec une partie r\'eelle nulle, ainsi
que d'autres m\'ethodes qui g\'en\`erent des dur\'ees, instances de
\clsind{Duration}, comme \mthind{Number}{hour}, \mthind{Number}{day}
et \mthind{Number}{week}
%ajout
(respectivement: heure, jour et semaine).

Les nombres supportent directement les \emph{fonctions
  math\'ematiques} telles que \mthind{Number}{sin},
\mthind{Number}{log}, \mthind{Number}{raiseTo:} 
%ajout
(puissance),
\mthind{Number}{squared}
%ajout
(carr\'e),
\mthind{Number}{sqrt}
%ajout
(racine carr\'ee).

\cmind{Number}{printOn:} utilise la m\'ethode abstraite
\ct{Number>>>printOn:base:} (la base par d\'efaut est 10).

Les m\'ethodes de test comprennent entre autres \mthind{Number}{even}
%ajout
(pair), 
\mthind{Number}{odd}
%ajout
(impair), 
\mthind{Number}{positive}
%ajout
(positif)
 et \mthind{Number}{negative}
%ajout
(n\'egatif).
Logiquement, \ct{Number} surcharge \lct{is\-Num\-ber} 
%ajout
(test d'appartenance \`a la hi\'erarchie de la classe des nombres).
Plus int\'eressant, \mthind{Number}{isInfinite} 
%ajout
(test d'infinit\'e)
renvoie \ct{false}.

Les m\'ethodes de \emph{troncature} incluent entre autres,
\mthind{Number}{floor}
%ajout
(arrondi \`a l'entier inf\'erieur),
\mthind{Number}{ceiling}
%ajout
(arrondi \`a l'entier sup\'erieur), 
\mthind{Number}{integerPart}
%ajout
(partie enti\`ere), 
\mthind{Number}{fractionPart}
%ajout
(partie apr\`es la virgule).

\begin{code}{@TEST}
1 + 2.5     --> 3.5             "Addition de deux nombres"
3.4 * 5      --> 17.0           "Multiplication de deux nombres"
8 / 2         --> 4                 "Division de deux nombres"
10 - 8.3   --> 1.7              "Soustraction de deux nombres"
12 = 11    --> false           !"\'Egalit\'e entre deux nombres"!
12 ~= 11 --> true            !"Teste si deux nombres sont diff\'erents"!
12 > 9      --> true            "Plus grand que"
12 >= 10  --> true            !"Plus grand ou \'egal \`a"!
12 < 10    --> false           "Plus petit que"
100@10   --> 100@10    !"Cr\'eation d'un point"!
\end{code}
\on{Should check how tabbing works in the listings package ...}

L'exemple suivant fonctionne \'etonnamment bien en \st:
\begin{code}{@TEST}
1000 factorial / 999 factorial --> 1000
\end{code}
Notons que \ct{1000 factorial} est r\'eellement calcul\'ee alors que dans beaucoup d'autres langages il peut \^etre difficile de le faire. Ceci est un excellent exemple de conversion automatique et d'une gestion exacte des nombres.
\cmindex{Integer}{factorial}

\dothis{Essayez d'afficher le r\'esultat de \ct{1000 factorial}. Il faut plus de temps pour l'afficher que pour le calculer!}

%-----------------------------------------------------------------
\subsection{Float}

\clsindmain{Float} impl\'emente les m\'ethodes de \ct{Number} abstraites pour les nombres \`a virgule flottante.

Plus int\'eressant, \ct{Float class} (\ie le c\^ot\'e classe de
\ct{Float}) contient des m\'ethodes pour renvoyer les
\emph{constantes}: \mthind{Float class}{e}, \mthind{Float
  class}{infinity}
%ajout
(infini), 
\mthind{Float class}{nan} 
%martial: ajout (definition wikipedia, pas besoin de plus) 
(acronyme de \emph{Not A Number} \cad ``n'est pas un nombre'':
r\'esultat d'un calcul num\'erique ind\'etermin\'e)
et \mthind{Float class}{pi}.

\begin{code}{@TEST}
Float pi                      --> 3.141592653589793
Float infinity               --> Infinity
Float infinity isInfinite --> true
\end{code}

%-----------------------------------------------------------------
\subsection{Fraction}

Les \clsind{fractions} sont repr\'esent\'ees par des variables d'instance pour le num\'erateur et le d\'enominateur, qui devraient \^etre des entiers. Les \ct{fractions} sont normalement cr\'e\'ees par division d'entiers (plut\^ot qu'en utilisant le constructeur \cmind{Fraction}{numerator:denominator:}):

\begin{code}{@TEST}
6/8             --> (3/4)
(6/8) class --> Fraction
\end{code}

Multiplier une fraction par un entier ou par une autre fraction peut renvoyer un entier:

\begin{code}{@TEST}
6/8 * 4 --> 3
\end{code}

\lr{Maybe mention to avoid fractions in results that one of the operands has to be a float, e.g. 6.0 / 8 or 6 asFloat / 8. (p. 213)}

%-----------------------------------------------------------------
\subsection{Integer}

\clsindmain{Integer} est le parent abstrait de trois impl\'ementations
concr\`etes d'entiers. En plus de fournir une impl\'ementation
concr\`ete de beaucoup de m\'ethodes abstraites de la classe
\ct{Number}, il ajoute \'egalement quelques m\'ethodes sp\'ecifiques
aux entiers, telles que \mthind{Integer}{factorial}
%ajout
(fractionnelle),
\mthind{Integer}{atRandom}
%ajout
(nombre al\'eatoire entre 1 et le receveur),
\mthind{Integer}{isPrime}
%ajout
(test de nombre premier), 
\mthind{Integer}{gcd:} 
%ajout
(le plus grand d\'enominateur commun)
et beaucoup d'autres.

La classe \clsindmain{SmallInteger} est particuli\`ere dans le sens que ses instances sont repr\'esent\'ees de mani\`ere compacte --- au lieu d'\^etre stock\'ees comme r\'ef\'erence, une instance de \ct{SmallInteger} est directement repr\'esent\'ee en utilisant les bits qui seraient normalement utilis\'es pour contenir la r\'ef\'erence.  Le premier bit de la r\'ef\'erence \`a un objet indique si l'objet est une instance de SmallInteger ou non.

Les m\'ethodes de classe \mthind{SmallInteger}{minVal} et \mthind{SmallInteger}{maxVal} nous donne la plage de valeurs d'une instance de \ct{SmallInteger}:

\begin{code}{@TEST}
SmallInteger maxVal = ((2 raisedTo: 30) - 1)      --> true
SmallInteger minVal = (2 raisedTo: 30) negated --> true
\end{code}

Quand un \ct{SmallInteger} d\'epasse cette plage de valeurs, il est automatiquement converti en une instance de \clsind{LargePositiveInteger} ou de \clsind{LargeNegativeInteger}, selon le besoin:

\begin{code}{@TEST}
(SmallInteger maxVal + 1) class --> LargePositiveInteger
(SmallInteger minVal - 1) class  --> LargeNegativeInteger
\end{code}

Les grands entiers sont de la m\^eme mani\`ere convertis en petits entiers quand il le faut.

Comme dans la plupart des langages de programmation, les entiers peuvent \^etre utiles pour sp\'ecifier une it\'eration.  Il existe une m\'ethode d\'edi\'ee \mthind{Integer}{timesRepeat:} pour l'\'evaluation r\'ep\'etitive d'un bloc.
Nous avons d\'ej\`a vu des exemples similaires dans le chapitre \charef{syntax}:
\begin{code}{@TEST | n |}
n := 2.
3 timesRepeat: [ n := n*n ].
n --> 256
\end{code}

%=================================================================
\section{Les caract\`eres}

\clsindmain{Character} est d\'efinie dans la cat\'egorie \scatind{Collections-Strings} comme une sous-classe de \clsind{Magnitude}. Les caract\`eres imprimables sont repr\'esent\'es en \pharo par \lct{\$$\langle$\emph{caract\`ere}$\rangle$}.  Par exemple:

\begin{code}{@TEST}
$a < $b --> true
\end{code}

Les caract\`eres non imprimables sont g\'en\'er\'es par diff\'erentes
m\'ethodes de classe. \mbox{\cmind{Character class}{value:}} prend la
valeur enti\`ere Unicode (ou ASCII) comme argument et renvoie le
caract\`ere correspondant. Le protocole \protind{accessing untypeable
  characters} contient un certain nombre de constructeurs utiles tels
que \mthind{Character class}{backspace} 
%ajout
(retour arri\`ere), 
\mthind{Character class}{cr} 
%ajout
(retour-chariot),
\mthind{Character class}{escape}
%ajout
(\'echappement),
\mthind{Character class}{euro}
%ajout
(signe \euro),
\mthind{Character class}{space}
%ajout
(espace), 
\mthind{Character class}{tab}
%ajout
(tabulation), 
parmi d'autres.

\begin{code}{@TEST}
Character space = (Character value: Character space asciiValue) --> true
\end{code}

La m\'ethode \mthind{Character}{printOn:} est assez adroite pour
savoir laquelle des trois mani\`eres utiliser pour g\'en\'erer les
caract\`eres de la fa\c{c}on la plus appropri\'ee:

\begin{code}{@TEST}
Character value: 1   --> Character home
Character value: 2   --> Character value: 2
Character value: 32  --> Character space
Character value: 97  --> $a
\end{code}\ignoredollar$ % CHANGE

Il existe plusieurs m\'ethodes de \emph{test} utiles:
\mthind{Character}{isAlphaNumeric}
%ajout
(si alphanum\'erique),
\mthind{Character}{isCharacter}
%ajout
(si caract\`ere),
\mthind{Character}{isDigit}
%ajout
(si num\'erique),
\mthind{Character}{isLowercase},
%ajout
(si minuscule),
\mthind{Character}{isVowel}
%ajout
(si voyelle non-accentu\'ee, voir page~\pageref{def:isVowel}), 
parmi d'autres.

Pour convertir un caract\`ere en une cha\^{\i}ne de caract\`eres contenant uniquement ce caract\`ere, il faut lui envoyer le message \mthind{Character}{asString}.  Dans ce cas \ct{asString} et \mthind{Character}{printString} donnent des r\'esultats diff\'erents:

\begin{code}{@TEST}
$a asString    --> 'a'
$a                  --> $a
$a printString --> '$a'
\end{code}\ignoredollar$

Chaque caract\`ere ASCII est une instance unique, stock\'ee dans la variable de classe \cvind{CharacterTable}:

\begin{code}{@TEST}
(Character value: 97) == $a --> true
\end{code}\ignoredollar$

Cependant, les caract\`eres au del\`a de la plage 0 \`a 255 ne sont pas uniques: 
\begin{code}{@TEST}
Character characterTable size                               --> 256
(Character value: 500) == (Character value: 500) --> false
\end{code}

%=================================================================
\section{Les cha\^{\i}nes de caract\`eres}

La classe \clsindmain{String} est \'egalement d\'efinie dans la cat\'egorie \scatind{Collections-Strings}.  Une cha\^{\i}ne de caract\`eres est une collection index\'ee contenant uniquement des caract\`eres.

\begin{figure}[ht]
\ifluluelse
	{\centerline {\includegraphics[width=0.4\textwidth]{StringHierarchy}}}
\caption{La hi\'erarchie de \ct{String}.\figlabel{strings}}
\end{figure}

En fait, \ct{String} est une classe abstraite et les cha\^{\i}nes de caract\`eres de \pharo sont en r\'ealit\'e des instances de la classe concr\`ete \clsindmain{ByteString}.

\begin{code}{@TEST}
'Bonjour Squeak' class --> ByteString
\end{code}

Une autre sous-classe importante de \ct{String} est
\clsindmain{Symbol}.  La diff\'erence fondamentale est qu'il n'y a
toujours qu'une instance unique de \ct{Symbol} pour une valeur
donn\'ee  (ceci est quelques fois appel\'e ``la propri\'et\'e de
l'instance unique'').  \`A l'oppos\'e, deux cha\^{\i}nes construites
s\'epar\'ement  et contenant la m\^eme s\'equence de caract\`eres
seront souvent des objets diff\'erents.

\begin{code}{@TEST}
'Sal','ut' == 'Salut' --> false
\end{code}

\begin{code}{@TEST}
('Sal','ut') asSymbol == #Salut --> true
\end{code}

%martial: j'ai traduit 'mutable' en 'modifiable'; on peut peut-etre
%juste garder 'mutable' pour les collections parce qu'on peut le
%rencontrer souvent dans le code
\noindent
Une autre diff\'erence importante est que \ct{String} est modifiable,
alors que \ct{Symbol} ne l'est pas.

\begin{code}{@TEST}
'hello' at: 2 put: $u; yourself --> 'hullo'
\end{code}\ignoredollar$

\begin{code}{NB: CANNOT TEST}
#hello at: 2 put: $u --> !erreur!
\end{code}\ignoredollar$

Il est facile d'oublier que, puisque les cha\^{\i}nes de caract\`eres
sont des collections, elles comprennent les m\^emes messages que les
autres collections
%ajout: parce qu'on n'a pas encore bien vu les collections
(ici, la m\'ethode \ct{indexOf:} de \ct{Collections} donne la position
du premier caract\`ere rencontr\'e): 

\begin{code}{@TEST}
#hello indexOf: $o --> 5
\end{code}\ignoredollar$

Bien que \ct{String} n'h\'erite pas de \clsind{Magnitude}, la classe supporte les m\'ethodes de  \protind{comparaison}, \ct{<}, \ct{=}, etc.  De plus, \cmind{String}{match:} est utile pour les recherches simples d'expressions r\'eguli\`eres:

\begin{code}{@TEST}
'*or*' match: 'zorro' --> true
\end{code}

\arelire{Si vous avez besoin d'un meilleur support pour les expressions
r\'eguli\`eres, vous pouvez jeter un \oe il sur le paquetage
\pkgind{Regex} de Vassili Bykov.} % CHANGE
\index{Bykov, Vassili}
\index{paquetage!expressions régulières}

Les cha\^{\i}nes de caract\`eres supportent un grand nombre de
m\'ethodes de conversion. Beaucoup sont des constructeurs-raccourci
\index{constructeur-raccourci}%\ind{shortcut constructor methods}
pour d'autres classes, comme \mthind{String}{asDate} 
%ajout
(pour cr\'eer une date)
ou \mthind{String}{asFileName}
%ajout
(pour cr\'eer un nom de fichier).
Il existe \'egalement un certain nombre de m\'ethodes utiles pour
transformer une cha\^{\i}ne de caract\`eres en une autre, comme
\mthind{String}{capitalized} 
%ajout
(pour capitaliser) 
et \mthind{String}{translateToLowercase}
%ajout
(pour mettre en minuscule).
%martial: peut-etre a redefinir \ind{shortcut constructor methods}
\seeindex{shortcut constructor method}{constructeur-raccourci}

Pour plus d'informations sur les cha\^{\i}nes de caract\`eres et les
collections, rendez-vous au chapitre \ref{cha:collections}.

\on{There is more material we could use here:
\url{http://www.dmu.com/crb/crb7.html}.}

%=================================================================
\section{Les bool\'eens}

La classe \clsindmain{Boolean} offre un aper\c{c}u fascinant de la
mani\`ere dont \st est construit autour de la biblioth\`eque de classes. 
%how much of the \st language has been pushed into the class library. 
\ct{Boolean} est la super-classe \subind{classe}{abstraite} des classes
singletons (de patron \patind{Singleton}): \clsindmain{True} et \clsindmain{False}.

\begin{figure}[ht]
	{\centerline {\includegraphics[width=0.5\textwidth]{BooleanHierarchy}}}
\caption{La hi\'erarchie des bool\'eens.\figlabel{booleans}}
\end{figure}

La plupart des comportements des bool\'eens peuvent \^etre compris en
regardant la m\'ethode \mthind{Boolean}{ifTrue:ifFalse:} (en
fran\c{c}ais, \codefrench{si vrai: si faux:}), qui prend deux blocs
comme arguments.

\begin{code}{@TEST}
(4 factorial > 20) ifTrue: [ 'plus grand' ] ifFalse: [ 'plus petit' ] --> 'plus grand'
\end{code}

La m\'ethode est abstraite dans \ct{Boolean}.
Les impl\'ementations dans les sous-classes concr\`etes sont toutes les deux triviales:

\begin{method}{Impl\'ementations de \lct{ifTrue:ifFalse:}}
True>>>ifTrue: trueAlternativeBlock ifFalse: falseAlternativeBlock 
    ^trueAlternativeBlock value

False>>>ifTrue: trueAlternativeBlock ifFalse: falseAlternativeBlock 
    ^falseAlternativeBlock value
\end{method}
\cmindex{True}{ifTrue:}
\cmindex{False}{ifTrue:}

En fait, ceci est l'essence m\^eme de la programmation orient\'ee
objet (POO): quand un message est envoy\'e \`a un objet, l'objet
lui-m\^eme d\'etermine quelle m\'ethode sera utilis\'ee pour
r\'epondre. Dans ce cas, une instance de \ct{True}  \'evalue
simplement l'alternative \emph{vraie}, alors qu'une instance de
\ct{False} evalue l'alternative \emph{fausse}. Toutes les m\'ethodes abstraites de la classe \ct{Boolean} sont impl\'ement\'ees de cette mani\`ere pour \ct{True} et \ct{False}. Par exemple:

\begin{method}{Impl\'ementer la n\'egation}
True>>>not
    "Negation--answer false since the receiver is true."
    ^false
\end{method}
\cmindex{True}{not}

%ajout
Le commentaire de la m\'ethode \ct{not} (n\'egation logique) nous informe que
la r\'eponse est toujours fausse (\ct{false}) puisque le receveur est
vrai (\ct{true}, instance de \ct{True}).

La classe \ct{Boolean} offre plusieurs m\'ethodes utiles, comme \mthind{Boolean}{ifTrue:}, \mthind{Boolean}{ifFalse:}, \mthind{Boolean}{ifFalse:ifTrue}. Vous avez \'egalement le choix entre les conjonctions et disjonctions optimis\'ees ou paresseuses.

\begin{code}{@TEST}
(1>2) & (3<4)              --> false    !"doit \'evaluer les deux cot\'es"!
(1>2) and: [ 3<4 ]        --> false    !"\'evalue seulement le receveur"!
(1>2) and: [ (1/0) > 0 ] --> false    !"le bloc pass\'e en argument n'est jamais \'evalu\'e, ainsi, pas d'exception"!
\end{code}

Dans le premier exemple, les deux sous-expressions bool\'eennes sont
\'evalu\'ees, puisque \mthind{Boolean}{&} 
%ajout
(\emph{et} logique) 
prend un argument bool\'een.
Dans le second et troisi\`eme exemple, uniquement la premi\`ere est
\'evalu\'ee, car \mthind{Boolean}{and:} 
%ajout
(\emph{et} non-\'evaluant) 
attend un bloc comme argument. Le  bloc est \'evalu\'e uniquement si le premier argument vaut \pvind{true}.

\dothis{Essayez d'imaginer comment \ct{and:} et \ct{or:} 
%ajout
(\emph{ou} non-\'evaluant)
sont impl\'ement\'es.
V\'erifiez les impl\'ementations dans \ct{Boolean}, \ct{True} et \ct{False}.}

%=================================================================
\section{R\'esum\'e du chapitre}
Nous avons vu que:

\begin{itemize}
%  \item Send \ct{yourself} to get back the receiver at the end of a cascade.

  \item si vous surchargez \ct{=} alors vous devez \'egalement
    surcharger la m\'ethode de hachage, \ct{hash};

  \item surchargez \ct{postCopy} pour impl\'ementer correctement la copie de vos objets;

  \item envoyez \ct{self halt} pour cr\'eer un point d'arr\^et;

  \item renvoyez \ct{self subclassResponsibility} pour faire une m\'ethode abstraite;

  \item pour donner la repr\'esentation en cha\^{\i}ne de caract\`eres d'un objet \ct{String}, vous devez surcharger \ct{printOn:};

  \item surchargez la m\'ethode g\'en\'erique \ct{initialize} pour instancier correctement vos objets;


% proposition de Rene:
%  \item Les m\'ethodes de la classe \ct{Number} convertissent automatiquement entre
%  flottants, fractions et entiers.
  \item les m\'ethodes de la classe \ct{Number} assurent, si
    n\'ecessaire, les conversions automatiques entre flottants, fractions et entiers;

  \item les fractions repr\'esentent vraiment des nombres r\'eels plut\^ot que des nombres \`a virgule flottante;

  \item les caract\`eres sont des instances uniques;

  \item les cha\^{\i}nes de caract\`eres sont modifiables mais les symboles ne le sont pas;
  cependant faites attention \`a ne pas modifier les cha\^{\i}nes de caract\`eres litt\'erales!

  \item ces symboles sont uniques mais les cha\^{\i}nes de caract\`eres ne le sont pas;

  \item les cha\^{\i}nes de caract\`eres et les symboles sont des collections et donc, supportent les m\'ethodes usuelles de la classe \ct{Collection}.

\end{itemize}

%=============================================================
\ifx\wholebook\relax\else
   \bibliographystyle{jurabib}
   \nobibliography{scg}
   \end{document}
\fi
%=============================================================

%-----------------------------------------------------------------

%%% Local Variables:
%%% coding: utf-8
%%% mode: latex
%%% TeX-master: t
%%% TeX-PDF-mode: t
%%% ispell-local-dictionary: "english"
%%% End:

%:Collections
% $Author: oscar $
% $Translation: martial $
% $Date: Fri Oct 12 13:57:53 CEST 2007 $
% $Revision: 12789 $
% translated by Martial.Boniou@ifrance.com start: (Fri, 12 Oct 2007)
% relecture : Rene Mages (9 jan 2008) de la version #14859
% adaptation pour PBE: martial Tue Sep  8 19:56:33 CEST 2009 from
% - oscar - 28661 - $Date: 2009-08-28 11:31:14 +0200 (Fri, 28 Aug
% 2009) $
% sync avec la revision: 29170
\ifx\wholebook\relax\else
% --------------------------------------------
% Lulu:
	\documentclass[a4paper,10pt,twoside]{book}
	\usepackage[
		papersize={6.13in,9.21in},
		hmargin={.75in,.75in},
		vmargin={.75in,1in},
		ignoreheadfoot
	]{geometry}
	\input{../common.tex}
	\pagestyle{headings}
	\setboolean{lulu}{true}
% --------------------------------------------
% A4:
%	\documentclass[a4paper,11pt,twoside]{book}
%	\input{../common.tex}
%	\usepackage{a4wide}
% --------------------------------------------
    \graphicspath{{figures/} {../figures/}}
	\begin{document}
	%\renewcommand{\nnbb}[2]{} % Disable editorial comments
	\sloppy
\fi
%=================================================================
\chapter{Les collections}
\chalabel{collections}
\ew{Stack is a popular construction. How does it fit in the collection hierarchy?}
% \ab{ The material here is based on a section that Andrew wrote for the \st Collections refactoring paper. It's not necessarily all appropriate for the book, but much of it is, I think, useful.}
%\on{recycled some material from \url{https://www.iam.unibe.ch/scg/svn_repos/Lectures/ST-H07/03StandardClasses.ppt}}
%\sd{Did a first pass: I should have a look at Xavier Briffault's book and at the material mentionned by andrew + lalonde book?}
%=============================================================
\section{Introduction}

% Rene propose un pluriel sur collections
% Martial s'était trompé en fait  (j'ai fait pareil pour les Streams)
% voir
% http://csl.ensm-douai.fr/noury/uploads/15/tourDeSmalltalkEn100Diapos.pdf
Les classes de collections forment un groupe de sous-classes de \clsindmain{Collection} et de \clsindmain{Stream} (pour flux de donn\'ees) faiblement coupl\'ees destin\'e \`a un usage g\'en\'erique.
Ce groupe de classes mentionn\'e dans la bible de \st nomm\'ee ``\ind{Blue Book}''~\cite{Gold83a} (le fameux livre bleu) comprend 17 sous-classes de \ct{Collection} 
et 9 issues de la classe \ct{Stream}. Formant un total de 28 classes, elles
ont d\'ej\`a \'et\'e remodel\'ees maintes fois avant la sortie du syst\`eme \st-80. Ce groupe de classes est souvent consid\'er\'e comme un exemple pragmatique
de mod\'elisation orient\'ee objet.

\arelire{Dans \pharo, les classes abstraites \ct{Collection} et \ct{Stream}
disposent respectivement de 101 et de 50 sous-classes mais beaucoup d'entre elles
(comme \mbox{\clsind{Bitmap},} \clsind{FileStream} et \clsind{CompiledMethod})
sont des classes d'usage sp\'ecifique d\'efinies pour \^etre employ\'ees
dans d'autres parties du syst\`eme ou dans des applications et ne sont par cons\'equent pas organis\'ees dans la cat\'egorie ``Collections''. 
Dans ce chapitre, nous r\'eunirons \ct{Collection} et ses 47 sous-classes
\emph{aussi} pr\'esentes dans les cat\'egories-syst\`eme de la forme \scat{Collections-*}
sous le terme de ``hi\'erarchie de Collections''
%``Collection Hierarchy'' 
et \ct{Stream} et ses 9 sous-classes de la cat\'egorie \scat{Collections-Streams} sous celui de ``hi\'erarchie de Streams''.
Ces 56 classes r\'epondent \`a 982 messages d\'efinissant un total de
1609 m\'ethodes!}

% % See test case in PBE-Collections for statistics

%c := (Collection allSubclasses select: [:each | each category beginsWith: 'Collections']).
%s := (Stream allSubclasses select: [:each | each category beginsWith: 'Collections-Streams']).
%m := (c,s) gather: #methods.
%Transcript
%	show: 'Collection subclasses: ', Collection allSubclasses size printString; cr;
%	show: 'Stream subclasses: ', Stream allSubclasses size printString; cr;
%	show: 'Collection package subclasses: ', c size printString; cr;
%	show: 'Collection package subclasses: ', c size printString; cr;
%	show: 'Stream package subclasses: ', s size printString; cr;
%	show: 'Total messages: ', m size printString; cr;
%	show: 'Total messages: ', (m collect: #selector) asSet size printString; cr

% \begin{figure}
% \begin{center}
% \ifluluelse
% 	{\fbox{\includegraphics[height=0.9\textheight]{CollectionHierarchyList}}}
% 	{\fbox{\includegraphics[width=3in]{CollectionHierarchyList}}}
% \caption{Les classes de collections dans \squeak. L'indentation indique la hi\'erarchie: Les classes \textit{\textsf{en italique}} sont abstraites.}
% \label{fig:CollClassesList}
% \end{center}
% \end{figure}

Dans ce chapitre, nous nous attarderons principalement sur le sous-ensemble
de classes de collections montr\'e sur 
\figref{CollClassesTree}.
Les flux de donn\'ees ou \emph{streams} seront abord\'es séparément dans 
\charef{streams}.

% \sd{should put stream related text in Stream chapter}.

% Note that the stream classes suffer from aging problems since they contain duplicated code and other code smells.
% ON: We should avoid statements like this!

%:FIGURE -- Key collection classes
\begin{figure}
\begin{center}
\ifluluelse
	{\includegraphics[width=\textwidth]{CollectionHierarchy}}
	{\includegraphics[width=0.8\textwidth]{CollectionHierarchy}}
\caption{Certaines des classes majeures de collection de \pharo.}
\figlabel{CollClassesTree}
\end{center}
\end{figure}

%=========================================================
\section{Des collections très variées}
\seclabel{varieties}

Pour faire bon usage des classes de collections, le lecteur devra conna\^{\i}tre
au moins superficiellement l'immense vari\'et\'e de collections que celles-ci
impl\'ementent ainsi que leurs similitudes et leurs diff\'erences.

Programmer avec des collections plut\^ot qu'avec des \'el\'ements 
ind\'ependants est une \'etape importante pour accro\^{\i}tre le degr\'e
d'abstraction d'un programme.
La fonction \ct{map} dans le langage \ind{Lisp} est un exemple
primaire de cette technique de programmation: cette fonction
applique une fonction entr\'ee en argument \`a tout \'el\'ement d'une
liste et retourne une nouvelle liste contenant le r\'esultat.
\st-80 a adopt\'e la programmation bas\'ee sur les collections
comme pr\'ecepte central.
Les langages modernes de programmation fonctionnelle tels que \ind{ML}
% martial: j'aurais eu envie de mettre le dynamique erlang, mon favori
et \ind{Haskell} ont suivi l'orientation de \st. 
%\st's lead.

Pourquoi est-ce une si bonne id\'ee?
Partons du principe que nous avons une structure de donn\'ees contenant
une collection d'enregistrements d'\'etudiants appel\'e \ct{students}
(pour \'etudiants, en anglais) et que nous voulons
accomplir une certaine action sur tous les \'etudiants remplissant un
certain crit\`ere.
Les programmeurs \'eduqu\'es aux langages imp\'eratifs vont se retrouver
imm\'ediatement à écrire une boucle. Mais le développeur en \st \'ecrira:
\begin{code}{}
students select: [ :each | each gpa < threshold ]
\end{code}
\noindent
ce qui donnera une nouvelle collection contenant pr\'ecisement les
\'el\'ements de \ct{students} (\'etudiants) pour lesquels la
fonction entre crochets renvoie une r\'eponse positive \cad \ct{true}~\footnote{L'expression entre crochets (brackets en anglais) peut \^etre vue comme une 
expression $\lambda$ d\'efinissant une fonction anonyme $\lambda x. x~{\sf gpa} < {\sf threshold}$.
\index{expression lambda}}.
Le code \st a la simplicit\'e et l'\'el\'egance des langages d\'edi\'es ou \emph{Domain-Specific Language} souvent abr\'eg\'es en DSL. 

Le message \ct{select:} est compris par \emph{toutes} les collections de \st.
Il n'est pas n\'ecessaire de chercher si la structure de donn\'ees des \'etudiants est un tableau ou une liste cha\^{\i}n\'ee: 
le message \ct{select:} est reconnu par les deux. 
Notez donc que c'est assez diff\'erent de l'usage d'une boucle avec laquelle
nous devons nous interroger pour savoir si \ct{students} est un tableau
ou une liste cha\^{\i}n\'ee avant que cette boucle puisse \^etre configur\'ee.

En \st, lorsque quelqu'un parle d'une collection sans \^etre plus pr\'ecis
sur le type de la collection, il mentionne un objet qui supporte des protocoles
bien d\'efinis pour tester l'appartenance et \'enum\'erer les \'el\'ements.
%well-defined protocols for testing membership and enumerating the elements. 
\emph{Toutes} les collections acceptent les messages 
de la cat\'egorie des tests nomm\'ee \protind{testing} tels que \mbox{\mthind{Collection}{includes:}} (test d'inclusion), \mthind{Collection}{isEmpty} (teste si la collection est vide) 
et \mbox{\mthind{Collection}{occurrencesOf:}} (test d'occurences d'un \'el\'ement). 
\emph{Toutes} les collections comprennent les messages du protocole 
\prot{enumeration} comme 
\mthind{Collection}{do:} (action sur chaque \'el\'ement), 
\mthind{Collection}{select:} (s\'election de certains \'el\'ements), 
\mthind{Collection}{reject:} (rejet \`a l'oppos\'e de \ct{select:}), 
\mthind{Collection}{collect:} (identique \`a la fonction \ct{map} de Lisp),
\mthind{Collection}{detect:ifNone:} (d\'etection tol\'erante \`a l'absence) 
\mthind{Collection}{inject:into:} (accumulation ou op\'eration par r\'eduction 
comme avec une fonction \emph{fold} ou \emph{reduce} dans d'autres langages) et
beaucoup plus encore.
C'est plus l'ubiquit\'e de ce protocole que sa diversit\'e qui le rend
si puissant.

\Figref{protocols} r\'esume les protocoles standards support\'es par la
plupart des classes de la hi\'erarchie de collections. 
Ces m\'ethodes sont d\'efinies, red\'efinies, optimis\'ees ou parfois
m\^eme interdites par les sous-classes de \ct{Collections}.

%:FIGURE -- Standard collection protocols
\begin{figure*}
\begin{center}
\ifluluelse
	{\begin{tabular}{|l|p{8cm}|}}
	{\begin{tabular}{|l|p{12cm}|}}
\hline
{\bf Protocole} & {\bf M\'ethodes}\\
\hline
\protind{accessing}	&	\lct{size}, \lct{capacity}, \lct{at: \emph{anIndex}}, \lct{at: \emph{anIndex} put: \emph{anElement}}	\\
\hline
\protind{testing}	&	\lct{isEmpty}, \lct{includes: \emph{anElement}}, \lct{contains: \emph{aBlock}}, \lct{occurrencesOf: \emph{anElement}}	\\
\hline
\protind{adding}	&	\lct{add: \emph{anElement}}, \lct{addAll: \emph{aCollection}}	\\
\hline
\protind{removing}		&	\lct{remove: \emph{anElement}}, \lct{remove: \emph{anElement} ifAbsent: \emph{aBlock}}, \lct{removeAll: \emph{aCollection}}	\\
\hline
\protind{enumerating}	&	\lct{do: \emph{aBlock}}, \lct{collect: \emph{aBlock}}, \lct{select: \emph{aBlock}}, \lct{ reject: \emph{aBlock}}, \lct{detect: \emph{aBlock}}, \lct{detect: \emph{aBlock} ifNone: aNoneBlock}, \lct{inject: \emph{aValue} into: \emph{aBinaryBlock}}	\\
\hline
\protind{converting}	&	\lct{asBag}, \lct{asSet}, \lct{asOrderedCollection}, \lct{asSortedCollection}, \lct{asArray}, \lct{asSortedCollection: \emph{aBlock}}	\\
\hline
\protind{creation}		&	\lct{with: \emph{anElement}}, \lct{with:with:}, \lct{with:with:with:}, \lct{ with:with:with:with:}, \lct{withAll:  \emph{aCollection}}	\\
\hline
\end{tabular}
\caption{Les protocoles standards de collections\figlabel{protocols}}
\end{center}
\end{figure*}

%\begin{figure*}
%\begin{center}
%\includegraphics[width=\textwidth]{CollectionsBySeq}
%\caption{Collections can be categorized according to whether or not they are sequenceable, \ie whether there are clearly defined first and last elements. All of the sequenceable collections except linked lists can also be indexed by an integer key. Of the non-sequenceable collections, dictionaries can be accessed by an arbitrary key, such as a string, while sets and bags cannot.\figlabel{CollsBySeq}}%
%\end{center}
%\end{figure*}
%\on{A useless diagram -- refer to the class hierarchy instead!}

Au-del\`a de cette homog\'en\'eit\'e apparente,
il y a diff\'erentes sortes de collections soit, supportant des protocoles 
diff\'erents soit, offrant un comportement diff\'erent pour une m\^eme requ\^ete.
Parcourons bri\`evement certaines de ces divergences essentielles:

\begin{itemize}
  \item {\bf Les s\'equentielles ou \emph{Sequenceable}:}
  les instances de toutes les sous-classes de \clsind{SequenceableCollection} 
d\'ebutent par un premier \'el\'ement dit \mthind{SequenceableCollection}{first} et progresse dans un ordre bien d\'efini jusqu'au
dernier \'el\'ement dit \mthind{SequenceableCollection}{last}.
Les instances de \clsind{Set}, \clsind{Bag} (ou multiensemble) et \clsind{Dictionary} ne sont pas des collections s\'equentielles.

  \item {\bf Les tri\'ees ou \emph{Sortable}:}
une \clsind{SortedCollection} maintient ses \'el\'ements dans un ordre de 
tri.
% sort order.

  \item {\bf Les index\'ees ou \emph{Indexable}:}
la majorit\'e des collections s\'equentielles sont aussi index\'ees, \cad
que ses \'el\'ements peuvent \^etre extraits par 	
\ct{at:} qui peut se traduire par l'expression ``\`a l'endroit indiqu\'e''.
	Le tableau \clsind{Array} est une structure de donn\'ees index\'ees famili\`ere avec une taille fixe;  \ct{anArray at: n} r\'ecup\`ere le
 $\mbox{\ct{n}}^{\ieme}$ \'el\'ement de \ct{anArray} alors que, 
\ct{anArray at: n put: v} change le $\mbox{\ct{n}}^{\ieme}$ \'el\'ement 
par \ct{v}.
	Les listes cha\^{\i}n\'ees de classe \ct{LinkedList} et
les listes \`a enjambements de classe \ct{SkipList} sont s\'equentielles mais non-index\'ees; autrement dit, elles acceptent \ct{first} et \ct{last}, mais pas \ct{at:}.
	\clsindex{LinkedList}
	\clsindex{SkipList}

  \item {\bf Les collections \`a cl\'es ou \emph{Keyed}:}
	les instances du dictionnaire \clsind{Dictionary} et ses sous-classes
sont accessibles via des cl\'es plut\^ot que par des indices.

  \item {\bf Les collections modifiables ou \emph{Mutable}:}
  	la plupart des collections sont dites \emph{mutables} \cad modifiables, mais les intervalles \ct{Interval} et les symboles \ct{Symbol} ne le sont pas.
	Un \clsind{Interval} est une collection non-modifiable ou \emph{immutable} repr\'esentant une rang\'ee d'entiers \ct{Integer}.  Par exemple, 
\ct{5 to: 16 by: 2} est un intervalle \ct{Interval} qui contient les 
\'el\'ements 5, 7, 9, 11, 13 et 15.  Il est indexable avec \mthind{Interval}{at:} mais ne peut pas \^etre chang\'e avec \ct{at:put:}.
	\clsindex{Symbol}

  \item {\bf Les collections extensibles:}
% ou \emph{Growable}:}
	les instances d'\ct{Interval} et de \clsind{Array} sont
toujours de taille fixe. D'autres types de collections (les collections tri\'ees \ct{SortedCollection}, ordonn\'ees \ct{OrderedCollection} et les listes cha\^{\i}n\'ees \ct{LinkedList}) peuvent \^etre \'etendues apr\`es leur cr\'eation.
	
	La classe \clsind{OrderedCollection} est plus g\'en\'erale que le tableau \mbox{\ct{Array};} la taille d'une \ct{OrderedCollection} grandit \`a la demande et
elle a aussi bien des m\'ethodes d'ajout en d\'ebut \mthind{OrderedCollection}{addFirst:} et en fin \mthind{OrderedCollection}{addLast:} que des m\'ethodes
\mthind{OrderedCollection}{at:} et \mthind{OrderedCollection}{at:put:}.
  
  \item {\bf Les collections \`a duplicat:}
%Accepts duplicates:}
  	un \clsind{Set} filtrera les \emph{duplicata} ou doublons mais un \clsind{Bag} (sac, en fran\c{c}ais) ne le fera pas.
	Les collections non-ordonn\'ees \clsind{Dictionary}, \ct{Set} et \ct{Bag} utilisent la m\'ethode
\ct{=} fournie par les \'el\'ements; les variantes \ct{Identity} de ces
classes (\ct{IdentityDictionary}, \ct{IdentitySet} et \ct{IdentityBag})
utilisent la m\'ethode \ct{==} qui teste si les arguments sont 
le m\^eme objet et les variantes \ct{Pluggable} emploient une \'equivalence
arbitraire d\'efinie par le cr\'eateur de la collection.
	\index{Collection!Pluggable}

  \item {\bf Les collections h\'et\'erog\`enes:}
%Heterogeneous:}
	La plupart des collections stockent n'importe quel type d'\'el\'ement.
	Un \clsind{String}, un \clsind{CharacterArray} ou \clsind{Symbol} ne contiennent cependant que des caract\`eres de classe \ct{Character}.
	Un \clsind{Array} pourra inclure un m\'elange de diff\'erents objets
 mais un tableau d'octets \lct{ByteArray} ne comprendra que des 
octets \ct{Byte}; tout comme un \clsind{IntegerArray} n'a que des entiers \ct{Integer}s et qu'un \clsind{FloatArray} ne peut contenir que des r\'eels \`a virgule flottante de classe \ct{Float}.
	Une liste cha\^{\i}n\'ee \clsind{LinkedList} est contrainte \`a ne 
pouvoir contenir que des \'el\'ements qui sont conformes au protocole
\prot{Link \go accessing}.

\end{itemize}

%  How is the order established? Sorted collections use a supplied total ordering function, intervals are implicitly ordered, while arrays and ordered collections are ordered explicitly when elements are inserted.

%=========================================================
\section{Les impl\'ementations des collections}
\seclabel{implementation}

\begin{figure*}
\begin{center}
\includegraphics[width=\textwidth]{CollectionsByImpl}
\caption{Certaines classes de collections rang\'ees selon leur technique d'impl\'ementation.
    \figlabel{collsByImpl}}
\end{center}
\end{figure*}

Consid\'erer ces cat\'egorisations par fonctionnalit\'e n'est pas 
suffisant; nous devons aussi regarder les classes de collections selon
leur impl\'ementation. Comme nous le montre 
% by functionality are not our only concern; we must also consider how the collection classes are implemented. 
\figref{collsByImpl}, cinq techniques d'impl\'ementations majeures 
sont employ\'ees.

\begin{enumerate}
  \item Les tableaux ou \emph{Arrays} stockent leurs \'el\'ements dans une
variable d'instance indexable de l'objet collection lui-m\^eme; 
d\`es lors, les tableaux doivent \^etre de taille fixe mais peuvent \^etre
cr\'e\'es avec une simple allocation de m\'emoire.
  \clsindex{Array}
%ajout
  \index{tableau}
  \item Les collections ordonn\'ees \clsind{OrderedCollection} et tri\'ees 
\clsind{SortedCollection} contiennent leurs \'el\'ements dans un tableau
qui est r\'ef\'erenc\'e par une des variables d'instance de la collection.
En cons\'equence, le tableau interne peut \^etre remplac\'e par un 
plus grand si la collection grossit au del\`a des capacit\'es de 
stockage.
  \item Les diff\'erents types d'ensemble (ou \emph{set}) et les dictionnaires
sont aussi r\'ef\'erenc\'es par un tableau de stockage subsidiaire
mais ils utilisent ce tableau comme une table de hachage (ou \emph{hash table}). Les ensembles dits sacs ou \emph{bags} (de classe \ct{Bag}) utilisent
un dictionnaire \ct{Dictionary} pour le stockage avec pour cl\'es
des \'el\'ements du \ct{Bag} et pour valeurs leur nombre d'occurences.
  \clsindex{Bag}
  \item Les listes cha\^{\i}n\'ees LinkedList utilisent une
repr\'esentation standard simplement cha\^{\i}n\'ee.
  \clsindex{LinkedList}
  \item Les intervalles \ct{Interval} sont repr\'esent\'ees 
par trois entiers qui enregistrent les deux points extr\^emes et la taille de pas.
  \clsindex{Interval}
\end{enumerate}
En plus de ces classes, il y a aussi les variantes de \mbox{\ct{Array},} de \ct{Set} et de plusieurs sortes de dictionnaires dites \`a liaisons faibles ou ``\subind{Collection}{weak}''. Ces collections maintiennent faiblement leurs \'el\'ements, \ie de mani\`ere \`a ce qu'elles n'emp\^echent pas ses \'el\'ements d'\^etre
recycl\'es par le ramasse-miettes ou \emph{garbage collector}.
La machine virtuelle \pharo est consciente de ces classes et les g\`ere d'une
fa\c{c}on particuli\`ere.
\index{Collection!à liaison faible}
\seeindex{weak collections}{collection!\`a liaison faible}

Les lecteurs int\'eress\'es dans l'apprentissage avanc\'e des collections
de \st sont renvoy\'es \`a la lecture de l'excellent livre de LaLonde et Pugh
\cite{LaLo90a}.
% remarque de martial: ca aussi etait ma bible d'apprentissage

%=========================================================
\section{Exemples de classes importantes}
Nous pr\'esentons maintenant les classes de collections les plus communes et les plus importantes via des exemples de code simples.
Les protocoles principaux de collections sont: \mthind{Collection}{at:}, \mthind{Collection}{at:put:} --- pour acc\'eder \`a un \'el\'ement, \mthind{Collection}{add:}, \mthind{Collection}{remove:} --- pour ajouter ou enlever un \'el\'ement, \mthind{Collection}{size}, \mthind{Collection}{isEmpty}, \mthind{Collection}{include:} --- pour obtenir des informations respectivement sur la taille, la virginit\'e (collection vide) et l'inclusion dans la collection, \mthind{Collection}{do:}, \mthind{Collection}{collect:}, \mthind{Collection}{select:} --- pour agir en it\'erations \`a travers la collection.
Chaque collection impl\'emente ou non de tels protocoles et quand elle 
le fait, elle les interpr\'ete pour \^etre en ad\'equation avec leurs s\'emantiques.
Nous vous sugg\'erons de naviguer dans les classes elles-m\^eme pour
identifier par vous m\^eme les protocoles sp\'ecifiques et plus
avanc\'es.
\seeindex{do:@\ct{do:}}{Collection, \ct{do:}}
\seeindex{at:@\ct{at:}}{Collection, \ct{at:}}
\seeindex{at:put:@\ct{at:put:}}{Collection, \ct{at:put:}}

Nous nous focaliserons sur les classes de collections les plus courantes:
\clsind{OrderedCollection}, \clsind{Set}, \clsind{SortedCollection}, \clsind{Dictionary}, \clsind{Interval} et \clsind{Array}.

\paragraph{Les protocoles communs de cr\'eation.}
Il existe plusieurs fa\c{c}ons de cr\'eer des instances de collections.
La technique la plus g\'en\'erale consiste \`a utiliser les m\'ethodes
\mthind{Collection class}{new:} et \mthind{Collection class}{with:}.
\ct{new: anInteger} cr\'ee une collection de taille \ct{anInteger} 
dont les \'el\'ements seront tous nuls \cad de valeur \ct{nil}. 
\mthind{Collection class}{with:} \ct{anObject} cr\'ee une collection
et ajoute \ct{anObject} \`a la collection cr\'e\'ee. 
Les collections r\'ealisent cela de diff\'erentes mani\`eres.

Vous pouvez cr\'eer des collections avec des \'el\'ements initiaux
en utilisant les m\'ethodes \mbox{\mthind{Collection class}{with:},}
\mthind{Collection class}{with:with:} \etc; et ce jusqu'\`a 
six \'el\'ements (donc six \ct{with:}).

\needlines{3} % CHANGE
\begin{code}{@TEST}
Array with: 1 --> #(1)
Array with: 1 with: 2 --> #(1 2)
Array with: 1 with: 2 with: 3 --> #(1 2 3)
Array with: 1 with: 2 with: 3 with: 4 --> #(1 2 3 4)
Array with: 1 with: 2 with: 3 with: 4 with: 5 --> #(1 2 3 4 5)
Array with: 1 with: 2 with: 3 with: 4 with: 5 with: 6 --> #(1 2 3 4 5 6)
\end{code}

Vous pouvez aussi utiliser la m\'ethode \ct{addAll:} pour ajouter tous les \'el\'ements d'une classe \`a une autre:

\begin{code}{@TEST}
(1 to: 5) asOrderedCollection addAll: '678'; yourself --> an OrderedCollection(1 2 3 4 5 $6 $7 $8)
\end{code}

\noindent
Prenez garde au fait que \mthind{Collection}{addAll:} renvoie aussi ses arguments et non pas le receveur!

Vous pouvez aussi cr\'eer plusieurs collections avec les m\'ethodes  
\mthind{Collection class}{withAll:} ou \mthind{Collection class}{newFrom:}

\begin{code}{@TEST}
Array withAll: #(7 3 1 3)                      --> #(7 3 1 3)
OrderedCollection withAll: #(7 3 1 3) --> an OrderedCollection(7 3 1 3)
SortedCollection withAll: #(7 3 1 3)    --> a SortedCollection(1 3 3 7)
Set withAll: #(7 3 1 3)                         --> a Set(7 1 3)
Bag withAll: #(7 3 1 3)                        --> a Bag(7 1 3 3)
Dictionary withAll: #(7 3 1 3)               --> a Dictionary(1->7 2->3 3->1 4->3 )
\end{code}

\begin{code}{@TEST}
Array newFrom: #(7 3 1 3)                                          --> #(7 3 1 3)
OrderedCollection newFrom: #(7 3 1 3)                     --> an OrderedCollection(7 3 1 3)
SortedCollection newFrom: #(7 3 1 3)                       --> a SortedCollection(1 3 3 7)
Set newFrom: #(7 3 1 3)                                            --> a Set(7 1 3)
Bag newFrom: #(7 3 1 3)                                           --> a Bag(7 1 3 3)
Dictionary newFrom: {1 -> 7. 2 -> 3. 3 -> 1. 4 -> 3} --> a Dictionary(1->7 2->3 3->1 4->3 )
\end{code}
\noindent
Notez que ces m\'ethodes ne sont pas identiques.
En particulier, \cmind{Dictionary class}{withAll:} interpr\`ete 
ses arguments comme un collection de valeurs alors que 
\cmind{Dictionary class}{newFrom:} s'attend \`a une collection
d'associations.

%---------------------------------------------------------
\subsection{Le tableau Array}
Un tableau \clsindmain{Array} est une collection de taille fixe
dont les \'el\'ements sont accessibles par des indices entiers.
Contrairement \`a la convention \'etablie dans le langage C,
le premier \'el\'ement d'un tableau \st est \`a la position 1 et
non \`a la position 0.
Le protocole principal pour acc\'eder aux \'el\'ements d'un tableau
est la m\'ethode
\mthind{Array}{at:} et la m\'ethode \mthind{Array}{at:put:}.
 \ct{at: anInteger} renvoie l'\'el\'ement \`a l'index 
\ct{anInteger}. \ct{at: anInteger put: anObject} met \ct{anObject} \`a 
l'index \ct{anInteger}. Comme les tableaux sont des collections de taille
fixe nous ne pouvons pas ajouter ou enlever des \'el\'ements \`a la fin du
tableau.
Le code suivant cr\'ee un tableau de taille 5, place des valeurs dans les 3
premi\`eres cases et retourne le premier \'el\'ement.

\begin{code}{@TEST | anArray | }
anArray := Array new: 5.
anArray at: 1 put: 4.
anArray at: 2 put: 3/2.
anArray at: 3 put: 'ssss'.
anArray at: 1 --> 4
\end{code}

Il y a plusieurs fa\c{c}ons de cr\'eer des instances de la classe 
\clsind{Array}. Nous pouvons utiliser \ct{new:}, \ct{with:} et les 
constructions bas\'ees sur \ct{#( )} et \ct|{ }|.

\paragraph{Cr\'eation avec \mthind{Array class}{new:}} \ct{new: anInteger} cr\'ee un tableau de taille \ct{anInteger}.
\ct{Array new: 5} cr\'ee un tableau de taille 5.

\paragraph{Cr\'eation avec \mthind{Array class}{with:}} les m\'ethodes 
\ct{with:} permettent de sp\'ecifier la valeur des \'el\'ements.
Le code suivant cr\'ee un tableau de trois \'el\'ements compos\'es
du nombre
\ct{4}, de la fraction \ct{3/2} et de la cha\^{\i}ne de caract\`eres
\ct{'lulu'}.

\begin{code}{@TEST | anArray |}
Array with: 4 with: 3/2 with: 'lulu' -->  {4 . (3/2) . 'lulu'}
\end{code}

\paragraph{Cr\'eation litt\'eral avec \ct|\#()|.}
\index{littéral!tableau}
\seeindex{Array!littéral}{littéral, tableau}
\ct{#()} cr\'ee des tableaux littéraux avec des \'el\'ements
statiques qui doivent \^etre connus
quand l'expression est compil\'ee et non lorsqu'elle est ex\'ecut\'ee.
Le code suivant cr\'ee un tableau de taille 2
dans lequel le premier \'el\'ement est le nombre \ct{1}
et le second la cha\^{\i}ne de caract\`eres \ct{'here'}: tous deux sont des litt\'eraux.

\seeindex{\#@{\textsf{\#( )}}}{littéral, tableau}
\seeindex{\{@{\textsf{\{ \}}}}{tableau, dynamique}
\seeindex{dynamique!tableau}{tableau, dynamique}
\seeindex{Array!dynamique}{tableau, dynamique}

\begin{code}{@TEST | anArray |}
#(1 'here') size --> 2
\end{code}

Si vous \'evaluez d\'esormais \ct{#(1+2)}, vous n'obtenez pas un tableau
avec un unique \'el\'ement \ct{3} mais vous obtenez plut\^ot le tableau
\ct{#(1 #+ 2)} \ie avec les trois \'el\'ements: \ct{1}, le symbole
\ct{#+} le chiffre \ct{2}.

\begin{code}{@TEST}
#(1+2) -->  #(1 #+ 2)
\end{code}

\noindent
Ceci se produit parce que la construction \ct{#()} fait que
le compilateur interpr\`ete litt\'erallement les expressions contenues dans
le tableau. L'expression est analys\'ee et les \'el\'ements r\'esultants
forment un nouveau tableau. Les tableaux litt\'eraux contiennent des
nombres, l'\'el\'ement \ct{nil}, des bool\'eens \ct{true} et \ct{false},
des symboles et des cha\^{\i}nes de caract\`eres.

\paragraph{Cr\'eation dynamique avec \ct|\{ \}|.}
Vous pouvez finalement cr\'eer un tableau dynamique en utilisant
la construction suivante: \ct|{}|. \ct|{ a . b }| est \'equivalent
\`a \lct{Array with: a with: b}. En particulier, les expressions incluses
entre \ct|{| et \ct|}| sont ex\'ecut\'ees. Chaque expression est séparée de la précédente par un point.

\begin{code}{@TEST | anArray |}
{ 1 + 2 } --> #(3)
{(1/2) asFloat} at: 1 --> 0.5
{10 atRandom . 1/3} at: 2 --> (1/3)
\end{code}

\paragraph{L'acc\`es aux \'el\'ements.}
Les \'el\'ements de toutes les collections s\'equentielles peuvent
\^etre acc\'ed\'es avec les messages \mthind{Array}{at:} et
 \mthind{Array}{at:put:}.

\begin{code}{@TEST | anArray |}
anArray := #(1 2 3 4 5 6) copy.
anArray at: 3 --> 3
anArray at: 3 put: 33.
anArray at: 3 --> 33
\end{code}
\cmindex{Array}{copy}
\seeindex{tableau!copie}{Array, \ct{copy}}

\noindent
Soyez attentif au fait que le code modifie les tableaux litt\'eraux!
Le compilateur essaie d'allouer l'espace n\'ecessaire aux tableaux litt\'eraux.
\`A moins que vous ne copiez le tableau, la seconde fois que vous \'evaluez
le code, votre tableau ``litt\'eral'' pourrait ne pas avoir la valeur que vous
attendez.
(sans clonage, la seconde fois, le tableau litt\'eral \ct{#(1 2 3 4 5 6)} sera en fait \ct{#(1 2 33 4 5 6)}!)
Les tableaux dynamiques n'ont pas ce probl\`eme.

%---------------------------------------------------------
\subsection{La collection ordonn\'ee OrderedCollection}
\clsindmain{OrderedCollection} est une des collections qui peut s'\'etendre
et auxquelles des \'el\'ements peuvent \^etre adjoints s\'equentiellement.
Elle offre une vari\'et\'e de m\'ethodes telles que \mthind{OrderedCollection}{add:}, \mthind{OrderedCollection}{addFirst:}, \mthind{OrderedCollection}{addLast:} et \mthind{OrderedCollection}{addAll:}.

\begin{code}{@TEST | ordCol |}
ordCol := OrderedCollection new.
ordCol add: 'Seaside'; add: 'SqueakSource'; addFirst: 'Monticello'.
ordCol --> an OrderedCollection('Monticello' 'Seaside' 'SqueakSource')
\end{code}

\paragraph{Effacer des \'el\'ements.} La m\'ethode \mthind{OrderedCollection}{remove:} \ct{anObject} efface la premi\`ere occurence d'un objet dans la collection. Si la collection n'inclut pas l'objet, elle l\`eve une erreur.

\begin{code}{@TEST | ordCol | ordCol := OrderedCollection with: 'Monticello' with: 'Seaside' with: 'SqueakSource'.}
ordCol add: 'Monticello'.
ordCol remove: 'Monticello'.
ordCol --> an OrderedCollection('Seaside' 'SqueakSource' 'Monticello')
\end{code}

Il y a une variante de \ct{remove:} nomm\'ee \mthind{OrderedCollection}{remove:ifAbsent:} qui permet de sp\'ecifier comme second argument un bloc ex\'ecut\'e dans le cas o\`u l'\'el\'ement \`a effacer n'est pas dans la collection. 

\begin{code}{@TEST | ordCol res | ordCol := OrderedCollection with: 'Seaside' with: 'SqueakSource' with: 'Monticello'.}
res := ordCol remove: 'zork' ifAbsent: [33].
res --> 33
\end{code}

\paragraph{La conversion.}
Il est possible d'obtenir une collection ordonn\'ee \ct{OrderedCollection} depuis
un tableau \ct{Array} (ou n'importe quelle autre collection) en envoyant le message \mthind{Collection}{asOrderedCollection}:

\begin{code}{@TEST}
#(1 2 3) asOrderedCollection --> an OrderedCollection(1 2 3)
'hello' asOrderedCollection --> an OrderedCollection($h $e $l $l $o)
\end{code}

%\paragraph{A word about automatic growth.}
%Note that while \ct{OrderedCollection} grows automatically when you add new elements it can be better to create an \ct{OrderedCollection} with a good approximation of the numbers of elements it will contain. The following code creates an ordered collection whose size is equal to the number of classes in \pharo and adds all the classes to it.

% \on{I don't like this example. No one should ever write code like this. Furthermore it is not good advice, as you almost never need to do this.}

%\begin{code}{@TEST | classes res | }
%classes := Smalltalk allClasses.
%res := OrderedCollection new: classes size.
%classes do: [:each | res add: each].
%classes size = res size --> true
%\end{code}

%\noindent
%This can be faster than creating the ordered collection with \ct{new} and letting it grow (see \charef{profiling}).

%---------------------------------------------------------
\subsection{L'intervalle Interval}
La classe \clsindmain{Interval} repr\'esente une suite de nombres.
Par exemple, l'intervalle compris entre 1 et 100 est d\'efini comme
suit:
\cmindex{Interval class}{from:to:}
\begin{code}{@TEST}
Interval from: 1 to: 100 --> (1 to: 100)
\end{code}

\noindent
L'imprim\'e ou l'affichage en mode \mthind{Interval class}{printString} de
cet intervalle nous r\'ev\`ele que la classe nombre \ct{Number} (repr\'esentant les nombres) dispose d'une m\'ethode de convenance appel\'ee \mthind{Number}{to:} (dans le sens de l'expression ``jusqu'\`a'') pour g\'en\'erer les intervalles:

\begin{code}{}
(Interval from: 1 to: 100) = (1 to: 100) --> true
\end{code}

Nous pouvons utiliser \cmind{Interval class}{from:to:by:} (mot \`a mot: depuis-jusque-par) ou
\cmind{Number}{to:by:} (jusque-par) pour sp\'ecifier le pas entre les deux nombres comme suit:

\begin{code}{@TEST}
(Interval from: 1 to: 100 by: 0.5) size --> 199
(1 to: 100 by: 0.5) at: 198 --> 99.5
(1/2 to: 54/7 by: 1/3) last --> (15/2)
\end{code}

%---------------------------------------------------------
\subsection{Le dictionnaire Dictionary}
Les dictionnaires sont des collections importantes dont les \'el\'ements
sont accessibles via des cl\'es.
Parmi les messages de dictionnaire les plus couramment utilis\'es, vous trouverez  
\mthind{Dictionary}{at:}, \mthind{Dictionary}{at:put:}, \mthind{Dictionary}{at:ifAbsent:}, \mthind{Dictionary}{keys} et \mthind{Dictionary}{values} (\emph{keys} et \emph{values} sont les mots anglais pour cl\'es et valeurs respectivement).
\seeindex{keys}{Dictionary, \ct{keys}}
\seeindex{values}{Dictionary, \ct{values}}
\seeindex{dictionnaire!clé}{Dictionary, \ct{keys}}
\seeindex{dictionnaire!valeur}{Dictionary, \ct{values}}

\begin{code}{@TEST | colors |}
colors := Dictionary new.
colors at: #yellow put: Color yellow.
colors at: #blue put: Color blue.
colors at: #red put: Color red.
colors at: #yellow --> Color yellow
colors keys          --> a Set(#blue #yellow #red)
colors values       --> {Color blue . Color yellow . Color red}
\end{code}

Les dictionnaires comparent les cl\'es par \'egalit\'e. Deux cl\'es sont
consid\'er\'ees comme \'etant la m\^eme si elles retournent \emph{true}
lorsqu'elles sont compar\'ees par \ct{=}. Une erreur commune et difficile \`a
identifier est d'utiliser un objet dont la m\'ethode \ct{=} a \'et\'e
red\'efinie mais pas sa m\'ethode de hachage \ct{hash}. Ces deux
m\'ethodes sont utilis\'ees dans l'impl\'ementation du dictionnaire
et lorsque des objets sont compar\'es.
\index{dictionnaire!surcharger \ct{=} et \ct{hash}}
\seeindex{Dictionary!surcharger \ct{=} et \ct{hash}}{dictionnaire!surcharger \ct{=} et \ct{hash}}

La classe \clsindmain{Dictionary} illustre clairement que la hi\'erarchie
de collections \arevoir{est bas\'ee sur l'h\'eritage et non sur du sous-typage}.
M\^eme si \ct{Dictionary} est une sous-classe de \clsind{Set}, nous
ne voudrions normalement pas utiliser un \ct{Dictionary} l\`a o\`u
un \ct{Set} est attendu. 
Dans son impl\'ementation pourtant un \ct{Dictionary} peut
clairement \^etre vu comme \'etant constitu\'e d'un ensemble d'associations
de valeurs et de cl\'es cr\'e\'e par le message \mthind{Object}{->}. Nous
pouvons cr\'eer un \ct{Dictionary} depuis une collection d'associations; nous
pouvons aussi convertir un dictionnaire en tableau d'associations.
\seeindex{association}{Object, \ct{->}}

\needlines{5}
\begin{code}{@TEST | colors |}
colors := Dictionary newFrom: { #blue->Color blue. #red->Color red. #yellow->Color yellow }.
colors removeKey: #blue.
colors associations --> {#yellow->Color yellow . #red->Color red}
\end{code}

\paragraph{IdentityDictionary.}
Alors qu'un dictionnaire utilise le r\'esultat des messages \ct{=} et \ct{hash} pour d\'eterminer si deux cl\'es sont la m\^eme, la classe \clsindmain{IdentityDictionary} utilise l'identit\'e (\cad le message \mthind{ProtoObject}{==}) de la cl\'e au lieu de celle de ses valeurs, \ie qu'il consid\`ere deux cl\'es comme \'egales \emph{seulement} si elles sont le m\^eme objet.

Souvent les symboles de classe \ct{Symbol} sont utilis\'es comme cl\'es, dans les cas o\`u le choix de \ct{IdentityDictionary} s'impose, car un symbole est toujours certain d'\^etre globalement unique. Si d'un autre c\^ot\'e, vos cl\'es sont des chaînes de caract\`eres 
\ct{String}, il est pr\'ef\'erable d'utiliser un \ct{Dictionary} ou sinon vous pourriez avoir des ennuis:

\begin{code}{@TEST | a b trouble |}
a := 'foobar'.
b := a copy.
trouble := IdentityDictionary new.
trouble at: a put: 'a'; at: b put: 'b'.
trouble at: a          --> 'a'
trouble at: b          --> 'b'
trouble at: 'foobar' --> 'a'
\end{code}

\noindent
Comme \ct{a} et \ct{b} sont des objets diff\'erents, ils sont trait\'es comme des objets diff\'erents.
%Interestingly,
Le litt\'eral \mbox{\ct{'foobar'}} est allou\'e une seule fois et
ce n'est vraiment pas le m\^eme objet que \ct{a}.
Vous ne voulez pas que votre code d\'epende d'un tel comportement!
Un simple \ct{Dictionary} vous donnerait la m\^eme valeur pour n'importe quelle
cl\'e \'egale \`a \ct{'foobar'}.

%ajout
Vous ne vous tromperez pas en utilisant seulement des \ct{Symbol}s comme cl\'e d'\ct{IdentityDictionary} et des \ct{String}s (ou d'autres objets) comme cl\'e de \ct{Dictionary} classique.

Notez que l'objet global \glbind{Smalltalk} est une instance de \clsind{SystemDictionary} sous-classe de  \ct{IdentityDictionary}; de ce fait, toutes ses cl\'es sont des \ct{Symbol}s (en r\'ealit\'e, des symboles de la classe \ct{ByteSymbol} qui contiennent des caract\`eres de 8 bits).

\begin{code}{@TEST}
Smalltalk keys collect: [ :each | each class ] --> a Set(ByteSymbol)
\end{code}
\noindent
Envoyer \ct{keys} ou \ct{values} \`a un \ct{Dictionary} nous renvoie 
un ensemble \ct{Set}; nous explorerons cette collection dans la section
qui suit.
% Since every key has the same class, the set of classes of keys contains only a single element, \ct{ByteSymbol}.

%---------------------------------------------------------
\subsection{L'ensemble Set}
La classe \clsindmain{Set} est une collection qui se comporte comme un ensemble
dans le sens math\'ematique \ie comme une collection sans doublons
et sans aucun ordre particulier. Dans un \ct{Set}, les \'el\'ements sont
ajout\'es en utilisant le message \mthind{Set}{add:} (signifiant
``ajoute'' en anglais) et ils ne peuvent pas \^etre accessibles par le message de recherche par indice \ct{at:}. 
Les objets \`a inclure dans \ct{Set} doivent impl\'ementer les m\'ethodes \ct{hash} et \ct{=}.

\begin{code}{@TEST | s | }
s := Set new.
s add: 4/2; add: 4; add:2.
s size --> 2
\end{code}

Vous pouvez aussi cr\'eer des ensembles via \cmind{Set class}{newFrom:} ou
par le message de conversion \cmind{Collection}{asSet}:

\begin{code}{@TEST}
(Set newFrom: #( 1 2 3 1 4 )) = #(1 2 3 4 3 2 1) asSet --> true
\end{code}

La m\'ethode \mthind{Collection}{asSet} offre une fa\c{c}on efficace pour \'eliminer les doublons dans une collection:
\begin{code}{@TEST}
{ Color black. Color white. (Color red + Color blue + Color green) } asSet size --> 2
\end{code}
\noindent
Notez que rouge (message \ct{red}) + bleu (message \ct{blue}) + vert (message \ct{green}) donne du blanc (message \ct{white}).

Une collection \clsindmain{Bag} ou \emph{sac} est un peu comme un \ct{Set} 
qui autorise le duplicata:
\begin{code}{@TEST}
{ Color black. Color white. (Color red + Color blue + Color green) } asBag size --> 3
\end{code}

Les op\'erations sur les ensembles telles que
l'\emph{union}, l'\emph{intersection} et le test d'\emph{appartenance} 
sont impl\'ement\'ees respectivement par les messages de \ct{Collection} \mthind{Collection}{union:}, \mthind{Collection}{intersection:} et \mthind{Collection}{includes:}.
Le receveur est d'abord converti en un \ct{Set}, ainsi ces op\'erations fonctionnent pour toute sorte de collections!

\seeindex{Set!union}{Collection, \ct{union:}}
\seeindex{Set!intersection}{Collection, \ct{intersection:}}
\seeindex{Set!membership}{Collection, \ct{includes:}}

\begin{code}{@TEST}
(1 to: 6) union: (4 to: 10)  --> a Set(1 2 3 4 5 6 7 8 9 10)
'hello' intersection: 'there' --> 'he'
#Smalltalk includes: $k     --> true
\end{code}

Comme nous l'avons expliqu\'e plus haut les \'el\'ements de \ct{Set} sont
accessibles en utilisant des \emph{méthodes d'itérations (itérateurs)} (voir \secref{iterators}).

%---------------------------------------------------------
\subsection{La collection tri\'ee SortedCollection}
Contrairement \`a une collection ordonn\'ee \ct{OrderedCollection}, 
une \clsindmain{SortedCollection} maintient ses \'el\'ements dans un ordre
de tri. 
Par d\'efaut, une collection tri\'ee utilise le message
\mthind{Magnitude}{<=} pour \'etablir l'ordre du tri, autrement
dit, elle peut trier des instances de sous-classes de la classe abstraite
\clsind{Magnitude} qui d\'efinit le protocole d'objets comparables
(\mthind{Magnitude}{<}, \mthind{Magnitude}{=}, \mthind{Magnitude}{>}, \mthind{Magnitude}{>=}, \mthind{Magnitude}{between:and:}...).
(voir \charef{basic}.)

Vous pouvez cr\'eer une \ct{SortedCollection} en cr\'eant une nouvelle 
instance et en lui ajoutant des \'el\'ements:
\begin{code}{@TEST}
SortedCollection new add: 5; add: 2; add: 50; add: -10; yourself. --> a SortedCollection(-10 2 5 50)
\end{code}

\noindent
Le message \mthind{Collection}{asSortedCollection} nous offre une bonne
technique de conversion souvent utilis\'ee.
%More usually, though, one will send the conversion message \mthind{Collection}{asSortedCollection} to an existing collection:
\begin{code}{@TEST}
#(5 2 50 -10) asSortedCollection --> a SortedCollection(-10 2 5 50)
\end{code}

Cet exemple r\'epond \`a la FAQ suivante:

\important{FAQ: Comment trier une collection?\\
{\sc R\'eponse}: En lui envoyant le message \ct{asSortedCollection}.}


\begin{code}{@TEST}
'hello' asSortedCollection --> a SortedCollection($e $h $l $l $o)
\end{code}

Comment retrouver une cha\^{\i}ne de caract\`eres \ct{String} depuis ce r\'esultat?
Malheureusement \ct{asString} retourne une repr\'esentation descriptive en \ct{printString}; ce n'est bien s\^ur pas ce que nous voulons :
\begin{code}{@TEST}
'hello' asSortedCollection asString --> 'a SortedCollection($e $h $l $l $o)'
\end{code}
\noindent
La bonne r\'eponse est d'utiliser 
%ajout
les messages de classe
\ct{String class>>>newFrom:} ou \ct{String class>>>withAll:}; ou bien
%ajout
le message de conversion g\'en\'erique \ct{Object>>>as:}:
\begin{code}{@TEST}
'hello' asSortedCollection as: String              --> 'ehllo'
String newFrom: ('hello' asSortedCollection) --> 'ehllo'
String withAll: ('hello' asSortedCollection)     --> 'ehllo'
\end{code}
\seeindex{Collection!tri}{Collection, \ct{asSortedCollection}}

Avoir diff\'erents types d'\'el\'ements dans une \ct{SortedCollection} est
possible tant qu'ils sont comparables.
Par exemple nous pouvons m\'elanger diff\'erentes sortes de nombres tels
que des entiers, des flottants et des fractions:
\begin{code}{@TEST | col |}
{ 5. 2/-3. 5.21 } asSortedCollection --> a SortedCollection((-2/3) 5 5.21)
\end{code}

Imaginez que vous voulez trier des objets qui ne d\'efinissent pas
la m\'ethode \ct{<=} ou que vous voulez trier selon une crit\`ere bien sp\'ecifique.
Vous pouvez le faire en sp\'ecifiant un bloc \`a deux arguments.
Par exemple, la classe de couleur \ct{Color} n'est pas une \ct{Magnitude} et
ainsi il n'impl\'emente pas \ct{<=} mais nous pouvons \'etablir un bloc
signalant que les couleurs devrait \^etre tri\'ees selon leur
luminance (une mesure de la brillance).

\begin{code}{@TEST | col |}
col := SortedCollection sortBlock: [:c1 :c2 | c1 luminance <= c2 luminance].
col addAll: { Color red. Color yellow. Color white. Color black }.
col --> a SortedCollection(Color black Color red Color yellow Color white)
\end{code}
\cmindex{SortedCollection class}{sortBlock:}

%---------------------------------------------------------
\subsection{La cha\^{\i}ne de caract\`eres String}
\seeindex{cha\^{\i}ne de caract\`eres}{String}
Un \clsindmain{String} en \st repr\'esente une collection de \ct{Character}s.
Il est s\'equentiel, index\'e, modifiable (\emph{mutable}) et homog\`ene, ne
contenant que des instances de \clsind{Character}.
Comme \ct{Array}, \ct{String} a une syntaxe d\'edi\'ee et est cr\'ee normalement
en d\'eclarant directement une cha\^{\i}ne de caract\`eres litt\'erale avec
de simples guillemets 
%ajout
(symbole \emph{apostrophe} sur votre clavier),
mais les m\'ethodes habituelles de cr\'eation de collection fonctionnent aussi.

\begin{code}{@TEST | s1 s2 |}
'Hello'                                             --> 'Hello'
String with: $A                               --> 'A'
String with: $h with: $i with: $BANG       --> 'hiBANG'
String newFrom: #($h $e $l $l $o) --> 'hello'
\end{code}

En fait, \ct{String} est abstrait.
Lorsque vous instanciez un \ct{String}, vous obtenez en r\'ealit\'e soit
un \clsind{ByteString} en 8 bits ou un \clsind{WideString}~\footnote{\emph{Wide} a le sens: \'etendu} en 32 bits.
% Martial: ou "Pour faire court" (il y a déjà 'simplement')
\arevoir{Pour simplifier}, nous ignorons habituellement la diff\'erence
et parlons simplement d'instances de \ct{String}.

Deux instances de \ct{String} peuvent \^etre concat\'en\'ees avec une virgule (en anglais, \emph{comma}).
\index{Collection!opérateur virgule}
\begin{code}{@TEST |s|}
s := 'no', ' ', 'worries'.
s -->  'no worries'
\end{code}

Comme une cha\^{\i}ne de caract\`eres est modifiable nous pouvons aussi la
changer en utilisant la m\'ethode \mthind{String}{at:put:}.

\begin{code}{@TEST |s| s := 'no', ' ', 'worries'.}
s at: 4 put: $h; at: 5 put: $u.
s --> 'no hurries'
\end{code}

Notez que la m\'ethode virgule est d\'efinie dans la classe \ct{Collection}.
Elle marche donc pour n'importe quelle sorte de collections!
\begin{code}{@TEST}
(1 to: 3) , '45' --> #(1 2 3 $4 $5)
\end{code}
\seeindex{String!concaténation}{Collection, opérateur virgule}
\seeindex{String!virgule}{Collection, opérateur virgule}
\index{Collection!opérateur virgule}

Nous pouvons aussi modifier une cha\^{\i}ne de caract\`eres existante
en utilisant les m\'ethodes \mthind{String}{replaceAll:with:} 
%ajout
(pour remplacer tout avec quelque chose d'autre)
ou \mthind{String}{replaceFrom:to:with:}
%ajout
(pour remplacer depuis tant jusqu'\`a un certain point par quelque chose)
comme nous pouvons le voir ci-dessous. Notez que le nombre de caract\`eres
et l'intervalle doivent \^etre de la m\^eme taille.

\begin{code}{@TEST |s| s := 'no hurries' copy.}
s replaceAll: $n with: $N.
s --> 'No hurries'
s replaceFrom: 4 to: 5 with: 'wo'.
s --> 'No worries'
\end{code}

D'une mani\`ere diff\'erente, \mthind{String}{copyReplaceAll:} cr\'ee 
une nouvelle cha\^{\i}ne de caract\`eres (curieusement, les arguments dans ce cas sont des sous-cha\^{\i}nes et non des caract\`eres ind\'ependants et leur taille n'a pas \`a \^etre identique).

\begin{code}{@TEST |s| s:= 'No worries' copy.}
s copyReplaceAll: 'rries' with: 'mbats' --> 'No wombats'
\end{code}

Un rapide aper\c{c}u de l'impl\'ementation de ces m\'ethodes nous r\'ev\`ele
qu'elles ne sont pas seulement d\'efinies pour les instances de
\ct{String}, mais également pour toutes sortes de collections 
s\'equentielles \ct{SequenceableCollection}; du coup, l'expression suivante
fonctionne aussi:

\begin{code}{@TEST}
(1 to: 6) copyReplaceAll: (3 to: 5) with: { 'three'. 'etc.' } --> #(1 2 'three' 'etc.' 6)
\end{code}

\paragraph{Appariement de cha\^{\i}nes de caractères}
\index{String!appariement de chaînes}
\seeindex{String!pattern matching}{String, appariement de chaînes}
\seeindex{String!filtrage}{String, appariement de chaînes}

Il est possible de demander si une cha\^{\i}ne de caract\`eres
s'apparie \`a  une expression-filtre ou \emph{pattern} en
envoyant le message \mthind{String}{match:}.
Ce \emph{pattern} ou filtre peut sp\'ecifier \ct{*} pour
comparer une s\'erie arbitraire de caract\`eres et \# 
pour repr\'esenter un simple caract\`ere quelconque.
%ajout: martial: j'ai mis en emphase car pour moi c'est important:
Notez que \emph{\ct{match:} est envoy\'e au filtre} et non pas \`a la cha\^{\i}ne
de caract\`eres \`a apparier.
\begin{code}{@TEST}
'Linux *' match: 'Linux mag'                      --> true
'GNU/Linux #ag' match: 'GNU/Linux tag' --> true
\end{code}

\ct{findString:} est une autre m\'ethode utile.
\begin{code}{@TEST}
'GNU/Linux mag' findString: 'Linux'                                                      --> 5
'GNU/Linux mag' findString: 'linux' startingAt: 1 caseSensitive: false  --> 5
\end{code}

\arelire{Des techniques d'appariements plus avanc\'ees par filtre 
offrant les m\^eme possibilit\'es que Perl sont disponibles dans le
paquetage \pkgind{Regex}.} % CHANGE
\index{paquetage!expressions régulières}
\seeindex{regular expression package}{paquetage, expressions régulières}


\paragraph{Quelques essais avec les cha\^{\i}nes de caract\`eres.} 
L'exemple suivant illustre l'utilisation de \mthind{String}{isEmpty}, \mthind{String}{includes:} et \mthind{String}{anySatisfy:} 
%ajout
(ce dernier sp\'ecifiant si la collection satisfait le test pass\'e en argument-bloc, au moins en un \'el\'ement);
ces messages ne sont pas seulement d\'efinis pour \ct{String} mais plus g\'en\'eralement pour toute collection.

\begin{code}{@TEST}
'Hello' isEmpty. --> false
'Hello' includes: $a --> false
'JOE' anySatisfy: [:c | c isLowercase] --> false
'Joe' anySatisfy: [:c | c isLowercase] --> true
\end{code} %$

\paragraph{Les gabarits ou \emph{String templating}.}
Il y a 3 messages utiles pour g\'erer les \emph{gabarits} ou \subind{String}{templating}: \mthind{String}{format:}, \mthind{String}{expandMacros} et \mthind{String}{expandMacrosWith:}.
\seeindex{gabarit}{String!templating}
\seeindex{chaîne de caractères!gabarit}{String!templating}

\begin{code}{@TEST}
'{1} est {2}' format: {'Pharo' . 'extra'}  --> 'Pharo est extra'
\end{code} % CHANGE
% 'Pharo is cool' dans PBE mais la capture d'écran est changée en fonction

Les messages de la famille \emph{expandMacros} offre une substitution
de variables en utilisant \ct{<n>} pour le retour-charriot, \ct{<t>} 
pour la tabulation, \ct{<1s>}, \ct{<2s>}, \ct{<3s>} pour les arguments
(\ct{<1p>}, \ct{<2p>} entourent la cha\^{\i}ne avec des simples guillemets),
et \ct{<1?value1:value2>} pour les clauses conditionnelles.


\begin{code}{@TEST}
'regardez-<t>-ici' expandMacros                                         --> 'regardez-	-ici'
'<1s> est <2s>' expandMacrosWith: 'Pharo' with: 'extra'   --> 'Pharo est extra'
'<2s> est <1s>' expandMacrosWith: 'Pharo' with: 'extra'   --> 'extra est Pharo'
'<1p> ou <1s>' expandMacrosWith: 'Pharo' with: 'extra'  --> '''Pharo'' ou Pharo'
'<1?Quentin:Thibaut> joue' expandMacrosWith: true     --> 'Quentin joue'
'<1?Quentin:Thibaut> joue' expandMacrosWith: false    --> 'Thibaut joue'
\end{code}

\paragraph{Des m\'ethodes utilitaires en vrac.}
La classe \ct{String} offre de nombreuses fonctionnalit\'es incluant les 
messages \mthind{String}{asLowercase} (pour mettre en minuscule), \mthind{String}{asUppercase} (pour mettre en majuscule) et \mthind{String}{capitalized} (pour mettre avec la premi\`ere lettre en capitale). 

\begin{code}{@TEST}
'XYZ' asLowercase --> 'xyz'
'xyz' asUppercase   --> 'XYZ'
'hilaire' capitalized   --> 'Hilaire'
'1.54' asNumber      --> 1.54
'cette phrase est sans aucun doute beaucoup trop longue' contractTo: 20 -->  'cette phr...p longue'
\end{code}

Remarquez qu'il y a g\'en\'eralement une diff\'erence entre demander une
repr\'esentation descriptive de l'objet en cha\^{\i}ne de caract\`eres
en envoyant le message
\mthind{Object}{printString} et en le convertissant en une cha\^{\i}ne de caract\`eres via le message \mthind{Object}{asString}.
Voici un exemple de diff\'erence:

\begin{code}{@TEST}
#ASymbol printString --> '#ASymbol'
#ASymbol asString    --> 'ASymbol'
\end{code}

Un symbole \ct{Symbol} est similaire \`a une cha\^{\i}ne de caract\`eres
mais nous sommes garantis de son unicit\'e globale. Pour cette raison,
les symboles sont pr\'ef\'er\'es aux \ct{String} comme cl\'e de dictionnaire,
en particulier pour les instances de \ct{IdentityDictionary}.
Voyez aussi \charef{basic} pour plus d'informations sur \clsind{String} et \clsind{Symbol}.

%=========================================================
\section{Les collections it\'eratrices ou iterators}
\seclabel{iterators}

En \st, les boucles et les clauses conditionnelles sont simplement
des messages envoy\'es \`a des collections ou d'autres objets
tels que des entiers ou des blocs (voir aussi \charef{syntax}).
En plus des messages de bas niveau comme \ct{to:do:} qui \'evalue un bloc avec un argument qui parcourt les valeurs entre un nombre initial et final,
%which evaluates a block with an argument ranging from an initial to a final number
la hi\'erarchie de collections \st offre de nombreux it\'erateurs de haut niveau.
Ceci vous permet de faire un code plus robuste et plus compact.
\index{Collection!itération}

%---------------------------------------------------------
\subsection{L'it\'eration par (\lct{do:})}
La m\'ethode \mthind{Collection}{do:} est un it\'erateur de collections basique.
Il applique son argument (un bloc avec un simple argument) \`a chaque
\'el\'ement du receveur.
L'exemple suivant imprime toutes les cha\^{\i}nes de caract\`eres
contenu dans le receveur vers le Transcript.

\begin{code}{}
#('bob' 'joe' 'toto') do: [:each | Transcript show: each; cr].
\end{code}

\paragraph{Les variantes.} Il existe de nombreuses variantes de \ct{do:}, 
telles que \mbox{\mthind{Collection}{do:without:},} 
\mbox{\mthind{SequenceableCollection}{doWithIndex:}} 
et \mthind{OrderedCollection}{reverseDo:};
pour les collections index\'ees (\ct{Array}, \ct{OrderedCollection}, \ct{SortedCollection}), la m\'ethode \mthind{SequenceableCollection}{doWithIndex:} 
vous donne acc\`es aussi \`a l'indice courant.
Cette m\'ethode est reli\'ee \`a \ct{to:do:} qui est d\'efinie dans la classe
\ct{Number}.

\begin{code}{@TEST}
#('bob' 'joe' 'toto') doWithIndex: [:each :i | (each = 'joe') ifTrue: [ ^ i ] ] --> 2
\end{code}

Pour des collections ordonn\'ees, \mthind{OrderedCollection}{reverseDo:} parcourt la collection dans l'ordre inverse.

Le code suivant montre un message int\'eressant:
\mthind{Collection}{do:separatedBy:} ex\'ecute un second bloc 
\`a ins\'erer entre les \'el\'ements.
\begin{code}{@TEST | res | }
res := ''.
#('bob' 'joe' 'toto') do: [:e | res := res, e ] separatedBy: [res := res, '.'].
res --> 'bob.joe.toto'
\end{code}
\noindent
Notez que ce code n'est pas tr\`es efficace puisqu'il cr\'ee une cha\^{\i}ne
de caract\`eres interm\'ediaire; il serait pr\'ef\'erable d'utiliser
un flux de donn\'ees en \'ecriture ou \emph{write stream} pour stocker
le r\'esultat dans un tampon (voir \charef{streams}):
\begin{code}{@TEST}
String streamContents: [:stream | #('bob' 'joe' 'toto') asStringOn: stream delimiter: '.' ] --> 'bob.joe.toto'
\end{code}

% DAMIEN: I would write it:
%res := String streamContents: [:stream |
%  #('bob' 'joe' 'toto')
%       do: [:e | stream nextPutAll: e]
%       separatedBy: [stream nextPut: $.]].
%res --> 'bob.joe.toto'
%Or even simpler:
%res := String streamContents: [:stream | #('bob' 'joe' 'toto')
%asStringOn: stream delimiter: '.' ].
%res --> 'bob.joe.toto'


\paragraph{Les dictionnaires.}
Quand la m\'ethode \mthind{Dictionary}{do:} est envoy\'ee \`a un dictionnaire,
les \'el\'ements pris en compte sont les valeurs et non pas les associations.
Les m\'ethodes appropri\'ees sont \mthind{Dictionary}{keysDo:}, \mthind{Dictionary}{valuesDo:} et \mthind{Dictionary}{associationsDo:} pour it\'erer respectivement sur les cl\'es, les valeurs ou les associations.

\begin{code}{}
colors := Dictionary newFrom: { #yellow -> Color yellow. #blue -> Color blue. #red -> Color red }.
colors keysDo: [:key | Transcript show: key; cr].                    "affiche les !cl\'es!"
colors valuesDo: [:value | Transcript show: value;cr].            "affiche les valeurs"
colors associationsDo: [:value | Transcript show: value;cr].  "affiche les associations"
\end{code}

%---------------------------------------------------------
\subsection{Collecter les r\'esultats avec \lct{collect:}}
Si vous voulez traiter les \'el\'ements d'une collection et produire
une nouvelle collection en r\'esultat, vous devez utiliser plut\^ot le
message \ct{collect:} ou d'autres m\'ethodes d'it\'erations au lieu
du message \ct{do:}.
La plupart peuvent \^etre trouv\'es dans le protocole \protind{enumerating} 
de la classe \ct{Collection} et de ses sous-classes.

Imaginez que nous voulions qu'une collection contienne le double des \'el\'ements d'une autre collection. 
En utilisant la m\'ethode \ct{do:}, nous devons \'ecrire le code suivant :

\begin{code}{@TEST | double |}
double := OrderedCollection new.
#(1 2 3 4 5 6) do: [:e | double add: 2 * e].
double --> an OrderedCollection(2 4 6 8 10 12)
\end{code}

\noindent
La m\'ethode \mthind{Collection}{collect:} ex\'ecute son bloc-argument
pour chaque \'el\'ement et renvoie une collection contenant les r\'esultats.
En utilisant d\'esormais \ct{collect:}, notre code se simplifie :
\begin{code}{@TEST}
#(1 2 3 4 5 6) collect: [:e | 2 * e] --> #(2 4 6 8 10 12)
\end{code}

Les avantages de \ct{collect:} sur \mthind{Collection}{do:} sont encore
plus d\'emonstratifs sur l'exemple suivant dans lequel nous g\'en\'erons
une collection de valeurs absolues d'entiers contenues dans une autre
collection :

\begin{code}{@TEST |aCol result|}
aCol :=  #( 2 -3 4 -35 4 -11).
result := aCol species new: aCol size.
1 to: aCol size do: [ :each | result at: each put: (aCol at: each) abs].
result --> #(2 3 4 35 4 11)
\end{code}
\noindent
Comparez le code ci-dessus avec l'expression suivante beaucoup plus simple:
\begin{code}{@TEST}
#( 2 -3 4 -35 4 -11) collect: [:each | each abs ] --> #(2 3 4 35 4 11)
\end{code}
\noindent
Le fait que cette seconde solution fonctionne aussi avec les \ct{Set} et les \ct{Bag} est un autre avantage.

Vous devriez g\'en\'eralement \'eviter d'utiliser \ct{do:} \`a moins que 
vous vouliez envoyer des messages \`a chaque \'el\'ement d'une collection.

Notez que l'envoi du message \ct{collect:} renvoie le m\^eme type de collection
que le receveur.
C'est pour cette raison que le code suivant \'echoue.
(Un \ct{String} ne peut pas stocker des valeurs enti\`eres.)
%hold integer values
\begin{code}{}
'abc' collect: [:ea | ea asciiValue ]      "erreur BANG!"
\end{code}
\noindent
Au lieu de \c{c}a, nous devons convertir d'abord la cha\^{\i}ne de caract\`eres
en \ct{Array} ou un \ct{OrderedCollection}:
\begin{code}{@TEST}
'abc' asArray collect: [:ea | ea asciiValue ] --> #(97 98 99)
\end{code}

En fait, \ct{collect:} ne garantit pas sp\'ecifiquement de retourner 
exactement la m\^eme classe que celle du receveur, mais seulement une classe
de la m\^eme \emph{``esp\`ece''}.  Dans le cas d'\ct{Interval}, l'esp\`ece est en r\'ealit\'e un tableau \ct{Array}!
En effet, dans ce cas, nous ne sommes pas assurés que le résultat pourra être transformé en intervalle.
\begin{code}{@TEST}
(1 to: 5) collect: [ :ea | ea * 2 ] --> #(2 4 6 8 10)
\end{code}

%---------------------------------------------------------
\subsection{S\'electionner et rejeter des \'el\'ements}
% (\ct{select:}, \ct{reject:}, \ct{detect:})}

\mthind{Collection}{select:} renvoie les \'el\'ements du receveur qui satisfont
une condition particuli\`ere:

\begin{code}{@TEST}
(2 to: 20) select: [:each | each isPrime] --> #(2 3 5 7 11 13 17 19)
\end{code}

\mthind{Collection}{reject:} fait le contraire:
\begin{code}{@TEST}
(2 to: 20) reject: [:each | each isPrime] --> #(4 6 8 9 10 12 14 15 16 18 20)
\end{code}

%---------------------------------------------------------
\subsection{Identifier un \'el\'ement avec \lct{detect:}}

La m\'ethode \mthind{OrderedCollection}{detect:} renvoie le premier
\'el\'ement du receveur qui rend vrai le test pass\'e en bloc-argument.
%ajout
\ct{isVowel} retourne vrai \cad \ct{true} si le receveur est une
voyelle non-accentu\'ee (pour plus d'explications, voir page~\pageref{def:isVowel}).
\begin{code}{@TEST}
'through' detect: [:each | each isVowel] --> $o
\end{code} %$

La m\'ethode \mthind{Collection}{detect:ifNone:} est une variante de la m\'ethode \ct{detect:}. Son second bloc est \'evalu\'e quand il n'y a pas d'\'el\'ement trouv\'e dans le bloc.

\begin{code}{@TEST}
Smalltalk allClasses detect: [:each | '*cobol*' match: each asString] ifNone: [ nil ] --> nil
\end{code} % CHANGE: cobol remplace java

%---------------------------------------------------------
\subsection{Accumuler les r\'esultats avec \lct{inject:into:}}
Les langages de programmation fonctionnelle offrent souvent une fonction d'ordre
sup\'erieure appel\'ee \emph{fold} ou \emph{reduce} pour accumuler un r\'esultat
en appliquant un op\'erateur binaire de mani\`ere it\'erative sur tous les
\'el\'ements d'une collection.
\pharo propose pour ce faire la m\'ethode \cmind{Collection}{inject:into:}.

Le premier argument est une valeur initiale et le second est un bloc-argument
\`a deux arguments qui est appliqu\'e au r\'esultat (\ct{sum}) et \`a chaque \'el\'ement (\ct{each}) \`a chaque tour.
%to the result this far, and each element in turn.
%martial: la phrase du dessus est a mieux tourner.

Une application triviale de  \ct{inject:into:} consiste \`a produire 
la somme de nombres stock\'es dans une collection.
%?En se r\'ef\'erant \`a Gauss,
%Following Gauss,
\arelire{D'après Gauss,} 
nous pouvons \'ecrire, en \pharo, cette expression pour sommer les 100 premiers entiers:
\begin{code}{@TEST}
(1 to: 100) inject: 0 into: [:sum :each | sum + each ] --> 5050
\end{code}


Un autre exemple est le bloc suivant \`a un argument pour calculer la factorielle:

\begin{code}{@TEST |factorial|}
factorial := [:n | (1 to: n) inject: 1 into: [:product :each | product * each ] ].
factorial value: 10 --> 3628800
\end{code}

%---------------------------------------------------------
\subsection{D'autres messages}

\paragraph{\mthind{Collection}{count:}} le message \ct{count:} (pour compter) renvoie le nombre d'\'el\'ements satisfaisant le bloc-argument: %appari\'es \`a un bloc.

\begin{code}{@TEST}
Smalltalk allClasses count: [:each | '*Collection*' match: each asString ] --> 3
\end{code} %CHANGE

\paragraph{\mthind{Collection}{includes:}} le message \ct{includes:} v\'erifie si l'argument est contenu dans la collection.

\begin{code}{@TEST | colors |}
colors := {Color white . Color yellow. Color red . Color blue . Color orange}.
colors includes: Color blue. --> true
\end{code}

\paragraph{\mthind{OrderedCollection}{anySatisfy:}} le message \ct{anySatisfy:} renvoie vrai si au moins un \'el\'ement satisfait \`a une condition. 

\begin{code}{@TEST | colors | colors := {Color white . Color yellow. Color red . Color blue . Color orange}.}
colors anySatisfy: [:c | c red > 0.5] --> true
\end{code}
%=========================================================
\section{Astuces pour tirer profit des collections}

\paragraph{Une erreur courante avec \mthind{OrderedCollection}{add:}} l'erreur
suivante est une des erreurs les plus fr\'equentes en \st.
\index{Collection!erreurs courantes}

\begin{code}{@TEST | collection | }
collection := OrderedCollection new add: 1; add: 2.
collection --> 2
\end{code}
\noindent
Ici la variable \ct{collection} ne contient pas la collection nouvellement cr\'e\'ee mais par le dernier nombre ajout\'e.
En effet, la m\'ethode \ct{add:} renvoie l'\'el\'ement ajout\'e et non le receveur.

Le code suivant donne le r\'esultat attendu:
\begin{code}{@TEST | collection |}
collection := OrderedCollection new.
collection add: 1; add: 2.
collection --> an OrderedCollection(1 2)
\end{code}

Vous pouvez aussi utiliser le message \mthind{Object}{yourself} pour
renvoyer le receveur d'une \ind{cascade} de messages:

\begin{code}{@TEST | collection |}
collection := OrderedCollection new add: 1; add: 2; yourself --> an OrderedCollection(1 2)
\end{code}

\paragraph{Enlever un \'el\'ement d'une collection en cours d'it\'eration.}
Une autre erreur que vous pouvez faire est d'effacer un \'el\'ement d'une collection que vous \^etes en train de parcourir de mani\`ere it\'erative en utilisant \mthind{Collection}{remove:}.
\begin{code}{@TEST |range|}
range := (2 to: 20) asOrderedCollection.
range do: [:aNumber | aNumber isPrime ifFalse: [ range remove: aNumber ] ].
range --> an OrderedCollection(2 3 5 7 9 11 13 15 17 19)
\end{code}
\noindent
Ce r\'esultat est clairement incorrect puisque 9 et 15 auraient du \'et\'e
filtr\'es!

La solution consiste \`a copier la collection avant de la parcourir.
\begin{code}{@TEST |range|}
range := (2 to: 20) asOrderedCollection.
range copy do: [:aNumber | aNumber isPrime ifFalse: [ range remove: aNumber ] ].
range --> an OrderedCollection(2 3 5 7 11 13 17 19)
\end{code}

\paragraph{Red\'efinir \`a la fois \ct{=} et \ct{hash}.}
Une erreur difficile \`a identifier se produit lorsque vous
red\'efinissez \ct{=} mais pas \ct{hash}. Les sympt\^omes sont
la perte d'\'el\'ements que vous mettez dans des ensembles ainsi
que d'autres ph\'enom\`enes plus \'etranges. Une solution propos\'ee
par Kent Beck est d'utiliser \ct{xor:} pour red\'efinir \ct{hash}.
Supposons que nous voulons que deux livres soient consid\'er\'es comme
\'egaux si leurs titres et leurs auteurs sont les m\^emes.
Alors nous red\'efinissons non seulement
\ct{=} mais aussi \ct{hash} comme suit:
\index{dictionnaire!surcharger \ct{=} et \ct{hash}}

\begin{method}{Red\'efinir \lct{=} et \lct{hash}.}
Book>>>= aBook
   self class = aBook class ifFalse: [^ false].
   ^ title = aBook title and: [ authors = aBook authors]

Book>>>hash 
   ^ title hash xor: authors hash
\end{method}

Un autre probl\`eme ennuyeux peut surgir lorsque vous utilisez des
objets modifiables ou \emph{mutables}: ils peuvent changer leur
code de hachage constamment quand ils sont \'el\'ements d'un \ct{Set}
ou cl\'es d'un dictionnaire. 
Ne le faites donc pas à moins que vous aimiez vraiment le d\'ebogage!

%=========================================================
\section{R\'esum\'e du chapitre}

La hi\'erarchie des collections en \st offre un vocabulaire commun pour la manipulation uniforme d'une grande famille de collections.

\begin{itemize}
  \item Une distinction essentielle est faite entre les collections s\'equentielles ou 
\ct{SequenceableCollection}s qui stockent leurs \'el\'ements dans un ordre
donn\'e, les dictionnaires de classe \ct{Dictionary} ou de ses sous-classes qui
enregistrent des associations cl\'e--valeur et les ensembles 
(\ct{Set}) ou multi-ensembles (\ct{Bag}) qui sont eux d\'esordonn\'es.
  \item Vous pouvez convertir la plupart des collections en d'autres sortes de 
collections en leur envoyant des messages tels que \ct{asArray}, \ct{asOrderedCollection} \etc.
  \item Pour trier une collection, envoyez-lui le message \ct{asSortedCollection}.
  \item Les tableaux litt\'eraux ou \emph{literal} \ct{Array} sont cr\'e\'es 
gr\^ace \`a une syntaxe sp\'eciale: \ct{#( ... )}.  Les tableaux dynamiques
sont cr\'e\'es avec la syntaxe \ct|{ ... }|.
  \item Un dictionnaire \ct{Dictionary} compare ses cl\'es par \'egalit\'e.
C'est plus utile lorsque les cl\'es sont des instances de \ct{String}. 
Un \ct{IdentityDictionary} utilise l'identit\'e entre objets pour comparer les cl\'es. Il est souhaitable que des \ct{Symbol}s soient utilis\'es comme cl\'es ou que la correspondance soit \'etablie sur les valeurs.
% cette phrase ci-dessus est a revoir: +when mapping object references to values.s
  \item Les cha\^{\i}nes de caract\`eres de classe \ct{String} comprennent
aussi les messages habituels de la collection. En plus, un \ct{String} 
supporte une forme simple d'appariement de formes ou \emph{pattern-matching}. 
Pour des applications plus avanc\'ees, vous aurez besoin du paquetage d'expressions r\'eguli\`eres RegEx.
  \item Le message de base pour l'it\'eration est \ct{do:}. Il est 
utile pour du code imp\'eratif tel que la modification de chaque \'el\'ement d'une collection ou l'envoi d'un message sur chaque \'el\'ement.
  \item Au lieu d'utiliser \ct{do:}, il est d'usage d'employer \ct{collect:}, \ct{select:}, \ct{reject:}, \ct{includes:}, \ct{inject:into:} et d'autres messages de haut niveau pour un traitement uniforme des collections.
  \item Ne jamais effacer un \'el\'ement d'une collection que vous parcourez it\'erativement. Si vous devez la modifier, it\'erez plut\^ot sur une copie.
  \item Si vous surchargez \ct{=}, souvenez-vous d'en faire de m\^eme pour le message \ct{hash} 
%ajout
qui renvoie le code de hachage!
\end{itemize}

%=========================================================
\ifx\wholebook\relax\else
   \bibliographystyle{jurabib}
   \nobibliography{scg}
   \end{document}
\fi
%=========================================================



%:Streams
% $Author: oscar $
% $Date: 2007-09-23 11:56:47 +0200 (Sun, 23 Sep 2007) $
% $Revisionn: 12130 $
% traduit par Martial 
% relecture par Rene (Fri, 21 Dec 2007)
% relecture par Rene (Sun, 13 Jan 2008)
% relecture par Rene (Fri,  8 Jan 2010)
% relecture par Rene (Mon,  9 Aug 2010)
% relecture par Rene (Sat, 16 Apr 2011)
% update: Tue Dec 25 12:52:59 CET 2007
% adaptation pour PBE: martial - Wed Sep  9 22:33:36 CEST 2009 from
% $Author: oscar $ % $Date: 2009-08-16 16:37:09 +0200 (Sun, 16 Aug
% 2009) $ % $Revision: 28477 $
% sync avec la version: 30278
%=================================================================
\ifx\wholebook\relax\else
% --------------------------------------------
% Lulu:
	\documentclass[a4paper,10pt,twoside]{book}
	\usepackage[
		papersize={6.13in,9.21in},
		hmargin={.75in,.75in},
		vmargin={.75in,1in},
		ignoreheadfoot
	]{geometry}
	\input{../common.tex}
	\pagestyle{headings}
	\setboolean{lulu}{true}
% --------------------------------------------
% A4:
%	\documentclass[a4paper,11pt,twoside]{book}
%	\input{../common.tex}
%	\usepackage{a4wide}
% --------------------------------------------
    \graphicspath{{figures/} {../figures/}}
	\begin{document}
	\renewcommand{\nnbb}[2]{} % Disable editorial comments
	\sloppy
\fi
%=================================================================
\newcommand{\stream}{\emph{stream}\xspace}
\newcommand{\streams}{\emph{streams}\xspace}
% remarque generale: beaucoup de commentaires et de chaines de caracteres dans les codes ont ete traduites dans ce chapitre
\chapter{Streams: les flux de données}\chalabel{streams}

\ew{Streams are presented as a way to navigate collection. From my point of view, stream are important not to navigate collection, but to produce/consume data:
(a)	memory constraint. Data can not hold into memory and must be processed in a stream fashion, e.g: encryption
(b)	blocking IO. A stream is a nice abstraction to deal with, and the stream manages internally data availability, buffering, etc. to simplify the consumption/production of data
Only few streams have random access capability.}

Les flux de données ou \streams sont utilisés pour itérer dans
une séquence d'éléments comme des
% sequenced
collections, des fichiers ou des flux réseau.
Les \streams peuvent être en lecture ou en écriture ou les deux.
La lecture et l'écriture est toujours relative à la position courante
dans le \stream. Les \streams peuvent être facilement convertis en
collections 
%ajout d'apres la remarque de lukas
(enfin presque toujours)
et les collections en \streams.
\lr{"Streams can easily be converted into collections." I wouldn't say it like this, because it is not true for all streams (infinite streams). According to Kent Beck we should only talk about conversion when the same protocol is supported. Collections and Streams do not support the same protocol. (p. 249)}

%=============================================================
\section{Deux séquences d'éléments}
Voici une bonne métaphore pour comprendre ce qu'est un flux de données:
un flux de données ou \stream peut être représenté comme deux
séquences d'éléments: une séquence d'éléments passée 
et une séquence d'éléments future.
Le \stream est positionné entre les deux séquences.
Comprendre ce modèle est important car toutes les opérations
sur les \streams en \st en dépendent.
C'est pour cette raison que la plupart des
% Attention \clsindmain n'est pas au tout debut 
classes \clsindmain{Stream} sont des sous-classes de \clsind{PositionableStream}.
\Figref{_abcde} présente un flux de données contenant cinq caractères.
Ce \stream est dans sa position originale \ie qu'il n'y a aucun élément
dans le passé. Vous pouvez revenir à cette position en envoyant le message 
\mthind{PositionableStream}{reset}.

\begin{figure}[ht]
\centerline{\includegraphics[scale=0.5]{_abcdeStef}}
\caption{Un flux de données positionné à son origine.}
\figlabel{_abcde}
\vspace{.2in}
\end{figure}

Lire un élément revient conceptuellement à effacer le premier élément de la séquence 
d'éléments future et le mettre après le dernier élément dans la séquence d'éléments passée.
Aprés avoir lu un élément avec le message \ct{next}, l'état de votre \stream est celui de \figref{a_bcde}.

\begin{figure}[ht]
\centerline{\includegraphics[scale=0.5]{a_bcdeStef}}
\caption{Le même flux de données après l'exécution de la méthode \ct{next}: le caractère \ct{a} est ``dans le passé'' alors que \ct{b}, \ct{c}, \ct{d} and \ct{e} sont ``dans le futur''.}
\figlabel{a_bcde}
\vspace{.2in}
\end{figure}

Écrire un élément revient à remplacer le premier élément de la séquence future par le nouveau et le déplacer dans le passé. \Figref{ax_cde} montre l'état du même \stream après avoir écrit un \ct{x} via le message \mthind{Stream}{nextPut:} \ct{anElement}.

\begin{figure}[h!t]
\centerline{\includegraphics[scale=0.5]{ax_cdeStef}}
\caption{Le même flux de données après avoir écrit un \ct{x}.}
\figlabel{ax_cde}
\vspace{.2in}
\end{figure}

%=============================================================
\section{Streams contre Collections}

Le protocole des collections supporte le stockage, l'effacement et l'énumération
des éléments d'une collection mais il ne permet pas que ces opérations
soient combinées ensemble. Par exemple, si les éléments d'une 
\clsind{OrderedCollection} sont traités par une méthode \mthind{OrderedCollection}{do:}, il n'est pas possible d'ajouter ou d'enlever des éléments à l'intérieur du bloc \ct{do:}.
Ce protocole ne permet pas non plus d'itérer dans deux collections
en même temps en choisissant quelle collection on itère, laquelle on n'itère pas.
% ?choosing which collection goes forward
% Serge : oui la phrase n'est pas très claire ... Précision à demander sur la liste sbe
De telles procédures requièrent qu'un index de parcours ou une référence
de position soit maintenu hors de la collection elle-même:
c'est exactement le rôle de  
%traversal index or
\clsind{ReadStream} (pour la lecture), \clsind{WriteStream} (pour l'écriture) et \clsind{ReadWriteStream} (pour les deux).

Ces trois classes sont définies pour \emph{glisser à travers}~\footnote{En anglais, nous dirions ``stream over''.} une collection.
Par exemple, le code suivant crée un \stream sur un intervalle puis y lit deux éléments.
\needlines{5}
\begin{code}{@TEST |r|}
r := ReadStream on: (1 to: 1000).
r next.   --> 1
r next.   --> 2
r atEnd.--> false
\end{code}

Les \ct{WriteStream}{}s peuvent écrire des données dans la collection:
\begin{code}{@TEST |w|}
w := WriteStream on: (String new: 5).
w nextPut: $a.
w nextPut: $b.
w contents. -->  'ab'
\end{code}

Il est aussi possible de créer des \ct{ReadWriteStream}{}s qui supportent
les protocoles de lecture et d'écriture.

Le principal problème de \ct{WriteStream} et de \ct{ReadWriteStream}
est que, dans \pharo, ils ne supportent que les tableaux et les 
chaînes de caractères. Cette limitation est en cours de
disparition grâce au développement d'une nouvelle librairie
nommée \emph{Nile}~\footnote{Disponible à \url{www.squeaksource.com/Nile.html}}. 
mais en attendant, vous obtiendrez une erreur
si vous essayez d'utiliser les \streams avec un autre type de collection:
% REVOIR 

\needlines{3}
\begin{code}{}
w := WriteStream on: (OrderedCollection new: 20).
w nextPut: 12. -->  !\normcomment{lève une erreur}!
\end{code}

Les \streams ne sont pas seulement destinés aux collections mais
aussi aux fichiers et aux \emph{sockets}.
L'exemple suivant crée un fichier appelé \ct{test.txt}, 
y écrit deux chaînes de caractères, séparées par un retour-chariot et enfin ferme le fichier.

\needlines{3}
\begin{code}{}
StandardFileStream
  fileNamed: 'test.txt'
  do: [:str | str
                nextPutAll: '123';
                cr;
                nextPutAll: 'abcd'].
\end{code}
\cmindex{FileStream class}{fileNamed:do:}

Les sections suivantes s'attardent sur les protocoles.

%=============================================================
\section{Utiliser les streams avec les collections}

Les \streams sont vraiment utiles pour traiter des collections d'éléments.
Ils peuvent être utilisés pour la lecture et l'écriture d'éléments
dans des collections. Nous allons explorer maintenant les caractéristiques
des \streams dans le cadre des collections.

%-----------------------------------------------------------------
\subsection{Lire les collections}

Cette section présente les propriétés utilisées pour lire des collections. 
Utiliser les flux de données pour lire une collection repose essentiellement 
sur le fait de disposer d'un pointeur sur le contenu de la collection.
Vous pouvez placer où vous voulez ce pointeur qui avancera dans le contenu pour lire.
La classe \clsindmain{ReadStream} devrait être utilisée pour lire les éléments dans 
les collections.

Les méthodes \mthind{ReadStream}{next} et \mthind{ReadStream}{next:} 
sont utilisées pour récupérer un ou plusieurs éléments dans la collection.

\begin{code}{@TEST |stream|}
stream := ReadStream on: #(1 (a b c) false).
stream next. -->   1
stream next. -->   #(#a #b #c)
stream next. -->   false
\end{code}
\cmindex{PositionableStream class}{on:}

\begin{code}{@TEST |stream|}
stream := ReadStream on: 'abcdef'.
stream next: 0. -->   ''
stream next: 1. -->   'a'
stream next: 3. -->   'bcd'
stream next: 2. -->   'ef'
\end{code}

Le message \mthind{PositionableStream}{peek} est utilisé quand vous voulez
connaître l'élément suivant dans le \stream sans avancer dans le flux.

\begin{code}{@TEST |stream negative number|}
stream := ReadStream on: '-143'.
negative := (stream peek = $-).    "regardez le premier !élément! sans le lire"
negative. --> true
negative ifTrue: [stream next].       "ignore le !caractère! moins"
number := stream upToEnd.
number. --> '143'
\end{code}%$
\noindent
Ce code affecte la variable booléenne \ct{negative} en fonction du signe du nombre dans le \stream et \ct{number} est assigné à sa valeur absolue. 
La méthode \mbox{\mthind{ReadStream}{upToEnd}} (qui en français se traduirait par ``jusqu'à la fin'') renvoie tout depuis la position courante jusqu'à
la fin du flux de données et positionne ce dernier à sa fin.
Ce code peut être simplifié grâce à \mthind{PositionableStream}{peekFor:} qui déplace le pointeur si et seulement si l'élément est égal au paramètre passé en argument.

\begin{code}{@TEST |stream negative number|}
stream := '-143' readStream.
(stream peekFor: $-) --> true
stream upToEnd         --> '143'
\end{code}%$
\noindent
\ct{peekFor:} retourne aussi un booléen indiquant si le paramètre est égal à l'élément courant.

Vous avez dû remarquer une nouvelle façon de construire un \stream dans l'exemple précédent: vous pouvez simplement envoyer  
\mthind{SequenceableCollection}{readStream} à une collection séquentielle pour avoir un flux de données en lecture seule sur une collection.

\paragraph{Positionner.} Il existe des méthodes pour positionner le pointeur du \stream. Si vous connaissez l'emplacement, vous pouvez vous y rendre directement en utilisant \mthind{PositionableStream}{position:}. Vous pouvez demander la position actuelle avec \mthind{PositionableStream}{position}. Souvenez-vous bien 
qu'un \stream n'est pas positionné sur un élément, mais entre deux éléments. L'index 0 correspond au début du flux.

Vous pouvez obtenir l'état du \stream montré dans 
\figref{ab_cde} avec le code suivant:

\begin{code}{@TEST |stream|}
stream := 'abcde' readStream.
stream position: 2.
stream peek --> $c
\end{code}%$

\begin{figure}[h!t]
\centerline{\includegraphics[scale=0.5]{ab_cdeStef}}
\caption{Un flux de données à la position 2.}
\figlabel{ab_cde}
\vspace{.2in}
\end{figure}

Si vous voulez aller au début ou à la fin, vous pouvez utiliser
   %martial: signaler la tournure lourde dans la VO: If you want to go to the beginning or at the end is what you want, you can use
\mthind{PositionableStream}{reset} ou \mthind{PositionableStream}{setToEnd}.
Les messages \mthind{PositionableStream}{skip:} et \mthind{PositionableStream}{skipTo:} sont utilisés pour avancer d'une position relative à la position actuelle: la méthode \ct{skip:} accepte un nombre comme
argument et saute sur une distance de ce nombre d'éléments alors que
\ct{skipTo:} saute tous les éléments dans le flux jusqu'à trouver
un élément égal à son argument. Notez que cette méthode positionne
le \stream après l'élément identifié.

\begin{code}{@TEST |stream|}
stream := 'abcdef' readStream.
stream next.      --> $a    "!le flux est à la position juste après a!"
stream skip: 3.                           "le flux est !après! d"
stream position.  -->   4
stream skip: -2.                          "le flux est !après! b"
stream position.  --> 2
stream reset.
stream position.  --> 0
stream skipTo: $e.                        "le flux est !après! e"
stream next.        --> $f
stream contents. --> 'abcdef'
\end{code}%$

Comme vous pouvez le voir, la lettre \ct{e} a été sautée.

La méthode \mthind{PositionableStream}{contents} retourne toujours une copie de l'intégralité du flux de données.

\paragraph{Tester.} Certaines méthodes vous permettent de tester l'état d'un \stream courant: 
la méthode \mthind{PositionableStream}{atEnd} renvoie \emph{true} si et seulement si aucun élément ne peut être trouvé aprés la position actuelle alors que \mthind{PositionableStream}{isEmpty} renvoie \emph{true} si et seulement si aucun élément ne se trouve dans la collection.

Voici une implémentation possible d'un algorithme utilisant \ct{atEnd} et prenant deux collections triées comme paramètres puis les fusionnant dans une autre collection triée:

\needlines{4}
\begin{code}{@TEST |stream1 stream2 result|}
stream1 := #(1 4 9 11 12 13) readStream.
stream2 := #(1 2 3 4 5 10 13 14 15) readStream.

"!La variable résultante contiendra la collection triée.!"
result := OrderedCollection new.
[stream1 atEnd not & stream2 atEnd not]
  whileTrue: [stream1 peek < stream2 peek
    "!Enlève le plus petit élément de chaque flux et l'ajoute au résultat!"
    ifTrue: [result add: stream1 next]
    ifFalse: [result add: stream2 next]].

"!Un des deux flux peut ne pas être à la position finale. Copie ce qu'il reste!"
result
  addAll: stream1 upToEnd;
  addAll: stream2 upToEnd.

result. -->   an OrderedCollection(1 1 2 3 4 4 5 9 10 11 12 13 13 14 15)
\end{code}
%	from either stream and add it to the result."

%-----------------------------------------------------------------
\subsection{Écrire dans les collections}

Nous avons déjà vu comment lire une collection en itérant sur ses
éléments via un objet \ct{ReadStream}. Apprenons maintenant à créer
des collections avec la classe \clsindmain{WriteStream}.

Les flux de données \ct{WriteStream} sont utiles pour adjoindre des données en plusieurs endroits dans une collection. Ils sont souvent utilisés pour construire des chaînes de caractères basées sur des parties à la fois statiques et dynamiques comme dans l'exemple suivant:

\begin{code}{NB: can't be tested due to the changing number of classes}
stream := String new writeStream.
stream
  nextPutAll: 'Cette image Smalltalk contient: ';
  print: Smalltalk allClasses size;
  nextPutAll: ' classes.';
  cr;
  nextPutAll: 'C'est vraiment beaucoup.'.

stream contents. --> 'Cette image Smalltalk contient: 2322 classes. C'est vraiment beaucoup.'
\end{code}

Par exemple, cette technique est utilisée dans différentes 
implémentations de la méthode \ct{printOn:}. Il existe une manière
plus simple et plus efficace de créer des flux de données si vous êtes
seulement interessé au contenu du \stream:

\begin{code}{@TEST |string|}
string := String streamContents:
		[:stream |
			stream
                 print: #(1 2 3);
                 space;
                 nextPutAll: 'size';
                 space;
                 nextPut: $=;
                 space;
                 print: 3.	].
string. -->   '#(1 2 3) size = 3'
\end{code}%$

La méthode \mthind{SequenceableCollection class}{streamContents:} \seclabel{streamContents} crée une collection et un \stream sur cette collection.
Elle exécute ensuite le bloc que vous lui donnez en passant le \stream comme argument de bloc. Quand le bloc se termine, \ct{streamContents:}
renvoie le contenu de la collection.

Les méthodes de \ct{WriteStream} suivantes sont spécialement utiles dans ce contexte:

\begin{description}
\item[\lct{nextPut:}] ajoute le paramètre au flux de données;
\item[\lct{nextPutAll:}] ajoute chaque élément de la collection passé en argument au flux;
\item[\lct{print:}] ajoute la représentation textuelle du paramètre au flux.
	\cmindex{Stream}{print:}
\end{description}

Il existe aussi des méthodes utiles pour imprimer différentes sortes
de caractères au \stream comme \mthind{WriteStream}{space} (pour un espace), 
   \mthind{WriteStream}{tab} (pour une tabulation) et
   \mthind{WriteStream}{cr} (pour \emph{Carriage Return} \cad le retour-chariot).
Une autre méthode s'avère utile pour s'assurer que le dernier caractère
dans le flux de données est un espace: il s'agit de \mthind{WriteStream}{ensureASpace}; si le dernier caractère n'est pas un espace, il en ajoute un.

\paragraph{Au sujet de la concaténation.}
L'emploi de \mthind{WriteStream}{nextPut:} et de \mthind{WriteStream}{nextPutAll:} sur un \ct{WriteStream} est souvent le meilleur moyen pour concaténer 
les caractères.
L'utilisation de l'opérateur virgule (\ct{,}) est beaucoup moins efficace:
\index{Collection!opérateur virgule}

\begin{code}{}
[| temp |
  temp := String new.
  (1 to: 100000)
    do: [:i | temp := temp, i asString, ' ']] timeToRun --> 115176 "(ms)"

[| temp |
  temp := WriteStream on: String new.
  (1 to: 100000)
    do: [:i | temp nextPutAll: i asString; space].
  temp contents] timeToRun --> 1262 "(milliseconds)"
\end{code}

La raison pour laquelle l'usage d'un \stream est plus efficace provient
du fait que l'opérateur virgule crée une nouvelle chaîne de caractères
contenant la concaténation du receveur et de l'argument, donc il doit
les copier tous les deux.
Quand vous concaténez de manière répétée sur le même receveur,
ça prend de plus en plus de temps à chaque fois; le nombre
de caractères copiés s'accroît de façon exponentielle.
Cet opérateur implique aussi une surcharge de travail pour le ramasse-miettes qui collecte ces chaînes. 
% ajout de 'Pour ce cas' pour suggerer que ca ne concerne specialement les gros travaux sur les chaines, pour les petites accumulations, je suis assez d'accord avec lukas (a discuter dans la sbe-discussion)
Pour ce cas, utiliser un \stream plutôt qu'une concaténation de chaînes est une optimisation bien connue.
\lr{About Concatenation. This is not true for real world examples (the example benchmark is unrealistic). Most of the time concatenation is simpler, cleaner and much faster, for example when being used on a small number of longer strings. (p. 257)}
En fait, vous pouvez utiliser la méthode de classe \mthind{SequenceableCollection class}{streamContents:} (mentionnée à la page \pageref{sec:streamContents}) pour parvenir à ceci:

\begin{code}{}
String streamContents: [ :tempStream |
  (1 to: 100000)
       do: [:i | tempStream nextPutAll: i asString; space]] 
\end{code}

%-----------------------------------------------------------------
\subsection{Lire et écrire en même temps}

Vous pouvez utiliser un flux de données pour accéder à une collection
en lecture et en écriture en même temps.
Imaginez que vous voulez créer une classe d'historique que nous appelerons \ct{History} et qui gérera
les boutons ``Retour'' (\emph{Back}) et ``Avant'' (\emph{Forward}) d'un
navigateur web.
Un historique réagirait comme le montrent les illustrations depuis \ref{fig:emptyStream} jusqu'à \ref{fig:page4}.

\begin{figure}[!ht]
\centerline{\includegraphics[scale=0.5]{emptyStef}}
\caption{Un nouvel historique est vide. Rien n'est affiché dans le navigateur web.}
\figlabel{emptyStream}
\vspace{.2in}
\end{figure}

\begin{figure}[!ht]
\centerline{\includegraphics[scale=0.5]{page1Stef}}
\caption{L'utilisateur ouvre la page 1.}
\figlabel{page1}
\vspace{.2in}
\end{figure}

\begin{figure}[!ht]
\centerline{\includegraphics[scale=0.5]{page2Stef}}
\caption{L'utilisateur clique sur un lien vers la page 2.}
\figlabel{page2}
\vspace{.2in}
\end{figure}

\begin{figure}[!ht]
\centerline{\includegraphics[scale=0.5]{page3Stef}}
\caption{L'utilisateur clique sur un lien vers la page 3.}
\figlabel{page3}
\vspace{.2in}
\end{figure}

\begin{figure}[!ht]
\centerline{\includegraphics[scale=0.5]{page2_Stef}}
\caption{L'utilisateur clique sur le bouton ``Retour'' (Back). Il visite désormais la page 2 à nouveau.}
\figlabel{page2_}
\vspace{.2in}
\end{figure}

\begin{figure}[!ht]
\centerline{\includegraphics[scale=0.5]{page1_Stef}}
\caption{L'utilisateur clique sur le bouton ``Retour'' (Back). La page 1 est affichée maintenant.}
\figlabel{page1_}
\vspace{.2in}
\end{figure}

\begin{figure}[!ht]
\centerline{\includegraphics[scale=0.5]{page4Stef}}
\caption{Depuis la page 1, l'utilisateur clique sur un lien vers la page 4. L'historique oublie les pages 2 et 3.}
\figlabel{page4}
\vspace{.2in}
\end{figure}

Ce comportement peut être programmé avec un \clsind{ReadWriteStream}.

\needlines{6}
\begin{code}{}
Object subclass: #History
  instanceVariableNames: 'stream'
  classVariableNames: ''
  poolDictionaries: ''
  category: 'PBE-Streams'

History>>initialize
    super initialize.
    stream := ReadWriteStream on: Array new.
\end{code}

Nous n'avons rien de compliqué ici; nous définissons une nouvelle classe
qui contient un \stream. Ce \stream est créé dans la méthode \ct{initialize} 
%ajout
depuis un tableau.

Nous avons besoin d'ajouter les méthodes \ct{goBackward} et \ct{goForward} pour aller respectivement en arrière (``Retour'') et en avant:

\begin{code}{}
History>>goBackward
  self canGoBackward ifFalse: [self error: '!\normcode{Déjà sur le premier élément}!'].
  stream skip: -2. 
  ^ stream next

History>>goForward
  self canGoForward ifFalse: [self error: '!\normcode{Déjà sur le dernier élément}!'].
  ^ stream next
\end{code}

Jusqu'ici le code est assez simple. Maintenant, nous devons nous occuper
de la méthode \ct{goTo:} (que nous pouvons traduire en français par ``aller à'') qui devrait être activée quand l'utilisateur clique sur un lien. Une solution possible est la suivante:

\begin{code}{}
History>>goTo: aPage
    stream nextPut: aPage.
\end{code}

Cette version est cependant incomplète. Ceci vient du fait que lorsque l'utilisateur clique sur un lien, il ne devrait plus y avoir de pages futurs \ie que le bouton ``Avant'' devrait être désactivé.
Pour ce faire, la solution la plus simple est d'écrire \ct{nil}
juste après la position courante pour indiquer la fin de l'historique:

\begin{code}{}
History>>goTo: anObject
  stream nextPut: anObject.
  stream nextPut: nil.
  stream back.
\end{code}

Maintenant, seules les méthodes \ct{canGoBackward} (pour dire si oui ou non nous pouvons aller en arrière) et \ct{canGoForward} (pour dire si oui ou non nous pouvons aller en avant) sont à coder.

Un flux de données est toujours positionné entre deux éléments.
Pour aller en arrière, il doit y avoir deux pages avant la position courante:
une est la page actuelle et l'autre est la page que nous voulons atteindre.

\begin{code}{}
History>>canGoBackward
  ^ stream position > 1

History>>canGoForward
  ^ stream atEnd not and: [stream peek notNil]
\end{code}

Ajoutons pour finir une méthode pour accéder au contenu du \stream:
\begin{code}{}
History>>contents
  ^ stream contents
\end{code}

Faisons fonctionner maintenant notre historique 
%ajout (plus claire)
comme dans la séquence illustrée plus haut:
\begin{code}{}
History new
	goTo: #page1;
	goTo: #page2;
	goTo: #page3;
	goBackward;
	goBackward;
	goTo: #page4;
	contents --> #(#page1 #page4 nil nil)
\end{code}

%=============================================================
\section{Utiliser les streams pour accéder aux fichiers}

Vous avez déjà vu comment glisser sur une collection d'éléments via
un \stream. Il est aussi possible d'en faire de même avec un flux 
sur des fichiers de votre disque dur.
Une fois créé, un \stream sur un fichier est comme un \stream sur
une collection: vous pourrez utiliser le même protocole pour lire, écrire
ou positionner le flux.
La principale différence apparaît à la création du flux de données.
Nous allons voir qu'il existe plusieurs manières de créer un \stream sur un fichier.

%-----------------------------------------------------------------
\subsection{Créer un flux pour fichier}
\seclabel{creat-file-stre}

Créer un \stream sur un fichier consiste à utiliser une des méthodes
de création d'instance suivantes mises à disposition par la classe
\clsindmain{FileStream}:

\begin{description}

\item[\lct{fileNamed:}] ouvre en lecture et en écriture un fichier 
  avec le nom donné. Si le fichier existe déjà, son contenu pourra
  être modifié ou remplacé mais le fichier ne sera pas tronqué
  à la fermeture. Si le nom n'a pas de chemin spécifié pour répertoire,
  le fichier sera créé dans le répertoire par défaut.
  \cmindex{FileStream class}{fileNamed:}
  
\item[\lct{newFileNamed:}] crée un nouveau fichier avec le nom donné
	et retourne un \stream ouvert en écriture pour ce fichier.
	Si le fichier existe déjà, il est demandé à l'utilisateur
	de choisir la marche à suivre.
  \cmindex{FileStream class}{newFileNamed:}
  
\item[\lct{forceNewFileNamed:}] crée un nouveau fichier avec le nom donné
	et répond un \stream ouvert en écriture sur ce fichier.
	Si le fichier existe déjà, il sera effacé avant qu'un nouveau
	ne soit créé.
  \cmindex{FileStream class}{forceNewFileNamed:}

\item[\lct{oldFileNamed:}] ouvre en lecture et en écriture un fichier 
	existant avec le nom donné. Si le fichier existe déjà, son 
	contenu pourra être modifié ou remplacé mais le fichier ne sera
	pas tronqué à la fermeture. Si le nom n'a pas de chemin spécifié
	pour répertoire, le fichier sera créé dans le répertoire par
	défaut.
  \cmindex{FileStream class}{oldFileNamed:}

\item[\lct{readOnlyFileNamed:}] ouvre en lecture seule un fichier 
	existant avec le nom donné.
  \cmindex{FileStream class}{readOnlyFileNamed:}

\end{description}

Vous devez vous remémorer de fermer le \stream sur le fichier que vous avez ouvert. Ceci se fait grâce à la méthode \mthind{FileStream}{close}.

\begin{code}{@TEST |stream|}
stream := FileStream forceNewFileNamed: 'test.txt'.
stream
    nextPutAll: '!\normcode{Ce texte est écrit dans un fichier nommé}! ';
    print: stream localName.
stream close.

stream := FileStream readOnlyFileNamed: 'test.txt'.
stream contents. --> '!\normcode{Ce fichier est écrit dans un fichier nommé}! ''test.txt'''
stream close.
\end{code}

% \on{need way to clean up test files afterwards}

%[:fileName | (FileDirectory forFileName: fileName)
%	deleteFileNamed: fileName
%	ifAbsent: [ 'Could not delete ', fileName ] ]
%	value: 'test.txt'

La méthode \mthind{FileStream}{localName} retourne le dernier composant du nom du fichier. 
Vous pouvez accéder au chemin entier en utilisant la méthode
\mthind{StandardFileStream}{fullName}.

Vous remarquerez bientôt que la fermeture manuelle de \stream de fichier
est pénible et source d'erreurs. C'est pourquoi \ct{FileStream}
offre un message appelé \mthind{FileStream class}{forceNewFileNamed:do:} 
pour fermer automatiquement un nouveau flux de données après
avoir évalué un bloc qui modifie son contenu.

\begin{code}{@TEST |string|}
FileStream
    forceNewFileNamed: 'test.txt'
    do: [:stream |
        stream
            nextPutAll: '!\normcode{Ce texte est écrit dans un fichier nommé}! ';
            print: stream localName].
string := FileStream
            readOnlyFileNamed: 'test.txt'
            do: [:stream | stream contents].
string --> '!\normcode{Ce fichier est écrit dans un fichier nommé}! ''test.txt'''
\end{code}

Les méthodes de création de flux de données prenant un bloc comme
argument créent d'abord un \stream sur un fichier, puis exécute
un argument et enfin ferme le \stream. Ces méthodes retournent ce qui est retourné par 
le bloc, \ie la valeur de la dernière
expression dans le bloc. C'est ce que nous avons utilisé dans
l'exemple précédent pour récupérer le contenu d'un fichier
et le mettre dans la variable \ct{string}.

%-----------------------------------------------------------------
\subsection{Les flux binaires}
\seclabel{binary-streams}

Par défaut, les \streams créés sont à base textuelle ce qui signifie
que vous lirez et écrirez des caractères.
Si votre flux doit être binaire, vous devez lui envoyer le message 
\mthind{FileStream}{binary}.

Quand votre \stream est en mode binaire, vous pouvez seulement écrire
des nombres de 0 à 255 (ce qui correspond à un octet). Si
vous voulez utiliser \ct{nextPutAll:} pour écrire plus d'un
nombre à la fois, vous devez passer comme
argument un tableau d'octets de la classe \clsind{ByteArray}.

\begin{code}{@TEST}
FileStream
  forceNewFileNamed: 'test.bin'
  do: [:stream |
          stream
            binary;
            nextPutAll: #(145 250 139 98) asByteArray].

FileStream
  readOnlyFileNamed: 'test.bin'
  do: [:stream |
          stream binary.
          stream size.         --> 4
          stream next.         --> 145
          stream upToEnd. --> #[250 139 98]
      ].
\end{code}

Voici un autre exemple créant une image dans un fichier nommé
``test.pgm'' que vous pourrez ouvrir avec votre outil graphique
préféré.

% The following does not assert anything, but @TEST is used to ensure
% that no error is thrown.
\begin{code}{@TEST}
FileStream
  forceNewFileNamed: 'test.pgm' 
  do: [:stream |
	stream
		nextPutAll: 'P5'; cr;
		nextPutAll: '4 4'; cr;
		nextPutAll: '255'; cr;
		binary;
		nextPutAll: #(255 0 255 0) asByteArray;
		nextPutAll: #(0 255 0 255) asByteArray;
		nextPutAll: #(255 0 255 0) asByteArray;
		nextPutAll: #(0 255 0 255) asByteArray
	]
\end{code}

Cela crée un échiquier 4 par 4 comme nous montre \figref{checkerboard4x4}.

\begin{figure}[!ht]
\centerline{\includegraphics[width=0.25\textwidth]{checkerboard4x4}}
\caption{Un échiquier 4 par 4 que vous pouvez dessiner en utilisant des \streams binaires.}
\figlabel{checkerboard4x4}
\vspace{.2in}
\end{figure}

%=============================================================
\section{Résumé du chapitre}

Par rapport aux collections, les flux de données ou \streams offrent
un bien meilleur moyen de lire et d'écrire de manière
incrémentale dans une séquence d'éléments.
Il est très facile de passer par conversion de \streams à collections et vice-versa.

\begin{itemize}
  \item Les flux peuvent être soit en lecture, soit en écriture, soit à la fois en lecture-écriture.
  \item Pour convertir une collection en un \stream, définissez un \stream
sur une collection grâce au message \ct{on:}, \eg \ct{ReadStream on: (1 to: 1000)}, ou via les messages \ct{readStream}, \etc ... sur la collection.
  \item Pour convertir un \stream en collection, envoyer le message \ct{contents}.
  \item Pour concaténer des grandes collections, il est plus efficace d'abandonner l'emploi de l'opérateur virgule \ct{,} et de créer un \stream et y adjoindre les collections avec le message \ct{nextPutAll:} puis extraire enfin le résultat en lui envoyant \ct{contents}.
  \item Par défaut, les \streams de fichiers sont à base de caractères.
Envoyer le message \ct{binary} en fait explicitement des \streams binaires.
\end{itemize}

%=================================================================
\ifx\wholebook\relax\else\end{document}\fi
%=================================================================

%-----------------------------------------------------------------

%%% Local Variables: 
%%% coding: utf-8
%%% mode: latex
%%% TeX-master: t
%%% TeX-PDF-mode: t
%%% End:

%:Morphic
%%% Morphic.tex --- 
%% 
%% Filename: Morphic.tex
%% Description: Traduction de la page Morphic.tex de Squeak By Example
%% + copier coller de l'article originale + ajout du maintainer
%% Author: Andrew Black
%% Maintainer: Martial Boniou
%% Created: 2007-11-09 08:07:45 +0100 (Fri, 09 Nov 2007)
%% Version: 13642
%% Last-Updated: Sat Sep 12 20:31:51 2009 (+0200)
%%           By: Martial Boniou
%%     Update #: 763
%% URL:
%% https://www.iam.unibe.ch/scg/svn_repos/SqueakByExample/FrenchBook/Morphic/Morphic.tex
%% Review: Rene Mages - Fri Dec 21 22:18:36 2007 (3600 CET)
%% Review: Rene Mages - Sat Jan 12 18:55:44 2008 (3600 CET)
%% Keywords: 
%% Compatibility: 
%% 
%%%%%%%%%%%%%%%%%%%%%%%%%%%%%%%%%%%%%%%%%%%%%%%%%%%%%%%%%%%%%%%%%%%%%%
%% 
%%% Commentary:
%%             - passage de premier paragraphe sur l'histoire de
%%             Morphic dans l'introduction 
%%             - changement de titre du premier chapitre 'premiere
%%             immersion...' +  ajouts de trois sous-sections pour
%%             aerer le document   
%%             - ajout dans la sous-section 'un monde de morphs' de
%%             deux paragraphes sur le caractere introspectif du
%%             Viewer + l'image du StringMorph qui pivote
%%             - ajout d'un tableau avec les commandes les plus
%%             communes des morphs
%%             - Dans le chapitre 'Interaction et animation', nous
%%             avons les sous-sections 'evenements souris' 'evenements
%%             clavier' 'animations Morphic'; j'ai donc ajoute pour la
%%             coherence avec le titre du chapitre l'ancien chapitre
%%             'interacteurs' comme sous-section.
%%             - redecoupe du 'resume du chapitre'; les *item*
%%             commencent tous par un verbe infinitif. Par rapport a
%%             la version d'Andrew Black, j'ai inserer dans la liste
%%             le fait de pouvoir trouver des morphs predefinis (avant-
%%             dernier *item*) et j'ai ajoute un *item* sur les
%%             methodes graphiques des canevas (derniere chapitre)
%%             
%%             D'une maniere generale tres peu de copier-coller du
%%             texte d'Hilaire Fernandes; Andrew a deja fait beaucoup
%%             de corrections. Reste la structure centrale des
%%             chapitres, les exemples et les graphiques associes.
%% 
%%%%%%%%%%%%%%%%%%%%%%%%%%%%%%%%%%%%%%%%%%%%%%%%%%%%%%%%%%%%%%%%%%%%%%
%% 
%%% Change log: 
%%             - Traduction et figures par Martial
%%             - Premiere et deuxième relecture par Rene 
%% 
%%%%%%%%%%%%%%%%%%%%%%%%%%%%%%%%%%%%%%%%%%%%%%%%%%%%%%%%%%%%%%%%%%%%%%
%=================================================================
\ifx\wholebook\relax\else
% --------------------------------------------
% Lulu:
	\documentclass[a4paper,10pt,twoside]{book}
	\usepackage[
		papersize={6.13in,9.21in},
		hmargin={.75in,.75in},
		vmargin={.75in,1in},
		ignoreheadfoot
	]{geometry}
	\input{../common.tex}
	\pagestyle{headings}
	\setboolean{lulu}{true}
% --------------------------------------------
% A4:
%	\documentclass[a4paper,11pt,twoside]{book}
%	\input{../common.tex}
%	\usepackage{a4wide}
% --------------------------------------------
    \graphicspath{{figures/} {../figures/}}
	\begin{document}
	\renewcommand{\nnbb}[2]{} % Disable editorial comments
	\sloppy
\fi
%=================================================================
\chapter{L'interface Morphic}

\tradalert{martial}{le reste du texte est encore du SBE}

%\sd{We should first give a conceptual overview.
%Then we need a cookbook of how to do simple things in Morphic.
%The observer pattern and its implementation with changed:  and update: messages could go here.  Or in ``Idiomatic design patterns'' later.}

%
\indmain{Morphic} est le nom de l'interface graphique de \pharo.
Elle est \'ecrite en \st, donc elle est pleinement portable entre
diff\'erents syst\`emes d'exploitation; en cons\'equence de quoi, \pharo
a le m\^eme aspect sur Unix, \macosx et Windows.
L'absence de distingo entre \emph{composition} et \emph{ex\'ecution}
de l'interface est la principale divergence de Morphic avec la plupart
des autres bo\^{\i}tes \`a outils graphiques: tous ses \'el\'ements
graphiques peuvent \^etre assembl\'es et d\'esassembl\'es \`a tout
moment par l'utilisateur.
%note de martial: il faudra remettre cette note lorsque nous serons
%fixes sur les droits. Je pense qu'il vaudrait mieux le remercier
%plus precisement dans le chapitre Preface (cad en precisant que c'est
%lui qui est a l'origine de ce chapitre.
%\footnote{We thank Hilaire Fernandes for permission to base this chapter on his original article in French.}

\ab{After the first printing, I took an editing pass, correcting some errors and grammatical infelicities.}

\on{I have commented out the LabelstickerMorph and PyramidMorph examples, as they do not really add much over the other examples we have already. The source code is now available in the example subdirectory, in case someone would like to try and use them after all.}

%martial: je prefere mettre la partie historique en intro car elle est
%trop courte; de plus le paragraphe porte mal son nom
Morphic a \'et\'e d\'evelopp\'ee par John Maloney et Randy Smith pour
le langage de programmation orient\'e objet \ind{Self} d\'evelopp\'e
chez Sun Microsystems:
%ajout
l'interface de ce langage bas\'e sur le concept de prototypes (comme
JavaScript) est apparue en 1993.
Maloney r\'e\'ecrivit ensuite une nouvelle version de Morphic pour
\pharo tout en conservant de la version originale son aspect 
%directness = franchise
\emph{direct} et 
%live(li)ness = vivacite
\emph{vivant}.
%ajout
Dans ce chapitre, nous ferons une immersion dans cet univers d'objets
graphiques, les \emph{morphs} et nous apprendrons \`a les modeler (\`a la
souris ou en programmation), \`a leur ajouter des fonctionnalit\'es (pour
accro\^{\i}tre leur capacit\'e d'interaction) et enfin, en pr\'eambule
d'un exemple complet, nous verrons comment il s'int\`egre non seulement
dans l'espace mais aussi dans le temps.

%=================================================================
\section{Premi\`ere immersion dans Morphic}

\subsection{R\'eponse au doigt et \`a l'\oe il}
Le caract\`ere direct de l'interface Morphic se traduit par le fait
que toutes les formes graphiques sont des objets 
%examined or chapnged directly
inspectables et modifiables directement par la souris.
%l'interactivit\'e offerte

De plus, le fait que toute action faite par l'utilisateur donne lieu
\`a une r\'eponse de la part de Morphic d\'efinit son caract\`ere
vivant: les informations affich\'ees sont constamment mise \`a jour
au fur et \`a mesure des changements du ``monde'' que l'interface
d\'ecrit. 
Comme preuve de cette vie et de toute la dynamique qui en r\'esulte,
nous vous proposons d'isoler un \'el\'ement du menu World et de vous
en faire un bouton hors du menu.

\dothis{Afficher le menu World. Cliquez une premi\`ere fois avec le
  \ind{bouton bleu} de la souris de mani\`ere \`a afficher le
  \emph{halo} Morphic. Recliquez \`a nouveau avec le bouton bleu mais
  cette fois-ci, placez le pointeur de la souris au-dessus de l'option
  de menu que vous voulez d\'etacher, disons \menu{new
    morph\ldots}. Le halo est donc affich\'e pour cet \'el\'ement du
  menu. D\'eplacez celui-ci n'importe o\`u sur l'\'ecran en glissant
  la poign\'ee noire \grabHandle{},
 comme le montre \figref{detachingMenu}.}
\index{Morphic!halo}

\begin{figure}[ht]
	\ifluluelse
		{\centerline{\includegraphics[width=0.6\textwidth]{detachingMenu}}}
		{\centerline{\includegraphics[width=0.4\textwidth]{detachingMenu}}}
	\caption{D\'etacher l'option de menu \menu{new morph\ldots} pour
      en faire un bouton ind\'ependant.\label{fig:detachingMenu}}
\end{figure}

\subsection{Un monde de morphs}
Tous les objets que vous voyez \`a l'\'ecran dans \pharo sont des
\emph{morphs}; tous sont des instances des sous-classes de \ct{Morph}.
\mbox{\ct{Morph}} est une grande classe avec de nombreuses m\'ethodes
qui permettent d'impl\'ementer des sous-classes avec un comportement
original avec tr\`es peu de code.
Vous pouvez cr\'eer un morph pour repr\'esenter n'importe quel objet. 
%, although how good a representation you get depends on the object!

\dothis{Pour cr\'eer un morph repr\'esentant une cha\^{\i}ne de
  caract\`eres, \'evaluer le code suivant dans un espace de travail,
  une ligne \`a la fois.}

\begin{code}{}
s := 'Morph' asMorph openInWorld.
s openViewerForArgument
\end{code}
\cmindex{Morph}{openInWorld}

La premi\`ere ligne cr\'ee un morph pour repr\'esenter la cha\^{\i}ne
de caract\`eres \ct{'Morph'} et l'affiche dans
l'\'ecran principal, le ``world'' (en fran\c{c}ais, nous dirions
``monde'' puisque la fen\^etre \pharo est un \emph{monde de morphs}).
Vous pouvez manipuler cet objet graphique gr\^ace au \ind{bouton rouge}
de la souris.
La seconde ligne ouvre un \ind{visualiseur de commandes} (ou
\emph{Viewer}) sous la forme d'un onglet
permettant la manipulation graphique de ce morph. Vous pouvez y voir
ses attributs tels que ses coordonn\'ees \ct{x} et \ct{y} dans
l'espace du \emph{world} dans lequel le morph est apparu.
Cliquer sur une ic\^one de point d'exclamation jaune sur noir envoie un
message au morph qui r\'epond promptement.
\seeindex{Viewer}{visualiseur de commandes}
\index{world}

%ajout massif
Par exemple, vous pouvez s\'electionner dans le menu de la bo\^{\i}te de
contr\^oles inf\'erieure de ce visualiseur de commandes l'option de
mouvement nomm\'ee \ct{motion} comme le montre \figref{turningInViewer}.
Remarquez que le titre du visualiseur n'est pas le nom de la classe
(ici \mbox{\ct{StringMorph})} mais le nom du morph; comme notre morph n'a pas
de nom, le visualiseur lui trouve un nom de morph en fonction de son
origine: \ct{String}. Si vous aviez l'id\'ee de cr\'eer un second
morph en reprenant le code plus haut et en repla\c{c}ant, par exemple,
le \ct{s} par un \ct{t}, vous obtiendrez certainement \ct{String1} comme nom
temporairement attribu\'e.

Nous partons ici du principe que le nom du morph est \ct{String}.

\dothis{Cliquez le point d'exclamation de la commande \ct{String turn
    by 5} dans la bo\^{\i}te de contr\^ole inf\'erieure du visualiseur
  de commandes. Vous pouvez voir le morph tourner sur son axe d'un
  angle de 5 degr\'es. Cliquez plusieurs fois pour faire faire un tour
  complet au Morph. Observez les valeurs de coordonn\'ees se mettre
  \`a jour \`a chaque clic.}

Vous pouvez vous amuser avec diff\'erentes commandes. Changez l'angle
avec les fl\`eches haut et bas. Cliquez sur \ct{String forward by 5}
pour faire avancer votre morph de 5 pixels. Faites appara\^{\i}tre le halo
Morphic via le \ind{bouton jaune} de la souris et manipuler votre
morph en cliquant sur les poign\'ees \grabHandle{} et \rotateHandle{}.
Remarquez les coordonn\'ees \'evoluer dans le visualiseur de commandes.

\begin{figure}[ht]
	\ifluluelse
		{\centerline{\includegraphics[width=0.6\textwidth]{turningInViewer}}}
		{\centerline{\includegraphics[width=0.4\textwidth]{turningInViewer}}}
	\caption{Tourner le morph gr\^ace au visualiseur de commandes.\label{fig:turningInViewer}}
\end{figure}

Pour fermer le visualiseur, cliquez sur le bouton en forme de cercle
en haut \`a gauche. Pour faire appara\^{\i}tre ce visualiseur de
commandes, vous pouvez aussi utiliser le halo Morphic et cliquer sur
la poign\'ee nomm\'ee Viewer \viewerHandle{}.
   
Notez que vous pouvez ranger le visualiseur au m\^eme titre que les
fen\^etres; pour ce faire, cliquez sur l'ic\^one miniature coll\'ee \`a
gauche du visualiseur. Pour rappeler le visualiseur, vous n'aurez
qu'\`a cliquer sur son ic\^one plac\'ee sur le bord droit du \emph{world}.

\subsection{Personnaliser sa repr\'esentation}

Revenons maintenant au code qui a cr\'e\'e ce morph.% 
%fin ajout massif
%ces deux lignes ne sont pas traduites!
%%%%Il est possible de d\'efinir des morphs plus int\'eressants en tant
%%%%%que repr\'esentation graphique. 
%ajout et transformation (plus clair) (j'ai mis aussi orange a la
%place de tan, comme ca il n'y a pas a traduire)
Tout repose sur la m\'ethode qui fabrique un morph \`a partir d'une
cha\^{\i}ne de caract\`eres:
cette m\'ethode \mthind{String}{asMorph} impl\'ement\'ee dans
\ct{String} cr\'ee un \ct{StringMorph}. \mthind{Object}{asMorph} est
impl\'ement\'ee par d\'efaut dans \ct{Object} donc tout objet peut
\^etre repr\'esent\'e par un morph. En r\'ealit\'e, la m\'ethode
\ct{asMorph} dans \ct{Object} fait appel \`a sa m\'ethode d\'eriv\'ee
dans \ct{String}. Ainsi, tant qu'une classe n'a pas surcharg\'ee cette
m\'ethode, elle sera repr\'esent\'ee par un \ct{StringMorph}.
Par exemple, \'evaluer \ct{Color orange asMorph openInWorld} ouvrira un
\ct{StringMorph} dont le label sera le r\'esultat de \clsind{Color}
\ct{orange printString} (comme en faisant un \short{p} sur \ct{Color orange}
dans un Workspace).
Voyons comment obtenir un rectangle de couleur plut\^ot que ce
\ct{StringMorph}.
%The method \mthind{Object}{asMorph} has a default implementation in class \ct{Object} class that just creates a StringMorph.
%So, for example, \ct{Color tan asMorph} returns a StringMorph labeled with the result of \clsind{Color} \ct{tan printString}.
%Let's change this so that we get a coloured rectangle instead.

\dothis{Ouvrez un navigateur de classes sur la classe \ct{Color} et
  ajoute la m\'ethode suivante
%ajout
dans le protocole \ct{creation}:}
\begin{method}{Obtenir un morph d'une instance de \ct{Color}}
Color>>>asMorph
	^ Morph new color: self
\end{method}
\noindent
Ex\'ecutez \ct{Color blue asMorph} \mthind{Morph}{openInWorld} dans un
espace de travail. Fini le texte d'affichage \ct{printString}! Vous
obtenez un joli rectangle bleu.


%=================================================================
\section{Manipuler les morphs}

Puisque les morphs sont des objets, nous pouvons les manipuler comme
n'importe quel autre objet dans \st \cad par envoi de messages. D\`es
lors nous pouvons entre autre changer leurs propri\'et\'es ou cr\'eer
de nouvelles sous-classes de \ct{Morph}.

Qu'il soit affich\'e \`a l'\'ecran ou non, tout morph a une position
et une taille. Tous les morphs sont inclus, par commodit\'e, dans une
bo\^{\i}te englobante, \cad une r\'egion rectangulaire occupant un
certain espace de l'\'ecran. Dans le cas des formes irr\'eguli\`eres,
leur position et leur taille correspondent \`a celles du plus petit
rectangle qui englobe la forme. Cette bo\^{\i}te englobante d\'efinit
les limites (ou \emph{bounds}) du morph. % ou frontiere?
La m\'ethode \mthind{Morph}{position} retourne un \ct{Point} qui
d\'ecrit la position du coin sup\'erieur gauche du morph (\cad le coin
sup\'erieur gauche de sa bo\^{\i}te englobante).
L'origine des coordonn\'ees du syst\`eme est le coin sup\'erieur
gauche de l'\'ecran: la valeur de la coordonn\'ee $y$ augmente
\emph{en descendant} l'\'ecran et la valeur de $x$ augmente en allant
de gauche \`a droite.
La m\'ethode \ct{extent} renvoie aussi un point, mais ce point
d\'efinit la largeur et la hauteur du morph plut\^ot qu'une position.

\dothis{Entrez le code suivant dans un espace de travail et
  \'evaluez-le (\menu{do it}):}
%%% futur?
%%%bill etait rouge (red); il devient vert (green) pour les raisons
%%%evoquees plus loin (contraste)
\begin{code}{}
joe := Morph new color: Color blue.
joe openInWorld.
bill := Morph new color: Color red.
bill openInWorld.
\end{code}
\noindent

%ajout
Ce code affiche deux nouveaux morphs r\'epondant aux noms de joe et
bill: par d\'efaut, un morph appara\^{\i}t comme  un rectangle de
position (0@0) et de taille (50@40).
Saisissez ensuite \ct{joe position} et affichez son r\'esultat par \menu{print it}.
Pour d\'eplacer joe, ex\'ecutez \ct{joe position: (joe position + (10@3))} plusieurs fois.
Vous pouvez modifier la taille aussi. Pour avoir la taille de joe,
vous pouvez \'evaluer par \menu{print it} l'expression \ct{joe}
\mthind{Morph}{extent}. Pour le faire grandir, ex\'ecutez \ct{joe extent: (joe extent * 1.1)}.
Pour changer la couleur d'un morph, envoyez-lui le message
\mthind{Morph}{color:} avec en argument un objet de classe \ct{Color},
correspondant \`a la couleur d\'esir\'ee. Par exemple,
\ct{joe color: Color orange}.
Pour ajouter la transparence, essayez
\ct{joe color: (Color blue alpha: 0.5)}.
%martial: a l'origine, c'est orange et pas blue (voir figure)
%%%martial: dans la version d'Andrew (SBE), joe est orange transparent ici et
%%%bleu transparent sur la figure: donc je prefere que joe soit orange
%%%transparent sur la figure; c'est pour le contraste que bill devient
%%%vert

\dothis{Pour faire en sorte que bill suive joe, vous pouvez ex\'ecuter
  ce code de mani\`ere r\'ep\'et\'ee:}
\begin{code}{}
bill position: (joe position + (100@0))
\end{code}
\noindent

Si vous d\'eplacez joe avec la souris et que vous ex\'ecutez ce code,
bill se d\'eplacera pour se positionner \`a 100 pixels \`a droite de joe.
\ab{It would seem that this would be a good place to introduce the \ct{step} method}

%=================================================================
\section{Composer des morphs}

Cr\'eer de nouvelles repr\'esentations graphiques peut se faire en
pla\c{c}ant un morph \`a l'int\'erieur d'un autre. C'est ce que nous
appelons la \emph{composition}; les morphs peuvent \^etre compos\'es
\`a l'infini. % to any depth
Pour ce faire, vous pouvez envoyer au morph contenant le message
\mthind{Morph}{addMorph:}. %container
\index{morph!composer}
\seeindex{morph!sous-morph}{sous-morph}
%
%To create new morphs, there are two main techniques that you can combine:
%\begin{enumerate}
%	\item by composing morphs one into another,
%	\item by subclassing \ct{Morph} and overriding \mthind{Morph}{drawOn:} to draw original morph shapes.
%\end{enumerate}
%}

\dothis{Ajoutez un morph \`a un autre avec le code suivant:}
\begin{code}{}
star := StarMorph new color: Color yellow.
joe addMorph: star.
star position: joe position.
\end{code}

\noindent
La derni\`ere ligne place l'\'etoile nomm\'ee star aux m\^emes
coordonn\'ees que joe. Notez que les coordonn\'ees du morph
contenu sont toujours \`a la position absolue d\'efinie par rapport
\`a l'\'ecran, et non \`a la position relative d\'efinie par rapport
au morph contenant.
Plusieurs m\'ethodes sont disponibles pour positionner un morph;
naviguez dans les m\'ethodes du protocole \protind{geometry} de la
classe \ct{Morph} pour le constater vous-m\^eme.
Par exemple, centrer l'\'etoile dans joe revient \`a ex\'ecuter
\ct{Star} \mthind{Morph}{center:} \ct{joe center}.

\begin{figure}[ht]
	\centerline{\includegraphics{joeStar}}
	\caption{L'\'etoile de classe StarMorph est contenue dans joe, le
      morph bleu translucide.\label{fig:joeStar}}
\end{figure}
%%%orange

Si vous attrapez l'\'etoile avec la souris, vous constaterez que vous
prenez en r\'ealit\'e joe et que les deux morphs sont ensemble:
l'\'etoile est \emph{incluse} %embedded inside a traduit par inclus dans
\`a l'int\'erieur de joe.
Il est possible d'inclure plus de morphs dans joe. %
%ajout
Les morphs inclus sont appel\'es des \ind{sous-morph}{}s (en anglais,
\emph{submorphs}). %
%la phrase est tourne differement dans SBE
Comme l'interface Morphic propose une interactivit\'e directe pour
tout morph, nous pouvons aussi faire notre inclusion de morphs en
rempla\c{c}ant la programmation par une simple manipulation \`a la souris.
%In addition to doing this programatically, you can also embed morphs by direct manipulation.
\seeindex{submorph}{sous-morph}

%martial: L'etoile est dite blanche mais elle est bleu claire en
%fait. De plus, il faut dire que certaines images n'ont plus du tout
%de supplies; donc j'ajoute le fait que ces morphs sont aussi dans Objects.
\dothis{Si vous n'avez pas d'onglet \menu{Supplies} (ou \menu{Accessoires})
  en bas d'\'ecran, ex\'ecutez la ligne \ct{Flaps enableGlobalFlaps}
  pour en cr\'eer un. %
Si vous n'obtenez rien, vous pouvez aussi cliquer sur \menu{World
  \go{} objects (o)} et vous rendre dans la cat\'egorie
\menu{Graphics} de cette fen\^etre rose d'objets. %
Depuis cette fen\^etre ou depuis l'onglet \menu{Supplies}, 
d\'eplacez une ellipse jaune nomm\'ee ``Ellipse'' et une \'etoile
  bleu p\^ale nomm\'ee ``Star''. Placez l'\'etoile sur l'ellipse et 
cliquez avec le bouton rouge de la souris en maintenant la touche
\texttt{Control} enfonc\'ee. Vous obtenez ainsi un menu~\footnote{Vous
  pouvez aussi cliquer avec le bouton bleu de la souris pour afficher
  le halo Morphic et cliquer sur la poign\'ee rouge de
  menu.}. S\'electionnez \menu{embed into \go Ellipse}. Maintenant
votre \'etoile et votre ellipse bougent ensemble.}

Pour d\'eplacer le sous-morph \emph{star}, \'evaluez \ct{joe}
\mthind{Morph}{removeMorph:} \ct{star} ou \ct{star}
\mthind{Morph}{delete}. L\`a encore, une manipulation directe est
possible: 

\dothis{Cliquez avec le \ind{bouton bleu} de la souris deux fois sur
  l'\'etoile bleu p\^ale. Glisser l'\'etoile hors de l'ellipse en
  utilisant la poign\'ee \grabHandle{}.}

\noindent
Le premier clic affiche le halo Morphic de l'ellipse; le second clic
affiche celui de l'\'etoile. Chaque clic change la mise au point en
descendant la hi\'erarchie des inclusions.
%The first click brings up the morphic halo on the ellipse; the second click the halo on the star.   Each click moves the focus down one level of embedding.


%=================================================================
\section{Dessiner ses propres morphs}

Bien qu'il soit possible de faire des repr\'esentations graphiques
utiles et int\'eressantes par composition de morphs, vous aurez
parfois besoin de cr\'eer quelque chose de compl\`etement diff\'erent.
\index{morph!sous-classer}
Pour ce faire, vous d\'efinissez une sous-classe de \ct{Morph} et
surchargez la m\'ethode \mthind{Morph}{drawOn:} pour personnaliser son
apparence.

L'interface Morphic envoie un message \ct{drawOn:} \`a un morph \`a
chaque fois qu'il est n\'ecessaire de rafra\^{\i}chir l'affichage du
morph \`a l'\'ecran. Le param\`etre pass\'e \`a \ct{drawOn:} est un
type de canevas de classe \clsind{Canvas}; le morph s'affichera alors
lui-m\^eme sur ce canevas dans ses limites. %
%the expected behaviour is that the morph will draw itself on that canvas, inside its bounds.
Utilisons cette connaissance pour cr\'eer un morph en forme de croix.
\index{morph!sous-classer}

\dothis{Utilisez le navigateur de classes, d\'efinissez une nouvelle
  classe \clsind{CrossMorph} h\'erit\'ee de \ct{Morph}:}
\begin{classdef}{D\'efinir la classe \ct{CrossMorph}}
Morph subclass: #CrossMorph
	instanceVariableNames: ''
	classVariableNames: ''
	poolDictionaries: ''
	category: 'PBE-Morphic'
\end{classdef}

Nous pouvons d\'efinir la m\'ethode \ct{drawOn:} ainsi:
%ajout du commentaire
\begin{method}[firstDrawOn]{Dessiner un \ct{CrossMorph}}
drawOn: aCanvas 
	"crossHeight est la hauteur de la barre horizontale horizontalBar
    et crossWidth est la largeur de la barre verticale verticalBar"
    | crossHeight crossWidth horizontalBar verticalBar |
	crossHeight := self height / 3.0 .
	crossWidth := self width / 3.0 .
	horizontalBar := self bounds insetBy: 0 @ crossHeight.
	verticalBar := self bounds insetBy: crossWidth @ 0.
	aCanvas fillRectangle: horizontalBar color: self color.
	aCanvas fillRectangle: verticalBar color: self color
\end{method}


\begin{figure}[hbt]
	\ifluluelse
		{\centerline{\includegraphics[width=0.3\textwidth]{NewCross}}}
		{\centerline{\includegraphics{NewCross}}}
	\caption{Un nouveau morph en forme de croix de classe
      \ct{CrossMorph} avec son halo. Vous pouvez redimensionner cette
      croix gr\^ace \`a la poign\'ee inf\'erieure droite
      de couleur jaune.\label{fig:cross}}
\end{figure}

Envoyer le message \mthind{Morph}{bounds} \`a un morph renvoie sa
bo\^{\i}te englobante, instance de la classe \clsind{Rectangle}.  Les
rectangles comprennent plusieurs messages qui cr\'eent d'autres
rectangles de m\^eme g\'eom\'etrie; % related geometry; 
dans notre m\'ethode, nous utilisons le message \ct{insetBy:} avec un
point comme argument pour cr\'eer une premi\`ere fois un rectangle de
hauteur (en anglais, \emph{height}) r\'eduite, puis pour cr\'eer un
autre rectangle de largeur (en anglais, \emph{width}) r\'eduite.

\dothis{Pour tester votre nouveau morph, \'evaluer l'expression \ct{CrossMorph new} \mthind{Morph}{openInWorld}.}
Le r\'esultat devrait \^etre semblable \`a celui de \figref{cross}.
Cependant, vous remarquerez que toute la bo\^{\i}te englobante est
sensible \`a la souris (vous pouvez cliquer en dehors de la croix et
interagir ou d\'eplacer celle-ci). Corrigeons ceci en rendant la seule
surface de la croix sensible \`a la souris.
%However, you will notice that the sensitive zone\,---\,where you can click to grab the morph\,---\,is still the whole bounding box.  Let's fix this.

Lorsque la librairie Morphic a besoin de trouver quels morphs se
trouvent sous le curseur, elle envoie le message \ct{containsPoint:}
 \`a tous les morphs qui ont leur bo\^{\i}te englobante sous le
pointeur de la souris. 
%ajout
Cette m\'ethode r\'epond vrai lorsque le point-argument est contenu dans la
forme d\'efinie.
Pour limiter la zone sensible du morph \`a la forme de la croix, vous
devez surcharger la m\'ethode \ct{containsPoint:}.

\dothis{D\'efinissez la m\'ethode \ct{containsPoint:} dans la classe \ct{CrossMorph}:}

\needlines{4}
\begin{method}[firstContains]{Modeler la zone sensible \`a la souris des instances de \ct{CrossMorph}}
containsPoint: aPoint
	| crossHeight crossWidth horizontalBar verticalBar |
	crossHeight := self height / 3.0.
	crossWidth := self width / 3.0.
	horizontalBar := self bounds insetBy: 0 @ crossHeight.
	verticalBar := self bounds insetBy: crossWidth @ 0.
	^ (horizontalBar containsPoint: aPoint)
		or: [verticalBar containsPoint: aPoint]
\end{method}

Cette m\'ethode suit la m\^eme logique que la m\'ethode \ct{drawOn:}, 
nous sommes donc s\^urs que les points pour lesquels
\ct{containsPoint:} retourne \ct{true} sont les m\^emes points qui
seront color\'es par \ct{drawOn:}.
Notez qu'\`a 
%ajout
la derni\`ere ligne
nous avons profit\'e de la m\'ethode
\mthind{Rectangle}{containsPoint:} 
de la classe \ct{Rectangle} pour faire l'essentiel du travail.
%Notice how we leverage the \mthind{Rectangle}{containsPoint:} method in class \ct{Rectangle} to do the hard work.

Il reste tout de m\^eme deux probl\`emes avec ce code dans les 
\mthsref{firstDrawOn} et \ref{mth:firstContains}.
Le plus remarquable est que nous avons du code dupliqu\'e.
C'est une erreur fondamentale: si vous avez besoin de modifier la
fa\c{c}on dont \ct{horizontalBar} ou \ct{verticalBar} sont
calcul\'ees, vous risquez d'oublier de reporter les changements
effectu\'es d'une m\'ethode \`a l'autre.
%reformulation
La solution consiste \`a \'eliminer la redondance en refactorisant ces
calculs dans deux nouvelles m\'ethodes que nous pla\c{c}ons dans le
protocole \ct{private}:

\needlines{4}
\begin{method}{\ct{horizontalBar}}
horizontalBar
	| crossHeight |
	crossHeight := self height / 3.0.
	^ self bounds insetBy: 0 @ crossHeight
\end{method}

\needlines{4}
\begin{method}{\ct{verticalBar}}
verticalBar
	| crossWidth |
	crossWidth := self width / 3.0.
	^ self bounds insetBy: crossWidth @ 0
\end{method}

\noindent
Nous pouvons ensuite d\'efinir les m\'ethodes \ct{drawOn:} et
\ct{containsPoint:} ainsi:

\needlines{4}
\begin{method}{Refactoriser \ct{CrossMorph>>>drawOn:}}
drawOn: aCanvas 
	aCanvas fillRectangle: self horizontalBar color: self color.
	aCanvas fillRectangle: self verticalBar color: self color
\end{method}

\needlines{4}
\begin{method}{Refactoriser \ct{CrossMorph>>>containsPoint:}}
containsPoint: aPoint
	^ (self horizontalBar containsPoint: aPoint)
		or: [self verticalBar containsPoint: aPoint]
\end{method}

Ce code est plus simple \`a comprendre principalement parce que nous
avons donn\'e des noms parlants \`a ces m\'ethodes priv\'ees. En fait,
notre simplification a mis en avant notre second probl\`eme: 
% it is so simple that you may have noticed the second problem:
la zone centrale de notre croix, \`a la crois\'ee des barres
horizontales et verticales, est dessin\'ee deux fois. Ce n'est pas
tr\`es probl\'ematique tant que notre croix est de couleur opaque,
mais l'erreur devient clairement apparente si nous dessinons une croix
semi-transparente, comme nous pouvons le voir sur \figref{overdrawBug}.

\begin{figure}[t]
\begin{minipage}{0.48\textwidth}
	\ifluluelse
		{\centerline{\includegraphics[scale=0.6]{overdrawBug}}}
		{\centerline{\includegraphics{overdrawBug}}}
	\caption{Le centre de la croix est rempli deux fois avec la
      couleur.	\label{fig:overdrawBug}}
\end{minipage}
\hfill
\begin{minipage}{0.48\textwidth}
	\ifluluelse
		{\centerline{\includegraphics[scale=0.6]{hairlineBug}}}
		{\centerline{\includegraphics{bug}}}
	\caption{Le morph en forme de croix pr\'esente une ligne de pixels
      non remplis.	\label{fig:bug}}
\end{minipage}
\end{figure}

\needlines{4}
\dothis{\'Evaluez ligne par ligne le code suivant dans un espace de travail:}

\begin{code}{}
m := CrossMorph new bounds: (0@0 corner: 300@300).
m openInWorld.
m color: (Color blue alpha: 0.3).

\end{code}

\noindent
La correction repose sur la division de la barre verticale en trois
morceaux et sur le remplissage uniquement des deux morceaux
sup\'erieurs et inf\'erieurs.
Encore une fois, nous trouvons une m\'ethode dans la classe
\ct{Rectangle} qui va bien nous aider: \ct{r1 areasOutside: r2} 
retourne un tableau de rectangles comprenant les parties de \ct{r1}
exclus de \ct{r2}. 

Le code revisit\'e de la m\'ethode \ct{drawOn:} peut s'\'ecrire comme suit:

%ajout du commentaire
\begin{method}{La m\'ethode \ct{drawOn:} revisit\'ee pour ne remplir le centre qu'une seule fois}
drawOn: aCanvas 
    "topAndBottom est un tableau des parties de verticalBar !tronqu\'e!"
	| topAndBottom |
	aCanvas fillRectangle: self horizontalBar color: self color.
	topAndBottom := self verticalBar areasOutside: self horizontalBar. 
	topAndBottom do: [ :each | aCanvas fillRectangle: each color: self color]
\end{method}

Ce code semble fonctionner mais, suivant la taille des croix 
%ajout
(que vous pouvez obtenir en les dupliquant et en les redimensionnant
avec le halo Morphic), vous pouvez constater qu'une ligne d'un pixel
de haut peut s\'eparer la base de la croix du reste, comme le montre
\figref{bug}.
Ceci est du \`a un probl\`eme de troncature: %temporaire  
lorsque la taille d'un rectangle \`a remplir n'est pas un entier,
\ct{fillRectangle: color:} semble mal arrondir et laisse donc une
ligne de pixels non remplis.
Nous pouvons r\'esoudre ce probl\`eme en arrondissant explicitement
lors du calcul des tailles des barres.

\needlines{5}
\begin{method}{\ct{CrossMorph>>>horizontalBar} avec troncature explicite}
horizontalBar
	| crossHeight |
	crossHeight := (self height / 3.0) rounded.
	^ self bounds insetBy: 0 @ crossHeight
\end{method}

\needlines{5}
\begin{method}{\ct{CrossMorph>>>verticalBar} avec troncature explicite}
verticalBar
	| crossWidth |
	crossWidth := (self width / 3.0) rounded.
	^ self bounds insetBy: crossWidth @ 0
\end{method}



%=================================================================
%\section{Composing Morphs}

%\on{The source code is in the examples directory.
%For the moment I prefer to leave out the examples, as they do not add much.}

%Below, we present a few morphs that were designed for a course project.

%\paragraph{An adhesive Label} The \ct{LabelStickerMorph} is a metaphor for an adhesive label with a colored border and three lines of text (\figref{labeler}, \egref{labeler}).

%\begin{figure}[ht]
%	\centerline{\includegraphics[width=0.25\textwidth]{labeler}}
%	\caption{The sticker label morph.
%		\label{fig:labeler}}
%\end{figure}

%\begin{example}[labeler]{Creating a sticker label}{}
%label := LabelstickerMorph new openInWorld.
%label text1: 'Confiture sans sucre';
%	text2: 'Fraises du jardin';
%	text3: '9 mai 2006'.
%label lineColor: Color blue
%\end{example}

%\paragraph{A Number Pyramid}
%The previous morph is designed by overriding the \ct{drawOn:} method.
%We built \ct{PyramidMorph} by composing morphs: we used \ct{TextMorph}s to make the blocks and added them to a base morph (\figref{pyramid}, \egref{pyramid}). \damien{figure does not match text... no numbers? Where is the code?}
%\begin{figure}[ht]
%	\centerline{\includegraphics{pyramid}}
%	\caption{The number pyramid morph.
%		\label{fig:pyramid}}
%\end{figure}

%\begin{example}[pyramid]{Manipulating the number pyramid}{}
%pyramid := (PyramidMorph base: 4) openInWorld.
%pyramid block: 8 value: 2
%\end{example}


%=================================================================
\section{Int\'eraction et animation}

Pour construire des interfaces utilisateur vivantes avec les morphs,
nous avons besoin de pouvoir interagir avec elles en utilisant la
souris et le clavier.
En outre, les morphs doivent \^etre capable de r\'epondre aux
int\'eractions de l'utilisateur en changeant leur apparence et leur
position, autrement dit, en s'animant eux-m\^emes.

\subsection{Les \'ev\'enements souris}

Quand un bouton de la souris est press\'e, Morphic envoie \`a chaque
morph sous le pointeur de la souris le message
\ct{handlesMouseDown:}. Si un morph r\'epond \ct{true}, Morphic lui
envoie imm\'ediatemment le message \mthind{Morph}{mouseDown:}. Lorsque
le bouton de la souris est rel\^ach\'e, Morphic envoie
\mthind{Morph}{mouseUp:}  \`a ces m\^eme morphs qui avaient r\'epondus
positivement. Si tous les morphs retournent \ct{false}, Morphic entame
une op\'eration de
%saisissement dans la version d'Hilaire (c'est en fait une emotion
%vive et soudaine
saisie en pr\'evision du glisser-d\'eposer.
Comme nous allons le voir, les messages \ct{mouseDown:} et \ct{mouseUp}
sont envoy\'es avec un argument\,---\,un objet de classe
\clsind{MouseEvent}\,---\,qui contient les d\'etails de l'action de la souris.
%that encodes the details of the mouse action.

Ajoutons la gestion des \'ev\'enements souris \`a notre classe
\ct{CrossMorph} en commen\c{c}ant par nous assurer que toutes nos
croix r\'epondent \ct{true} au message \mthind{Morph}{handlesMouseDown:}.

\dothis{Ajoutez la m\'ethode suivante \`a la classe \ct{CrossMorph}:}
\begin{method}{D\'eclarer que \ct{CrossMorph} r\'eagit aux clics de souris}
CrossMorph>>>handlesMouseDown: anEvent
	^true
\end{method}

Supposons que vous voulez que la couleur de la croix passe au rouge
(\ct{Color red})
\`a chaque clic du bouton rouge de la souris et qu'elle passe au jaune
(\ct{Color yellow})
lorsque le bouton enfonc\'e est le bouton jaune. 
Nous devons cr\'eer la \mthref{mouseDown}.

\begin{method}[mouseDown]{R\'eagir aux clics de la souris en changeant la couleur de la croix}
CrossMorph>>>mouseDown: anEvent
	anEvent redButtonPressed
		ifTrue: [self color: Color red].
	anEvent yellowButtonPressed
		ifTrue: [self color: Color yellow].
	self changed
\end{method}

\ab{I added this note:}
Remarquez que non seulement cette m\'ethode change la couleur de notre
morph, mais qu'elle envoie aussi le message \ct{self changed}.
Ce message assure que Morphic envoie \ct{drawOn:}
%in a timely fashion
de fa\c{c}on assez rapide.
\ab{However, the \ct{self changed} message seems to be entirely unnecessary; the colour changes instantly without it.}
Notez aussi qu'une fois qu'un morph g\`ere les événements \subind{événement}{souris}, vous ne pouvez plus l'attraper avec la souris pour le d\'eplacer.
D\`es lors, vous devez utiliser le halo Morphic en cliquant dessus avec le
bouton bleu: les poign\'ees sup\'erieurs noir \grabHandle{}
et marron \moveHandle{} 
vous permettent respectivement de prendre et d\'eplacer ce morph.
\seeindex{souris!événement}{événement, souris}

L'argument \ct{anEvent} de \ct{mouseDown:} est une instance de
\mbox{\clsind{MouseEvent},} sous-classe de \ct{MorphicEvent}.
%\lct{Mor\-phic\-Event}{.} 
\ct{MouseEvent} d\'efinit les m\'ethodes
\mthind{MouseEvent}{redButtonPressed} pour la gestion du clic au
bouton rouge de la souris et \mthind{MouseEvent}{yellowButtonPressed}
pour celle du clic au bouton jaune. Parcourez cette classe pour
en savoir plus sur les autres m\'ethodes disponibles pour la gestion
des \'ev\'enements souris.

\subsection{Les \'ev\'enements clavier}

La capture des événements \subind{événement}{clavier} se d\'eroule en trois
\'etapes. Morphic devra:

%modification
%``keyboard focus'' to a specific morph: for instance we can give focus to our morph when the mouse is over it.

\begin{enumerate}
	\item activer votre morph pour la gestion du clavier par la ``mise
      au point'' sous une certaine condition, disons, lorsque la souris est au-dessus du morph; 
	\item g\'erer l'\'ev\'enement proprement dit avec la m\'ethode
      \mthind{Morph}{handleKeystroke:} --- ce message est envoy\'e au
      morph quand vous pressez une touche et qu'il a d\'ej\`a re\c{c}u
      la mise au point  (en anglais, \emph{keyboard focus});
	\item lib\'erer la mise au point lorsque la condition de la
      premi\`ere \'etape n'est plus remplie, disons, quand la souris
      n'est plus au-dessus du morph.
\end{enumerate}

Occupons-nous de \ct{CrossMorph} pour que nos croix r\'eagissent \`a
certaines touches du clavier. Tout d'abord, nous avons besoin d'\^etre
inform\'e que la souris est au-dessus de la surface de notre morph:
dans ce cas, le morph doit r\'epondre \ct{true} au message
\mthind{Morph}{handlesMouseOver:}.

\dothis{D\'eclarez que \ct{CrossMorph} r\'eagit lorsque il est sous le
  pointeur de la souris.}

\begin{method}{G\'erer les \'ev\'enements souris ``mouse over''} 
CrossMorph>>>handlesMouseOver: anEvent
	^true
\end{method}

\noindent
Ce message est \'equivalent \`a \mthind{Morph}{handlesMouseDown:}
utilis\'e pour la position de la souris.
Les messages \mthind{Morph}{mouseEnter:} et
\mthind{Morph}{mouseLeave:} sont envoy\'es respectivement lorsque le
pointeur de la souris entre dans l'espace du morph ou sort de celui-ci.

\dothis{D\'efinissez deux m\'ethodes gr\^ace auxquelles un morph
  \ct{CrossMorph} peut activer et lib\'erer la mise au point sur le
  clavier. Cr\'eez ensuite une troisi\`eme m\'ethode pour g\'erer
  l'interaction via la saisie des touches.}
\begin{method}{Activer la mise au point sur le clavier lorsque la souris entre dans l'espace du morph}
CrossMorph>>>mouseEnter: anEvent
	anEvent hand newKeyboardFocus: self
\end{method}

\begin{method}{Lib\'erer la mise au point sur le clavier lorsque la souris sort de l'espace du morph}
CrossMorph>>>mouseLeave: anEvent
	anEvent hand newKeyboardFocus: nil
\end{method}

\begin{method}[handleKeystroke]{Capturer et g\'erer les \'ev\'enements clavier}
CrossMorph>>>handleKeystroke: anEvent
	| keyValue |
	keyValue := anEvent keyValue.
	keyValue = 30	 "!fl\`eche du haut!"
		ifTrue: [self position: self position - (0 @ 1)].
	keyValue = 31	 "!fl\`eche du bas!"
		ifTrue: [self position: self position + (0 @ 1)].
	keyValue = 29	 "!fl\`eche de droite!"
		ifTrue: [self position: self position + (1 @ 0)].
	keyValue = 28	 "!fl\`eche de gauche!"
		ifTrue: [self position: self position - (1 @ 0)]
\end{method}

La m\'ethode que nous venons d'\'ecrire vous permet de d\'eplacer
notre croix avec les touches fl\'ech\'ees. Remarquez que quand la
souris n'est pas sur la croix, le message
\mthind{Morph}{handleKeystroke:} n'est pas envoy\'e: dans ce cas, la croix
ne r\'epond pas aux commandes clavier.
Vous pouvez conna\^{\i}tre la valeur des touches saisies au clavier en
ouvrant une fen\^etre Transcript et en ajoutant \`a 
\mthref{handleKeystroke} la ligne 
\glbind{Transcript} \ct{show: anEvent keyValue}.
L'\'ev\'enement-argument \ct{anEvent} de \ct{handleKeystroke} est une
instance de la classe \clsind{KeyboardEvent}, sous-classe de
\clsind{MorphicEvent}. Naviguez dans cette classe pour conna\^itre les
m\'ethodes de gestion des \'ev\'enements clavier.

% dans le document original, on parle avant de
% MorphicEvent mais pas dans la version d'andrew

\subsection{Les animations Morphic}

%martial: j'ai chang\'e deux methodes + 3 en quatre + 1
Pour l'essentiel, Morphic permet de composer et d'automatiser de
simples animations gr\^ace \`a quatre m\'ethodes:
\begin{itemize}
\item \mthind{Morphic}{step} qui est envoy\'e au morph \`a un
  \emph{tempo} r\'egulier pour construire le comportement de l'animation;
\item \mthind{Morphic}{stepTime} qui d\'efinit l'intervalle de temps en
  millisecondes entre chaque envoi du message \ct{step}~\footnote{\ct{stepTime} est
    en r\'ealit\'e le temps \emph{minimum} entre les envois du message
    \ct{step}. Si vous demandez un \emph{tempo} \ct{stepTime} de
    1\,ms, ne soyez pas \'etonn\'e si \pharo est trop occup\'e pour que
    le rythme de l'animation de votre morph tienne cette cadence.};
\item \mthind{Morphic}{startStepping} d\'emarre l'animation au rythme
  du m\'etronome \ct{stepTime};
\item \mthind{Morphic}{stopStepping} arr\^ete l'animation.
\end{itemize}

\`A ces m\'ethodes s'ajoutent une m\'ethode de test
\mthind{Morphic}{isStepping} pour savoir si le morph est en cours
d'animation.% currently being stepped.
\index{Morphic!animation}

\dothis{Faites clignoter le \ct{CrossMorph} en d\'efinissant les
  m\'ethodes suivantes:}
\begin{method}{D\'efinir la p\'eriodicit\'e de l'animation}
CrossMorph>>>stepTime
	^ 100
\end{method}
\begin{method}{Construire le comportement de l'animation}
CrossMorph>>>step
	(self color diff: Color black) < 0.1
		ifTrue: [self color: Color red]
		ifFalse: [self color: self color darker]
\end{method}
\noindent
Pour d\'emarrer l'animation, vous pouvez ouvrir un inspecteur sur
votre objet \ct{CrossMorph}: cliquez sur la poign\'ee de d\'ebogage 
\debugHandle{} du halo Morphic de votre croix (activ\'e avec le
bouton bleu de la souris) puis choisissez \menu{inspect morph} dans le
menu flottant. Entrez l'expression \ct{self startStepping} dans le
mini-espace de travail situ\'e dans le bas de l'inspecteur et faites
un \menu{do it}.
%ajout
Pour arr\^eter l'animation, vous n'avez qu'\`a \'evaluer \ct{self stopStepping} dans l'inspecteur. 
Pour d\'emarrer et arr\^eter l'animation de fa\c{c} plus efficace, vous pouvez
ajouter des contr\^oles suppl\'ementaires au clavier. Par exemple,
vous pouvez modifier la m\'ethode \ct{handleKeystroke:} pour que la
touche $+$ d\'emarre le clignotement de la croix et que la touche $-$
le stoppe.

\dothis{Ajoutez le code suivant \`a \mthref{handleKeystroke}:}

\begin{code}{}
	keyValue = $+ asciiValue 
		ifTrue: [self startStepping].
	keyValue = $- asciiValue
		ifTrue: [self stopStepping].
\end{code}

% \on{You can also \menu{debug \go inspect morph} and evaluate: \ct{self currentWorld startStepping: self}.}

%=================================================================
%martial: passage de cette section en sous-section
\subsection{Les interacteurs} %interactors

%ajout
Morphic dispose de morphs commodes pour cr\'eer en quelques lignes de
code des interactions avec l'utilisateur. Parmi eux, nous avons la
classe \clsind{FillInTheBank} offre quelques bo\^{\i}tes de dialogue
pr\^etes \`a l'emploi pour fournir \`a l'utilisateur une zone de saisie.
La m\'ethode \mthind{FillInTheBlank}{request:initialAnswer:} renvoie
une cha\^{\i}ne de caract\`eres entr\'ee par l'utilisateur (voir
\figref{dialogName}).

\begin{figure}[htb]
\begin{minipage}{0.55\textwidth}
	\ifluluelse
		{\centerline{\includegraphics[scale=0.65]{dialog}}}
		{\centerline{\includegraphics[width=5cm]{dialog}}}
	\caption{Une bo\^{\i}te de dialogue affich\'ee par
      \ct{FillInTheBlank request: 'Quel est votre nom?' initialAnswer: 'sans nom'}.
		\label{fig:dialogName}}
\end{minipage}
\hfill
\begin{minipage}{0.38\textwidth}
	\vfill
	\ifluluelse
		{\centerline{\includegraphics [scale=0.65]{popup}}}
		{\centerline{\includegraphics[width=3cm]{popup}}}
	\vfill
	\vspace{4ex}
	\caption{Un menu flottant affich\'e gr\^ace \`a \ct{PopUpMenu>>>startUpWithCaption:}.}
\end{minipage}
\end{figure}

%\begin{figure}[ht]
%	\centerline{\includegraphics[width=5cm]{dialog}}
%	\caption{Dialog displayed by \ct{FillInTheBlank request: 'What''s your name?' initialAnswer: 'no name'}.
%		\label{fig:dialogName}}
%\end{figure}

Pour afficher le menu flottant (en anglais, \emph{pop-up menu}), vous
devez faire appel \`a la classe \clsind{PopupMenu}:
\begin{code}{}
menu := PopUpMenu
	labelArray: #('cercle' 'ovale' '!\normcode{carr\'e}!' 'rectangle' 'triangle')
	lines: #(2 4).
menu startUpWithCaption: 'Choisissez une forme'
\end{code}

%\begin{figure}[ht]
%	\centerline{\includegraphics[width=3cm]{popup}}
%	\caption{PopUp displayed by \ct{PopUpMenu>>>startUpWithCaption:}.}
%\end{figure}

%=================================================================
\section{Le glisser-d\'eposer}

Morphic supporte aussi le glisser-d\'eposer. \'Etudions l'exemple
suivant. Cr\'eons tout d'abord un morph receveur qui n'acceptera un
morph que si le d\'ep\^ot de ce morph se fait dans une certaine
condition. Cr\'eons ensuite un second morph que nous appelons morph
d\'epos\'e. Le fait que le morph soit bleu (\ct{Color blue}) sera
notre condition pour que le glisser-d\'epos\'e se fasse ici.

\dothis{D\'efinissez la classe pour le morph receveur et cr\'eez une
  m\'ethode d'initialisation comme suit:}
\begin{classdef}{D\'efinir un morph sur lequel un autre morph pourra \^etre d\'epos\'e}
Morph subclass: #ReceiverMorph
	instanceVariableNames: ''
	classVariableNames: ''
	poolDictionaries: ''
	category: 'PBE-Morphic'
\end{classdef}

%\dothis{Now define the initialization method in the usual way:}
\begin{method}{Initialiser un objet \ct{ReceiverMorph}}
ReceiverMorph>>>initialize
	super initialize.
	color := Color red.
	bounds := 0 @ 0 extent: 200 @ 200
\end{method}

Comment d\'ecidons-nous si le receveur va accepter ou refuser le morph
d\'epos\'e? En g\'en\'eral, ces deux morphs devront s'accorder sur
leur interaction. Le receveur fait cela en r\'epondant au message
 \mthind{Morph}{wantsDroppedMorph:event:}; le premier argument est le
 morph que nous voulons d\'eposer et le second est l'\'ev\'enement
 souris. Ce dernier argument permet, par exemple, au receveur de
 savoir si une (ou plusieurs) touche de modification a \'et\'e
 maintenue enfonc\'ee durant la phase de d\'ep\^ot de l'autre morph.
Le morph d\'epos\'e, quant \`a lui, se doit de v\'erifier s'il est
compatible avec le morph sur lequel il est d\'epos\'e; le message
\ct{wantsToBeDroppedInto:} doit r\'epondre \ct{true} si le morph
receveur 
%ajout
pass\'e en argument est d\'efini comme compatible. L'impl\'ementation
de cette m\'ethode dans la classe m\`ere des morphs \ct{Morph} renvoie
toujours \ct{true} 
donc, par d\'efaut, tous les morphs sont accept\'es en tant que
receveur.

\begin{method}{Accepter les morphs d\'epos\'es selon leur couleur}
ReceiverMorph>>>wantsDroppedMorph: aMorph event: anEvent
	^ aMorph color = Color blue
\end{method}

Qu'arrive-t-il au morph d\'epos\'e si le morph receveur ne veut pas de lui?
Le comportement par d\'efaut de l'interface Morphic est de ne rien
%reformule
faire, \cad de laisser le morph d\'epos\'e au-dessus du morph receveur
sans aucune interaction avec celui-ci. 
Le morph d\'epos\'e aurait un comportement plus intuitif s'il
retournait \`a sa position d'origine en cas de refus.
Nous pouvons faire cela en disant au receveur de r\'epondre \ct{true}
au message  \mthind{Morph}{repelsMorph:event:} lorsque celui-ci ne
veut pas du morph d\'epos\'e:

\needlines{4}
\begin{method}{Changer le comportement du morph d\'epos\'e lorsqu'il est rejet\'e}
ReceiverMorph>>>repelsMorph: aMorph event: anEvent
	^ (self wantsDroppedMorph: aMorph event: anEvent) not
\end{method}

C'est tout ce dont nous avons besoin.
%That's all we need as far as the receiver is concerned.

\dothis{Cr\'eez des instances de \clsind{ReceiverMorph} et de
  \clsind{EllipseMorph} dans un espace de travail:}
\begin{code}{}
ReceiverMorph new openInWorld.
EllipseMorph new openInWorld.
\end{code}
\noindent
Essayez de faire un glisser-d\'eposer de l'ellipse jaune
\ct{EllipseMorph} sur le morph receveur rouge. Il sera rejet\'e et
retournera \`a sa position initiale.

%martial: j'ai ajoute le texte entre parentheses
\dothis{Changez la couleur de l'ellipse pour du bleu via l'inspecteur
  (que vous pouvez activer avec le menu de la poign\'ee du d\'ebogage
  du halo Morphic en cliquant sur \menu{inspect morph}): \'evaluez
  \ct{self color: Color blue}.  Les morphs bleus \'etant accept\'es par
  le \ct{ReceiverMorph}: essayez \`a nouveau le glisser-d\'eposer.}

%ajout (encouragement)
Bravo! Vous venez de faire un glisser-d\'eposer.

Continuons \`a explorer le glisser-d\'eposer en cr\'eant un morph
d\'epos\'e sp\'ecifique nomm\'e \ct{DroppedMorph},
sous-classe de \ct{Morph}:

\begin{classdef}{D\'efinir un morph que nous pouvons glisser-d\'eposer sur un \ct{ReceiverMorph}}
Morph subclass: #DroppedMorph
	instanceVariableNames: ''
	classVariableNames: ''
	poolDictionaries: ''
	category: 'PBE-Morphic'
\end{classdef}

\begin{method}{Initialiser \ct{DroppedMorph}}
DroppedMorph>>>initialize
	super initialize.
	color := Color blue.
	self position: 250@100
\end{method}

% reformulation
Nous voulons que le morph d\'epos\'e ait un nouveau comportement 
lorsqu'il est rejet\'e par le receveur; cette fois-ci, il restera
attach\'e au pointeur de la souris:
\begin{method}{R\'eagir lorsque le morph est rejet\'e lors du d\'ep\^ot}
DroppedMorph>>>rejectDropMorphEvent: anEvent
	| h |
	h := anEvent hand.
	WorldState
		addDeferredUIMessage: [h grabMorph: self].
	anEvent wasHandled: true
\end{method}

L'envoi du message \mthind{MorphicEvent}{hand} \`a un \'ev\'enement
r\'epond la ``main'' (en anglais, \emph{hand}), instance de
\ct{HandMorph} qui repr\'esente le pointeur de la souris et tout ce
qu'il tient.
Dans notre m\'ethode, nous disons \`a l'\'ecran \pharo, \ct{World}, que
la main 
%ajout
(stock\'ee dans la variable temporaire \ct{h}) doit capturer
le morph rejet\'e 
%ajout + l'index de la m\'ethode
\ct{self} gr\^ace au message \mthind{HandMorph}{grabMorph:}.
%ajout (commentaire de la methode dans MorphicEvent et non dans DropEvent)
La m\'ethode \ct{wasHandled:} d\'etermine si l'\'ev\'enement \'etait captur\'e.

\dothis{Cr\'eer deux instances de \ct{DroppedMorph} et faites un
  glisser-d\'eposer pour chacune sur le receveur.}
\begin{code}{}
ReceiverMorph new openInWorld.
(DroppedMorph new color: Color blue) openInWorld.
(DroppedMorph new color: Color green) openInWorld.
\end{code}
\noindent
Le morph vert (\ct{Color green}) est rejet\'e et reste ainsi attach\'e
au pointeur de la souris.

%=================================================================
%\section{Un exemple complet}
\section{Le jeu du d\'e}

Lan\c{c}ons-nous maintenant dans la cr\'eation d'un jeu
du d\'e complet. Nous voulons faire d\'efiler toutes les faces d'un
d\'e dans une boucle rapide suite \`a un premier clic de
souris sur la surface de ce d\'e puis, lors
d'un second clic, arr\^eter l'animation sur une face.

\begin{figure}[ht]
	\centerline{\includegraphics[scale=0.65]{die}}
	\caption{Le d\'e dans Morphic.\label{fig:dialogDie}}
\end{figure}

\dothis{D\'efinissez un d\'e comme une sous-classe de
  \clsind{BorderedMorph} d\'efinissant un \ct{Morph} avec un bord:
  appelez-le \ct{DieMorph} (d\'e se dit \emph{die} en anglais).}

\needlines{6}
\begin{classdef}{D\'efinir le d\'e DieMorph}
BorderedMorph subclass: #DieMorph
	instanceVariableNames: 'faces dieValue isStopped'
	classVariableNames: ''
	poolDictionaries: ''
	category: 'PBE-Morphic'
\end{classdef}

La variable d'instance \ct{faces} stocke le nombre de faces de notre
d\'e; nous nous autorisons \`a avoir des d\'es jusqu'\`a neuf faces!
\ct{dieValue} contient la valeur de la face affich\'ee en ce moment et
\ct{isStopped} est un bool\'een que est \ct{true} si et seulement si
l'animation est \`a l'arr\^et.
Nous allons d\'efinir la \emph{m\'ethode de classe} \ct{faces: n} dans
le c\^ot\'e classe de \clsind{DieMorph} pour pouvoir cr\'eer un
nouveau d\'e \`a \ct{n} faces.

\begin{method}{Cr\'eer un nouveau d\'e avec un nombre de faces d\'etermin\'e}
DieMorph class>>>faces: aNumber
	^ self new faces: aNumber
\end{method}

La m\'ethode \ct{initialize} est d\'efinie dans le c\^ot\'e instance
de la classe; souvenez-vous que \ct{new} envoie \ct{initialize} \`a
toute instance nouvellement cr\'e\'ee.

\begin{method}{Initialiser les instances de \ct{DieMorph}}
DieMorph>>>initialize
	super initialize.
	self extent: 50 @ 50.
	self useGradientFill; borderWidth: 2; useRoundedCorners.
	self setBorderStyle: #complexRaised.
	self fillStyle direction: self extent.
	self color: Color green.
	dieValue := 1.
	faces := 6.
	isStopped := false
\end{method}

Nous utilisons quelques m\'ethodes de la classe \ct{BorderedMorph}
pour donner un aspect sympathique \`a notre d\'e: bordure \'epaisse
avec un effet de relief, coins arrondis et d\'egrad\'e de couleur sur
la face visible.
%ajout
Nous d\'efinissons ensuite la m\'ethode d'instance \ct{faces:} pour
affecter la variable d'instance\,---\,il s'agit d'une m\'ethode
d'acc\`es de type mutateur\,---\, en v\'erifiant que le param\`etre est
bien valide:
\begin{method}{Affecter le nombre correspondant \`a la face visible de d\'e}
DieMorph>>>faces: aNumber
	"Affecter le num\'ero de la face"
	(aNumber isInteger
			and: [aNumber > 0]
			and: [aNumber <= 9])
		ifTrue: [faces := aNumber]
\end{method}
\on{Why not make this a pre-condition, \ie an assertion?}

Comprenez bien l'ordre dans lequel les messages sont envoy\'es lors de
la cr\'eation d'un d\'e. Si nous \'evaluons \ct{DieMorph faces: 9}:
\begin{enumerate}
	\item la m\'ethode de classe \ct{DieMorph class>>>faces:} envoie
      \ct{new} \`a \ct{DieMorph class};
	\item la m\'ethode pour \ct{new} (h\'erit\'ee par \ct{DieMorph class} de \ct{Behavior}) cr\'ee la nouvelle instance et lui envoie
       le message \ct{initialize};
	\item la m\'ethode \ct{initialize} de \ct{DieMorph} affecte la
      valeur initiale 6 \`a \ct{faces};
	\item \ct{DieMorph class>>>new} retourne \`a la m\'ethode de
      classe \ct{DieMorph class>>>faces:} qui envoie ensuite le
      message \ct{faces: 9} \`a la nouvelle instance;
	\item la m\'ethode d'instance \ct{DieMorph>>>faces:} s'ex\'ecute
      maintenant en affectant \`a la valeur 9 la variable d'instance
      \ct{faces}.
\end{enumerate}

%Avant de passer \`a la d\'efinition de la m\'ethode \ct{drawOn:}
Pour positionner les points noirs sur la face du d\'e, nous devons
besoin de d\'efinir autant de m\'ethodes qu'il y a de faces possibles:

\begin{methods}{Neuf m\'ethodes pour placer les points noirs sur la face visible du d\'e}
DieMorph>>>face1
	^{0.5@0.5}
DieMorph>>>face2
	^{0.25@0.25 . 0.75@0.75}
DieMorph>>>face3
	^{0.25@0.25 . 0.75@0.75 . 0.5@0.5}
DieMorph>>>face4
	^{0.25@0.25 . 0.75@0.25 . 0.75@0.75 . 0.25@0.75}
DieMorph>>>face5
	^{0.25@0.25 . 0.75@0.25 . 0.75@0.75 . 0.25@0.75 . 0.5@0.5}
DieMorph>>>face6
	^{0.25@0.25 . 0.75@0.25 . 0.75@0.75 . 0.25@0.75 . 0.25@0.5 . 0.75@0.5}
DieMorph>>>face7
	^{0.25@0.25 . 0.75@0.25 . 0.75@0.75 . 0.25@0.75 . 0.25@0.5 . 0.75@0.5 . 0.5@0.5}
DieMorph >>>face8
	^{0.25@0.25 . 0.75@0.25 . 0.75@0.75 . 0.25@0.75 . 0.25@0.5 . 0.75@0.5 . 0.5@0.5 . 0.5@0.25}
DieMorph >>>face9
	^{0.25@0.25 . 0.75@0.25 . 0.75@0.75 . 0.25@0.75 . 0.25@0.5 . 0.75@0.5 . 0.5@0.5 . 0.5@0.25 . 0.5@0.75}
\end{methods}
\on{kind of ugly boilerplate code -- should be a nice way to map these more elegantly to coordinates.}

Ces m\'ethodes d\'efinissent des collections de coordonn\'ees de
points pour chaque configuration de faces possible. Les coordonn\'ees
sont dans un carr\'e de dimension $1\times1$. Pour placer nos points,
nous effectuons simplement un changement d'\'echelle.

%reformulation
Enfin, pour dessiner la face du d\'e, nous d\'efinissons la m\'ethode
\ct{drawOn:} qui fera d'abord un envoi sur \ct{super}, utilisant la m\'ethode
d\'efinie dans une classe-m\`ere pour dessiner le fond de la face, et
qui exploitera, dans un deuxi\`eme temps, les m\'ethodes cr\'e\'ees
pr\'ec\'edemment pour dessiner les points noirs.

\begin{method}{Dessiner le d\'e}
DieMorph>>>drawOn: aCanvas
	super drawOn: aCanvas.
	(self perform: ('face' , dieValue asString) asSymbol)
		do: [:aPoint | self drawDotOn: aCanvas at: aPoint]
\end{method}

Les capacit\'es r\'eflexives de \st sont utilis\'ees dans la
derni\`ere expression de cette m\'ethode. Dessiner les points noirs
d'une face revient \`a it\'erer sur la collection 
%ajout
de coordonn\'ees retourn\'ee par la m\'ethode \ct{faceX} 
%ajout
(\ct{X} est issu de la variable d'instance \ct{dieValue}
correspondant au num\'ero de la face en cours),
en envoyant le message \ct{drawDotOn:at:} pour chacune de ces
coordonn\'ees. Pour joindre la bonne m\'ethode %martial: joindre plutot qu'appeler
\ct{faceX}, nous utilisons la m\'ethode \mthind{Object}{perform:} qui
envoie le message construit \`a partir d'une cha\^{\i}ne de
caract\`eres \ct{('face', dieValue asString) asSymbol}.
Cet usage de la m\'ethode \ct{perform:} est tr\`es fr\'equent.

\index{réfléctivité}
\begin{method}{Dessiner un simple point noir sur une face}
DieMorph>>>drawDotOn: aCanvas at: aPoint
	aCanvas
		fillOval: (Rectangle
			center: self position + (self extent * aPoint)
			extent: self extent / 6)
		color: Color black
\end{method}

Puisque les coordonn\'ees sont norm\'ees dans l'intervalle $[0{:}1]$,
elles sont mises \`a l'\'echelle des dimensions du d\'e avec 
\ct{self extent * aPoint}.

\dothis{Cr\'eez une instance de d\'e dans un espace de travail:}
\begin{code}{}
(DieMorph faces: 6) openInWorld.
\end{code}

Pour pouvoir modifier la valeur de la face visible, nous devons
cr\'eer un mutateur aussi pour \ct{dieValue}. 
%modification
Gr\^ace \`a elle, nous pourrions, par exemple, afficher la face \`a 4
points depuis une nouvelle m\'ethode de la classe en y \'ecrivant
\ct{self dieValue: 4}.
%To change the displayed face, we create an accessor that we can use as \ct{myDie dieValue: 4}:

\begin{method}{Affecter un nombre \`a la valeur courante du d\'e}
DieMorph>>>dieValue: aNumber
	(aNumber isInteger
			and: [aNumber > 0]
			and: [aNumber <= faces])
		ifTrue:
			[dieValue := aNumber.
			self changed]
\end{method}

Nous allons utiliser le syst\`eme d'animation pour faire d\'efiler rapidement 
%ajout
et al\'eatoirement (avec le message \ct{atRandom}) toutes les faces du d\'e:
\index{Morphic!animation}
\begin{methods}{Animer le d\'e}
DieMorph>>>stepTime
	^ 100

DieMorph>>>step
	isStopped ifFalse: [self dieValue: (1 to: faces) atRandom]
\end{methods}
D\'esormais, notre d\'e ``roule''!

Pour d\'emarrer ou arr\^eter  l'animation par un clic de souris, nous
utiliserons ce que nous avons pr\'ealablement appris sur les
\'ev\'enements souris.
Nous activons la r\'eception des \'ev\'enements de la souris et nous
d\'ecrivons notre gestion du bouton rouge de la souris dans la
m\'ethode \ct{mouseDown:}.       
%ou \mthind{Morph}{mouseDown:}

\begin{methods}{G\'erer les clics de souris pour d\'emarrer et arr\^eter l'animation}
DieMorph>>>handlesMouseDown: anEvent
	^ true

DieMorph>>>mouseDown: anEvent
	anEvent redButtonPressed
		ifTrue: [isStopped := isStopped not]
\end{methods}

%ajout
Maintenant notre d\'e ``roule'' ou se fige quand nous cliquons dessus.

% That's all for the essentials of Morphic!

% Most of the work on \ct{DieMorph} was done with an instance of it living in the environment; this is quite nice when to tweak programs.

%=================================================================
%\section{More about the canvas}
%\section{Encore quelques mots sur le canevas}
\section{Gros plan sur le canevas}

La m\'ethode \ct{drawOn:} a un canevas, instance de
\clsindmain{Canvas}, comme unique argument;
le canevas est l'espace dans lequel le morph se dessine.
En utilisant les m\'ethodes graphiques du canevas, vous \^etes libre
de donner l'apparence que vous voulez \`a votre morph.
Si vous parcourez la hi\'erarchie d'h\'eritage de la classe
\ct{Canvas}, vous constaterez plusieurs variantes. Par d\'efaut, nous
utilisons \clsind{FormCanvas}. Cette classe et sa classe-m\`ere
\ct{Canvas} contiennent les m\'ethodes graphiques essentielles pour
dessiner des points, des lignes, des polygones, des rectangles, des
ellipses, du texte et des images avec rotation et changement d'\'echelle. 

%ajout de la note de bas de page
Vous pouvez aussi utiliser d'autres types de canevas pour obtenir, par
exemple, des m\'ethodes suppl\'ementaires ou encore, ajouter la
transparence ou l'anti-cr\'enelage (ou
\emph{anti-aliasing}~\footnote{Ce rendu est utilis\'e pour att\'enuer
  ou \'eliminer l'effet escalier du aux pixels.}) aux morphs.  
Vous aurez besoin dans ces cas-l\`a de canevas tels que
\clsind{AlphaBlendingCanvas} ou \clsind{BalloonCanvas}.
Pour obtenir un canevas diff\'erent dans la m\'ethode \ct{drawOn:}
alors que son argument est une instance de \ct{FormCanvas}, vous devrez
court-circuiter le canevas courant par un autre. 
%martial: ou transformer un type de canevas en un autre.
%comme dans:
%But how can you obtain such a canvas in a \ct{drawOn:} method, when \ct{drawOn:} receives an instance of \ct{FormCanvas} as its argument?
%Fortunately, you can transform one kind of canvas into another.

\dothis{Red\'efinissez \ct{drawOn:} de la classe \ct{DieMorph} pour
  utiliser un canevas semi-transparent:}

%martial: je force la nouvelle page tant que le titre est sur l'autre page
%\newpage
\begin{method}{Dessiner un d\'e semi-transparent}
DieMorph>>>drawOn: aCanvas
	| theCanvas |
	theCanvas := aCanvas asAlphaBlendingCanvas: 0.5.
	super drawOn: theCanvas.
	(self perform: ('face' , dieValue asString) asSymbol)
		do: [:aPoint | self drawDotOn: theCanvas at: aPoint]
\end{method}
\noindent

C'est tout ce dont nous avons besoin! 

Vous pouvez parcourir la m\'ethode
\mthind{Canvas}{asAlphaBlendingCanvas:} par curiosit\'e.
Pour profiter de l'anti-cr\'enelage, vous pouvez aussi utiliser
\clsind{BalloonCanvas} et changer la m\'ethode d'affichage des points
noirs de notre d\'e comme dans \mthsref{aadie}.

\begin{figure}[ht]
	\centerline{\includegraphics[scale=0.7]{multiMorphs}}
	\caption{Le d\'e semi-transparent.\label{fig:multiMorphs}}
\end{figure}

\needlines{6}
\begin{methods}[aadie]{Dessiner un d\'e avec anti-cr\'enelage}
DieMorph>>>drawOn: aCanvas
	| theCanvas |
	theCanvas := aCanvas asBalloonCanvas aaLevel: 3.
	super drawOn: aCanvas.
	(self perform: ('face' , dieValue asString) asSymbol)
		do: [:aPoint | self drawDotOn: theCanvas at: aPoint]

DieMorph>>>drawDotOn: aCanvas at: aPoint
	aCanvas
		drawOval: (Rectangle
			center: self position + (self extent * aPoint)
			extent: self extent / 6)
		color: Color black
		borderWidth: 0
		borderColor: Color transparent
\end{methods}

%=================================================================
\section{R\'esum\'e du chapitre}

Morphic est une librairie graphique dans laquelle les \'el\'ements de
l'interface graphique peuvent \^etre compos\'es dynamiquement.
Vous pouvez:
\begin{itemize}
  \item convertir un objet en \emph{morph} et l'afficher
    sur l'\'ecran de \pharo, le \emph{world}, en lui envoyant le message
    \ct{asMorph openInWorld};
  \item faire appara\^{\i}tre le halo Morphic en cliquant avec le
    bouton bleu de la souris sur un morph et manipuler ce morph
    gr\^ace aux poign\'ees du halo. Ces poign\'ees ont des ballons
    d'aide (ou \emph{help balloons}) qui d\'etaillent leur action;
  \item composer des morphs en les embo\^{\i}tant les uns dans les autres,
    soit par glisser-d\'eposer, soit par envoi du message \ct{addMorph:};
  \item d\'eriver la classe d'un morph et red\'efinir ses
    m\'ethodes-cl\'es telles que \ct{initialize} et \ct{drawOn:};
  \item contr\^oler la fa\c{c}on dont r\'eagit un morph avec les
    \'ev\'enements issus de la souris et du clavier en red\'efinissant les
    m\'ethodes comme, par exemple, \ct{handlesMouseDown:} et
    \ct{handlesMouseOver:};
  \item animer un morph en d\'efinissant les m\'ethodes \ct{step}
    (ce que fait le morph) et \ct{stepTime} (le nombre de
    millisecondes entre les pas);
  \item trouver diff\'erents morphs pr\'e-d\'efinis pour
    l'interactivit\'e de l'utilisateur comme
    \ct{PopUpMenu} ou \ct{FillInTheBlank};
%martial: ajout a signaler a andrew
  \item explorer les m\'ethodes graphiques des diff\'erents canevas,
    instances de \ct{Canvas} ou de sous-classes,
    pour exploiter leurs ressources pour le dessin des morphs.
\end{itemize}

%=================================================================
\ifx\wholebook\relax\else\end{document}\fi
%=================================================================

%-----------------------------------------------------------------

%=================================================================
%:PART 3 -- Advanced Pharo
\part{Pharo avanc\'e}
%:Metaclasses
% $Author: oscar $
% $Date: 2007-09-23 11:56:47 +0200 (Sun, 23 Sep 2007) $
% $Revision: 12130 $
%=================================================================
% Luc Fabresse (LF)
%% relecture (et reponse): Martial Boniou (remarque: j'ai mis les morphs au masculin apres mure reflexion.
% respecter la typographie française ? si oui, alors il faut revoir tous les itemize car il ne doit pas y avoir de majuscule après un ":"
%% reponse: normalement il ne doit pas y avoir de majuscules
% "System browser" laissé tel quel comme dans les autres chapitres ddéjà traduits
%% reponse: parfait
%  traduction de "class side"  j'ai mis "côté classe"
%% reponse: c'est ce que j'ai mis aussi
% problème de sens dans le SBE anglais: caption fig 12.3 : "The metaclasses of Translucent its superclasses"
%   traduit par : "Les méta-classes de TranslucentColor ET ses super-classes"
% relecture : Rene Mages 21-12-2007
% update: Tue Dec 25 22:32:45 CET 2007
% martial: convertir ``mon texte'' en «~mon texte~» pour franciser +
% j'ai mis des \ct{} au lieu de \lct{}
% relecture : Rene Mages 10-01-2008
% adaptation pour PBE - martial - Thu Sep 10 19:46:40 CEST 2009 from
% $Author: oscar $ $Date: 2009-08-17 12:10:55 +0200 (Mon, 17 Aug 2009) $ $Revision: 28502 $

\ifx\wholebook\relax\else
% --------------------------------------------
% Lulu:
	\documentclass[a4paper,10pt,twoside]{book}
	\usepackage[
		papersize={6.13in,9.21in},
		hmargin={.75in,.75in},
		vmargin={.75in,1in},
		ignoreheadfoot
	]{geometry}
	\input{../common.tex}
	\pagestyle{headings}
	\setboolean{lulu}{true}
% --------------------------------------------
% A4:
%	\documentclass[a4paper,11pt,twoside]{book}
%	\input{../common.tex}
%	\usepackage{a4wide}
% --------------------------------------------
    \graphicspath{{figures/} {../figures/}}
	\begin{document}
	\renewcommand{\nnbb}[2]{} % Disable editorial comments
	\sloppy
\fi
%=================================================================
\chapter{Classes et méta-classes}
\chalabel{metaclasses}

\on{The section on responsibilities of Class, Behavior and Metaclass are weak, and need to be fleshed out with convincing examples. Marcus, can you help?}

Comme nous l'avons vu dans~\charef{model}, en \st, \mantra, et tout objet est une instance d'une classe.
Les classes ne sont pas des cas particuliers:
les classes sont des objets, et les objets représentant les classes sont des instances d'autres classes.
Ce modèle objet capture l'essence de la programmation orientée objet: il est petit, simple, élégant et uniforme.
Cependant, les implications de cette uniformité peuvent prêter à confusion pour les débutants.
L'objectif de ce chapitre est de montrer qu'il n'y a rien de compliqué, de ``magique'' ou de spécial ici: juste des règles simples appliquées uniformément. 
En suivant ces règles, vous pourrez toujours comprendre le code, quelque soit la situation.

%=================================================================
\section{Les règles pour les classes et les méta-classes}

Le modèle objet de \st est basé sur un nombre limité de concepts appliqués uniformément.
Les concepteurs de \st ont appliqué le principe du ``rasoir d'Occam'': toute considération conduisant à un modèle plus complexe que nécessaire a été abandonnée.
Rappelons ici les règles du modèle objet qui ont été présentées dans~\charef{model}.

\begin{enumerate}[label={\textbf{Règle \arabic{*}}.}, ref={la règle~\arabic{*}}, leftmargin=*, widest=10]
% NB: rule labels must not be multiply defined across chapters!
\item{} % \rulelabel{everything}
	\Mantra.

\item{} % \rulelabel{instance}
	Tout objet est instance d'une classe.

\item{} % \rulelabel{inheritance}
	Toute classe a une super-classe.

\item{} % \rulelabel{message}
	Tout se passe par envoi de messages. 
% REVOIR dans PBE: Everything happens by message sends. -> Everything happens by sending messages.
\item{} % \rulelabel{lookup}
	La recherche de méthodes suit la chaîne d'héritage.
\end{enumerate}

Comme nous l'avons mentionné en introduction de ce chapitre, une conséquence de la \ruleref{everything} est que les \emph{classes sont des objets aussi}, dans ce cas la \ruleref{instance} dit que les classes sont obligatoirement des instances de classes.
La classe d'une classe est appelée une \emph{méta-classe}.

\seclabel{metaclassIntro}
Une \indmain{méta-classe} est automatiquement créée pour chaque nouvelle classe.
La plupart du temps, vous n'avez pas besoin de vous soucier ou de penser aux méta-classes.
Cependant, chaque fois que vous utilisez le Browser pour naviguer du  ``\subind{Browser}{côté classe}'' d'une classe, il est utile de se rappeler que vous êtes en train de naviguer dans une classe différente.
Une classe et sa méta-classe sont deux classes inséparables, même si la première est une instance de la seconde.
Pour expliquer correctement les classes et les méta-classes, nous devons étendre les règles du chapitre~\ref{cha:model} en ajoutant les règles suivantes:

\begin{enumerate}[label={\textbf{Règle \arabic{*}}.}, ref={Règle \arabic{*}}, leftmargin=*, widest=10]
\setcounter{enumi}{5}
\item{} \rulelabel{metaclass}
	Toute classe est une instance d'une méta-classe.

\item{} \rulelabel{parallelhierarchy}
	La \subind{méta-classe}{hiérarchie} des méta-classes est parallèle à celle des classes.

\item{} \rulelabel{behavior}
	Toute méta-classe hérite de \clsind{Class} et de \clsind{Behavior}.

\item{} \rulelabel{metaclassclass}
	Toute méta-classe est une instance de \clsind{Metaclass}.

\item{} \rulelabel{metaclassmetaclass}
	La méta-classe de \ct{Metaclass} est une instance de \ct{Metaclass}.

\end{enumerate}

% \noindent
%%Ensembles, 
Ensemble, ces 10 règles complètent le modèle objet de \st.
Nous allons tout d'abord revoir les 5 règles issues du chapitre~\ref{cha:model} à travers un exemple simple.
Ensuite, nous examinerons ces nouvelles règles à travers le même exemple.

%=================================================================
\section{Retour sur le modèle objet de \st}

Puisque \mantra, la couleur bleue est aussi un objet en \st.

\begin{code}{@TEST}
Color blue --> Color blue
\end{code}

\noindent
Tout objet est une instance d'une classe.
La classe de la couleur bleue est la classe \clsind{Color}:
\begin{code}{@TEST}
Color blue class --> Color
\end{code}

\noindent
Toutefois, si l'on fixe la valeur \emph{alpha} d'une couleur, nous obtenons une instance d'une classe différente, nommée \clsind{TranslucentColor}:
\begin{code}{@TEST}
(Color blue alpha: 0.4) class --> TranslucentColor
\end{code}

\noindent
Nous pouvons créer un morph et fixer sa couleur à cette couleur translucide:
\begin{code}{}
EllipseMorph new color: (Color blue alpha: 0.4); openInWorld
\end{code}
\noindent
Vous pouvez voir l'effet produit dans la \figref{translucentEllipse}.

\begin{center}
\begin{figure}[!ht]
\ifluluelse
	{\centerline {\includegraphics[scale=0.7]{TranslucentEllipse}}}
	{\centerline {\includegraphics[width=8cm]{TranslucentEllipse}}}
\caption{Une ellipse translucide.\figlabel{translucentEllipse}}
\end{figure}
\end{center}

%%Par la \ruleref{inheritance}, 
D'apr\`es la \ruleref{inheritance},
toute classe possède une super-classe.
La super-classe de \clsind{TranslucentColor} est \clsind{Color} et la super-classe de \ct{Color} est \clsind{Object}:
\begin{code}{@TEST}
TranslucentColor superclass --> Color
Color superclass                   --> Object
\end{code}

Comme tout se produit par \subind{message}{envoi} de messages
(\ruleref{message}), nous pouvons en déduire que \mthind{Color
  class}{blue} est un message à destination de \ct{Color};
\mthind{Object}{class} et \mthind{Color}{alpha:} sont des messages à
destination de la couleur bleue; \mthind{Morph}{openInWorld} est un
message à destination d'une ellipse morph et
\mthind{Behavior}{superclass} est un message à destination de
\ct{TranslucentColor} et \ct{Color}. % REVOIR PBE message send ->
                                % sending messages
Dans chaque cas, le receveur est un objet puisque \mantra bien que certains de ces objets soient aussi des classes.

La recherche de méthodes suit la chaîne d'héritage (\ruleref{lookup}), donc quand nous envoyons le message \ct{class} au résultat de 
\ct{Color blue alpha: 0.4}, le message est traité quand la méthode correspondante est trouvée dans la classe \ct{Object}, comme illustré par \figref{classmessage}.

\begin{center}
\begin{figure}[!ht]
\ifluluelse
	{\centerline{\includegraphics[width=\textwidth]{TranslucentClassMessage}}}
	{\centerline{\includegraphics[width=0.8\textwidth]{TranslucentClassMessage}}}
\caption{Envoyer un message à une couleur translucide.\figlabel{classmessage}}
\end{figure}
\end{center}

Cette figure capture l'essence de la relation \emphind{est un}{}(e).
Notre objet bleu translucide \emphind{est un}{}e instance de
\ct{TranslucentColor}, mais nous pouvons aussi dire qu'il
\emph{est un}{}e  \ct{Color} et qu'il \emph{est un} \ct{Object}, puisqu'il répond aux messages définis dans toutes ces classes.
En fait, il y a un message, \mthind{Object}{isKindOf:}, qui peut être envoyé à n'importe quel objet pour déterminer s'il est en relation ``\emph{est un}'' avec une classe donnée:

\needlines{4}
\begin{code}{@TEST | translucentBlue |}
translucentBlue := Color blue alpha: 0.4.
translucentBlue isKindOf: TranslucentColor --> true
translucentBlue isKindOf: Color                    --> true
translucentBlue isKindOf: Object                  --> true
\end{code}

%=================================================================
\section{Toute classe est une instance d'une méta-classe}

% \ruleref{metaclass}

Comme nous l'avons mentionné dans \secref{metaclassIntro}, les classes dont les instances sont aussi des classes sont appelées des \ind{méta-classe}{}s.

\paragraph{Les méta-classes sont implicites.}
Les méta-classes sont automatiquement créées quand une classe est définie.
On dit qu'elles sont \emph{implicites} car en tant que programmeur, vous n'avez jamais à vous en soucier.
Une méta-classe \subind{méta-classe}{implicite} est créée pour chaque classe que vous créez donc chaque méta-classe n'a qu'une seule instance.

Alors que les classes ordinaires sont nommées par des variables globales, les méta-classes sont anonymes.
Cependant, nous pouvons toujours les référencer à travers la classe qui est leur instance.
Par exemple, la classe de \clsind{Color} est \clsind{Color class} et la classe de \ct{Object} est \clsind{Object class}:
\begin{code}{@TEST}
Color class   --> Color class
Object class --> Object class
\end{code}

\noindent
\Figref{translucentmetaclasses} montre que chaque classe est une instance de sa méta-classe (\subind{méta-classe}{anonyme}).

\begin{center}
\begin{figure}[!ht]
\ifluluelse
	{\centerline {\includegraphics[width=\textwidth]{TranslucentMetaclasses}}}
	{\centerline {\includegraphics[width=0.8\textwidth]{TranslucentMetaclasses}}}
\caption{Les méta-classes de TranslucentColor et ses super-classes.\figlabel{translucentmetaclasses}}
\end{figure}
\end{center}


Le fait que les classes soient aussi des objets facilite leur interrogation par envoi de messages.
Voyons cela:
\begin{code}{@TEST}
Color subclasses                           --> {TranslucentColor}
TranslucentColor subclasses         --> #()
TranslucentColor allSuperclasses  --> an OrderedCollection(Color Object ProtoObject)
TranslucentColor instVarNames     --> #('alpha')
TranslucentColor allInstVarNames --> #('rgb' 'cachedDepth' 'cachedBitPattern' 'alpha')
TranslucentColor selectors             -->  an IdentitySet(#pixelValueForDepth: #pixelWord32
#convertToCurrentVersion:refStream: #isTransparent #scaledPixelValue32 #bitPatternForDepth: #storeArrayValuesOn: #setRgb:alpha: #alpha #isOpaque #pixelWordForDepth: #isTranslucentColor #hash #isTranslucent #alpha: #storeOn: #asNontranslucentColor #privateAlpha #balancedPatternForDepth:)
\end{code}
\cmindex{Class}{subclasses}
\cmindex{Behavior}{allSuperclasses}
\cmindex{Behavior}{instVarNames}
\cmindex{Behavior}{allInstVarNames}
\cmindex{Behavior}{selectors}


%=================================================================
\section{La hiérarchie des méta-classes est parallèle à celle des classes}
% \ruleref{parallelhierarchy}

La \ruleref{parallelhierarchy} dit que la super-classe d'une méta-classe ne peut pas être une classe arbitraire: elle est contrainte à être la méta-classe de la super-classe de l'unique instance de cette méta-classe.
\begin{code}{@TEST}
TranslucentColor class superclass --> Color class
TranslucentColor superclass class --> Color class
\end{code}

\noindent
C'est ce que nous voulons dire par le fait que la \subindmain{méta-classe}{hiérarchie} des méta-classes est parallèle à la hiérarchie des classes; \figref{parallelHierarchies} montre comment cela fonctionne pour la hiérarchie de \clsind{TranslucentColor}.
 
\begin{center}
\begin{figure}[!ht]
\ifluluelse
	{\centerline {\includegraphics[width=\textwidth]{TranslucentMetaclassHierarchy}}}
	{\centerline {\includegraphics[width=0.8\textwidth]{TranslucentMetaclassHierarchy}}}
\caption{La hiérarchie des méta-classes est parallèle à la hiérarchie des classes.\figlabel{parallelHierarchies}}
\end{figure}
\end{center}

\begin{code}{@TEST}
TranslucentColor class                                     --> TranslucentColor class
TranslucentColor class superclass                   --> Color class
TranslucentColor class superclass superclass --> Object class
\end{code}

\paragraph{L'uniformité entre les classes et les objets.}
Il est intéressant de revenir en arrière un moment et de réaliser qu'il n'y a pas de différence entre envoyer un message à un objet et à une classe.
Dans les deux cas, la recherche de la méthode correspondante commence dans la classe du receveur et chemine le long de le chaîne d'héritage.

Ainsi, les messages envoyés à des classes doivent suivre la chaîne d'héritage des méta-classes. 
Considérons, par exemple, la méthode \mthind{Color class}{blue} qui est implémentée du \subind{Browser}{côté classe} de \ct{Color}.
Si nous envoyons le message \ct{blue} à \ct{TranslucentColor}, alors il sera traité de la même façon que les autres messages.
La recherche commence dans \ct{TranslucentColor class} et continue dans la hiérarchie des méta-classes jusqu'à trouver dans  \ct{Color class} (voir \figref{metaclasslookup}).

\begin{code}{@TEST}
TranslucentColor blue --> Color blue
\end{code}
\noindent
Notons que l'on obtient comme résultat un \ct{Color blue}  ordinaire, et non pas un translucide\,---\,il n'y a pas de magie!

\begin{center}
\begin{figure}[!ht]
\ifluluelse
	{\centerline {\includegraphics[width=\textwidth]{TranslucentColorBlue}}}
	{\centerline {\includegraphics[width=0.8\textwidth]{TranslucentColorBlue}}}
\caption{Le traitement des messages pour les classes est le même que pour les objets ordinaires.\figlabel{metaclasslookup}}
\end{figure}
\end{center}

Nous voyons donc qu'il y a une \subind{méthode}{recherche} de méthode uniforme en \st.
Les classes sont juste des objets et se comportent comme tous les autres objets.
Les classes ont le pouvoir de créer de nouvelles instances uniquement parce qu'elles répondent au message \ct{new} et que la méthode pour \ct{new} sait créer de nouvelles instances.
Normalement, les objets qui ne sont pas des classes ne comprennent pas
ce message, mais si vous avez une bonne raison pour faire cela, il n'y a rien qui vous empêche d'ajouter une méthode \ct{new} à une classe qui n'est pas une méta-classe.

Comme les classes sont des objets, nous pouvons aussi les inspecter.
\index{inspecteur}

\dothis{Inspectez \ct{Color blue} et \ct{Color}.}

%martial: la tournure reste lourde: Tue Dec 25 20:50:20 CET 2007
\noindent
Notons que vous inspectez, dans un cas, une instance de \ct{Color} et
dans l'autre cas, la classe \ct{Color} elle-même.
Cela peut prêter à confusion parce que la barre de titre de l'inspecteur contient le nom de la \emph{classe} de l'objet en cours d'inspection.
L'inspecteur sur \ct{Color} vous permet de voir 
%% ojout
entre autre
la super-classe, les variables d'instances, le \subind{méthode}{dictionnaire} des méthodes 
%% etc.
de  la classe \ct{Color}, comme indiqu\'e dans \figref{inspectingColor}.

\begin{center}
\begin{figure}[!ht]
\ifluluelse
	{\centerline{\includegraphics[width=\textwidth]{InspectingColor}}}
	{\centerline{\includegraphics[width=10cm]{InspectingColor}}}
\caption{Les classes sont aussi des objets.\figlabel{inspectingColor}}
\end{figure}
\end{center}


%=================================================================
\section{Toute m{\'e}ta-classe h{\'e}rite de \ct{Class} et de \ct{Behavior}}
% \ruleref{behavior}

Toute \ind{méta-classe} \emphind{est un}{}e classe, donc hérite de \clsind{Class}.
À son tour, \ct{Class} hérite de ses super-classes, \clsind{ClassDescription} et \clsind{Behavior}.
En \st, puisque tout \emph{est un} objet, ces classes héritent finalement toutes de \ct{Object}.
Nous pouvons voir le schéma complet dans \figref{inheritbehavior}.

\begin{center}
\begin{figure}
\ifluluelse
	{\centerline{\includegraphics[width=\textwidth]{TranslucentBehavior}}}
	{\centerline{\includegraphics[width=0.8\textwidth]{TranslucentBehavior}}}
\caption{Les méta-classes héritent de \ct{Class} et de \ct{Behavior}.\figlabel{inheritbehavior}}
\end{figure}
\end{center}


\paragraph{Où est défini \ct{new}?}
Pour comprendre l'importance du fait que les méta-classes héritent de \ct{Class} et de \ct{Behavior}, 
%% cela aide de demander 
demandons-nous
où est défini \ct{new} et comment cette définition est trouvée.
Quand le message \ct{new} est envoyé à une classe, il est recherché dans sa chaîne de méta-classes et finalement dans ses super-classes \ct{Class}, \ct{ClassDescription} et \ct{Behavior} comme montré dans \figref{sendingnew}.

La question ``\emph{Où est défini \ct{new}?}'' est cruciale.
La m\'ethode \mthind{Behavior}{new} est définie en premier dans la classe \ct{Behavior} et peut être redéfinie dans ses sous-classes, ce qui inclut toutes les méta-classes des classes que nous avons définies, si cela est nécessaire.
Maintenant, quand un message \ct{new} est envoyé à une classe, il est recherché, comme d'habitude, dans la méta-classe de cette classe, en continuant le long de la chaîne de super-classes jusqu'à la classe \ct{Behavior} si aucune redéfinition n'a été rencontrée sur le chemin.

Notons que le résultat de l'envoi de message \ct{TranslucentColor new} est une instance de  \clsind{TranslucentColor}  et \emph{non} de \ct{Behavior}, même si la méthode est trouvée dans la classe \ct{Behavior}!  \ct{new} retourne toujours une instance de \self, la classe qui a reçu le message, même si cela est implémenté dans une autre classe.

\begin{code}{@TEST}
TranslucentColor new class --> TranslucentColor    "et non pas Behavior!"
\end{code}

\begin{center}
\begin{figure}
\ifluluelse
	{\centerline{\includegraphics[width=\textwidth]{TranslucentSendingNew}}}
	{\centerline{\includegraphics[width=0.8\textwidth]{TranslucentSendingNew}}}
\caption{\ct{new} est un message ordinaire recherché dans la chaîne des méta-classes.\figlabel{sendingnew}}
\end{figure}
\end{center}

Une erreur courante est de rechercher \ct{new} dans la super-classe de la classe du receveur.
La même chose se produit pour \ct{new:}, le message standard pour créer un objet d'une taille donnée.
Par exemple, \ct{Array new: 4} crée un tableau de 4 éléments.
Vous ne trouverez pas la définition de cette méthode dans \ct{Array} ni 
%% ajout
dans
aucune de ses super-classes.
À la place, vous devriez regarder dans \ct{Array class} et ses super-classes puisque c'est là que la recherche commence.

\on{The text below needs more details and convincing examples ...}

\paragraph{Les responsabilités de \ct{Behavior}, \ct{ClassDescription} et \ct{Class}.}
\clsind{Behavior} fournit l'état minimum et nécessaire à des objets possédant des instances: cela inclut un lien super-classe, un dictionnaire de méthodes et une description des instances (\ie représentation et nombre).
\on{not sure I understand the last point}% LF same for me...
\ct{Behavior} hérite de \ct{Object}, donc elle, ainsi que toutes ses sous-classes peuvent se comporter comme des objets. 

\ct{Behavior} est aussi l'interface basique pour le compilateur.
Elle fournit des méthodes pour créer un dictionnaire de méthodes, compiler des méthodes, créer des instances (\ie \mthind{Behavior}{new}, \mthind{Behavior}{basicNew}, \mthind{Behavior}{new:}, et \mthind{Behavior}{basicNew:}),
manipuler la hiérarchie de classes (\ie
\mthindex{Behavior}{superclass:} \lct{superclass:}, % CHANGE
 \mthind{Behavior}{addSubclass:}), 
accéder aux méthodes (\ie \mthind{Behavior}{selectors}, \lmthind{Behavior}{allSelectors}, \mthind{Behavior}{compiledMethodAt:}),
accéder aux instances et aux variables (\ie \mthind{Behavior}{allInstances}, \mthind{Behavior}{instVarNames}\ldots),
accéder à la hiérarchie de classes (\ie \mthind{Behavior}{superclass}, \mthind{Behavior}{subclasses})
et interroger  (\ie \mthind{Behavior}{hasMethods}, \mthind{Behavior}{includesSelector}, \mthind{Behavior}{canUnderstand:}, \mthind{Behavior}{inheritsFrom:}, \mthind{Behavior}{isVariable}).

\clsind{ClassDescription} est une classe abstraite qui fournit des facilités utilisées par ses deux sous-classes directes,  \clsind{Class} et \clsind{Metaclass}.
\clsind{ClassDescription} ajoute des facilités fournies à la base par \ct{Behavior}:
des variables d'instances nommées,
la catégorisation des méthodes dans des protocoles,
la notion de nom (abstrait),
la maintenance de \changesets, la journalisation des changements
et la plupart des mécanismes requis pour l'exportation de \changesets.


\clsind{Class} représente le comportement commun de toutes les classes.
Elle fournit un nom de classe, des méthodes de compilation, des méthodes de stockage et des variables d'instance.
Elle fournit aussi  une représentation concrète pour les noms des variables de classe et des variables de pool (\mthind{Class}{addClassVarName:}, \mthind{Class}{addSharedPool:}, \mthind{Class}{initialize}).
\ct{Class} sait comment créer des instances donc toutes les méta-classes doivent finalement hériter de \ct{Class}.


%=================================================================
\section{Toute méta-classe est une instance de \ct{Metaclass}}
% \ruleref{metaclassclass}

Les méta-classes sont aussi des objets; elles sont des instances de la classe \clsind{Metaclass} comme montré dans \figref{metaclassclass}.
Les instances de la classe \ct{Metaclass} sont les méta-classes anonymes; chacune ayant exactement une unique instance qui est une classe.

\begin{center}
\begin{figure}
\ifluluelse
	{\centerline{\includegraphics[width=\textwidth]{TranslucentMetaclassClass}}}
	{\centerline{\includegraphics[width=0.8\textwidth]{TranslucentMetaclassClass}}}
\caption{Toute méta-classe est une \ct{Metaclass}.\figlabel{metaclassclass}}
\end{figure}
\end{center}


\ct{Metaclass} représente le comportement commun des méta-classes.
Elle fournit des méthodes pour la création d'instances (\ct{sub\-class\-Of:}) permettant de créer des instances initialisées de l'unique instance \ct{Metaclass} pour l'initialisation des variables de classe, la compilation de méthodes et l'obtention d'informations à propos des classes (liens d'héritage, variables d'instance, \etc).

% It provides methods for instance creation (\ct{sub\-class\-Of:})
% creating initialized instances of the metaclass's sole instance,
% initialization of class variables,
% metaclass instance, % LF "metaclass instance," non traduit car non compris dans la phrase anglaise ...
% % (\ct{name:inEnvironment:subclassOf:})
% % Actually, this is in ClassBuilder, it seems!
% method compilation, % (different semantics can be introduced)
% and class information (inheritance links, instance variables, \etc).
% \ab{This is too cryptic for me.} % LF for me too !

%=================================================================
\section{La méta-classe de \ct{Metaclass} est une instance de \ct{Metaclass}}
% \ruleref{metaclassmetaclass}

La dernière question à laquelle il faut répondre est: quelle est la classe de \clsind{Metaclass class}?
La réponse est simple: il s'agit d'une méta-classe, donc forcément une instance de \ct{Metaclass}, exactement comme toutes les autres méta-classes dans le système (voir \figref{metaclassclassclass}).


\begin{center}
\begin{figure}
\ifluluelse
	{\centerline{\includegraphics[width=\textwidth]{TranslucentMetaclassClassClass}}}
	{\centerline{\includegraphics[width=0.8\textwidth]{TranslucentMetaclassClassClass}}}
\caption{Toutes les méta-classes sont des instances de la classe \ct{Metaclass},  même la méta-classe de \ct{Metaclass}. \figlabel{metaclassclassclass}}
\end{figure}
\end{center}

La figure montre que toutes les méta-classes sont des instances de \ct{Metaclass}, ce qui inclut aussi la méta-classe de \ct{Metaclass}.
Si vous comparez les figures \ref{fig:metaclassclass} et \ref{fig:metaclassclassclass}, vous verrez comment la \subind{méta-classe}{hiérarchie} des méta-classes reflète parfaitement la hiérarchie des classes, tout le long du chemin jusqu'à \ct{Object class}.


Les exemples suivants montrent comment il est possible d'interroger la hiérarchie de classes afin de démontrer que \figref{metaclassclassclass} est correcte.
%martial: j'ai enleve les parentheses
En réalité, vous verrez que nous avons dit un pieux mensonge\,---\,\ct{Object class superclass -->} {\clsind{ProtoObject class}, et non \ct{Class}. En \pharo, il faut aller une super-classe plus haut dans la hiérarchie pour atteindre \ct{Class}.

\needlines{2}
\begin{example}{La hiérarchie des classes}{@TEST}
TranslucentColor superclass			-->		Color
Color superclass							-->		Object
\end{example}

\needlines{4}
% Martial: trouver mieux que "saute". Les ! sont nécessaires pour l'espace avant le : 
\begin{example}{La hi{\'e}rarchie parall{\`e}le des méta-classes}{@TEST}
TranslucentColor class superclass   --> Color class
Color class superclass                     --> Object class
Object class superclass superclass --> Class    "!Attention: saute ProtoObject class!"
Class superclass                              --> ClassDescription
ClassDescription superclass            --> Behavior
Behavior superclass                         --> Object
\end{example}

\begin{example}{Les instances de \ct{Metaclass}}{@TEST}
TranslucentColor class class --> Metaclass
Color class class                   --> Metaclass
Object class class                 --> Metaclass
Behavior class class              --> Metaclass
\end{example}

\begin{example}{\ct{Metaclass class} est une \ct{Metaclass}}{@TEST}
Metaclass class class --> Metaclass
Metaclass superclass --> ClassDescription
\end{example}

%=================================================================
\section{Résumé du chapitre}

Maintenant, vous devriez mieux comprendre la façon dont les classes sont organisées et l'impact de l'uniformité du modèle objet.
Si vous vous perdez ou vous embrouillez, vous devez toujours vous rappeler que l'envoi de messages est la clé: cherchez alors la méthode dans la classe du receveur.
Cela fonctionne pour \emph{tous} les receveurs.
Si une méthode n'est pas trouvée dans la classe du receveur, elle est recherchée dans ses super-classes.


\begin{enumerate}

\item Toute classe est une instance d'une méta-classe.
Les méta-classes sont implicites.
Une méta-classe est créée automatiquement à chaque fois que vous créez une classe; cette dernière étant sa seule instance.


\item La \subind{méta-classe}{hiérarchie} des méta-classes est parallèle à celle des classes.
La recherche de méthodes pour les classes est analogue à la recherche de méthodes pour les objets ordinaires et suit la chaîne des super-classes entre méta-classes.


\item Toute méta-classe hérite de \clsind{Class} et de \clsind{Behavior}.
Toute classe \emph{est un}{}e \ct{Class}.
Puisque les méta-classes sont aussi des classes, elles doivent hériter de  \ct{Class}.
	\ct{Behavior} fournit un comportement commun à toutes les entités ayant des instances.

\item Toute méta-classe est une instance de \clsind{Metaclass}.
	\ct{ClassDescription} fournit tout ce qui est commun à \ct{Class} et \`a \ct{Metaclass}.
	

\item La méta-classe de \ct{Metaclass} est une instance de \ct{Metaclass}.
	La relation \emph{instance-de} forme une boucle fermée, donc \ct{Metaclass class class --> Metaclass}.
\end{enumerate}

%=================================================================
\ifx\wholebook\relax\else\end{document}\fi
%=================================================================

%-----------------------------------------------------------------

%:Reflectivité
% $Author: oscar $
% $Date: 2009-10-07 14:15:45 +0200 (Wed, 07 Oct 2009) $
% $Revision: 29389 $

% HISTORY: [see also Metaprogramming2.tex]
% 2007-05-22 - Damien Pollet started (translation from French article by ...?)
% 2008-01-15 - Alex added text
% 2008-12-15 - Oscar revised
% 2009-03-24 - Stef started new chapter (acttalk ... see separate file)
% 2009-06-01 - Oscar started to revise again and add new material
% 2009-06-08 - Lukas -- unsent messages
% 2009-06-15 - Oscar completed revision
% 2009-06-16 _ Stef comments
% 2009-06-17 - Alexandre completed revision
% 2009-06-19 - Lukas comments
% 2009-07-07 - Oscar migrated to Pharo; fixed broken tests
% 2009-08-16 - Oscar indexing and cleaning up loose ends

%=================================================================
\ifx\wholebook\relax\else
% --------------------------------------------
% Lulu:
	\documentclass[a4paper,10pt,twoside]{book}
	\usepackage[
		papersize={6.13in,9.21in},
		hmargin={.75in,.75in},
		vmargin={.75in,1in},
		ignoreheadfoot
	]{geometry}
	\input{../common.tex}
	\pagestyle{headings}
	\setboolean{lulu}{true}
% --------------------------------------------
% A4:
%	\documentclass[a4paper,11pt,twoside]{book}
%	\input{../common.tex}
%	\usepackage{a4wide}
% --------------------------------------------
    \graphicspath{{figures/} {../figures/}}
	\begin{document}
	% \renewcommand{\nnbb}[2]{} % Disable editorial comments
	\sloppy
\fi

%=================================================================
\chapter{La réflexivité}\chalabel{reflection}

%\alex{I addressed lukas comments using editorial macros. Please, if you read this, remove this comment and remove the editorial macros, else their purpose is diminished}

\indexmain{reflection}
\st est un langage de programmation réflexif. Brièvement, cela signifie que les programmes ont la possibilité d'agir sur leur propre exécution et structure.
% \lr{not only on execution, also on the static model}
Plus techniquement, cela signifie que les \emphind{méta-objets} du système en cours d'exécution peuvent être \emph{réifiés} sous forme d'objets ordinaires qui peuvent alors recevoir des requêtes et être inspectés.
Les méta-objets dans \st sont les classes, méta-classes, dictionnaires de méthodes, méthodes compilées, pile d'exécution, etc ...
Cette forme de réflexivité est également appelée \emphind{introspection} et est disponible dans de nombreux langages de programmation.

\begin{figure}[ht]\centering
	\includegraphics[width=\linewidth]{reflect}
	\caption{Réification et réflexivité.\figlabel{reflect}} % \lr{not referenced, not sure if I understand it}
\end{figure}
Inversement il est possible en \st de modifier les méta-objets réifiés et que leurs modifications soient prises en compte lors de l'exécution du système (voir \figref{reflect}).
Ceci est appelé \emph{intercession} et est principalement utilisé dans les langages de programmation dynamique et de manière plus limitée dans les langages statiques.

Un programme qui manipule d'autres programmes (ou même lui-même) est un \emphind{méta-programme}.
Pour qu'un langage de programmation soit réflexif, il faut qu'il supporte à la fois l'\ind{introspection} et l'\ind{intercession}.

L'introspection est la capacité \emph{d'examiner} les structures de données qui définissent le programme comme les objets, classes, méthodes ou pile d'exécution.
L'intercession est la capacité de modifier ces structures, en d'autre terme de changer la sémantique du langage et le comportement d'un programme depuis le programme lui-même.
La \emph{réflexivité structurelle} s'intéresse à l'exploration et à la modification des structures lors de l'exécution du système, alors que la \emph{réflexivité de comportement} concerne l'interprétation de ces structures.

Dans ce chapitre, nous allons principalement nous intéresser à la \ind{réflexivité structurelle}.
Nous illustrerons avec plusieurs exemples pratiques comment \st supporte l'introspection et la méta-programmation.

%======================================
\section{Introspection}

En utilisant un inspecteur, il est possible d'examiner un objet, de changer les valeurs de ses variables d'instances et même de lui envoyer un message.

\dothis{Évaluer le code qui suit dans un workspace:}
\begin{code}{| w |}
w := Workspace new.
w openLabel: 'My Workspace'.
w inspect
\end{code}

Ceci va ouvrir un deuxième workspace et un inspecteur.
L'inspecteur montre l'état interne de ce nouveau workspace : la liste de ses variables d'instances dans la partie gauche (\ct!dependents!, \ct!contents!, \ct!bindings!...) et la valeur de la variable d'instance sélectionnée dans la partie droite.
La variable d'instance \ct!contents! représente ce que le workspace affiche dans sa zone de texte. Ainsi si vous la sélectionnez, la partie droite montrera une chaîne de caractères vide.

\begin{figure}[ht]\centering
	\includegraphics[width=\linewidth]{workspaceInspector}
	\caption{Inspecter un \ct!Workspace!.\figlabel{workspaceInspector}}
\end{figure}

\dothis{Maintenant tapez \ct!'hello'! à la place de la chaîne de caractères vide, puis ensuite faites \emph{accept}.}
La valeur de la variable \ct!contents! change, mais la fenêtre du workspace n'en sera pas notifiée, c'est-à-dire qu'elle ne ré-affiche pas son contenu.
Afin d'activer le rafraîchissement de la fenêtre, évaluez \ct!self contentsChanged! dans la partie inférieure de l'inspecteur.

%-----------------------------------------------------------------
\subsection{Accéder aux variables d'instances}

Comment fonctionne l'inspecteur ?
En \st, toutes les variables d'instances sont protégées.
En théorie, il est impossible d'y accéder depuis un autre objet si la classe ne définit pas d'accesseur.
En pratique, l'inspecteur peut accéder aux variables sans avoir besoin d'accesseurs, parce qu'il utilise les capacités réflexives de \st.
En \st, les classes définissent les variables d'instances soient par nom ou au moyen d'indices numériques.
L'inspecteur utilisent des méthodes définies dans la classe \ct!Object! afin d'y accéder :  \lct{instVarAt: \emph{index}} et \lct{instVarNamed: \emph{aString}} peuvent être utilisé pour avoir respectivement la valeur de la variable d'instance à la position \lct{\emph{index}} ou identifié par \lct{\emph{aString}}; pour associer de nouvelles valeurs à ces variables d'instances, on utilise \ct!instVarAt:put:! et \ct!instVarNamed:put:!.
\mthindex{Object}{instVarAt:}
\mthindex{Object}{instVarNamed:}
\mthindex{Object}{instVarAt:put:}
\mthindex{Object}{instVarNamed:put:}

Par exemple, vous pouvez changer la valeur de la variable d'instance \ct!contents! de \ct!w! précédemment définie en évaluant :
\begin{code}{}
w instVarNamed: 'contents' put: 'howdy!'; contentsChanged
\end{code}

\important{\emph{Caveat:} Bien que ces méthodes soient utiles pour construire les outils de l'environnement de développement, les utiliser dans le cadre d'une application conventionnelle est une mauvaise idée: ces méthodes réflexives rompent l'encapsulation des objets et peuvent rendre votre code plus difficile à comprendre et à maintenir.}
% \lr{Why? The access does not show up when looking for all readers/writers in the code browser.}

\ct!instVarAt:! et \ct!instVarAt:put:! sont des \ind{méthodes primitives}, c'est-à-dire qu'elles sont implémentées comme des opérations primitives de la machine virtuelle \pharo.
Si vous consultez le code de ces méthodes, la syntaxe spéciale d'un \ind{pragma} \ct!<primitive: N>! où \ct!N! est un entier.
% \lr{actually this is the syntax of pragmas (method annotations), \ct!primitive:! is just a special kind of pragma}

\needlines{5}
\begin{code}{}
Object>>>instVarAt: index 
	"Primitive. Answer a fixed variable in an object. ..."
	!\textbf{<primitive: 73>}!
	"Access beyond fixed variables."
	^self basicAt: index - self class instSize		
\end{code}

Généralement le code après l'invocation de la primitive n'est pas exécuté. Il est exécuté seulement si la primitive échoue. Dans le cas qui nous intéresse, si on essaie d'accéder à une variable qui n'existe pas, alors le code qui suit la primitive sera essayé. Ceci permet aussi d'utiliser le débogueur sur des méthodes primitives.
Bien qu'il soit possible de modifier le code des méthodes primitives, ceci est risqué pour la stabilité de votre image \pharo.

\begin{figure}[ht]\centering
	\includegraphics[width=\linewidth]{allInstanceVariables}
	\caption{Afficher toutes les variables d'instance d'un \ct!Workspace!.\figlabel{allInstanceVariables}}
\end{figure}

\figref{AllInstanceVariables} montre comment afficher les valeurs des variables d'instances d'une instance (\ct!w!)  de la classe \ct!Workspace!.
La méthode \ct!allInstVarNames! retourne l'ensemble des noms de variables d'instances d'une classe donnée.

De la même façon, il est possible de collecter les instances qui ont certaines propriétés.
Par exemple, pour avoir toutes les instances de la classe \ct!SketchMorph! dont la variable d'instance \ct!owner! est initialisée avec un morph de type world (\ie les morphs qui sont affichés à chaque instant à l'écran), essayez cette expression:
\begin{code}{}
SketchMorph allInstances select: [:c | (c instVarNamed: 'owner') isWorldMorph]
\end{code}

%-----------------------------------------------------------------
\subsection{Parcourir les variables d'instances}

\mthindex{Object}{instanceVariableValues}
Considérons le message \ct!instanceVariableValues!, qui retourne une collection de toutes les valeurs des variables d'instances définies dans la classe, en excluant les variables d'instances héritées.
Par exemple :
\begin{code}{@TEST}
(1@2) instanceVariableValues --> an OrderedCollection(1 2)
\end{code}

La méthode est implémentée dans \ct{Object} de la manière qui suit :
\needlines{9}
\begin{code}{}
Object>>>instanceVariableValues
	"Answer a collection whose elements are the values of those instance variables of the receiver which were added by the receiver's class."	
	| c |
	c := OrderedCollection new.
	self class superclass instSize + 1
		to: self class instSize
		do: [ :i | c add: (self instVarAt: i)].
	^ c
\end{code}

Cette méthode parcourt par indice les variables d'instances que la classe définit, à partir du dernier index utilisé par les superclasses.
(La méthode \ct!instSize! retourne le nombre des variables d'instances nommées que la classe définit.)

%-----------------------------------------------------------------
\subsection{Faire des requêtes sur les classes et interfaces}

Les outils de développement \pharo (navigateur de code, débogueur, inspecteur, ...) utilisent tous les mécanismes réflexifs que nous avons vu jusqu'à présent.

Voici quelques messages supplémentaires qui peuvent être utiles afin de construire des outils de développement :

\lct{isKindOf: \emph{aClass}} retourne vrai si le receveur est une instance de \lct{\emph{aClass}} ou d'une de ses superclasses.
Par exemple :
\begin{code}{@TEST}
1.5 class                     --> Float
1.5 isKindOf: Number --> true
1.5 isKindOf: Integer   --> false
\end{code}
\mthindex{Object}{class}
\mthindex{Object}{isKindOf:}

\lct{respondsTo: \emph{aSymbol}} retourne vrai si le receveur a une méthode dont le sélecteur est \lct{\emph{aSymbol}}.
Par exemple :
\needlines{3}
\begin{code}{@TEST}
1.5 respondsTo: #floor      --> true    "car Number !implémente! floor"
1.5 floor                            --> 1
Exception respondsTo: #, --> true    "exception classes can be grouped"
\end{code}
\mthindex{Object}{respondsTo:}

\important{\emph{Caveat:} Bien que tous ces outils soient particulièrement utiles pour construire des outils de développements, ils ne sont pas appropriés pour une application classique.
Demander à un objet sa classe ou bien l'interroger pour connaître les messages qu'il comprend, sont un signe indiquant une mauvaise conception objet, puisque généralement cela signifie une violation du principe d'encapsulation.
Les outils de développements ne sont pas considérés comme des applications comme les autres, puisque leur domaine d'applications porte sur le logiciel. Ces outils doivent nécessairement accéder aux détails internes du code.}

%There also exist mechanisms for introspecting on various parts of the run-time system, such as  the process scheduler, the memory manager and so on. For now we will focus on navigating through objects, classes and methods, and we will look more closely at rest of the runtime system in an other chapter.
%\on{let's not mention this if we don't actually write such a chapter!}

%-----------------------------------------------------------------
\subsection{Métriques de code}

Voyons maintenant comment nous pouvons utiliser les mécanismes d'introspection de \st pour rapidement pouvoir construire des métriques de code. Les \ind{métriques} de code mesure certains aspects comme la profondeur de l'arbre d'héritage, le nombre de sous-classes directes ou indirectes, le nombre de méthodes ou de variables d'instances de chaque classe ou enfin le nombre local de méthodes ou de variables d'instances.
Voici quelques résultats de métriques pour la classe \ct!Morph!, qui est la superclasse de tous les objets graphiques de \pharo, révélant qu'il s'agit d'une classe d'une taille conséquente et qu'elle est la racine d'une hiérarchie importante. Peut-être qu'elle nécessiterait d'être refactorisé !

\mthindex{Behavior}{allSuperclasses}
\mthindex{Behavior}{allSelectors}
\mthindex{Behavior}{allInstVarNames}
\mthindex{Behavior}{selectors}
\mthindex{Behavior}{instVarNames}
\mthindex{Behavior}{subclasses}
\mthindex{Behavior}{allSubclasses}
\mthindex{ClassDescription}{linesOfCode}
\begin{code}{}
Morph allSuperclasses size.  -->       2 "profondeur d'!héritage!"
Morph allSelectors size.        --> 1378 "nombre de !méthodes!"
Morph allInstVarNames size. -->      6 "nombre de variables d'instances"
Morph selectors size.             -->  998 "nombre de nouvelles !méthodes!"
Morph instVarNames size.     -->      6 "nombre de nouvelles variables"
Morph subclasses size.          -->    45 "sous-classes directes"
Morph allSubclasses size.      -->  326 "total de sous-classes"
Morph linesOfCode.               --> 5968 "nombre total de lignes de codeBANG"
\end{code}

Une des métriques les plus intéressantes dans le domaine de la programmation par objet est le nombre de méthodes qui étendent les méthodes héritées de la superclasse. Ceci nous informe de la relation entre une classe et ses superclasses.
Dans les prochaines sections, nous verrons comment exploiter notre connaissance des mécanismes d'exécution pour répondre à de telles questions.

%======================================
\section{Parcourir le code}

En \st, tout est objet. Les classes sont en particulier des objets qui fournissent des mécanismes utiles afin de parcourir leurs instances.
La plupart des messages que nous allons voir maintenant sont implémentés dans la classe \ct{Behavior}. Ils sont donc compris de toutes les classes.

Comme nous avons vu précédemment, nous pouvons obtenir une instance particulière d'une classe donnée en lui envoyant le message \ct!#someInstance!.
\mthindex{Behavior}{someInstance}
\begin{code}{@TEST} % Possibly fragile!
Point someInstance --> 0@0
\end{code}

Vous pouvons également rassembler toutes les instances avec \ct!#allInstances! ou déterminer le nombre d'instances en mémoire avec \ct!#instanceCount!.

%\alex{In a Pharo0.1-10342dev09.96.3, I have "ByteString instanceCount --> 63607"}
\mthindex{Behavior}{allInstances}
\mthindex{Behavior}{instanceCount}
\mthindex{Behavior}{allSubInstances}
\begin{code}{} % Cannot test this
ByteString allInstances        --> #('collection' 'position'  ...)
ByteString instanceCount    --> 104565
String allSubInstances size -->  101675
\end{code}

Ces caractéristiques peuvent être très utiles lors du deboguage d'une application, car il est possible de demander à une classe d'énumérer les méthodes possédant des propriétés spécifiques.
\begin{itemize}
\item \mthind{Behavior}{whichSelectorsAccess:} retourne la liste de tous les sélecteurs de méthodes qui lisent ou écrivent dans une variable dont le nom est passé en argument
\item \mthind{Behavior}{whichSelectorsStoreInto:} retourne les sélecteurs des méthodes qui modifient la valeur d'une variable d'instance
\item \mthind{Behavior}{whichSelectorsReferTo:} retourne les sélecteurs des méthodes qui envoie un certain message
\item \mthind{Behavior}{crossReference} associe chaque message avec l'ensemble des méthodes qui l'envoie.
\end{itemize}

\begin{code}{} % TOO FRAGILE TO TEST
Point whichSelectorsAccess: 'x'    --> an IdentitySet(#'\\' #= #scaleBy: ...)
Point whichSelectorsStoreInto: 'x' --> an IdentitySet(#setX:setY: ...)
Point whichSelectorsReferTo: #+  --> an IdentitySet(#rotateBy:about: ...)
Point crossReference --> an Array(
		an Array('*' an IdentitySet(#rotateBy:about: ...))
		an Array('+' an IdentitySet(#rotateBy:about: ...))
		...)
\end{code}

Les messages qui suivent prennent en compte l'héritage :
\begin{itemize}
\item \mthind{Behavior}{whichClassIncludesSelector:} retourne la super-classe qui implémente le message concerné
\item \mthind{Behavior}{unreferencedInstanceVariables} retourne la liste des variables d'instances qui ne sont ni utilisées dans la classe du receveur, ni dans aucune de ses sous-classes
\end{itemize}

\begin{code}{@TEST}
Rectangle whichClassIncludesSelector: #inspect --> Object
Rectangle unreferencedInstanceVariables            --> #()
\end{code}

\clsind{SystemNavigation} is a facade that supports various useful methods for querying and browsing the source code of the system.
\ct{SystemNavigation} \mthind{SystemNavigation class}{default} returns an instance you can use to navigate the system.

\clsind{SystemNavigation} est une façade qui comporte plusieurs méthodes utiles pour examiner et parcourir le code source du système.
\ct{SystemNavigation} \mthind{SystemNavigation class}{default} retourne une instance que vous pouvez utiliser pour naviguer dans le système. 

Par exemple :

\mthindex{SystemNavigation}{allClassesImplementing:}
\begin{code}{@TEST}
SystemNavigation default allClassesImplementing: #yourself --> {Object}
\end{code}

Les messages suivants devraient être également compréhensibles par eux-même :

\mthindex{SystemNavigation}{allSentMessages}
\mthindex{SystemNavigation}{allUnsentMessages}
\mthindex{SystemNavigation}{allUnimplementedCalls}
\begin{code}{}
SystemNavigation default allSentMessages size          --> 24930
SystemNavigation default allUnsentMessages size      --> 6431
SystemNavigation default allUnimplementedCalls size --> 270
\end{code}

Notons que les messages implémentés mais non envoyés ne sont pas nécessairement inutiles, car ils peuvent être envoyés implicitement (\eg en utilisant \ct{perform:}).
Les messages envoyés mais non implémentés sont plus problématiques, car les méthodes envoyant ces messages vont échouer à l'exécution. Ceci peut être le signe d'une implémentation non finie, d'une API obsolète ou bien de librairies manquantes.

\mthindex{SystemNavigation}{allCallsOn:}
\ct!SystemNavigation default allCallsOn: #Point! retourne tous les messages envoyés explicitement à \ct!Point! comme receveur du message.

Toutes ces fonctionnalités sont intégrées dans l'environnement de programmation \pharo, en particulier dans les navigateurs de code.
Comme vous avez déjà pu vous en apercevoir, il existe des raccourcis claviers pour parcourir tous les i\underline{m}plémenteurs (\short{m}) et e\underline{n}voyeurs (\short{n}) d'un message particulier.

Ce qui est moins connu est qu'il existe un certain nombre de méthodes pour faire des requêtes similaires dans le protocole \prot{browsing} dans la classe \ct{SystemNavigation}.

Par exemple, vous pouvez parcourir de manière programmatique tous les implémenteurs du message \ct{ifTrue:} en évaluant:
\mthindex{SystemNavigation}{browseAllImplementorsOf:}
\begin{code}{}
SystemNavigation default browseAllImplementorsOf: #ifTrue:
\end{code}

\begin{figure}[ht]\centering
	\includegraphics[width=\linewidth]{implementors}
	\caption{Parcourir toutes les implémentations de \ct!\#ifTrue:!.\figlabel{implementors}}
\end{figure}

Des méthodes qui sont particulièrement utiles sont les méthodes \ct{browserAllSelect:} et \lct{browserMethodsWithSourceString:}. 
Voici deux différentes façons de parcourir les méthodes d'un système qui utilisent des appels à super (la première façon est plutôt brutale; la deuxième est meilleure et élimine certains faux positifs):
\mthindex{SystemNavigation}{browseMethodsWithSourceString:}
\mthindex{SystemNavigation}{browseAllSelect:}
\begin{code}{}
SystemNavigation default browseMethodsWithSourceString: 'super'.
SystemNavigation default browseAllSelect: [:method | method sendsToSuper ].
\end{code}

%======================================
\section{Classes, dictionnaires de méthodes et méthodes}

Comme les classes sont des objets, il est possible de les inspecter ou de les explorer de la même manière que les objets.

\mthindex{Object}{explore}
\dothis{\'Evaluer \ct{Point explore}.}

Dans \figref{CompiledMethod}, l'\ind{explorateur} montre la structure de la classe \clsind{Point}.
Vous pouvez remarquer que la classe stocke ses méthodes dans un dictionnaire, indexées par leur sélecteur.
Le sélecteur \ct{#*} pointe vers le \ind{bytecode} décompilé de \ct!Point>>>*!.

\begin{figure}[ht]\centering
	\includegraphics[width=.5\linewidth]{CompiledMethod}
	\caption{l'explorateur de la classe \ct!Point! et le bytecode de sa méthode \ct!\#*! method.\figlabel{CompiledMethod}}
\end{figure}

Examinons la relation entre classes et méthodes.
Dans \figref{MethodsAsObjects} nous voyons que classes et méta-classes ont en commun la superclasse \ct{Behavior}. C'est dans cette superclasse que la méthode \mthind{Behavior}{new} est définie, parmi d'autres méthodes clés pour ces classes.
Chaque classe possède un dictionnaire de méthode, qui associe chaque sélecteur de méthodes à sa \ind{méthode compilée}.
Chaque méthode compilée connait la classe dans laquelle elle est installée.
Dans \figref{CompiledMethod}, nous pouvons même remarquer que l'information est conservée au moyen d'une association dans \ct{literal5}.

\begin{figure}[ht]\centering
	\includegraphics[width=\linewidth]{MethodsAsObjects}
	\caption{Classes, dictionnaires de méthodes et méthodes compilées\figlabel{MethodsAsObjects}}
\end{figure}

On peut exploiter les relations établies entre classes et méthodes, pour effectuer des requêtes sur le système.
Par exemple, pour connaitre quelles méthodes viennent d'être introduites dans une classe donnée, \ie celles qui ne surchargent pas les méthodes de la superclasse, nous pouvons naviguer depuis la classe vers le dictionnaire de méthodes de cette manière :

\mthindex{Behavior}{methodDict}
\begin{code}{}
[:aClass| aClass methodDict keys select: [:aMethod |
  (aClass superclass canUnderstand: aMethod) not ]] value: SmallInteger
  --> an IdentitySet(#threeDigitName #printStringBase:nDigits: ...)
\end{code}

Une méthode compilée ne stocke pas simplement le bytecode de la méthode.
C'est aussi un objet qui fournit de nombreuses méthodes utiles pour interroger le système. 
Une de ces méthodes se nomme \ct{isAbstract} (qui nous renseigne si la méthode envoie \ct{subclassResponsibility}).
Nous pouvons l'utiliser pour identifier toutes les méthodes abstraites d'une classe abstraite.

\needlines{4}
\begin{code}{}
[:aClass| aClass methodDict keys select: [:aMethod |
  (aClass>>aMethod) isAbstract ]] value: Number
  --> an IdentitySet(#storeOn:base: #printOn:base: #+ #- #* #/ ...)
\end{code}

Remarquez que ce code envoie le message \ct{>>} à la classe pour obtenir la méthode compilée d'un sélecteur donné.

% As a slightly more complex example, we can browse 

Pour parcourir les méthodes d'une classe-mère au sein d'une hiérarchie donnée, par exemple de la hiérarchie de Collections, nous pouvons poser une requête plus sophistiquée : 
\begin{code}{}
class := Collection.
SystemNavigation default
  browseMessageList: (class withAllSubclasses gather: [:each |
    each methodDict associations
      select: [:assoc | assoc value sendsToSuper]
      thenCollect: [:assoc | MethodReference class: each selector: assoc key]])
  name: 'Supersends of ' , class name , ' and its subclasses'
\end{code}
Remarquez comment nous naviguons en partant des classes pour aller vers les dictionnaires de méthodes puis vers les méthodes compilées pour identifier les méthodes recherchées. 
Une \ct{MethodReference} est un proxy léger pour une méthode compilée, utilisé par de nombreux outils.
Il existe une méthode adaptée \clsmthind{CompiledMethod}{methodReference} qui retourne la référence de la méthode pour une méthode compilée.

\begin{code}{@TEST}
(Object>>#=) methodReference methodSymbol --> #=
\end{code}

%======================================
\section{Environnements de navigation du code}

Même si \clsind{SystemNavigation} offre quelques façons utiles d'interroger le système par programmes et de parcourir le code système, il existe une meilleure manière. Le \ind{Refactoring Browser}, qui est intégré dans \pharo, permet de poser des questions complexes à la fois de manière interactive et par programme.

Supposons que nous voulions découvrir quelle méthode dans la hiérarchie \lct{Collection} envoie un message à \super qui soit différent depuis le sélecteur de méthodes. Ceci est généralement considéré comme un mauvais \ind{code smell}, puisque un tel \super-send devrait normalement être remplacé par un \self-send. (Pensez à cela --- vous ne devriez avoir besoin de \super que pour étendre une méthode que vous êtes en train de surcharger; toutes les autres méthodes héritées peuvent être accédées par un \self! )

Le Refactoring Browser nous permet de manière élégante de restreindre nos interrogations uniquement aux classes et méthodes qui nous intéressent. 


\dothis{Ouvrir un Browser sur la classe \ct{Collection}.
\actclick sur le nom de la classe et sélectionner \menu{refactoring scope>subclasses with}.
Ceci ouvrira un nouveau browser Environnement sur la hierarchie \ct{Collection}.
Dans ce champ restreint sélectionner \menu{refactoring scope>super-sends} pour ouvrir un nouvel environnement avec toutes les méthodes qui font des super-sends dans la hiérarchie \ct{Collection}. Maintenant \click sur n'importe quelle méthode et sélectionner \menu{refactor>code critics}.
Naviguer dans \menu{Lint checks>Possible bugs>Sends different super message} et \actclick pour sélectionner \menu{browse}.}

Dans \figref{sendDifferentSuper} nous pouvons voir que 19 méthodes de la sorte ont été trouvées dans la hiérarchie \ct{Collection}, incluant \ct{Collection>>>printNameOn:}, laquelle envoie \ct{super printOn:}.
\begin{figure}[ht]\centering
	\includegraphics[width=\linewidth]{sendDifferentSuper}
	\caption{Trouver les méthodes qui envoient un message différent de super.
 	\figlabel{sendDifferentSuper}}
\end{figure}

Un environnement de navigation du code peut aussi être créé par programme.
Ici, par exemple, nous créons un nouveau \clsind{BrowserEnvironment} pour \clsind{Collection} et ses sous-classes, nous sélectionnons les méthodes super-sending et nous ouvrons l'environnement résultant.
\needlines{4}
\begin{code}{}
((BrowserEnvironment new forClasses: (Collection withAllSubclasses))
	selectMethods: [:method | method sendsToSuper])
	label: 'Collection methods sending super';
	open.
\end{code}{}

Notez à quel point ce moyen est considérablement plus compact que les exemples précédents utilisant \ct{SystemNavigation}.

Finallement, nous pouvons trouver uniquement les méthodes qui envoient un message différent de super comme ceci :
\needlines{6}
\begin{code}{}
((BrowserEnvironment new forClasses: (Collection withAllSubclasses))
	selectMethods: [:method | 
		method sendsToSuper
		and: [(method parseTree superMessages includes: method selector) not]])
	label: 'Collection methods sending different super';
	open
\end{code}
Dans cet exemple nous demandons à chaque méthode compilée son arbre d'analyse (Refactoring Browser), dans le but de trouver quels super messages diffèrent des sélecteurs de méthodes.  
Regardez le protocole \prot{querying} de la classe \ct{RBProgramNode} pour voir un certain nombre de choses que nous pouvons demander aux arbres d'analyse. 

% Here we ask each compiled method for its (Refactoring Browser) parse tree, in order to find out whether the super messages differ from the method's selector.
% Have a look at the \prot{querying} protocol of the class \ct{RBProgramNode} to see some the things we can ask of parse trees.

%======================================
\section{Accéder au contexte d'exécution}

Nous avons vu comment les capacités réflexives de \st, nous permettent d'interroger et d'explorer les objets, les classes et les méthodes. Qu'en est-il de l'environnement d'exécution ?

%-----------------------------------------------------------------
\subsection{Contextes des méthodes}

En fait, le contexte d'exécution d'une méthode se trouve dans la machine virtuelle et pas dans l'image. Mais visiblement, le \ind{débogueur} a accès à cette information et on peut explorer le contexte d'exécution, comme n'importe quel autre objet ? Comment cela est possible ?

En fait, il n'y rien de magique avec le débogueur.
Le secret réside dans la pseudo-variable \pvind{thisContext}, que nous avons brièvement rencontré précédemment.
Lorsque l'on accède à \ct{thisContext} dans une méthode qui s'exécute, tout le contexte d'exécution de cette méthode est réifié et rendu disponible dans l'image comme une liste chainée d'objets \clsind{MethodContext}.
Nous pouvons facilement expérimenter ce mécanisme par nous-même.

\dothis{Changer la définition de \ct{Integer>>>factorial} en insérant l'expression soulignée ci-dessous:}
\mthindex{Object}{halt}
\begin{code}{}
Integer>>>factorial
	"Answer the factorial of the receiver."
	self = 0 ifTrue: [!\underline{thisContext explore. self halt.}! ^ 1].
	self > 0 ifTrue: [^ self * (self - 1) factorial].
	self error: 'Not valid for negative integers'
\end{code}

\dothis{Maintenant évaluons \ct{3 factorial} dans un espace de travail. Nous devons normalement obtenir une fenêtre de déboguage et un explorateur comme on peut le voir dans \figref{exploringThisContext}.}

\begin{figure}[ht]\centering
	\includegraphics[width=\linewidth]{exploringThisContext}
	\caption{Explorer \lct{thisContext}.\figlabel{exploringThisContext}}
\end{figure}

Bienvenue dans le débogueur du pauvre ! 
Si maintenant nous parcourons la classe de l'objet exploré (\ie en évaluant \ct{self browse} dans le panneau du bas de l'explorateur), vous aller découvrir que c'est une instance de la classe \lct{MethodContext}, comme tous les \ct{envoyeurs} dans la chaîne.
% All of these objects have been created dynamically in the image by the \st virtual machine at the point where \ct{thisContext} was referred to in the \ct{factorial} method. \lr{Not actually. In all the currently available VMs the context objects are created with every method activation, no matter if they are accessed using \ct{thisContext} or not.}

\ct{thisContext} n'est pas destiné à être utilisé dans la programmation de tous les jours, mais il est essentiel pour réaliser des outils comme des débogueurs et lorsque l'on a besoin d'accéder à des information concernant la pile d'appels de méthodes.
On peut évaluer l'expression suivante pour découvrir quelles méthodes utilisent \ct{thisContext}:

\begin{code}{}
SystemNavigation default browseMethodsWithSourceString: 'thisContext'
\end{code}

Comme on pourrait s'en douter, une des applications les plus répandues est de découvrir l'envoyeur d'un message.
Voici une application courante :
\begin{code}{}
Object>>>subclassResponsibility
	"This message sets up a framework for the behavior of the class' subclasses.
	Announce that the subclass should have implemented this message."

	self error: 'My subclass should have overridden ', thisContext sender selector printString
\end{code}

Par convention, les méthodes \st qui envoient le message \ct{self subclassResponsibility} sont considérées comme abstraites. Mais
comment \clsmthind{Object}{subclassResponsibility} indique un message d'erreur utile indiquant quelle méthode abstraite a été appelée ? Simplement, en interrogant la pseudo-variable \ct{thisContext} de l'envoyeur du message.

%\lr{I think co-routines and continuations should at least mentioned here. Another very practical application that is simple and could be shown here is the ``escaper''. Store the current context into a temp or inst-var \ct{target := thisContext} and jump back to that stack frame at a later point in time using \ct{target return: 123}.}
%\sd{lukas maybe we should have another chapter showing such kind of beasts. I would love to read it. Showing how to use block to build exception and such a kind of point. I think that this chapter should be an introduction may be we should have a Reflection applied chapter}
%\lr{I would love to help writing such a chapter}
%-----------------------------------------------------------------
\subsection{Points d'arrêts intelligents}

\mthindex{Object}{halt}
La façon \st de mettre des points d'arrêts dans un programme consiste à évaluer \ct{self halt} aux endroits correspondants. Ceci va provoquer la réification de \ct{thisContext} et une fenêtre de \ind{déboguage} va s'ouvrir sur ce point d'arrêt.
Malheureusement ceci peut poser des problèmes pour des méthodes qui sont utilisées partout dans le système.

Supposons par exemple, que nous voulons explorer l'exécution de \ct{OrderedCollection>>>add:}.
Mettre un point d'arrêt sur cette méthode est problématique.

\dothis{Prendre une \emph{nouvelle} image et ajouter le point d'arrêt suivant :}
\needlines{3}
\begin{code}{}
OrderedCollection>>>add: newObject
	!\underline{self halt.}!
	^self addLast: newObject
\end{code}

Nous remarquons que notre image se fige instantanément ! Nous n'avons même pas eu de fenêtre de déboguage.
Le problème devient évident lorsque nous comprenons que (i) \ct{OrderedCollection>>>add:} est utilisé en de nombreux endroit du système, de telle sorte que le point d'arrêt est déclenché peut après que nous ayons accepté cette modification et (ii) que le \emph{débogueur lui-même} envoie le message \ct{add:} à une instance de \ct{OrderedCollection}, l'empéchant d'ouvrir la fenêtre de déboguage !
Ce dont nous avons besoin est de pouvoir faire \emph{un point d'arrêt conditionnel} seulement lorsque nous sommes dans le contexte qui nous intéresse.
C'est exactement ce qu'offre \clsmthind{Object}{haltIf:}.

Supposons maintenant que nous voulons arrêter le programme seulement si le message \ct{add} est envoyé dans le contexte de \ct{OrderedCollectionTest>>>testAdd}.

\dothis{Ouvrir une nouvelle image de nouveau et mettre le point d'arrêt ci-dessous :}
\begin{code}{}
OrderedCollection>>>add: newObject
	!\underline{self haltIf: \#testAdd.}!
	^self addLast: newObject
\end{code}

Cette fois ci, l'image ne se fige pas. Essayer d'exécuter \ct{OrderedCollectionTest}.
(Vous pouvez trouver cette classe dans la catégorie \scat{CollectionsTests-Sequenceable}.)

Comment cela fonctionne-t-il ? Regardons le code de \clsmthind{Object}{haltIf:}:
\begin{code}{}
Object>>>haltIf: condition
	| cntxt |
	condition isSymbol ifTrue: [
		"only halt if a method with selector symbol is in callchain"
		cntxt := thisContext.
		[cntxt sender isNil] whileFalse: [
			cntxt := cntxt sender. 
			(cntxt selector = condition) ifTrue: [Halt signal]. ].
		^self.
	].
	...
\end{code}

À partir de \ct!thisContext!, \ct!haltIf:! parcourt la pile d'exécution en vérifiant que le nom de la méthode appelante est le même que celle passé en paramètre. Si c'est le cas, la méthode déclenche une exception qui par défaut déclenche le débogueur.

Il est également possible de fournir une valeur booléenne ou un bloc qui retourne un booléen comme argument de \ct{haltIf:}, mais dans ce cas, cela devient plus simple et on n'utilise pas \ct{thisContext}.

%======================================
\section{Intercepter les messages non compris}
\seclabel{msgnotunderstood}

Pour l'instant, nous avons utilisé les capacités réflexives de \st principalement pour interroger et explorer les objets, classes, méthodes et pile d'exécution du système. Nous allons maintenant voir comment utiliser notre connaissance de \st pour intercepter des messages et modifier le comportement du système à l'exécution.

Lorsque un objet reçoit un message, il regarde d'abord dans son dictionnaire de méthodes pour la méthode correspondante pour répondre au message.
Si aucune méthode correspondante n'existe, il va continuer son exploration en remontant dans la hiérarchie de classe, jusqu'à atteindre la classe \ct{Object}. If still no method is found for that message, the object will \emph{send itself} the message \ct{doesNotUnderstand:} with the message selector as its argument.
The process then starts all over again, until \clsmthind{Object}{doesNotUnderstand:} is found, and the debugger is launched.

But what if \ct{doesNotUnderstand:} is overridden by one of the subclasses of \ct{Object} in the lookup path?
As it turns out, this is a convenient way of realizing certain kinds of very dynamic behaviour. An object that does not understand a message can, by overriding \ct{doesNotUnderstand:}, fall back to an alternative strategy for responding to that message.

Two very common applications of this technique are (1) to implement \ind{lightweight proxies} for objects, and (2) to dynamically compile or load missing code.

%-----------------------------------------------------------------
\subsection{Proxy légers}

In the first case, we introduce a ``\ind{minimal object}'' to act as a proxy for an existing object.
Since the proxy will implement virtually no methods of its own, any message sent to it will be trapped by \ct{doesNotUnderstand:}. By implementing this message, the proxy can then take special action before delegating the message to the real subject it is the proxy for.

Let us have a look at how this may be implemented\footnote{You can also load \pkg{PBE-Reflection} from \url{http://www.squeaksource.com/PharoByExample/}}.

We define a \ct{LoggingProxy} as follows:
\begin{code}{}
ProtoObject subclass: #LoggingProxy
	instanceVariableNames: 'subject invocationCount'
	classVariableNames: ''
	poolDictionaries: ''
	category: 'PBE-Reflection'
\end{code}
Note that we subclass \ct{ProtoObject} rather than \ct{Object} because we do not want our proxy to inherit over 400 methods (!) from \ct{Object}.

\begin{code}{}
Object methodDict size --> 408
\end{code}

Our proxy has two instance variables: the \ct{subject} it is a proxy for, and a \ct{count} of the number of messages it has intercepted.
We initialize the two instance variables and we provide an accessor for the message count.
Initially the \ct{subject} variable points to the proxy object itself.
\begin{code}{}
LoggingProxy>>>initialize
	invocationCount := 0.
	subject := self.
\end{code}

\begin{code}{}
LoggingProxy>>>invocationCount
	^ invocationCount
\end{code}

We simply intercept all messages not understood, print them to the Transcript, update the message count, and forward the message to the real subject.
\begin{code}{}
LoggingProxy>>>doesNotUnderstand: aMessage 
	Transcript show: 'performing ', aMessage printString; cr.
	invocationCount := invocationCount + 1.
	^ aMessage sendTo: subject
\end{code}

Here comes a bit of magic.
We create a new \ct{Point} object and a new \ct{LoggingProxy} object, and then we tell the proxy to \mthind{ProtoObject}{become:} the point object:
\seeindex{\ct{become:}}{\ct{ProtoObject>>>become:}}
\begin{code}{}
point := 1@2.
LoggingProxy new !\underline{become:}! point.
\end{code}

This has the effect of swapping all references in the image to the point to now refer to the proxy, and vice versa. Most importantly, the proxy's \ct{subject} instance variable will now refer to the point!

\begin{code}{}
point invocationCount --> 0
point + (3@4)             --> 4@6
point invocationCount --> 1
\end{code}

This works nicely in most cases, but there are some shortcomings:
\begin{code}{}
point class --> LoggingProxy
\end{code}
Curiously, the method \ct{class} is not even implemented in \ct{ProtoObject} but in \ct{Object}, which \ct{LoggingProxy} does not inherit from!
The answer to this riddle is that \ct{class} is never sent as a message but is directly answered by the virtual machine.\footnote{\ct{yourself} is also never truly sent.
Other messages that may be directly interpreted by the VM, depending on the receiver, include:
\ct{+- < > <= >= = ~= * / \ ==}
\ct{@ bitShift: // bitAnd: bitOr:}
\ct{at: at:put: size}
\ct{next nextPut: atEnd}
\ct{blockCopy: value value: do: new new: x y}.
Selectors that are never sent, because they are inlined by the compiler and transformed to comparison and jump bytecodes:
\ct{ifTrue: ifFalse: ifTrue:ifFalse: ifFalse:ifTrue:}
\ct{and: or:}
\ct{whileFalse: whileTrue: whileFalse whileTrue}
\ct{to:do: to:by:do:}
\ct{caseOf: caseOf:otherwise:}
\ct{ifNil: ifNotNil:  ifNil:ifNotNil: ifNotNil:ifNil:}
Attempts to send these messages to non-boolean objects can be intercepted and execution can be resumed with a valid boolean value by overriding \ct{mustBeBoolean} in the receiver or by catching the \ct{NonBooleanReceiver} exception.
}% NB: Notes by Lukas Renggli

Even if we can ignore such special message sends, there is another fundamental problem which cannot be overcome by this approach: \self-sends cannot be intercepted:
\begin{code}{}
point := 1@2.
LoggingProxy new become: point.
point invocationCount --> 0
point rect: (3@4)        --> 1@2 corner: 3@4
point invocationCount --> 1
\end{code}

Our proxy has been cheated out of two \self-sends in the \ct{rect:} method:
\begin{code}{}
Point>>>rect: aPoint 
	^ Rectangle  origin: (self min: aPoint) corner: (self max: aPoint)
\end{code}

Although messages can be intercepted by proxies using this technique, one should be aware of the inherent limitations of using a proxy.  In \secref{wrapper} we will see another, more general approach for intercepting messages.

%-----------------------------------------------------------------
\subsection{Générer des méthodes manquantes}

The other most common application of intercepting not understood messages is to dynamically load or generate the missing methods.
Consider a very large library of classes with many methods.  Instead of loading the entire library, we could load a stub for each class in the library. The stubs know where to find the source code of all their methods.  The stubs simply trap all messages not understood, and dynamically load the missing methods on-demand.  At some point, this behaviour can be deactivated, and the loaded code can be saved as the minimal necessary subset for the client application.

%\on{Stef sez: check ObjectOut -- I looked, but this seems to be very old. Depends on SqueakPage.}

Let us look at a simple variant of this technique where we have a class that automatically adds accessors for its instance variables on-demand:
% \lr{the last statement should return the result of the message, otherwise you cannot proceed with the debugger} \alex{all redefinition of doesNotUnderstand: includes a return statement. However, I do not see your comment lukas, I tried to insert a 'self halt' in the method, I was able to proceed. I added the return in the function} \lr{Of course it depends on the exact circumstances. If you perform a message on self that returns self it does not matter, but in any other case a forgotten return can introduce strange side effects. There was no return in the listing below, but now there is and the problem is solved.}

\begin{code}{}
DynamicAcccessors>>>doesNotUnderstand: aMessage
	| messageName |
	messageName := aMessage selector asString.
	(self class instVarNames includes: messageName)
		ifTrue: [
			self class compile: messageName, String cr, ' ^ ', messageName.
			^ aMessage sendTo: self ].
	^ super doesNotUnderstand: aMessage
\end{code}
Any message not understood is trapped here. If an instance variable with the same name as the message sent exists, then we ask our class to compile an accessor for that instance variables and we re-send the message.

Suppose the class \ct{DynamicAcccessors} has an (uninitialized) instance variable \ct{x} but no pre-defined accessor. Then the following will generate the accessor dynamically and retrieve the value:
\needlines{2}
\begin{code}{}
myDA := DynamicAccessors new.
myDA x --> nil
\end{code}

Let us step through what happens the first time the message \ct{x} is sent to our object (see \figref{DynamicAccessors}).

\begin{figure}[ht]\centering
	\includegraphics[width=\linewidth]{DynamicAccessors}
	\caption{Création dynamique d'accesseurs.\figlabel{DynamicAccessors}}
	% \alex{not sure whether the figure is highly necessary. The code is rather simple and the figure complete to follow in my opinion.}}
	% \on{trust me, it is useful to see all the steps.}
\end{figure}

(1) We send \ct{x} to \ct{myDA}, (2) the message is looked up in the class, and (3) not found in the class hierarchy. (4) This causes \ct{self doesNotUnderstand: #x} to be sent back to the object, (5) triggering a new lookup. This time \ct{doesNotUnderstand:} is found immediately in \ct{DynamicAccessors}, (6) which asks its class to compile the string \ct{'x ^ x'}. The \ct{compile} method is looked up (7), and (8) finally found in \ct{Behavior}, which (9-10) adds the new compiled method to the method dictionary of \ct{DynamicAccessors}. Finally, (11-13) the message is resent, and this time it is found.

The same technique can be used to generate setters for instance variables, or other kinds of boilerplate code, such as visiting methods for a Visitor.

Note the use of \clsmthind{Object}{perform:} in step (13) which can be used to send messages that are composed at run-time:
\begin{code}{@TEST}
5 perform: #factorial                                             --> 120
6 perform: ('fac', 'torial') asSymbol                       --> 720
4 perform: #max: withArguments: (Array with: 6) --> 6
\end{code}

%======================================
\section{Objects as method wrappers}
\seclabel{wrapper}

We have already seen that compiled methods are ordinary objects in \st, and they support a number of methods that allow the programmer to query the run-time system.
What is perhaps a bit more surprising, is that \emph{any object} can play the role of a compiled method. All it has to do is respond to the method \ct{run:with:in:} and a few other important messages.

\dothis{Define an empty class \ct{Demo}. Evaluate \ct{Demo new answer42} and notice how the usual ``Message Not Understood'' error is raised.}

Now we will install a plain \st object in the method dictionary of our \ct{Demo} class.

\dothis{Evaluate \lct{Demo methodDict at: \#answer42 put: ObjectsAsMethodsExample new.}
Now try again to print the result of \ct{Demo new answer42}. This time we get the answer \ct{42}.}

If we take look at the class \clsind{ObjectsAsMethodsExample} we will find the following methods:
%\alex{I would prefer having return 42 in the run:with:in: method, and not having answer42 defined in ObjectsAsMethodsExample, this could be confusing I imagine}
%\on{ObjectsAsMethodsExample is part of the standard pharo image -- it is not in PBE-Reflection}
\needlines{5}
\begin{code}{}
answer42
	^42

run: oldSelector with: arguments in: aReceiver
	^self perform: oldSelector withArguments: arguments
\end{code}

When our \ct{Demo} instance receives the message \ct{answer42}, method lookup proceeds as usual, however the virtual machine will detect that in place of a compiled method, an ordinary \st object is trying to play this role.
The VM will then send this object a new message \ct{run:with:in:} with the original method selector, arguments and receiver as arguments.
Since \ct{ObjectsAsMethodsExample} implements this method, it intercepts the message and delegates it to itself.

We can now remove the fake method as follows:
\begin{code}{}
Demo methodDict removeKey: #answer42 ifAbsent: []
\end{code}

If we take a closer look at \ct{ObjectsAsMethodsExample}, we will see that its superclass also implements the methods \ct{flushcache}, \ct{methodClass:} and \lct{selector:}, but they are all empty.  These messages may be sent to a compiled methods, so they need to be implemented by an object pretending to be a compiled method.  (\ct{flushcache} is the most important method to be implemented; others may be required depending on whether the method is installed using \clsmthind{Behavior}{addSelector:withMethod:} or directly using \clsmthind{MethodDictionary}{at:put:}.)

%-------------------------------------------------------------------------
\subsection{Using methods wrappers to perform test coverage}

Method wrappers are a well-known technique for intercepting messages \cite{Bran98a}.
In the original implementation\footnote{http://www.squeaksource.com/MethodWrappers.html}, a method wrapper is an instance of a subclass of \ct{CompiledMethod}. When installed, a method wrapper can perform special actions before or after invoking the original method.
When uninstalled, the original method is returned to its rightful position in the method dictionary.

In Pharo, \ind{method wrappers} can be implemented more easily by implementing \ct{run:with:in:} instead of by subclassing \ct{CompiledMethod}. In fact, there exists a lightweight implementation of objects as method wrappers\footnote{http://www.squeaksource.com/ObjectsAsMethodsWrap.html}, but it is not part of standard Pharo at the time of this writing.

Nevertheless, the Pharo Test Runner uses precisely this technique to evaluate test coverage.
Let's have a quick look at how it works.

The entry point for test coverage is the method \clsmthind{TestRunner}{runCoverage}:
\begin{code}{}
TestRunner>>>runCoverage
	| packages methods |
	... "identify methods to check for coverage"
	self collectCoverageFor: methods
\end{code}

The method \clsmthind{TestRunner}{collectCoverageFor:} clearly illustrates the coverage checking algorithm:
\begin{code}{}
TestRunner>>>collectCoverageFor: methods
	| wrappers suite |
	wrappers := methods collect: [ :each | TestCoverage on: each ].
	suite := self
		reset;
		suiteAll.
	[ wrappers do: [ :each | each install ].
	  [ self runSuite: suite ] ensure: [ wrappers do: [ :each | each uninstall ] ] ] valueUnpreemptively.
	wrappers := wrappers reject: [ :each | each hasRun ].
	wrappers isEmpty 
		ifTrue: 
			[ UIManager default inform: 'Congratulations. Your tests cover all code under analysis.' ]
		ifFalse: ...
\end{code}
A wrapper is created for each method to be checked, and each wrapper is installed.
The tests are run, and all wrappers are uninstalled.
Finally the user obtains feedback concerning the methods that have not been covered.

How does the wrapper itself work?
The \ct{TestCoverage} wrapper has three instance variables, \ct{hasRun}, \ct{reference} and \ct{method}.
They are initialized as follows:
\begin{code}{}
TestCoverage class>>>on: aMethodReference
	^ self new initializeOn: aMethodReference

TestCoverage>>>initializeOn: aMethodReference
	hasRun := false.
	reference := aMethodReference.
	method := reference compiledMethod
\end{code}

The install and uninstall methods simply update the method dictionary in the obvious way:
\begin{code}{}
TestCoverage>>>install
	reference actualClass methodDictionary
		at: reference methodSymbol
		put: self

TestCoverage>>>uninstall
	reference actualClass methodDictionary
		at: reference methodSymbol
		put: method
\end{code}
\noindent
and the \ct{run:with:in:} method simply updates the \ct{hasRun} variable, uninstalls the wrapper (since coverage has been verified), and resends the message to the original method
\begin{code}{}
run: aSelector with: anArray in: aReceiver
	self mark; uninstall.
	^ aReceiver withArgs: anArray executeMethod: method

mark
	hasRun := true
\end{code}
(Have a look at \clsmthind{ProtoObject}{withArgs:executeMethod:} to see how a method displaced from its method dictionary can be invoked.)

That's all there is to it!

Method wrappers can be used to perform any kind of suitable behaviour before or after the normal operation of a method.  Typical applications are instrumentation (collecting statistics about the calling patterns of methods), checking optional pre- and post-conditions, and memoization (optionally cacheing computed values of methods).

%======================================
\section{Pragmas}

Un \emphind{pragma} est une annotation qui donne des information sur un programme, mais qui n'est pas directement impliqué dans l'exécution de ce programme. Les pragmas n'ont pas d'effet direct lors du déroulement d'une méthode annotée.
Les pragmas sont très utiles notamment pour :
\begin{itemize}
\item Donner de l'information au compilateur : les \indmain{pragmas} peuvent être utilisés par le compilateur pour qu'une méthode appelle une fonction primitive. Cette fonction doit être définie par la machine virtuelle ou au moyen d'un greffon externe.
\item Donner de l'information à l'exécution.
\end{itemize}

Les pragmas s'utilisent uniquement lors de la déclaration des méthodes d'un programme. Une méthode peut déclarer un ou plusieurs pragmas qui sont écrits avant toutes expressions Smalltalk. En réalité, un pragma défini une sorte de message statique avec des arguments qui sont des litéraux.

Nous avons déjà parlé brièvement des pragmas lorsque nous avons briévement introduit la notion de primitives précédemment dans ce chapitre. Une primitive n'est rien moins qu'une déclaration de pragma. 
Consider \ct{<primitive: 73>} as contained in \ct{instVarAt:}. The pragma's selector is \ct{primitive:} and its arguments is an immediate literal value, \ct{73}. 

Le compilateur Smalltalk est probablement l'un des utilisateurs les plus important des pragmas. SUnit is another tool that makes use of annotations. SUnit is able to estimate the coverage of an application from a test unit. One may want to exclude some methods from the coverage. This is the case of the \ct!documentation! method in \ct!SplitJointTest class!:

\begin{code}{}
SplitJointTest class>>>documentation
	<ignoreForCoverage>
	"self showDocumentation"
	
	^ 'This package provides function.... "
\end{code}

By simply annotating a method with the pragma \ct!<ignoreForCoverage>! one can control the scope of the coverage.

%Beside the compiler, Lint is a heavy user of pragmas. Lint is a static code analyzer that flags suspicious, non-portable constructs and code that is likely to contain bugs. It may happen that a method needs to be excluded from Lint analysis. This is the case here:

%\begin{code}{}
%MorphObjectOut>>>doesNotUnderstand: aMessage 
%	"Bring in the object, install, then resend aMessage"
%	"Transcript show: thisContext sender selector; cr."
%	"useful for debugging"
%	
%	! \textbf{<lint: 'Unnecessary "= true"' rationale: 'recursionFlag may be nil' author: 'stephane.ducasse'>}!
%	...
%\end{code}	

%One of the pragmas used by Lint to filter out methods is \ct{lint:rationale:author:}.

As instances of the class \clsind{Pragma}, pragmas are first class objects. A compiled method answers to the message \mthind{CompiledMethod}{pragmas}. This method returns an array of pragmas. 

\begin{code}{@TEST}
(SplitJoinTest class >> #showDocumentation) pragmas
  --> an Array(<ignoreForCoverage>)
(Float>>#+) pragmas --> an Array(<primitive: 41>)
\end{code}

Methods defining a particular query may be retrieved from a class. The class side of \ct!SplitJoinTest! contains some methods annotated with \ct!<ignoreForCoverage>!:

\begin{code}{@TEST}
Pragma allNamed: #ignoreForCoverage in: SplitJoinTest class  --> an Array(<ignoreForCoverage> <ignoreForCoverage> <ignoreForCoverage>)
\end{code}

A variant of \ct{allNamed:in:} may be found on the class side of \ct{Pragma}.

A pragma knows in which method it is defined (using \ct{method}), the name of the method (\ct{selector}), the class that contains the method (\ct{methodClass}), its number of arguments (\ct{numArgs}), about the literals the pragma has for arguments (\ct{hasLiteral:} and \ct{hasLiteralSuchThat:}). 

%\lr{Typically pragmas are performed on an interpreter object that understands the pragma message.}

%======================================
\section{Résumé du chapitre}

Reflection refers to the ability to query, examine and even modify the metaobjects of the run-time system as ordinary objects.

\begin{itemize}
\item The Inspector uses \ct{instVarAt:} and related methods to query and modify ``private'' instance variables of objects.
\item Send \ct{Behavior>>>allInstances} to query instances of a class.
\item The messages \ct{class}, \ct{isKindOf:}, \ct{respondsTo:} \etc  are useful for gathering metrics or building development tools, but they should be avoided in regular applications: they violate the encapsulation of objects and make your code harder to understand and maintain.
\item \ct{SystemNavigation} is a utility class holding many useful queries for navigation and browsing the \ct class hierarchy. For example, use \ct{SystemNavigation default browseMethodsWithSourceString: 'pharo'.} to find and browse all methods with a given source string. (Slow, but thorough!)
\item Every \st class points to an instance of \ct{MethodDictionary} which maps selectors to instances of \ct{CompiledMethod}. A compiled method knows its class, closing the loop.
\item \ct{MethodReference} is a leightweight proxy for a compiled method, providing additional convenience methods, and used by many \st tools. 
\item \ct{BrowserEnvironment}, part of the Refactoring Browser infrastructure, offers a more refined interface than \ct{SystemNavigation} for querying the system, since the result of a query can be used as a the scope of a new query. Both GUI and programmatic interfaces are available.
\item \ct{thisContext} is a pseudo-variable that reifies the run-time stack of the virtual machine. It is mainly used by the debugger to dynamically construct an interactive view of the stack. It is also especially useful for dynamically determining the sender of a message.
\item Intelligent breakpoints can be set using \ct{haltIf:}, taking a method selector as its argument. \ct{haltIf:} halts only if the named method occurs as a sender in the run-time stack.
\item A common way to intercept messages sent to a given target is to use a ``minimal object'' as a proxy for that target. The proxy implements as few methods as possible, and traps all message sends by implementing \ct{doesNotunderstand:}. It can then perform some additional action and then forward the message to the original target.
\item Send \ct{become:} to swap the references of two objects, such as a proxy and its target.
\item Beware, some messages, like \ct{class} and \ct{yourself} are never really sent, but are interpreted by the VM.  Others, like \ct{+}, \ct{-} and \ct{ifTrue:} may be directly interpreted or inlined by the VM depending on the receiver.
\item Another typical use for overriding \ct{doesNotUnderstand:} is to lazily load or compile missing methods.
\item \ct{doesNotUnderstand:} cannot trap \self-sends.
\item A more rigorous way to intercept messages is to use an object as a method wrapper. Such an object is installed in a method dictionary in place of a compiled method. It should implement \ct{run:with:in:} which is sent by the VM when it detects an ordinary object instead of a compiled method in the method dictionary. This technique is used by the SUnit Test Runner to collect coverage data.
\end{itemize}

%=========================================================
\ifx\wholebook\relax\else
   \bibliographystyle{jurabib}
   \nobibliography{scg}
   \end{document}
\fi
%=========================================================

%Other stuff:
%- anonymous classes (uses compile: and primitiveChangeClassTo:) ???
%- collect direct senders; class collaborations
%- Object primitiveChangeClassTo: become: and becomeForward: (see tests and slides with minimal object example)
%- PointerFinder?
%- anonymous classes (see slides) ?

%Test  Coverage using ObjectsAsMethodsWrap package:
%\begin{code}{}
%category := 'SCGPier'.
%w := (ObjectAsOneTimeMethodWrapper installOnClassCategory: category).
%tr := TestRunner new.
%ToolBuilder open: tr.
%[tr
%	categoryAt: (tr categoryList indexOf: 'SCGPier') put: true;
%	selectAllClasses;
%	runAll.]
%ensure: [[w do: [:each| each uninstall ]] valueUnpreemptively].
%((w select: [:each | each executed not ])
%	collect: [:each | each wrappedClass name, '>>', each selector name ]) explore.
%\end{code}

\appendix
\part{Annexes}
%:FAQ
% $Author: oscar $
% $Date: 2007-09-23 11:56:47 +0200 (dim, 23 sep 2007) $
% $Revision: 12130 $
% Traduction: Benoît TUDURI 18-10-2007
% Relecture: Martial Boniou - Fri Nov  9  16:49:46 CET 2007
% Relecture: Rene Mages     - Fri Jan  10 22:23:44 CET 2008
% Relecture: Martial Boniou - Wed Jan 30 23:21:34 CET 2008
% adaptation pour Pharo: martial - Fri Sep 11 10:50:59 CEST 2009 from
% $Author: oscar $ % $Date: 2009-09-09 09:57:42 +0200 (Wed, 09 Sep 2009) $ % $Revision: 29006 $
% martial: remplacement de la faq:omnibrowser en faq:packagebrowser pour la selection via "choose new default browser"
% Relecture: Rene Mages     - Thu Aug  12 22:23:44 CET 2010
%=================================================================
\ifx\wholebook\relax\else
% --------------------------------------------
% Lulu:
	\documentclass[a4paper,10pt,twoside]{book}
	\usepackage[
		papersize={6.13in,9.21in},
		hmargin={.75in,.75in},
		vmargin={.75in,1in},
		ignoreheadfoot
	]{geometry}
	\input{../common.tex}
	\pagestyle{headings}
	\setboolean{lulu}{true}
% --------------------------------------------
% A4:
%	\documentclass[a4paper,11pt,twoside]{book}
%	\input{../common.tex}
%	\usepackage{a4wide}
% --------------------------------------------
    \graphicspath{{figures/} {../figures/}}
	\begin{document}
	\renewcommand{\nnbb}[2]{} % Disable editorial comments
	\sloppy
\fi
  
%=================================================================
%\chapter{Frequently Asked Questions}
%\chapter{Questions Fréquemments Posées}
%% note de Martial: le terme FAQ est utilise dans certaines pages; sinon "questions freq. posees" est valable aussi
\chapter{Foire Aux Questions}
\label{cha:faq}

%\on{These should be *real* (not invented) FAQs.
%Here are a few that I have collected.
%Check the ST lecture notes for more FAQs.
%We should also try to mine more from newbies.}
\on{Ceci devrait être une *vraie* FAQs.
Celle-ci contient quelques questions que j'ai collectées.
Pour plus de FAQ, vous pouvez consulter les références de lectures sur ST.
Nous devrions essayer de nous mêmes pour les newbies...}

%=================================================================
%\section{Getting started}
\section{Prémisses}
\begin{faq}
%Where do I get the latest Squeak?
Où puis-je trouver la dernière version de Pharo?
\end{faq}
\answer
\pharoweb
\index{download}

\begin{faq}
%Which \pharo image should I use with this book?
Quelle image de \pharo devrais-je utiliser avec ce livre?
\end{faq}
\answer
Vous pouvez utiliser n'importe quelle image \pharo 
% ajout vf
de version 1.0
mais nous vous recommandons d'utiliser l'image préparée sur le site web de \pharo 
Par l'Exemple: {\ppe}.
% ajout one-click
Celle-ci inclut une version de la machine virtuelle compilée pour votre système d'exploitation
ainsi que des scripts pour lancer votre image en \emph{un clic}.
Utiliser une autre image, c'est courir le risque d'avoir des comportements surprenants lors
de la saisie des exercices proposés dans ce livre.

\begin{faq} % ajout vf : special OneClick version 1.0
Comment puis-je démarrer \pharo{} convenablement~?
\end{faq}
\answer
Cela varie en fonction de votre système d'exploitation:
\begin{itemize}
\item sous Windows, double-cliquez sur l'icône \textsf{pharo.exe} à la racine du répertoire \textsf{\pharooneclick{}.app};
\item sous \macosx{}, double-cliquez sur l'icône d'application \textsf{\pharooneclick} (ou \textsf{\pharooneclick{}.app});
\item sous Linux, double-cliquez sur l'icône \textsf{pharo.sh} depuis le répertoire \textsf{\pharooneclick{}.app} ou, grâce à un terminal, naviguer jusqu'au répertoire \textsf{\pharooneclick{}.app} et lancer la commande:
\begin{code}{}
./pharo.sh
\end{code}
\end{itemize}
\index{lancement de \pharo}

\begin{faq}\faqlabel{saveas-oneclick} % ajout vf : special OneClick version 1.0 
Comment puis-je changer d'image à la sauvegarde et être sûr de démarrer la bonne image lors du démarrage de \mbox{\pharo~?}
\end{faq}
\answer
Lorsque que vous sauvegarder votre image sous un autre nom en \clickant sur \menu{World\go{}Save as\ldots}, vous créez deux nouveaux fichiers dans le même répertoire que votre image initiale. En appelant la nouvelle image ``myPharo'' comme sur \figref{saveas-oneclick}, vous pourriez donc sauvegarder dans l'état courant votre image dans deux fichiers à la racine du dossier \mbox{\textsf{Contents/Resources}:} ``myPharo.image'' contenant le \emph{byte-code} et ``myPharo.changes'' contenant les changements de code source. L'intégralité du code source de notre image ``myPharo.image'' est l'union de code de ``myPharo.changes'' avec le fichier ``PharoV10.sources''.
En continuant de travailler dans \pharo{}, vous travaillez donc dans votre nouvelle image.
\begin{figure}[htb]
	{\centerline{\includegraphics[width=0.35\textwidth]{saveAsOneClick}}}
 	\caption{La boîte de dialogue \menu{save as\ldots}.\figlabel{saveas-oneclick}}
\end{figure}
Pour pouvoir lancer cette nouvelle image, la machine virtuelle a besoin de connaître le nouveau nom. Pour ce faire:
\begin{itemize} % RELIRE
\item sous Windows, éditez le fichier \textsf{pharo.ini} à la racine de \textsf{\pharooneclick{}.app} et remplacez le champ \textsf{ImageFile}. Dans notre cas, remplacez ``PBE.image'' par notre nouvelle image pour obtenir \mbox{\textsf{ImageFile=Contents\symbol{92}Resources\symbol{92}myPharo.image};}
\item sous \macosx{}, éditez le fichier \textsf{Info.plist} à la racine de \textsf{Contents} après avoir affiché le contenu du paquet en cliquant avec le bouton droit de la souris sur le programme \textsf{\pharooneclick}. Pour vous faciliter la navigation dans ce code \texttt{XML}, \macosx{} dispose de l'utilitaire \emph{Property List Editor}: trouver le champ \textsf{SqueakImageName} et renommez l'image du nouveau nom ``myPharo.image'';
\item sous Linux, éditez le script \textsf{pharo.sh} à la racine de \textsf{\pharooneclick{}.app} de sorte que le nom de l'image lancée par votre machine virtuelle change; ainsi la dernière ligne de code s'écrira: 
\begin{code}{}
exec "$BASE/squeakvm" \
	-plugins "$BASE" \
	-encoding latin1 \
	-vm-display-X11 \
	"$ROOT/Contents/Resources/myPharo.image"
\end{code}%$
Notez que les antislashs \ct{\} indiquent au \emph{shell} Linux de passer une ligne sans exécuter le code immédiatement (comme cela se fait normalement après un retour-chariot).
\end{itemize}
\index{sauvegarde de \pharo}
\index{fichier!image}
\index{fichier!changes}

%=================================================================
\section{Collections}

\begin{faq}
%How do I sort an \clsind{OrderedCollection}?
Comment puis-je trier une \clsind{OrderedCollection}~?

\end{faq}
\answer
%Send it the message \mthind{Collection}{asSortedCollection}.
Envoyez le message suivant \mthind{Collection}{asSortedCollection}.

\begin{code}{@TEST}
#(7 2 6 1) asSortedCollection --> a SortedCollection(1 2 6 7)
\end{code}

\begin{faq}
%How do I convert a collection of characters to a \clsind{String}?
Comment puis-je convertir une collection de caractères en une 
%% ajout
chaîne de caractères
\clsind{String}~?
\end{faq}
\answer
\begin{code}{@TEST}
String streamContents: [:str | str nextPutAll: 'hello' asSet] --> 'hleo'
\end{code}

%=================================================================
%\section{Browsing the system}
\section{Naviguer dans le système}

% voir \faqref{packagebrowser}
\begin{faq}\faqlabel{packagebrowser}
Le navigateur de classes ne ressemble pas à celui décrit dans le
livre. Que se passe-t-il?
\end{faq}
\answer
Vous utilisez probablement une image disposant d'une version
différente d'\ind{OmniBrowser}
(abrégé en OB) % ajout: martial: Fri Sep 11 11:41:49 CEST 2009
installé comme Browser par défaut.
Dans ce livre, nous présumons que le navigateur Omnibrowser
\emph{Package Browser} (navigateur par paquetages) est installé par
défaut.
Vous pouvez changer cela en \clickant sur la bulle grise 
à droite de la barre de titre du navigateur, puis en sélectionnant
dans le menu du Browser ``Choose new default Browser'' (en français,
\emph{choisissez le nouveau Browser par défaut}). Dans la liste des
navigateurs proposés, \clickz sur O2PackageBrowserAdaptor.
Le prochain navigateur de classes que vous ouvrirez sera le Package Browser. %CHANGE

\begin{figure}[tbh]
	\centering
	\subfigure[Choisir un nouveau Browser.]{\figlabel{chooseNewBrowser}
		\includegraphics[width=0.45\linewidth]{chooseNewBrowser}}\hfill
	\subfigure[Sélectionnez l'OB Package Browser]{\figlabel{O2PackageBrowserAdaptor}
		\includegraphics[width=0.45\linewidth]{O2PackageBrowserAdaptor}}\hfill
	\caption{Changer le navigateur par défaut.}
\end{figure}
\seeindex{halo}{Morphic} % \seeindex{morphic halo}{Morphic} dans PBE

\begin{faq}
%How to I search for a class?
Comment puis-je chercher une classe~?
\end{faq}
\answer
%\short{b} (browse) on the class name, or \short{f} in the category pane of the class browser.
\short{b} (pour \emph{browse} \cad parcourir à l'aide du navigateur)
sur le nom de la classe ou \short{f} (pour \emph{find} \cad trouver)
dans le panneau des catégories du Browser. % REVOIR
\index{raccourci-clavier!browse it}
\index{raccourci-clavier!find...}

\begin{faq}
%How do I find/browse all sends to super?
Comment puis-je trouver/naviguer dans tous les envois à \super~?
\end{faq}
\answer
%The second solution is much faster:
La deuxième solution est la plus rapide:
\begin{code}{}
SystemNavigation default browseMethodsWithSourceString: 'super'.
SystemNavigation default browseAllSelect: [:method | method sendsToSuper ].
\end{code}
%\index{browsing programmatically}
\index{naviguer de manière pragmatique} % REVOIR
\clsindex{SystemNavigation}

\begin{faq}
%How do I browse all super \subind{super}{send}{}s within a hierarchy?
Comment puis-je naviguer au travers de tous les 
\subpvind{super}{envoi}{}s de messages à \super dans une hiérarchie~?
\end{faq}
\answer
\begin{code}{}
browseSuperSends:= [:aClass | SystemNavigation default
	browseMessageList: (aClass withAllSubclasses gather: [:each |
		(each methodDict associations
			select: [:assoc | assoc value sendsToSuper ])
				collect: [:assoc | MethodReference class: each selector: assoc key ] ])
	name: 'Les envois !à! super de ' , aClass name , ' et de ses sous-classes'].
browseSuperSends value: OrderedCollection.
\end{code}

\begin{faq}
%How do I find out which are the new methods implemented in a class?
Comment puis-je découvrir quelles sont les nouvelles méthodes implémentées dans une classe? (autrement, dit comment obtenir la liste des méthodes non surchargées d'une classe~?) 
\end{faq}
\answer
%Here we ask which new methods are introduced by \ct{True}:
Dans le cas présent nous demandons quelles sont les nouvelles méthodes introduites par la classe \ct{True}:
\begin{code}{@TEST | newMethods |}
newMethods:= [:aClass| aClass methodDict keys select:
	[:aMethod | (aClass superclass canUnderstand: aMethod) not ]].
newMethods value: True --> an IdentitySet(#asBit)
\end{code}

\begin{faq}
%How do I tell which methods of a class are abstract?
Comment puis-je trouver les méthodes d'une classe qui sont abstraites~?
\end{faq}
\answer
\begin{code}{@TEST | abstractMethods |}
abstractMethods:=
	[:aClass | aClass methodDict keys select:
		[:aMethod | (aClass>>aMethod) isAbstract ]].
abstractMethods value: Collection --> an IdentitySet(#remove:ifAbsent: #add: #do:)
\end{code}

\begin{faq}
%How do I generate a view of the \ind{AST} of an expression?
Comment puis-je créer une vue de l'arbre syntaxique abstrait ou 
\ind{AST} d'une expression~?
\end{faq}
\answer
%Load AST from squeaksource.com. Then evaluate:
Charger le paquetage AST depuis \url{http://squeaksource.com/AST}. Ensuite évaluer:
\begin{code}{}
(RBParser parseExpression: '3+4') explore
\end{code}
%(Alternatively \emph{explore it}.)
\clsindex{RBParser}

\begin{faq}
%How do I find all the Traits in the system?
Comment puis-je trouver tout les \emph{Traits} dans le système~?
\end{faq}
\answer
\begin{code}{}
Smalltalk allTraits
\end{code}

\begin{faq}
%How do I find which classes use traits?
Comment puis-je trouver quelles classes utilisent les \emph{Traits}~?
\end{faq}
\answer
\begin{code}{}
Smalltalk allClasses select: [:each | each hasTraitComposition and: [each traitComposition notEmpty]]
\end{code}

%=================================================================
%\section{Using Monticello and SqueakSource}
\section{Utilisation de Monticello et de SqueakSource}

\begin{faq}
%How do I load a \ind{SqueakSource} project?
Comment puis-je charger un projet du \ind{SqueakSource}~?
\index{Monticello}
\end{faq}
\answer
\begin{enumerate}
%  \item Find the project you want in \url{squeaksource.com}
  \item Trouvez le projet que vous souhaitez sur \url{http://squeaksource.com}
%  \item Copy the registration code snippet
  \item Copiez le code d'enregistrement
%  \item Select \menu{open \go Monticello browser}
  \item Sélectionnez \menu{open \go Monticello browser}
%  \item Select \menu{+Repository \go HTTP}
  \item Sélectionnez \menu{+Repository \go HTTP}
%  \item Paste and accept the Registration code snippet; enter your password
  \item Collez et acceptez le code d'enregistrement ; entrez votre mot de passe
%  \item Select the new repository and \menu{Open} it
  \item Sélectionnez le nouveau dépôt et ouvrez-le avec le bouton \menu{Open}
%  \item Select and load the latest version
  \item Sélectionnez et chargez la version la plus récente
\end{enumerate}

\begin{faq}
%How do I create a SqueakSource project?
Comment puis-je créer un projet SqueakSource~?
\end{faq}
\answer
\begin{enumerate}
%  \item Go to \url{http://squeaksource.com}
  \item Allez à \url{http://squeaksource.com}
%  \item Register yourself as a new member
  \item Enregistrez-vous comme un nouveau membre
%  \item Register a project (name = category)
  \item Enregistrez un projet (nom = catégorie)
%  \item Copy the Registration code snippet
  \item Copiez le code d'enregistrement
%  \item \menu{open \go Monticello browser}
  \item \menu{open \go Monticello browser}
%  \item \menu{+Package} to add the category
  \item \menu{+Package} pour ajouter une catégorie
%  \item Select the package
  \item Sélectionnez le package
%  \item \menu{+Repository \go HTTP}
  \item \menu{+Repository \go HTTP}
%  \item Paste and accept the Registration code snippet; enter your password
  \item Collez et acceptez le code d'enregistrement; entrez votre mot de passe
%  \item \menu{Save} to save the first version
  \item \menu{Save} pour enregistrer la première version
\end{enumerate}

\begin{faq}
%How do I extend \ct{Number} with \ct{Number>>>chf} but have Monticello recognize it as being part of my \ct{Money} project?
Comment puis-je étendre \ct{Number} avec
%% ajout
la méthode \ct{Number>>>chf} 
%% changement
tel que Monticello la reconnaissent comme étant une partie de mon projet \ct{Money}~?
\end{faq}
\answer
%Put it in a method-category named \ct{*Money}.
%% changement
Mettez-la 
dans une catégorie de méthodes nommée \ct{*Money}.
%Monticello gathers all methods that are in other categories named like *package and includes them in your package.
Monticello réunit toutes les méthodes 
dont les noms de catégories 
%% changement: nommées autrement telles que 
ont la forme \emph{*package} et les insére dans votre package.

%=================================================================
%\section{Tools}
\section{Outils}

\begin{faq}
%How do I programmatically open the \ind{SUnit} \ind{TestRunner}?
Comment puis-je ouvrir de manière pragmatique le \ind{SUnit} \ind{TestRunner}~?
\end{faq}
\answer
%Evaluate \ct{TestRunner open}.
Évaluez \ct{TestRunner open}.

\begin{faq}
%Where can I find the \ind{Refactoring Browser}?
Où puis-je trouver le \ind{Refactoring Browser}~?
\end{faq}
\answer
%Load AST then Refactoring Engine from squeaksource.com:
Chargez le paquetage AST puis le moteur de 
%%plutot que refrabrication 
refactorisation sur le site \url{http://squeaksource.com}:\\
\url{http://www.squeaksource.com/AST}\\
\url{http://www.squeaksource.com/RefactoringEngine} % pas de problème avec Stef de laisser AST bien qu'OBSOLETE

\begin{faq}
%How do I register the browser that I want to be the default?
Comment puis-je enregistrer le navigateur comme navigateur par défaut~?
\end{faq}
\answer
%Click the menu icon in the top left of the Browser window.
\Clickz{} sur l'icône (une bulle grise) du menu situé à droite dans la barre de
titre de la fenêtre du Browser. % REVOIR (erreur dans PBE)
%% à côté de \aretirer{la croix de destruction de la fenêtre}. % martial - ce n'est plus une croix mais une bulle rouge 
%% ajout (pour la traduction)
Choisissez \menu{Register this Browser as default} pour enregistrer le navigateur courant comme navigateur par défaut ou bien, sélectionnez \menu{Choose new default Browser} pour obtenir un menu flottant d'où vous pourrez faire votre choix parmi les différentes classes de Browser.

%=================================================================
%\section{Regular expressions and parsing}
\section{Expressions régulières et analyse grammaticale}

%\begin{faq}
%%How can I work with regular expressions?
%Comment puis-je travailler avec les expressions régulières~?
%\index{paquetage!expressions régulières}
%\end{faq}
%\answer
%Chargez le paquetage de RegEx de Vassili Bykov à l'adresse: \\
%\url{http://www.squeaksource.com/Regex.html}
%\index{Bykov, Vassili}

\begin{faq}
%Where is the documentation for the RegEx package?
Où est la documentation pour le \arevoir{paquetage RegEx~?}
% rene : paquetage VB-Regex ?
\end{faq}
\answer
%Look at the \menu{DOCUMENTATION} protocol of \ct{RxParser class} in the \menu{VB-Regex} category.
%% changement
Regardez dans le protocole \menu{DOCUMENTATION} de \ct{RxParser class} situé dans la catégorie \menu{VB-Regex}.

\begin{faq}
%Are there tools for writing parsers?
Y a-t'il des outils pour l'écriture d'un outil d'analyse grammaticale~?
\end{faq}
\answer
%Use \ind{SmaCC}\,---\,the Smalltalk Compiler Compiler.
Utilisez \ind{SmaCC}\,---\,le compilateur de compilateur (ou générateur de compilateur)~\footnote{En anglais, Compiler-Compiler.} Smalltalk.
%You should install at least SmaCC-lr.13.
Vous devrez installer au moins SmaCC-lr.13.
%Load it from \url{http://www.squeaksource.com/SmaccDevelopment.html}.
Chargez-le depuis \url{http://www.squeaksource.com/SmaccDevelopment.html}.
%There is a nice tutorial online:
Il y a un bon tutoriel en ligne à l'adresse:
\url{http://www.refactoryworkers.com/SmaCC/Tutorial.html} % martial 2011; CHANGE URL: http://www.refactory.com/Software/SmaCC/Tutorial.html}

\begin{faq}
%Which packages should I load from SqueakSource SmaccDevelopment to write parsers?
Quels paquetages devrais-je charger depuis \emph{SqueakSource SmaccDevelopment} pour écrire un analyseur grammatical~?
\end{faq}
\answer
%Load the latest version of \ind{SmaCCDev}{}\,---\,the runtime is already there.
Chargez la dernière version de \ind{SmaCCDev}{}\,---\,le lanceur de programme est déjà actif.
(Attention: SmaCC-Development est destiné à la version 3.8 de \squeak)

\ifseaside{}{%
% \section{Pharo for web developers}
\section{Pharo pour les développeurs web}
\begin{faq}\faqlabel{seaside}
\pharo{} dispose-t-il d'un environnement pour développer des sites web~?
\end{faq}
\answer
Oui. Et mieux que ça: \pharo{} propose Seaside, un \emph{framework} dont l'architecture orientée composants et la gestion totale du bouton de \emph{retour} du navigateur permettent de développer de formidables applications web modernes et dynamiques. Cette bibliothèque est portée sur d'autres \st{}s: \pharo{} est la plateforme de développement choisie par son équipe de programmeurs.\\
Un livre dédié à cet environnement est disponible pour l'heure en anglais sur: \dwdseaside{}. Nous vous encourageons à acquérir une version imprimable auprès de notre éditeur: \sba{}.% 
}
%=================================================================
\ifx\wholebook\relax\else\end{document}\fi
%=================================================================

%\begin{faq}
%How do I run a headless image with a file argument?
%\end{faq}
%\answer
%Right now you can't do it with the MacOSX VM.

%\begin{faq}
%How do I find out which methods access the Smalltalk dictionary?
%\end{faq}
%\answer
%???

%\begin{faq}
%How do I get the tree view of an AST?
%\end{faq}
%\answer
%???

%\begin{faq}
%How do I browse all references to a given class?
%\end{faq}
%\answer
%???

%%%%%%%%%%%%%%%%%%%%%%%%%%%%%%%%%%%%%

%  Benoît

% Rappel LaTeX:
%  é    \'e
%  è    \`e
%  ê    \^e
%  ë    \"e
%  ç    \c{c}
%  $    \$
%  &    \&
%  %    \%
%  #    \#
%  _    \_
%  {    \{
%  }    \}
%  ~    \~
%  ^    \^
%  \    $\backslash$

% Paramètre VIM 7.1
% pour l'édition de ce document
% :set encoding=UTF-8
%%%%%%%%%%%%%%%%%%%%%%%%%%%%%%%%%%%%%


% LocalWords:  retour-chariot

%=================================================================
%:BIBLIOGRAPHY
% \printglossary
\pagestyle{newheadings}
\bibliographystyle{jurabib}
\bibliography{scg}
\cleardoublepage
%=================================================================
%:INDEX
{\small\raggedright\printindex}
%\printindex
%=================================================================
% Round out to multiple of 4 pages
\newcommand{\pagefinale}{\cleardoublepage\thispagestyle{empty}
~ % Force some space
}
\pagefinale % une seule feuille blanche
%\pagefinale
\cleardoublepage
%=================================================================
\end{document}
